%% The following is a directive for TeXShop to indicate the main file
%%!TEX root = diss.tex


\begin{thebibliography}{}
%--------------------------------------------------------------------------------------------------------------------------------------------%

\bibitem{ACHP}
M. K. Agrawal, J. H. Coates, D. C. Hunt and A. J. van der Poorten,
\emph{Elliptic curves of conductor $11$},
Math. Comp. 35 (1980), 991--1002.

\bibitem{AkBh}
S. Akhtari and M. Bhargava,
\emph{A positive proportion of locally soluble Thue equations are globally insoluble}, 
American Journal of Mathematics, (2017).

\bibitem{BaSh} 
R. Balasubramanian and T. N. Shorey,
\newblock On the equation $a(x^m-1)/(x-1) = b (y^n-1)/(y-1)$,
\newblock {\em Math. Scand.}  46 (1980), 177--182.

\bibitem{Bak1}
A. Baker, 
\emph{Linear forms in the logarithms of algebraic numbers}, 
Mathematika. \textit{12} (1966), 204-216.

\bibitem{Bak}
A. Baker,
\newblock Bounds for the solutions of the hyperelliptic equation, 
\newblock {\em Math. Proc. Camb. Phil. Soc.}. 65 (1969), 439--444. 

\bibitem{BaBe}
M. Bauer and M. A. Bennett,
\newblock Applications of the hypergeometric method to the Generalized Ramanujan-Nagell equation,
\newblock {\em The Ramanujan J.} 6 (2002), 209--270.

\bibitem{Be}
K. Belabas,
\emph{A fast algorithm to compute cubic fields,} Math. Comp. 66 (1997), 1213--1237.

\bibitem{BeCo}
K. Belabas and H. Cohen,
\emph{Binary cubic forms and cubic number fields},
Organic Mathematics (Burnaby, BC, 1995), 175--204. CMS Conf. Proc., 20 Amer. Math. Soc. 1997.

\bibitem{BeGh}
M. A. Bennett and A. Ghadermarzi,
\emph{Mordell's equation : a classical approach},
L.M.S. J. Comput. Math. 18 (2015), 633--646.

\bibitem{BeGhKr}
M. A. Bennett, A. Gherga and D. Kreso, 
\emph{An old and new approach to Goormaghtigh's equation},
Submitted

\bibitem{BeGhRe}
M. A. Bennett, A. Gherga and A. Rechnitzer,
\emph{Computing elliptic curves over $\mathbb{Q}$},
Math. Comp. 88 (2019), no. 317, 1341-1390

\bibitem{BR-survey}
M. A. Bennett and A. Rechnitzer,
\emph{Computing elliptic curves over $\mathbb{Q}$ : bad reduction at one prime}, 
Proceedings of 2015 AMMCS-CAIMS Congress

\bibitem{BeMa}
W. E. H. Berwick and G. B. Mathews,
\emph{On the reduction of arithmetical binary cubic forms which have a negative determinant},
Proc. London Math. Soc. (2) 10 (1911), 43--53.

\bibitem{Beu1}
F. Beukers,
\newblock On the generalized Ramanujan-Nagell equation I,
\newblock {Acta Arith.} XXXVIII (1981), 389--410.

\bibitem{Beu2}
F. Beukers,
\newblock On the generalized Ramanujan-Nagell equation, II,
\newblock {Acta Arith.} XXXIX (1981), 113--123.

\bibitem{BK}
B. J. Birch and W. Kuyk (Eds.),
\emph{Modular Functions of One Variable IV},
Lecture Notes in Math., vol. 476, Springer-Verlag, Berlin and New York, 1975.

\bibitem{BoGu}
E. Bombieri and W. Gubler,
\emph{Heights in Diophantine Geometry}, 
Cambridge University Press, 2006.

\bibitem{BS}
Z. I. Borevich and I. R. Shafarevich, \emph{Number theory},
Academic Press, 1966.

\bibitem{magma}
W.\ Bosma, J.\ Cannon and C.\ Playoust,
\newblock The Magma Algebra System I: The User Language,
\newblock {\em J.\ Symb.\ Comp.} {\bf 24} (1997), 235--265. (See also \url{http://magma.maths.usyd.edu.au/magma/})

\bibitem{BCDT}
C. Breuil, B. Conrad, F. Diamond and R. Taylor,
\emph{On the Modularity of Elliptic Curves over $\mathbb{Q}$ : Wild 3-adic Exercises},
J. Amer. Math. Soc. 14 (2001), 843--939.

\bibitem{BrMc}
A. Brumer and O. McGuinness,
\emph{The behaviour of the Mordell-Weil group of elliptic curves},
Bull. Amer. Math. Soc. 23 (1990), 375--382.

\bibitem{BrSi}
A. Brumer and J. H. Silverman,
\emph{The number of elliptic curves over $\mathbb{Q}$ with conductor $N$},
Manuscripta Math. 91 (1996), 95--102.

\bibitem{BugeaudGyory} 
Y. Bugeaud and K. Gy\"{o}ry, 
\newblock Bounds for the solutions of Thue-Mahler equations and norm form equations, 
\newblock{ \em Acta Arith.} 74.3 (1996), 273--292.

\bibitem{BuSh}
Y. Bugeaud and T.N. Shorey,
\newblock On the diophantine equation $\frac{x^m-1}{x-1} = \frac{y^n-1}{y-1}$,
\newblock {\em Pacific J. Math.}  207 (2002), 61--75.

\bibitem{Ca}
J. W. S. Cassels, \emph{Local fields},
Cambridge University Press, 1986.

\bibitem{Chud}
G. V. Chudnovsky, 
\newblock On the method of Thue-Siegel, 
\newblock {\em Ann. of Math.} (2) 117 (1983), 325--382. 

\bibitem{Coa}
J. Coates,
\emph{An effective $p$-adic analogue of a theorem of Thue. III. The diophantine equation $y^2=x^3+k$},
Acta Arith. 16 (1969/1970), 425--435.

%%\bibitem{Coa2}
%J. Coates,
%\emph{Verification of Weil's conjecture on elliptic curves over $\mathbb{Q}$ in some special cases},
%Proceedings of AMS Number Theory Conference, Boulder, Colorado, 1972.

\bibitem{Cog}
F. Coghlan,
\emph{Elliptic Curves with Conductor $2^m3^n$},
Ph.D. thesis, Manchester, England, 1967.

\bibitem{Coh1}
H. Cohen,
\emph{A course in computational algebraic number theory}, 
Springer Verlag, 1995.

\bibitem{Coh2}
H. Cohen,
\emph{Number theory volume I: tools and diophantine equations}, 
Springer Science + Business Media, LLC, 2007.
%Acta Arith. 65 (1993), 367--381.

\bibitem{Cre1}
J. E. Cremona,
\emph{Elliptic curve tables}, 
\url{http://johncremona.github.io/ecdata/}

\bibitem{Cre2}
J. E. Cremona,
\emph{Algorithms for modular elliptic curves}, second ed.,
Cambridge University Press, Cambridge, 1997. Available online at 
\url{http://homepages.warwick.ac.uk/staff/J.E.Cremona/book/fulltext/index.html}

\bibitem{Cr}
J. E. Cremona,
\emph{Reduction of binary cubic and quartic forms,}
LMS J. Comput. Math. 4 (1999), 64--94.

 \bibitem{mwrank}
J. E. Cremona, 
\emph{mwrank and related programs for elliptic curves over $\mathbb{Q}$}, 1990--2017,
 \url{http://www.warwick.ac.uk/staff/J.E. Cremona/mwrank/index.html}
  
\bibitem{CrLi}
J. E. Cremona and M. Lingham,
\emph{Finding all elliptic curves with good reduction outside a given set of primes},
Experiment. Math. 16 (2007), 303--312.

\bibitem{Dav}
H. Davenport,
\emph{The reduction of a binary cubic form. I.},
J. London Math. Soc. 20 (1945), 14--22.

\bibitem{Dav2}
H. Davenport,
\emph{The reduction of a binary cubic form. II.},
J. London Math. Soc. 20 (1945), 139--147.

\bibitem{Dav3}
H. Davenport,
\emph{On the class-number of binary cubic forms. I.},
J. London Math. Soc. 26 (1951), 183--192; ibid,  27 (1952), 512.

\bibitem{DaHe}
H. Davenport and H. Heilbronn,
\emph{On the density of discriminants of cubic fields. II.},
Proc. Roy. Soc. London Ser. A. 322 (1971), 405--420.

\bibitem{DaLeSc}
H. Davenport, D. J. Lewis and A. Schinzel,
\newblock Equations of the form $f(x)=g(y)$,
\newblock {\em Quart. J. Math. Oxford set (2)} 12 (1961), 304--312.

\bibitem{EGT}
B. Edixhoven, A. de Groot and J. Top,
\emph{Elliptic curves over the rationals with bad reduction at only one prime},
Math. Comp.  54 (1990), 413--419.

\bibitem{Elk1}
N. D. Elkies,
\emph{How many elliptic curves can have the same prime conductor?}, 
\url{http://math.harvard.edu/~elkies/condp_banff.pdf}

\bibitem{Elk2}
N. D. Elkies,
\emph{Rational points near curves and small nonzero $|x^3-y^2|$ via lattice reduction}, 
Lecture Notes in Computer Science 1838 (proceedings of ANTS-4, 2000; W.Bosma, ed.), 33--63.

\bibitem{ElWa}
N. D. Elkies and M. Watkins,
\emph{Elliptic curves of large rank and small conductor},
 Algorithmic number theory, 42--56, Lecture Notes in Comput. Sci., 3076, Springer, Berlin, 2004.

\bibitem{FP}
U. Fincke and M. Pohst, 
\emph{Improved methods for calculating vectors of short length in a lattice, including a complexity analysis}, Mathematics of Computation \textbf{44} (1985), no. 170, 463-471.

\bibitem{GhKaMaSi}
A. Gherga, R. von K\"{a}nel, B. Matschke and S. Siksek, 
\newblock Efficient resolution of Thue-Mahler equations, 
\newblock {Manuscript in preparation} (2018).

\bibitem{Go}
R. Goormaghtigh,
\newblock {\em L'Interm\'ediaire des Math\'ematiciens} 24 (1917), 88.

\bibitem{Had}
T. Hadano,
\emph{On the conductor of an elliptic curve with a rational point of order $2$},
Nagoya Math. J. 53 (1974), 199--210.

\bibitem{Hai}
B. Haible, 
\emph{CLN, a class library for 
numbers}, available from \url{http://www.ginac.de/CLN/}

\bibitem{Ham}
K. Hambrook,
\newblock Implementation of a Thue-Mahler solver,
\newblock M.Sc. thesis, University of British Columbia, 2011.
	
\bibitem{Has}
H. Hasse,
\emph{Arithmetische Theorie der kubischen Zahlk\"{o}per auf klassenk\"{o}rpertheoretischer Grundlage},
Math. Z. 31 (1930), 565--582.

\bibitem{Has2}
H. Hasse, 
\emph{Number theory}, 
Springer-Verlag, 1980.

\bibitem{He}
B. He,
\newblock A remark on the Diophantine equation $(x^3-1)/(x-1)=(y^n-1)/(y-1)$,
\newblock {\em Glasnik Mat.} 44 (2009), 1--6.

\bibitem{HeTo}
B. He and A. Togb\'e,
\newblock On the number of solutions of Goormaghtigh equation for given $x$ and $y$,
\newblock {\em Indag. Mathem.} 19 (2008), 65--72.

\bibitem{Her1}
C. Hermite,
\emph{Note sur la r\'eduction des formes homog\`enes \`a 
coefficients entiers et  \`a deux ind\'etermin\'ees},
J. Reine Angew. Math. 36 (1848), 357--364.

\bibitem{Her2}
C. Hermite,
\emph{Sur la r\'eduction des formes cubiques \`a deux ind\'etermin\'ees},
C. R. Acad. Sci. Paris 48 (1859), 351--357.

\bibitem{Ju}
G. Julia,
\emph{\'Etude sur les formes binaires non quadratiques \`a ind\'etermin\'ees r\'eelles, ou complexes, ou \`a 
ind\'etermin\'ees conjugu\'ees},
Mem. Acad. Sci. l'Inst. France 55 (1917), 1--293.

\bibitem{KanMat}
R. von Kanel and B. Matschke,
\emph{Solving $S$-unit, Mordell, Thue, Thue-Mahler and generalized Ramanujan-Nagell equations via Shimura-Taniyama conjecture}, 
preprint, arXiv:1605.06079.

\bibitem{Ka}
C. Karanicoloff,
\newblock Sur une \'equation diophantienne consid\'er\'ee par Goormaghtigh,
\newblock {\em Ann. Polonici Math.} XIV (1963), 69--76.

\bibitem{Ko}
N. Koblitz, 
\emph{p-adic numbers, p-adic analysis, and zeta-functions}, 
Springer-Verlag,1977.

\bibitem{Kou}
A. Koutsianas,
\emph{Computing all elliptic curves over an arbitrary number field with prescribed primes of bad reduction}, 
Experiment. Math., \url{http://www.tandfonline.com/doi/full/10.1080/10586458.2017.1325791}

\bibitem{Le1}
M. Le,
\newblock On the Diophantine equation $(x^3-1)/(x-1)=(y^n-1)/(y-1)$,
\newblock {\em Trans. Amer. Math. Soc.} 351 (1999), 1063--1074.

\bibitem{Le2}
M. Le,
\newblock Exceptional solutions to the exponential Diophantine equation $(x^3-1)/(x-1)=(y^n-1)/(y-1)$,
\newblock {\em J. Reine Angew. Math.}  543 (2002), 187--192.

\bibitem{Le3}
M. Le,
\newblock On Goormaghtigh's equation $(x^3-1)/(x-1)=(y^n-1)/(y-1)$,
\newblock {\em Acta Math. Sinica (Chin. Ser.)} 45 (2002), 505--508.

\bibitem{LLL}
A.K. Lenstra, H.W. Lenstra Jr., and L. Lovasz, 
\emph{Factoring polynomials with rational coefficients}, 
Math. Ann. \textbf{261} (1982), 515-534.

\bibitem{Liu}
J. Liouville,
\emph{Sur des classes tr\`es \'etendues de quantit\'es dont la valuer $n^e$ est ni algebrique, ni m\^eme r\'eductible \'a des irrationnelles algebriques},
C.R. Acad. Sci. Paris \textbf{18} (1844), 883-885, 910-911.

\bibitem{Lju}
W. Ljunggren,
\newblock Noen Setninger om ubestemte likninger av formen $(x^n-1)/(x-1)=y^q$,
\newblock {\em Norsk. Mat. Tidsskr.} 25 (1943), 17--20.

\bibitem{LMFDB}
The LMFDB Collaboration, The L-functions and Modular Forms Database, 
{\url http://www.lmfdb.org}

\bibitem{Mahler}
K. Mahler,
\emph{Zur Approximation algebraischer Zahlen, I: Ueber den gr\"ossten
Primteiler bin\"arer Formen}, Math. Ann. 107 (1933), 691--730. 

\bibitem{Mah0}
K. Mahler,
\emph{An application of Jensen's formula to polynomials},
Mathematika 7 (1960), 98--100.

\bibitem{Mah}
K. Mahler,
\emph{An inequality for the discriminant of a polynomial},
Michigan Math. J.  11 (1964), 257--262.

\bibitem{Maple}
L. Bernardin et al, 
\newblock Maple Programming Guide, Maplesoft, 
2017, Waterloo ON, Canada.

\bibitem{Mar}
D. A. Marcus, 
\emph{Number Fields},
Springer-Verlag, 1977.

\bibitem{MO}
J.-F. Mestre and J. Oesterl\'e,
\emph{Courbes de Weil semi-stables de discriminant une puissance $m$-i\`eme},
J. Reine Angew. Math 400 (1989), 173--184.

\bibitem{Mih}
P. Mih{\u{a}}ilescu,
\newblock Primary cyclotomic units and a proof of Catalan's conjecture.
\newblock {\em J. Reine Angew. Math.} 572  (2004), 167--195.

\bibitem{Mil}
G. L.  Miller, 
\emph{Riemann's hypothesis and tests for primality} in Proceedings 
of seventh annual ACM symposium on Theory of computing, 234--239 (1975).

\bibitem{Mor1}
L. J. Mordell,
\emph{The diophantine equation $y^2-k=x^3$},
Proc. London. Math. Soc. (2) 13 (1913), 60--80.
  
\bibitem{Mor2}
L. J. Mordell,
\emph{Indeterminate equations of the third and fourth degree},
Quart. J. of Pure and Applied Math. 45 (1914), 170--186.
  
\bibitem{Mor}
L. J. Mordell,
\emph{Diophantine Equations,}
Academic Press, London,
1969.

\bibitem{Nag}
T. Nagell, 
\newblock Note sur l'\'equation ind\'etermin\'ee $(x^n-1)/(x-1)=y^q$, 
\newblock {\em Norsk. Mat. Tidsskr.} 2 (1920), 75--78.

\bibitem{Nag2}
T. Nagell, 
\emph{Introduction to Number Theory},
New York, 1951.

\bibitem{Nark}
W. Narkiewicz, 
\emph{Elementary and analytic theory of algebraic numbers}, 
3rd ed., Springer-Verlag, 2004.

\bibitem{NeSh}
Y. V. Nesterenko and T. N. Shorey,
\newblock On an equation of Goormaghtigh,
\newblock {\em Acta Arith.} 83 (1998), 381--389.

\bibitem{Neu}
O. Neumann,
\emph{Elliptische Kurven mit vorgeschriebenem Reduktionsverhalten II},
Math. Nach. 56 (1973), 269--280.

\bibitem{Neuk}
J. Neukirch
\emph{Algebraic Number Theory},
Springer-Verlag, 1999.

\bibitem{Ogg1}
A. P. Ogg. 
\emph{Abelian curves of 2-power conductor},
Proc. Cambridge Philos. Soc., $62$ ($1966$), $143-148$.

\bibitem{Ogg2}
A. P. Ogg. 
\emph{Abelian curves of small conductor},
J. Reine Angew. Math., $226$ ($1967$), $204-215$.

\bibitem{Pap}
I. Papadopoulos,
\emph{Sur la classification de N\'eron des courbes elliptiques  
en caract\'eristique r\'esiduelle $2$ et $3$}, 
J. Number Theory 44 (1993), 119--152.

\bibitem{PARI2}
The PARI~Group, Bordeaux.
PARI/GP version {\tt 2.7.1}, 2014.
available at {\tt http://pari.math.u-bordeaux.fr/}.

\bibitem{PPVW}
J. Park, B. Poonen, J. Voight and M. Matchett Wood,
\emph{A heuristic for boundedness of ranks of elliptic curves},
J. European Math. Soc \url{https://www.ems-ph.org/doi/10.4171/JEMS/893}

\bibitem{Pet1}
A. Peth\H{o},
\emph{On the resolution of Thue inequalities},
J. Symbolic Computation 4 (1987), 103--109.

\bibitem{Pet2}
A. Peth\H{o},
\emph{On the representation of $1$ by binary cubic forms of positive discriminant},
Number Theory, Ulm 1987 (Springer LNM 1380), 185--196.

\bibitem{Rab}
M. O. Rabin, 
 \emph{Probabilistic algorithm for testing primality}, J.
Number Theory 12 (1980) 128--138.

\bibitem{Ra}
R. Ratat,
\newblock {\em L'Interm\'ediaire des Math\'ematiciens} 23 (1916), 150.

\bibitem{Ro}
G. Robin,
\newblock Estimation de la fonction de Tchebychef $\theta$ sur le $k$-i\`eme nombre premier et grandes valeurs de la fonction $\omega (n)$ nombre de diviseurs premiers de $n$.
\newblock {\em Acta Arith.} XLII (1983), 367--389.

\bibitem{RuSi}
K. Rubin and A. Silverberg,
\emph{Mod $2$ representations of elliptic curves}, Proc. Amer. Math. Soc. 129 (2001), 53--57.

\bibitem{Sage}
The Sage Developers,
\newblock SageMath, the Sage Mathematics Software System (Version 8.1),
 \newblock \url{http://www.sagemath.org},
\newblock  2018.

\bibitem{Set}
B. Setzer,
\emph{Elliptic curves of prime conductor},
J. London Math. Soc. 10 (1975), 367--378.

\bibitem{Shaf}
I. R. Shafarevich,
\emph{Algebraic number theory}, Proc. Internat. Congr. Mathematicians,
Stockholm, Inst. Mittag-Leffler, Djursholm (1962), 163--176.

\bibitem{Shanks}
D. Shanks,
\emph{Five number-theoretic algorithms},
 Proceedings of the Second Manitoba Conference on Numerical Mathematics, (1973), 51--70.

\bibitem{ShoSur}
T. N. Shorey,
\newblock An equation of Goormaghtigh and Diophantine approximations,
\newblock Current Trends in Number Theory, edited by S.D.Adhikari, S.A.Katre and B.Ramakrishnan, Hindustan Book Agency, New Delhi (2002), 185--197.

\bibitem{ShTi}
T. N. Shorey and R. Tijdeman,
\newblock New applications of diophantine approximation to diophantine equations,
\newblock {\em Math. Scand.} 39 (1976), 5--18.

\bibitem{Sie}
C. L. Siegel,
\emph{\:Uber einige Anwendungen Diophantischer Approximationen},
Abh. Preuss. Acad. Wiss. Phys.-Mat. Kl.\textbf{1}, (1929), 41-69.

\bibitem{SSW}
A. K. Silvester, B. K. Spearman and K. S. Williams,
\emph{Cyclic cubic fields of given conductor
and given index}, Canad. Math. Bull. Vol. 49 (2006)  472--480.

\bibitem{Sm}
N.P. Smart, 
\emph{The algorithmic resolution of diophantine equations}, 
Chapman and Hall, Cambridge University Press, 1998.

\bibitem{SW}
 J. P. Sorenson and J. Webster, 
  \emph{Strong Pseudoprimes to Twelve 
Prime Bases}, Math. Comp. 86 (2017), 985--1003.

\bibitem{Spri}
V. G. Sprindzuk,
\emph{Classical Diophantine Equations},
Springer-Verlag, Berlin, 1993.

\bibitem{SpVi}
V. G. Sprindzuk,  A.I. Vinogradov,
\emph{The representation of numbers by binary forms (Russian)},
Matematicheskie Zametki \textbf{3} (1968), 369-376.

\bibitem{StWa}
W. Stein and M. Watkins,
\emph{A database of elliptic curves -- first report},
Algorithmic Number Theory (Sydney, 2002), Lecture Notes in Compute. Sci., vol. 2369, 
Springer, Berlin, 2002, pp. 267--275.
  
\bibitem{Ste}
N. M. Stephens, 
The Birch Swinnerton-Dyer Conjecture for Selmer
curves of positive rank,
\emph{Ph.D. Thesis}, Manchester, 1965.

\bibitem{Tange2011a}
O. Tange,
\emph{GNU Parallel - The Command-Line Power Tool},
;login: The USENIX Magazine, (2011), 42--47.

\bibitem{Th}
A. Thue,
\emph{\"Uber Ann\"aherungswerte algebraischer Zahlen},
J. Reine Angew. Math. 135 (1909), 284--305.

\bibitem{Ti}
R. Tijdeman,
\newblock On the equation of Catalan, 
\newblock {\em Acta Arith.} 29 (1976), 197--209.

\bibitem{TW}
N. Tzanakis and B. M. M. de Weger,
\emph{On the practical solutions of the Thue equation},
J. Number Theory 31 (1989), 99--132.

\bibitem{TW2}
N. Tzanakis and B. M. M. de Weger, 
\emph{Solving a specific Thue-Mahler equation},
Math. Comp. 57 (1991) 799--815.

\bibitem{TW3}
N. Tzanakis and B. M. M. de Weger, 
\newblock How to explicitly solve a Thue-Mahler equation,
\newblock {\em Compositio Math}. 84 (1992), 223--288.

\bibitem{Watetal}
M. Watkins, S. Donnelly, N. D. Elkies, T. Fisher, A. Granville and N. F. Rogers, 
\emph{Ranks of quadratic twists of elliptic curves},
Num\'ero consacr\'e au trimestre ``M\'ethodes arithm\'etiques et applications'', automne 2013, 63--98, Publ. Math. Besan\c{c}on Alg\`ebre Th\'eorie Nr., 2014/2, Presses Univ. Franche-Comt\'e, Besan\c{c}on, 2015. 

\bibitem{Weg0}
B. M. M. de Weger,
\newblock Algorithms for diophantine equations,
\newblock CWI-Tract No. 65, Centre for Mathematics and Computer Science, Amsterdam, 1989.

\bibitem{Weg}
B. M. M. de Weger,
\emph{The weighted sum of two $S$-units being a square},
Indag. Mathem. 1 (1990), 243--262.

\bibitem{Wi}
A. Wiles,
\emph{Modular elliptic curves and Fermat's Last Theorem},
Ann. Math. 141 (1995), 443--551.

\bibitem{Yu0}
P. Yuan,
\newblock On the Diophantine equation $ax^2+by^2=ck^n$,
\newblock {\em Indag. Mathem.} 16 (2) (2005), 301--320.

\bibitem{Yu}
P. Yuan,
\newblock On the diophantine equation $\frac{x^3-1}{x-1}=\frac{y^n-1}{y-1}$,
\newblock {\em J. Number Theory} 112 (2005), 20--25.

\end{thebibliography}

%--------------------------------------------------------------------------------------------------------------------------------------------%

\endinput

Any text after an \endinput is ignored.
You could put scraps here or things in progress.


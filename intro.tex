%% The following is a directive for TeXShop to indicate the main file
%%!TEX root = diss.tex

\chapter{Introduction}
\label{ch:Introduction}

A Diophantine equation is a polynomial equation in several variables defined over the integers. The term \textit{Diophantine} refers to the Greek mathematician Diophantus of Alexandria, who studied such equations in the 3rd century A.D.  Let $f(x_1, \dots, x_n)$ be a polynomial with integer coefficients. We wish to study the set of solutions $(x_1, \dots, x_n) \in \mathbb{Z}^n$ to the equation
\begin{equation}\label{Introduction:Diophantine}
f(x_1, \dots, x_n) = 0.
\end{equation}
There are several different approaches for doing so, arising from three basic
problems concerning Diophantine equations. The first such problem is to
determine whether \eqref{Introduction:Diophantine} has any
solutions. Indeed, one of the most
famous theorems in mathematics states that for $f(x,y,z) = x^n + y^n - z^n$, where $n \geq 3$, there are no
solutions in the positive integers $x,y,z$. This equation is known as Fermat's Last Theorem and was proven by Wiles in 1995. Qualitative questions of this type are often studied using algebraic methods.

Suppose now that \eqref{Introduction:Diophantine} is solvable, that is, has at least one solution. The second basic problem is to determine whether the number of solutions is finite or infinite. For example, consider the \textit{Thue equation}, 
\begin{equation}\label{Introduction:Thue}
f(x,y) = a,
\end{equation}
where $f(x,y)$ is an integral binary form of degree $n \geq 3$ and $a$ is a fixed nonzero rational integer. In 1909, Thue \cite{Th} proved that this equation has only finitely many
solutions. This result followed from a sharpening of Liouville's inequality \cite{Liu}, an
observation that algebraic numbers do not admit very strong approximation by
rational numbers. That is, if $\alpha$ is a real algebraic number of degree
$n \geq 2$ and $p,q$ are integers, Liouville's 
observation states that
\begin{equation}\label{Introduction:Liouville}
\left|\alpha - \frac{p}{q}\right| > \frac{c_1}{q^{n}},
\end{equation}
where $c_1 >0 $ is a value depending explicitly on $\alpha$. The finitude of the number of solutions to \eqref{Introduction:Thue} follows directly from a sharpening of \eqref{Introduction:Liouville} of the type
\begin{equation}\label{Introduction:Sharpening}
\left|\alpha - \frac{p}{q}\right| > \frac{\lambda(q)}{q^{n}}, \quad \lambda(q) \to \infty.
\end{equation}
Indeed, if
$\alpha$ is a real root of $f(x,1)$ and $\alpha^{(i)}$, $i = 1, \dots, n$ are
its conjugates, it follows from \eqref{Introduction:Thue} that
\[\prod_{i=1}^n\left|\alpha^{(i)}-\frac{x}{y}\right| = \frac{a}{|a_0||y|^n}\]
where $a_0$ is the leading coefficient of the polynomial $f(x,1)$. If the Thue equation has integer solutions with arbitrarily large $|y|$, the product $\prod_{i=1}^n|\alpha^{(i)}-x/y|$ must take arbitrarily small values for solutions $x,y$ of \eqref{Introduction:Thue}. As all the $\alpha^{(i)}$ are different, $x/y$ must be correspondingly close to one of the real numbers $\alpha^{(i)}$, say $\alpha$. Thus we obtain
\[\left|\alpha - \frac{x}{y}\right| < \frac{c_2}{|y|^n}\]
where $c_2$ depends only on $a_0$, $n$, and the conjugates $\alpha^{(i)}$ (cf. Chapter 4 of \cite{Spri}). Comparison of this inequality with 
\eqref{Introduction:Sharpening} shows that $|y|$ cannot be arbitrarily large, and so the number of solutions of the Thue equation is finite. Using this argument, an explicit bound can be constructed on the solutions of \eqref{Introduction:Thue} provided that an effective inequality \eqref{Introduction:Sharpening} is known. The sharpening of the Liouville inequality however, especially in effective form, proved to be very difficult. 

In \cite{Th}, Thue published a proof that 
\[\left|\alpha - \frac{p}{q}\right| < \frac{1}{q^{\frac{n}{2} + 1 + \varepsilon}}\]
has only finitely many solutions in integers $p,q > 0$ for all algebraic
numbers $\alpha$ of degree $n \geq 3$ and any $\varepsilon > 0$. In essence, he
obtained the inequality \eqref{Introduction:Sharpening} with
$\lambda(q) = c_3q^{\frac{1}{2}n - 1 - \varepsilon}$, where $c_3 > 0$ depends on $\alpha$ and $\varepsilon$,
thereby confirming that all Thue equations have only finitely many
solutions (cf \cite{Spri}). Unfortunately, Thue's arguments do not allow one to find the
explicit dependence of $c_3$ on $\alpha$ and $\varepsilon$, and so the bound
for the number of solutions of the Thue equation cannot be given in explicit
form either. That is, Thue's proof is ineffective, meaning that it provides no means to find the solutions to \eqref{Introduction:Thue}. 
  
Nonetheless, the investigation of Thue's equation and its generalizations was central to the development of the theory of Diophantine equations in the early 20th century when it was discovered that many Diophantine equations in two unknowns could be reduced to it. In particular, the thorough development and enrichment of Thue's method led Siegel \cite{Sie} to his theorem on the finitude of the number of integral points on an algebraic curve of genus greater than zero. However, as Siegel's result relies on Thue's rational approximation to algebraic numbers, it too is ineffective. 

Shortly following Thue's result, Goormaghtigh \cite{Go} conjectured that the only non-trivial integer solutions of the exponential Diophantine equation
\begin{equation} \label{Introduction:Goormaghtigh}
\frac{x^m-1}{x-1} = \frac{y^n-1}{y-1}
\end{equation}
satisfying $x > y > 1$ and $n,m > 2$ are
\[31 = \frac{2^{5}-1}{2-1} = \frac{5^{3}-1}{5-1} \quad \text{ and } \quad 8191 = \frac{2^{13}-1}{2-1}=\frac{90^{3}-1}{90-1}.\]
These correspond to the known solutions $(x,y,m,n)=(2,5,5,3)$ and $(2,90,13,3)$ to what is nowadays termed {\it Goormaghtigh's equation}. The Diophantine equation \eqref{Introduction:Goormaghtigh} asks for integers having all digits equal to one with respect to two distinct bases, yet whether it has finitely many solutions is still unknown. By fixing the exponents $m$ and $n$ however, Davenport, Lewis, and Schinzel \cite{DaLeSc} were able to prove that \eqref{Introduction:Goormaghtigh} has only finitely many solutions. Unfortunately, this result rests on Siegel's aforementioned finiteness theorem, and is therefore ineffective. 

In 1933, Mahler \cite{Mahler} published a paper on the investigation of the Diophantine equation
\[f(x,y) = p_1^{z_1}\cdots p_v^{z_v}, \quad (x,y) = 1,\] in which
$S= \{p_1, \dots, p_v\}$ denotes a fixed set of prime numbers,
$x,y,z_i \geq 0$, $i = 1, \dots, v$ are unknown integers, and $f(x,y)$ is an
integral irreducible binary form of degree $n \geq 3$. Generalizing the
classical result of Thue, Mahler proved that this equation has only finitely
many solutions. Unfortunately, like Thue, Mahler's argument is also
ineffective.

This leads us to the third basic problem regarding Diophantine equations and the main focus of this thesis: given a solvable Diophantine equation, determine all of its solutions. Until long after Thue's work, no method was known for the construction of bounds for the number of solutions of a Thue equation in terms of the parameters of the equation. Only in 1968 was such a method introduced by Baker \cite{Bak1}, based on his theory of bounds for linear forms in the logarithms of algebraic numbers. Generalizing Baker's ground-breaking result to the $p$-adic case, Sprind\u zuk and Vinogradov \cite{SpVi} and Coates \cite{Coa} proved that the solutions of any \textit{Thue-Mahler equation},
\begin{equation}\label{Introduction:ThueMahler}
f(x,y) = ap_1^{z_1}\cdots p_v^{z_v}, \quad (x,y) = 1,
\end{equation}
where $a$ is a fixed integer, could, at least in principal, be effectively determined.  The first practical method for solving the general Thue-Mahler equation \eqref{Introduction:ThueMahler} over $\mathbb{Z}$ is attributed to Tzanakis and de Weger (cf. \cite{Weg0}, \cite{TW}, \cite{TW2}, \cite{TW3}), whose ideas were inspired in part by the method of Agrawal, Coates, Hunt, and van der Poorten \cite{ACHP} in their work to solve the specific Thue-Mahler equation
\[x^3 - x^2y + xy^2 + y^3 = \pm 11^{z_1}.\]
Using optimized bounds arising from the theory of linear forms in logarithms, a refined, automated version of this explicit method has since been implemented by Hambrook \cite{Ham} as a Magma package \cite{magma}.

As for Goormaghtigh's equation, when $m$ and $n$ are fixed and 
\begin{equation}\label{Introduction:GoormaghtighCondition}
\gcd(m-1,n-1) > 1,
\end{equation}
Davenport, Lewis, and Schinzel \cite{DaLeSc} were able to replace Siegel's result by an effective argument due to Runge. This result was improved by Nesterenko and Shorey \cite{NeSh}, and Bugeaud and Shorey \cite{BuSh} using Baker's theory of linear forms in logarithms. In either case, in order to deduce effectively computable bounds  upon the polynomial variables $x$ and $y$, one must impose the constraints upon $m$ and $n$ that either $m=n+1$, or that the assumption \eqref{Introduction:GoormaghtighCondition} holds. In the extensive literature on this problem, there are a number of striking results that go well beyond what we have mentioned here. By way of example, work of Balasubramanian and 
 Shorey \cite{BaSh} shows that equation \eqref{Introduction:Goormaghtigh} has at most finitely many solutions if we fix only the set of prime divisors of $x$ and $y$, while Bugeaud and Shorey \cite{BuSh} prove an analogous finiteness result, under the additional assumption of \eqref{Introduction:GoormaghtighCondition}, provided the quotient $(m-1)/(n-1)$ is bounded above. Additional results on special cases of equation \eqref{Introduction:Goormaghtigh} are available in, for example, \cite{HeTo}, \cite{Le1}, \cite{Le2},\cite{ShoSur} and \cite{Le3}. 

%---------------------------------------------------------------------------------------------------------------------------------------------%

\section{Statement of the results}

The novel contributions of this thesis concern the development and implementation of efficient algorithms to determine all solutions of certain Goormaghtigh equations and Thue-Mahler equations. In particular, we follow [REF: BeGhKr] to prove that, in fact, under assumption \eqref{Introduction:GoormaghtighCondition}, equation \eqref{Introduction:Goormaghtigh} has at most finitely many solutions which may be found effectively, even if we fix only a single exponent. \\

\begin{theorem}[BeGhKr]\label{IntroductionTheorem:Goormaghtigh1}
If there is a solution in integers $x,y, n$ and $m$ to equation \eqref{Introduction:Goormaghtigh}, satisfying \eqref{Introduction:GoormaghtighCondition}, then
\begin{equation} \label{IntroductionTheorem:Goormaghtigh1Eq}
x <  (3d)^{4n/d} \leq  36^n.
\end{equation}
In particular, if $n$ is fixed, there is an effectively computable constant $c=c(n)$ such that
$\max \{ x, y, m \} < c$.
\end{theorem}
We note that the latter conclusion here follows immediately from \eqref{IntroductionTheorem:Goormaghtigh1Eq}, in conjunction with, for example, work of Baker ([REF]). The constants present in our upper bound \eqref{IntroductionTheorem:Goormaghtigh1Eq} may be sharpened somewhat at the cost of increasing the complexity of our argument. By refining our approach, in conjunction with some new results from computational Diophantine approximation, we are able to achieve the complete solution of equation \eqref{Introduction:Goormaghtigh}, subject to condition \eqref{Introduction:GoormaghtighCondition}, for small fixed values of $n$. \\

\begin{theorem}[BeGhKr] \label{IntroductionTheorem:Goormaghtigh2Eq}
If there is a solution in integers $x,y$ and $m$ to equation \eqref{Introduction:Goormaghtigh}, with $n \in \{ 3, 4, 5 \}$ and satisfying \eqref{Introduction:GoormaghtighCondition}, then
\[(x,y,m,n) = (2,5,5,3)  \; \mbox{ and } \; (2,90,13,3).\]
\end{theorem}

In the case $n = 5$ of Theorem \eqref{IntroductionTheorem:Goormaghtigh2Eq} ``off-the-shelf'' techniques for finding integral points on models of elliptic curves or for solving {\it Ramanujan-Nagell} equations of the shape $F(x)=z^n$ (where $F$ is a polynomial and $z$ a fixed integer) do not apparently permit the full resolution of this problem in a reasonable amount of time. Instead, we sharpen the existing techniques of [TdW] and [Hambrook] for solving Thue-Mahler equations and specialize them to this problem. 

A direct consequence and primary motivation for developing an efficient Thue-Mahler algorithm is the computation of elliptic curves over $\mathbb{Q}$. Let $S$ be a finite set of rational primes. In $1963$, Shafarevich [CITE] proved that there are at most finitely many $\mathbb{Q}$-isomorphism classes of elliptic curves defined over $\mathbb{Q}$ having good reduction outside $S$. The first effective proof of this statement was provided by Coates [CITE] in 1970 for the case $K = \mathbb{Q}$ and $S = \{2,3\}$ using bounds for linear forms in $p$-adic and complex logarithms. Early attempts to make these results explicit for fixed sets of small primes overlap with the arguments of [COATES], in that they reduce the problem to that of solving a number of degree $3$ Thue-Mahler equations of the form
\[F(x,y) = au,\]
where $u$ is an integer whose prime factors all lie in $S$. 

In the $1950$'s and $1960$'s, Taniyama and Weil asked whether all elliptic curves over $\mathbb{Q}$ of a given conductor $N$ are related to modular functions. While this conjecture is now known as the Modularity Theorem, until its proof in $2001$ \cite{Breuil}, attempts to verify it sparked a large effort to tabulate all elliptic curves over $\mathbb{Q}$ of given conductor $N$. In 1966, Ogg (\cite{Ogg1}, \cite{Ogg2}) determined all elliptic curves defined over $\mathbb{Q}$ with conductor of the form $2^a$. Coghlan, in his dissertation \cite{Coghlan}, studied the curves of conductor $2^a3^b$ independently of Ogg, while Setzer \cite{Setzer} computed all $\mathbb{Q}$-isomorphism classes of elliptic curves of conductor $p$ for certain small primes $p$. Each of these examples corresponds, via the [BR] approach, to cases with reducible forms. The first analysis on irreducible forms in \eqref{Eq:tmEquation} was carried out by Agrawal, Coates, Hunt and van der Poorten \cite{Agrawal}, who determined all elliptic curves of conductor $11$ defined over $\mathbb{Q}$ to verify the (then) conjecture of Taniyama-Weil.

There are very few, if any, subsequent attempts in the literature to find elliptic curves of given conductor via Thue-Mahler equations. Instead, many of the approaches involve a completely different method to the problem, using modular forms. This method relies upon the Modularity Theorem of Breuil, Conrad, Diamond and Taylor \cite{Breuil}, which was still a conjecture (under various guises) when these ideas were first implemented. Much of the success of this approach can be attributed to Cremona (see e.g. \cite{Cremona1}, \cite{Cremona2}) and his collaborators, who have devoted decades of work to it. In fact, using this method, all elliptic curves over $\mathbb{Q}$ of conductor $N$ have been determined for values of $N$ as follows
\begin{itemize} \itemsep0em
\item Antwerp IV ($1972$): $N \leq 200$
\item Tingley ($1975$): $N \leq 320$
\item Cremona ($1988$): $N \leq 600$
\item Cremona ($1990$): $N \leq 1000$
\item Cremona ($1997$): $N \leq 5077$
\item Cremona ($2001$): $N \leq 10000$
\item Cremona ($2005$): $N \leq 130000$
\item Cremona ($2014$): $N \leq 350000$
\item Cremona ($2015$): $N \leq 364000$
\item Cremona ($2016$): $N \leq 390000$.
\end{itemize}

In this thesis, we follow [BeGhRe] wherein we return to techniques based upon solving Thue-Mahler equations, using a number of results from classical invariant theory. In particular, we illustrate the connection between elliptic curves over $\mathbb{Q}$ and cubic forms and subsequently describe an effective algorithm for determining all elliptic curves over $\mathbb{Q}$ having good reduction outside $S$. This result can be summarized as follows. If we wish to find an elliptic curves $E$ of conductor $N = p_1^{a_1}\cdots p_v^{a_v}$ for some $a_i \in \mathbb{N}$, by Theorem 1 of [BeGhRe], there exists an integral binary cubic form $F$ of discriminant $N_0 \mid 12 N$ and relatively prime integers $u,v$ satisfying
\[F(u,v) = w_0u^3 + w_1u^2v + w_2uv^2 + w_3v^3 = 2^{\alpha_1}3^{\beta_1}\prod_{p|N_0}p^{\kappa_p}\]
for some $\alpha_1, \beta_1, \kappa_p$. Then $E$ is isomorphic over $\mathbb{Q}$ to the elliptic curve $E_{\mathcal{D}}$, where $E_{\mathcal{D}}$ is determined by the form $F$ and $(u,v)$. It is worth noting that Theorem 1 of [BeGhRe] very explicitly describes how to generate $E_{\mathcal{D}}$; once a solution $(u,v)$ to the Thue-Mahler equation $F$ is known, a quick computation of the Hessian and Jacobian discriminant of $F$ evaluated at $(u,v)$ yields the coefficients of $E_{\mathcal{D}}$. Using this theorem, all $E/\mathbb{Q}$ of conductor $N$ may be computed by generating all of the relevant binary cubic forms, solving the corresponding Thue-Mahler equations, and outputting the elliptic curves that arise. The first and last steps of this process are straightforward. Indeed, Bennett and Rechnitzer describe an efficient algorithm for carrying out the first step \aaron{REF}. In fact, they having carried out a one-time computation of all irreducible forms that can arise in Theorem 1 of absolute discriminant bounded by $10^{10}$. The bulk of the work is therefore concentrated in step 2, solving a large number of degree 3 Thue-Mahler equations. 

Unfortunately, despite many refinements, [Hambrook's] MAGMA implementation of a Thue-Mahler solver encounters a multitude of bottlenecks which often yield unavoidable timing and memory problems, even when parallelization is considered. As our aim is to use the results of [BeGhRe] to generate all elliptic curves over $\mathbb{Q}$ of conductor $N < 10^6$, in its current state, the Hambrook algorithm is inefficient for this task, and in many cases, simply unusable due to its memory requirements. The main novel contribution of this thesis is therefore the efficient resolution of an arbitrary degree $3$ Thue-Mahler equation and the implementation of this algorithm as a MAGMA package. This work is based on ideas of Matshke, von Kanel [CITE], and Siksek and is summarized in the following steps.



% OLD INTRO

%A Thue-Mahler equation is a Diophantine equation of the form
%\begin{equation} \label{Eq:tmEquation}
%F(x,y) = ap_1^{z_1}\cdots p_v^{z_v},
%\end{equation}
%where
%\[F(x,y) = f_0x^n + f_1x^{n-1}y + \cdots + f_{n-1}xy^{n-1} + f_ny^n\]
%is an irreducible binary form of degree at least $3$, $a$ is a nonzero integer, and $p_1, \dots, p_v$ are rational primes. The number of solutions in relatively prime integers $x$ and $y$ is known to be finite via the work of Mahler [CITE]. Though this proof is ineffective, it generalizes a classical result of Thue [CITE], who had proved an analogous statement for equations of the form $F(x,y) = a$. In the mid-1960's, Baker [CITE] proved his ground-breaking results on effective lower bounds for linear forms in logarithms of algebraic numbers. 
%
%Generalizing Baker's result to the $p$-adic case, Sprind\u zuk and Vinogradov [CITE] and Coates [CITE] proved that the solutions of any Thue-Mahler equation could, at least in principal, be effectively determined. The first practical method for solving the general Thue-Mahler equation over $\mathbb{Z}$ is attributed to Tzanakis and de Weger [CITE], whose ideas were inspired in part by the method of Agrawal, Coates, Hunt, and van der Poorten [CITE] in their work to solve the specific Thue-Mahler equation
%\[x^3 - x^2y + xy^2 + y^3 = \pm 11^{z_1}.\]

%In 2011, using optimized bounds arising from the theory of linear forms in logarithms, Hambrook [CITE] implemented a refined, automated version of the explicit method of [TdW] as a MAGMA package. Similar to [TdW], this algorithm begins by reducing the problem to one of solving a collection of finitely many $S$-unit equations in a certain algebraic number field $K$. Of course, by an $S$-unit, we mean an integer whose prime factors all lie in $S$. For each such equation, a very large upper bound on the solutions is generated using the theory of linear forms in logarithms. This bound is then reduced via Diophantine approximation techniques. Finally, the algorithm searches below this reduced bound using a combination of clever sieves and brute force. 
%
%Unfortunately, despite Hambrook's refinements, this algorithm encounters a multitude of bottlenecks which often yield unavoidable timing and memory problems, even when parallelization is considered. As we will outline shortly, one of our primary aims in this thesis is to solve a very large number of Thue-Mahler equations. In its current state, the Hambrook algorithm is inefficient for this task, and in many cases, simply unusable due to its memory requirements. The main novel contribution of this thesis is therefore the efficient resolution of an arbitrary Thue-Mahler equation and the implementation of this algorithm as a MAGMA package. This work is based on ideas of Matshke, von Kanel [CITE], and Siksek and is summarized in the following steps.

\textbf{Step 1.}
Following [TdW] and [Hambrook], we reduce the problem of solving the given Thue-Mahler equation to the problem of solving a collection of finitely many $S$-unit equations in a certain algebraic number field $K$. These are equations of the form
\begin{equation}\label{Introduction:SUnit}
\mu_0 y - \lambda_0 x = 1
\end{equation}
for some $\mu_0, \lambda_0 \in K$ and unknowns $x,y$. The collection of forms is such that if we know the solutions of each equation in the collection, then we can easily derive all of the solutions of the Thue-Mahler equation. This reduction is performed in two steps. First, \eqref{Introduction:ThueMahler} is reduced to a finite number of ideal equations over $K$. Here, we employ new results by Siksek [Cite?] to significantly reduce the number of ideal equations to consider. Next, we reduce each ideal equation to a number of certain $S$-unit equations \eqref{Introduction:SUnit} via a finite number of principalization tests. The method of [TdW] reduces \eqref{Introduction:ThueMahler} to $(m/2) h^v$ $S$-unit equations, where $m$ is the number of roots of unity of $K$, $h$ is the class number, and $v$ is the number of rational primes $p_1, \dots, p_v$. The method of Siksek that we employ gives only $m/2$ $S$-unit equations. The principle computational work here consists of computing an integral basis, a system of fundamental units, and a splitting field of $K$, as well as computing the class group of $K$ and the factorizations of the primes $p_1,\dots,p_v$ into prime ideals in the ring of integers of $K$. 

The remaining steps are performed for each of the $S$-unit equations in our collection. 

\textbf{Step 2.}
In place of the logarithmic sieves used in [TdW] to derive a large upper bound, we work with the global logarithmic Weil height
\[h: \mathbb{G}_m(\overline{\mathbb{Q}}) \to \mathbb{R}_{\geq 0}.\]
For a given \eqref{Introduction:SUnit}, we show that the height $h(1/x)$ admits a decomposition into local heights at each place of $K$ appearing in the $S$-unit equation. Using [CITE : Matshke, von Kanel], we generate a very large upper bound on the height $h(1/x)$, and subsequently, on the local heights. This step is a straightforward computation, whereas the analogous step in Hambrook and TdW is a complex and lengthy derivation which involves factoring rational primes into prime ideals in a splitting field of $K$ and computing heights of certain elements of the splitting field. 

\textbf{Step 3.}
For each place of $K$ appearing in \eqref{Introduction:SUnit}, we drastically reduce the upper bounds derived in Step 2 by using computational Diophantine approximation techniques applied to the intersection of a certain ellipsoid and translated lattice. This technique involves using the Finke-Pohst algorithm to enumerate all short vectors in the intersection. Here, working with the Weil height $h(1/x)$ has the advantage that it leads to ellipsoids whose volumes are smaller than the ellipsoids implicitly used in [TdW] by a factor of $\sim r^{r/2}$ for $r$ the number of places of $K$ appearing in our $S$-unit equation. In this way, we reduce the number of short vectors appearing from the Fincke-Pohst algorithm, and consequently reduce our running time and memory requirements. 

\textbf{Step 5.}
Finally, we use a sieving procedure to find all the solutions of the Diophantine equation that live in the box defined by the bounds derived in the previous steps. To carry out this step, we run through all the possible solutions in the box and �sieve� out the vast majority of non-solutions. This is done via certain low-cost congruence tests. The candidate solutions passing this test are then verified directly against \eqref{Introduction:SUnit}. Though we expect the bounds defining the box to be small, there can still be a very large number of possible solutions to check, especially if the number of rational primes involved in the Thue-Mahler equation is large. The computations performed on each individual candidate solution are relatively simple, but the sheer number of candidates often makes this step the computational bottleneck of the entire algorithm. 

\textbf{Step 6.}
Having performed Steps 2-5 for each $S$-unit equation in our collection, we now have all the solutions of each such equation, and we use this knowledge to determine all the solutions of the Thue-Mahler equation. 

The reader will notice several parallels between this refined algorithm and the aforementioned Goormaghtigh equation solver in the case $n=5$. In particular, both algorithms share the same setup and refinements of the [TdW] and [Hambrook] solver. For \eqref{Introduction:Goormaghtigh}, however, we are left to solve
\[f(y) = x^m,\]
a Thue-Mahler-like equation of degree $4$ in explicit values of $x$ and unknown integers $y$ and $m$. In this case, we are permitted simplifications which allow us to omit the Fincke-Pohst algorithm and final congruence sieves. Instead, for each $x$, we rely on only a few iterations of the LLL algorithm to reduce our initial bound on the exponents before entering a naive search to complete our computation. Of course, this algorithm can be refined further for efficiency, however, in the context of [BeGhKr], such improvements are not needed. 

The outline of this thesis is as follows. ADD


\edit{Intro from Goormaghtigh:}
More than a century ago, Ratat \cite{Ra} and Goormaghtigh \cite{Go} observed the identities
$$
31 = \frac{2^5-1}{2-1} = \frac{5^3-1}{5-1} \; \; \mbox{ and } \; \; 8191 = \frac{2^{13}-1}{2-1} = \frac{90^3-1}{90-1}.
$$
These correspond to the known solutions $(x,y,m,n)=(2,5,5,3)$ and $(2,90,13,3)$
to what is nowadays termed {\it Goormaghtigh's equation}
\begin{equation} \label{eq-main}
\frac{x^m-1}{x-1} = \frac{y^n-1}{y-1}, \; \; y>x>1, \; m > n > 2.
\end{equation}
This is a classical example of a {\it polynomial-exponential equation} and shares a number of common characteristics with other frequently-studied Diophantine equations of this type, such as those of 
Catalan
\begin{equation} \label{Cat}
x^m-y^n=1, \; \; x, y, m, n > 1,
\end{equation}
and 
Nagell-Ljunggren 
\begin{equation} \label{NL}
\frac{x^m-1}{x-1} = y^n, \; \; x, y, n >1, \; m > 2.
\end{equation}
In a certain sense, however, equation (\ref{eq-main}) appears to be rather harder to treat than (\ref{Cat}) or (\ref{NL}). Techniques from Diophantine approximation (specifically, bounds for linear forms in complex and $p$-adic logarithms) have been applied by Tijdeman \cite{Ti} to show that equation (\ref{Cat}) has at most finitely many solutions in the four variables $x,y, m$ and $n$ (a result subsequently sharpened by Mih{\u{a}}ilescu \cite{Mih} via a different method to solve (\ref{Cat}) completely). Similarly, Shorey and Tijdeman \cite{ShTi} showed that equation (\ref{NL}) has at most finitely many solutions if any one of the variables $x, y$ or $m$ is fixed (though we do not have a like result for fixed odd $n$; the case where $n$ is even was resolved earlier by Nagell \cite{Nag} and Ljunggren \cite{Lju}). In the case of equation (\ref{eq-main}), on the other hand, to obtain finiteness results with current technology, we apparently need to assume that two of the variables $x, y, m$ and $n$ are fixed (see \cite{BaSh} for references). In addition, if the two variables fixed are the exponents $m$ and $n$, then in order to deduce effectively computable bounds upon the polynomial variables $x$ and $y$, via either Runge's method (Davenport, Lewis and Schinzel \cite{DaLeSc}) or from bounds upon linear forms in logarithms (see e.g. Nesterenko and Shorey \cite{NeSh}, and Bugeaud and Shorey \cite{BuSh}), we require constraints upon $m$ and $n$, that either $m=n+1$, or that
 \begin{equation} \label{condition}
 \gcd (m-1,n-1) = d > 1.
 \end{equation}
 In the extensive literature on this problem, there are a number of striking results that go well beyond what we have mentioned here. By way of example, work of Balasubramanian and 
 Shorey \cite{BaSh} shows that equation (\ref{eq-main}) has at most finitely many solutions if we fix only the set of prime divisors of $x$ and $y$, while Bugeaud and Shorey \cite{BuSh} prove an analogous finiteness result, under the additional assumption of (\ref{condition}), provided the quotient $(m-1)/(n-1)$ is bounded above. Additional results on special cases of equation (\ref{eq-main}) are available in, for example, \cite{HeTo}, \cite{Le1}, \cite{Le2} and \cite{Le3}.  An excellent overview of results on this problem  can be found in the survey of Shorey \cite{ShoSur}.
 
In this paper, we prove that, in fact, under assumption (\ref{condition}), equation (\ref{eq-main}) has at most finitely many solutions which may be found effectively, even if we fix only a single exponent.
\begin{theorem} \label{main-thm}
If there is a solution in integers $x,y, n$ and $m$ to equation (\ref{eq-main}), satisfying (\ref{condition}), then
\begin{equation} \label{hive}
x <  (3d)^{4n/d} \leq  36^n.
\end{equation}
In particular, if $n$ is fixed, there is an effectively computable constant $c=c(n)$ such that
$\max \{ x, y, m \} < c$.
\end{theorem}
We note that the latter conclusion here follows immediately from (\ref{hive}), in conjunction with, for example, work of Baker \cite{Bak}.
The constants present in our upper bound (\ref{hive}) may be sharpened somewhat at the cost of increasing the complexity of our argument. By refining our approach, in conjunction with some new results from computational Diophantine approximation, we are able to achieve the complete solution of equation (\ref{eq-main}), subject to condition (\ref{condition}),  for small fixed values of $n$.

\begin{theorem} \label{main-thm2}
If there is a solution in integers $x,y$ and $m$ to equation (\ref{eq-main}), with $n \in \{ 3, 4, 5 \}$ and satisfying (\ref{condition}), then
$$
(x,y,m,n) = (2,5,5,3)  \; \mbox{ and } \; (2,90,13,3).
$$
\end{theorem}

Essentially half of the current paper is concerned with developing Diophantine approximation machinery for the case $n=5$ in Theorem \ref{main-thm2}. Here, ``off-the-shelf'' techniques for finding integral points on models of elliptic curves or for solving {\it Ramanujan-Nagell} equations of the shape $F(x)=z^n$ (where $F$ is a polynomial and $z$ a fixed integer) do not apparently permit the full resolution of this problem in a reasonable amount of time. The new ideas introduced here are explored more fully in the general setting of {\it Thue-Mahler} equations in the forthcoming paper  \cite{GhKaMaSi}. These are polynomial-exponential equations of the form $F(x,y)=p_1^{\alpha_1} \cdots p_k^{\alpha_k}$ where $F$ is a binary form of degree three or greater and $p_1, \ldots, p_k$ are fixed rational primes. Here, we take this opportunity to specialize these refinements to the case of Ramanujan-Nagell equations, and to introduce some further sharpenings which enable us to complete the proof of Theorem \ref{main-thm2}.

We observe that, in case $n=3$, Theorem \ref{main-thm2} was obtained earlier by Yuan \cite{Yu} (see also He \cite{He}).  The techniques employed in both \cite{He} and \cite{Yu}, however, depend essentially upon the fact that $n=3$ (whereby $n-1=2$ and one can appeal to specialized techniques from the theory of quadratic fields) and cannot apparently be generalized to other values of $n$.

The title of this paper reflects the fact that the machinery of Pad\'e approximation to binomial functions has been applied to the problem of solving equation (\ref{eq-main}) in earlier work of Bugeaud and Shorey \cite{BuSh}. We will employ these tools  here rather differently.

The outline of this paper is as follows. In Section \ref{proof}, we derive ``good'' rational approximations to certain algebraic numbers associated to solutions of (\ref{eq-main}). Section \ref{Pade} contains relevant details about Pad\'e approximation to the binomial function. In Sections \ref{sec-main-thm} and \ref{sec-main-thm2}, we find the proofs of Theorems \ref{main-thm} and \ref{main-thm2}, respectively. In the latter case, to treat small fixed values of $n$ and $x$ in equation (\ref{eq-main}), we appeal to a variety of techniques from computational Diophantine approximation. Most interestingly, in case $n=5$, we sharpen existing techniques for solving  Thue-Mahler equations, and specialize them to our problem. We note that this section may essentially be read independently of the rest of the paper. For each $x$, we restrict the problem to that of solving a number of related $S$-unit equations, where $S$ is the set of primes dividing $x$. We then generate a large upper bound on the exponents of these equations using bounds for linear forms in logarithms, both Archimedean and non-Archimedean. Finally, unlike traditional examples of Thue-Mahler equations, where extensive use of geometric and $p$-adic reduction techniques are typically required, using only a few iterations of the LLL algorithm, we reduce this bound significantly, after which we apply a naive search to complete our computation. We will, in fact, employ  two quite different algorithms for solving Thue-Mahler equations, one for which we must compute the class group of a number field and one which avoids this computation altogether. For a given value of $x$, one of these versions may be significantly faster than the other; we list some timings for examples to illustrate this difference. 




%--------------------------------------------------------------------------------------------------------------------------------------------%
%--------------------------------------------------------------------------------------------------------------------------------------------%

\endinput

Any text after an \endinput is ignored.
You could put scraps here or things in progress.



This document provides a quick set of instructions for using the
\class{ubcdiss} class to write a dissertation in \LaTeX. 
Unfortunately this document cannot provide an introduction to using
\LaTeX.  The classic reference for learning \LaTeX\ is
\citeauthor{lamport-1994-ladps}'s
book~\cite{lamport-1994-ladps}.  There are also many freely-available
tutorials online;
\webref{http://www.andy-roberts.net/misc/latex/}{Andy Roberts' online
    \LaTeX\ tutorials}
seems to be excellent.
The source code for this docment, however, is intended to serve as
an example for creating a \LaTeX\ version of your dissertation.

We start by discussing organizational issues, such as splitting
your dissertation into multiple files, in
\autoref{sec:SuggestedThesisOrganization}.
We then cover the ease of managing cross-references in \LaTeX\ in
\autoref{sec:CrossReferences}.
We cover managing and using bibliographies with \BibTeX\ in
\autoref{sec:BibTeX}. 
We briefly describe typesetting attractive tables in
\autoref{sec:TypesettingTables}.
We briefly describe including external figures in
\autoref{sec:Graphics}, and using special characters and symbols
in \autoref{sec:SpecialSymbols}.
As it is often useful to track different versions of your dissertation,
we discuss revision control further in
\autoref{sec:DissertationRevisionControl}. 
We conclude with pointers to additional sources of information in
\autoref{sec:Conclusions}.

%--------------------------------------------------------------------------------------------------------------------------------------------%
%--------------------------------------------------------------------------------------------------------------------------------------------%
\section{Suggested Thesis Organization}
\label{sec:SuggestedThesisOrganization}

The \acs{UBC} \acf{GPS} specifies a particular arrangement of the
components forming a thesis.\footnote{See
    \url{http://www.grad.ubc.ca/current-students/dissertation-thesis-preparation/order-components}}
This template reflects that arrangement.

In terms of writing your thesis, the recommended best practice for
organizing large documents in \LaTeX\ is to place each chapter in
a separate file.  These chapters are then included from the main
file through the use of \verb+\include{file}+.  A thesis might
be described as six files such as \file{intro.tex},
\file{relwork.tex}, \file{model.tex}, \file{eval.tex},
\file{discuss.tex}, and \file{concl.tex}.

We also encourage you to use macros for separating how something
will be typeset (\eg bold, or italics) from the meaning of that
something. 
For example, if you look at \file{intro.tex}, you will see repeated
uses of a macro \verb+\file{}+ to indicate file names.
The \verb+\file{}+ macro is defined in the file \file{macros.tex}.
The consistent use of \verb+\file{}+ throughout the text not only
indicates that the argument to the macro represents a file (providing
meaning or semantics), but also allows easily changing how
file names are typeset simply by changing the definition of the
\verb+\file{}+ macro.
\file{macros.tex} contains other useful macros for properly typesetting
things like the proper uses of the latinate \emph{exempli grati\={a}}
and \emph{id est} (\ie \verb+\eg+ and \verb+\ie+), 
web references with a footnoted \acs{URL} (\verb+\webref{url}{text}+),
as well as definitions specific to this documentation
(\verb+\latexpackage{}+).

%--------------------------------------------------------------------------------------------------------------------------------------------%
\section{Making Cross-References}
\label{sec:CrossReferences}

\LaTeX\ make managing cross-references easy, and the \latexpackage{hyperref}
package's\ \verb+\autoref{}+ command\footnote{%
    The \latexpackage{hyperref} package is included by default in this
    template.}
makes it easier still. 

A thing to be cross-referenced, such as a section, figure, or equation,
is \emph{labelled} using a unique, user-provided identifier, defined
using the \verb+\label{}+ command.  
The thing is referenced elsewhere using the \verb+\autoref{}+ command.
For example, this section was defined using:
\begin{lstlisting}
    \section{Making Cross-References}
    \label{sec:CrossReferences}
\end{lstlisting}
References to this section are made as follows:
\begin{lstlisting}
    We then cover the ease of managing cross-references in \LaTeX\
    in \autoref{sec:CrossReferences}.
\end{lstlisting}
\verb+\autoref{}+ takes care of determining the \emph{type} of the 
thing being referenced, so the example above is rendered as
\begin{quote}
    We then cover the ease of managing cross-references in \LaTeX\
    in \autoref{sec:CrossReferences}.
\end{quote}

The label is any simple sequence of characters, numbers, digits,
and some punctuation marks such as ``:'' and ``--''; there should
be no spaces.  Try to use a consistent key format: this simplifies
remembering how to make references.  This document uses a prefix
to indicate the type of the thing being referenced, such as \texttt{sec}
for sections, \texttt{fig} for figures, \texttt{tbl} for tables,
and \texttt{eqn} for equations.

For details on defining the text used to describe the type
of \emph{thing}, search \file{diss.tex} and the \latexpackage{hyperref}
documentation for \texttt{autorefname}.


%--------------------------------------------------------------------------------------------------------------------------------------------%
\section{Managing Bibliographies with \BibTeX}
\label{sec:BibTeX}

One of the primary benefits of using \LaTeX\ is its companion program,
\BibTeX, for managing bibliographies and citations.  Managing
bibliographies has three parts: (i) describing references,
(ii)~citing references, and (iii)~formatting cited references.

\subsection{Describing References}

\BibTeX\ defines a standard format for recording details about a
reference.  These references are recorded in a file with a
\file{.bib} extension.  \BibTeX\ supports a broad range of
references, such as books, articles, items in a conference proceedings,
chapters, technical reports, manuals, dissertations, and unpublished
manuscripts. 
A reference may include attributes such as the authors,
the title, the page numbers, the \ac{DOI}, or a \ac{URL}.  A reference
can also be augmented with personal attributes, such as a rating,
notes, or keywords.

Each reference must be described by a unique \emph{key}.\footnote{%
    Note that the citation keys are different from the reference
    identifiers as described in \autoref{sec:CrossReferences}.}
A key is a simple sequence of characters, numbers, digits, and some
punctuation marks such as ``:'' and ``--''; there should be no spaces. 
A consistent key format simiplifies remembering how to make references. 
For example:
\begin{quote}
   \fbox{\emph{last-name}}\texttt{-}\fbox{\emph{year}}\texttt{-}\fbox{\emph{contracted-title}}
\end{quote}
where \emph{last-name} represents the last name for the first author,
and \emph{contracted-title} is some meaningful contraction of the
title.  Then \citeauthor{kiczales-1997-aop}'s seminal article on
aspect-oriented programming~\cite{kiczales-1997-aop} (published in
\citeyear{kiczales-1997-aop}) might be given the key
\texttt{kiczales-1997-aop}.

An example of a \BibTeX\ \file{.bib} file is included as
\file{biblio.bib}.  A description of the format a \file{.bib}
file is beyond the scope of this document.  We instead encourage
you to use one of the several reference managers that support the
\BibTeX\ format such as
\webref{http://jabref.sourceforge.net}{JabRef} (multiple platforms) or
\webref{http://bibdesk.sourceforge.net}{BibDesk} (MacOS\,X only). 
These front ends are similar to reference manages such as
EndNote or RefWorks.


\subsection{Citing References}

Having described some references, we then need to cite them.  We
do this using a form of the \verb+\cite+ command.  For example:
\begin{lstlisting}
    \citet{kiczales-1997-aop} present examples of crosscutting 
    from programs written in several languages.
\end{lstlisting}
When processed, the \verb+\citet+ will cause the paper's authors
and a standardized reference to the paper to be inserted in the
document, and will also include a formatted citation for the paper
in the bibliography.  For example:
\begin{quote}
    \citet{kiczales-1997-aop} present examples of crosscutting 
    from programs written in several languages.
\end{quote}
There are several forms of the \verb+\cite+ command (provided
by the \latexpackage{natbib} package), as demonstrated in
\autoref{tbl:natbib:cite}.
Note that the form of the citation (numeric or author-year) depends
on the bibliography style (described in the next section).
The \verb+\citet+ variant is used when the author names form
an object in the sentence, whereas the \verb+\citep+ variant
is used for parenthetic references, more like an end-note.
Use \verb+\nocite+ to include a citation in the bibliography
but without an actual reference.
\nocite{rowling-1997-hpps}
\begin{table}
\caption{Available \texttt{cite} variants; the exact citation style
    depends on whether the bibliography style is numeric or author-year.}
\label{tbl:natbib:cite}
\centering
\begin{tabular}{lp{3.25in}}\toprule
Variant & Result \\
\midrule
% We cheat here to simulate the cite/citep/citet for APA-like styles
\verb+\cite+ & Parenthetical citation (\eg ``\cite{kiczales-1997-aop}''
    or ``(\citeauthor{kiczales-1997-aop} \citeyear{kiczales-1997-aop})'') \\
\verb+\citet+ & Textual citation: includes author (\eg
    ``\citet{kiczales-1997-aop}'' or
    or ``\citeauthor{kiczales-1997-aop} (\citeyear{kiczales-1997-aop})'') \\
\verb+\citet*+ & Textual citation with unabbreviated author list \\
\verb+\citealt+ & Like \verb+\citet+ but without parentheses \\
\verb+\citep+ & Parenthetical citation (\eg ``\cite{kiczales-1997-aop}''
    or ``(\citeauthor{kiczales-1997-aop} \citeyear{kiczales-1997-aop})'') \\
\verb+\citep*+ & Parenthetical citation with unabbreviated author list \\
\verb+\citealp+ & Like \verb+\citep+ but without parentheses \\
\verb+\citeauthor+ & Author only (\eg ``\citeauthor{kiczales-1997-aop}'') \\
\verb+\citeauthor*+ & Unabbreviated authors list 
    (\eg ``\citeauthor*{kiczales-1997-aop}'') \\
\verb+\citeyear+ & Year of citation (\eg ``\citeyear{kiczales-1997-aop}'') \\
\bottomrule
\end{tabular}
\end{table}

\subsection{Formatting Cited References}

\BibTeX\ separates the citing of a reference from how the cited
reference is formatted for a bibliography, specified with the
\verb+\bibliographystyle+ command. 
There are many varieties, such as \texttt{plainnat}, \texttt{abbrvnat},
\texttt{unsrtnat}, and \texttt{vancouver}.
This document was formatted with \texttt{abbrvnat}.
Look through your \TeX\ distribution for \file{.bst} files. 
Note that use of some \file{.bst} files do not emit all the information
necessary to properly use \verb+\citet{}+, \verb+\citep{}+,
\verb+\citeyear{}+, and \verb+\citeauthor{}+.

There are also packages available to place citations on a per-chapter
basis (\latexpackage{bibunits}), as footnotes (\latexpackage{footbib}),
and inline (\latexpackage{bibentry}).
Those who wish to exert maximum control over their bibliography
style should see the amazing \latexpackage{custom-bib} package.

%--------------------------------------------------------------------------------------------------------------------------------------------%
\section{Typesetting Tables}
\label{sec:TypesettingTables}

\citet{lamport-1994-ladps} made one grievous mistake
in \LaTeX: his suggested manner for typesetting tables produces
typographic abominations.  These suggestions have unfortunately
been replicated in most \LaTeX\ tutorials.  These
abominations are easily avoided simply by ignoring his examples
illustrating the use of horizontal and vertical rules (specifically
the use of \verb+\hline+ and \verb+|+) and using the
\latexpackage{booktabs} package instead.

The \latexpackage{booktabs} package helps produce tables in the form
used by most professionally-edited journals through the use of
three new types of dividing lines, or \emph{rules}.
% There are times that you don't want to use \autoref{}
Tables~\ref{tbl:natbib:cite} and~\ref{tbl:LaTeX:Symbols} are two
examples of tables typeset with the \latexpackage{booktabs} package.
The \latexpackage{booktabs} package provides three new commands
for producing rules:
\verb+\toprule+ for the rule to appear at the top of the table,
\verb+\midrule+ for the middle rule following the table header,
and \verb+\bottomrule+ for the bottom-most at the end of the table.
These rules differ by their weight (thickness) and the spacing before
and after.
A table is typeset in the following manner:
\begin{lstlisting}
    \begin{table}
    \caption{The caption for the table}
    \label{tbl:label}
    \centering
    \begin{tabular}{cc}
    \toprule
    Header & Elements \\
    \midrule
    Row 1 & Row 1 \\
    Row 2 & Row 2 \\
    % ... and on and on ...
    Row N & Row N \\
    \bottomrule
    \end{tabular}
    \end{table}
\end{lstlisting}
See the \latexpackage{booktabs} documentation for advice in dealing with
special cases, such as subheading rules, introducing extra space
for divisions, and interior rules.

%--------------------------------------------------------------------------------------------------------------------------------------------%
\section{Figures, Graphics, and Special Characters}
\label{sec:Graphics}

Most \LaTeX\ beginners find figures to be one of the more challenging
topics.  In \LaTeX, a figure is a \emph{floating element}, to be
placed where it best fits.
The user is not expected to concern him/herself with the placement
of the figure.  The figure should instead be labelled, and where
the figure is used, the text should use \verb+\autoref+ to reference
the figure's label.
\autoref{fig:latex-affirmation} is an example of a figure.
\begin{figure}
    \centering
    % For the sake of this example, we'll just use text
    %\includegraphics[width=3in]{file}
    \Huge{\textsf{\LaTeX\ Rocks!}}
    \caption{Proof of \LaTeX's amazing abilities}
    \label{fig:latex-affirmation}   % label should change
\end{figure}
A figure is generally included as follows:
\begin{lstlisting}
    \begin{figure}
    \centering
    \includegraphics[width=3in]{file}
    \caption{A useful caption}
    \label{fig:fig-label}   % label should change
    \end{figure}
\end{lstlisting}
There are three items of note:
\begin{enumerate}
\item External files are included using the \verb+\includegraphics+
    command.  This command is defined by the \latexpackage{graphicx} package
    and can often natively import graphics from a variety of formats.
    The set of formats supported depends on your \TeX\ command processor.
    Both \texttt{pdflatex} and \texttt{xelatex}, for example, can
    import \textsc{gif}, \textsc{jpg}, and \textsc{pdf}.  The plain
    version of \texttt{latex} only supports \textsc{eps} files.

\item The \verb+\caption+ provides a caption to the figure. 
    This caption is normally listed in the List of Figures; you
    can provide an alternative caption for the LoF by providing
    an optional argument to the \verb+\caption+ like so:
    \begin{lstlisting}
    \caption[nice shortened caption for LoF]{%
	longer detailed caption used for the figure}
    \end{lstlisting}
    \ac{GPS} generally prefers shortened single-line captions
    in the LoF: multiple-line captions are a bit unwieldy.

\item The \verb+\label+ command provides for associating a unique, user-defined,
    and descriptive identifier to the figure.  The figure can be
    can be referenced elsewhere in the text with this identifier
    as described in \autoref{sec:CrossReferences}.
\end{enumerate}
See Keith Reckdahl’s excellent guide for more details,
\webref{http://www.ctan.org/tex-archive/info/epslatex.pdf}{\emph{Using
imported graphics in LaTeX2e}}.

\section{Special Characters and Symbols}
\label{sec:SpecialSymbols}

\LaTeX\ appropriates many common symbols for its own purposes,
with some used for commands (\ie \verb+\{}&%+) and
mathematics (\ie \verb+$^_+), and others are automagically transformed
into typographically-preferred forms (\ie \verb+-`'+) or to
completely different forms (\ie \verb+<>+).
\autoref{tbl:LaTeX:Symbols} presents a list of common symbols and
their corresponding \LaTeX\ commands.  A much more comprehensive list 
of symbols and accented characters is available at:
\url{http://www.ctan.org/tex-archive/info/symbols/comprehensive/}
\begin{table}
\caption{Useful \LaTeX\ symbols}\label{tbl:LaTeX:Symbols}
\centering\begin{tabular}{ccp{0.5cm}cc}\toprule
\LaTeX & Result && \LaTeX & Result \\
\midrule
    \verb+\texttrademark+ & \texttrademark && \verb+\&+ & \& \\
    \verb+\textcopyright+ & \textcopyright && \verb+\{ \}+ & \{ \} \\
    \verb+\textregistered+ & \textregistered && \verb+\%+ & \% \\
    \verb+\textsection+ & \textsection && \verb+\verb!~!+ & \verb!~! \\
    \verb+\textdagger+ & \textdagger && \verb+\$+ & \$ \\
    \verb+\textdaggerdbl+ & \textdaggerdbl && \verb+\^{}+ & \^{} \\
    \verb+\textless+ & \textless && \verb+\_+ & \_ \\
    \verb+\textgreater+ & \textgreater && \\
\bottomrule
\end{tabular}
\end{table}

%--------------------------------------------------------------------------------------------------------------------------------------------%\section{Changing Page Widths and Heights}

The \class{ubcdiss} class is based on the standard \LaTeX\ \class{book}
class~\cite{lamport-1994-ladps} that selects a line-width to carry
approximately 66~characters per line.  This character density is
claimed to have a pleasing appearance and also supports more rapid
reading~\cite{bringhurst-2002-teots}.  I would recommend that you
not change the line-widths!

\subsection{The \texttt{geometry} Package}

Some students are unfortunately saddled with misguided supervisors
or committee members whom believe that documents should have the
narrowest margins possible.  The \latexpackage{geometry} package is
helpful in such cases.  Using this package is as simple as:
\begin{lstlisting}
    \usepackage[margin=1.25in,top=1.25in,bottom=1.25in]{geometry}
\end{lstlisting}
You should check the package's documentation for more complex uses.

\subsection{Changing Page Layout Values By Hand}

There are some miserable students with requirements for page layouts
that vary throughout the document.  Unfortunately the
\latexpackage{geometry} can only be specified once, in the document's
preamble.  Such miserable students must set \LaTeX's layout parameters
by hand:
\begin{lstlisting}
    \setlength{\topmargin}{-.75in}
    \setlength{\headsep}{0.25in}
    \setlength{\headheight}{15pt}
    \setlength{\textheight}{9in}
    \setlength{\footskip}{0.25in}
    \setlength{\footheight}{15pt}

    % The *sidemargin values are relative to 1in; so the following
    % results in a 0.75 inch margin
    \setlength{\oddsidemargin}{-0.25in}
    \setlength{\evensidemargin}{-0.25in}
    \setlength{\textwidth}{7in}       % 1.1in margins (8.5-2*0.75)
\end{lstlisting}
These settings necessarily require assuming a particular page height
and width; in the above, the setting for \verb+\textwidth+ assumes
a \textsc{US} Letter with an 8.5'' width.
The \latexpackage{geometry} package simply uses the page height and
other specified values to derive the other layout values.
The
\href{http://tug.ctan.org/tex-archive/macros/latex/required/tools/layout.pdf}{\texttt{layout}}
package provides a
handy \verb+\layout+ command to show the current page layout
parameters. 


\subsection{Making Temporary Changes to Page Layout}

There are occasions where it becomes necessary to make temporary
changes to the page width, such as to accomodate a larger formula. 
The \latexmiscpackage{chngpage} package provides an \env{adjustwidth}
environment that does just this.  For example:
\begin{lstlisting}
    % Expand left and right margins by 0.75in
    \begin{adjustwidth}{-0.75in}{-0.75in}
    % Must adjust the perceived column width for LaTeX to get with it.
    \addtolength{\columnwidth}{1.5in}
    \[ an extra long math formula \]
    \end{adjustwidth}
\end{lstlisting}


%--------------------------------------------------------------------------------------------------------------------------------------------%
\section{Keeping Track of Versions with Revision Control}
\label{sec:DissertationRevisionControl}

Software engineers have used \acf{RCS} to track changes to their
software systems for decades.  These systems record the changes to
the source code along with context as to why the change was required.
These systems also support examining and reverting to particular
revisions from their system's past.

An \ac{RCS} can be used to keep track of changes to things other
than source code, such as your dissertation.  For example, it can
be useful to know exactly which revision of your dissertation was
sent to a particular committee member.  Or to recover an accidentally
deleted file, or a badly modified image.  With a revision control
system, you can tag or annotate the revision of your dissertation
that was sent to your committee, or when you incorporated changes
from your supervisor.

Unfortunately current revision control packages are not yet targetted
to non-developers.  But the Subversion project's
\webref{http://tortoisesvn.net/docs/release/TortoiseSVN_en/}{TortoiseSVN}
has greatly simplified using the Subversion revision control system
for Windows users.  You should consult your local geek.

A simpler alternative strategy is to create a GoogleMail account
and periodically mail yourself zipped copies of your dissertation.

%--------------------------------------------------------------------------------------------------------------------------------------------%
\section{Recommended Packages}

The real strength to \LaTeX\ is found in the myriad of free add-on
packages available for handling special formatting requirements.
In this section we list some helpful packages.

\subsection{Typesetting}

\begin{description}
\item[\latexpackage{enumitem}:]
    Supports pausing and resuming enumerate environments.

\item[\latexpackage{ulem}:]
    Provides two new commands for striking out and crossing out text
    (\verb+\sout{text}+ and \verb+\xout{text}+ respectively)
    The package should likely
    be used as follows:
    \begin{verbatim}
    \usepackage[normalem,normalbf]{ulem}
    \end{verbatim}
    to prevent the package from redefining the emphasis and bold fonts.

\item[\latexpackage{chngpage}:]
    Support changing the page widths on demand.

\item[\latexpackage{mhchem}:] 
    Support for typesetting chemical formulae and reaction equations.

\end{description}

Although not a package, the
\webref{http://www.ctan.org/tex-archive/support/latexdiff/}{\texttt{latexdiff}}
command is very useful for creating changebar'd versions of your
dissertation.


\subsection{Figures, Tables, and Document Extracts}

\begin{description}
\item[\latexpackage{pdfpages}:]
    Insert pages from other PDF files.  Allows referencing the extracted
    pages in the list of figures, adding labels to reference the page
    from elsewhere, and add borders to the pages.

\item[\latexpackage{subfig}:]
    Provides for including subfigures within a figure, and includes
    being able to separately reference the subfigures.  This is a
    replacement for the older \texttt{subfigure} environment.

\item[\latexpackage{rotating}:]
    Provides two environments, sidewaystable and sidewaysfigure,
    for typesetting tables and figures in landscape mode.  

\item[\latexpackage{longtable}:]
    Support for long tables that span multiple pages.

\item[\latexpackage{tabularx}:]
    Provides an enhanced tabular environment with auto-sizing columns.

\item[\latexpackage{ragged2e}:]
    Provides several new commands for setting ragged text (\eg forms
    of centered or flushed text) that can be used in tabular
    environments and that support hyphenation.

\end{description}


\subsection{Bibliography Related Packages}

\begin{description}
\item[\latexpackage{bibunits}:]
    Support having per-chapter bibliographies.

\item[\latexpackage{footbib}:]
    Cause cited works to be rendered using footnotes.

\item[\latexpackage{bibentry}:] 
    Support placing the details of a cited work in-line.

\item[\latexpackage{custom-bib}:]
    Generate a custom style for your bibliography.

\end{description}


%--------------------------------------------------------------------------------------------------------------------------------------------%
\section{Moving On}
\label{sec:Conclusions}

At this point, you should be ready to go.  Other handy web resources:
\begin{itemize}
\item \webref{http://www.ctan.org}{\ac{CTAN}} is \emph{the} comprehensive
    archive site for all things related to \TeX\ and \LaTeX. 
    Should you have some particular requirement, somebody else is
    almost certainly to have had the same requirement before you,
    and the solution will be found on \ac{CTAN}.  The links to
    various packages in this document are all to \ac{CTAN}.

\item An online
    \webref{http://www.ctan.org/get/info/latex2e-help-texinfo/latex2e.html}{%
	reference to \LaTeX\ commands} provides a handy quick-reference
    to the standard \LaTeX\ commands.

\item The list of 
    \webref{http://www.tex.ac.uk/cgi-bin/texfaq2html?label=interruptlist}{%
	Frequently Asked Questions about \TeX\ and \LaTeX}
    can save you a huge amount of time in finding solutions to
    common problems.

\item The \webref{http://www.tug.org/tetex/tetex-texmfdist/doc/}{te\TeX\
    documentation guide} features a very handy list of the most useful
    packages for \LaTeX\ as found in \ac{CTAN}.

\item The
\webref{http://www.ctan.org/tex-archive/macros/latex/required/graphics/grfguide.pdf}{\texttt{color}}
    package, part of the graphics bundle, provides handy commands
    for changing text and background colours.  Simply changing
    text to various levels of grey can have a very 
    \textcolor{greytext}{dramatic effect}.


\item If you're really keen, you might want to join the
    \webref{http://www.tug.org}{\TeX\ Users Group}.

\end{itemize}



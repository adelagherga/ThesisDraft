%% The following is a directive for TeXShop to indicate the main file
%%!TEX root = diss.tex

\chapter{Abstract}

We present a practical and efficient algorithm for solving an arbitrary Thue-Mahler equation. This algorithm uses explicit height bounds with refined sieves, combining Diophantine approximation techniques of Tzanakis-de Weger with new geometric ideas. We begin by using methods of algebraic number theory to reduce the problem of solving the Thue-Mahler equation to the problem of solving a finite collection of related Diophantine equations. In the first part of this thesis, we establish the key results which allow us to drastically reduce the number of such Diophantine equations and subsequently reduce the running time. 

In the second part of this thesis, we show that, by fixing one exponent, there exists an effectively computable constant bounding the solutions of a Goormaghtigh equation under certain conditions. For small values of this fixed exponent, we solve the equation completely. For one such small exponent, we modify and specialize our Thue-Mahler algorithm to the resulting equation in order to fully resolve this case.

In the third part, we discuss an algorithm for finding all elliptic curves over $\mathbb{Q}$ with a given conductor. Though based on 
classical ideas derived from reducing the problem to one of solving associated Thue-Mahler equations,  our approach, in many cases at least, appears to be reasonably efficient  computationally. We provide 
details of the output derived from running the algorithm, concentrating on the cases of conductor $p$ or $p^2$, for $p$ prime, with comparisons to existing 
data. 

Finally, we specialize the Thue-Mahler algorithm to degree $3$, applying an analogue of Matshke-von K\"anel's elliptic logarithm sieve to construct a global sieve, leading to reduced search spaces. The algorithm is implemented in the Magma computer algebra system, and is part of an ongoing collaborative project. 



%--------------------------------------------------------------------------------------------------------------------------------------------%

\endinput

This document provides brief instructions for using the \class{ubcdiss}
class to write a \acs{UBC}-conformant dissertation in \LaTeX.  This
document is itself written using the \class{ubcdiss} class and is
intended to serve as an example of writing a dissertation in \LaTeX.
This document has embedded \acp{URL} and is intended to be viewed
using a computer-based \ac{PDF} reader.

Note: Abstracts should generally try to avoid using acronyms.

Note: at \ac{UBC}, both the \ac{GPS} Ph.D. defence programme and the
Library's online submission system restricts abstracts to 350
words.

% Consider placing version information if you circulate multiple drafts
%\vfill
%\begin{center}
%\begin{sf}
%\fbox{Revision: \today}
%\end{sf}
%\end{center}

Any text after an \endinput is ignored.
You could put scraps here or things in progress.
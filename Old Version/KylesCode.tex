\documentclass[11pt]{article}

% relevant packages
\usepackage[parfill]{parskip} 	% Activate to begin paragraphs with an empty line rather than an indent
\usepackage{graphicx}
\usepackage{amssymb}
\usepackage{amsmath}
\usepackage{mathrsfs }
\usepackage{amsthm}
\usepackage{epstopdf}
\usepackage{enumerate}
\usepackage{tikz}
\usetikzlibrary{matrix}
\usepackage{multicol}
\usepackage{listings}
\usepackage{color}
\usepackage{tikz-cd}
\usepackage[all]{xy}
\usepackage[english]{babel}
\usepackage{setspace}

\DeclareGraphicsRule{.tif}{png}{.png}{`convert #1 `dirname #1`/`basename #1 .tif`.png}

% stylistic environment
\definecolor{grau}{rgb}{0.3,0.3,0.3}
\usepackage[colorlinks, linkcolor=grau, citecolor=grau, urlcolor=grau]{hyperref}
\urlstyle{same}
\usepackage[top=1.3in, bottom=1.6in, left=1.3in, right=1.3in]{geometry}
\frenchspacing
\sloppy
\usepackage{booktabs}
\usepackage[ruled,vlined]{algorithm2e}
\usepackage{enumerate}
\onehalfspacing

% relevant environments
\newtheorem{theorem}{Theorem}[section]
\newtheorem{conjecture}[theorem]{Conjecture}
\newtheorem{corollary}[theorem]{Corollary}
\newtheorem{lemma}[theorem]{Lemma}
\newtheorem{properties}[theorem]{Properties}
\newtheorem{proposition}[theorem]{Proposition}
\newtheorem{problem}[theorem]{Problem}
\newtheorem{question}[theorem]{Question}

\theoremstyle{definition}
\newtheorem{Algorithm}[theorem]{Algorithm}
\newtheorem{definition}[theorem]{Definition}
\newtheorem{example}[theorem]{Example}
\newtheorem{remark}[theorem]{Remark}

% math operators
\DeclareMathOperator{\ord}{ord}
\DeclareMathOperator{\sgn}{sgn}
\DeclareMathOperator{\Cl}{Cl}
\DeclareMathOperator{\Gal}{Gal}

% code environment
\definecolor{dkgreen}{rgb}{0,0.6,0}
\definecolor{gray}{rgb}{0.5,0.5,0.5}
\definecolor{mauve}{rgb}{0.58,0,0.82}

\lstset{%frame=tb,
  language=Java,
  aboveskip=3mm,
  belowskip=3mm,
  showstringspaces=false,
  columns=flexible,
  basicstyle={\small\ttfamily},
  numbers=none,
  numberstyle=\tiny\color{gray},
  keywordstyle=\color{blue},
  commentstyle=\color{dkgreen},
  stringstyle=\color{mauve},
  breaklines=true,
  breakatwhitespace=true,
  tabsize=3
}

% new commands
\newcommand{\eps}{\varepsilon}


% draft edit commands
\usepackage{soul}
\newcommand{\edit}[1]{\textcolor{blue}{#1}}
\newcommand{\aaron}[1]{\textcolor{purple}{\footnotesize #1}}
\newcommand{\strike}[1]{\textcolor{red}{\st{#1}}}


%---------------------------------------------------------------------------------------------------------------------------------------------%
%---------------------------------------------------------------------------------------------------------------------------------------------%

\title{Kyle's Code}
	
\author{}
\date{\today}                                           % Activate to display a given date or no date

\begin{document}
\maketitle

%---------------------------------------------------------------------------------------------------------------------------------------------%

\begin{multicols}{2}
Kyle's Code
\begin{itemize}
\item Create $K$, $OK$, integral basis of $OK$, $U$, compute $r$
\item Compute FFF, rootsofginFFF, OF, which are $L$, $\theta$ in $L$ and $OL$, the splitting field of $K$
\item Set precision and start massive precision loop
\item Compute the decomposition of primes in $OK$ with their ramification, inertial degrees, compute $h_{ij}$
\item Compute Completions of $K$ at prime ideals, Kpp, with map mKpp
\item Generate minimal polynomial gp of mKpp(theta) in Qp
\item Compute decomposition of prime ideal in $OL$ (store only 1), find ramification, inertial degrees
\item Compute the completion of FFF at one prime ideal above p in FFF, FFFppF. This is the completion of $L$ at a single prime ideal in $L$ over $p$, Lp, along with mFFFpF the map $L \mapsto Lp$
\end{itemize}

\columnbreak

My Code
\begin{itemize}
\item Create $K$, $OK$, integral basis of $OK$, $U$, compute $r$
\item Compute FFF, rootsofginFFF, OF, which are $L$, $\theta$ in $L$ and $OL$, the splitting field of $K$ 
\item Generate all automorphisms of L, AutL and set ijkL as $\sigma, \sigma^2, \text{id}$
\item Generate large upper bound
\end{itemize}
\end{multicols}

\begin{multicols}{2}
Kyle's Code
\begin{itemize}
\item Generate thetap, roots of gp in FFFppF, the completion of $L$ at a single prime over $p$, Lp
\item \textcolor{blue}{Generate ImageOfIntegralBasisElementp, the image of the integral basis for $K$ in Kp, where the map is defined by $\theta \mapsto thetap[i][j]$}
\item \textcolor{blue}{Generate ImageOfpip, ImageOfepsp, the image of $\pi, \eps$ under $K$ in Kp, where the map is defined by $\theta \mapsto thetap[i][j]$}
\item Compute conjugates of theta in C, thetaC
\item \textcolor{red}{Generate ImageOfIntegralBasisElementC, the image of the integral basis for $K$ in $C$, where the map is defined by $\theta \mapsto Conjugates(\theta)$}
\begin{itemize}
\item This is where we invoke the use of the hom function, as Kyle's method sometimes sends elements to $0$
\end{itemize}
\item \textcolor{red}{Generate ImageOfpiC, ImageOfepsC, the image of $\pi, \eps$ under $K$ in C, where the map is defined by $\theta \mapsto Conjugates(\theta)$}
\item Relabel the $\theta$ in $L$ from earlier as thetaF
\item \textcolor{purple}{Generate ImageOfIntegralBasisElementF, the image of the integral basis for $K$ in $L$, where the map is defined by $\theta \mapsto thetaF[i]$. That is, $\theta \mapsto \theta[i][j] in L$}
\end{itemize}

\columnbreak 

My Code
\begin{itemize}
\item
\end{itemize}
\end{multicols}

\begin{multicols}{2}
Kyle's Code
\begin{itemize}
\item \textcolor{purple}{Generate ImageOfpiF, ImageOfepsF, the image of $\pi, \eps$ under $K$ in L, where the map is defined by $\theta \mapsto thetaF[i]$}
\item Apply prime ideal removing lemma \\
\end{itemize}

\columnbreak 

My Code
\begin{itemize}
\item
\end{itemize}
\end{multicols}

Begin to iterate through the cases. For each case, we now have $\zeta, \alpha$. Hence, for each prime

\begin{multicols}{2}
Kyle's Code
\begin{itemize}
\item 
\end{itemize}

\columnbreak 

My Code
\begin{itemize}
\item Begin p adic precision loop
\item Compute completion of $L$ at one prime ideal above p, FFFppF, with map mFFFpF, Lp, mapLLp. This is 
\item Compute the decomposition of primes in $OK$
\item Compute Completions of $K$ at prime ideals, Kpp, with map mKpp, Kp, mKp
\item Generate minimal polynomial gp of mKpp(theta) in Qp (mKp(th))
\end{itemize}
\begin{center}
\begin{tikzcd}
L \to L \arrow[r, "\phi"] \arrow[d]
& Lp \arrow[d] \\
K \arrow[r]
& Kp
\end{tikzcd}
\end{center}
\begin{itemize}
\item Take th in $K$ into $L$, apply each automorphism $ijkL: L \to L$, apply $mapLLp: L \to Lp$ (mFFFpF). These are the roots of gp in Lp (thetap)
\item Find which ijkL[k] map corresponds to thetap[i][j], mapsLL[i][j]
\end{itemize}
\end{multicols}

\begin{multicols}{2}
Kyle's Code
\begin{itemize}
\item Compute images of zeta, alpha in Qp
\item \textcolor{blue}{Generate ImageOfzetap, ImageOfalphap, the image of $\zeta, \alpha$ under $K$ in Kp, where the map is defined by $\theta \mapsto thetap[i][j]$}
\item \textcolor{red}{Generate ImageOfzetaC, ImageOfalphaC, the image of $\zeta, \alpha$ under $K$ in C, where the map is defined by $\theta \mapsto Conjugates(\theta)$}
\item Determine $i_0, j, k$ by computing $\delta_1, \delta_2$ in FFFppF, the completion of $L$ at prime ideal, Lp (using thetap)
\item Verify Special Case 1
\item If Special Case 1 holds, recompute $\alpha$, ImageOfalphap, ImageOfalphaC
\item \textcolor{blue}{Generate LogarithmicAlphap, the pAdicLof of $\delta_1$, ImageOfpip[k]/ImageOfpip[j], ImageOfepsp[k]/ImageOfepsp[j], etc}
\item Verify Special Case 2 and see if we can find $i_0, j,k$ for which it holds
\item Compute $\hat{i}$
\item \textcolor{blue}{Generate betas:= -LogarithmicAlphap[i] /LogarithmicAlphap[ihat]}
\item Compute rootsofginFFFppF[l][i]:= mFFFppF(theta[i][j]), taking theta[i][j], the roots $\theta$ in $L$, into Lp
\end{itemize}
\begin{center}
\begin{tikzcd}
L \arrow[r, "\phi"] \arrow[d]
& Lp \arrow[d] \\
K \arrow[r]
& Kp
\end{tikzcd}
\end{center}

\columnbreak 

My Code
\begin{itemize}
\item \textcolor{purple}{Define thetaL (thetap) as ijk[k](L!th)}
\item Find i0,jj,kk among ijkL using thetaL
\item \textcolor{purple}{Compute tauL, gammalistL, epslistL, the image of $\alpha*\zeta$, gammalist, epslist using mapsLL[i][j]}
\item \textcolor{purple}{Compute $\delta_1, \delta_2$ in $L$}
\item \textcolor{blue}{Compute $\delta_1, \delta_2$ in $Lp$ via mapLLp}
\item Check Special Case 1, Special Case 2; determine small bound if true
\item Generate heuristic precision
\item \textcolor{blue}{Compute pAdicLog of gammalistL,epslistL in Lp, mapped there via mapLLp}
\item Compute ihat
\item \textcolor{blue}{Compute beta:= -LogList[i]/Loglist[ihat]}
\item Generate ellipsoid [DETAILS]
\item Generate pAdicLattice [DETAILS]
\end{itemize}
\end{multicols}

\begin{multicols}{2}
Kyle's Code
\begin{itemize}
\item In $L$, we have rootsofginFFF, theta[i][j]. To compute thetap, generate minimal polynomial of theta in Kp, compute roots in Lp. Now, when generating Lp, we also generated the map $\phi:L \to Lp$. Applying $\phi(theta[i][j])$ gives us the same roots theta[i][j] in possibly different order. Hence thetap are the same as rootsofginFFFppF
\item Find which rootsofginFFFppF[l][i] corresponds to thetap[i0], thetap[j], thetap[k]; label these as ii0, ijjj, ikkk
\item \textcolor{purple}{Generate the preimage of $\zeta, \alpha$ in $L$ corresponding to i0, j,k using the above ii0,ijjj,ikkk}
\item \textcolor{purple}{Now, we have thetaF, ImageOfpiF, PreimageOfalpha, PreimageOfzeta, ImageOfepsF}
\item \textcolor{purple}{Generate alphaALGEBRAIC, images of $\delta_1$, ImageOfpiF[ikkk]/ImageOfpiF[ijjj] in $L$}
\end{itemize}

\columnbreak 

My Code
\begin{itemize}
\item
\end{itemize}
\end{multicols}

\newpage

\begin{multicols}{2}
Kyle's Code
\begin{itemize}
\item
\end{itemize}

\columnbreak

My Code
\begin{itemize}
\item
\end{itemize}
\end{multicols}





\newpage
\begin{itemize}
\item \textcolor{red}{Choose one conjugate of $L.1$ in C, giving a map $L \mapsto C$. Use this to compute rootsofginC, theta[i][j] under this map}
\end{itemize}
\begin{center}
\begin{tikzcd}
L \arrow[r, "\phi"] \arrow[d]
& \mathbb{C} \arrow[d] \\
K \arrow[r]
& \mathbb{C}
\end{tikzcd}
\end{center}
\begin{itemize}
\item In C, we have Conjugates(theta), also known as thetaC. We now have theta[i][j] in L mapped into C, rootsofginC, and these items should be the same
\item Find which rootsofginC[i] correspond to thetaC[i0], thetaC[j], thetaC[k]; label these as ii0,ijjj,ikkk
\item \textcolor{purple}{Generate the preimage of $\zeta, \alpha$ in $L$ corresponding to i0, j,k using the above ii0,ijjj,ikkk}
\item \textcolor{purple}{Generate alphaALGEBRAIC, images of $\delta_1$, ImageOfpiF[ikkk]/ImageOfpiF[ijjj] in $L$}
\item Compute upper bound $C22$ for cases $s = 0,1,2, \geq 3$, use this to determine UpperBoundForn \\
\item alphaALGEBRAIC (and therefore preimages) are only used to generate the upper bound
\end{itemize}

Begin basic $p$-adic reduction (LLL)
\begin{itemize}
\item \textcolor{blue}{Use betas:= -LogarithmicAlphap[i] /LogarithmicAlphap[ihat] to generate $p$-adic approximation matrix; from imagesinp}
\end{itemize}

Begin basic real reduction (LLL)
\begin{itemize}
\item \textcolor{red}{Use LogarithmicAlphaC to generate complex approximation matrix; from imagesinC}
\end{itemize}






















\end{document}
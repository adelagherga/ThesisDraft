%% The following is a directive for TeXShop to indicate the main file
%%!TEX root = diss.tex

\chapter{Computing Elliptic Curves over $\mathbb{Q}$}
\label{ch:EllipticCurves} 

In the chapter at hand, we outline an algorithm to compute elliptic curves over $\mathbb{Q}$, based upon techniques of solving Thue-Mahler equations. Our aim is to give a straightforward demonstration of the link 
between the conductors of the elliptic curves in question and the corresponding equations, and to make the Diophantine approximation problem 
that follows as easy to tackle as possible. It is worth noting here that these connections are quite straightforward for primes $p > 3$, but require 
careful analysis at the primes $2$ and $3$. We will demonstrate our approach for a number of specific conductors and sets $S$, and then focus 
our main computational efforts on curves with bad reduction at a single prime (i.e. curves of conductor $p$ or $p^2$ for 
$p$ prime).  In these cases, the computations simplify significantly and we 
are able to find all curves of prime conductor up to $2 \times 10^9$ ($10^{10}$ in the case of curves of positive 
discriminant) and conductor $p^2$ for $p \leq 5\times 10^5$. We then extend these computations in the case of conductor 
$p$,  for prime $p \leq 2 \times 10^{13}$, and conductor $p^2$ for prime $p \leq 10^{10}$. We are not, however, able to 
guarantee completeness for these extended computations (we will discuss this further in what follows). 

%We are in the process of making our data more easily available through the LMFDB\footnote{The $L$-function and modular 
%form database, currently accessible at \url{http://www.lmfdb.org/}}. Until this is completed, we invite 
%the interested reader to contact the authors.


%-----------------------------------------------
\section{Elliptic curves} \label{elliptic}
%-----------------------------------------------

Our basic problem is to find a model for each isomorphism class of elliptic curves over $\mathbb{Q}$ with a given 
conductor. Let $S=\{ p_1, p_2, \ldots, p_k \}$ where the $p_i$ are distinct primes, and fix a conductor $N= p_1^{\eta_1} \cdots p_k^{\eta_k}$ for
$\eta_i \in \mathbb{N}$.  Any curve of conductor $N$ has a minimal 
model
\begin{align*}
E:&\phantom{=} y^2 + a_1 xy + a_3 y = x^3 + a_2 x^2 + a_4 x + a_6
\end{align*}
with the $a_i$ integral and discriminant 
\begin{align*}
\Delta_E &= (-1)^\delta p_1^{\gamma_1} \cdots p_k^{\gamma_k},
\end{align*}
where the $\gamma_i$ are positive integers satisfying $\gamma_i \geq \eta_i$, for each $i = 1, 2, \ldots, k$, and $\delta \in \{ 0, 1 \}$. 

Writing
$$
b_2 = a_1^2+4a_2, \; \; b_4 = a_1 a_3 + 2 a_4, \; \; 
b_6 = a_3^2+4a_6, \; \; 
c_4 = b_2^2-24 b_4 
$$
and
$$
c_6 = -b_2^3+ 36 b_2 b_4 -216 b_6,
$$
we have
$1728 \Delta_E = c_4^3-c_6^2$ and
$j_E = c_4^3/\Delta_E$.
It follows that
\begin{align} \label{first}
c_6^2 &= c_4^3 + (-1)^{\delta +1} 2^6 \cdot 3^3 \cdot p_1^{\gamma_1} \cdots p_k^{\gamma_k}.
\end{align}
In fact, it is equation~\eqref{first} that lies at the heart of our method (see also Cremona and 
Lingham \cite{CrLi} for an approach to the problem that takes as its starting point equation~\eqref{first}, but 
subsequently heads in a rather different direction).

Let $\nu_p(x)$ be the largest power of a prime $p$ dividing a nonzero integer $x$. Since our model is minimal, we 
may suppose  (via Tate's algorithm; see, for example, Papadopoulos \cite{Pap}) that 
$$
\min \{ 3 \nu_p (c_4), 2 \nu_p (c_6) \} < 12 + 12 \nu_p(2) + 6 \nu_p(3),
$$
for each prime $p$, while
$$
\nu_p (N_E) \leq 2 + \nu_p (1728).
$$ 
For future use, it will be helpful to have a somewhat more precise determination of the possible 
values of $\nu_p(c_4)$ and $\nu_p(c_6)$ we encounter. We compile this data from Papadopoulos~\cite{Pap} and
summarize it in Tables~\ref{tab nu2},~\ref{tab nu3} and~\ref{tab nup}.

\begin{table}[h]
$$
\begin{array}{|r|r|r|r|}
\hline
\nu_2 (c_4) & \nu_2 (c_6) & \nu_2 (\Delta_E) & \nu_2 (N)  \\ \hline
0 & 0 & \geq 0  & \min \{ 1, \nu_2 (\Delta_E)  \}  \\
\geq 4 &  3 & 0 & 0  \\
\geq 4 & 5 & 4 & 2, 3 \mbox{ or } 4  \\
\geq 4 & \geq 6 & 6 & 5 \mbox{ or } 6  \\
4 & 6 & 7 & 7  \\
4 & 6 & 8 & 2, 3 \mbox{ or } 4  \\
4 & 6 & 9 &  5   \\
4 & 6 & 10 \mbox{ or } 11 & 3 \mbox{ or } 4   \\
4 & 6 & \geq 12 &  4  \\
5 & 7 & 8 & 7  \\
\geq 6 & 7 & 8 & 2, 3 \mbox{ or } 4  \\
\hline
\end{array}
\quad
\begin{array}{|r|r|r|r|}
\hline
\nu_2 (c_4) & \nu_2 (c_6) & \nu_2 (\Delta_E) & \nu_2 (N)  \\ \hline
 5 & \geq 8 & 9 & 8 \\
  \geq 6 & 8 & 10 & 6 \\
 6 & \geq 9 & 12 & 5 \mbox{ or } 6 \\
 6 & 9 & \geq 14 &  6 \\
 7 & 9 & 12 & 5 \\
  \geq 8 & 9 & 12 & 4 \\
 6 & 9 & 13 & 7 \\
 7 & 10 & 14 & 7 \\
 7 & \geq 11 & 15 & 8 \\
  \geq 8 &  10 & 14 & 6 \\
  & & & \\
\hline
\end{array}
$$


\caption{The possible values of $\nu_2(c_4), \nu_2(c_6), \nu_2(\Delta_E)$ and $\nu_2(N)$.}
\label{tab nu2}
\end{table}

\begin{table}[h]
$$
\begin{array}{|r|r|r|r|}
\hline
\nu_3 (c_4) & \nu_3 (c_6) & \nu_3 (\Delta_E) & \nu_3 (N)  \\ \hline
0 & 0 & \geq 0  & \min \{ 1, \nu_3 (\Delta_E)  \}  \\
1 & \geq 3 & 0 & 0  \\
\geq 2 & 3 & 3 & 2 \mbox{ or } 3  \\
2 & 4 & 3 & 3  \\
2 & \geq 5 & 3 & 2  \\
2 & 3 & 4 & 4  \\
2 & 3 & 5 & 3  \\
2 & 3 & \geq 6 & 2  \\
\geq 3 & 4 & 5 & 5  \\
3 & 5 & 6 & 4  \\
\hline
\end{array}
\quad
\begin{array}{|r|r|r|r|}
\hline
\nu_3 (c_4) & \nu_3 (c_6) & \nu_3 (\Delta_E) & \nu_3 (N)  \\ \hline
 3 & \geq 6 & 6 & 2 \\
 \geq 4 & 5 & 7 & 5 \\
 \geq 4 & 6 & 9 & 2 \mbox{ or } 3 \\
 4 & 7 & 9 & 3 \\
 4 & \geq 8 & 9 & 2 \\
 4 & 6 & 10 & 4 \\
 4 & 6 & 11 & 3 \\
 \geq 5 & 7 & 11 & 5 \\
 5 & 8 & 12 & 4 \\
 \geq 6 & 8 & 13 & 5 \\
\hline
\end{array}
$$

\caption{The possible values of $\nu_3(c_4), \nu_3(c_6), \nu_3(\Delta_E)$ and $\nu_3(N)$.}
\label{tab nu3}
\end{table}

\begin{table}[h]
$$
\begin{array}{|r|r|r|r|}
\hline
\nu_p (c_4) & \nu_p (c_6) & \nu_p (\Delta_E) & \nu_p (N)  \\ \hline
0 & 0 & \geq 1  & 1  \\
\geq 1 & 1 & 2 & 2  \\
1 & \geq 2 & 3 & 2  \\
\geq 2 & 2 & 4 & 2  \\
\geq 2 & \geq 3 & 6 & 2 \\
\hline
\end{array}
\quad
\begin{array}{|r|r|r|r|}
\hline
\nu_p (c_4) & \nu_p (c_6) & \nu_p (\Delta_E) & \nu_p (N)  \\ \hline
 2 & 3 & \geq 7 & 2 \\
 \geq 3 & 4 & 8 & 2 \\
 3 & \geq 5 & 9 & 2 \\
 \geq 4 & 5 & 10 & 2 \\
  &  &  &  \\
\hline
\end{array}
$$

\caption{The possible values of $\nu_p(c_4), \nu_p(c_6), \nu_p(\Delta_E)$ and $\nu_p(N)$ when $p > 3$ is prime and 
$p\mid \Delta_E$.}
\label{tab nup}
\end{table}


%-----------------------------------------------------------------------------------------
\section{Cubic forms : the main theorem and algorithm} \label{forms}
%-----------------------------------------------------------------------------------------

Having introduced the notation we require for elliptic curves, we now turn our attention to cubic forms and our main result. Fix integers $a, b, 
c$ and $d$, and consider the binary cubic form 
\begin{align} \label{form0}
F(x,y)&=ax^3+bx^2y+cxy^2+dy^3,
\end{align}
with discriminant
\begin{equation} \label{claire-bear}
D_F = -27 a^2 d^2 + b^2 c^2 + 18 abcd -4 ac^3 -4 b^3 d.
\end{equation}
To any such form, we can associate a pair of covariants, the Hessian  $H=H_F$:
\begin{align*}
H=  H_F (x,y)=  - \frac{1}{4} \left(\frac{\partial^2 F}{\partial x^2} \frac{\partial^2 F}{\partial y^2} - 
\left(\frac{\partial^2 F}{\partial x \partial y}\right)^2 \right) 
\end{align*}
and the Jacobian determinant of $F$ and $H$,  a cubic form $G=G_F$ defined by
\begin{align*}
G&=G_F (x,y)=\frac{\partial F}{\partial x}\frac{\partial H}{\partial y}-  \frac{\partial F}{\partial y} \frac{\partial 
H}{\partial x}.
\end{align*}
A quick computation reveals that, explicitly, 
$$
H= (b^2-3ac) x^2 + (bc-9ad) xy + (c^2-3bd) y^2  
$$
and
$$
\arraycolsep=1.4pt\def\arraystretch{1.4}
\begin{array}{ll}
G = & (-27 a^2d+9abc-2b^3)  x^3 + (-3b^2c-27 abd+18ac^2) x^2 y \\
   & + (3bc^2-18b^2d+27acd)  x y^2 + (-9bcd+2c^3+27ad^2) y^3.\\
\end{array}
$$
These satisfy the syzygy
\begin{align} \label{syz}
4H(x,y)^3 &= G(x,y)^2+27D_F F(x,y)^2
\end{align}
as well as the resultant identities:
\begin{equation} \label{resultant}
\mbox{Res} (F,G) = -8 D_F^3 \; \; \mbox{ and } \; \; 
 \mbox{Res} (F,H) = D_F^2.
\end{equation}
Note here that we could just as readily work with $-G$ instead of $G$ here (corresponding to taking the Jacobian determinant of $H$ and $F$, rather than of $F$ and $H$). Indeed, as we shall observe in Section \ref{note}, for our applications we will, in some sense, need to consider both possibilities.

Notice that if we set $(x,y)=(1,0)$ and multiply through by $\mathcal{D}^6/4$ (for any rational $\mathcal{D}$), then 
this syzygy can be rewritten as
\begin{align*}
  ( \mathcal{D}^2 H(1,0))^3 - \left( \frac{\mathcal{D}^3}{2} G(1,0) \right)^2 
  &=  1728 \cdot \frac{\mathcal{D}^6 D_F}{256} F(1,0)^2.
\end{align*}
Given an elliptic curve with corresponding invariants $c_4, c_6$ and $\Delta_E$, we will show that it is always possible to construct a binary 
cubic form $F$, with corresponding  $\mathcal{D}$ for which 
$$
\mathcal{D}^2 H(1,0) = c_4, \; \; -\frac{1}{2} \mathcal{D}^3 G(1,0) = c_6 \; \mbox{ and } \; \Delta_E =  \frac{\mathcal{D}^6  D_F F(1,0)^2}{256}
$$
(and hence equation (\ref{first}) is satisfied). This is the basis of the proof of our main result, which provides an algorithm for computing all isomorphism classes of elliptic curves $E/\mathbb{Q}$ with conductor a fixed positive integer $N$. Though we state our result for curves with $j_E \neq 0$, the case $j_E=0$ is easy to treat separately (see Section \ref{Mordell}). 

\begin{theorem} \label{fisk}
Let $E/\mathbb{Q}$ be an elliptic curve of conductor $N=2^\alpha 3^\beta N_0$, where $N_0$ is coprime to $6$ and $0 \leq \alpha \leq 8$, $0 \leq \beta \leq 5$. Suppose further that $j_E \neq 0$.
Then there exists an integral binary cubic form $F$ of discriminant 
\begin{align*}
D_F &= \text{sign}(\Delta_E) 2^{\alpha_0} 3^{\beta_0} N_1,
\end{align*}
and relatively prime integers $u$ and $v$ with 
\begin{equation} \label{TM-eq}
F(u,v) =  \omega_0 u^3 + \omega_1 u^2v + \omega_2 uv^2 + \omega_3 v^3 = 2^{\alpha_1} \cdot 3^{\beta_1} \cdot \prod_{p \mid N_0} p^{\kappa_p},
\end{equation}
such that $E$ is isomorphic over $\mathbb{Q}$ to $E_{\mathcal{D} }$, where
\begin{equation} \label{curvey}
E_{\mathcal{D}} \; \; : \; \;  3^{[\beta_0/3]} y^2 = x^3 -27 \mathcal{D}^2 H_F(u,v) x +27 \mathcal{D}^3 G_F(u,v)
\end{equation}
and, for $[r]$ the greatest integer not exceeding a real number $r$,
\begin{align} \label{Dee}
\mathcal{D} &= \prod_{p \mid \gcd (c_4(E), c_6(E))} p^{\min \{ [\nu_p (c_4(E))/2], [\nu_p (c_6(E))/3] \}}.
\end{align}
The $\alpha_0$, $\alpha_1$, $\beta_0$, $\beta_1$ and $N_1$ are nonnegative integers satisfying  $N_1 \mid N_0$, 
\begin{align*}
(\alpha_0, \alpha_1) &=
\begin{cases}
(2, 0)  \mbox{ or } (2,3)  
    & \mbox{ if }  \alpha =0, \\
(3,\geq 3) \mbox{ or } (2,\geq 4)  
    & \mbox{ if }  \alpha =1, \\
(2,1), (4,0) \mbox{ or }  (4,1)  
    & \mbox{ if }  \alpha =2, \\
(2,1), (2,2), (3,2), (4,0)  \mbox{ or }  (4,1)  
    & \mbox{ if }  \alpha =3, \\
(2, \geq 0), (3, \geq 2), (4,0)  \mbox{ or }  (4, 1) 
    & \mbox{ if }  \alpha =4,  \\
(2, 0) \mbox{ or } (3,1) 
    & \mbox{ if }  \alpha =5, \\
(2, \geq 0), (3, \geq 1), (4,0) \mbox{ or }  (4, 1) 
    & \mbox{ if }  \alpha =6, \\
(3,0) \mbox{ or } (4,0) 
    & \mbox{ if }  \alpha =7,  \\
(3, 1) 
    & \mbox{ if }  \alpha =8
\end{cases}
\intertext{and}
(\beta_0, \beta_1) &=
\begin{cases}
(0, 0) 
    & \mbox{ if } \beta =0, \\
(0, \geq 1) \mbox{ or } (1, \geq 0) 
    & \mbox{ if } \beta =1, \\
(3,0), (0, \geq 0) \mbox{ or } (1, \geq 0) 
    & \mbox{ if } \beta =2, \\
(\beta, 0) \mbox{ or } (\beta,1) 
    & \mbox{ if } \beta \geq 3.
\end{cases}
\end{align*}
% $$
% \; \; \; \; \; \; \; \; \; \; \; 
% (\alpha_0, \alpha_1)  = \left\{
% \begin{array}{l}
% (2, 0)  \mbox{ or } (2,3)  \; \mbox{ if } \; \alpha =0, \\
% (3,\geq 3) \mbox{ or } (2,\geq 4)  \; \mbox{ if } \; \alpha =1, \\
% (2,1), (4,0) \mbox{ or }  (4,1)  \; \mbox{ if } \; \alpha =2, \\
% (2,1), (2,2), (3,2), (4,0)  \mbox{ or }  (4,1)  \; \mbox{ if } \; \alpha =3, \\
% (2, \geq 0), (3, \geq 2), (4,0)  \mbox{ or }  (4, 1) \; \mbox{ if } \; \alpha =4,  \\
% (2, 0) \mbox{ or } (3,1)  \; \mbox{ if } \; \alpha =5, \\
% (2, \geq 0), (3, \geq 1), (4,0) \mbox{ or }  (4, 1) \; \mbox{ if } \; \alpha =6, \\
% (3,0) \mbox{ or } (4,0)  \; \mbox{ if } \; \alpha =7, \\
% (3, 1)  \; \mbox{ if } \; \alpha =8, \\
% \end{array}
% \right.
% $$
% $$
% (\beta_0, \beta_1)  = \left\{
% \begin{array}{l}
% (0, 0) \; \mbox{ if } \; \beta =0, \\
% (0, \geq 1) \mbox{ or } (1, \geq 0) \; \mbox{ if } \; \beta =1, \\
% (3,0), (0, \geq 0) \mbox{ or } (1, \geq 0) \; \mbox{ if } \; \beta =2, \\
% (\beta, 0) \mbox{ or } (\beta,1) \; \mbox{ if } \; \beta \geq 3, \\
% \end{array}
% \right.
% $$
The  $\kappa_p$ are nonnegative integers with
\begin{equation} \label{term0}
\nu_p (\Delta_E)  =
\left\{
\begin{array}{lc}
 \nu_p (D_F) + 2 \kappa_p & \mbox{ if }  p \nmid \mathcal{D}, \\
\nu_p (D_F) + 2 \kappa_p + 6 & \mbox{ if }  p \mid \mathcal{D} \\
\end{array}
\right.
\end{equation}
and
\begin{equation} \label{term1}
\kappa_p \in \{ 0, 1 \} \; \; \mbox{ whenever } \; \; p^2 \mid N_1.
\end{equation}
Further, we have
\begin{equation} \label{term2}
 \mbox{ if } \; \; \beta_0 \geq 3, \; \mbox{ then } \; 3 \mid \omega_1 \mbox{ and } 3 \mid \omega_2,
\end{equation}
and
\begin{equation} \label{term3}
 \mbox{ if } \nu_p(N)=1, \mbox{ for } p \geq 3, \mbox{ then } p \mid D_F F(u,v).
\end{equation}
\end{theorem}

Here, as we shall make explicit in the next subsection, the form $F$ corresponding to the curve $E$
in Theorem \ref{fisk} determines the $2$-division field of $E$. This connection was noted by Rubin and Silverberg \cite{RuSi} in a somewhat different context -- they proved that if $K$ is a field of characteristic $\neq 2, 3$,  $F(u,v)$ is a binary cubic form defined over $K$,  $E$ is an elliptic curve defined by $y^2=F(x,1)$, and $E_0$ is another elliptic curve over $K$ with the property that
$E[2] \cong E_0[2]$ (as Galois modules), then $E_0$ is isomorphic to the curve
$$
y^2 = x^3 - 3 H_F(u,v) x + G_F(u,v),
$$
for some $u, v \in K$. 

%--------------------------------------------------------------
\subsection{Remarks}
%---------------------------------------------------------------
Before we proceed, there are a number of observations we should make regarding Theorem \ref{fisk}. 

\subsubsection{Historical comments}
Theorem \ref{fisk} is based upon a generalization of classical work of Mordell \cite{Mor1} (see also Theorem 3 of 
Chapter 24 of Mordell \cite{Mor}), in which the Diophantine equation 
$$
X^2+kY^2 = Z^3
$$
is treated through reduction to binary cubic forms\index{cubic forms} and their covariants, under the assumption that 
$X$ and $Z$ are coprime. That this last restriction can, with some care, be eliminated, was noted by Sprindzuk (see 
Chapter VI of \cite{Spri}). A similar approach to this problem can be made through the invariant theory of binary quartic forms, where one is led to solve, instead, equations of the shape
$$
X^2 + k Y^3 = Z^3.
$$
We will not carry out the analogous analysis here.

\subsubsection{$2$-division fields and reducible forms}
It might happen that the form $F$ whose existence is guaranteed by Theorem \ref{fisk} is reducible over $\mathbb{Z}[x,y]$. This occurs precisely when the elliptic curve $E$ has a nontrivial rational $2$-torsion point. This follows from the more general fact that the cubic form $F(u,v) =  \omega_0 u^3 + \omega_1 u^2v + \omega_2 uv^2 + \omega_3 v^3$ corresponding to an elliptic curve $E$ has the property that the splitting field of $F(u,1)$ is isomorphic to the $2$-division field of $E$. This is almost immediate from the identity
$$
\arraycolsep=1.4pt\def\arraystretch{1.4}
\begin{array}{ll}
3^3 \, \omega_0^2 \, F \left( \frac{x-\omega_1}{3 \omega_0},1 \right) & = x^3+(9 \omega_0 \omega_2-3 \omega_1^2) x+27 \omega_0^2 \omega_3-9 \omega_0 \omega_1 \omega_2+2 \omega_1^3 \\
 & = x^3 - 3 H_F(1,0) x  + G_F (1,0). \\
\end{array}
$$
Indeed, from (\ref{curvey}), the elliptic curve defined by the equation
$y^2=x^3 - 3 H_F(1,0) x  + G_F (1,0)$
is a quadratic twist of that given by the model $y^2 = x^3 -27 c_4(E) x -54 c_6(E)$, and hence also of $E$ (whereby they have the same $2$-division field).

\subsubsection{Imprimitive forms}
It is also the case that the cubic forms arising need not be primitive (in the sense that $\gcd 
(\omega_0,\omega_1,\omega_2,\omega_3)=1$). This situation can occur if each of 
the coefficients of $F$ is divisible by some integer $g \in \{ 2, 3, 6 \}$. Since the discriminant is a quartic form in the coefficients of $F$, for this to take place one requires that
$$
D_F \equiv 0 \mod{g^4}.
$$
This is a necessary but not sufficient condition for the form $F$ to be imprimitive. It follows, if we wish to restrict attention to primitive forms in Theorem \ref{fisk}, that the possible values for $\nu_p (D_F)$ that can arise are
\begin{equation} \label{lumpy}
\nu_2 (D_F) \in \{ 0, 2, 3, 4 \}, \; \;  \nu_3 (D_F) \in \{ 0, 1, 3, 4, 5 \} \; \; \mbox{ and } \; \; \nu_p (D_F) \in \{ 0, 1, 2 \}, \; \mbox{ for } p > 3.
\end{equation}


\subsubsection{Possible twists}
We note that necessarily
\begin{align} \label{froggie}
\mathcal{D} &\mid 2^3 \cdot 3^2 \cdot \prod_{p \mid N_0} p,
\end{align}
so that, given $N$, there is a finite set of $E_{\mathcal{D}}$ to consider (we can restrict our attention to quadratic twists of the curve defined via
$y^2=x^3 - 3 H_F(1,0) x  + G_F (1,0)$,
by squarefree divisors of $6N$). In  case we are dealing with squarefree conductor $N$ (i.e. for semistable curves $E$), then, from Tables \ref{tab nu2}, \ref{tab nu3} and \ref{tab nup}, it follows that $\mathcal{D} \in \{ 1, 2 \}$.


\subsubsection{Necessity, but not sufficiency}
If we search for elliptic curves of conductor $N$, say, there may exist a cubic form $F$ for which the corresponding Thue-Mahler equation \eqref{TM-eq} \index{Thue-Mahler equations} has a 
solution, where all of the conditions of Theorem \ref{fisk} are satisfied, but for which the corresponding  $E_{\mathcal{D}}$ has conductor $N_{E_{\mathcal{D}}} \neq N$ for all possible $\mathcal{D}$. This can happen when 
certain local conditions at primes dividing $6N$ are not met; these local conditions are, in practice, easy to check and only a minor issue when 
performing computations. Indeed, when producing tables of elliptic curves of conductor up to some given bound, we will, in many cases,  apply Theorem \ref{fisk} to find all curves with good reduction outside a fixed set of primes -- in effect, working with 
multiple conductors simultaneously. For such a computation, the conductor of every twist $E_{\mathcal{D}}$ we encounter will be of interest to 
us. 

\subsubsection{Special binary cubic forms} \label{dahlia}
If, for a given binary form $F(x,y)=a x^3 + b x^2 y + c xy^2 + d y^3$, 3 divides both the coefficients $b$ and $c$ (say $b = 3 b_0$ and 
$c=3 c_0$), then  $27 \mid D_F$ and, consequently, we can write $D_F=27 \widetilde{D}_F$, where
\begin{align*}
\widetilde{D}_F &= -a^2d^2+6ab_0c_0d+3b_0^2c_0^2-4ac_0^3-4b_0^3d.
\end{align*}
One can show that the set of binary cubic forms with $b \equiv c \equiv 0 \mod{3}$ is closed within the larger set of all binary 
cubic forms in $\mathbb{Z}[x,y]$, under the action of either $\mbox{SL}_2 ( \mathbb{Z})$ or $\mbox{GL}_2 ( \mathbb{Z})$. 
Also note that for such forms we have
$$
 \widetilde{H}_F(x,y) = \frac{H_F(x,y)}{9}= (b_0^2-ac_0) x^2 + (b_0c_0-ad) xy + (c_0^2-b_0d) y^2 
$$
and $\widetilde{G}_F (x,y) =  G_F(x,y)/27$, so that
$$
\arraycolsep=1.4pt\def\arraystretch{1.4}
\begin{array}{cl} 
\widetilde{G}_F (x,y) = & (-a^2d+3ab_0c_0-2b_0^3) x^3 + 3 (-b_0^2c_0-ab_0d+2ac_0^2) x^2 y \\
 & + 3 (b_0c_0^2-2b_0^2d+ac_0d) x y^2 + (-3b_0c_0d+2c_0^3+ad^2) y^3. \\
 \end{array}
$$
The syzygy now becomes
\begin{align} \label{syz2}
4\widetilde{H}_F (x,y)^3 &=\widetilde{G}_F(x,y)^2+\widetilde{D}_F F(x,y)^2.
\end{align}
We note, from Theorem \ref{fisk}, that we will be working exclusively with forms of this shape whenever we wish to treat elliptic curves of conductor $N \equiv 0 \mod{3^3}$.

\subsubsection{The case $j_E = 0$} \label{Mordell}
This case is treated over a general number field in Proposition 4.1 of Cremona and Lingham \cite{CrLi}.
The elliptic curves $E/\mathbb{Q}$ with $j_E=0$ and a given conductor $N$ are particularly easy to determine, since a curve with this property is necessarily isomorphic over $\mathbb{Q}$ to a {\it Mordell} curve with a model of the shape $Y^2 = X^3  - 54 c_6$ where $c_6=c_6(E)$. Such a model is minimal except possibly at $2$ and $3$ and has discriminant 
$-2^6 \cdot 3^9 \cdot c_6^2$ (whereby any primes $p > 2$ which divide $c_6$  necessarily also divide $N$). Here, without loss of generality, we may suppose that $c_6$ is sixth-power-free.
Further, from Tables \ref{tab nu2}, \ref{tab nu3}, and \ref{tab nup}, we have that $\nu_2(N) \in \{ 0, 2, 3, 4, 6 \}$, that $\nu_3 (N) \in \{ 2, 3, 5 \}$, and that $\nu_p(N)=2$ whenever $p \mid N$ for $p > 3$. Given a positive integer $N$ satisfying these constraints, it is therefore a simple matter to check to see if there are elliptic curves $E/\mathbb{Q}$ with conductor $N$ and $j$-invariant $0$. One needs only to compute the conductors of the curves given by $Y^2 = X^3  - 54 c_6$ for each sixth-power-free integer (positive or negative) $c_6$ dividing $64 N^3$.


\subsection{The algorithm}
It is straightforward to convert Theorem~\ref{fisk} into an algorithm for finding all $E/\mathbb{Q}$ of conductor $N$. We can proceed as follows.
\begin{enumerate}
\item Begin by finding all $E/\mathbb{Q}$ of conductor $N$ with $j_E=0$, as outlined in Section \ref{Mordell}.
\item Next, compute $\mbox{GL}_2(\mathbb{Z})$-representatives for every binary form $F$ with discriminant
\begin{align*}
\Delta_F &= \pm 2^{\alpha_0} 3^{\beta_0} N_1
\end{align*}
for each divisor $N_1$ of $N_0$, and each possible pair $(\alpha_0,\beta_0)$ given in the statement of Theorem 
\ref{fisk} (see (\ref{lumpy}) for specifics). We describe an algorithm for listing these forms in Section~\ref{rep}. 
\item Solve the corresponding Thue-Mahler equations\index{Thue-Mahler equations}, finding pairs of integers $(u,v)$ such that
$F(u,v)$ is an $S$-unit, where $S = \{ p \mbox{ prime } : p \mid N \} \cup \{2 \}$ and $F(u,v)$ satisfies the additional conditions given in the statement of Theorem 
\ref{fisk}.
\item For each cubic form $F$ and pair of integers $(u,v)$, consider the elliptic curve
$$
E_1 \; \; : \; \; y^2 = x^3 -27 H_F(u,v) x +27 G_F(u,v)
$$
and all its quadratic twists by squarefree divisors of $6N$. Output those curves with conductor $N$ (if any).
\end{enumerate}
The first, second and fourth steps here are straightforward; the first and second can be done efficiently, while the fourth is essentially 
trivial. The main bottleneck is step~(3). While there is a deterministic procedure for carrying this out (see Tzanakis and de 
Weger \cite{TW2}, \cite{TW3}), it is both involved and, often, computationally taxing. An earlier  implementation of this method in Magma due to Hambrook \cite{Ham} has subsequently been refined by the second author \cite{Ghe}; the most up-to-date version of this code (which we will reference here and henceforth as UBC-TM) is available at 
\begin{center}
\url{http://www.nt.math.ubc.ca/BeGhRe/Code/UBC-TMCode}
\end{center}

We give a number of examples of this general procedure in Section~\ref{examples}. In Section~\ref{primes}, we show that  in the 
special cases where the conductor is prime or the square of a prime, the Thue-Mahler equations~\eqref{TM-eq} (happily) 
reduce to Thue equations (i.e. the exponents on the right hand side of~\eqref{TM-eq} are absolutely bounded).  This situation occurs because, for such elliptic curves, a very strong form of Szpiro's conjecture (bounding the minimal discriminant of an elliptic curve from above in terms of its conductor) is known to hold. Thue equations 
can be solved by routines that are computationally much easier than is currently the case for Thue-Mahler equations; such procedures have been implemented in Pari/GP \cite{PARI2}  and Magma 
\cite{magma}. Further, in this situation,  it is possible to apply a much more computationally efficient argument to find all such elliptic 
curves heuristically but not, perhaps, completely (see Section~\ref{compy}).
 

%------------------------------------------------------------
\section{Proof of Theorem \ref{fisk}} \label{BigProof}
%------------------------------------------------------------

\begin{proof}
Given an elliptic curve $E/\mathbb{Q}$ of conductor $N=2^\alpha 3^\beta N_0$ and invariants $c_4= c_4(E) \neq 0$ and 
$c_6=c_6(E)$, we will construct a corresponding cubic form $F$ explicitly. In fact, our form $F$ will have the property that its leading coefficient will be supported on the primes dividing $6N$, i.e. that
$$
F(1,0) = 2^{\alpha_1} \cdot 3^{\beta_1} \cdot \prod_{p \mid N_0} p^{\kappa_p}.
$$
Define $\mathcal{D}$ as in~\eqref{Dee}, i.e. take 
$\mathcal{D}$ to be the largest integer whose square divides $c_4$ and whose cube divides $c_6$. We then set 
$$
X = c_4/\mathcal{D}^2 \; \; \mbox{ and } \; \; Y = c_6/\mathcal{D}^3,
$$
whereby, from (\ref{first}),
\begin{align} \label{first2}
Y^2 &= X^3 + (-1)^{\delta +1}  M,
\end{align}
for
\begin{align*}
M &=\mathcal{D}^{-6} \cdot 2^6 \cdot 3^3 \cdot |\Delta_E|.
\end{align*}
Note that the assumption that $c_4(E) \neq 0$ ensures that both the $j$-invariant $j_E \neq 0$ and that $X \neq 0$.

It will prove useful to us later to understand precisely the possible common factors among $X, Y, \mathcal{D}$ and $M$.
For any $p>3$, we have $\nu_p(N) \leq 2$. When 
$\nu_p(N)=1$, from Table \ref{tab nup} we find that
\begin{equation} \label{super-1}
( \nu_p (\mathcal{D}), \nu_p (X), \nu_p (Y), \nu_p (M)) = (0,0,0, \geq 1),
\end{equation}
while, if $\nu_p (N)=2$, then either 
\begin{equation} \label{super0}
\nu_p (\mathcal{D}) = 1 \mbox{ and }  \min \{ \nu_p (X), \nu_p (Y) \} = 0, \; \nu_p (M)=0 ,
\end{equation}
or
%
\begin{equation} \label{super}
\nu_p (\mathcal{D}) \leq 1, \;
(\nu_p (X), \nu_p (Y), \nu_p (M) ) = (0,0, \geq 1), (\geq 1, 1, 2), (1, \geq 2, 3) 
\mbox{ or } (\geq 2, 2, 4).
\end{equation}
Things are rather more complicated for the primes $2$ and $3$; we summarize this in Tables~\ref{tab nu2 nxym} and~\ref{tab nu3 
nxym} (which are, in turn, compiled from the data in Tables~\ref{tab nu2} and ~\ref{tab nu3}). 

\begin{table}[h]
$$
\begin{array}{|c|l|} \hline
\nu_2 (N) & (\nu_2 (X), \nu_2 (Y), \nu_2 (M), \nu_2 (\mathcal{D}) ) \\ \hline
0 &  (\geq 2, 0, 0, 1)  \mbox{ or } (0,0,6, 0)  \\
1 & (0,0,\geq 7, 0)  \\
2 & (\geq 2, 2, 4, 1), (\geq 2, 1, 2, 2) \mbox{ or } (0,0,2, 2) \\
3 & (\geq 2, 2, 4, 1), (\geq 2, 1, 2, 2)  \mbox{ or }  (0,0,t, 2), t= 2, 4 \mbox{ or } 5  \\
4 &  (\geq 2, 2, 4, 1), (\geq 2, 1, 2, 2), (\geq 2, 0, 0, 3)  \mbox{ or }  (0,0,t, 2), t= 2 \mbox{ or } t \geq 4  \\
5 & (\geq 0, \geq 0, 0, 2), (0, \geq 0, 0, 3), (0,0,3, 2) \mbox{ or } (1,0,0, 3)  \\
6 & (\geq 0, \geq 0, 0, 2), (0, \geq 0, 0, 3), (\geq 2, 2, 4, 2), (\geq 2, 1, 2, 3) \mbox{ or }  (0,0,\geq 2, 3) \\
7 & (0,0,1,2), (0,0,1,3), (1,1,2,2) \mbox{ or } (1,1,2,3) \\
8 & (1, \geq 2, 3, 2) \mbox{ or } (1, \geq 2, 3, 3). \\
 \hline
\end{array}
$$
\caption{The possible values of $\nu_2(N), \nu_2 (X), \nu_2 (Y), \nu_2(M)$ and $\nu_2 (D)$}
\label{tab nu2 nxym}
\end{table}

\begin{table}[h]
$$
\begin{array}{|c|l|} \hline
\nu_3 (N) & (\nu_3 (X), \nu_3 (Y), \nu_3 (M), \nu_3 (\mathcal{D}) ) \\ \hline
0 & (1, \geq 3, 3, 0)  \mbox{ or } (0,0,3, 0)  \\
1 & (0,0,\geq 4, 0)  \\
2 & (\geq 0, 0, 0, 1), (0, \geq 2, 0, 1), (0,0, \geq 3, 1), (1, \geq 3, 3, 1), (\geq 0,0,0,2) 
\mbox{ or } (0,\geq 2, 0, 2)  \\
3 & (\geq 0, 0, 0, 1),  (\geq 0, 0, 0, 2), (0,1,0, 1), (0,1,0, 2), (0,0,2,1) \mbox{ or } (0,0,2, 2) \\
4 &  (0,0,1,1), (0,0,1,2), (1,2,3,1) \mbox{ or } (1, 2, 3, 2) \\
5 &   (\geq 1, 1, 2, 1), (\geq 1, 1, 2, 2), (\geq 2, 2, 4, 1) \mbox{ or }  (\geq 2, 2, 4,2). \\
 \hline
\end{array}
$$
\caption{The possible values of $\nu_3(N), \nu_3(X), \nu_3 (Y), \nu_3(M)$ and $\nu_3 (D)$}
\label{tab nu3 nxym}
\end{table}

We will construct a cubic form
$$
 F_1(x,y) = ax^3 + 3b_0 x^2y + 3c_0 xy^2 + dy^3,
$$
one coefficient at a time; our main challenge will be to ensure that the $a, b_0, c_0$ and $d$ we produce are actually integral rather than just rational. The form $F$ whose existence is asserted
in the statement of Theorem \ref{fisk} will turn out to be either $F_1$ or $F_1/3$. 

Let us write
\begin{align*}
  M = M_1 \cdot M_2
\end{align*}
where $M_2$ is the largest integer divisor of $M$ that is coprime to $X$, so that
$$
M_1 = \prod_{p \, \mid  \, X} p^{\nu_p (M)} \; \; \mbox{ and } \; \;  M_2 = \prod_{p \, \nmid \, X} p^{\nu_p (M)}.
$$
We define
\begin{equation} \label{a1}
a_1 = \prod_{p \mid M_1} p^{\left[ \frac{\nu_p (M)-1}{2} \right]}
\end{equation}
and set
\begin{equation} \label{a2}
a_2 = \left\{
\begin{array}{cl}
3^{-1} \, \prod_{p \mid M_2} p^{\left[ \frac{\nu_p (M)}{2} \right]} & \mbox{ if }  \nu_3 (X)=0, \, \nu_3(M) =2t, \, t \in \mathbb{Z}, t \geq 2,  \\
\prod_{p \mid M_2} p^{\left[ \frac{\nu_p (M)}{2} \right]} & \mbox{ otherwise. } \\
\end{array}
\right.
\end{equation}
Define $a = a_1 \cdot a_2$.
It follows that $a_1^2 \mid M_1$ and, from (\ref{super-1}), (\ref{super0}), (\ref{super}), and Tables \ref{tab nu2 nxym} and \ref{tab nu3 nxym}, that both
$$
a_1 \mid X \; \; \mbox{ and } \; \;  a_1^2 \mid Y.
$$
We write $X = a_1 \cdot X_1$ and observe that $a_2^2 \mid M_2$. Note that $a_2$ is coprime to $X$ and hence to $a_1$. 
Since $a^2 \mid M$, we may thus define a positive integer $K$ via $K = M/a^2$, so that (\ref{first2}) becomes
\begin{align*}
Y^2-X^3 &= (-1)^{\delta+1} Ka^2.
\end{align*}
From the fact that $\gcd (a_2,X)=1$ and $X \neq 0$, we may choose $B$ so that
\begin{align*}
  a_2 B &\equiv -Y/a_1 \mod{X^3},
\end{align*}
whereby 
\begin {equation} \label{truth}
a B+Y \equiv 0 \mod{a_1X^3}.
\end{equation}  
Note that, since $a_1^2 \mid Y$ and $a_1 \mid X$, it follows that $a_1 \mid B$.
Let us define
\begin{equation} \label{definite}
b_0 = \frac{aB+Y}{X}, \; \; 
c_0 = \frac{b_0^2-X}{a} \; \; \mbox{ and } \; \; 
d = \frac{b_0c_0 - 2B}{a}.
\end{equation}
We now demonstrate that these are all integers. That $b_0 \in \mathbb{Z}$ is immediate from (\ref{truth}).
Since $b_0 X - Y = aB$, we know that $b_0 X \equiv Y \mod{a}$. Squaring both sides 
thus gives 
$$
b_0^2 X^2 \equiv Y^2 \equiv X^3 + (-1)^{\delta+1} K a^2  \equiv X^3 \mod{a_1 \cdot a_2},
$$
and, since $\gcd (a_2,X)=1$,
$$
 b_0^2 \equiv X \mod{a_2}.
$$
From (\ref{truth}), we have $b_0 \equiv 0 \mod{a_1 X^2}$, whereby, since $a_1 \mid X$, 
$$
b_0^2 \equiv X \equiv 0 \mod{a_1}.
$$
The fact that $\gcd(a_1,a_2)=1$ thus allows us to conclude that $b_0^2 \equiv X \mod{a}$ and hence that $c_0 \in \mathbb{Z}$.

It remains to show that $d$ is an integer.
Let us rewrite $ad$ as
$$
ad = b_0c_0-2B = \left( \frac{aB+Y}{aX} \right) \left( \left( \frac{aB+Y}{X} \right)^2 - X \right) - 2B,
$$
so that
$$
ad =\left( \frac{aB+Y}{aX} \right) \left( \frac{(-1)^{\delta+1} K a^2 + 2 a B Y + a^2 B^2}{X^2} \right) - 2B.
$$
Expanding, we find that
\begin{equation} \label{three}
X^3 d = (-1)^{\delta+1} KY+ 3 Y B^2 + a B^3 + (-1)^{\delta+1} 3 KaB.
\end{equation}
We wish to show that
$$
(-1)^{\delta+1} KY+ 3 Y B^2 + a B^3 + (-1)^{\delta+1} 3 KaB \equiv 0 \mod{X^3}.
$$
From (\ref{truth}), we have that
$$
(-1)^{\delta+1} KY+ 3 Y B^2 + a B^3 + (-1)^{\delta+1} 3 KaB \equiv 2Y \left( B^2 + (-1)^\delta K \right) \mod{a_1 X^3}.
$$
Multiplying congruence (\ref{truth}) by $aB-Y$ (which, from our prior discussion, is divisible by $a_1^2$), 
we find that 
$$
a^2 B^2 \equiv Y^2 \equiv X^3 + (-1)^{\delta+1} K a^2 \mod{a_1^3 X^3}
$$
and hence, dividing through by $a_1^2$,
$$
a_2^2 B^2 \equiv  a_1 X_1^3 + (-1)^{\delta+1} K a_2^2 \mod{a_1 X^3}.
$$
It follows that
\begin{equation} \label{three-2}
B^2 + (-1)^\delta K \equiv a_2^{-2} a_1 X_1^3  \mod{a_1 X^3},
\end{equation}
and so, since $a_1^2 \mid Y$, 
$$
Y \left( B^2 + (-1)^\delta K \right) \equiv 0 \mod{X^3},
$$
whence we conclude that $d$ is an integer, as desired.

With these values of $a, b_0, c_0$ and $d$, we can then confirm (with a quick computation) that the cubic form
\begin{align*}
  F_1(x,y) &= ax^3 + 3b_0 x^2y + 3c_0 xy^2 + dy^3
\end{align*}
has discriminant
\begin{align*}
  D_{F_1} &= \frac{108}{a^2} (X^3-Y^2) = (-1)^\delta \cdot 2^2 \cdot 3^3 \cdot K
\end{align*}
We also note that 
$$
F_1(1,0) = a, \; \; \widetilde{H}_{F_1}(1,0) = b_0^2 -a c_0 = X
$$
and
$$
 -\frac{1}{2} \widetilde{G}_{F_1}(1,0) = \frac{1}{2}(a^2d -3ab_0c_0+2b_0^3) = Y,
$$
where $\widetilde{G}_{F}$ and $\widetilde{H}_{F}$ are as in Section \ref{dahlia}.

Summarizing Table \ref{tab nu3 nxym},  we find that we are in one of the following four cases :

\begin{enumerate}
\item[(i)] $\nu_3 (X)=1, \;  \nu_3(Y) =2,  \; \nu_3(M)=3 \; \mbox{ and } \;  \nu_3(N)=4,$
\item[(ii)] $\nu_3(X) \geq 2, \;  \nu_3(Y) =2, \; \nu_3(M) =4, \; \nu_3(N)=5,$
\item[(iii)] $\nu_3(M) \leq 2 $ and $ \nu_3(N) \geq 2$, or
\item[(iv)] $\nu_3 (M)  \geq 3$ and either $\nu_3 (XY)=0 $ or $\nu_3(X)=1, \,  \nu_3(Y) \geq 3$.
\end{enumerate}
In cases (i), (ii), and (iii), we choose $F=F_1$, i.e.
\begin{align*}
(\omega_0,\omega_1,\omega_2,\omega_3) = (a, 3 b_0, 3 c_0, d),
\end{align*}
so that
$$
F(1,0)=a, \; \; D_{F} = (-1)^\delta 2^2 \cdot 3^3 \cdot K, \; \; 
c_4 = \mathcal{D}^2 \widetilde{H}_{F} (1,0) \; \; \mbox{ and } \; \; 
c_6 = - \frac{1}{2}  \mathcal{D}^3 \widetilde{G}_{F} (1,0).
$$
It follows that $E$ is isomorphic over $\mathbb{Q}$ to the curve
$$
y^2 = x^3 -27 c_4 x -54 c_6 = x^3 - 3 \mathcal{D}^2 H_{F}(1,0) x + \mathcal{D}^3 G_{F}(1,0).
$$

In  case (iv), observe that, from definitions (\ref{a1}) and (\ref{a2}),
\begin{equation} \label{foster}
\nu_3(a) = \left[ \frac{\nu_3(M)-1}{2} \right] \; \; \mbox{ and } \; \; 
\nu_3(K) = \nu_3(M) - 2 \nu_3 (a),
\end{equation}
so that $3 \mid a$ and $3 \mid K$. From equation (\ref{three}),  $3 \mid X^3 d$.
If $\nu_3(X)=0$ this implies that $3 \mid d$. On the other hand, 
if $\nu_3(X)=1$, then, from (\ref{three-2}), we may conclude that $3 \mid B$. Since each of $a, B$ and $K$ is divisible by $3$, while $\nu_3 (X) =1$ and $\nu_3 (Y) \geq 3$,  equation (\ref{three})
once again implies that $3 \mid d$.
In this case, we can therefore write $a = 3a_0$ and $d=3d_0$, for integers $a_0$ and $d_0$ and 
set $F = F_1/3$, i.e. take
\begin{align*}
(\omega_0,\omega_1,\omega_2,\omega_3) = (a_0, b_0, c_0, d_0).
\end{align*}
We have
$$
F(1,0) =a/3, \; \;  D_{F}  = (-1)^\delta 2^2  \cdot K/3, \; \; 
c_4 = \mathcal{D}^2 H_{F} (1,0)  \; \; \mbox{ and } \; \;  
c_6 = - \frac{1}{2}  \mathcal{D}^3 G_{F} (1,0).
$$
The curve $E$ is now isomorphic over $\mathbb{Q}$ to the model
$$
y^2 = x^3 -27 c_4 x -54 c_6 = x^3 - 27 \mathcal{D}^2 H_{F}(1,0) x + 27 \mathcal{D}^3 G_{F}(1,0).
$$

Since $|D_F|/D_F = (-1)^\delta$ and $a^2 K \mid 1728 \Delta_E$, we may write
$$
F(1,0) = 2^{\alpha_1} \cdot 3^{\beta_1} \cdot \prod_{p \mid N_0} p^{\kappa_p}
\; \; \mbox{ and } \; \; D_F = (|\Delta_E|/\Delta_E) 2^{\alpha_0} 3^{\beta_0} N_1,
$$
for nonnegative integers $\alpha_0, \alpha_1, \beta_0, \beta_1, \kappa_p$ and a positive integer $N_1$,  divisible only by primes dividing $N_0$. More explicitly, we have 
$$
\alpha_0 = \nu_2 (K) + 2\; \; \mbox{ and } \;  \; \beta_0 =\nu_3 (K) + 
\left\{ 
\begin{array}{rl}
3 & \mbox{ in case (i), (ii) or (iii), or } \\
-1& \mbox{ in case (iv),} \\
\end{array}
\right.
$$
and
$$
\alpha_1 = \nu_2 (a) \; \; \mbox{ and } \;  \; \beta_1 =\nu_3 (a) + 
\left\{ 
\begin{array}{rl}
0 & \mbox{ in case (i), (ii) or (iii), or } \\
-1& \mbox{ in case (iv). } \\
\end{array}
\right.
$$
It remains for us to prove that these integers satisfy the conditions listed in the statement of the theorem. It is straightforward to check this,  considering in turn each possible triple $(X,Y,M)$ from 
(\ref{super-1}), (\ref{super0}), (\ref{super}), and Tables~\ref{tab nu2 nxym} and ~\ref{tab nu3 nxym}, and using the fact that $K = M/a^2$. 
%, where
%$$
%a = a_3 \mbox{ or } a_3/3, \mbox{ according to (\ref{a2}), } \; \mbox{and } \; 
%a_3 = \prod_{p \mid M} p^{\left[ \frac{\nu_p (M)-\delta_p}{2} \right]} \; \mbox{ for } \; \; 
%\delta_p =
%\left\{
%\begin{array}{ll}
%1 & \mbox{ if } p \mid X, \\
%0 & \mbox{ if } p \nmid X. \\
%\end{array}
%\right.
%$$

In particular, if $p > 3$, we have $\nu_p(\Delta_E) = 6 \nu_p(\mathcal{D}) + \nu_p (D_F) + 2 \kappa_p$. From Table \ref{tab nup} and (\ref{Dee}), we have $\nu_p(\mathcal{D}) \leq 1$, whereby (\ref{term0}) follows. 
Further,
\begin{equation} \label{toothbrush}
\nu_p(a) = 
\left\{\begin{array}{ll}
\left[ \frac{\nu_p(M)-1}{2} \right] & \mbox{ if } p \mid X, \\
\left[ \frac{\nu_p(M)}{2} \right] & \mbox{ if } p \nmid X, \\
\end{array}
\right.
\end{equation}
and so, if $p \nmid X$, 
$$
\nu_p(M)-2 \nu_p(a) \leq 1.
$$
Since $a^2K=M$, if $p^2 \mid D_F$, then $\nu_p(N)=2$ and it follows that we are in case (\ref{super}), with $p \mid X$. We may thus conclude that $\nu_p (M) \in \{ 2, 3, 4 \}$ and hence, from (\ref{toothbrush}), that $\nu_p(a) \leq 1$. This proves (\ref{term1}).

For (\ref{term2}), note that, in cases (i), (ii) and (iii), we clearly have that  $3 \mid \omega_1$ and $3 \mid \omega_2$.  In case (iv), from (\ref{foster}),
$$
\beta_0 = \nu_3 (D_F) = \nu_3(K) -1 = \nu_3(M) - 2  \left[ \frac{\nu_3(M)-1}{2} \right] - 1 \in \{ 0, 1 \}.
$$
Finally, to see (\ref{term3}), note that if $\nu_p(N)=1$, for  $p > 3$, then we have (\ref{super-1}) and hence
$$
\nu_p(D_F) + 2 \nu_p (F(u,v)) = \nu_p(M) \geq 1,
$$
whereby $p \mid D_F$ or $p \mid F(u,v)$.
We may also readily check that the same conclusion obtains for $p=3$ (since, equivalently, $\beta_0+\beta_1 \geq 1$). This completes the proof of Theorem \ref{fisk}.


\end{proof}

To illustrate this argument, suppose we consider the elliptic curve (denoted 109a1 in Cremona's database) defined via
$$
E \; \; : \; \; y^2 + xy = x^3 - x^2 -8 x -7,
$$
with $\Delta_E=-109$.
We have
$$
c_4(E) = 393 \; \; \mbox{ and } \; \; c_6(E)=7803,
$$
so that $\gcd (c_4(E), c_6(E))=3$. It follows that
$$
\mathcal{D} = 1, \; X=393, \; Y=7803, \; \delta=1, \; M = 2^6 \cdot 3^3 \cdot 109,
$$
and hence we have
$$
M_1 = 3^3, \; M_2=2^6 \cdot 109, \; a_1 = 3, \; a_2 = 2^3, \; a=2^3 \cdot 3 \; \mbox{ and } \; K = 3 \cdot 109.
$$
We solve the congruence $8B \equiv -2601 \mod{393^3}$ to find that we may choose $B=7586982$, so that
$$
b_0 =463347, \; \; 
c_0 = 8945435084 \; \; \mbox{ and } \; \; 
d =172701687278841.
$$
We are in case (iv) and thus set
$$
F(x,y) = 8 x^3 + 463347 x^2 y + 8945435084 x y^2 + 57567229092947 y^3,
$$
with discriminant $D_F = -4 \cdot 109$,
$$
G_F(1,0) = -15606 = -2 c_6(E) \; \; \mbox{ and } \; \; H_F(1,0)=393 = c_4(E).
$$
The curve $E$ is thus isomorphic to the model
\begin{equation} \label{flag}
E_{\mathcal{D}} \; \; : \; \; y^2 = x^3 - 27 \mathcal{D}^2 H_{F}(1,0) x + 27 \mathcal{D}^3 G_{F}(1,0) = x^3 - 10611x-421362.
\end{equation}

We observe that the form $F$ is $\mbox{GL}_2(\mathbb{Z})$-equivalent to a ``reduced'' form (see Section \ref{rep} for details),  given by
$$
\tilde{F} (x,y) = x^3 + 3 x^2y + 4 x y^2 + 6 y^3.
$$
In fact, this is the only form (up to $\mbox{GL}_2(\mathbb{Z})$-equivalence) of discriminant $\pm 4 \cdot 109$. We can check that the solutions to the Thue equation
$\tilde{F}(u,v)=8$ are given by $(u,v)=(2,0)$ and $(u,v)=(-7,3)$. 
The minimal quadratic twist of 
$$
y^2 = x^3 -27 H_{\tilde{F}}(2,0) x +27 G_{\tilde{F}}(2,0)
$$
has conductor $2^5 \cdot 109$ and hence cannot correspond to $E$. For the solution $(u,v)=(-7,3)$, we find that the curve given by the model
$$
y^2 = x^3 -27 H_{\tilde{F}}(-7,3) x +27 G_{\tilde{F}}(-7,3) =x^3 - 10611x+421362,
$$
is the quadratic twist by $-1$ of the curve (\ref{flag}). This situation arises from the fact that $G_F$ is an $\mbox{SL}_2(\mathbb{Z})$-covariant, but not a $\mbox{GL}_2(\mathbb{Z})$-covariant of $F$ (we will discuss this more in the next section).

%---------------------------------------------------------------
\section{Finding representative forms} \label{rep}
%---------------------------------------------------------------

As Theorem \ref{fisk} illustrates, we are able to tabulate elliptic curves over $\mathbb{Q}$ with good reduction outside a given set of primes, 
by finding a set of representatives for $\mbox{GL}_2 ( \mathbb{Z})$-equivalence classes of binary cubic forms with 
certain discriminants, and then solving a number of Thue-Mahler equations. In this section, we will provide a brief description of techniques to find 
distinguished \emph{reduced} representatives for equivalence classes of cubic forms over a given range of 
discriminants. For both positive and negative discriminants, the notion of \emph{reduction} arises from associating a 
particular definite quadratic form to a given cubic form. 
%We do not make these quadratic forms explicit, but rather 
%state our definitions of reduction solely in terms of the coefficients of the cubic forms.

%-------------------------------------------------------
\subsection{Irreducible Forms}
%-------------------------------------------------------
For forms of positive discriminant, there is a well developed classical theory of reduction dating back to work of
Hermite \cite{Her1}, \cite{Her2} and, later, Davenport (see e.g. \cite{Dav}, \cite{Dav2} and \cite{DaHe}).  We can actually apply 
this method to both reducible and irreducible forms. Initially, though, we will assume the forms are irreducible, since we will treat 
the elliptic curves corresponding to reducible forms by a somewhat different approach (see Section~\ref{ssec reducible}). 
Note that when one speaks of ``irreducible, reduced forms'',
as Davenport observes,
``the terminology is unfortunate, but can hardly be avoided'' (\cite{Dav3}, page 184).

In each of Belabas \cite{Be}, Belabas and Cohen \cite{BeCo} and Cremona 
\cite{Cr}, we find very efficient algorithms for computing cubic forms of both positive and negative discriminant, refining classical work of Hermite, Berwick 
and Mathews \cite{BeMa}, and Julia \cite{Ju}. These are  readily translated into computer code to loop over valid $(a,b,c,d)$-values (with corresponding forms $ax^3+bx^2y+cxy^2+dy^3$).  The running time in each case is linear in the upper bound $X$.
Realistically, this step (finding representatives for our cubic forms) is highly unlikely to be the bottleneck in our computations.
  
%----------------------------------------------------------------------
\subsection{Reducible forms} \label{ssec reducible}
%----------------------------------------------------------------------
One can make similar definitions of reduction for reducible forms (see \cite{BeGh} for example). However, for our 
purposes, it is sufficient to note that a reducible form is equivalent to 
$$
F(x,y) = b x^2 y + c x y^2 + d y^3 \; \; \mbox{ with }  \; \; 0 \leq d \leq c,
$$
which has discriminant
\begin{align*}
\Delta_F &= b^2 (c^2 - 4 b d).
\end{align*}

To find all elliptic curves with good reduction outside $S = \{ p_1, p_2, \ldots, p_k \}$, corresponding to 
reducible cubics in Theorem~\ref{fisk} (i.e. those $E$ with at least one rational $2$-torsion point), it is enough to find all such triples $(b,c,d)$ for which there exist integers $x$ and $y$ so that both
$$
b^2(c^2-4bd) \; \; \mbox{ and } \; \;  b x^2 y + c x y^2 + d y^3
$$
are $S^*$-units (with $S^* = S \cup \{ 2 \}$). For this to be true, it is necessary that each of the integers
$$
b, \;  \; c^2 - 4 bd,\;  \; y \; \; \mbox{ and } \; \;  \mu = bx^2 + cxy + dy^2
$$
is an $S^*$-unit. Taking the discriminant of $\mu$ as a function of $x$, we thus require that 
\begin{align} \label{weger}
(c^2 - 4 bd )  y^2  + 4 b \mu &=Z^2,
\end{align}
for some integer $Z$. This is an equation of the shape 
\begin{align} \label{S-square}
 X+Y &=Z^2
\end{align}
in $S^*$-units $X$ and $Y$. 

An algorithm for solving such equations is described in detail in Chapter~7 of de~Weger \cite{Weg0} (see also 
\cite{Weg}); it relies on bounds for linear forms in $p$-adic and complex logarithms and various reduction 
techniques from Diophantine approximation. An implementation of this is available at 
\begin{center}
\url{http://www.nt.math.ubc.ca/BeGhRe/Code/UBC-TMCode}.
\end{center}

While \emph{a priori} equation~\eqref{S-square} arises as only a necessary condition for the existence 
of an elliptic curve of the desired form, given any solution to~\eqref{S-square} in $S^*$-units $X$ and $Y$ and integer $Z$, the curves 
$$
E_1(X,Y) \; \; : \; \; y^2 = x^3 + Z x^2 + \frac{X}{4} x
$$
and
$$
E_2(X,Y) \; \; : \; \; y^2 = x^3 + Z x^2 + \frac{Y}{4} x
$$
have nontrivial rational $2$-torsion (i.e. the point corresponding to $(x,y)=(0,0)$) and discriminant $X^2Y$ and $XY^2$, respectively (and hence good reduction at all primes outside $S^*$). 

Though a detailed analysis of running times for solving equations of the shape (\ref{S-square}), or for solving more general cubic Thue-Mahler equations, has not to our knowledge been carried out, our experience from carrying out such computations for several thousand sets $S$ is that, typically,  the former can be done significantly faster than the latter. By way of example, solving (\ref{S-square}) for $S=\{ 2, 3, 5, 7, 11 \}$ takes only a few hours on a laptop, while treating the analogous problem of determining all elliptic curves over $\mathbb{Q}$ with trivial rational $2$-torsion and good reduction outside $S$ (see Section \ref{gummy-bear}) requires many thousand machine-hours.

%--------------------------------------------------------------------
\subsection{Computing forms of fixed discriminant} \label{pickles}
%--------------------------------------------------------------------

For our purposes, we will typically compute and tabulate a large list of irreducible forms of absolute discriminant bounded by a given positive number $X$ (of size up to $10^{12}$ of so, beyond which storage becomes problematical). In certain situations, however, we will want to compute all forms of a given fixed, larger discriminant (perhaps up to size $10^{15}$). To carry this out and find desired forms of the shape $ax^3+bx^2y+cxy^2+dy^3$, we can argue as in, for example, Cremona \cite{Cr}, to restrict our attention to $O(X^{3/4})$ triples $(a,b,c)$.
From (\ref{claire-bear}), the definition of $D_F$, we have that
$$
d = \frac{9abc-2b^3 \pm \sqrt{4 (b^2-3ac)^3-27 a^2 D_F}}{27 a^2}
$$
and hence it remains to check that the quantity $4 (b^2-3ac)^3-27 a^2 D_F$ is an integer square, that the relevant conditions modulo $27a^2$ are satisfied, and that a variety of further inequalities from \cite{Cr} are satisfied. The running time for finding forms with discriminants of absolute value of size $X$ via this approach is of order $X^{3/4}$.

%----------------------------------------------------------------------------------------------------------------
\subsection{$\mbox{GL}_2(\mathbb{Z})$ vs $\mbox{SL}_2(\mathbb{Z})$} \label{note}
%----------------------------------------------------------------------------------------------------------------

One last observation which is very important to make before we proceed, is that while $G_F^2$ is $\mbox{GL}_2(\mathbb{Z})$-covariant\index{covariants}, the same is not actually true for $G_F$ (it is, however, an $\mbox{SL}_2(\mathbb{Z})$-covariant). This may seem like a subtle point, but what it means for us in practice is that, having found our $\mbox{GL}_2(\mathbb{Z})$-representative forms $F$ and corresponding curves of the shape $E_{\mathcal{D}}$ from Theorem \ref{fisk}, we need, in every case, to also check to see if  
$$
\tilde{E}_{\mathcal{D}} \; \; : \; \; 3^{[\beta_0/3]} y^2 = x^3 -27 \mathcal{D} ^2 H_F(u,v) x -27 \mathcal{D} ^3 G_F(u,v),
$$
the quadratic twist of $E_{\mathcal{D}}$ by $-1$, yields a curve of the desired conductor\index{conductor}.

%--------------------------------------------------------------------------------------------
\section{Examples}\label{examples}
%--------------------------------------------------------------------------------------------

In this section, we will describe a few applications of Theorem \ref{fisk} to computing all elliptic curves of a fixed conductor $N$, or all curves with good reduction outside
a given set of primes $S$. We restrict our attention to examples with composite conductors, since the case of conductors $p$ and $p^2$, for $p$ prime, 
will be treated at length in Section \ref{primes} (and subsequently). For the examples in Sections \ref{lapel}, \ref{exe1}, \ref{exe2} and \ref{exe3}, since the conductors under discussion are not ``square-full'', there are necessarily no curves $E$ encountered with $j_E=0$.

In our computations in this section, we executed all jobs in parallel via the shell tool \cite{Tange2011a}. We note that our Magma code lends itself easily to parallelization, and we made full use of this fact throughout.

We carried out a one-time computation of all irreducible cubic forms that can arise in Theorem \ref{fisk}, of absolute discriminant bounded by $10^{10}$. This computation took slightly more than $3$ hours on a cluster of $40$ cores; roughly half this time was taken up with sorting and organizing output files.  There are $996198693$  classes of  irreducible cubic forms of positive discriminant and $3079102475$ of negative discriminant in the range in question; storing them requires roughly $120$ gigabytes. We could also have tabulated and stored representatives for each class of reducible form of absolute discriminant up to $10^{10}$, but chose not to since our approach to solving equation (\ref{S-square}) does not require them.

%--------------------------------------------------------------------------------------------
\subsection{Cases without irreducible forms} \label{lapel}
%--------------------------------------------------------------------------------------------

We begin by noting an obvious corollary to Theorem \ref{fisk} that, in many cases, makes it a relatively routine matter to determine all elliptic curves of a given conductor, provided  we can show the nonexistence of certain corresponding cubic forms.
\begin{corollary} \label{impulse}
Let $N$ be a square-free positive integer with $\gcd(N,6)=1$ and suppose that there do not exist irreducible binary cubic forms in $\mathbb{Z}[x,y]$ of discriminant $\pm 4N_1$, for each positive integer $N_1 \mid N$. Then every elliptic curve over $\mathbb{Q}$ of conductor $N_1$,  for each  $N_1 \mid N$, has nontrivial rational $2$-torsion.
\end{corollary}

We will apply this result to  a pair of examples (chosen somewhat  arbitrarily). Currently, such an approach is feasible for forms of absolute discriminant (and hence potentially conductors) up to roughly $10^{15}$. 
We observe that, among the positive integers $N < 10^8$ satisfying
$$
\nu_2(N) \leq 8, \; \; \nu_3 (N) \leq 5 \; \; \mbox{ and } \; \; \nu_p (N) \leq 2 \; \mbox{ for } \; p > 3,
$$
i.e. those for which there might actually exist elliptic curves $E/\mathbb{Q}$ of conductor $N$, we find that $708639$ satisfy the hypotheses of Corollary \ref{impulse}. 

It is somewhat harder to modify the statement of Corollary \ref{impulse} to include reducible forms (with corresponding elliptic curves having nontrivial rational $2$-torsion). One of the difficulties one encounters is that there actually do exist reducible forms of, by way of example, discriminant $4p$ for every $p \equiv 1 \mod{8}$; writing $p = 8k+1$, for instance, the form
$$
F(x,y) = 2 x^2y+xy^2-ky^3
$$
has this property.

%--------------------------------------------------------------------------------------------
\subsubsection{Conductor $2655632887 = 31 \cdot 9007 \cdot 9511$}
%--------------------------------------------------------------------------------------------

In the notation of Theorem \ref{fisk}, we have $\alpha=\beta=0$ and hence $\alpha_0 =2$ and $\beta_0 =0$, so that, in order for there to be an elliptic curve with trivial rational $2$-torsion and this conductor, we require the existence of an irreducible cubic form of discriminant $4 N_1$ where 
$N_1 \mid  31 \cdot 9007 \cdot 9511$,
i.e. discriminant $\pm 4 \cdot 31^{\delta_1} \cdot 9007^{\delta_2} \cdot 9511^{\delta_3}$, for $\delta_i \in \{ 0, 1 \}$. We check that there are no such forms, directly from our table of forms, except for the possibility of $D_F = \pm 4 \cdot 31 \cdot 9007 \cdot 9511$, which exceeds $10^{10}$ in absolute value. For these latter possibilities, we argue as in Section \ref{pickles} to show that no such forms exist. We may thus appeal to Corollary \ref{impulse}.

For the possible cases with rational $2$-torsion, we solve $X+Y=Z^2$ with $X$ and $Y$ $S$-units for $S = \{ 2, 31, 9007, 9511 \}$. The solutions to this equation with $X \geq Y$, $Z > 0$ and $\gcd (X,Y)$ squarefree are precisely those with
$$
\begin{array}{lll}
(X,Y) &= &  (2,-1), (2,2), (8,1), (32,-31), (62,2), (256,-31), (961,128), \\
& & (992,-31), (3968,1), (76088,-9007), (294841, 8) \mbox{ and } (492032,-9007).  \\
\end{array}
$$
A short calculation confirms that each elliptic curve arising from these solutions via quadratic twist has bad reduction at the prime $2$ (and, in particular, cannot have conductor $2655632887$). There are thus no elliptic curves over $\mathbb{Q}$ with conductor $2655632887$.
Observe that these calculations in fact ensure that there do not exist elliptic curves over $\mathbb{Q}$ with conductor dividing $2655632887$. 

Full computational details are available at
\begin{center}
\url{http://www.nt.math.ubc.ca/BeGhRe/Examples/2655632887-data}.
\end{center}
We should observe that it is much more challenging computationally to try to extend this argument to tabulate curves $E$ with good reduction outside 
$S = \{ 31, 9007, 9511 \}$. To do this, we would have to first determine whether or not there exist irreducible cubic forms of discriminant, say,
$D_F = \pm 4 \cdot 31^2 \cdot 9007^2 \cdot 9511^2 > 2.8 \times 10^{19}$. This appears to be at or beyond current computational limits.

%--------------------------------------------------------------------------------------------
\subsubsection{Conductor $3305354359 = 41 \cdot 409 \cdot 439 \cdot 449$}
%--------------------------------------------------------------------------------------------

For there to exist an elliptic curve with trivial rational $2$-torsion and conductor $3305354359$, we require the existence of an irreducible cubic form of discriminant $\pm 4  \cdot 41^{\delta_1} \cdot 409^{\delta_2} \cdot 439^{\delta_3} \cdot 449^{\delta_4}$, with $\delta_i \in \{ 0, 1 \}$. We check that, again, there are no such forms (once more employing a short auxiliary computation in the case $D_F =\pm 4  \cdot 41 \cdot 409 \cdot 439 \cdot 449$). If we solve $X+Y=Z^2$ with $X$ and $Y$ $S$-units for $S = \{ 2, 41, 409, 439, 449 \}$, we find that the solutions to this equation with $X \geq Y$, $Z > 0$ and $\gcd (X,Y)$ squarefree are precisely 
$$
\begin{array}{l}
(X,Y) =  (2,-1), (2,2), (8,1), (41,-16),  (41,-32), (41,8), (82,-1), (128,41), ( 409, -328),  \\
 \hskip9.6ex (409, 32), (439, 2), (449, -328), (449, -8), (512, 449), (818, 82), (898, 2), \\
\hskip9.6ex   (3272, 449), (3362, 2), (7184, 41), (16769, -128), (16769, -14368),  (18409, -16384), \\
 \hskip9.6ex  (33538, -18409), (36818, 818), (41984, 41), (68921, -57472), (183641, -1312),\\
\hskip9.6ex   (183641, -56192), (183641, 41984), (359102, 898), (403202, -33538), \\
\hskip9.6ex   (403202, -359102), (403202, 17999), (737959, 183641), (754769, -6544), \\
\hskip9.6ex  (6858521, -919552), (8265641, -16) \mbox{ and } (7095601778, -5610270178). \\
\end{array}
$$
Once again, a short calculation confirms that each elliptic curve arising from these solutions via twists has even conductor. There are thus no elliptic curves over $\mathbb{Q}$ with conductor $3305354359$.

Full computational details are available at
\begin{center}
\url{http://www.nt.math.ubc.ca/BeGhRe/Examples/3305354359-data}.
\end{center}

%------------------------------------------------------------------------------------------------------
\subsection{Cases with fixed conductor (and corresponding irreducible forms)}
%------------------------------------------------------------------------------------------------------

%--------------------------------------------------------------------------------------------
\subsubsection{Conductor $399993 = 3 \cdot 11 \cdot 17 \cdot 23 \cdot 31$} \label{exe1}
%--------------------------------------------------------------------------------------------

We next choose an example where full data is already available for comparison in the LMFDB \cite{LMFDB}. In particular, there are precisely $10$ isogeny classes of curves of this conductor (labelled $399993a$ to $399993j$ in the LMFDB), containing a total of $21$ isomorphism classes. Of these, $7$ isogeny classes (and $18$ isomorphism classes) have nontrivial rational $2$-torsion.

According to Theorem \ref{fisk}, the curves arise from  consideration of cubic forms of discriminant discriminant $\pm 4  K$, where $K \mid 3 \cdot 11 \cdot 17 \cdot 23 \cdot 31$. 
The (reduced)  irreducible cubic forms $F(u,v)$ of these discriminants are as follows, where $F(u,v) =  \omega_0 u^3 + \omega_1 u^2v + \omega_2 uv^2 + \omega_3 v^3$.

$$
\begin{array}{cc|cc} 
(\omega_0,\omega_1,\omega_2,\omega_3) & D_F & (\omega_0,\omega_1,\omega_2,\omega_3)  & D_F \\ \hline
(1,1,1,3) & -4 \cdot 3 \cdot 17 & (2, 4, -6, -3) & 4 \cdot 3 \cdot 17 \cdot 23 \\
(1,2,2,2) & - 4 \cdot 11 & (2, 5, 2, 6) & -4 \cdot 3 \cdot 17 \cdot 23 \\
(1,2,2,6) & -4 \cdot 11 \cdot 17 & (3, 3, -8, -2) & 4 \cdot 3 \cdot 23 \cdot 31 \\
(1, 4, -16, -2) & 4 \cdot 11 \cdot 17 \cdot 31 & (3, 3, 44, 66) &  -4 \cdot 3 \cdot 11 \cdot 17 \cdot 23 \cdot 31 \\
(1, 8, -2, 42) & -4 \cdot 3 \cdot 17 \cdot 23 \cdot 31 & (3, 4, 10, 14) & -4  \cdot 11 \cdot 23 \cdot 31 \\
(1,11,-12,-6) & 4 \cdot 3 \cdot 11 \cdot 17 \cdot 31 & (3, 7, 5, 7) & -4 \cdot 3 \cdot 23 \cdot 31 \\
(2,0,7,1) & -4 \cdot  23 \cdot 31  & (4, 17, 10, 28) & -4 \cdot 11 \cdot 17 \cdot 23 \cdot 31 \\ 
(2, 1, 14, -2) & -4 \cdot 11 \cdot 17 \cdot 31 & & \\
\end{array}
$$

\vskip1.2ex
In each case, we are thus led to solve the Thue-Mahler equation
\begin{equation} \label{lonely}
F(u,v) = 2^{3 \delta} 3^{\beta_1} 11^{\kappa_{11}} 17^{\kappa_{17}} 23^{\kappa_{23}} 31^{\kappa_{31}},
\end{equation}
where $\gcd(u,v) =1$, $\delta \in \{ 0, 1 \}$ and $\beta_1$, $\kappa_{11}$, $\kappa_{17}$, $\kappa_{23}$ and $\kappa_{31}$ are arbitrary nonnegative integers. Applying (\ref{term3}), in order to find a curve of conductor $399993$, we require additionally that, for a corresponding solution to (\ref{lonely}), 
\begin{equation} \label{mary}
F(u,v) \, D_F \equiv 0 \mod{ 3 \cdot 11 \cdot 17 \cdot 23 \cdot 31}.
\end{equation}
We readily check that the congruence $F(u,v) \equiv 0 \mod{p}$ has only the solution $u \equiv v \equiv 0 \mod{p}$ for the following forms $F$ and primes $p$ (whereby (\ref{mary}) cannot be satisfied by coprime integers $u$ and $v$ for these forms) :

$$
\begin{array}{cc|cc} 
(\omega_0,\omega_1,\omega_2,\omega_3) & p & (\omega_0,\omega_1,\omega_2,\omega_3)  & p \\ \hline
(1,1,1,3) & 11,23 & (2,0,7,1) & 3, 17 \\
(1,2,2,2) & 3, 23, 31 & (2, 5, 2, 6) & 11, 31 \\
(1, 4, -16, -2) & 3, 23 & (3,3,-8,-2) & 11  \\
(1, 8, -2, 42) & 11 & (4, 17, 10, 28) & 3 \\
(1,11,-12,-6) & 23 & & \\
\end{array}
$$

For the remaining $6$ forms under consideration, we appeal to UBC-TM. The only solutions we find satisfying (\ref{mary})
are as follows.
$$
\begin{array}{c|c} 
(\omega_0,\omega_1,\omega_2,\omega_3) & (u,v) \\ \hline
(1, 2, 2, 6) & (-1851, 892), (14133, -3790) \\
(2, 1, 14, -2) & (13, -5), (-29, -923) \\
(2, 4, -6, -3) & (10,-3), (64, 49), (-95, 199), (-3395, 1189), \\
& (3677, -1069), (5158, 4045), (-23546, 57259),  (-77755, 30999) \\
(3,3,44,66) & (1,0), (1,2), (-3,4), (3,-2), (-11,9), (25,-3), \\
& (231,2), (-317,240), (489,61), (1263, -878), (6853, -4119) \\
(3, 7, 5, 7) & (1, 12), (-29, 26), (78, 1),  (423, -160) \\
(3, 4, 10, 14) & (-41, 84), (95, -69), (307, 90) \\
\end{array}
$$ 
From these, we compute the conductors of $E_{\mathcal{D}}$ in (\ref{curvey}), where $\mathcal{D} \in \{ 1, 2 \}$, together with their twists by $-1$.
The only curves with conductor $399993$ arise from the form $F$ with $(\omega_0,\omega_1,\omega_2,\omega_3)=(2,4,-6,-3)$ and the solutions
$$
(u,v) \in \left\{ (10,-3), (5158,4045),  (-23546, 57259) \right\}.
$$
In each case, $\mathcal{D}=2$. The solution $(u,v)=(10,-3)$ corresponds to, in the notation of the LMFBD, curve 399993.j1, $(u,v)=(5158,4045)$ to 399993.i1, and $(u,v)=(-23546, 57259)$ to 399993.h1. Note that every form and solution we consider leads to elliptic curves with good reduction outside $\{ 2, 3, 11, 17, 23, 31 \}$, just not necessarily of  conductor $399993$. By way of example, if $(\omega_0,\omega_1,\omega_2,\omega_3) =(2, 4, -6, -3)$ and $(u,v)=(-77755, 30999)$, we find curves with minimal quadratic twists of conductor 
$$
2^5 \cdot 3 \cdot 11 \cdot 17 \cdot 23 \cdot 31 = 2^5 \cdot 399993.
$$

To determine the curves of conductor $399993$ with nontrivial rational $2$-torsion, we are led to solve the equation $X+Y=Z^2$ in $S$-units $X$ and $Y$, and integers $Z$, where $S=\{2,3,11,17,23,31 \}$.
We employ Magma code available at
\begin{center}
\url{http://nt.math.ubc.ca/BeGhRe/Code/UBC-TMCode}
\end{center}
to find precisely $2858$ solutions with $X \geq |Y|$ and  $\gcd (X,Y)$ squarefree (this computation took slightly less than $2$ hours). Of these, $1397$ have $Z > 0$, with $Z$ largest for the solution corresponding to the identity
$$
48539191572432 - 40649300451407 = 2^4 \cdot 3^4 \cdot 11 \cdot 23^7 - 17^5 \cdot 31^5 = 2808895^2.
$$
As in subsection \ref{ssec reducible}, we attach to each solution a pair of elliptic curves $E_1(X,Y)$ and $E_2(X,Y)$.
Of these, the only twists we find to have conductor $399993$ are the quadratic twists by $t$ of $E_i(X,Y)$  given in the following table. Note that there is some duplication -- the curve labelled 399993.f2 in the LMFDB, for example, arises from three distinct solutions to $X+Y=Z^2$.


$$
\begin{array}{c|c|c|c|c||c|c|c|c|c}
X & Y & E_i & t & \mbox{ LMFDB}  & X & Y & E_i & t & \mbox{ LMFDB} \\ \hline
16192 & -4743 & E_1 & -1 & 399993.g2 & 534336 & -506447 & E_2 & 2 & 399993.e1 \\
16192 & -4743 & E_2 & 2 & 399993.g1 & 1311552 & -527 & E_1 & 1 & 399993.a2 \\
23529 & 18496 & E_1 & -2 & 399993.f2  & 1311552 & -527 & E_2 & -2 & 399993.a1\\
23529 & 18496 & E_2 & 1 & 399993.f3 & 1414017 & -1045568 & E_1 & 2 & 399993.b2 \\
116281 & -75072 & E_1 & 2 & 399993.f4 & 1414017 & -1045568 & E_2 & -1 & 399993.b1 \\
116281 & -75072 & E_2 & -1 & 399993.f2 & 6305121 & 3027904 & E_1 & 2 & 399993.c1  \\
371008 & 4761 & E_1 & 1 & 399993.f2 & 6305121 & 3027904 & E_2 & -1 & 399993.c2 \\
371008 & 4761 & E_2 & -2 & 399993.f1& 6988113 & 18496 & E_1 & 2 & 399993.c2 \\
519777 & -131648 & E_1 & 2 & 399993.d2  & 6988113 & 18496 & E_2 & -1 & 399993.c3 \\
519777 & -131648 & E_2 & -1 & 399993.d1 & 7745089 & -2731968 & E_1 & 2 & 399993.c4\\
534336 & -506447 & E_1 & -1 & 399993.e2 & 7745089 & -2731968 & E_2 & -1 & 399993.c2 \\
\end{array}
$$

Full computational details are available at
\begin{center}
\url{http://www.nt.math.ubc.ca/BeGhRe/Examples/399993-data}.
\end{center}

%--------------------------------------------------------------------------------------------
\subsubsection{Conductor $10^6-1$}  \label{exe2}
%--------------------------------------------------------------------------------------------

We next treat a slightly larger conductor, which is not available in the LMFDB currently (but probably within computational range).
We have 
$$
10^6-1 = 3^3 \cdot 7 \cdot 11 \cdot 13 \cdot 37.
$$
From Theorem \ref{fisk}, we thus need to consider binary cubic forms $F(u,v) =  \omega_0 u^3 + \omega_1 u^2v + \omega_2 uv^2 + \omega_3 v^3$ of discriminant $D_F = \pm 108 N_1$, where $N_1 \mid 7 \cdot 11 \cdot 13 \cdot 37$ and $\omega_1 \equiv \omega_2 \equiv 0 \mod{3}$. The irreducible forms of this shape are as follows.
$$
\begin{array}{ccc|ccc} 
(\omega_0,\omega_1,\omega_2,\omega_3) & D_F & p & (\omega_0,\omega_1,\omega_2,\omega_3)  & D_F & p \\ \hline
(1,0,-6,-2) & 108 \cdot 7 & 37 &   (2,3,-78,-26) & 108 \cdot 7 \cdot 11 \cdot 13 \cdot 37 & \mbox{ none } \\
(1,0,21,16) & -108 \cdot 11 \cdot 37 & 7, 13 &  (2,3,6,3) & -108 \cdot 7 & 11, 37 \\
(1,0,30,2) & -108 \cdot 7 \cdot 11 \cdot 13 & \mbox{ none } &  
(2,3,6,8) & -108 \cdot 37 & 7 \\
(1, 3, 3, 3) & -108 & 7, 13, 37 &   (2, 6, -12, 1) & 108 \cdot 11 \cdot 13 & 7 \\
(1, 3, 6, 16) & -108 \cdot 37 & 7 &  (2,6,21,88) & -108 \cdot 11 \cdot 13 \cdot 37 &  \mbox{ none } \\
 %(2,10,-26,-13) & 108 \cdot 7 \cdot 13 \cdot 37 & 11  \\
(1,3,12,26) & -108 \cdot 7 \cdot 13 & \mbox{ none } &  (2,12,0,13) & -108 \cdot 7 \cdot 11 \cdot 13 &  \mbox{ none } \\
(1,3,33,117) & -108 \cdot 7 \cdot 11 \cdot 37 & \mbox{ none } & (2,21,-6,80) &  -108 \cdot 7 \cdot 11 \cdot 13 \cdot 37 &  \mbox{ none }  \\ 
(1,6,-36,-34) & 108 \cdot 7 \cdot 13 \cdot 37 & 11 & (3,3,18,20) &  -108 \cdot 7 \cdot 11 \cdot 13 &  \mbox{ none } \\
(1,6,3,6) & -108 \cdot 37 & 7 &  (4,6,15,14) & -108 \cdot 13 \cdot  37 & 11 \\
(1,6,9,26) & -108 \cdot 11 \cdot 13 & \mbox{ none } & (5,6,27,14) & -108 \cdot 7 \cdot 11 \cdot 37 &  \mbox{ none } \\
(1,9,0,74) & -108 \cdot 7 \cdot 13 \cdot 37 & \mbox{ none } & (5,9,3,21) & -108 \cdot 7 \cdot 11 \cdot 37 &  \mbox{ none } \\
(1,12,12,14) & -108 \cdot 13 \cdot 37 & 11 &  (7,0,12,14) &  -108 \cdot 7 \cdot 11 \cdot 37 &  \mbox{ none }  \\
%(6,32,-35,-7) & 108 \cdot 7 \cdot 11 \cdot 13 \cdot 37 & \mbox{ none } \\
(2, 0, -18, -5) & 108 \cdot 11 \cdot 37 & 13 & (10,3,42,-16) & -108 \cdot 7 \cdot 11 \cdot 13 \cdot 37 &  \mbox{ none }  \\
(2,0,3,3) & -108 \cdot 11 & 7, 37 & (10,6,12,3) & -108 \cdot 13 \cdot  37 &  \mbox{ none }  \\
(2,0,15,3) & -108 \cdot 7 \cdot 37 & 11, 13 & (11,6,12,6) & -108 \cdot 7 \cdot 11 \cdot 13  &  \mbox{ none } \\
(2,0,18,7) & -108 \cdot 13 \cdot 37 & 11 & (21,12,27,20) &  -108 \cdot 7 \cdot 11 \cdot 13 \cdot 37 &  \mbox{ none }  \\
\end{array}
$$
Here, we list primes $p$ for which a local obstruction exists modulo $p$ to finding coprime integers  $u$ and $v$ satisfying (\ref{term3}). 
It is worth noting at this point that the restriction to forms with $\omega_1 \equiv \omega_2 \equiv 0 \mod{3}$ that follows from the fact that we are considering a conductor divisible by $3^3$ is a helpful one. There certainly can and do exist irreducible forms $F$ with $108 \mid D_F$ that fail to satisfy $\omega_1 \equiv \omega_2 \equiv 0 \mod{3}$.

We are thus left to treat $17$ Thue-Mahler equations which we solve using UBC-TM; see 
\begin{center}
\url{http://www.nt.math.ubc.ca/BeGhRe/Examples/999999-data}
\end{center}
 for computational details. From (\ref{term3}), we require that
$D_F F(u,v) \equiv 0 \mod{7 \cdot 11 \cdot 13 \cdot 37}$; the only solutions we find satisfying this constraint are as follows.

$$
\begin{array}{c|c} 
(\omega_0,\omega_1,\omega_2,\omega_3) & (u,v) \\ \hline
(1, 0, 30, 2) & (-1,21), (1,16), (27,25) \\
(1,3,33,117) & (26,-7) \\
(1,9,0,74) &  (-19,2) \\
(2,3,-78,-26) & (-1,3), (-3,2), (-5,-1), (9,-1), (13,2), (-17,-58), (-39,-61), \\
&   (-57,-10), (-59,9), (65,-6), (79,-330), (159,-23) \\
(2,6,21,88) & (3,1), (165,-43) \\
(2,12,0,13) & (-1,9), (18,23) \\
(2,21,-6,80) & (1, -10), (2,1), (4, -3), (4,-1), (17, 1),\\
& 	(19, -5),(21, -2 ),(138, -11 ),(1356, -127) \\
(3,3,18,20) & (9,13), (97,-12) \\
(5,6,27,14) & (14,1),  (19,6),  (-21,44) \\
(5,9,3,21) & (-1,2), (6,1), (8,-3), (-649,284), (1077,-464)  \\
% (6,32,-35,-7)  & (0,-1), (2,-1), (2,-3),(16,3), (17,-3), (28,25), (95, -13), (859,-5009) \\
(7,0,12,14) & (-1,5), (-7,9), (301,-62), (-459,553)  \\
(10,3,42,-16) & (1,1), (1,2), (2,-1), (3,1), (4,-17), (20,19), (-22,-69),  (127,339)  \\
(10,6,12,3) & (2,-1), (5,-13), (-12,83), (-24,89), (81,-107), (125,-437) \\
(11,6,12,6) & (-1,22), (47,-72), (223,-429) \\
(21,12,27,20) & (1, -3 ),(1, 0 ),(1, 5 ),(4, -9),(4, 3 ),(9, -29 ),\\
& 	(19, -15 ),(29, -40 ),(316, -455 ),(551, -805)\\
\end{array}
$$ 


The only ones of these for which we find an $E_{\mathcal{D}}$ in (\ref{curvey})  of conductor $999999$ are as follows, where $E_{\mathcal{D}}$ is isomorphic over $\mathbb{Q}$ to a curve with model
$$
y^2 + a_1 xy + a_3 y = x^3 + a_2 x^2 + a_4 x + a_6.
$$

$$
\begin{array}{c|c|c|c|c|c|c|c} 
(\omega_0,\omega_1,\omega_2,\omega_3) & (u,v) & \mathcal{D} & a_1 & a_2 & a_3 & a_4 & a_6 \\ \hline
(1,0,30,2) & (27,25) & 6 & 0 & 0 & 1 & -40395 & 5402579  \\
(1,0,30,2) & (27,25) & -2 & 0 & 0 & 1 & -363555 & -145869640  \\
(5,6,27,14) & (14,1) & 1 & 1 & -1 & 0 & 14700 & 55223 \\
(5,6,27,14) & (14,1) & -3 & 1 & -1 & 1 & 1633 & -2590 \\
(5,9,3,21) & (-1,2) & 6 & 0 & 0 & 1 & 30 & 2254 \\
(5,9,3,21) & (-1,2) & -2 & 0 & 0 & 1 & 270 & -60865 \\
(10,6,12,3) & (125,-437) & 2 & 0 & 0 & 1 & -17205345 & -27554570341 \\
(10,6,12,3) & (125,-437) & -6 & 0 & 0 & 1 & -1911705 & 1020539642 \\
(21,12,27,20) & (4,3) & -1 & 1 & -1 & 0 & 12432 & -164125 \\
(21,12,27,20) & (4,3) & 3 & 1 & -1 & 1 & 1381 & 5618 \\
\end{array}
$$ 

Each of these listed curves has trivial rational $2$-torsion. To search for curves of conductor $999999$ with nontrivial rational $2$-torsion, we solve the equation $X+Y=Z^2$ in $S$-units $X$ and $Y$, and integers $Z$, where $S=\{2,3,7, 11,13, 37 \}$.
We find that there are precisely $4336$ solutions with $X \geq |Y|$ and  $\gcd (X,Y)$ squarefree. Of these, $2136$ have $Z > 0$, with $Z$ largest for the solution corresponding to the identity
$$
103934571636753 - 68209863326528 = 3^{15} \cdot 11 \cdot 13 \cdot 37^3 - 2^6 \cdot 7^{13} \cdot 11 =   5977015^2.
$$
Once again, we attach to each solution a pair of elliptic curves $E_1(X,Y)$ and $E_2(X,Y)$. We find $505270$ isomorphism classes of $E/\mathbb{Q}$ with good reduction outside of $\{ 2, 3, 7, 11, 13, 37 \}$ and nontrivial rational $2$-torsion. None of them have  conductor $999999$, whereby we conclude that there are precisely  $10$ isomorphism classes of elliptic curves over $\mathbb{Q}$ with conductor $10^6-1$. Checking that these curves each have distinct traces of Frobenius $a_{47}$ shows that they are nonisogenous.


%--------------------------------------------------------------------------------------------
\subsubsection{Conductor $10^9-1$}  \label{exe3}
%--------------------------------------------------------------------------------------------
 
This example is chosen to be somewhat beyond the current scope of the LMFDB.
We have
$$
10^9-1 = 3^4 \cdot 37 \cdot 333667
$$
and so, applying Theorem \ref{fisk}, we are led to consider binary cubic forms of discriminant $\pm 4 \cdot 3^4 \cdot 37^{\delta_1} \cdot 333667^{\delta_2}$, where $\delta_i \in \{ 0, 1 \}$. These include imprimitive forms with the property that each of its coefficients $\omega_i$ is divisible by $3$. For such forms, from Theorem \ref{fisk}, we necessarily have $\beta_1 \in \{ 0, 1 \}$ and hence $\beta_1=1$. Dividing through by $3$, we may thus 
restrict our attention to primitive forms of discriminant $\pm 4 \cdot 3^\kappa \cdot 37^{\delta_1} \cdot 333667^{\delta_2}$, where $\delta_i \in \{ 0, 1 \}$
 and $\kappa \in \{ 0, 4 \}$.
For the irreducible forms, we have, by slight abuse of notation (since, for the $F$ listed here with $D_F \not\equiv 0 \mod{3}$, the form whose existence is guaranteed by Theorem \ref{fisk} is actually $3F$), the following.
$$
\begin{array}{ccc|ccc} 
(\omega_0,\omega_1,\omega_2,\omega_3) & D_F & p & (\omega_0,\omega_1,\omega_2,\omega_3)  & D_F & p \\ \hline
(1,1,-3,-1) & 4 \cdot 37 & 333667 & (5,14,19,54) &  -4 \cdot 333667 & 37 \\
(1,4,52,250) & -4 \cdot 333667 & 37 & (6,18,168,323) & -4 \cdot 3^4 \cdot 333667 & 37 \\
(1,9,37,279) & -4 \cdot 333667 &   \mbox{ none } & (6,27,42,356) & -4 \cdot 3^4 \cdot 333667 & 37 \\
(1,21,117,2135) & -4 \cdot 3^4 \cdot 333667 & 37 & (6,54,-48,115) & -4 \cdot 3^4 \cdot 333667 & 37 \\
(2,0,3,1) & -4 \cdot 3^4 & 37 & (10,18,96,229) &  -4 \cdot 3^4 \cdot 333667 & 37 \\
(2,17,-26,-31) & 4 \cdot 333667 & 37 & (26,9,102,4) & -4 \cdot 3^4 \cdot 333667 & \mbox{ none } \\
(4,30,117,665) & -4 \cdot 3^4 \cdot 333667 & 37 & (27,7,70,32) & -4 \cdot 37 \cdot 333667 & \mbox{ none } \\
(4,35,14,216) & -4 \cdot 37 \cdot 333667 & \mbox{ none } & (31,9,87,-25) & -4 \cdot 3^4 \cdot 333667 & \mbox{ none } \\
(5,6,9,6) & -4 \cdot 3^4 \cdot 37 & \mbox{ none } & (49,51,63,55) & -4 \cdot 3^4 \cdot 333667  & \mbox{ none } \\ 
(5,7,19,51) & -4 \cdot 333667 & 37 & (52,55,72,37) & -4 \cdot 37 \cdot 333667 & \mbox{ none } \\
\end{array}
$$
Once again, we list primes $p$ for which a local obstruction exists modulo $p$ to finding coprime integers  $u$ and $v$ satisfying (\ref{term3}). There are thus $8$ Thue-Mahler equations left to solve.
In the (four) cases where  $D_F \not\equiv 0 \mod{3}$, these take the shape
$$
F(u,v) = 2^{3 \delta_1} \cdot 37^{\gamma_1} \cdot 333667^{\gamma_2},
$$
where $\delta_1 \in \{ 0, 1 \}$, $\gamma_1$ and $\gamma_2$ are nonnegative integers, and $u$ and $v$ are coprime integers. For the remaining $F$, the analogous equation is
$$
F(u,v) = 2^{3 \delta_1} \cdot 3^{\delta_2} \cdot 37^{\gamma_1} \cdot 333667^{\gamma_2},
$$
where $\delta_i \in \{ 0, 1 \}$, $\gamma_1, \gamma_2 \in \mathbb{Z}^+$ and $u, v \in \mathbb{Z}$ with $\gcd(u,v)=1$. We solve these equations using the UBC-TM Thue-Mahler solver. The only cases where we find  that
$$
D_F F(u,v) \equiv 0 \mod{37 \cdot 333667}
$$
occur for $(\omega_0,\omega_1,\omega_2,\omega_3)=(4,35,14,216)$ and $(u,v)=(-8,1)$ or $(u,v)=(-2,1)$, for
 $(\omega_0,\omega_1,\omega_2,\omega_3)=(27,7,70,32)$ and $(u,v)=(1,-2)$ or $(2,-1)$, and for $(\omega_0,\omega_1,\omega_2,\omega_3)=(52,55,72,37)$ and 
 $(u,v)=(0,1)$ or $(-3,5)$. In each case, all resulting  twists have bad reduction at $2$ (and hence cannot have conductor $10^9-1$).
 
 To search for curves with nontrivial rational $2$-torsion and conductor $10^9-1$, we solve the equation $X+Y=Z^2$ in $S$-units $X$ and $Y$, and integers $Z$, where 
 $S=\{2,3, 37, 333667 \}$. There are precisely $98$ solutions with $X \geq |Y|$ and  $\gcd (X,Y)$ squarefree. Of these, $41$ have $Z > 0$, with $Z$ largest for the solution coming from  the identity
$$
27027027 -101306 = 3^4 \cdot 333667 -  2 \cdot 37^3 =  5189^2.
$$
These correspond via twists to elliptic curves of conductor as large as $2^8 \cdot 3^2 \cdot 37^2 \cdot 333667^2$, but none of conductor $10^9-1$. There thus exist no curves $E/\mathbb{Q}$  of conductor $10^9-1$.

Full computational details are available at
\begin{center}
\url{http://www.nt.math.ubc.ca/BeGhRe/Examples/999999999-data}.
\end{center}



%----------------------------------------------------------------------------------------------------------------------
\subsection{Curves with good reduction outside $\{ 2, 3, 23 \}$ : an example of Koutsianis and of von Kanel and Matchke}
%----------------------------------------------------------------------------------------------------------------------

This case was worked out by Koutsianis \cite{Kou} (and also by von Kanel and Matschke \cite{KanMat}, who actually computed $E/\mathbb{Q}$ with good reduction outside $\{ 2, 3, p \}$ for all prime $p \leq 163$), by rather different methods from those employed here. We include it here to provide an example where we determine all $E/\mathbb{Q}$ with good reduction outside a specific set $S$, which is of somewhat manageable size in terms of the set of cubic forms encountered. Our data agrees with that  of \cite{KanMat} and \cite{Kou}.

To begin, we observe that the elliptic curves with good reduction outside $\{ 2, 3, 23 \}$ and $j$-invariant $0$ are precisely those with models of the shape
$$
E \; \; : \; \; Y^2 = X^3 \pm 2^a 3^b 23^c, \; \; \mbox{ where } \; 0 \leq a, b, c \leq 5.
$$
Appealing to (\ref{lumpy}), we next  look through our precomputed list to find  all the irreducible primitive cubic forms of discriminant $\pm 2^{\alpha} 3^\beta 23^\gamma$,
where 
$$
\alpha \in \{ 0, 2, 3, 4 \}, \; \; \beta \in \{ 0, 1, 3, 4, 5 \} \; \; \mbox{ and } \; \; \gamma \in \{ 0, 1, 2 \}. 
$$
The imprimitive forms we need consider correspond to primitive
forms $F$ with either $\nu_2 (D_F) = 0$ or $\nu_3 (D_F) \in \{ 0, 1 \}$. We find precisely $95$ irreducible, primitive cubic forms of the desired discriminants.
\newpage
$$
\begin{array}{cc|cc|cc} 
(\omega_0,\omega_1,\omega_2,\omega_3) & D_F & (\omega_0,\omega_1,\omega_2,\omega_3)  & D_F & (\omega_0,\omega_1,\omega_2,\omega_3)  & D_F \\ \hline
(1,0,-18,-6) & 2^2 \cdot 3^5 \cdot 23 & (2, 0, 3, 4) & -2^3 \cdot 3^5  & (4, 9, 24, 29) & -2^2 \cdot 3^4 \cdot 23^2 \\
(1,0,-3,-1) & 3^4 &   (2, 3, 6, 4) & -2^2 \cdot 3^5  & (4,12,12,27) & -2^4 \cdot 3^3 \cdot 23^2 \\
(1,0,3,2) & -2^3 \cdot 3^3 & (2, 3, 12, 8) & -2^4 \cdot 3^3 \cdot 23 & (4,12,12,73) & -2^4 \cdot 3^5 \cdot 23^2 \\
(1,0,6,2) & -2^2 \cdot 3^5 & (2, 3, 36, 29) & -2^3 \cdot 3^4 \cdot 23^2  & (4, 18, 9, 24) & -2^2 \cdot 3^5 \cdot 23^2  \\
(1,0,6,4) & -2^4 \cdot 3^4 & (2, 3, 36, 98) & -2^3 \cdot 3^5 \cdot 23^2  & (4, 18, 27, 48) & -2^2 \cdot 3^5 \cdot 23^2 \\
(1,0,9,6) & -2^4 \cdot 3^5 & (2, 5, 8, 15) & -2^3 \cdot 3 \cdot 23^2  & (5, 6, 7, 4) & -2^3 \cdot  23^2 \\
(1,0,33,32) & -2^2 \cdot 3^4 \cdot 23^2 & (2, 6, -12, -1) & 2^2 \cdot 3^5 \cdot 23 & (5, 6, 15, 8) & -2^3 \cdot 3^5 \cdot 23 \\
(1,1,2,1) & -23 & (2, 6, 6, 5) & -2^2 \cdot 3^5  & (5, 9, 12, 10) & -2^2 \cdot 3^5 \cdot 23 \\
(1,1,8,6) & -2^2 \cdot 23^2 & (2, 6, 6, 25) & -2^2 \cdot 3^3 \cdot 23^2  & (5, 12, 18, 20) & -2^4 \cdot 3^5 \cdot 23 \\
(1, 3, -9, -4) & 3^5 \cdot 23 & (2, 6, 27, 117) & -2^3 \cdot 3^5 \cdot 23^2 & (5, 18, 30, 46) & -2^2 \cdot 3^5 \cdot 23^2 \\
(1,3,-6,-4) & 2^2 \cdot 3^3 \cdot 23 & (2, 9, -6, -4) & 2^2 \cdot 3^5 \cdot 23 & (5,24,-3,26) & -2^4 \cdot 3^5 \cdot 23^2  \\
(1,3,-3,-2) & 3^3\cdot 23 & (2, 9, 0, -4) & 2^4 \cdot 3^3 \cdot 23 & (6, 3, 12, -7) & -2^3 \cdot 3^3 \cdot 23^2 \\
(1,3,-6,-2) & 2^3 \cdot 3^5 & (2,9,48,185) & -2^4 \cdot 3^5 \cdot 23^2 & (6, 3, 12, 16) & -2^4 \cdot 3^3 \cdot 23^2 \\
(1,3,3,3) & -2^2 \cdot 3^3 & (2, 12, 24, 85) & -2^2 \cdot 3^5 \cdot 23^2 & (6, 6, 9, 13) & -2^3 \cdot 3^3 \cdot 23^2 \\
(1,3,3,5) & -2^4 \cdot 3^3 & (2, 18, -15, 31) & -2^3 \cdot 3^5 \cdot 23^2 & (6, 9, 12, 23) & -2^3 \cdot 3^4 \cdot 23^2 \\
(1,3,3,7) & -2^2 \cdot 3^5 & (3, 0, 3, 2) & -2^4 \cdot 3^4 & (6, 18, 18, 29) & -2^2 \cdot 3^5 \cdot 23^2 \\
(1,3,3,13) & -2^4 \cdot 3^5 & (3, 4, 12, 12) & -2^4 \cdot 3 \cdot 23^2 & (7, 6, 9, 4) & -2^3 \cdot 3^4 \cdot 23 \\
(1, 3, 18, 50) & -2^3 \cdot 3^5 \cdot 23 & (3, 6, 4, 6) & -2^2 \cdot 3 \cdot 23^2  & (7, 15, 3, 17) & -2^2 \cdot 3^5 \cdot 23^2 \\
( 1, 6, -24, -4) & 2^4 \cdot 3^5 \cdot 23 & (3, 6, 9, 8) & -2^3 \cdot 3^3 \cdot 23 & (8, 9, 12, 13) & -2^2 \cdot 3^4 \cdot 23^2 \\
(1, 6, 3, 32) & -2^3 \cdot 3^5 \cdot 23 & (3, 9, 9, 7) & -2^4 \cdot 3^5 & (8, 15, 18, 21) & -2^3 \cdot 3^4 \cdot 23^2 \\
(1, 6, 6, 16) & -2^4 \cdot 3^3 \cdot 23 & (3, 9, 9, 49) & -2^2 \cdot 3^5 \cdot 23^2 & (9,9,3,31) & -2^4 \cdot 3^5 \cdot 23^2 \\
(1,6,12,54) & -2^2 \cdot 3^3 \cdot 23^2 & (3,18,36,116) & -2^4 \cdot 3^5 \cdot 23^2 & (10, 6, 15, 1) & -2^3 \cdot 3^3 \cdot 23^2 \\
(1,6,12,100) & -2^4 \cdot 3^3 \cdot 23^2 & (3,27,9,29) & -2^4 \cdot 3^5 \cdot 23^2 & (11, 6, 12, 2) & -2^2 \cdot 3^3 \cdot 23^2 \\
( 1, 9, -12, -16) & 2^4 \cdot 3^5 \cdot 23 & ( 4, 0, -18, -3) & 2^4 \cdot 3^5 \cdot 23 & (11, 15, 15, 17) & -2^2 \cdot 3^5 \cdot 23^2 \\
(1,9,-9,-3) & 2^2 \cdot 3^5 \cdot 23 & (4, 0, 6, 1) & -2^4 \cdot 3^5  & (12,9,36,16) & -2^4 \cdot 3^5 \cdot 23^2 \\
(1,9,27,165) & -2^2 \cdot 3^5 \cdot 23^2 & (4, 2, 8, 3) & -2^4  \cdot 23^2 & (12,36,36,35) & -2^4 \cdot 3^5 \cdot 23^2 \\
(1,9,27,303) & -2^4 \cdot 3^5 \cdot 23^2 & (4, 3, 6, 2) & -2^2 \cdot 3^3 \cdot 23 & (13, 9, 18, 12) & -2^2 \cdot 3^5 \cdot 23^2 \\
(1, 12, 9, 18) & -2^4 \cdot 3^5 \cdot 23 & (4, 3, 12, 10) & -2^3 \cdot 3^5 \cdot 23 & (13, 15, 27, 7) & -2^2 \cdot 3^5 \cdot 23^2 \\
(1,12,12,44) & -2^4 \cdot 3^3 \cdot 23^2 & (4, 3, 18, 13) & -2^3 \cdot 3^3 \cdot 23^2  & (21,9,27,11) & -2^4 \cdot 3^5 \cdot 23^2 \\
( 1, 15, 3, -7) & 2^4 \cdot 3^5 \cdot 23 & (4, 3, 18, 36) & -2^2 \cdot 3^5 \cdot 23^2 & (23,30,36,20) & -2^4 \cdot 3^5 \cdot 23^2  \\
(2, 0, 3, 1) & -2^2 \cdot 3^4 & (4, 4, 9, 1) & -2^4 \cdot 23^2 & (24,27,36,16) & -2^4 \cdot 3^5 \cdot 23^2 \\
(2, 0, 3, 2) & -2^3 \cdot 3^4 & (4, 6, 3, 12) & -2^2 \cdot 3^3 \cdot 23^2 & & \\
\end{array}
$$

In each case, we solve the corresponding Thue-Mahler equation specified by Theorem \ref{fisk}. For example, if $D_F = \pm 2^4 \cdot 3^t \cdot 23^2$, with $t \geq 3$, then we actually need only solve the (eight) Thue equations of the shape
$$
F(u,v) = 2^{\delta_1} 3^{\delta_2} 23^{\delta_3}, \; \; \mbox{ where } \; \; \delta_i \in \{ 0, 1 \}.
$$
For all other discriminants, we must treat ``genuine'' Thue-Mahler equations (where at least one of the exponents on the right-hand-side of equation (\ref{TM-eq}) is, {\it a priori}, unconstrained). Details of this computation are available at 
\begin{center}
\url{http://www.nt.math.ubc.ca/BeGhRe/Examples/2-3-23-data}.
\end{center}
In total, we found precisely $730$ solutions to these equations, leading, after twisting, to $3856$ isomorphism classes of $E/\mathbb{Q}$ with good reduction outside $\{ 2, 3, 23 \}$ and trivial rational $2$-torsion.

Once again, to find the curves with nontrivial rational $2$-torsion, we solved $X+Y=Z^2$ in $S$-units $X$ and $Y$, and integers $Z$, where 
 $S=\{2,3, 23 \}$. There are precisely $118$ solutions with $X \geq |Y|$ and  $\gcd (X,Y)$ squarefree (this computation took less than 1 hour). Of these, $55$ have $Z > 0$, with $Z$ largest for the solution coming from  the identity
$$
89424 - 23 = 2^4\cdot 3^5\cdot 23 - 23 = 299^2.
$$
These correspond via twists to elliptic curves of conductor as large as $2^8\cdot 3^2 \cdot 23^2$, a total of $1664$ isomorphism classes. There thus exist a total of $5520$ isomorphism classes (in $3968$ isogeny classes) of elliptic curves $E/\mathbb{Q}$ with good reduction outside $\{ 2, 3, 23 \}$. Note that $432 = 2 \times 6^3$ of these have $j_E=0$.

%MAYBE NOTE ANY OTHER TORSION! Theorem 5.2 of Hadano \cite{??}
%Manuscripta finds $43$ isomorphism classes of curves with nontrivial torsion (all types) and good reduction outside $\{ 3, 5 \}$.

%----------------------------------------------------------------------------------------------------------------------------------------------------
\subsection{Curves with good reduction outside $\{ 2, 3, 5, 7, 11 \}$ : an example of von Kanel and Matschke} \label{gummy-bear}
%----------------------------------------------------------------------------------------------------------------------------------------------------

This is the largest computation carried out along these lines by von Kanel and Matschke \cite{KanMat} (and also a very substantial computation via our approach, 
taking many thousand machine hours on $80$ cores).

As in the preceding example, note that
the curves with models of the shape
$$
E \; \; : \; \; Y^2 = X^3 \pm 2^a 3^b 5^c 7^d 11^e, \; \; 0 \leq a, b, c, d, e \leq 5
$$
are precisely the $E/\mathbb{Q}$  with good reduction outside $\{ 2, 3, 5, 7, 11 \}$ and $j$-invariant $0$. We next proceed  by searching our precomputed list for all irreducible primitive cubic forms of discriminant $2^{\alpha} 3^\beta M$,
where 
$$
\alpha \in \{ 0, 2, 3, 4 \}, \; \; \beta \in \{ 0, 1, 3, 4, 5 \} \; \; \mbox{ and } \; \; M \mid 5^2 \cdot 7^2 \cdot 11^2. 
$$
The imprimitive forms we need consider again correspond to primitive 
forms $F$ with either $\nu_2 (D_F) = 0$ or $\nu_3 (D_F) \in \{ 0, 1 \}$.
We encounter $1796$ irreducible cubic forms, which we tabulate at 
\begin{center}
\url{http://www.nt.math.ubc.ca/BeGhRe/Examples/2-3-5-7-11-data}
\end{center}
where details on the resulting Thue-Mahler computation may also be found. Confirming the results of von Kanel and Matschke \cite{KanMat}, we find that there exist a total of $592192$ isomorphism classes (in $453632$ isogeny classes) of elliptic curves $E/\mathbb{Q}$ with good reduction outside $\{ 2, 3, 5, 7, 11\}$, including $15552=2 \times 6^5$ with $j_E=0$.


%------------------------------------------------------------------------------------
\section{Good reduction outside a single prime} \label{primes}
%------------------------------------------------------------------------------------

For the remainder of this chapter, we will focus our attention on the case of elliptic curves with bad reduction at a single prime, i.e. curves of conductor $p$ or $p^2$, for $p$ prime.
In this case, our approach simplifies considerably and 
rather than being required to solve Thue-Mahler equations, the problem reduces to one of solving \emph{Thue} equations, i.e. equations of the shape $F(x,y)=m$, where $F$ is a form and $m$ is a fixed integer. While, once again, we do not have a detailed computational complexity analysis of either algorithms for solving Thue equations or more general algorithms for solving Thue-Mahler equations, computations to date strongly support the contention that the former is, usually, much, much faster than the latter, particularly if the set of primes $S$ considered for the Thue-Mahler equations is anything other than tiny. Since none of these conductors are divisible by $9$, we may always suppose that $j_E \neq 0$. We note that the data we have produced in these cases totals several terabytes. As a result, we have not yet determined how best to make it publicly available; interested readers should contact the authors for further details.

\subsection{Conductor $N=p$}
Suppose that $E$ is a curve with conductor $N=p$ prime with invariants $c_4$ and $c_6$. From Tables \ref{tab nu2}, \ref{tab nu3} and \ref{tab nup}, we 
necessarily have one of 
\begin{align*}
(\nu_2 (c_4), \nu_2(c_6)) &= (0,0) \mbox{ or } (\geq 4, 3), \mbox{ and } \nu_2 (\Delta_E) = 0,  \; \mbox{ or} \\
(\nu_3 (c_4), \nu_3 (c_6)) &= (0,0) \mbox{ or } (1, \geq 3), \mbox{ and } \nu_3 (\Delta_E) = 0, \; \mbox{ or} \\
(\nu_p(c_4), \nu_p(c_6)) &= (0, 0) \mbox{ and } \nu_p(\Delta_E) \geq 1.
\end{align*}
From this we see that $\mathcal{D}= 1$ or $2$. Theorem \ref{fisk} then implies that there is a cubic form of 
discriminant\index{discriminant} $\pm 4$ or $\pm 4p$, and integers $u, v$, with
$$
F(u,v) = p^{\kappa_p} \mbox{ or } 8 p^{\kappa_p}, \; \; 
c_4 = \mathcal{D}^2 H_F(u,v)  \; \; \mbox{ and } \; \; 
c_6 = - \frac{1}{2} \mathcal{D}^3 G_F (u,v),
$$
for $\mathcal{D} \in \{1,2\}$ and $\kappa_p$ a nonnegative integer. Note that, while the smallest absolute discriminant for an irreducible cubic form in $\mathbb{Z}[x,y]$ is $23$, there do exist reducible cubic forms of discriminants $4$ and $-4$ which we must consider.

Appealing to Th\'eor\`eme~2 of Mestre and Oesterl\'e \cite{MO} (and using \cite{BCDT}), we can actually restrict the choices for $n$ dramatically. In fact, we 
have 3 possibilities --  either  $p \in \{ 11, 17, 19, 37 \}$, or $p=t^2+64$ for some integer $t$, or, in all other cases, $\Delta_E = \pm p$. There are precisely $14$ isomorphism classes of $E/\mathbb{Q}$ with conductor in $\{ 11, 17, 19, 37 \}$; one may consult Cremona \cite{Cre1} for details. If we can write $p=t^2+64$ for an integer $t$ (which we may, without loss of generality, assume to satisfy $t \equiv 1\mod{4}$),  then the ($2$-isogenous) curves defined by 
$$
 y^2 + xy = x^3 + \frac{t-1}{4} \cdot x^2  - x
$$
and
$$
 y^2 + xy = x^3 + \frac{t-1}{4} \cdot x^2 + 4x + t
$$
have  rational points of order $2$ given by $(x,y)=(0,0)$ and $(x,y) = (-t/4,t/8)$, respectively, and discriminants $t^2+64$ and  $-(t^2+64)^2$, respectively. In the 
final case (in which $\Delta_E=\pm p$), we have (using the notation of Section~\ref{forms} and, in particular, appealing to (\ref{term0}) which, in this case yields the equation $1 = \nu_p(\Delta_E) =  \nu_p (D_F) + 2 \kappa_p$)
$$
\alpha_0 =2, \;  \alpha_1 \in \{ 0, 3 \}, \;  \beta_0=\beta_1=0, \;
\kappa_p = 0 \; \; \mbox{ and } \; \;  N_1  \in \{1,p\}.
$$
Theorem~\ref{fisk} thus tells us that to determine the elliptic curves of conductor $p$, we are led to 
to find all binary cubic forms (reducible and  irreducible) $F$ of discriminant $\pm 4$ and $\pm 4p$ and then solve the 
Thue equations
$$
F(x,y) =1 \; \; \mbox{ and } \; \; F(x,y)=8.
$$
Since for any solution $(x,y)$ to the equation $F(x,y)=1$, we have $F(2x,2y)=8$, we may thus restrict our attention to the equation  $F(x,y)=8$ (where we assume that $\gcd(x,y) \mid 2$). 

\subsection{Conductor $N=p^2$}

In case $E$ has conductor $N=p^2$, we have that either $E$ is a either a quadratic twist of a curve of conductor $p$, or we have $\nu_p (\Delta_E) \in \{ 2, 3, 4 \}$.
To see this, note that, via  Table~\ref{tab nup}, $p \mid c_4$, $p \mid c_6$ and 
$\mathcal{D} \mid 2p$, and we may suppose that 
$(\nu_p (c_4(E)),  \nu_p (c_6(E)), \nu_p (\Delta_E) )$ is one of
$$
(\geq 1, 1, 2),   (1, \geq 2, 3),   (\geq 2, 2,4),
(\geq 2, \geq 3, 6),  (2,3,\geq 7),  (\geq 3, 4, 8), 
(3, \geq 5, 9 ), (\geq 4, 5, 10).
$$
In each case with $\nu_p (c_6(E)) \geq 3$, denote by $E_1$ the quadratic twist of $E$ by $(-1)^{(p-1)/2} p$. For curves $E$ with 
$$
(\nu_p (c_4(E)),  \nu_p (c_6(E)), \nu_p (\Delta_E) )=(\geq 2, \geq 3, 6), 
$$
one may verify that $E_1$ has good 
reduction at $p$ and hence conductor $1$, a contradiction. If we have
$$
(\nu_p (c_4(E)),  \nu_p (c_6(E)), \nu_p (\Delta_E) )=(2, 3, \geq 7),
$$
then
$$
(\nu_p (c_4(E_1)),  \nu_p (c_6(E_1)), \nu_p (\Delta_{E_1}) ) = (0, 0, \nu_p (\Delta_E)-6)
$$
and so $E_1$ has conductor $p$. In the remaining cases, where
$$
(\nu_p (c_4(E)),  \nu_p (c_6(E)), \nu_p (\Delta_E) ) \in \{ (\geq 3, 4, 8), (3, \geq 5, 9 ), (\geq 4, 5, 10) \},
$$
we check that
$$
(\nu_p (c_4(E_1)),  \nu_p (c_6(E_1)), \nu_p (\Delta_{E_1}) ) \in \{ (\geq 1, 1, 2),   (1, \geq 2, 3),   (\geq 2, 2,4) \}.
$$

It follows that, in order to determine all isomorphism classes of $E/\mathbb{Q}$ of conductor $p^2$, it suffices to carry out the following program.
\begin{itemize}
\item Find all curves of conductor $p$.
\item Find $E/\mathbb{Q}$ with minimal discriminant\index{discriminant}
$\Delta_E \in \{ \pm p^2, \pm p^3, \pm p^4 \}$, and then
\item consider all appropriate quadratic twists of these curves.
\end{itemize}
The fact that we can essentially restrict attention to $E/\mathbb{Q}$ with minimal discriminant\index{discriminant}
\begin{equation} \label{submarine}
\Delta_E \in \{ \pm p^2, \pm p^3, \pm p^4 \}
\end{equation}
(once we have all curves of conductor $p$) was noted by Edixhoven, de Groot and Top in Lemma~1 of \cite{EGT}. To find the $E$ satisfying (\ref{submarine}), Theorem~\ref{fisk} (with specific appeal to (\ref{term0}))  leads us 
to consider Thue equations of the shape
\begin{equation} \label{fish1}
F(x,y) = 8 \; \mbox{ for  $F$ a form of discriminant }  \pm 4 p^2,
\end{equation}
\begin{equation} \label{fish2}
F(x,y) = 8p \; \mbox{ for  $F$ a form of discriminant }  \pm 4 p
\end{equation}
and
\begin{equation} \label{fish3}
F(x,y) = 8p \; \mbox{ for  $F$ a form of discriminant }  \pm 4 p^2,
\end{equation}
corresponding to $\Delta_E = \pm p^2$, $\pm p^3$ and $\pm p^4$, respectively.



%----------------------------------------------------
\subsection{Reducible forms}
%----------------------------------------------------

To find all elliptic curves $E/\mathbb{Q}$ with conductor $p$ or $p^2$ arising from reducible forms, 
via Theorem~\ref{fisk} we are led to solve equations
\begin{equation} \label{loaf}
 F(x,y)=8 p^n \; \; \mbox{ with } \; \;  n \in \mathbb{Z} \; \; \mbox{ and } \; \; \gcd(x,y) \mid 2,
\end{equation}
where $F$ is a reducible binary cubic form of discriminant\index{discriminant} $\pm 4$, $\pm 4p$ and $\pm 4 p^2$. 
This is an essentially elementary, though rather painful, exercise. Alternatively, we may observe that curves of 
conductor $p$ or $p^2$ arising from reducible cubic forms are exactly those with at least one rational $2$-torsion 
point. We can then use Theorem~I of Hadano \cite{Had} to show that the only such $p$ are $p=7, 17$ and $p=t^2+64$ for 
integer $t$.
In any case, after some rather tedious but straightforward work, we can show that the elliptic curves  of conductor $p$ 
or $p^2$ corresponding to reducible forms, are precisely those given in Table~\ref{tab red p curves} (up to quadratic 
twists by $\pm p$). 

\begin{table}
$$
% \begin{array}{|l|l||l|c|c|} \hline
%   c_4 & c_6 & p & \Delta_E & N_E \\ \hline
%   273 & 4455 & 17 & 17^2 & 17 \\
%  33 & 12015 & 17 & -17^4 & 17 \\
%  p-256 &  -t(p+512) & t^2+64 & -p^2 &  p \\
%  105 & 1323 & 7 & -7^3 & 7^2 \\
%  1785 & 75411 & 7 & 7^3 & 7^2 \\
%  33 & -81 & 17 & 17^3 & 17 \\
%  4353 & 287199 & 17 & 17 & 17 \\
%  p-16 & -t (p+8) & t^2+64 & p &  p \\ \hline
% \end{array}
\begin{array}{|c|c||c|c|c|} \hline
  c_4 & c_6 & p & \Delta_E & N_E \\ \hline
 4353 & 287199 & 17 & 17 & 17 \\
 33 & -81 & 17 & 17 & 17 \\
 t^2+48 & -t (t^2+72) & t^2+64 & t^2+64 &  t^2+64 \\ \hline
  273 & 4455 & 17 & 17^2 & 17 \\
 t^2-192 &  -t(t^2+576) & t^2+64 & -(t^2+64)^2 &  t^2+64 \\
\hline
 1785 & 75411 & 7 & 7^3 & 7^2 \\
 105 & 1323 & 7 & -7^3 & 7^2 \\
 \hline
 33 & 12015 & 17 & -17^4 & 17 \\
\hline
\end{array}
$$
\caption{All curves of conductor $p$ and $p^2$, for $p$ prime,  corresponding to reducible forms (i.e. with nontrivial rational $2$-torsion). Note 
that $t$ is any integer so that $t^2+64$ is prime. For the sake of brevity, we have omitted curves that are quadratic 
twists by $\pm p$ of curves of conductor $p$. }\label{tab red p 
curves}
\end{table}



%----------------------------------------------------
\subsection{Irreducible forms : conductor $p$}
%----------------------------------------------------
A quick search demonstrates that there are no irreducible cubic forms of discriminant $\pm 4$. Consequently if we wish 
to find  elliptic curves of conductor $p$ coming from irreducible cubics in Theorem \ref{fisk}, we need to solve equations of the shape 
$F(x,y)=8$ for all cubic forms of discriminant $\pm 4p$. An almost immediate consequence of this is the following.
\begin{proposition} \label{Setzer-prop}
Let $p > 17$ be prime. If there exists an elliptic curve $E/\mathbb{Q}$ of conductor $p$, then either $p=t^2+64$ for some integer $t$, or there exists an irreducible binary cubic form of discriminant $\pm 4p$.
\end{proposition}

On the other hand, if we denote by $h(K)$  the class number of a number field $K$, classical results of Hasse \cite{Has} imply the following.

\begin{proposition} \label{Setzer-prop2}
Let $p \equiv \pm 1 \mod{8}$ be prime and $\delta \in \{ 0, 1 \}$. If there exists an irreducible cubic form of discriminant $(-1)^\delta 4 p$, then 
$$
h \left( \mathbb{Q} (\sqrt{(-1)^\delta p}) \right) \equiv 0 \mod{3}.
$$
\end{proposition}

Combining Propositions \ref{Setzer-prop} and \ref{Setzer-prop2}, we thus have
\begin{corollary}[Theorem 1 of Setzer \cite{Set}] \label{CorSet}
Let $p \equiv \pm 1 \mod{8}$ be prime. If there exists an elliptic curve $E/\mathbb{Q}$ of conductor $p$, then either $p=t^2+64$ for some integer $t$, or we have
$$
h \left( \mathbb{Q} (\sqrt{p}) \right) \cdot h \left( \mathbb{Q} (\sqrt{-p}) \right) \equiv 0 \mod{3}.
$$
\end{corollary}

We remark that Proposition \ref{Setzer-prop} is actually a rather stronger criterion for guaranteeing the non-existence of elliptic curves of conductor $p$ than Corollary \ref{CorSet}. Indeed, by way of example, we may readily check that there are no irreducible cubic forms of discriminant $\pm 4p$ for 
$$
p \in \{ 23, 31, 199, 239, 257, 367, 439 \},
$$
(and hence no elliptic curves of conductor $p$ for these primes)
while, in each case, we have that
$ 3 \mid h \left( \mathbb{Q} (\sqrt{p}) \right) \cdot h \left( \mathbb{Q} (\sqrt{-p}) \right)$.

%----------------------------------------------------
\subsection{Irreducible forms : conductor $p^2$}
%---------------------------------------------------- 

As noted earlier, to determine all  elliptic curves of conductor $p^2$ for $p$ prime, arising via Theorem \ref{fisk} from irreducible cubics, it suffices to find those 
of conductor $p$  and those of conductor $p^2$ with $\Delta_F = \pm p^2, \pm p^3$ and $\pm p^4$ (and subsequently twist 
them). We explore these cases below.

%--------------------------------------------------------------------
\subsubsection{Elliptic curves of discriminant $\pm p^3$} \label{winky}
%--------------------------------------------------------------------
To find elliptic curves of discriminant $\pm p^3$, we need to solve Thue equations of the shape $F(x,y) = 8p$, where $F$ runs over all  cubic forms of discriminant $\Delta_F = 
\pm 4p$. These forms are already required to compute curves of conductor $p$. Now, we can either proceed directly to 
solve $F(x,y)=8p$ or transform the problem into one of solving a pair of new Thue equations of the shape $G_i(x,y)=8$. 
In practice, we used the former when solving rigorously and the latter when solving heuristically (see 	
Section~\ref{ssec heuristic}).

We now describe this transformation. Let $F(x,y)=ax^3+bx^2y+cxy^2+dy^3$ be a reduced form of discriminant $\pm 4p$. 
Since $p \mid \Delta_F$, we have
\begin{align*}
F(x,y) \equiv a (x-r_0 y)^2 (x-r_1y) \mod{p},
\end{align*}
where we must have that $p \nmid a$, since $F$ is a reduced form (which implies that $1 \leq a < p$). Comparing 
coefficients of $x$ shows that
$$
2 r_0 + r_1  \equiv -b/a \mod{p}, \; \; 
r_0^2 + 2 r_0 r_1 \equiv c/a \mod{p}
$$
and
$$
r_0^2 r_1 \equiv -d/a \mod{p}.
$$
Multiply the first two of these by $a$ and add them to get
\begin{align*}
3 a r_0^2 + 2 b r_0 + c  & \equiv 0 \mod{p}.
\end{align*}
We can solve this for $r_0$ and hence $r_1$:
\begin{align*}
(r_0,r_1) &\equiv (3a)^{-1} \left(-b \pm t, -b \mp 2t \right)  \mod{p},
\end{align*}
where we require that $t$ satisfies $t^2 \equiv b^2 - 3 ac \mod{p}$. Finding square roots modulo $p$ can be done efficiently via the Tonelli-Shanks 
algorithm, for example (see e.g. Shanks \cite{Shanks}), and almost trivially if, say, $p \equiv 3 \mod{4}$. Once we have these $(r_0,r_1$), we can 
readily check which pair satisfies $r_0^2 r_1 \equiv -d/a \mod{p}$.

Now if $F(x,y)=8p$ then we must have either
$$
 x \equiv r_0 y \mod{p} \; \; \mbox{ or } \; \; 
 x \equiv r_1 y \mod{p}.
$$
In either case, write $x=r_i y + p u$, which maps the equation $F(x,y)=8p$ to a pair of equations of the shape
$$
G_i(u,y) = 8,
$$
where
$$
G_i(u,y) = ap^2 u^3 + (3 a p r_i + bp) u^2 y + (3 a r_i^2 + 2 b r_i + c) u y^2 + \frac{1}{p} (a r_i^3 + b r_i^2 + c 
r_i + d) y^3.
$$
Notice that $\Delta_{G_i}= p^2 \Delta_F$. In practice, for our deterministic approach, we will actually solve the 
equation $F(x,y)=8p$ directly. For our heuristic approach (where a substantial increase in the size of the form's 
discriminant\index{discriminant} is not especially problematic), we will reduce to consideration of the equations 
$G_i(x,y)=8$.

\subsubsection{A (conjecturally infinite) family of forms and solutions}
We note that there are families of primes for which we can guarantee that the equation $F(x,y)=8p$ has solutions. By way of
example, define a quartic form $p_{r,s}$ via
\begin{align*}
p_{r,s}=r^4+9 r^2 s^2 + 27 s^4.
\end{align*}
Then for a given $r,s$ and $p=p_{r,s}$ the cubic form 
\begin{align*}
F(x,y) = s x^3 + r x^2 y - 3 s x y^2 -r y^3
\end{align*}
has discriminant $4 p$. Additionally one can readily verify the polynomial identities
$$
F(2r^2/s+6s,-2r) = 8p \; \; \mbox{ and } \; \;  F(6s,-18s^2/r-2r) = 8p.
$$
If we set $s \in \{ 1, 2 \}$ in the first of these, or $r \in \{ 1, 2 \}$ in the second, then we arrive at four one-parameter families 
of forms of discriminant $4p$ for which the equation $F(x,y)=8p$ has a solution, namely:
\begin{align*}
(p,x,y) &= (r^4+9r^2+27,2r^2+6,-2r), (r^4+36 r^2 + 432,r^2+12,-2r), \\
 &\phantom{=+}  (27 s^4+9 s^2+1,6s,-18s^2-2), (27s^4+36s^2+16,6s,-9s^2-4).
\end{align*}
Similarly, if we define 
\begin{align*}
p_{r,s}=r^4-9 r^2 s^2 + 27 s^4
\end{align*}
then the form
\begin{align*}
F(x,y) &= s x^3 + r x^2 y + 3 s x y^2 +r y^3
\end{align*}
has discriminant $-4 p$, and the equation $F(x,y)=8p$ has solutions
\begin{align*}
(x,y) &= (-2r^2/s + 6s, 2r)  \text{ and } (6s,-18s^2/r + 2r)
\end{align*}
and hence we again find  (one parameter) families  of primes corresponding to either $r\in \{ 1,2 \}$ or $s\in \{1,2 \}$:
\begin{align*}
(p,x,y)= & (r^4-9r^2+27,-2r^2+6,2r), (r^4-36 r^2 + 432,-r^2+12,2r), \\
 &\phantom{=+} (27 s^4-9 s^2+1,6s,-18s^2+2), (27s^4-36s^2+16,6s,-9s^2+4).
\end{align*}
We expect that each of the quartic families described here attains infinitely many prime values, but proving this is 
beyond current technology.

%------------------------------------------------------------------------------------
\subsubsection{Elliptic curves of discriminant $p^2$ and $p^4$} \label{pinky}
%------------------------------------------------------------------------------------
To find elliptic curves of discriminant $p^2$ and $p^4$ via Theorem~\ref{fisk}, we need to 
solve Thue equations $F(x,y)=8$ and $F(x,y)=8p$, respectively, for cubic forms $F$ of discriminant 
$4p^2$. Such forms are quite special and it turns out that they form a $2$-parameter family.

Indeed, in order for there to exist a cubic form of discriminant $4 p^2$, it is necessary and 
sufficient that we are able to write $p=r^2+27 s^2$ for positive integers $r$ and $s$, whereby $F$ 
is equivalent to the form
\begin{align*}
F_{r,s} (x,y) &= s x^3 + r x^2y - 9 s x y^2 - r y^3.
\end{align*}
To see this, note that the existence of an irreducible cubic form of discriminant $4p^2$ for prime $p$ necessarily implies that of a (cyclic) cubic field of discriminant $p^2$ and field index $2$. From Silvester, Spearman and Williams \cite{SSW}, there is a unique such field up to isomorphism, which exists precisely when the prime $p$ can be represented by the quadratic form $r^2+27 s^2$. We conclude as desired upon observing that
$$
D_{F_{r,s}} = 4 \left( r^2+27 s^2 \right)^2.
$$

It remains, then, to solve the Thue equations
$$
F_{r,s} (x,y) =8 \; \; \mbox{ and } \; \;   F_{r,s} (x,y)=8 p.
$$
We can transform the problem of solving the second of these equations to one of solving a related Thue 
equation of the form $G_{r,s}(x,y)=8$. This transformation is quite similar to the one described in 
the previous subsection. 

First note that we may assume that $p \nmid y$, since otherwise, we would require that $p \mid sx$, contradicting the facts that $s < \sqrt{p}$ and $p^2 \nmid F$. Since $p^2 \mid \Delta_F$, it follows that the congruence
\begin{align*}
  su^3 +ru^2 -9su - r &\equiv 0 \mod{p}
\end{align*}
has a unique solution modulo $p$; one may readily check that this satisfies $u \equiv 9s/r \mod{p}$:
\begin{align*}
  su^3 +ru^2 -9su - r &\equiv -r^{-3} \cdot (r^2-27s^2)(r^2+27s^2) \equiv 0 \mod{p}.
\end{align*}
Consequently, we know that $x \equiv uy \mod{p}$. Substituting $x = uy + vp$ into $F$ gives
\begin{align*}
F_{r,s}(uy+vp,y) &= a_0 v^3 + b_0 v^2 y + c_0 v y^2 + d_0 y^3
\end{align*}
so, with a quick renaming of variables, we obtain
\begin{align*}
G_{r,s}(x,y) &= a_0 x^3 + b_0 x^2 y + c_0 x y^2 + d_0 y^3 =8,
\end{align*} 
where
$$
a_0 = s p^2, \; \; b_0 = (3 u s+r) p, \; \; 
c_0 = 3 u^2 s+2 r u - 9s \;  \; \mbox{ and } \; \; d_0 = (u^3 s+r u^2 - 9 u s - r)/p.
$$
A little algebra confirms that 
\begin{align*}
\Delta_{G_{r,s}} = 4 p^4.
\end{align*}
As noted in the previous subsection, we have solved $F_{r,s}(x,y)=8p$ directly in our deterministic 
approach, while we solved equation $G_{r,s}(x,y)=8$ for our heuristic method.

%------------------------------------------------------------------------------------
\subsubsection{Elliptic curves of discriminant $-p^2$ and $-p^4$} \label{pinky2}
%------------------------------------------------------------------------------------
Elliptic curves of discriminant $-p^2$ and $-p^4$ can be found through Theorem~\ref{fisk} by solving 
the Thue equations $F(x,y)=8$ and $F(x,y)=8p$, respectively, this time  for cubic forms $F$ of 
discriminant $-4p^2$. As in the cases treated in the preceding subsection, these forms can be described as a $2$-parameter family.
Specifically, such forms arise precisely when there exist integers $r$ and $s$ such that
$p = | r^2 - 27 s^2|$, in which case the form $F$ is equivalent to 
 \begin{align*}
F_{r,s}(x,y)&=sx^3+r x^2y + 9s x y^2 + r y^3.
\end{align*}
The primes $p$ for which we can write  $p = | r^2 - 27 s^2|$ are those  with $p \equiv \pm 1 \mod{12}$.
To see this, note first that if $p \equiv 1 \mod{3}$ and $p = | r^2 - 27 s^2|$, then necessarily $p=r^2-27s^2$, so that $p \equiv 1 \mod{4}$, while, if $p \equiv -1 \mod{3}$ and
 $p = | r^2 - 27 s^2|$, then $p=27s^2-r^2$ so that $p \equiv -1 \mod{4}$. It follows that necessarily $p \equiv \pm 1 \mod{12}$. To show that this is sufficient to have $p =|r^2-27s^2|$ for integers $r$ and $s$, we appeal
 to the following.
\begin{proposition} \label{neg-pee2}
If $p \equiv 1 \mod{12}$ is prime, there exist positive integers $r$ and $s$ such that
$$
r^2-27 s^2 = p, \; \mbox{ with } \; r < \frac{3}{2} \sqrt{6p} \; \; \mbox{ and } \; \;  s < \frac{5}{18} \sqrt{6p}.
$$
If $p \equiv -1 \mod{12}$ is prime, there exist positive integers $r$ and $s$ such that
$$
r^2-27 s^2 = -p,  \; \mbox{ with } \; r < \frac{5}{2} \sqrt{2p}  \; \; \mbox{ and } \; \; s < \frac{1}{2} \sqrt{2p}.
$$
\end{proposition}
This result is, in fact, an almost  direct consequence of the following.
\begin{theorem}[Theorem~112 from Nagell \cite{Nag2}]\label{Nag}
If $p \equiv 1 \mod{12}$ is prime, there exist positive integers $u$ and $v$ such that
$$
p = u^2-3 v^2, \; \; u  < \sqrt{3p/2}  \; \; \mbox{ and } \; \; v < \sqrt{p/6}.
$$
If $p \equiv -1 \mod{12}$ is prime, there exist positive integers $u$ and $v$ such that
$$
-p = u^2-3v^2, \; \; u < \sqrt{p/2} \; \; \mbox{ and } \; \;  v < \sqrt{p/2}.
$$
\end{theorem}
\begin{proof}[Proof of Proposition \ref{neg-pee2}]
If $p \equiv \pm 1 \mod{12}$, Theorem  \ref{Nag} guarantees the existence of integers $u$ and $v$ such that $p = |u^2-3v^2|$.
If $3\mid v$ then we set $r=u, s=v/3$. 
Clearly $3\nmid u$, so if $3 \nmid v$ then we have (replacing $v$ by $-v$ is necessary) that $u \equiv v \mod{3}$. If we now 
set $r=2u+3v$ and $s=(2v+u)/3$, then it follows that
\begin{align*}
|r^2 - 27 v^2| &= |(2u+3v)^2 - 3 (2v+u)^2| = |u^2 - 3 v^2|  = p
\end{align*}
and hence either 
$$
|r| \leq 2 \sqrt{3p/2} + 3 \sqrt{p/6} = \frac{3}{2} \sqrt{6 p}  \; \; \mbox{ and } \; \; 
|s| \leq \frac{1}{3} (2 \sqrt{p/6}  + \sqrt{3p/2} ) = \frac{5}{18} \sqrt{6p},
$$
or
$$
|r| \leq 2 \sqrt{p/2} + 3 \sqrt{p/2} = \frac{5}{2} \sqrt{2 p} \; \; \mbox{ and } \; \; 
|s| \leq \frac{1}{3} (2 \sqrt{p/2}  + \sqrt{p/2} ) = \frac{1}{2} \sqrt{2 p}.
$$
\end{proof}

Again, we are able to reduce the problem of solving $F_{r,s}(x,y)=8p$ to that of treating a related 
equation $G_{r,s}(x,y)=8$. As before, note that if $u \equiv -9s/r 
\mod{p}$, then
\begin{align*}
s u^3 + r u^2 + 9 s u + r  \equiv r^{-3} (r^2-27 s^2) (r^2+27s^2) \equiv 0 \mod{p}.
\end{align*}
Again, write $x=r_0y+vp$ so that, after renaming $v$, we have
\begin{align*}
G_{r,s}(x,y) &= a_0 x^3 + b_0 x^2 y + c_0 x y^2 + d_0 y^3 =8,
\end{align*}
where
$$
a_0 = s p^2, \; \; 
b_0 = (3 u s+r) p, \; \;
c_0 = 3 u^2 s+2 r u + 9s \; \; \mbox{ and } \; \; 
d_0 = (u^3 s+r u^2 + 9 u s + r)/p.
$$

Note that, in contrast to the case where $p =r^2 + 27s^2$, here $p$ is represented by an indefinite quadratic form and so the presence of infinitely many units in $\mathbb{Q}(\sqrt{3})$ implies that a given representation is not unique. 
If, however, we have two solutions to the equation $|r^2-27s^2|=p$, say $(r_1,s_1)$ and $(r_2,s_2)$, then the corresponding forms
$$
s_1x^3+r_1 x^2y + 9s_1 x y^2 + r_1 y^3 \; \mbox{ and } \; s_2x^3+r_2 x^2y + 9s_2 x y^2 + r_2 y^3
$$
may be shown to be $\mbox{GL}_2 ( \mathbb{Z})$-equivalent.

% 
% As a final comment, we note that if we have two solutions to the equation $|r^2-27s^2|=p$, say $(r_1,s_1)$ and $(r_2,s_2)$, then the corresponding forms
% $$
% s_1x^3+r_1 x^2y + 9s_1 x y^2 + r_1 y^3 \; \mbox{ and } \; s_2x^3+r_2 x^2y + 9s_2 x y^2 + r_2 y^3
% $$
% are readily seen to be $\mbox{GL}_2 ( \mathbb{Z})$-equivalent.

%-----------------------------------------------------------------
 \section{Computational details} \label{compy}
%-----------------------------------------------------------------

As noted earlier, the computations required to generate curves of prime conductor  $p$  (and subsequently conductor $p^2$) fall into a small number of 
distinct parts. 

%--------------------------------------------------------
\subsection{Generating the required forms}
%--------------------------------------------------------

To find the irreducible forms  potentially corresponding to elliptic curves of prime conductor $p \leq X$ for some fixed 
positive real $X$, arguing as in Section \ref{rep}, we tabulated all reduced forms $F(x,y)=ax^3+bx^2y+cxy^2+d$ with 
discriminants in $(0,4X]$ and $[-4X,0)$, separately. As each form was generated, we checked to see if it actually 
satisfied the desired definition of reduction. Of course, this does not only produce forms with discriminant $\pm 4p$ -- 
as each form was produced, we kept only those whose discriminant was in the appropriate range, and equal 
to $\pm 4p$ for some prime $p$. Checking primality was done using the Miller-Rabin primality test (see \cite{Mil}, 
\cite{Rab}; to make this deterministic for the range we require, we appeal to \cite{SW}). While it is straightforward to 
code the above in computer algebra packages such as \texttt{sage} \cite{Sage}, \texttt{maple} \cite{Maple} or Magma \cite{magma}, we 
instead implemented it in \texttt{c++} for speed. To avoid possible numerical overflows, we used the \texttt{CLN} 
library \cite{Hai} for \texttt{c++}. 


We computed forms of discriminant $\pm 4p$ in two separate runs --- first to $p\leq 10^{12}$ and then a second run to $p 
\leq 2\times 10^{13}$. In the first of these, we constructed all the forms and explicitly saved them to files. 
Constructing all the required positive discriminant forms took approximately 40 days of CPU time on a modern server, and 
about 300 gigabytes of disc space. Thankfully, the computation is easily parallelised and it only took about 1 day of 
real time. We split the jobs by running a manager which distributed $a$-values to the other cores. The output from each 
$a$-value was stored as a tab-delimited text file with one tuple of $p,a,b,c,d$ on each line. Generating all forms of 
negative discriminant took about 3 times longer and required about 900 gigabytes of disc space. The distribution of 
forms is heavily weighted to small values of $a$. To allow us to spread the load across many CPUs we actually split the 
task into 2 parts. We first ran $a \geq 3$, with the master node distributing $a$-values to the other cores. We then ran 
$a=1$ and $2$ with the master node distributing $b$-values to the other cores. The total CPU time was about three times longer 
than for the positive case (there being essentially three times as many forms), but more real-time was required due to 
these complications. Thus generating all forms took less than 1 week of real time but required about 1.2 terabytes of 
disc space. 

These forms were then sorted by discriminant while keeping positive and negative discriminant forms separated. Sorting 
a terabyte of data is a non-trivial task, and in practice we did this by first sorting\footnote{Using the standard unix 
\texttt{sort} command and taking advantage of multiple cores.} the forms for each $a$-value and then splitting them into 
files of discriminants in the ranges $[n \times 10^9, (n+1)\times 10^9)$ for $n\in[0,999]$. Finally, all the files of 
each discriminant range were sorted together. This process for positive and negative discriminant forms took around two days of real 
time. We found $9247369050$ forms of positive discriminant $4p$  and  $27938060315$ of negative discriminant $-4p$, with $p$
bounded by $10^{12}$. Of these, $475831852$ and $828238359$, respectively had $F(x,y)=8$ solvable (by the 
heuristic method described below), leading to $159552514$ and $276339267$ elliptic curves of positive and negative 
discriminant, respectively, with prime conductor up to $10^{12}$.

The second run to $p \leq 2\times 10^{13}$ required a different workflow due to space constraints. Saving all 
forms to disc was simply impractical --- we estimated it to require over 20 terabytes of space! Because of this we 
combined the form-generation code with the heuristic solution method (see below) and kept only those forms $F(x,y)$ 
for which solutions to $F(x,y)=8$ existed. Since only a small fraction of forms (asymptotically likely $0$)  have 
solutions, the disc space required was considerably less. Indeed to store all the required forms took about 250 and 400 
gigabytes for positive and negative forms respectively. This then translated into about 65 and 115 gigabytes of 
positive and negative discriminant curves, respectively, with prime conductor up to $2\times 10^{13}$. This 
second computation took roughly 20 times longer than the first, requiring about 4 months of real-time. This led to 
a final count of $1738595275$ and $3011354026$ (isomorphism classes of) curves of positive and negative discriminant, respectively, with prime 
conductor up to $2\times 10^{13}$.


\subsection{Complete solution of Thue equations : conductor $p$}
For each form encountered, we needed to solve the Thue equation\index{Thue equations}
\begin{align*}
  ax^3+bx^2y+cxy^2+dy^3 &= 8
\end{align*}
in integers $x$ and $y$ with $\gcd(x,y) \in\{ 1, 2 \}$.
We approached this in two distinct ways.

To solve the Thue equation rigorously, we appealed to by now well-known arguments of Tzanakis and de Weger \cite{TW}, based upon lower bounds for linear forms in complex logarithms, together with lattice basis reduction; these are implemented in several computer algebra packages, including  Magma \cite{magma} and Pari/GP \cite{PARI2}. The main computational bottleneck in this approach is typically that of computing the fundamental units in the corresponding cubic fields; for computations $p$ of size up to $10^9$ or so, we encountered no difficulties with any of the Thue equations arising (in particular, the fundamental units occurring can be certified without reliance upon the Generalized Riemann Hypothesis).

We ran this computation in Magma \cite{magma}, using  its 
built-in Thue equation solver. Due to memory consumption issues, we fed the forms into 
Magma in small batches, restarting Magma after each set. We saved the 
output as a tuple 
$$
p,a,b,c,d,n,\{(x_1,y_1),\dots,(x_n,y_n)\}, 
$$
where $p,a,b,c,d$ came 
from the form, $n$ counts the number of solutions of the Thue equation and $(x_i,y_i)$ the 
solutions. These solutions can then be converted into corresponding elliptic curves in minimal form using Theorem \ref{fisk} and standard techniques. 

For positive discriminant, this approach works without issue for $p < 10^{10}$. For forms of negative discriminant $-4p$, however, the fundamental unit $\epsilon_p$ in the associated cubic field can be extremely large (i.e. $\log |\epsilon_p|$ can be roughly of size  $\sqrt{p}$). For this reason, finding all negative discriminant curves with prime conductor exceeding $2 \cdot 10^9$ or so proves to be extremely time-consuming.
Consequently, for large $p$, we turned to a non-exhaustive method, which, 
though it finds solutions to the Thue equation, is not actually guaranteed to find them all.

%----------------------------------------------------------------------------------------
\subsection{Non-exhaustive, heuristic solution of Thue equations}\label{ssec heuristic}
%----------------------------------------------------------------------------------------
If we wish to find all ``small'' solutions to a Thue equation (which, subject to various well-accepted conjectures, might actually prove to be all solutions), there is an obvious and very  computationally efficient approach we can take, based upon the idea that, given any solution to the equation $F(x,y)=m$ for fixed integer $m$, we necessarily either have that $x$ and $y$ are (very) small, relative to $m$, or that $x/y$ is a convergent in the infinite simple continued fraction\index{continued fractions} expansion to a root of the equation $F(x,1)=0$.

Such techniques were developed in detail by Peth\H{o} \cite{Pet1}, \cite{Pet2}; in particular, he provides a precise and computationally efficient distinction  between ``large'' and ``small'' solutions. Following this, for each form $F$ under consideration,
we expanded the roots of $F(x,1)=0$ to high precision, again using the 
\texttt{CLN} library for \texttt{c++}. We then computed the continued fraction expansion for each real root, 
along with its associated convergents. Each convergent $x/y$ was then substituted into $F(x,y)$ and  
checked to see if $F(x,y)=\pm 1, \pm 8$. Replacing $(x,y)$ by one of $(-x,-y), (2x,2y)$ or $(-2x,-2y)$, if necessary, then provided the required solutions of $F(x,y)=8$. The precision was chosen so that we could 
compute convergents $x/y$ with $|x|,|y| \leq 2^{128} \approx 3.4\times10^{38}$. We then looked for solutions of small height using a 
brute force search over a relatively small range of values. 

To ``solve'' $F(x,y)=8$ by this method,  for all forms with discriminant $\pm 4p$ with $p\leq 10^{12}$, took about 1 
week of real time using 80 cores. The resulting solutions files (in which we stored also  forms with no corresponding 
solutions) required about 1.5 terabytes of disc space. Again, the files were split into files of absolute discriminant 
(or more precisely absolute discriminant divided by 4) in the ranges $[n \times 10^9, (n+1)\times 10^9)$ for 
$n\in[0,999]$. For the second computation run to $p \leq 2\times 10^{13}$, we combined the form-generation and 
heuristic-solutions steps, storing only forms which had solutions. This produced about $235$ and $405$ gigabytes of 
data for positive and negative discriminants, respectively. 

%---------------------------------------------
\subsection{Conversion to curves}
%---------------------------------------------
Once one has a tuple $(a,b,c,d,x,y)$, one then computes $G_F(x,y)$ and $H_F(x,y)$, appeals to Theorem \ref{fisk} and 
checks twists. This leaves us with a list of pairs $(c_4,c_6)$ corresponding to elliptic curves. It is now 
straightforward to derive $a_1,a_2,a_3,a_4$ and $a_6$ for a corresponding elliptic curve in minimal form (see e.g. 
Cremona \cite{Cre2}).  For each curve, we saved a tuple $p,a_1,a_2,a_3,a_4,a_6,\pm1$ with the last entry being the sign 
of the discriminant of the form used to generate the curve (which coincides with the sign of the discriminant of the 
curve). We then merged the curves with positive and negative discriminants and added the curves with prime conductor 
arising from reducible forms (i.e. of small conductor or for primes of the form $t^2+64$). After sorting by conductor, 
this formed a single file of about $17$ gigabytes for all curves with prime conductor $p < 10^{12}$ and about $180$ 
gigabytes for all curves with conductor $p<2\times10^{13}$.


%------------------------------------------------
\subsection{Conductor $p^2$}
%-----------------------------------------------
The conductor $p^2$ computation was quite similar, but was split further into parts. 
\subsubsection{Twisting conductor $p$}
The vast majority of curves of conductor $p^2$ that we encountered arose as quadratic twists of curves of conductor 
$p$. To compute these, we took all curves with conductor $p \leq 10^{10}$ and calculated the invariants 
$c_4$ and $c_6$. The twisted curve then has corresponding $c$-invariants
$$
  c_4' = p^2 c_4  \; \; \mbox{ and } \; \;  c_6' = (-1)^{(p-1)/2}  p^3 c_6.
$$
The minimal $a$-invariants were then computed as for curves of 
conductor $p$. 

We wrote a simple \texttt{c++} program to read curves of conductor $p$ and then twist 
them, recompute the $a$-invariants and output them as a tuple 
$p^2,a_1,a_2,a_3,a_4,a_6,\pm1$. The resulting code only took a few minutes to process the 
approximately $1.1\times10^7$ curves.

\subsubsection{Solving $F(x,y)=8p$ with $F$ of discriminant $\pm 4p$}
There was no need to retabulate forms for this computation; we reused the positive and negative forms of discriminant 
$\pm 4p$ with $p \leq 10^{10}$ from the conductor-$p$ computations. We subsequently rigorously solved the corresponding 
equations $F(x,y)=8p$ for $p \leq 10^8$. To solve the Thue equation $F(x,y)=8p$ for $10^8 < p \leq 10^{10}$, using the 
non-exhaustive, heuristic method, we first converted the equation to a pair of new Thue equations of the form 
$G_i(u,y)=8$ as described in Section \ref{winky} and then applied  Peth\H{o}'s solution search method (where we searched for solutions to these new equations with 
$|y|$ bounded by $2^{128}$ and $|u|=|(x-r_iy)/p|$ bounded in such way as to guarantee that our original $|x|$  is also bounded by $2^{128}$). 

The solutions were then processed into curves as for the conductor $p$ case 
above, and the resulting curves were twisted by $\pm p$ in order to obtain more 
curves of conductor $p^2$. 

\subsubsection{Solving $F(x,y) \in \{ 8, 8p \}$ with $F$ of  discriminant $\pm 4p^2$}
To find forms of discriminant $4 p^2$ with $p\leq 10^{10}$ we need only check to see which primes are of the form $p = r^2+27s^2$ in the desired range. To 
do so, we simply looped over $r$ and $s$ values and then again checked primality\index{primality testing} using 
Miller-Rabin. As each prime was found, the corresponding $p,r,s$ tuple was converted to a 
form  as in Section \ref{pinky}, and the Thue equations $F(x,y)=8$ and $F(x,y)=8p$ were solved,  using the rigorous
approach for $p < 10^6$ and  the 
non-exhaustive method described previously  for $10^6 < p \leq 10^{10}$. Again, in the latter situation, the equation $F(x,y)=8p$ was converted to a 
new equation $G(x,y)=8$ as described in Section \ref{pinky}. The process for forms of discriminant $-4p^2$ was very similar, 
excepting that more care is required with the range of $r$ and $s$ (appealing to Proposition \ref{neg-pee2}). The non-exhaustive 
method solving both $F(x,y)=8$ and $F(x,y)=8p$ for positive and negative forms took a total of 
approximately 5 days of real time on a smaller server of 20 cores. The rigorous approach, even restricted to prime $p < 10^6$ was much, much slower.

The solutions were then converted to curves as with the previous cases and each resulting 
curve was twisted by $\pm p$ to find other curves of conductor $p^2$. 

 %------------------------------------------
\section{Data} \label{data}
%-----------------------------------------

%------------------------------------------
\subsection{Previous work}
%------------------------------------------
The principal prior work on computing table of elliptic curves of prime conductor was carried out in two lengthy computations, by Brumer and McGuinness \cite{BrMc} in the late 1980s and by Stein and Watkins \cite{StWa} slightly more than ten years later. 
For the first of these computations, the authors  fixed the $a_1, a_2$ and $a_3$ invariants ($12$ possibilities) and looped over $a_4$ and $a_6$ chosen to make the corresponding discriminant small.
By this approach, they were able to find $311243$ curves of prime conductor $p < 10^8$ (representing approximately $99.6 \%$ of such curves). In the latter case, the authors looped instead over $c_4$ and $c_6$, subject to (necessary) local conditions. They obtained a large collection of elliptic curves of general conductor to $10^8$, and $11378912$ of those with prime conductor to $10^{10}$ (which we estimate to be slightly in excess of $99.8 \%$ of such curves).

%--------------------------------------------------
\subsection{Counts : conductor $p$}
%--------------------------------------------------

By way of comparison, we found the following numbers of isomorphism classes of
elliptic curves\index{elliptic curves}  over $\mathbb{Q}$ with prime conductor $p \leq X$:
 $$
 \begin{array}{|c|c|c|c|c|c|c|} \hline
 X    & \Delta_E > 0  &   \Delta_E < 0 & \mbox{Ratio}^2 &  \mbox{Total}  & \mbox{Expected} & \mbox{Total / Expected} \\ 
\hline
10^3 &  33 &  51 &  2.3884 &  84 &  68 & 1.2353 \\
10^4 &  129 &  228 &  3.1239 &  357 &  321 & 1.1122 \\
10^5 &  624 &  1116 &  3.1986 &  1740 &  1669 & 1.0425 \\
10^6 &  3388 &  5912 &  3.0450 &  9300 &  9223 & 1.0084 \\
10^7 &  19605 &  34006 &  3.0087 &  53611 &  52916 & 1.0131\\
10^8 &  114452 &  198041 &  2.9941 &  312493 &  311587 & 1.0029\\
10^9 &  685278 &  1187686 &  3.0038 &  1872964 &  1869757 & 1.0017 \\
2\times10^9 &  1178204 &  2040736 &  3.0001 &  3218940 &  3216245 & 1.0008\\
\hline
10^{10} &  4171055 &  7226982 &  3.0021 &  11398037 &  11383665 & 1.0013\\
10^{11} &  25661634 &  44466339 &  3.0026 &  70127973 &  70107401 & 1.0003\\
10^{12} &  159552514 &  276341397 &  2.9997 &  435893911 &  435810488 & 1.0002\\
10^{13} &  999385394 &  1731017588 &  3.0001 &  2730402982 &  2730189484 & 1.00008 \\
2\times10^{13} &  1738595275 &  3011354026 &  3.0000 &  4749949301 &  4749609116 & 1.00007 \\
\hline
\end{array}
$$


The data above the line is rigorous; for positive discriminant, we actually have a rigorous 
result to $10^{10}$. For the positive forms this took about one week of real time using 80 cores. 
Unfortunately, the negative discriminant forms took significantly longer, roughly 2 months of real time using 80 cores. 
Heuristics given by Brumer and McGuinness \cite{BrMc} suggest that the number of elliptic curves of negative 
discriminant of absolute discriminant up to $X$ should be asymptotically $\sqrt{3}$ times as many as those of positive 
discriminant in the same range -- here we report the square of this ratio in the given ranges. The aforementioned 
heuristic count of Brumer and McGuinness suggests that the 
expected number of $E$ with prime $N_E \leq X$ 
should be 
  $$
  \frac{\sqrt{3}}{12} \, \left( \int_{1}^{\infty} \frac{1}{\sqrt{u^3-1}} du + \int_{-1}^{\infty} \frac{1}{\sqrt{u^3+1}} du  \right) \mbox{ Li} (X^{5/6}),
  $$
  which we list (after rounding) in the table above. It should not be surprising that this ``expected'' number of curves appears to slightly undercount the actual number, since it does not take into account the roughly $\sqrt{X}/\log X$ curves of conductor $p=n^2+64$ and discriminant $-p^2$ (counting only curves of discriminant $\pm p$).

\subsection{Counts : conductor $p^2$}
To compile the final list of curves of conductor $p^2$, we combined the five lists of 
curves: twists of curves of conductor $p$, curves from forms of discriminant $+4p$ and 
$-4p$, and curves from discriminant $+4p^2$ and $-4p^2$. The list was then sorted and any 
duplicates removed. The resulting list is approximately one gigabyte in size. The counts of curves 
are as follows; here we list numbers of isomorphism classes of curves of conductor $p^2$ for $p$ prime with $p \leq X$.
\vskip2ex

\begin{center}
 \begin{tabular}{|c|c|c|c|c|}
\hline
$X$ & $\Delta_E>0$ &$\Delta_E<0$ & Total &Ratio$^2$ \\
\hline
$10^3$  &  53  &  93  &  146  &  3.0790 \\
$10^4$  &  191  &  322  &  513  &  2.8421 \\
$10^5$  &  764  &  1304  &  2068  &  2.9132 \\
$10^6$  &  3764  &  6356  &  10120  &  2.8515 \\
$10^7$  &  20539  &  35096  &  55635  &  2.9198 \\
$10^8$  &  116894  &  200799  &  317693  &  2.9508 \\
$10^9$  &  691806  &  1195262  &  1887068  &  2.9851\\
$10^{10}$  &  4189445  &  7247980  &  11437425  &  2.9931\\

\hline
 \end{tabular}
\end{center}

\vskip2ex
Subsequently we decided that we should recompute the discriminants of these curves as a sanity check,  by reading the curves into \texttt{sage} and using its built-in 
elliptic curve \index{elliptic curves} routines to compute and then factor the discriminant. This took about one day 
on a single core.

The only curves of genuine interest are those that do not arise from twisting, i.e. those of discriminant $\pm p^2$, $\pm p^3$ and $\pm p^4$. In the last of these categories, we found only $5$ curves, of conductors $11^2$, $43^2$, $431^2$, $433^2$ and $33013^2$. The first four of these were noted by 
Edixhoven, de Groot and Top \cite{EGT} (and are of small enough conductor to now appear in Cremona's tables). The fifth, satisfying
$$
(a_1,a_2,a_3,a_4,a_6)=(1, -1, 1, -1294206576, 17920963598714),
$$
has discriminant $33013^4$. For discriminants $\pm p^2$ and $\pm p^3$, we found the following numbers of curves, for conductors $p^2$ with $p \leq X$ :
\vskip2ex

\begin{center}
 \begin{tabular}{|c|c|c|c|c|}
\hline
$X$ & $\Delta_E=-p^2$ & $\Delta_E=p^2$ & $\Delta_E=-p^3$ & $\Delta_E=p^3$  \\
\hline
$10^3$ & 12 & 4 & 7 & 4 \\
$10^4$ & 36 & 24 & 9 & 5 \\
$10^5$ & 80 & 58 & 12 & 9 \\
$10^6$ & 203 & 170 & 17 & 15 \\
$10^7$ & 519 & 441 & 24 & 23 \\
$10^8$ & 1345 & 1182 & 32 & 36 \\
$10^9$ & 3738 & 3203 & 48 & 58 \\
$10^{10}$ & 10437 & 9106 & 60 & 86 \\
\hline
 \end{tabular}
\end{center}

\vskip2ex
It is perhaps worth observing that the majority of these curves arise from, in the case of discriminant $\pm p^2$, forms with, in the notation of Sections \ref{pinky} and \ref{pinky2}, either $r$ or $s$ in $\{1, 8 \}$. Similarly, for $\Delta_E=\pm p^3$, most of the curves we found come from forms in the eight one-parameter families described in Section \ref{winky}. We are unaware of a heuristic predicting the number of curves of conductor $p^2$ up to $X$ that do not arise from twisting curves of conductor $p$.

%----------------------------------------
\subsection{Thue equations}
%----------------------------------------
 
  
It is noteworthy that all solutions we encountered to the Thue equations\index{Thue equations} $F(x,y)=8$ and $F(x,y)=8p$ under consideration satisfied $|x|, |y| < 2^{30}$. The ``largest'' such solution corresponded to the equation
$$
355 x^3 + 293 x^2 y -1310 x y^2-292 y^3=8,
$$
where we have
$$
(x,y)=(188455233,-82526573).
$$
This leads to the elliptic curve of conductor
$948762329069$,
$$
E \; \; : \; \; y^2+xy+y=x^2-2x^2+a_4x+a_6,
$$
with
$$
a_4=-1197791024934480813341
$$
and 
$$
a_6=15955840837175565243579564368641.
$$
Note that this curve does not actually correspond to a particularly impressive $abc$ or Hall conjecture (see  Section \ref{spartan} for the definition of this term) example.

In the following table, we collect data on the number of $\mbox{GL}_2 ( \mathbb{Z})$-equivalence classes of irreducible binary cubic forms of discriminant $4p$ or $-4p$ for $p$ in $[0,X]$, denoted $P_3(0,X)$ and $P_3(-X,0)$, respectively. We also provide counts for those forms where the corresponding equation $F(x,y)=8$ has at least one integer solution, denoted $P_3^*(0,X)$ and 
$P_3^*(-X,0)$ for positive and negative discriminant forms, respectively.

 $$
 \begin{array}{|c|c|c|c|c|} \hline
 X    & P_3(0,X)  & P_3^*(0,X)    &  P_3(-X,0) &  P_3^*(-X,0) \\ \hline
10^3   & 23  & 22  &   78 & 61    \\
10^4 & 204  &  163  &  740  & 453 \\
10^5 &  1851   &  1159 & 6104  & 2641 \\
10^6 & 16333  & 7668 &  53202  &  16079 \\
10^7 &  147653  & 49866 &  466601  & 97074  \\
10^8 & 1330934 & 314722 & 4126541 & 582792 \\
10^9 &  12050910 & 1966105 & 36979557 & 3530820 \\
2 \times 10^9 &  23418535 & 3408656 & 71676647  & 6080245 \\  \hline 
10^{10} &  109730653 & 12229663 & 334260481 & 21576585 \\    
10^{11} & 1004607003 & 76122366 & 3045402451 & 133115651 \\
10^{12} &  9247369050 & 475831852  & 27938060315  &   828238359 \\ \hline
\end{array}
$$
Due to space limitations we did not compute these statistics in the second large computational run.


\vskip2ex
Our expectation is that the number of forms for which the equation $F(x,y)=8$ has solutions with absolute discriminant up to $X$ is  $o(X)$ (i.e. this occurs for essentially ``zero'' percent of forms; a first step in proving something is this direction can be found in recent work of Akhtari and Bhargava \cite{AkBh}).


%----------------------------------------------------------------------------
\subsection{Elliptic curves with the same prime conductor}
%----------------------------------------------------------------------------

One might ask how many isomorphism classes of curves of a given prime conductor can occur. If one accepts recent 
heuristics that predict that the Mordell-Weil rank of $E/\mathbb{Q}$ is absolutely bounded (see e.g. \cite{PPVW} and \cite{Watetal}), then this number should also be so bounded. As noted by Brumer and Silverman \cite{BrSi}, 
there are $13$ curves of conductor $61263451$. Up to $p < 10^{12}$, the largest number we encountered was for $ p=530956036043$,
with $20$ isogeny classes, corresponding to $(a_1,a_2,a_3,a_4,a_6)$ as follows :
 $$
 \begin{array}{l}
 \left( 0, -1, 1, -1003, 37465 \right), \left( 0, -1, 1, -1775, 45957 \right), \left( 0, -1, 1, -38939, 2970729 \right),  \\
 \left( 0, -1, 1, -659, -35439 \right), \left(0, -1, 1, 2011, 4311\right), \left(0, -2, 1, -27597, -1746656\right), \\
\left( 0, -2, 1, 57, 35020\right),  \left(1, -1, 0, -13337473,18751485796\right),\left(0, 0, 1, -13921, 633170\right), \\
\left( 0, 0, 1, -30292,-2029574\right),\left(0, 0, 1, -6721, -214958\right), \left(0, 0, 1, -845710, -299350726\right), \\
\left( 0, 0, 1, -86411851, 309177638530\right),  \left(0,0, 1, -10717, 428466\right), \left(1, -1, 0, -5632177, 5146137924\right), \\
 \left( 1, -1, 0, 878,33379\right),  \left(1, -1, 1, 1080, 32014\right), \left(1, -2, 1, -8117, -278943\right), \\
\left(1, -3, 0, -2879, 71732\right), \left(1, -3, 0, -30415, -2014316\right). \\
\end{array}
$$
% Of these $20$ curves, $2$ have rank $3$, $3$ have rank $2$, $9$ have rank $1$ and $6$ have rank $0$.
  All have discriminant $-p$.
 %The class group of $\mathbb{Q}(\sqrt{3 \cdot 530956036043})$ is isomorphic to
%$$
%\mathbb{Z}/3 \mathbb{Z} \oplus \mathbb{Z}/3 \mathbb{Z} \oplus \mathbb{Z}/3 \mathbb{Z},
%$$
%which, via a classical result of Hasse \cite{Has}, explains the existence of a large number of cubic forms of discriminant $-4p$.
 Elkies \cite{Elk1} found examples of rather larger conductor with more curves, including
 $21$ classes for $p=14425386253757$ and discriminant $p$, and $24$ classes for $p=998820191314747$ and discriminant $-p$. Our computations confirm, with high likelihood, that, for $p < 2 \times 10^{13}$, the number of isomorphism classes of elliptic curves of conductor a fixed prime $p$ is at most $21$.
 
 
 %-----------------------------------------------------
 \subsection{Rank and discriminant records}
 %-----------------------------------------------------
 
In the following table, we list the smallest prime conductor with a given Mordell-Weil rank\index{Mordell-Weil rank}. These were computed by running through our data, using Rubinstein's upper bounds for analytic ranks (as implemented in Sage) to search for candidate curves of ``large'' rank which were then checked using mwrank \cite{mwrank}.\index{rank records}
The last entry corresponds to a curve of rank $6$ with minimal positive prime  discriminant; we have not yet ruled out the existence of a rank $6$ curve with smaller absolute (negative) discriminant.

 $$
 \begin{array}{|c|c|c|c|} \hline
 N    & (a_1,a_2,a_3,a_4,a_6)  &  \mbox{sign}(\Delta_E) &  rk(E(\mathbb{Q})   \\ \hline
37  &   (0, 0, 1, -1, 0) & + & 1 \\
389 & (0, 1, 1, -2, 0) & + & 2 \\
5077 & (0, 0, 1, -7, 6) & + & 3  \\
501029 & (0, 1, 1, -72, 210)& + & 4  \\
19047851 & ( 0, 0, 1, -79, 342) & - & 5  \\
6756532597 & (0, 0, 1, -547, -2934) & + & 6 \\  \hline
\end{array}
$$
 \vskip1.3ex
 It is perhaps noteworthy that the curve listed here of rank $6$ has the smallest known minimal discriminant for such a curve (see Table 4 of Elkies and Watkins \cite{ElWa}).
 
If we are interested in similar records over all curves, including composite conductors, we have

$$
 \begin{array}{|c|c|c|c|} \hline
 N    & (a_1,a_2,a_3,a_4,a_6)  &  \mbox{sign}(\Delta_E) &  rk(E(\mathbb{Q})   \\ \hline
37  &   (0, 0, 1, -1, 0) & + & 1 \\
389 & (0, 1, 1, -2, 0] & + & 2 \\
5077 & (0, 0, 1, -7, 6) & + & 3  \\
234446 & (1, -1, 0, -79, 289) & + & 4  \\ \hline
19047851 & (0, 0, 1, -79, 342) & - & 5  \\
5187563742 & (1, 1, 0, -2582, 48720) & + & 6 \\  
382623908456 & (0,0,0,-10012,346900) & + & 7 \\ \hline
\end{array}
$$

\vskip2ex
Here, the curves listed above the line are proven to be those of smallest conductor with the given rank. Those listed below the line have the smallest known conductor for the corresponding rank. It is our belief that the techniques of this chapter should enable one to determine whether the  curve listed here of rank $5$ has the smallest conductor of any curve with this property.


%---------------------------------------------------
\section{Completeness of our data} \label{spartan}
%--------------------------------------------------

As a final result, we will present something that might, optimistically, be viewed as evidence that our ``heuristic'' approach, in practice, enables us to actually find all elliptic curves of prime conductor $p < 2 \times 10^{13}$.

A conjecture of Hall, admittedly one that without modification is widely disbelieved at present,  is that if $x$ and $y$ are integers for which $x^3-y^2$ is nonzero, then the {\it Hall ratio} 
$$
\mathcal{H}_{x,y}=\frac{|x|^{1/2}}{|x^3-y^2|}
$$
is absolutely bounded. The pair $(x,y)$ corresponding to the largest known Hall ratio comes from the identity
$$
5853886516781223^3 - 447884928428402042307918^2 = 1641843,
$$
noted by Elkies \cite{Elk2}, with $\mathcal{H}_{x,y} > 46.6$. All other examples known currently have $\mathcal{H}_{x,y} < 7$. We prove the following.

\begin{proposition} \label{total}
 If there is an elliptic curve $E$ with  conductor $p < 2 \times 10^{13}$, corresponding via Theorem \ref{fisk} to a cubic form $F$ and  $u, v \in \mathbb{Z}$, such that 
 $$
 F(u,v)=8 \; \; \mbox{ and  } \; \; \max \{ |u|, |v| \} \geq 2^{128},
 $$
 then
 \begin{equation} \label{trauma}
 \mathcal{H}_{c_4(E),c_6(E)} > 1.5 \times 10^6.
 \end{equation}
\end{proposition}

In other words, if there is an elliptic curve $E$  with  conductor $p < 2 \times 10^{13}$ that we have missed in our heuristic search, then we necessarily have inequality (\ref{trauma}) (and hence a record-setting Hall ratio).

\begin{proof}
The main idea behind our proof is that the roots of the Hessian $H_F(x,1)$ have no particularly good reason to be close to those of the polynomial $F(x,1)$. It follows that, if we have relatively large integers $u$ and $v$ satisfying the Thue equation $F(u,v)=8$ (so that $u/v$ is close to a root of $F(x,1)=0$), our expectation is that not only does $H_F(u,v)$ fail to be small, but, in fact, we should have inequalities of the order of
$$
H_F(u,v) \gg \left( \max\{|u|,|v|\} \right)^2 \; \; \mbox{ and } \; \; G_F(u,v) \gg \left( \max\{|u|,|v|\} \right)^3
$$
(where the Vinogradov symbol hides a possible dependence on $p$).
Since
$$
c_4(E) = \mathcal{D}^2 H_{F} (u,v) \mbox{ and }  c_6(E) = - \frac{1}{2}  \mathcal{D}^3 G_{F} (u,v),
$$
where $\mathcal{D} \in \{ 1, 2 \}$, these would imply that
$$
\mathcal{H}_{c_4(E),c_6(E)} \gg_p \frac{1}{p}  \max\{|u|,|v|\}.
$$
In fact, for forms (and curves) of positive discriminant, we can deduce inequalities of the shape
$$
\mathcal{H}_{c_4(E),c_6(E)} \gg_p p^{-3/4}  \min\{|u|,|v|\} \gg p^{-5/4}  \max\{|u|,|v|\},
$$
where the implicit constants are absolute. For curves of negative discriminant, we have a slightly weaker result :
$$
\mathcal{H}_{c_4(E),c_6(E)} \gg_p p^{-1}  \min\{|u|,|v|\} \gg p^{-3/2}  \max\{|u|,|v|\}.
$$

To make this argument precise, let us write, for concision, $c_4=c_4(E)$ and $c_6=c_6(E)$.  From the identity $|c_4^3-c_6^2|=1728p$, we have a Hall ratio 
$$
\mathcal{H}_{c_4,c_6} = \frac{|c_4|^{1/2}}{1728p} > \frac{|c_4|^{1/2}}{3.456 \times 10^{16}} \geq \frac{|H_F(u,v)|^{1/2}}{3.456 \times 10^{16}}.
$$
Our goal will thus be to obtain a lower bound upon $|H_{F} (u,v)|$ --
we claim that, in fact, $|H_{F} (u,v)| > 3 \times 10^{45}$, whereby this Hall ratio exceeds $1.5 \times 10^6$, as stated. Suppose that we have a cubic form $F$ and integers $u$ and $v$ with
 $D_F = \pm 4 p$ for $p$ prime,
\begin{equation} \label{size}
\max \{ |u|, |v| \} \geq 2^{128} \; \; \mbox{ and } \; \; 2 \times 10^{9} < p < 2 \times 10^{13}.
\end{equation}
Notice that $F(u,0) = \omega_0 u^3 = 8$ and hence (\ref{size}) implies that $v \neq 0$.

Write
$$
F(u,v) = \omega_0 (u -\alpha_1 v) (u-\alpha_2v) (u-\alpha_3v)
$$
and suppose that 
$$
|u - \alpha_1 v| = \min \{ |u-\alpha_iv|, \; i = 1, 2, 3 \}.
$$
We may further assume, without loss of generality, that the form $F$ is reduced. From (\ref{resultant}), we have 
\begin{equation} \label{rezzie}
\omega_0^2 \, \left|  H_F (\alpha_1,1) \, H_F (\alpha_2,1) \, H_F (\alpha_3,1) \right|=  16 \, p^2.
\end{equation}
For future use, we note that  the main result of Mahler \cite{Mah0} implies the inequality
\begin{equation} \label{Mahler}
|\omega_0| \prod_{i=1}^{3} \max \{ 1, |\alpha_i| \} \leq |\omega_0|+|\omega_1|+|\omega_2|+|\omega_3|.
\end{equation}

Let us assume first  that $D_F > 0$, whereby $H_F$ has negative discriminant ($D_{H_F}=-3 D_F$).  Since $F$ is reduced, we have
$$
| \omega_1 \omega_2 -9 \omega_0 \omega_3| \leq \omega_1^2-3\omega_0 \omega_2 \leq \omega_2^2 - 3 \omega_1 \omega_3,
$$
and hence the identity
\begin{equation} \label{bleep}
(\omega_1 \omega_2 -9 \omega_0 \omega_3)^2 - 4 (\omega_1^2-3\omega_0 \omega_2) (\omega_2^2 - 3 \omega_1 \omega_3) = - 3 D_F
\end{equation}
yields the inequalities
\begin{equation} \label{bleep2}
D_F \geq  (\omega_1^2-3\omega_0 \omega_2) (\omega_2^2 - 3 \omega_1 \omega_3) \geq (\omega_1^2-3\omega_0 \omega_2)^2.
\end{equation}

Since (\ref{bleep}) and $D_F > 0$ imply that $\omega_1^2-3\omega_0 \omega_2 \neq 0$,
we may write
$$
\frac{H_F (\alpha_1,1)}{ \omega_1^2-3\omega_0 \omega_2} =  \left(\alpha_1 -  \frac{9 \omega_0 \omega_3 - \omega_1 \omega_2 + \sqrt{-3 D_F}}{2 (\omega_1^2-3\omega_0 \omega_2)} \right) \left(\alpha_1 -  \frac{9 \omega_0 \omega_3 - \omega_1 \omega_2 - \sqrt{-3 D_F}}{2 (\omega_1^2-3\omega_0 \omega_2)} \right).
$$
Defining
$$
\Gamma_1 = \alpha_1 -  \frac{9 \omega_0 \omega_3 - \omega_1 \omega_2 }{2 (\omega_1^2-3\omega_0 \omega_2)} \; \; \mbox{ and } \; \; 
\Gamma_2 = \frac{\sqrt{3 D_F}}{2 (\omega_1^2-3\omega_0 \omega_2)},
$$
we have
$$
H_F (\alpha_1,1) = \left( \omega_1^2-3\omega_0 \omega_2 \right) \left( \Gamma_1^2+\Gamma_2^2 \right)
$$
and so
\begin{equation} \label{pos-case}
|H_F(\alpha_1,1)| > \frac{3 D_F}{4 (\omega_1^2-3\omega_0 \omega_2)}.
\end{equation}

Since $\alpha_1$ is ``close'' to $u/v$, it follows that the same is true for $H_F(\alpha_1,1)$ and $H_F(u/v,1) = v^{-2} H_F(u,v)$. To make this precise, note that, 
via the Mean Value Theorem,
\begin{equation} \label{general}
\left| H_F (\alpha_1,1) - H_F (u/v,1) \right| = \left| 2 ( \omega_1^2-3\omega_0 \omega_2) y + \omega_1 \omega_2 -9 \omega_0 \omega_3 \right|
\left| \alpha_1 - \frac{u}{v} \right|,
\end{equation}
for some $y$ lying between $\alpha_1$ and $u/v$. We thus have
\begin{equation} \label{wobbly}
\left| H_F (\alpha_1,1) - H_F (u/v,1) \right| \leq  ( \omega_1^2-3\omega_0 \omega_2) \left( 2 \left( |\alpha_1| + \left| \alpha_1 - \frac{u}{v} \right| \right) +1 \right) \left| \alpha_1 - \frac{u}{v} \right|.
\end{equation}

To derive an upper bound upon $\left| \alpha_1 - \frac{u}{v} \right|$, we can argue as in the proof of Theorem 2 of Peth\H{o} \cite{Pet2} to obtain the inequality
\begin{equation} \label{tombstone}
\left| \alpha_1 - \frac{u}{v} \right| \leq 2^{7/3} D_F^{-1/6} v^{-2}.
\end{equation}
Since $|v| \geq 1$ and $D_F =4p > 8 \times 10^9$, we thus have that 
\begin{equation} \label{rince}
\left| \alpha_1 - \frac{u}{v} \right| < 0.12.
\end{equation}

We may suppose that $F$ is reduced, whereby, crudely, from Lemma \ref{fisher},
$$
|\omega_0| \leq \frac{2D_F^{1/4}}{3 \sqrt{3}} \; \mbox{ and } \; |\omega_1| \leq  \frac{3\omega_0}{2}+  \left( \sqrt{D_F}-\frac{27\omega_0^2}{4} \right)^{1/2}
< \left( 1 + \frac{1}{\sqrt{3}} \right) D_F^{1/4}.
$$
From Proposition 5.5 of Belabas and Cohen \cite{BeCo}, 
$$
|\omega_2| \leq \left( \frac{35 + 13 \sqrt{13}}{216} \right)^{1/3} D_F^{1/3} \; \mbox{ and } \;
|\omega_3| \leq \frac{4}{27} D_F^{1/2},
$$
whence, after a little computation, we find that
$$
|\omega_0|+|\omega_1|+|\omega_2|+|\omega_3| < D_F^{1/2} = 2 p^{1/2}.
$$
From (\ref{Mahler}), it follows that
$$
|\alpha_1| \leq |\omega_0|+|\omega_1|+|\omega_2|+|\omega_3| < 2 p^{1/2},
$$
whereby inequalities (\ref{rince}) and (\ref{size}) thus yield 
$$
|u/v| < 2 p^{1/2} + 0.12 < 2^{23.1},
$$
and so, again appealing to (\ref{size}), $\min \{ |u|, |v| \} > 2^{104}$. Returning to inequality (\ref{wobbly}), we find that, after applying (\ref{bleep2}),
$$
\left| H_F (\alpha_1,1) - H_F (u/v,1) \right| \leq 2 p^{1/2}  \left( 4 p^{1/2} + 1.24 \right) 2^{7/3} (2 p)^{-1/6} v^{-2}.
$$
From $p < 2 \times 10^{13}$ and $|v| > 2^{104}$, it follows that
$$
\left| H_F (\alpha_1,1) - H_F (u/v,1) \right| < 10^{-50}.
$$
Combining this with (\ref{bleep2}) and (\ref{pos-case}) yields the inequality
$$
\left| H_F(u/v,1) \right| >  \frac{2 p}{|\omega_1^2-3\omega_0 \omega_2|},
$$
whence
$$
 |H_F(u,v)| = v^2 \left| H_F(u/v,1) \right| > \frac{2 v^2 p}{|\omega_1^2-3\omega_0 \omega_2|} \geq v^2 \sqrt{p},
$$
where the last inequality follows from  (\ref{bleep2}).
From  (\ref{size}) and the fact that $|v| > 2^{104}$, we conclude that
$$
 |H_F(u,v)|> 10^{67}.
$$

Next, suppose that $F$ has negative discriminant, so that $H_F$ has positive discriminant $D_{H_F}=-3 D_F$.  
If $\omega_1^2-3\omega_0 \omega_2=0$, then, from (\ref{bleep}), we have that
$$
3 p = - (\omega_1^2-3\omega_0 \omega_2) (\omega_2^2 - 3 \omega_1 \omega_3),
$$
which implies that
$$
\max \left\{ |\omega_1^2-3\omega_0 \omega_2|, |\omega_2^2 - 3 \omega_1 \omega_3| \right\} \geq p.
$$
On the other hand, from Lemma 6.4 of Belabas and Cohen \cite{BeCo}, we have
\begin{equation} \label{iota}
\arraycolsep=1.8pt\def\arraystretch{1.8}
\begin{array}{cc}
|\omega_0| \leq \frac{2^{3/2} p^{1/4}}{3^{3/4}}, \; 
|\omega_1| \leq \frac{2^{3/2} p^{1/4}}{3^{1/4}}, \; 
\max \{ |\omega_0 \omega_2^3|, |\omega_1^3 \omega_3| \} \leq \frac{(11+5 \sqrt{5}) p}{2}, \\
|\omega_1 \omega_2| \leq \frac{8 p^{1/2}}{3^{1/2}}
 \; \mbox{ and } \;
|\omega_0 \omega_3| \leq \frac{2 p^{1/2}}{3^{1/2}}, \\
\end{array}
\end{equation}
whereby a short calculation, together with the fact that $p > 2 \times 10^9$, yields a contradiction.
We may thus suppose that $\omega_1^2-3\omega_0 \omega_2 \neq 0$. We have
$$
H_F(\alpha_i,1) =  (\omega_1^2-3\omega_0 \omega_2)  \left(\alpha_i - \beta_1 \right) \left( \alpha_i - \beta_2 \right),
$$
where
$$
\beta_j = \frac{9 \omega_0 \omega_3 - \omega_1 \omega_2 + (-1)^j \sqrt{12 p}}{2 (\omega_1^2-3\omega_0 \omega_2)} \; \; \mbox{ for } \; \; j \in \{ 1, 2 \}.
$$
It follows that
$$
|\beta_j| \leq |\omega_1^2-3\omega_0 \omega_2|^{-1} 44 \cdot 3^{-1/2} p^{1/2}
$$
and, again from (\ref{Mahler}),
$$
|\omega_0 \alpha_i| \leq |\omega_0|+|\omega_1|+|\omega_2|+|\omega_3|,
$$
whereby
$$
|\omega_0 \alpha_i| \leq \frac{2^{3/2} p^{1/4}}{3^{3/4}}+\frac{2^{3/2} p^{1/4}}{3^{1/4}}+\frac{2^{2/3} \left(11+5 \sqrt{5} \right)^{1/3} p^{1/2}}{3^{1/2} |\omega_0|}+\frac{2 p^{1/2}}{3^{1/2}|\omega_0|},
$$
whence we find that
$$
|\alpha_i| \leq \frac{3.4 \, p^{1/4}}{|\omega_0|} + \frac{2.1 \, p^{1/2}}{|\omega_0|^2} < \frac{6.4 \, p^{1/2}}{|\omega_0|^2}.
$$
From (\ref{rezzie}), we thus have 
$$
\left|  H_F (\alpha_1,1)  \right| \geq  \omega_0^{-2} (\omega_1^2-3\omega_0 \omega_2)^{-2} \min \left\{ \frac{\omega_0^2}{3.2}, \frac{|\omega_1^2-3\omega_0 \omega_2|}{12.8} \right\}^{2}.
$$

If $|\omega_1^2-3\omega_0 \omega_2| > 4 \omega_0^2$, it follows that
$$
\left|  H_F (\alpha_1,1)  \right| \geq \frac{\omega_0^2}{10.24 \, (\omega_1^2-3\omega_0 \omega_2)^{2}}
$$
and so
$$
\left|  H_F (\alpha_1,1)  \right| \geq \frac{1}{10.24 \, (2^3 3^{-1/2} p^{1/2}+2^{2/3} 3^{1/2} \left(11+5 \sqrt{5} \right)^{1/3} p^{1/2})^{2}}
$$
which implies that
\begin{equation} \label{foggie}
\left|  H_F (\alpha_1,1)  \right| > \frac{1}{1561 \, p}.
\end{equation}
If, conversely, $|\omega_1^2-3\omega_0 \omega_2| \leq 4 \omega_0^2$, then
$$
\left|  H_F (\alpha_1,1)  \right| \geq \frac{1}{163.84 \, \omega_0^2} > \frac{1}{253 \sqrt{p}}
$$
and hence (\ref{foggie}) holds in either case.

Now if $\alpha_1 \not\in \mathbb{R}$, then, via Mahler \cite{Mah},
$$
\left| \mbox{Im} (\alpha_1) \right|  \geq \frac{1}{18} \left( |\omega_0|+|\omega_1|+|\omega_2|+|\omega_3| \right)^{-2} > \frac{\omega_0^2}{738 \, p},
$$
so that
$$
\left| \alpha_1 - \frac{u}{v} \right|> \frac{\omega_0^2}{738 \, p}
$$
and hence
$$
8 = |\omega_0| |v|^3 \left| \alpha_1 - \frac{u}{v} \right| \left| \alpha_2 - \frac{u}{v} \right| \left| \alpha_3 - \frac{u}{v} \right| > |\omega_0| |v|^3 \left( \frac{\omega_0^2}{738 \, p} \right)^3.
$$
It follows that
$$
|v| < 1476 p < 2.952 \times 10^{16},
$$
via (\ref{size}). Since $\max \{ |u|, |v| \} > 2^{128}$, we thus have
$$
\left| u/v \right| > 1.15 \times 10^{22}.
$$
From
$$
|\alpha_1| < 6.4 p^{1/2} < 6.4 \left( 2 \times 10^{13} \right)^{1/2} <3 \times 10^7, 
$$
we may thus conclude that
$$
\left| \alpha_1 - \frac{u}{v} \right|> 1.14 \times 10^{22}
$$
and so
$$
8 \geq \left( 1.14 \times 10^{22} \right)^3,
$$
an immediate contradiction.

We may thus suppose that $\alpha_1 \in \mathbb{R}$ (so that $\alpha_2, \alpha_3 \not\in \mathbb{R}$).  It follows from Mahler \cite{Mah} that
$$
\left| \alpha_i - \frac{u}{v} \right|> \frac{\omega_0^2}{738 \, p}, \; \; \mbox{ for } \; \; i \in \{ 2, 3 \},
$$
and so
\begin{equation} \label{frenchy}
\left| \alpha_1 - \frac{u}{v} \right| < \frac{8}{|\omega_0| |v|^3} \left( \frac{738 p }{\omega_0^2} \right)^2.
\end{equation}
Appealing to (\ref{size}) and the inequalities $|\alpha_1| < 3 \times 10^7$ and $|v| \geq 1$, we thus have that
$$
|u/v| <1.75 \times 10^{33} + 3 \times 10^7 < 1.76 \times 10^{33},
$$
and so, from $\max \{ |u|, |v| \} > 2^{128}$, $|v| > 1.9 \times 10^5$. Inequality (\ref{frenchy}) thus now implies 
$$
|u/v| < 2.6 \times 10^{17},
$$
whence $|v| > 1.3 \times 10^{21}$. Substituting this a third time into (\ref{frenchy}), 
$$
\left| \alpha_1 - \frac{u}{v} \right| <10^{-30},
$$
so that $|u/v| < 3.1 \times 10^7$ and $|v| > 10^{31}$. One final use of (\ref{frenchy}) thus yields the inequality
$$
\left| \alpha_1 - \frac{u}{v} \right| <10^{-59}.
$$
Appealing to (\ref{size}), (\ref{general}), (\ref{iota}), and the fact that $|\alpha_1| < 3 \times 10^7$, we thus have, after a little work,
$$
\left| H_F (\alpha_1,1) - H_F (u/v,1) \right| < 3.4 \times 10^{-44}.
$$
With (\ref{foggie}), this implies that
$$
\left|  H_F (u/v,1)  \right| > \frac{1}{1562 \, p}
$$
and so
$$
|H_F(u,v)| = v^2 \left| H_F(u/v,1) \right| > \frac{v^2 }{1562 p} > \frac{10^{62}}{3124 \times 10^{13}} > 3 \times 10^{45},
$$
as claimed.
\end{proof}


%We have
%$$
%H_F(u,v)= (\omega_1^2-3\omega_0 \omega_2) u^2 + (\omega_1 \omega_2-9\omega_0 \omega_3) uv + (\omega_2^2-3\omega_1\omega_3) v^2.
%$$

%$$
%\omega_3^2 - \omega_0^2 >  \omega_1 \omega_3-\omega_0 \omega_2, \; \; -(\omega_0-\omega_1)^2-\omega_0 \omega_2< \omega_0 \omega_3-\omega_1 \omega_2 < (\omega_0+\omega_1)^2+\omega_0 \omega_2.
%$$

%---------------------------------------------------
\section{Concluding remarks}
%--------------------------------------------------

Many of the techniques of this chapter can be generalized to potentially treat the problem of determining elliptic curves of a given conductor over a number field $K$. 
In case $K$ is an imaginary quadratic field of class number $1$, then, in fact, such an approach works without any especially new ingredients. 



%---------------------------------------------------------------------------------------------------------------------------------------------%

\endinput

Any text after an \endinput is ignored.
You could put scraps here or things in progress.

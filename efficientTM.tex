%% The following is a directive for TeXShop to indicate the main file
%%!TEX root = diss.tex

\chapter{Towards Efficient Resolution of Thue-Mahler Equations}
\label{ch:EfficientTMSolver}

Let $a$ denote a nonzero integer and let $S=\{p_1,\dotsc,p_v\}$ be a set of rational primes. In this section, we specialize the results of \autoref{ch:AlgorithmsForTM} to the degree $3$ Thue--Mahler equation
\begin{equation} \label{Eq:TM1}
F(X,Y) = c_0 X^3 + c_1 X^{2}Y + c_2XY^2 + c_3Y^3 = a p_1^{Z_1}\cdots p_v^{Z_v},
\end{equation}
where $(X,Y) \in \mathbb{Z}^2$, $\gcd(X,Y)=1$, and $Z_i \geq 0$ for $i = 1, \dots, v$. In particular, to enumerate the set of solutions $\{X,Y, Z_1, \dots, Z_v\}$ to this equation, we follow \autoref{sec:FactorizationTM} to reduce the problem of solving \eqref{Eq:TM1} to solving finitely many so-called ``$S$-unit'' equations
\begin{equation} \label{eq:EfficientSunit}
\lambda = \delta_1 \prod_{i = 1}^r\left( \frac{\varepsilon_i^{(k)}}{\varepsilon_i^{(j)}}\right)^{a_i}\prod_{i = 1}^{\nu} \left( \frac{\gamma_i^{(k)}}{\gamma_i^{(j)}}\right)^{n_i} - 1 = \delta_2 \prod_{i = 1}^{r}\left( \frac{\varepsilon_i^{(i_0)}}{\varepsilon_i^{(j)}}\right)^{a_i} \prod_{i = 1}^{\nu} \left( \frac{\gamma_i^{(i_0)}}{\gamma_i^{(j)}}\right)^{n_i},
\end{equation}
where
\[\delta_1 = \frac{\theta^{(i_0)} - \theta^{(j)}}{\theta^{(i_0)} - \theta^{(k)}}\cdot\frac{\alpha^{(k)}\zeta^{(k)}}{\alpha^{(j)}\zeta^{(j)}}, \quad \delta_2 = \frac{\theta^{(j)} - \theta^{(k)}}{\theta^{(k)} - \theta^{(i_0)}}\cdot \frac{\alpha^{(i_0)}\zeta^{(i_0)}}{\alpha^{(j)}\zeta^{(j)}}\]
are constants. Here, we adopt the notation of \autoref{ch:AlgorithmsForTM} and recall that we reduce \eqref{Eq:TM1} to a homogenous equation of the form
\begin{equation} \label{eq:Efficientpoly}
f(x,y) = x^3 + C_1x^2y + C_2xy^2 + C_3y^3 = cp_1^{z_1}\cdots p_v^{z_v},
\end{equation}
where $\gcd(x,y) = 1$ and $\gcd(c,p_i) = 1$ for $i = 1, \dots, p_v$. Moreover, we set
\begin{equation} \label{eq:Efficientg}
g(t) = f(t,1) = t^3 + C_1t^2 + C_2t + C_3
\end{equation}
so that $K = \mathbb{Q}(\theta)$ with $g(\theta) = 0$. Recall that $\zeta$ in \eqref{eq:EfficientSunit} denotes a root of unity in $K$, while $\{\eps_1, \dots, \eps_r\}$ is a set of fundamental units of $\mathcal{O}_K$. In this case, as $K$ is a degree $3$ extension of $\mathbb{Q}$, we either have $3$ real embeddings of $K$ into $\mathbb{C}$, or one real embedding of $K$ into $\mathbb{C}$ and a pair of complex conjugate embeddings of $K$ into $\mathbb{C}$. Thus either $r = 1$ or $r = 2$. 

In this section, we describe new techniques to solve equation~\eqref{eq:EfficientSunit} via a global Weil height. This work is part of the on-going collaborative project \cite{GhKaMaSi}. Notably, the ideas presented in this chapter do not yet yield a full degree $3$ Thue-Mahler solver. Indeed, for the time being, only those Thue-Mahler equations with $r = 2$ are considered. However, when $r = 1$, the general setup established in this chapter remains the same. 

%---------------------------------------------------------------------------------------------------------------------------------------------%
%---------------------------------------------------------------------------------------------------------------------------------------------%
\section{Decomposition of the Weil height} 
\label{sec:DecompositionofWeilHeight}

The sieves of \cite{TW3} involve logarithms which are of local nature. To obtain a global sieve, we work instead with the global logarithmic Weil height. This height is invariant under conjugation and admits a decomposition into local heights which can be related to complex and $p$-adic logarithms. 

Let $n_1, \dots, n_{\nu}, a_1, \dots, a_r$ be a solution to \eqref{eq:EfficientSunit} and consider the Weil height of
\[\frac{\delta_2}{\lambda}= \prod_{i = 1}^{r}\left( \frac{\varepsilon_i^{(j)}}{\varepsilon_i^{(i_0)}}\right)^{a_i} \prod_{i = 1}^{\nu} \left( \frac{\gamma_i^{(j)}}{\gamma_i^{(i_0)}}\right)^{n_i}.\]
Given the global Weil height of $\delta_2/\lambda$, or all the local heights of $\delta_2/\lambda$, we will construct several ellipsoids `containing' $n_1, \dots, n_{\nu}, a_1, \dots, a_r$ such that the volume of the ellipsoids are as small as possible. We begin by computing the height of $\delta_2/\lambda$. 

Let $L$ be the splitting field of $K$. Recall that for cubic extensions $K$, the Galois group $\Gal(L/\mathbb{Q})$ is isomorphic to either the alternating group $A_3$ or the symmetric group $S_3$. 

\begin{lemma}\label{lem:cancellation}
Let $\mathfrak{p}$ be a prime ideal of $\mathcal{O}_K$ and let $\mathfrak{P}$ denote an ideal of $\mathcal{O}_L$ lying above it. Suppose $\sigma_{i_0}: L \to L$, $\theta \mapsto \theta^{(i_0)}$ and $\sigma_{j}: L \to L$, $\theta \mapsto \theta^{(j)}$ are two automorphisms of $L$ such that $(i_0,j,k)$ forms a subgroup of $\Gal(L/\mathbb{Q})$ of order $3$. Let $\mathfrak{P}^{(i_0)} = \sigma_{i_0}(\mathfrak{P})$ and $\mathfrak{P}^{(j)} = \sigma_{j}(\mathfrak{P})$ be the prime ideals lying over $\mathfrak{p}^{(i_0)}$, $\mathfrak{p}^{(j)}$ respectively. For $i = 1, \dots, \nu$, 
\[\left( \frac{\gamma_i^{(j)}}{\gamma_i^{(i_0)}}\right)\mathcal{O}_L 
	 = \left(\prod_{\mathfrak{P}\mid\mathfrak{p}_1} \frac{\mathfrak{P}^{(j) \ e(\mathfrak{P}^{(j)}\mid\mathfrak{p}_1^{(j)})}}{\mathfrak{P}^{(i_0) \ e(\mathfrak{P}^{(i_0)}\mid\mathfrak{p}^{(i_0)}_1)}}\right)^{a_{1i}} \cdots \left(\prod_{\mathfrak{P}\mid\mathfrak{p}_{\nu}} \frac{\mathfrak{P}^{(j) \ e(\mathfrak{P}^{(j)}\mid\mathfrak{p}^{(j)}_{\nu})}}{\mathfrak{P}^{(i_0) \ e(\mathfrak{P}^{(i_0)}\mid\mathfrak{p}^{(i_0)}_{\nu})}}\right)^{a_{\nu i}}\]
where $\mathfrak{P}^{(j)} \neq \mathfrak{P}^{(i_0)}$ for all $\mathfrak{P}$ lying above $\mathfrak{p}$ in $K$. 
\end{lemma}

\begin{proof}
Since 
\[(\gamma_i)\mathcal{O}_K = \mathfrak{p}_1^{a_{1i}} \cdots \mathfrak{p}_{\nu}^{a_{\nu i}},\]
for $i = 1, \dots, \nu$, where
\[\mathfrak{p}_i\mathcal{O}_L=\prod_{\mathfrak{P}\mid\mathfrak{p}_i} \mathfrak{P}^{e(\mathfrak{P}\mid\mathfrak{p}_i)},\]
it holds that
\[(\gamma_i)\mathcal{O}_L = \left(\prod_{\mathfrak{P}\mid\mathfrak{p}_1} \mathfrak{P}^{e(\mathfrak{P}\mid\mathfrak{p}_1)}\right)^{a_{1i}} \cdots \left(\prod_{\mathfrak{P}\mid\mathfrak{p}_{\nu}} \mathfrak{P}^{e(\mathfrak{P}\mid\mathfrak{p}_{\nu})}\right)^{a_{\nu i}}.\]

Let $\mathfrak{P}^{(i_0)},\mathfrak{P}^{(j)}$ denote the ideal $\mathfrak{P}$ under the automorphisms of $L$
\[\sigma_{i_0}: L \to L, \quad \theta \mapsto \theta^{(i_0)} \quad \text{ and } \quad \sigma_{j}: L \to L, \quad \theta \mapsto \theta^{(j)},\]
respectively. That is, $\mathfrak{P}^{(i_0)} = \sigma_{i_0}(\mathfrak{P})$ and $\mathfrak{P}^{(j)} = \sigma_{j}(\mathfrak{P})$. Then
\[\left( \frac{\gamma_i^{(j)}}{\gamma_i^{(i_0)}}\right)\mathcal{O}_L 
	 = \left(\prod_{\mathfrak{P}\mid\mathfrak{p}_1} \frac{\mathfrak{P}^{(j) \ e(\mathfrak{P}^{(j)}\mid\mathfrak{p}_1^{(j)})}}{\mathfrak{P}^{(i_0) \ e(\mathfrak{P}^{(i_0)}\mid\mathfrak{p}^{(i_0)}_1)}}\right)^{a_{1i}} \cdots \left(\prod_{\mathfrak{P}\mid\mathfrak{p}_{\nu}} \frac{\mathfrak{P}^{(j) \ e(\mathfrak{P}^{(j)}\mid\mathfrak{p}^{(j)}_{\nu})}}{\mathfrak{P}^{(i_0) \ e(\mathfrak{P}^{(i_0)}\mid\mathfrak{p}^{(i_0)}_{\nu})}}\right)^{a_{\nu i}}.\]

To show that $\mathfrak{P}^{(j)} \neq \mathfrak{P}^{(i_0)}$ for all $\mathfrak{P}$ lying above $\mathfrak{p}$ in $K$, we consider the decomposition group of $\mathfrak{P}$, 
\[D(\mathfrak{P}|p) = \{\sigma \in G \ : \ \sigma(\mathfrak{P}) = \mathfrak{P}\}.\]
Iterating through all possible decompositions of $\mathfrak{p}$ in $L$, we observe that $\mathfrak{P}^{(i_0)} \neq \mathfrak{P}^{(j)}$ whenever $D(\mathfrak{P}_i|p)$ does not have cardinality $2$. Since $(i_0,j,k)$ forms an order $3$ subgroup of $\Gal(L/\mathbb{Q})$, it cannot coincide with $D(\mathfrak{P}|p)$ and therefore cannot lead to $\mathfrak{P}^{(i_0)} = \mathfrak{P}^{(j)}$. 
\end{proof}

For the remainder of this paper, we assume that $(i_0,j,k)$ are automorphisms of $L$ selected as in Lemma~\ref{lem:cancellation}.

\begin{lemma}\label{lem:ordpz}
Let $\mathfrak{p}$ be a prime ideal of $\mathcal{O}_K$ and let $\mathfrak{P}$ denote an ideal of $\mathcal{O}_L$ lying above it. Let $\mathfrak{P}^{(i_0)} = \sigma_{i_0}(\mathfrak{P})$ and $\mathfrak{P}^{(j)} = \sigma_{j}(\mathfrak{P})$ be the prime ideals lying over $\mathfrak{p}^{(i_0)}$, $\mathfrak{p}^{(j)}$ respectively. We have
\[\ord_{\mathfrak{P}}\left(\frac{\delta_2}{\lambda}\right)=
\begin{cases}
(u_l - r_l)e(\mathfrak{P}^{(j)}|\mathfrak{p}_l^{(j)})	
	& \textnormal{ if } \mathfrak{P}^{(j)} \mid p_l , \ p_l \in \{p_1,\dots, p_{\nu}\}\\
(r_l - u_l)e(\mathfrak{P}^{(i_0)}|\mathfrak{p}_l^{(i_0)})
	& \textnormal{ if } \mathfrak{P}^{(i_0)}\mid p_l, \ p_l \in \{p_1,\dots, p_{\nu}\}\\
0 	& \textnormal{ otherwise}.
\end{cases}\]
\end{lemma}
\begin{proof}

By Lemma~\ref{lem:cancellation}, we have 
\begin{align*}
\left(\frac{\delta_2}{\lambda}\right)\mathcal{O}_L
	& = \left( \frac{\gamma_1^{(j)}}{\gamma_1^{(i_0)}}\right)^{n_1}\cdots \left( \frac{\gamma_{\nu}^{(j)}}{\gamma_{\nu}^{(i_0)}}\right)^{n_{\nu}} \mathcal{O}_L\\
	& = \left(\prod_{\mathfrak{P}\mid\mathfrak{p}_1} \frac{\mathfrak{P}^{(j) \ e(\mathfrak{P}^{(j)}\mid\mathfrak{p}_1^{(j)})}}{\mathfrak{P}^{(i_0) \ e(\mathfrak{P}^{(i_0)}\mid\mathfrak{p}^{(i_0)}_1)}}\right)^{\sum_{i = 1}^\nu n_ia_{1i}} \cdots \left(\prod_{\mathfrak{P}\mid\mathfrak{p}_{\nu}} \frac{\mathfrak{P}^{(j) \ e(\mathfrak{P}^{(j)}\mid\mathfrak{p}^{(j)}_{\nu})}}{\mathfrak{P}^{(i_0) \ e(\mathfrak{P}^{(i_0)}\mid\mathfrak{p}^{(i_0)}_{\nu})}}\right)^{\sum_{i=1}^{\nu} n_ia_{\nu i}}\\
	& = \left(\prod_{\mathfrak{P}\mid\mathfrak{p}_1} \frac{\mathfrak{P}^{(j) \ e(\mathfrak{P}^{(j)}\mid\mathfrak{p}_1^{(j)})}}{\mathfrak{P}^{(i_0) \ e(\mathfrak{P}^{(i_0)}\mid\mathfrak{p}^{(i_0)}_1)}}\right)^{u_1 - r_1} \cdots \left(\prod_{\mathfrak{P}\mid\mathfrak{p}_{\nu}} \frac{\mathfrak{P}^{(j) \ e(\mathfrak{P}^{(j)}\mid\mathfrak{p}^{(j)}_{\nu})}}{\mathfrak{P}^{(i_0) \ e(\mathfrak{P}^{(i_0)}\mid\mathfrak{p}^{(i_0)}_{\nu})}}\right)^{u_{\nu} - r_{\nu}}.
\end{align*}
It follows that 
\[\ord_{\mathfrak{P}}\left( \frac{\delta_2}{\lambda}\right)=
\begin{cases}
(u_l - r_l)e(\mathfrak{P}^{(j)}|\mathfrak{p}_l^{(j)})	
	& \textnormal{ if } \mathfrak{P}^{(j)} \mid p_l , \ p_l \in \{p_1,\dots, p_{\nu}\}\\
(r_l - u_l)e(\mathfrak{P}^{(i_0)}|\mathfrak{p}_l^{(i_0)})
	& \textnormal{ if } \mathfrak{P}^{(i_0)}\mid p_l, \ p_l \in \{p_1,\dots, p_{\nu}\}\\
0 	& \textnormal{ otherwise}.
\end{cases}\]
\end{proof}

Let $\log^+(\cdot)$ denote the real valued function $\max(\log(\cdot), 0)$ on $\mathbb{R}_{\geq 0}$. 

\begin{proposition}\label{prop:heightdecomp}
The height $h\left(\frac{\delta_2}{\lambda}\right)$ admits a decomposition 
\begin{equation} \label{eq:hdecomp}
h\left(\frac{\delta_2}{\lambda}\right) = \frac{1}{[K:\mathbb{Q}]}\sum_{l = 1}^{\nu} \log(p_l)|u_l - r_l| + \frac{1}{[L:\mathbb{Q}]}\sum_{w :L \to \mathbb{C}} \log \max \left\{ \left|w\left(\frac{\delta_2}{\lambda}\right)\right|, 1\right\}.
\end{equation}
\end{proposition}
For ease of notation, let $S^* = S \cup \{v : L \to \mathbb{C}\}$ and write 
\[h\left(\frac{\delta_2}{\lambda}\right) = \frac{1}{[K:\mathbb{Q}]}\sum_{v \in S^*}h_{v}\left(\frac{\delta_2}{\lambda}\right).\]
By Proposition~\ref{prop:heightdecomp}, when $v = p_l$ is a finite place, 
\[h_{v}\left(\frac{\delta_2}{\lambda}\right) = \log(p_l)|u_l - r_l|,\]
whereas we write
\[h_{v}\left(\frac{\delta_2}{\lambda}\right) = \frac{1}{[L:K]} \log \max \left\{ \left|w\left(\frac{\delta_2}{\lambda}\right)\right|, 1\right\}\]
for all infinite places $v = w$. Finally, let $m$ denote the number of embeddings of $L$ into $\mathbb{C}$,  $m = \# \{v : L \to \mathbb{C}\}$. 

\begin{proof}[Proof of Proposition~\ref{prop:heightdecomp}]Since $\frac{\delta_2}{\lambda} \in L$, the definition of the absolute logarithmic Weil height gives
\[h\left(\frac{\delta_2}{\lambda}\right)=\frac{1}{[L:\mathbb{Q}]}\sum_{w \in M_L} \log \max \left\{ \left\|\frac{\delta_2}{\lambda}\right\|_{w}, 1\right\}\]
where $||z||_w$ is the usual norms and $M_L$ is a set of inequivalent absolute values on $L$. In particular, if $w: L \to \mathbb{C}$ is an infinite place, we obtain
\[ \log \max \left\{ \left\|\frac{\delta_2}{\lambda}\right\|_{w}, 1\right\} = \log \max \left\{ \left|w\left(\frac{\delta_2}{\lambda}\right)\right|, 1\right\}.\]
Writing $z = \frac{\delta_2}{\lambda}$ and $w = \mathfrak{P}$ a finite place, we have
\[ \log \max \{ \|z\|_{w}, 1\} = \max \left\{ \log\left(\frac{1}{N(\mathfrak{P})^{\ord_{\mathfrak{P}}(z)}} \right), 0\right\}. \]
By Lemma~\ref{lem:ordpz}, 
\[\ord_{\mathfrak{P}}\left( \frac{\delta_2}{\lambda}\right)=
\begin{cases}
(u_l - r_l)e(\mathfrak{P}^{(j)}|\mathfrak{p}_l^{(j)})	
	& \textnormal{ if } \mathfrak{P}^{(j)} \mid p_l , \ p_l \in \{p_1,\dots, p_{\nu}\}\\
(r_l - u_l)e(\mathfrak{P}^{(i_0)}|\mathfrak{p}_l^{(i_0)})
	& \textnormal{ if } \mathfrak{P}^{(i_0)}\mid p_l, \ p_l \in \{p_1,\dots, p_{\nu}\}\\
0 	& \textnormal{ otherwise}.
\end{cases}\]
That is, for $\mathfrak{P}^{(j)}\mid p_l$ where $p_l \in \{p_1, \dots, p_{\nu}\}$, we have
\begin{align*}
 \log \max \{ ||z||_{w}, 1\}	
 	& = \max \left\{ \log\left(\frac{1}{N(\mathfrak{P})^{\ord_{\mathfrak{P}}(z)}} \right), 0\right\}\\
	& = \max \left\{ \log\left(\frac{1}{N(\mathfrak{P})^{(u_l - r_l)e(\mathfrak{P}^{(j)}|\mathfrak{p}_l^{(j)})}} \right), 0\right\}\\
	& = \max \left\{ -(u_l - r_l)f(\mathfrak{P}^{(j)}\mid p_l)e(\mathfrak{P}^{(j)}|\mathfrak{p}_l^{(j)})\log(p_l), 0\right\}.
\end{align*}
Moreover, for $p_l \in \{p_1, \dots, p_{\nu}\}$, there is one unique prime ideal $\mathfrak{p}_l$ in the ideal equation \eqref{eq:TMfactored} lying above $p_l$ in $K$. Hence, each $\mathfrak{P}$ lying over $p_l$ must also lie over $\mathfrak{p}_l$. Now, 
\begin{align*}
\sum_{\mathfrak{P}^{(j)} \mid \mathfrak{p}_l^{(j)}} \log \max \left\{ \left\|\frac{\delta_2}{\lambda}\right\|_{w}, 1\right\}
	& = \sum_{\mathfrak{P}^{(j)} \mid \mathfrak{p}_l^{(j)}} \max \left\{ -(u_l - r_l)f(\mathfrak{P}^{(j)}\mid p_l)e(\mathfrak{P}^{(j)}|\mathfrak{p}_l^{(j)})\log(p_l), 0\right\}\\
	& = \max \left\{ (r_l - u_l)\log(p_l), 0\right\}f(\mathfrak{p}_l^{(j)}\mid p_l)[L:\mathbb{Q}(\theta^{(j)})]\\
	& = \max \left\{ (r_l - u_l)\log(p_l), 0\right\}f(\mathfrak{p}_l^{(j)}\mid p_l)[L:K],
\end{align*}
where the last inequality follows from $K = \mathbb{Q}(\theta) \cong \mathbb{Q}(\theta^{(j)})$.

Similarly, for $\mathfrak{P}^{(i_0)}\mid p_l$ where $p_l \in \{p_1, \dots, p_{\nu}\}$, we have
\begin{align*}
 \log \max \{ \|z\|_{w}, 1\}	
 	& = \max \left\{ \log\left(\frac{1}{N(\mathfrak{P})^{\ord_{\mathfrak{P}}(z)}} \right), 0\right\}\\
	& = \max \left\{ \log\left(\frac{1}{N(\mathfrak{P})^{(r_l - u_l)e(\mathfrak{P}^{(i_0)}|\mathfrak{p}_l^{(i_0)})}} \right), 0\right\}\\
	& = \max \left\{ -(r_l - u_l)f(\mathfrak{P}^{(i_0)}\mid p_l)e(\mathfrak{P}^{(i_0)}|\mathfrak{p}_l^{(i_0)})\log(p_l), 0\right\},
\end{align*}
and so
\begin{align*}
\sum_{\mathfrak{P}^{(i_0)} \mid \mathfrak{p}_l^{(i_0)}} \log \max \left\{ \left\|\frac{\delta_2}{\lambda}\right\|_{w}, 1\right\}
	& = \max \left\{ (u_l - r_l)\log(p_l), 0\right\}f(\mathfrak{p}_l^{(i_0)}\mid p_l)[L:K].
\end{align*}

Lastly, if $w = \mathfrak{P}$ such that $\mathfrak{P} \neq \mathfrak{P}^{(i_0)},  \mathfrak{P}^{(j)}$, we have
\[\log \max \{ \|z\|_{w}, 1\} = \max \left\{ \log\left(\frac{1}{N(\mathfrak{P})^{\ord_{\mathfrak{P}}(z)}} \right), 0\right\}=0.\]
Putting this all together, we obtain
\begin{align*}
h\left(\frac{\delta_2}{\lambda}\right)
	& = \frac{1}{[L:\mathbb{Q}]}\sum_{w :L \to \mathbb{C}} \log \max \left\{ \left|w\left(\frac{\delta_2}{\lambda}\right)\right|, 1\right\} + \frac{1}{[L:\mathbb{Q}]}\sum_{\mathfrak{P} \in \mathcal{O}_L \text{ finite }} \log \max \left\{ \left\|\frac{\delta_2}{\lambda}\right\|_{\mathfrak{P}}, 1\right\}\\
	& = \frac{1}{[L:\mathbb{Q}]}\sum_{w :L \to \mathbb{C}} \log \max \left\{ \left|w\left(\frac{\delta_2}{\lambda}\right)\right|, 1\right\} + \frac{1}{[K:\mathbb{Q}]}\sum_{l = 1}^{\nu} \log(p_l)|u_l - r_l|.
\end{align*}
\end{proof}

%---------------------------------------------------------------------------------------------------------------------------------------------%
\section{Initial height bounds}
\label{sec:InitialHeightBounds}

We seek solutions to equaition~\eqref{eq:EfficientSunit}. We recall this equation presently, 
\begin{equation*}
\lambda = \delta_1 \prod_{i = 1}^r\left( \frac{\varepsilon_i^{(k)}}{\varepsilon_i^{(j)}}\right)^{a_i}\prod_{i = 1}^{\nu} \left( \frac{\gamma_i^{(k)}}{\gamma_i^{(j)}}\right)^{n_i} - 1 = \delta_2 \prod_{i = 1}^{r}\left( \frac{\varepsilon_i^{(i_0)}}{\varepsilon_i^{(j)}}\right)^{a_i} \prod_{i = 1}^{\nu} \left( \frac{\gamma_i^{(i_0)}}{\gamma_i^{(j)}}\right)^{n_i}.
\end{equation*}
To simplify notation, we write
\[\tilde{y} =  \prod_{i = 1}^r\left( \frac{\varepsilon_i^{(k)}}{\varepsilon_i^{(j)}}\right)^{a_i}\prod_{i = 1}^{\nu} \left( \frac{\gamma_i^{(k)}}{\gamma_i^{(j)}}\right)^{n_i}, \quad 
\tilde{x} = \prod_{i = 1}^{r}\left( \frac{\varepsilon_i^{(i_0)}}{\varepsilon_i^{(j)}}\right)^{a_i} \prod_{i = 1}^{\nu} \left( \frac{\gamma_i^{(i_0)}}{\gamma_i^{(j)}}\right)^{n_i}\]
so that equation~\eqref{eq:EfficientSunit} becomes
\begin{equation} \label{eq:TrueSunit}
\delta_1\tilde{y} - \delta_2\tilde{x} = 1.
\end{equation}
Let $z= \frac{1}{\tilde{x}} = \frac{\delta_2}{\lambda}$ and denote by $\Sigma$ the set of pairs $(\tilde{x},\tilde{y})$ satisfying \eqref{eq:TrueSunit}. That is, $\Sigma$ denotes the set of tuples $(n_1, \dots, n_{\nu}, a_1, \dots, a_r)$ corresponding to $(\tilde{x},\tilde{y})$ which satisfy \eqref{eq:TrueSunit}.

Let $\mathbf{l},\mathbf{h}\in\mathbb{R}^{\nu + m}$ with $\mathbf{0}\leq \mathbf{l}\leq \mathbf{h}$. Then we define $\Sigma(\mathbf{l},\mathbf{h})$ as the set of all $(\tilde{x},\tilde{y}) \in \Sigma$ such that $\left(h_v\left(\frac{\delta_2}{\lambda}\right)\right)\leq \mathbf{h}$ and such that $\left(h_v\left(\frac{\delta_2}{\lambda}\right)\right)\nleq \mathbf{l}$, and write $\Sigma(\mathbf{h})=\Sigma(\mathbf{l},\mathbf{h})$ if $\mathbf{l}=\mathbf{0}$. Additionally, for each place $w$, we denote by $\Sigma_w(\mathbf{l},\mathbf{h})$ the set of all $(\tilde{x},\tilde{y})\in\Sigma(\mathbf{h})$ such that $h_w(z)>l_w$. 

Recall the minimal polynomial $g(t)$ of $K$, \eqref{eq:Efficientg}, derived from 
\[f(x,y) = x^3 + C_1 x^{2}y + C_2xy^2 + C_3y^3 = cp_1^{z_1}\cdots p_v^{z_v}.\]
For $S = \{p_1, \dots, p_v\}$, let $N_S = \prod_{p\in S}p$ and set 
\[b_S	 = 1728 N_S^2 \prod_{p \notin S} p^{\min(2,\ord_p(b))}\]
for any integer $b$. In particular, we take $b = 432 \Delta c^2$ with $\Delta$ the discriminant of $f$. Denote by $h(f-c)$ the maximum logarithmic Weil heights of the coefficients of the polynomial $f - c$,
\[h(f-c) = \max(\log|C_1|, \log|C_2|, \log|C_3|, \log|c|).\]

Now, setting
\[\Omega = 2b_S \log(b_S) + 172h(f-c),\]
we obtain, by Corollary J (ii) of \cite{KanMat}, the following height bound on any solution $(x,y)$ of \eqref{eq:Efficientpoly}
\[\max(h(x),h(y))\leq \Omega.\]
To translate this result for use with our logarithmic Weil height \eqref{eq:hdecomp}, we have the following lemma. 
\begin{lemma} \label{lem:TMinitialheight}
Let ${\mathbf{m} = (n_1, \dots, n_{\nu}, a_1, \dots, a_r) \in \mathbb{R}^{r + \nu}}$ be any solution of \eqref{eq:EfficientSunit} and let 
\begin{equation} \label{eq:Omegaprime}
\Omega' = 2h(\alpha) + 4\Omega + 2h(\theta) + 2\log(2).
\end{equation}
If $\mathbf{h} \in\mathbb{R}^{\nu + m}$ with $\mathbf{h} = (\Omega')$, then $\mathbf{m}\in \Sigma(h)$.
\end{lemma}

\begin{proof}
Let $(\tilde{x},\tilde{y}) \in \Sigma$. We show that the corresponding value $z = \frac{1}{\tilde{x}} = \frac{\delta_2}{\lambda}$ arising from this choice of $\tilde{x},\tilde{y}$ satisfies
\[\mathbf{0} < \left(h_v\left(\frac{\delta_2}{\lambda}\right)\right)\leq \mathbf{h}.\]

As stated earlier, any solution $x,y$ of $f(x,y) = c p_1^{z_1}\cdots p_v^{z_v}$ satisfies
\[\max(h(x),h(y)) \leq \Omega.\]
Taking the height of 
\[\beta = x-y\theta = \alpha \zeta \varepsilon_1^{a_1} \cdots \varepsilon_r^{a_r}\cdot \gamma_1^{n_1}\cdots \gamma_{\nu}^{n_{\nu}},\]
we obtain
\[h(\beta) = h(x) + h(\theta) + h(y) + \log{2}  \leq 2\Omega + h(\theta) + \log{2}.\]
In particular, as $h(\beta) = h(\beta^{(i)})$, 
\[h(\beta^{(i)}) \leq 2\Omega + h(\theta) + \log{2}.\]
Now, 
\[\delta_2\tilde{x} 
	= \frac{\theta^{(j)} - \theta^{(k)}}{\theta^{(k)} - \theta^{(i_0)}}\cdot \frac{\alpha^{(i_0)}\zeta^{(i_0)}}{\alpha^{(j)}\zeta^{(j)}} \prod_{i = 1}^{r}\left( \frac{\varepsilon_i^{(i_0)}}{\varepsilon_i^{(j)}}\right)^{a_i} \prod_{i = 1}^{\nu} \left( \frac{\gamma_i^{(i_0)}}{\gamma_i^{(j)}}\right)^{n_i}  
	= \frac{\theta^{(j)} - \theta^{(k)}}{\theta^{(k)} - \theta^{(i_0)}}\cdot \frac{\beta^{(i_0)}}{\beta^{(j)}},\]
meaning that $\tilde{x}$ may be written as
\[\tilde{x} =\frac{\beta^{(i_0)}}{\beta^{(j)}}\cdot \frac{\alpha^{(j)}\zeta^{(j)}}{\alpha^{(i_0)}\zeta^{(i_0)}}.\]
Hence, 
\[h(\tilde{x})	 = 2h(\beta) + 2h(\alpha) \leq 4\Omega + 2h(\theta) + 2\log{2} + 2h(\alpha) = \Omega'.\]
Finally, we observer that 
\[h(z) = h(1/\tilde{x}) \leq \Omega'.\]
Together with $\displaystyle h_v\left(\frac{\delta_2}{\lambda}\right) \leq h\left(\frac{\delta_2}{\lambda}\right)$, this implies
\[h_v\left(\frac{\delta_2}{\lambda}\right) \leq \Omega'.\]
Of course, by definition, we have $h_v\left(\frac{\delta_2}{\lambda}\right) \geq 0$, so that $(\tilde{x},\tilde{y}) \in \Sigma(h)$ as required. 
\end{proof}

%---------------------------------------------------------------------------------------------------------------------------------------------%
\section{Coverings of $\Sigma$}
\label{sec:CoveringsofSigma}

From \autoref{sec:InitialHeightBounds}, we now know that all solutions $(\tilde{x},\tilde{y}) \in \Sigma$ satisfy $\mathbf{m}\in \Sigma(h)$ if ${\mathbf{h} = (\Omega')}$. In the notation of \autoref{sec:InitialHeightBounds}, we have the following result. 

\begin{lemma}\label{lem:covering}
Let $\mathbf{l},\mathbf{h}\in\mathbb{R}^{\nu+m}$ with $\mathbf{0}\leq \mathbf{l}\leq \mathbf{h}$. It holds that $\Sigma(\mathbf{h})=\Sigma(\mathbf{l},\mathbf{h})\cup \Sigma(\mathbf{l})$ and $\Sigma(\mathbf{l},\mathbf{h})=\cup_{v \in S^*}\Sigma_v(\mathbf{l},\mathbf{h})$.
\end{lemma}

\begin{proof}
Suppose $(\tilde{x},\tilde{y}) \in \Sigma(\mathbf{h})$. By definition this means, $(h_v(z))\leq \mathbf{h}$ and that $h_v(z) > 0$ for at least one coordinate $v$. Since $\mathbf{0} \leq \mathbf{l} \leq \mathbf{h}$, it follows that either $(h_v(z))\leq \mathbf{l}$ or $(h_v(z))\nleq \mathbf{l}$. That is, either all coordinates satisfy $h_v(z) \leq l_v$, or there is at least one coordinate for which $h_v(z) > l_v$. This means that either $(\tilde{x},\tilde{y}) \in \Sigma(\mathbf{l})$ or $(\tilde{x},\tilde{y}) \in \Sigma(\mathbf{l},\mathbf{h})$, and so $\Sigma(\mathbf{h}) \subseteq \Sigma(\mathbf{l},\mathbf{h}) \cup \Sigma(\mathbf{l})$.

Conversely, suppose  $(\tilde{x},\tilde{y}) \in \Sigma(\mathbf{l},\mathbf{h}) \cup \Sigma(\mathbf{l})$. It follows that either $(h_v(z))\leq \mathbf{h} \text{ and } (h_v(z))\nleq \mathbf{l}$ or $(h_v(z))\leq \mathbf{l} \text{ and } (h_v(z))\nleq \mathbf{0}$. In either case, this means that $(h_v(z)) \leq \mathbf{h}$ and $(h_v(z)) \nleq \mathbf{0}$. Hence $(\tilde{x},\tilde{y}) \in \Sigma(\mathbf{h})$ and $\Sigma(\mathbf{h}) \supseteq \Sigma(\mathbf{l},\mathbf{h}) \cup \Sigma(\mathbf{l})$.

To prove the second equality, let $(\tilde{x},\tilde{y}) \in \Sigma(\mathbf{l},\mathbf{h})$. Then there exists $w\in S^*$ with $h_w(z)>l_w$ so that $(\tilde{x},\tilde{y})$ lies in $\Sigma_w(\mathbf{l},\mathbf{h})$. Hence $\Sigma(\mathbf{l},\mathbf{h}) \subseteq \cup_{v\in S^*}\Sigma_v(\mathbf{l},\mathbf{h})$. Lastly, since each set $\Sigma_v(\mathbf{l},\mathbf{h})$ is contained in $\Sigma(\mathbf{l},\mathbf{h})$ it follows that $\Sigma(\mathbf{l},\mathbf{h})=\cup_{v\in S^*}\Sigma_v(\mathbf{l},\mathbf{h})$ as required. 
\end{proof}

Let $\mathbf{h}_0 = (\Omega', \dots, \Omega')$ denote the vector consisting of the initial bound $\Omega'$. By Proposition~\ref{lem:TMinitialheight}, every solution of \eqref{eq:EfficientSunit} is contained in $\mathbf{h}_0$. Therefore, we write $\Sigma = \Sigma(\mathbf{h}_0)$. Consider the pairs $(\mathbf{l}_n,\mathbf{h}_n)\in \mathbb{R}^{\nu + m}\times \mathbb{R}^{\nu + m}$ with $\mathbf{0}\leq \mathbf{l}_n\leq \mathbf{h}_n$ and $\mathbf{h}_{n+1}=\mathbf{l}_{n}$ for $n=0,\dotsc,N$. Then we can cover $\Sigma$: 
$$\Sigma=\Sigma(\mathbf{l}_{N})\cup\bigl(\cup_{n=0}^{N}\cup_{v\in S^*}\Sigma_v(\mathbf{l}_n,\mathbf{h}_n)\bigl).$$
Indeed this follows directly by applying Lemma~\ref{lem:covering} $N$ times. In particular, Lemma~\ref{lem:covering} gives  $$\Sigma=\Sigma(\mathbf{h}_0), \quad \Sigma(\mathbf{h})=\Sigma(\mathbf{l},\mathbf{h})\cup \Sigma(\mathbf{l}) \quad \textnormal{and} \quad \Sigma(\mathbf{l},\mathbf{h})=\cup_{v\in S^*}\Sigma_v(\mathbf{l},\mathbf{h}).$$
After choosing a good sequence of lower and upper bounds $\mathbf{l}_n,\mathbf{h}_n$ covering the whole space $\Sigma$, we are reduced to computing $\Sigma_v(\mathbf{l},\mathbf{h})$ for each $v \in S^*$. In the following section, we construct the ellipsoids associated to each $\Sigma_v(\mathbf{l},\mathbf{h})$, after which we describe the sieve allowing us to compute the solutions of each $\Sigma_v(\mathbf{l},\mathbf{h})$. 


%---------------------------------------------------------------------------------------------------------------------------------------------%

\section{Construction of the ellipsoids}
\label{sec:ConstructionofEllipsoids}

In \autoref{sec:CoveringsofSigma}, we establish that for a suitable pair of vectors $\mathbf{l}, \mathbf{h}$, solving \eqref{eq:EfficientSunit} reduces to computing $\Sigma_v(\mathbf{l},\mathbf{h})$ for each $v \in S^*$. In this section, we construct the ellipsoids associated to each $\Sigma_v(\mathbf{l},\mathbf{h})$, which will subsequently allow us to compute all solutions of $\Sigma_v(\mathbf{l},\mathbf{h})$. 

We begin with the quadratic form $q_f=A^TD^2A$ on $\mathbb{Z}^{\nu}$, where $D^2$ is a $\nu \times \nu$ diagonal matrix with diagonal entries $\lfloor\frac{\log(p_i)^2}{\log(2)^2}\rfloor$ for $p_i \in S$. Recall that $A$ is the matrix generated in either \autoref{subsec:FactorizationTMwithoutOK} or \autoref{subsec:FactorizationTMwithoutOK}. As $A$ is invertible, our choice of entries in $D$ guarantees that this quadratic form is positive definite. This will become very important later in the sieve when we will need to apply many instances of the Fincke-Pohst algorithm. 

\begin{lemma} \label{lem:boundqf}
For any solution $(x,y, n_1, \dots, n_{\nu}, a_1, \dots, a_r)$ of \eqref{eq:EfficientSunit} with $\mathbf{n} = (n_1, \dots, n_{\nu})$, we have 
\[\frac{\log(2)^2}{[K:\mathbb{Q}]}q_f(\mathbf{n}) < \frac{1}{[K:\mathbb{Q}]}\sum_{l = 1}^{\nu}\log(p_l)^2|u_l -r_l|^2.\] 
\end{lemma}

\begin{proof}
Recall from \autoref{subsec:FactorizationTMwithoutOK} and \autoref{subsec:FactorizationTMwithOK} that
\[A\mathbf{n} = \mathbf{u} - \mathbf{r}.\]
Assume first that $2 \notin S$ so that
\begin{align*}
q_f(\mathbf{n})	
	 = (A\mathbf{n})^{\text{T}}D^2A\mathbf{n}
	 = (\mathbf{u} - \mathbf{r})^{\text{T}}D^2(\mathbf{u} - \mathbf{r})
	 = \sum_{l = 1}^{\nu}\left\lfloor\frac{\log(p_l)^2}{\log(2)^2}\right\rfloor|u_l-r_l|^2.
\end{align*}
Multiplication by $\frac{\log(2)^2}{[K:\mathbb{Q}]}$ then gives
\begin{align*}
\frac{\log(2)^2}{[K:\mathbb{Q}]}q_f(\mathbf{n})  
	= \frac{\log(2)^2}{[K:\mathbb{Q}]}\sum_{l = 1}^{\nu} \left\lfloor\frac{\log(p_l)^2}{\log(2)^2}\right\rfloor|u_l -r_l|^2 
 	\leq \frac{1}{[K:\mathbb{Q}]}\sum_{l = 1}^{\nu}\log(p_l)^2|u_l -r_l|^2,
\end{align*}
where all terms in the summand are positive. 

If $2 \in S$, we have
\begin{align*}
q_f(\mathbf{n})	
	 = (A\mathbf{n})^{\text{T}}D^2A\mathbf{n}
	 = |u_1 - r_1|^2 + \sum_{l = 2}^{\nu}\left\lfloor\frac{\log(p_l)^2}{\log(2)^2}\right\rfloor|u_l-r_l|^2.
\end{align*}
It follows that
\begin{align*}
\frac{\log(2)^2}{[K:\mathbb{Q}]}q_f(\mathbf{n}) 
	& \leq \frac{\log(2)^2}{[K:\mathbb{Q}]}\left( |u_1 - r_1|^2 + \sum_{l = 2}^{\nu} \frac{\log(p_l)^2}{\log(2)^2}|u_l -r_l|^2\right) \\
	& = \frac{1}{[K:\mathbb{Q}]}\sum_{l = 1}^{\nu}\log(p_l)^2|u_l -r_l|^2.
\end{align*}
\end{proof}

We now briefly re-examine the decomposition of $h\left(\frac{\delta_2}{\lambda}\right)$ into local heights, 
\[h\left(\frac{\delta_2}{\lambda}\right) = \frac{1}{[K:\mathbb{Q}]}\sum_{l = 1}^{\nu} \log(p_l)|u_l - r_l| + \frac{1}{[L:\mathbb{Q}]}\sum_{w :L \to \mathbb{C}} \log \max \left\{ \left|w\left(\frac{\delta_2}{\lambda}\right)\right|, 1\right\}.\]
In particular, for every finite place $v$, Lemma~\ref{lem:boundqf} tells us that any bound $h_v$ on $h_v\left(\frac{\delta_2}{\lambda}\right)$ yields a bound on $\frac{\log(2)^2}{[K:\mathbb{Q}]}q_f(\mathbf{n})$. In the remainder of this section, we build analogous bounds on the exponents $a_1, \dots, a_r$ of the fundamental units. 

Recall $r = 1$ or $r = 2$ for the degree $3$ Thue-Mahler equation~\eqref{eq:Efficientpoly} in question. Choose a set $I$ of embeddings $L \rightarrow \mathbb{C}$ of cardinality $r$. For $r = 1$, this is simply
\[R = \begin{pmatrix}
	\log\left|\left(\frac{\varepsilon_1^{(j)}}{\varepsilon_1^{(i_0)}}\right)^{\iota_1}\right| \end{pmatrix}.\]
Clearly, as long as we choose $\iota_1$ such that $\log\left|\left(\frac{\varepsilon_1^{(j)}}{\varepsilon_1^{(i_0)}}\right)^{\iota_1}\right| \neq 0$, this matrix is invertible.

When $r = 2$, we let $I$ be the set of embeddings $L \to \mathbb{C}$ of cardinality $2$ such that for any $\alpha \in K$, it holds that $I\alpha^{(i_0)} \cup I\alpha^{(j)} = \Gal(L/\mathbb{Q})\alpha$. Let $R$ be the $2 \times 2$ matrix
\[R = \begin{pmatrix}
	\log\left|\left(\frac{\varepsilon_1^{(j)}}{\varepsilon_1^{(i_0)}}\right)^{\iota_1}\right| &
	\log\left|\left(\frac{\varepsilon_2^{(j)}}{\varepsilon_2^{(i_0)}}\right)^{\iota_1}\right|\\
	\log\left|\left(\frac{\varepsilon_1^{(j)}}{\varepsilon_1^{(i_0)}}\right)^{\iota_2}\right| &
	\log\left|\left(\frac{\varepsilon_2^{(j)}}{\varepsilon_2^{(i_0)}}\right)^{\iota_2}\right|\\
	\end{pmatrix}.\]
\begin{lemma}
When $r = 2$, the matrix $R$ has an inverse,
\[R^{-1} = \begin{pmatrix}
	\overline{r}_{11} & \overline{r}_{12} \\
	\overline{r}_{21} & \overline{r}_{22}
\end{pmatrix}.\]
\end{lemma}
	
\begin{proof}
Suppose that $\mathbf{m}\in\mathbb{Z}^{r}$ satisfies $R\mathbf{m}=\mathbf{0}$. Then for each $\iota\in I$ it holds that 
\[\sum_{i=1}^r m_{\eps_i} \log\left|\left(\frac{\varepsilon_i^{(j)}}{\varepsilon_i^{(i_0)}}\right)^{\iota_1}\right| =0,\] and hence 
\[\prod_{i=1}^r \left|\left(\frac{\varepsilon_i^{(j)}}{\varepsilon_i^{(i_0)}}\right)^{\iota_1}\right|^{m_{\eps_i}}=1.\] This together with $I(i)\cup I(j)=Gal(L/\mathbb{Q})$ implies that all conjugates of $\alpha=\prod_{i=1}^r \eps_i^{m_{\eps_i}}$ have the same absolute value. Since all $\eps_i$ are units of $\mathcal{O}_K$, it follows that $|\alpha|^{[L:\mathbb{Q}]}=N(\alpha)=1$ and hence $\alpha$ is a root of unity in $K$. On using that  the elements $\eps_i$ are multiplicatively independent, we obtain that $\mathbf{m}=\mathbf{0}$.  Then linear algebra gives $R^{-1}\in\mathbb{R}^{r\times r}$, completing the proof. 
\end{proof}

For the remainder of this chapter, we specialize to the real case, $r = 2$. The setup for $r = 1$ follows closely the work described here, yet poses other difficulties when defining the corresponding sieves. This case is treated in the on-going results of \cite{GhKaMaSi}. 

Now, for any solution $(x,y, n_1, \dots, n_{\nu}, a_1, a_2)$ of \eqref{eq:EfficientSunit}, set
\[\mathbf{\eps} = \begin{pmatrix} a_1 & a_2 \end{pmatrix}^{\text{T}}.\]
We have 
\begin{align*}
R{\varepsilon}
	& = \begin{pmatrix} 
		\log\left|\left(\frac{\varepsilon_1^{(j)}}{\varepsilon_1^{(i_0)}}\right)^{\iota_1 \ a_1} \cdot 
		 \left(\frac{\varepsilon_2^{(j)}}{\varepsilon_2^{(i_0)}}\right)^{\iota_1 \ a_2}\right| \\ 
		\log\left|\left(\frac{\varepsilon_1^{(j)}}{\varepsilon_1^{(i_0)}}\right)^{\iota_2\ a_2} \cdot 
		 \left(\frac{\varepsilon_2^{(j)}}{\varepsilon_2^{(i_0)}}\right)^{\iota_2 \ a_2}\right|
		 \end{pmatrix}.
\end{align*}
Since $R$ is invertible with $R^{-1} = (\overline{r}_{nm})$, we find
\begin{align*}
{\varepsilon} = \begin{pmatrix} a_1 \\ a_2 \end{pmatrix} 
	& = \begin{pmatrix} 
		\overline{r}_{11}\log\left|\left(\frac{\varepsilon_1^{(j)}}{\varepsilon_1^{(i_0)}}\right)^{\iota_1 \ a_1} 		\cdot \left(\frac{\varepsilon_2^{(j)}}{\varepsilon_2^{(i_0)}}\right)^{\iota_1 \ a_2}\right| + 
		\overline{r}_{12}\log\left|\left(\frac{\varepsilon_1^{(j)}}{\varepsilon_1^{(i_0)}}\right)^{\iota_2\ a_1}
		\cdot \left(\frac{\varepsilon_2^{(j)}}{\varepsilon_2^{(i_0)}}\right)^{\iota_2 \ a_2}\right| \\
		\overline{r}_{21}\log\left|\left(\frac{\varepsilon_1^{(j)}}{\varepsilon_1^{(i_0)}}\right)^{\iota_1 \ a_1} 		\cdot \left(\frac{\varepsilon_2^{(j)}}{\varepsilon_2^{(i_0)}}\right)^{\iota_1 \ a_2}\right| +
		\overline{r}_{22}\log\left|\left(\frac{\varepsilon_1^{(j)}}{\varepsilon_1^{(i_0)}}\right)^{\iota_2\ a_1}
		\cdot \left(\frac{\varepsilon_2^{(j)}}{\varepsilon_2^{(i_0)}}\right)^{\iota_2 \ a_2}\right|
		\end{pmatrix},
\end{align*}
giving
\[a_l = \overline{r}_{l1}\log\left|\left(\frac{\varepsilon_1^{(j)}}{\varepsilon_1^{(i_0)}}\right)^{\iota_1 \ a_1} 		\cdot \left(\frac{\varepsilon_2^{(j)}}{\varepsilon_2^{(i_0)}}\right)^{\iota_1 \ a_2}\right| + 
	\overline{r}_{l2}\log\left|\left(\frac{\varepsilon_1^{(j)}}{\varepsilon_1^{(i_0)}}\right)^{\iota_2\ a_1}
	\cdot \left(\frac{\varepsilon_2^{(j)}}{\varepsilon_2^{(i_0)}}\right)^{\iota_2 \ a_2}\right|\]
for $l = 1,2$. 

To estimate $|a_l|$, we begin to estimate the sum on the right hand side. For this, we consider
\[\frac{\delta_2}{\lambda}= \left( \frac{\varepsilon_1^{(j)}}{\varepsilon_1^{(i_0)}}\right)^{a_1}\left( \frac{\varepsilon_2^{(j)}}{\varepsilon_2^{(i_0)}}\right)^{a_2}\prod_{i = 1}^{\nu} \left( \frac{\gamma_i^{(j)}}{\gamma_i^{(i_0)}}\right)^{n_i}.\]
For any embedding $\iota: L \to \mathbb{C}$, we have 
\[\left(\frac{\delta_2}{\lambda}\right)^{\iota} \prod_{i = 1}^{\nu} \left( \frac{\gamma_i^{(i_0)}}{\gamma_i^{(j)}}\right)^{\iota \ n_i} =  \left( \frac{\varepsilon_1^{(j)}}{\varepsilon_1^{(i_0)}}\right)^{\iota \ a_1}\left( \frac{\varepsilon_2^{(j)}}{\varepsilon_2^{(i_0)}}\right)^{\iota \ a_2}.\] 

Taking absolute values, we obtain
\[\left|\left(\frac{\delta_2}{\lambda}\right)^{\iota} \prod_{i = 1}^{\nu} \left( \frac{\gamma_i^{(i_0)}}{\gamma_i^{(j)}}\right)^{\iota \ n_i}\right| = \left|\left( \frac{\varepsilon_1^{(j)}}{\varepsilon_1^{(i_0)}}\right)^{\iota \ a_1}\left( \frac{\varepsilon_2^{(j)}}{\varepsilon_2^{(i_0)}}\right)^{\iota \ a_2}\right|,\]
so that
\begin{align*}
\log\left|\left( \frac{\varepsilon_1^{(j)}}{\varepsilon_1^{(i_0)}}\right)^{\iota \ a_1}\left( \frac{\varepsilon_2^{(j)}}{\varepsilon_2^{(i_0)}}\right)^{\iota \ a_2}\right|
	& = \log\left|\left(\frac{\delta_2}{\lambda}\right)^{\iota}\right| - \log\left| \prod_{i = 1}^{\nu} \left( \frac{\gamma_i^{(j)}}{\gamma_i^{(i_0)}}\right)^{\iota \ n_i}\right|. 
\end{align*}

Hence, for $l =1,2$,
\begin{align*}
a_l	& = \overline{r}_{l1}\log\left|\left(\frac{\varepsilon_1^{(j)}}{\varepsilon_1^{(i_0)}}\right)^{\iota_1 \ a_1} 		\cdot \left(\frac{\varepsilon_2^{(j)}}{\varepsilon_2^{(i_0)}}\right)^{\iota_1 \ a_2}\right| + 
	\overline{r}_{l2}\log\left|\left(\frac{\varepsilon_1^{(j)}}{\varepsilon_1^{(i_0)}}\right)^{\iota_2\ a_1}
	\cdot \left(\frac{\varepsilon_2^{(j)}}{\varepsilon_2^{(i_0)}}\right)^{\iota_2 \ a_2}\right|\\
	& = \overline{r}_{l1}\left( \log\left|\left(\frac{\delta_2}{\lambda}\right)^{\iota_1}\right| - \log\left| \prod_{i = 1}^{\nu} \left( \frac{\gamma_i^{(j)}}{\gamma_i^{(i_0)}}\right)^{\iota_1 \ n_i}\right|\right) + \\ 
	& \quad \quad + \overline{r}_{l2}\left( \log\left|\left(\frac{\delta_2}{\lambda}\right)^{\iota_2}\right| - \log\left| \prod_{i = 1}^{\nu} \left( \frac{\gamma_i^{(j)}}{\gamma_i^{(i_0)}}\right)^{\iota_2 \ n_i}\right|\right)\\
	& = \overline{r}_{l1}\log\left|\left(\frac{\delta_2}{\lambda}\right)^{\iota_1}\right| + \overline{r}_{l2}\log\left|\left(\frac{\delta_2}{\lambda}\right)^{\iota_2}\right| - n_1\beta_{\gamma l 1} - \dots - n_{\nu}\beta_{\gamma l \nu},
\end{align*}
where
\[\beta_{\gamma l k} = \left(\overline{r}_{l1} \log\left| \left( \frac{\gamma_k^{(j)}}{\gamma_k^{(i_0)}}\right)^{\iota_1}\right|+ \overline{r}_{l2}\log\left| \left( \frac{\gamma_k^{(j)}}{\gamma_k^{(i_0)}}\right)^{\iota_2}\right|\right)\]
for $k = 1, \dots, \nu$. Recall that $\mathbf{n} = A^{-1}(\mathbf{u} - \mathbf{r})$ and suppose $A^{-1} = (\overline{a}_{nm})$. We have
\begin{align*}
\mathbf{n}  = A^{-1}(\mathbf{u} - \mathbf{r})
	 = \begin{pmatrix} \sum_{k=1}^{\nu} \overline{a}_{1k}(u_k-r_k)\\  \vdots \\ \sum_{k=1}^{\nu} \overline{a}_{\nu k}(u_k-r_k) \end{pmatrix}.
\end{align*}

Now, 
\begin{align*}
a_l 	& = \overline{r}_{l1}\log\left|\left(\frac{\delta_2}{\lambda}\right)^{\iota_1}\right| + \overline{r}_{l2}\log\left|\left(\frac{\delta_2}{\lambda}\right)^{\iota_2}\right| - n_1\beta_{\gamma l 1} - \dots - n_{\nu}\beta_{\gamma l \nu}\\
	& = \overline{r}_{l1}\log\left|\left(\frac{\delta_2}{\lambda}\right)^{\iota_1}\right| + \overline{r}_{l2}\log\left|\left(\frac{\delta_2}{\lambda}\right)^{\iota_2}\right| - \sum_{k=1}^{\nu}(u_k-r_k)\alpha_{\gamma l k},
\end{align*}
where
\[\alpha_{\gamma l k} = \overline{a}_{1k}\beta_{\gamma l 1} + \cdots + \overline{a}_{\nu k}\beta_{\gamma l \nu}\]
and
\[\beta_{\gamma l k} = \left(\overline{r}_{l1} \log\left| \left( \frac{\gamma_k^{(j)}}{\gamma_k^{(i_0)}}\right)^{\iota_1}\right|+ \overline{r}_{l2}\log\left| \left( \frac{\gamma_k^{(j)}}{\gamma_k^{(i_0)}}\right)^{\iota_2}\right|\right)\]
for $k = 1, \dots, \nu$.

Since $\frac{\delta_2}{\lambda}$ is a quotient of elements which are conjugate to one another, by taking the norm on $L$ of $\frac{\delta_2}{\lambda}$, we obtain $N\left(\frac{\delta_2}{\lambda}\right) = 1.$ On the other hand, by definition, we have 
\[1 = N\left(\frac{\delta_2}{\lambda}\right) = \prod_{\sigma: L \to \mathbb{C}} \sigma \left(\frac{\delta_2}{\lambda}\right).\]
Taking absolute values and logarithms, 
\[0 = \sum_{\sigma: L \to \mathbb{C}} \log\left|\sigma \left(\frac{\delta_2}{\lambda}\right)\right|\]
so that
\[-\log\left|\left(\frac{\delta_2}{\lambda}\right)^{\iota}\right| = -\log\left|\iota \left(\frac{\delta_2}{\lambda}\right)\right| = \sum_{\shortstack{\footnotesize $\sigma: L \to \mathbb{C} $\\ \footnotesize $ \sigma \neq \iota$}} \log\left|\sigma \left(\frac{\delta_2}{\lambda}\right)\right|.\]

Therefore,
\begin{align*}
|a_l| 	& = \left|\overline{r}_{l1}\log\left|\left(\frac{\delta_2}{\lambda}\right)^{\iota_1}\right| + \overline{r}_{l2}\log\left|\left(\frac{\delta_2}{\lambda}\right)^{\iota_2}\right| - \sum_{k=1}^{\nu}(u_k-r_k)\alpha_{\gamma l k}\right|\\
	& \leq |\overline{r}_{l1}|\left|\log\left|\left(\frac{\delta_2}{\lambda}\right)^{\iota_1}\right|\right| + |\overline{r}_{l2}|\left|\log\left|\left(\frac{\delta_2}{\lambda}\right)^{\iota_2}\right|\right| + \sum_{k=1}^{\nu}|u_k-r_k||\alpha_{\gamma l k}|.
\end{align*}

Suppose $\log\left|\left(\frac{\delta_2}{\lambda}\right)^{\iota_1}\right| \geq 0$ and $\log\left|\left(\frac{\delta_2}{\lambda}\right)^{\iota_2}\right| \geq 0$. Then, 
\begin{align*}
|a_l| 	& \leq |\overline{r}_{l1}|\left|\log\left|\left(\frac{\delta_2}{\lambda}\right)^{\iota_1}\right|\right| + |\overline{r}_{l2}|\left|\log\left|\left(\frac{\delta_2}{\lambda}\right)^{\iota_2}\right|\right| + \sum_{k=1}^{\nu}|u_k-r_k||\alpha_{\gamma l k}|\\
	& = |\overline{r}_{l1}|\log\left|\left(\frac{\delta_2}{\lambda}\right)^{\iota_1}\right| + |\overline{r}_{l2}|\log\left|\left(\frac{\delta_2}{\lambda}\right)^{\iota_2}\right| + \sum_{k=1}^{\nu}|u_k-r_k||\alpha_{\gamma l k}|\\
	& \leq \max\{|\overline{r}_{l1}|, |\overline{r}_{l2}|\}|\log\max\left\{\left|\left(\frac{\delta_2}{\lambda}\right)^{\iota_1}\right|,1\right\} + \\
	& \quad \quad +\max\{|\overline{r}_{l1}|, |\overline{r}_{l2}|\}\log\max\left\{\left|\left(\frac{\delta_2}{\lambda}\right)^{\iota_2}\right|,1\right\} + \sum_{k=1}^{\nu}|u_k-r_k||\alpha_{\gamma l k}|\\
	&  \leq \sum_{w: L \to \mathbb{C}}\max\{|\overline{r}_{l1}|, |\overline{r}_{l2}|\}|\log\max\left\{\left|w\left(\frac{\delta_2}{\lambda}\right)\right|,1\right\} + \sum_{k=1}^{\nu}|u_k-r_k||\alpha_{\gamma l k}|.
\end{align*}

Alternatively, suppose that both $\log\left|\left(\frac{\delta_2}{\lambda}\right)^{\iota_1}\right| < 0$ and $\log\left|\left(\frac{\delta_2}{\lambda}\right)^{\iota_2}\right| < 0$. In this case,
\begin{align*}
|a_l| 	& \leq |\overline{r}_{l1}|\left|\log\left|\left(\frac{\delta_2}{\lambda}\right)^{\iota_1}\right|\right| + |\overline{r}_{l2}|\left|\log\left|\left(\frac{\delta_2}{\lambda}\right)^{\iota_2}\right|\right| + \sum_{k=1}^{\nu}|u_k-r_k||\alpha_{\gamma l k}|\\
	& = |\overline{r}_{l1}|\sum_{\shortstack{\footnotesize $\sigma: L \to \mathbb{C} $\\ \footnotesize $ \sigma \neq \iota_1$}} \log\left|\sigma \left(\frac{\delta_2}{\lambda}\right)\right|+ |\overline{r}_{l2}|\left(-\log\left|\left(\frac{\delta_2}{\lambda}\right)^{\iota_2}\right|\right) + \sum_{k=1}^{\nu}|u_k-r_k||\alpha_{\gamma l k}|\\
	& \leq \max\{|\overline{r}_{l1}|, |\overline{r}_{l2}|\}\sum_{\shortstack{\footnotesize $\sigma: L \to \mathbb{C} $\\ \footnotesize $ \sigma \neq \iota_1$}} \log\left|\sigma \left(\frac{\delta_2}{\lambda}\right)\right| +\\
	& \quad \quad +\max\{|\overline{r}_{l1}|, |\overline{r}_{l2}|\}\left(-\log\left|\left(\frac{\delta_2}{\lambda}\right)^{\iota_2}\right|\right) + \sum_{k=1}^{\nu}|u_k-r_k||\alpha_{\gamma l k}|\\
	& \leq \max\{|\overline{r}_{l1}|, |\overline{r}_{l2}|\}\sum_{\shortstack{\footnotesize $w: L \to \mathbb{C} $\\ \footnotesize $ \sigma \neq \iota_1, \iota_2$}} \log\left|\sigma \left(\frac{\delta_2}{\lambda}\right)\right|  + \sum_{k=1}^{\nu}|u_k-r_k||\alpha_{\gamma l k}|\\
	&  \leq \sum_{w: L \to \mathbb{C}}\max\{|\overline{r}_{l1}|, |\overline{r}_{l2}|\}|\log\max\left\{\left|w\left(\frac{\delta_2}{\lambda}\right)\right|,1\right\} + \sum_{k=1}^{\nu}|u_k-r_k||\alpha_{\gamma l k}|.
\end{align*}

Lastly, if, without loss of generality, we have $\log\left|\left(\frac{\delta_2}{\lambda}\right)^{\iota_1}\right| < 0$ and $\log\left|\left(\frac{\delta_2}{\lambda}\right)^{\iota_2}\right| \geq 0$, then
\begin{align*}
|a_l| 	& \leq |\overline{r}_{l1}|\left|\log\left|\left(\frac{\delta_2}{\lambda}\right)^{\iota_1}\right|\right| + |\overline{r}_{l2}|\left|\log\left|\left(\frac{\delta_2}{\lambda}\right)^{\iota_2}\right|\right| + \sum_{k=1}^{\nu}|u_k-r_k||\alpha_{\gamma l k}|\\
	& = |\overline{r}_{l1}|\sum_{\shortstack{\footnotesize $\sigma: L \to \mathbb{C} $\\ \footnotesize $ \sigma \neq \iota_1$}} \log\left|\sigma \left(\frac{\delta_2}{\lambda}\right)\right|+ |\overline{r}_{l2}|\log\left|\left(\frac{\delta_2}{\lambda}\right)^{\iota_2}\right| + \sum_{k=1}^{\nu}|u_k-r_k||\alpha_{\gamma l k}|\\
	& \leq \max\{|\overline{r}_{l1}|, |\overline{r}_{l2}|\}\sum_{w: L \to \mathbb{C}} \log \max \left\{\left|w\left(\frac{\delta_2}{\lambda}\right)\right|, 1\right\} +  |\overline{r}_{l2}|\log\left|\left(\frac{\delta_2}{\lambda}\right)^{\iota_2}\right| +\sum_{k=1}^{\nu}|u_k-r_k||\alpha_{\gamma l k}|\\
	& = \sum_{w: L \to \mathbb{C}}\max\{|\overline{r}_{l1}|, |\overline{r}_{l2}|\}|\log\max\left\{\left|w\left(\frac{\delta_2}{\lambda}\right)\right|,1\right\} + |\overline{r}_{l1}|\log\max\left\{\left|\iota_1\left(\frac{\delta_2}{\lambda}\right)\right|,1\right\}+ \\
	&\quad\quad + |\overline{r}_{l2}|\log\max\left\{\left|\iota_2\left(\frac{\delta_2}{\lambda}\right)\right|,1\right\}+ \sum_{k=1}^{\nu}|u_k-r_k||\alpha_{\gamma l k}|.
\end{align*}
In all three cases, it follows that
\begin{align*}
|a_l|	& \leq \sum_{w: L \to \mathbb{C}}\max\{|\overline{r}_{l1}|, |\overline{r}_{l2}|\}|\log\max\left\{\left|w\left(\frac{\delta_2}{\lambda}\right)\right|,1\right\} + |\overline{r}_{l1}|\log\max\left\{\left|\iota_1\left(\frac{\delta_2}{\lambda}\right)\right|,1\right\} + \\
	&\quad\quad + |\overline{r}_{l2}|\log\max\left\{\left|\iota_2\left(\frac{\delta_2}{\lambda}\right)\right|,1\right\}+ \sum_{k=1}^{\nu}|u_k-r_k||\alpha_{\gamma l k}|
\end{align*}
where
\[\alpha_{\gamma l k} = \overline{a}_{1k}\beta_{\gamma l 1} + \cdots + \overline{a}_{\nu k}\beta_{\gamma l \nu}\]
and
\[\beta_{\gamma l k} = \left(\overline{r}_{l1} \log\left| \left( \frac{\gamma_k^{(j)}}{\gamma_k^{(i_0)}}\right)^{\iota_1}\right|+ \overline{r}_{l2}\log\left| \left( \frac{\gamma_k^{(j)}}{\gamma_k^{(i_0)}}\right)^{\iota_2}\right|\right)\]
for $k = 1, \dots, \nu$.

Hence for $l = 1,2$, we write
\begin{equation} \label{eq:albound}
|a_l|	\leq \frac{1}{[L:\mathbb{Q}]}\sum_{\sigma :L \to \mathbb{C}} w_{\varepsilon l \sigma}\log \max \left\{ \left|\sigma\left(\frac{\delta_2}{\lambda}\right)\right|, 1\right\} + \frac{1}{[K:\mathbb{Q}]}\sum_{k = 1}^{\nu} w_{\gamma l k}\log(p_k)|u_k - r_k|,
\end{equation}
where
\begin{equation} \label{eq:welsigma}
w_{\varepsilon l \sigma} = 
\begin{cases}
\max\{|\overline{r}_{l1}|, |\overline{r}_{l2}|\}[L:\mathbb{Q}] & \text{ for } \sigma \notin I\\
\left(\max\{|\overline{r}_{l1}|, |\overline{r}_{l2}|\} + |\overline{r}_{li}|\right)[L:\mathbb{Q}] & \text{ for } \sigma = \iota_i \in I\\
\end{cases}
\end{equation}
and 
\begin{equation} \label{eq:wgammalk}
w_{\gamma l k} = |\alpha_{\gamma l k}|\frac{[K:\mathbb{Q}]}{\log(p_k)}.
\end{equation}

To summarize, we have proven the following lemma.
\begin{lemma}\label{lem:mepsbound}
For any solution $(x,y,a_1, \dots, a_r, n_1, \dots, n_{\nu})$ of \eqref{eq:EfficientSunit}, for $l =1,2$
\[|a_l| \leq \frac{1}{[L:\mathbb{Q}]}\sum_{\sigma :L \to \mathbb{C}} w_{\varepsilon l \sigma}\log \max \left\{ \left|\sigma\left(\frac{\delta_2}{\lambda}\right)\right|, 1\right\} + \frac{1}{[K:\mathbb{Q}]}\sum_{k = 1}^{\nu} w_{\gamma l k}\log(p_k)|u_k - r_k|.\]
\end{lemma}

%---------------------------------------------------------------------------------------------------------------------------------------------%

\subsection{The Archimedean ellipsoid: the real case}
\label{subsec:ArchEllipsoid}

Let $\tau:L\to\mathbb{R} \subset \mathbb{C}$ be an embedding and let $l_\tau\geq c_\tau$ and $c>0$ be given real numbers for $c_\tau=\log^+(2|\tau(\delta_2)|)= \log \max\{2|\tau(\delta_2)|,1\}$. We define 
\[\alpha_0 = [c\log|\tau(\delta_1)|] \quad \text{ and } \quad \alpha_{\varepsilon 1} =  \left[c\log\left|\tau\left(\frac{\varepsilon_1^{(k)}}{\varepsilon_1^{(j)}}\right)\right|\right],\ \  \alpha_{\varepsilon 2} =  \left[c\log\left|\tau\left(\frac{\varepsilon_2^{(k)}}{\varepsilon_2^{(j)}}\right)\right|\right].\]
For $i = 1, \dots, \nu$, define
\[\alpha_{\gamma i} = \left[c\log\left|\tau\left(\frac{\gamma_i^{(k)}}{\gamma_i^{(j)}}\right)\right|\right].\]
Here, $[ \ \cdot\  ]$ denotes the nearest integer function. 

Let
\begin{equation} \label{eq:wsigma}
w_{\sigma} = (w_{\varepsilon 1 \sigma} + w_{\varepsilon 2 \sigma}), \quad \quad w_k = (w_{\gamma 1 k} + w_{\gamma 2 k}) + \frac{[K:\mathbb{Q}]}{\log(p_k)}\sum_{i=1}^{\nu}|\overline{a}_{ik}|
\end{equation}
for $\sigma: L \to \mathbb{C}$ and $k = 1, \dots, \nu$. Here $w_{\eps 1 \sigma}, w_{\eps 3 \sigma}$ and $w_{\gamma 1 k}, w_{\gamma 3 k}$ are the coefficients \eqref{eq:welsigma} and \eqref{eq:wgammalk}, respectively. Let $\kappa_{\tau} = 3/2$ and 
\[h_{\tau}\left(\frac{\delta_2}{\lambda}\right) = \frac{1}{[L:K]}\log \max \left\{ \left|\tau\left(\frac{\delta_2}{\lambda}\right)\right|, 1\right\},\]
the local height at $\tau$ in the decomposition of $h\left(\frac{\delta_2}{\lambda}\right)$.

\begin{lemma}\label{lem:archellest}
Suppose $(x,y, n_1, \dots, n_{\nu}, a_1, \dots, a_r)$ is a solution of \eqref{eq:EfficientSunit}. If ${h_{\tau} \left(\frac{\delta_2}{\lambda}\right) > c_{\tau}}$, then  
\begin{align*}
&\left|\alpha_0+\sum_{i = 1}^r a_i \alpha_{\varepsilon i} + \sum_{i = 1}^{\nu} n_i \alpha_{\gamma i}\right|\\
	& \leq \frac{1}{2}\left(\frac{1}{[K:\mathbb{Q}]}\sum_{l = 1}^{\nu}w_l \log(p_l)|u_l - r_l| + \frac{1}{[L:\mathbb{Q}]}\sum_{\sigma :L \to \mathbb{C}} w_{\sigma}\log \max \left\{ \left|\sigma\left(\frac{\delta_2}{\lambda}\right)\right|, 1\right\} \right) + \\
	& \quad \quad \quad + \left(\frac{1}{2} + c\kappa_{\tau}e^{-h_{\tau}\left(\frac{\delta_2}{\lambda}\right)}\right).
\end{align*} 
\end{lemma}

\begin{proof}
Let 
\begin{align*}
\alpha	
	& = \alpha_0+\sum_{i = 1}^r a_i \alpha_{\varepsilon i} + \sum_{i = 1}^{\nu} n_i \alpha_{\gamma i}\\
	& = [c\log|\tau(\delta_1)|] +\sum_{i = 1}^r a_i \left[c\log\left|\tau\left(\frac{\varepsilon_i^{(k)}}{\varepsilon_i^{(j)}}\right)\right|\right] + \sum_{i = 1}^{\nu} n_i \left[c\log\left|\tau\left(\frac{\gamma_i^{(k)}}{\gamma_i^{(j)}}\right)\right|\right]
\end{align*}
and
\[\Lambda_{\tau} = \log\left|\tau\left(\delta_1 \prod_{i = 1}^r\left( \frac{\varepsilon_i^{(k)}}{\varepsilon_i^{(j)}}\right)^{a_i}\prod_{i = 1}^{\nu} \left( \frac{\gamma_i^{(k)}}{\gamma_i^{(j)}}\right)^{n_i}\right)\right|= \log\left(\tau\left(\delta_1 \prod_{i = 1}^r\left( \frac{\varepsilon_i^{(k)}}{\varepsilon_i^{(j)}}\right)^{a_i}\prod_{i = 1}^{\nu} \left( \frac{\gamma_i^{(k)}}{\gamma_i^{(j)}}\right)^{n_i}\right)\right) \]
where the above equality follows from 
\[\tau\left(\delta_1 \prod_{i = 1}^r\left( \frac{\varepsilon_i^{(k)}}{\varepsilon_i^{(j)}}\right)^{a_i}\prod_{i = 1}^{\nu} \left( \frac{\gamma_i^{(k)}}{\gamma_i^{(j)}}\right)^{n_i}\right) > 0.\]
Indeed, by assumption,
\[ h_{\tau}\left(\frac{\delta_2}{\lambda}\right) > c_\tau = \log \max\{2|\tau(\delta_2)|,1\},\]
so that 
\begin{align*}
\exp\left(h_{\tau}\left(\frac{\delta_2}{\lambda}\right)\right)	& > \exp(c_{\tau}) \\
\exp\left(\log \max \left\{ \left|\tau\left(\frac{\delta_2}{\lambda}\right)\right|, 1\right\}\right) & > \exp \left(\log \max\{2|\tau(\delta_2)|,1\}\right)\\
\max \left\{ \left|\tau\left(\frac{\delta_2}{\lambda}\right)\right|, 1\right\} & > \max\{2|\tau(\delta_2)|,1\}.
\end{align*}
From this last inequality, we must have that 
\[\max \left\{ \left|\tau\left(\frac{\delta_2}{\lambda}\right)\right|, 1\right\} = \left|\tau\left(\frac{\delta_2}{\lambda}\right)\right|.\]
In this case, 
\[\max\{2|\tau(\delta_2)|,1\} < \max \left\{ \left|\tau\left(\frac{\delta_2}{\lambda}\right)\right|, 1\right\} = \left|\tau\left(\frac{\delta_2}{\lambda}\right)\right|.\]
It follows that
\[2|\tau(\delta_2)| \leq \max\{2|\tau(\delta_2)|,1\} < \max \left\{ \left|\tau\left(\frac{\delta_2}{\lambda}\right)\right|, 1\right\} = \left|\tau\left(\frac{\delta_2}{\lambda}\right)\right|\]
and therefore
\[2|\tau(\delta_2)| < \left|\tau\left(\frac{\delta_2}{\lambda}\right)\right| = \frac{|\tau(\delta_2)|}{|\tau(\lambda)|} \implies |\tau(\lambda)| < \frac{1}{2}.\]
Recall that $\delta_1\tilde{y} - \delta_2\tilde{x} = 1$. This is equation \eqref{eq:TrueSunit} defined earlier. In particular, observe that $\lambda = \delta_2\tilde{x}$, so that applying $\tau$, we obtain
\[\tau(\lambda) = \tau(\delta_2\tilde{x}) = \tau(\delta_1\tilde{y}) - 1.\]
Thus
\[|\tau(\lambda)| < \frac{1}{2} \implies \tau(\delta_1\tilde{y}) = \tau(\lambda) + 1 > 0.\]
This proves that
\[\tau(\delta_1\tilde{y}) = \tau\left(\delta_1 \prod_{i = 1}^r\left( \frac{\varepsilon_i^{(k)}}{\varepsilon_i^{(j)}}\right)^{a_i}\prod_{i = 1}^{\nu} \left( \frac{\gamma_i^{(k)}}{\gamma_i^{(j)}}\right)^{n_i}\right) > 0.\]
Having established this, we write, 
\[\Lambda_{\tau} = \log\left|\tau\left(\delta_1 \prod_{i = 1}^r\left( \frac{\varepsilon_i^{(k)}}{\varepsilon_i^{(j)}}\right)^{a_i}\prod_{i = 1}^{\nu} \left( \frac{\gamma_i^{(k)}}{\gamma_i^{(j)}}\right)^{n_i}\right)\right|= \log\left(\tau\left(\delta_1 \prod_{i = 1}^r\left( \frac{\varepsilon_i^{(k)}}{\varepsilon_i^{(j)}}\right)^{a_i}\prod_{i = 1}^{\nu} \left( \frac{\gamma_i^{(k)}}{\gamma_i^{(j)}}\right)^{n_i}\right)\right) \]
or equivalently, 
\begin{align*}
\Lambda_{\tau}	
	& = \log\left(\tau\left(\delta_1\right)\right) + \sum_{i=1}^r a_i\log\left(\tau\left( \frac{\varepsilon_i^{(k)}}{\varepsilon_i^{(j)}}\right) \right) + \sum_{i=1}^{\nu}n_i\log \left(\tau\left( \frac{\gamma_i^{(k)}}{\gamma_i^{(j)}}\right)\right).
\end{align*}
By the triangle inequality, 
\[|\alpha| \leq |\alpha - c\Lambda_{\tau}| + c|\Lambda_{\tau}|,\]
where
\begin{align*}
|\alpha-c\Lambda_\tau|
	& = \left|[c\log(\tau(\delta_1))] +\sum_{i = 1}^r a_i \left[c\log\left(\tau\left(\frac{\varepsilon_i^{(k)}}{\varepsilon_i^{(j)}}\right)\right)\right] + \sum_{i = 1}^{\nu} n_i \left[c\log\left(\tau\left(\frac{\gamma_i^{(k)}}{\gamma_i^{(j)}}\right)\right)\right]\right. +\\
	& \quad \quad \left. - c \log\left(\tau\left(\delta_1 \prod_{i = 1}^r\left( \frac{\varepsilon_i^{(k)}}{\varepsilon_i^{(j)}}\right)^{a_i}\prod_{i = 1}^{\nu} \left( \frac{\gamma_i^{(k)}}{\gamma_i^{(j)}}\right)^{n_i}\right)\right)\right|\\
	& \leq \left| [c\log(\tau(\delta_1))] - c\log\left(\tau\left(\delta_1\right)\right)\right| + \sum_{i = 1}^r |a_i|\left| \left[c\log\left(\tau\left(\frac{\varepsilon_i^{(k)}}{\varepsilon_i^{(j)}}\right)\right)\right] - c\log\left(\tau\left( \frac{\varepsilon_i^{(k)}}{\varepsilon_i^{(j)}}\right) \right)\right| \\
	& \quad \quad +  \sum_{i = 1}^{\nu} |n_i|\left| \left[c\log\left(\tau\left(\frac{\gamma_i^{(k)}}{\gamma_i^{(j)}}\right)\right)\right] - c\log \left(\tau\left( \frac{\gamma_i^{(k)}}{\gamma_i^{(j)}}\right)\right)\right|.
\end{align*}
Since $[ \ \cdot \ ]$ denotes the nearest integer function, it is clear that $|[ \ c \ ] - c| \leq 1/2$ for any integer $c$, 
\begin{align*}
|\alpha-c\Lambda_\tau|
	& \leq \frac{1}{2} + \frac{1}{2}\sum_{i = 1}^r |a_i| + \frac{1}{2}\sum_{i = 1}^{\nu} |n_i|\\
	& \leq \frac{1}{2}\left(1 + \sum_{i = 1}^r |a_i| + |u_1-r_1|\sum_{i=1}^{\nu}|\overline{a}_{i1}| + \cdots + |u_{\nu} - r_{\nu}| \sum_{i=1}^{\nu}|\overline{a}_{i\nu}|\right).
\end{align*}
By Lemma~\ref{lem:mepsbound}, 
\[|a_l| \leq \frac{1}{[L:\mathbb{Q}]}\sum_{\sigma :L \to \mathbb{C}} w_{\varepsilon l \sigma}\log \max \left\{ \left|\sigma\left(\frac{\delta_2}{\lambda}\right)\right|, 1\right\} + \frac{1}{[K:\mathbb{Q}]}\sum_{k = 1}^{\nu} w_{\gamma l k}\log(p_k)|u_k - r_k|\]
for $l = 1,2$. Applying this result to $|\alpha-c\Lambda_\tau|$, we obtain
\begin{align*}
|\alpha-c\Lambda_\tau| 
	& \leq \frac{1}{2} + \frac{1}{2}|u_1-r_1|\sum_{i=1}^{\nu}|\overline{a}_{i1}| + \cdots + \frac{1}{2}|u_{\nu} - r_{\nu}| \sum_{i=1}^{\nu}|\overline{a}_{i\nu}| + \\
	& \quad + \frac{1}{2}\left(\frac{1}{[L:\mathbb{Q}]}\sum_{\sigma :L \to \mathbb{C}} w_{\varepsilon_1 \sigma}\log \max \left\{ \left|\sigma\left(\frac{\delta_2}{\lambda}\right)\right|, 1\right\} + \frac{1}{[K:\mathbb{Q}]}\sum_{k = 1}^{\nu} w_{\gamma_1 k}\log(p_k)|u_k - r_k| \right) + \\
	&\quad + \frac{1}{2}\left(\frac{1}{[L:\mathbb{Q}]}\sum_{\sigma :L \to \mathbb{C}} w_{\varepsilon_2 \sigma}\log \max \left\{ \left|\sigma\left(\frac{\delta_2}{\lambda}\right)\right|, 1\right\} + \frac{1}{[K:\mathbb{Q}]}\sum_{k = 1}^{\nu} w_{\gamma_2 k}\log(p_k)|u_k - r_k| \right) \\
	& = \frac{1}{2} + \frac{1}{[L:\mathbb{Q}]}\sum_{\sigma :L \to \mathbb{C}} \frac{(w_{\varepsilon_1 \sigma} + w_{\varepsilon_2 \sigma})}{2}\log \max \left\{ \left|\sigma\left(\frac{\delta_2}{\lambda}\right)\right|, 1\right\} + \\
	& \quad + |u_1 - r_1|\left( \frac{(w_{\gamma_1 1} + w_{\gamma_2 1})}{2[K:\mathbb{Q}]}\log(p_1) + \frac{1}{2}\sum_{i=1}^{\nu}|\overline{a}_{i1}|\right) + \cdots \\
	& \quad + |u_{\nu} - r_{\nu}|\left( \frac{(w_{\gamma_1 {\nu}} + w_{\gamma_2 {\nu}})}{2[K:\mathbb{Q}]}\log(p_{\nu}) + \frac{1}{2}\sum_{i=1}^{\nu}|\overline{a}_{i{\nu}}|\right). 
\end{align*}
Altogether, we have
\begin{align*}
|\alpha-c\Lambda_\tau| 
	& \leq \frac{1}{2} + \frac{1}{2[L:\mathbb{Q}]}\sum_{\sigma :L \to \mathbb{C}} (w_{\varepsilon_1 \sigma} + w_{\varepsilon_2 \sigma})\log \max \left\{ \left|\sigma\left(\frac{\delta_2}{\lambda}\right)\right|, 1\right\} + \\
	& \quad + \frac{1}{2[K:\mathbb{Q}]}\sum_{k = 1}^{\nu} \log(p_k)|u_k - r_k|\left( (w_{\gamma_1 k} + w_{\gamma_2 k}) + \frac{[K:\mathbb{Q}]}{\log(p_k)}\sum_{i=1}^{\nu}|\overline{a}_{ik}|\right).
\end{align*}
Finally, using the notation of \eqref{eq:wsigma}, this inequality reduces to
\begin{align*}
|\alpha-c\Lambda_\tau|
	& \leq \frac{1}{2} + \frac{1}{2[L:\mathbb{Q}]}\sum_{\sigma :L \to \mathbb{C}} w_{\sigma}\log \max \left\{ \left|\sigma\left(\frac{\delta_2}{\lambda}\right)\right|, 1\right\} + \frac{1}{2[K:\mathbb{Q}]}\sum_{l = 1}^{\nu}w_l \log(p_l)|u_l - r_l|.
\end{align*}

Now the following upper bound for $|\Lambda_\tau|$ implies the statement. On using power series definition of exponential function, we obtain $$\Lambda_\tau(1+\sum_{n\geq 2} (\Lambda_\tau)^{n-1}/n!)=\Lambda_\tau+\sum_{n\geq 2} (\Lambda_\tau)^n/n!=e^{\Lambda_\tau}-1=\tau(\lambda).$$
If $\Lambda_\tau\geq 0$ then $1+\sum_{n\geq 2} (\Lambda_\tau)^{n-1}/n!>1$ which implies that $|\Lambda_\tau|\leq |\tau(\lambda)|.$ Suppose now that $\Lambda_\tau<0$. Our assumption $h_{\tau}(z)\geq \log^+ (2|\lambda_0|)$ means that $|\tau(\lambda)|\leq 1/2$ and thus $|\Lambda_\tau|=-\log (\tau(\lambda)+1)\leq -\log(1/2)=\log 2.$ Therefore, the absolute value of $\sum_{n\geq 2} (\Lambda_\tau)^{n-1}/n!$ is at most $$\sum_{n\geq 2} |\Lambda_\tau|^{n-1}/n!=\sum_{n\geq 1} |\Lambda_\tau|^{n}/(n+1)!\leq \tfrac{1}{2}\sum_{n\geq 1} |\Lambda_\tau|^{n}/n!\leq
\tfrac{1}{2}e^{\log 2}-1/2=1/2.$$
More precisely, for any even $N\geq 2$, we obtain 
$$|\sum_{n\geq 2} (\Lambda_\tau)^{n-1}/n!|=|\sum_{n\geq 1} (\Lambda_\tau)^{n}/(n+1)!|\leq |\sum_{N\geq n\geq 1} (\Lambda_\tau)^{n}/(n+1)!|+\tfrac{1}{N+2}|\sum_{n>N} (\Lambda_\tau)^{n}/n!|$$
$$\leq |\sum_{N\geq n\geq 1} (\Lambda_\tau)^{n}/(n+1)!|+\tfrac{1}{N+2}e^{|\Lambda_\tau|}\leq |\sum_{N\geq n\geq 1} (\Lambda_\tau)^{n}/(n+1)!|+\tfrac{2}{N+2}:=k_N.$$
We now give an upper bound for $k_N$. Since $\Lambda_\tau<0$, we obtain
$$\sum_{n\geq 2} (\Lambda_\tau)^{n-1}/n!=\sum_{N\geq n\geq 1} (\Lambda_\tau)^{n}/(n+1)!=\sum_{N\geq n\geq 2, \, 2\mid n}\tfrac{|\Lambda_\tau|^n}{(n+1)!}-\tfrac{|\Lambda_\tau|^{n-1}}{n!}$$
$$=\sum_{N\geq n\geq 2, \, 2\mid n}\tfrac{|\Lambda_\tau|^{n-1}}{n!}(\tfrac{|\Lambda_\tau|}{n+1}-1)=\tfrac{|\Lambda_\tau|}{2}(\tfrac{|\Lambda_\tau|}{3}-1)+\sum_{N\geq n\geq 4, \, 2\mid n}\tfrac{|\Lambda_\tau|^{n-1}}{n!}(\tfrac{|\Lambda_\tau|}{n+1}-1)$$
$$\geq \tfrac{\log 2}{2}(\tfrac{\log 2}{4}-1)+\sum_{N\geq n\geq 4, \, 2\mid n}\tfrac{(\log 2)^{n-1}}{n!}(\tfrac{3/4(\log 2)}{n+1}-1):=-k_N.$$
The last inequality follows by distinguishing two cases whether  $|\Lambda_\tau|\leq 3/4\cdot \log 2$ or not; note that $\ln(2)/2\cdot(\ln(2)/4-1)/(-\ln (2)\cdot3/8)\geq 1$.  Now, on using that $-k_N$ is negative, it follows that $|1+\sum_{n\geq 2} (\Lambda_\tau)^{n-1}/n!|\geq 1-|\sum_{n\geq 2} (\Lambda_\tau)^{n-1}/n!|\geq 1-k_N$ and thus 
$$|\Lambda_\tau|\leq \kappa_\tau|\tau(x)|, \quad \kappa_\tau=\tfrac{1}{1-k_N}|\tau(\lambda_0)|,  \quad c_\tau=\log^+(2|\lambda_0|).$$
The constant $\kappa_\tau$ depends on $N$ which can be taken arbitrarily as long as $N\geq 2$ is even. Further, the value $k_N$ can be slightly improved when one finds the maximum of the functions $x^{n-1}(\tfrac{x}{n+1}-1)$ on the interval $[0,\log 2]$ for each even $n\geq 2$. This is our reason for taking $\kappa_{\tau} = \frac{3}{2}$. Currently this is not the optimal choice of $\kappa_{\tau}$, but it suffices for our present case. 

Finally, we now have
\begin{align*}
&\left|\alpha_0+\sum_{i = 1}^r a_i \alpha_{\varepsilon i} + \sum_{i = 1}^{\nu} n_i \alpha_{\gamma i}\right|\\
	& \leq \frac{1}{2}\left(\frac{1}{[L:\mathbb{Q}]}\sum_{\sigma :L \to \mathbb{C}} w_{\sigma}\log \max \left\{ \left|\sigma\left(\frac{\delta_2}{\lambda}\right)\right|, 1\right\} + \frac{1}{[K:\mathbb{Q}]}\sum_{l = 1}^{\nu}w_l \log(p_l)|u_l - r_l|\right) + \\
	& \quad \quad \quad + \left(\frac{1}{2} + c\kappa_{\tau}e^{-h_{\tau}\left(\frac{\delta_2}{\lambda}\right)}\right).
\end{align*} 

%
%If $v=p$ then Lemma~\ref{} gives optimal $c_v$ and $\kappa_v$. In the real case, if $v=\tau:L\to \mathbb{C}$ then we can take $\kappa_\tau$ as defined in \eqref{} and $c_v=\log 2$.
%
%Let now  $v=\tau$. Suppose first that we are in the real case. It holds that $\mu-1=\lambda$ and $\Lambda_\tau=\log \tau (\mu)$. Then, on using power series definition of exponential function, we obtain $$\Lambda_\tau(1+\sum_{n\geq 2} (\Lambda_\tau)^{n-1}/n!)=\Lambda_\tau+\sum_{n\geq 2} (\Lambda_\tau)^n/n!=e^{\Lambda_\tau}-1=\tau(\lambda).$$
%If $\Lambda_\tau\geq 0$ then $1+\sum_{n\geq 2} (\Lambda_\tau)^{n-1}/n!>1$ which implies that $|\Lambda_\tau|\leq |\tau(\lambda)|.$ Suppose now that $\Lambda_\tau<0$. Our assumption $h_v(z)\geq \log 2$ means that $|\tau(z)|\leq 1/2$ and thus $|\Lambda_\tau|=-\log (\tau(z)+1)\leq -\log(1/2)=\log 2.$ Therefore, the absolute value of $\sum_{n\geq 2} (\Lambda_\tau)^{n-1}/n!$ is at most $$\sum_{n\geq 2} |\Lambda_\tau|^{n-1}/n!=\sum_{n\geq 1} |\Lambda_\tau|^{n}/(n+1)!\leq \tfrac{1}{2}\sum_{n\geq 1} |\Lambda_\tau|^{n}/n!\leq
%\tfrac{1}{2}e^{\log 2}-1/2=1/2.$$
%More precisely, for any even $N\geq 2$, we obtain 
%$$|\sum_{n\geq 2} (\Lambda_\tau)^{n-1}/n!|=|\sum_{n\geq 1} (\Lambda_\tau)^{n}/(n+1)!|\leq |\sum_{N\geq n\geq 1} (\Lambda_\tau)^{n}/(n+1)!|+\tfrac{1}{N+2}|\sum_{n>N} (\Lambda_\tau)^{n}/n!|$$
%$$\leq |\sum_{N\geq n\geq 1} (\Lambda_\tau)^{n}/(n+1)!|+\tfrac{1}{N+2}e^{|\Lambda_\tau|}\leq |\sum_{N\geq n\geq 1} (\Lambda_\tau)^{n}/(n+1)!|+\tfrac{2}{N+2}:=k_N.$$
%We now give an upper bound for $k_N$. Since $\Lambda_\tau<0$, we obtain
%$$\sum_{n\geq 2} (\Lambda_\tau)^{n-1}/n!=\sum_{N\geq n\geq 1} (\Lambda_\tau)^{n}/(n+1)!=\sum_{N\geq n\geq 2, \, 2\mid n}\tfrac{|\Lambda_\tau|^n}{(n+1)!}-\tfrac{|\Lambda_\tau|^{n-1}}{n!}$$
%$$=\sum_{N\geq n\geq 2, \, 2\mid n}\tfrac{|\Lambda_\tau|^{n-1}}{n!}(\tfrac{|\Lambda_\tau|}{n+1}-1)=\tfrac{|\Lambda_\tau|}{2}(\tfrac{|\Lambda_\tau|}{3}-1)+\sum_{N\geq n\geq 4, \, 2\mid n}\tfrac{|\Lambda_\tau|^{n-1}}{n!}(\tfrac{|\Lambda_\tau|}{n+1}-1)$$
%$$\geq \tfrac{\log 2}{2}(\tfrac{\log 2}{4}-1)+\sum_{N\geq n\geq 4, \, 2\mid n}\tfrac{(\log 2)^{n-1}}{n!}(\tfrac{3/4(\log 2)}{n+1}-1):=-k_N.$$
%The last inequality follows by distinguishing two cases whether  $|\Lambda_\tau|\leq 3/4\cdot \log 2$ or not; note that $ln(2)/2*(ln(2)/4-1)/(-ln (2)*3/8)\geq 1$.  Now, on using that $-k_N$ is negative, it follows that $|1+\sum_{n\geq 2} (\Lambda_\tau)^{n-1}/n!|\geq 1-|\sum_{n\geq 2} (\Lambda_\tau)^{n-1}/n!|\geq 1-k_N$ and thus 
%$$|\Lambda_\tau|\leq \kappa_\tau|\tau(z)|, \quad \kappa_\tau=\tfrac{1}{1-k_N},  \quad c_\tau=\log 2.$$
%The constant $\kappa_\tau$ depends on $N$ which can be taken arbitrarily as long as $N\geq 2$ is even. Further, the value $k_N$ can be slightly improved when one finds the maximum of the functions $x^{n-1}(\tfrac{x}{n+1}-1)$ on the interval $[0,\log 2]$ for each even $n\geq 2$. 
\end{proof}

To summarize the results of this section, let $\mathbf{m} = (n_1, \dots, n_{\nu}, a_1, \dots, a_r) \in \mathbb{R}^{r + \nu}$ be any solution of \eqref{eq:EfficientSunit} with corresponding vector $\mathbf{n} = (n_1, \dots, n_{\nu})$. Take $\mathbf{l},\mathbf{h}\in\mathbb{R}^{\nu+m}$ such that $\mathbf{0} \leq \mathbf{l} \leq \mathbf{h}$ and suppose $h_v(z)\leq h_v$ for all $v\in S^*$. By Lemma~\ref{lem:boundqf}, we deduce
\begin{equation} \label{def:bbound}
\log(2)^2q_f(\mathbf{n}) \leq \sum_{k = 1}^{\nu} \log(p_k)^2|u_k -r_k|^2 \leq \sum_{k = 1}^{\nu} h_k^2=:b.
\end{equation}
For $l = 1, 2$, Lemma~\ref{lem:mepsbound} gives us
\begin{align}\label{def:bepsbound}
|a_l|^2 &\leq \left( \frac{1}{[K:\mathbb{Q}]}\sum_{k = 1}^{\nu} w_{\gamma l k}\log(p_k)|u_k - r_k| + \frac{1}{[L:\mathbb{Q}]}\sum_{\sigma :L \to \mathbb{C}} w_{\varepsilon l \sigma}\log \max \left\{ \left|\sigma\left(\frac{\delta_2}{\lambda}\right)\right|, 1\right\} \right)^2\\
	&\leq \left( \frac{1}{[K:\mathbb{Q}]}\sum_{k = 1}^{\nu} w_{\gamma l k}h_k + \frac{1}{[L:\mathbb{Q}]}\sum_{\sigma:L\to \mathbb{C}} w_{\varepsilon l \sigma}h_{\sigma}\right)^2 =: b_{\eps_l}.
\end{align}
Finally, suppose in addition that
\[h_{\tau}\left(\frac{\delta_2}{\lambda}\right) \geq l_{\tau} > c_{\tau}.\]
Then by Lemma~\ref{lem:archellest}, we obtain
\begin{align*}
&\left|\alpha_0+\sum_{i = 1}^r a_i \alpha_{\varepsilon i} + \sum_{i = 1}^{\nu} n_i \alpha_{\gamma i}\right|\\
	& \leq \frac{1}{2}\left(\frac{1}{[K:\mathbb{Q}]}\sum_{l = 1}^{\nu}w_l \log(p_l)|u_l - r_l| + \frac{1}{[L:\mathbb{Q}]}\sum_{\sigma :L \to \mathbb{C}} w_{\sigma}\log \max \left\{ \left|\sigma\left(\frac{\delta_2}{\lambda}\right)\right|, 1\right\} \right) + \\
	& \quad \quad \quad + \left(\frac{1}{2} + c\kappa_{\tau}e^{-h_{\tau}\left(\frac{\delta_2}{\lambda}\right)}\right)\\
	& \leq \frac{1}{2}\left(\frac{1}{[K:\mathbb{Q}]}\sum_{l = 1}^{\nu}w_l h_l + \frac{1}{[L:\mathbb{Q}]}\sum_{\sigma :L \to \mathbb{C}} w_{\sigma}h_{\sigma} \right) + \frac{1}{2} + c\kappa_{\tau}e^{-l_{\tau}}, 
\end{align*} 
and we set
\begin{equation} \label{bepstarbound}
b_{\eps_l^*}:= \frac{1}{2}\left(\frac{1}{[K:\mathbb{Q}]}\sum_{l = 1}^{\nu}w_l h_l + \frac{1}{[L:\mathbb{Q}]}\sum_{\sigma :L \to \mathbb{C}} w_{\sigma}h_{\sigma} \right) + \frac{1}{2} + c\kappa_{\tau}e^{-l_{\tau}}.
\end{equation}

It is of particular importance to note that the assumptions $h_{\tau}(z) \geq l_{\tau}$ and  $h_v(z)\leq h_v$ for all $v\in S^*$ are not arbitrary. Indeed, for the vectors $\mathbf{l}, \mathbf{h}$, these conditions imply precisely that $(\tilde{x},\tilde{y}) \in \Sigma_{\tau}(\mathbf{l}, \mathbf{h})$, where $(\tilde{x},\tilde{y})$ are solutions to \eqref{eq:EfficientSunit} corresponding to $\mathbf{m}$.


%
%\begin{align*}
%& \left|\alpha_0+\sum_{i = 1}^r a_i \alpha_{\varepsilon i} + \sum_{i = 1}^{\nu} n_i \alpha_{\gamma i}\right| ^2 \\ & \leq
%\begin{cases}
%\left( \frac{2}{[L:\mathbb{Q}]}\max_{\sigma:L\to \mathbb{C}} w_{\sigma}h_{\sigma}  + \frac{1}{[K:\mathbb{Q}]}\sum_{k = 1}^{\nu} w_{k}h_k + \frac{1}{2} + c\kappa_{\tau}e^{-l_{\tau}}\right)^2 & \text{ if } \sqrt{\Delta}\notin\mathbb{Q} \\
%\left( \frac{1}{[L:\mathbb{Q}]}\max_{\sigma:L\to \mathbb{C}} w_{\sigma}h_{\sigma}  + \frac{1}{[K:\mathbb{Q}]}\sum_{k = 1}^{\nu} w_{k}h_k + \frac{1}{2} + c\kappa_{\tau}e^{-l_{\tau}}\right)^2 & \text{ if } \sqrt{\Delta}\in\mathbb{Q}.\\
%\end{cases}
%\end{align*}

%We now fix $\epsilon^*\in\unit_\infty$ and we take $h\in\RR^{S^*}$ with $h\geq 0$.  Let $(x,y)\in\Sigma$ with $m\in\ZZ^\unit$ and $h_\tau(z)\geq l_\tau$, and suppose that $h_v(z)\leq h_v$ for all $v\in S^*$. Then, on combining Lemma~\ref{lem:archellest} with \eqref{eq:hvjbound}, we see that  $\left|\alpha_0+\sum_{u\in\unit}m_u\alpha_u\right|^2$ is at most
%\begin{equation}\label{def:bepsstarbound}
%b_{\epsilon^*}=\left(\sum_{v:L\to\CC}w_{v}h_v+\sum_{v\in T}\max\bigl(w_{v}h_v,w_{v^{(j)}}a_p\log N(v^{(j)})\bigl)+c\kappa_\tau e^{-l_{\tau}}\right)^2.
%\end{equation}
%By Remark~\ref{rem:i12}, we can replace $\sum_{v:L\to\CC}w_{v}h_v$ by $3[L:K]\|r_\epsilon\|_\infty\max_{v:L\to\CC}h_v$ if $|I|=2$.
%

We are finally in position to define the ellipsoid corresponding to $\Sigma_{\tau}(\mathbf{l}, \mathbf{h})$. Fix any $\eps_l^* \in \{\varepsilon_1, \dots, \varepsilon_r\}$. 
For each $\varepsilon_l$ in $\{\varepsilon_1, \dots, \varepsilon_r\}$ such that $\varepsilon_l \neq \varepsilon_l^*$, we associate the bound $b_{\eps_l}$. For $\varepsilon_l$, we associate the value $b_{\eps_l^*}$.
%
%\[b_{\varepsilon_l} = 
%\begin{cases}
%\left( \frac{2}{[L:\mathbb{Q}]}\max_{\sigma:L\to \mathbb{C}} w_{\sigma}h_{\sigma}  + \frac{1}{[K:\mathbb{Q}]}\sum_{k = 1}^{\nu} w_{k}h_k + \frac{1}{2} + c\kappa_{\tau}e^{-l_{\tau}}\right)^2 & \text{ if } \sqrt{\Delta}\notin\mathbb{Q} \\
%\left( \frac{1}{[L:\mathbb{Q}]}\max_{\sigma:L\to \mathbb{C}} w_{\sigma}h_{\sigma}  + \frac{1}{[K:\mathbb{Q}]}\sum_{k = 1}^{\nu} w_{k}h_k + \frac{1}{2} + c\kappa_{\tau}e^{-l_{\tau}}\right)^2 & \text{ if } \sqrt{\Delta}\in\mathbb{Q}\\
%\end{cases}\]
%where
%\begin{align*}
%\left|\alpha_0+\sum_{i = 1}^r a_i \alpha_{\varepsilon i} + \right.& \left.\sum_{i = 1}^{\nu} n_i \alpha_{\gamma i}\right| ^2 
%	\leq b_{\varepsilon_l} \\
%\\ & =
%\begin{cases}
%\left( \frac{2}{[L:\mathbb{Q}]}\max_{\sigma:L\to \mathbb{C}} w_{\sigma}h_{\sigma}  + \frac{1}{[K:\mathbb{Q}]}\sum_{k = 1}^{\nu} w_{k}h_k + \frac{1}{2} + c\kappa_{\tau}e^{-l_{\tau}}\right)^2 & \text{ if } \sqrt{\Delta}\notin\mathbb{Q} \\
%\left( \frac{1}{[L:\mathbb{Q}]}\max_{\sigma:L\to \mathbb{C}} w_{\sigma}h_{\sigma}  + \frac{1}{[K:\mathbb{Q}]}\sum_{k = 1}^{\nu} w_{k}h_k + \frac{1}{2} + c\kappa_{\tau}e^{-l_{\tau}}\right)^2 & \text{ if } \sqrt{\Delta}\in\mathbb{Q}.\\
%\end{cases}
%\end{align*}

Let
\[\mathbf{x} = (x_1, \dots, x_{\nu}, x_{\varepsilon_1}, \dots, x_{\varepsilon_{r}}) \in \mathbb{R}^{\nu + r}.\]
Then we define the ellipsoid $\mathcal{E_\tau}\subseteq \mathbb{R}^{r+\nu}$ by
\begin{align}\label{def:ellreal}
& \mathcal{E_\tau}=\{q_\tau(\mathbf{x})\leq (1 + r)(bb_{\varepsilon_1}\cdots b_{\varepsilon_r}); \ \mathbf{x}\in\mathbb{R}^{r+\nu}\}\end{align}
where
\[q_{\tau}(\mathbf{x})= (b_{\varepsilon_1}\cdots b_{\varepsilon_r})\left( q_f(x_1, \dots, x_{\nu}) + \sum_{i = 1}^r\frac{b}{b_{\varepsilon_i}}x_{\varepsilon_i}^2\right)\]
and
\[q_f(\mathbf{y}) = (A\mathbf{y})^{\text{T}}D^2A\mathbf{y}.\]

\edit{here}
\edit{here}
Let $M=M_\tau$ be the matrix defining the ellipsoid 
\[\mathcal E_\tau: z^tM^tMz\leq (1 + r)(bb_{\varepsilon_1}\cdots b_{\varepsilon_r}),\]
that is,  
\begin{align*}
M &=\sqrt{b_{\varepsilon_1}\cdots b_{\varepsilon_r}}\begin{pmatrix}
	DA & 0 & \dots & 0 & 0\\
	0 & \sqrt{\frac{b}{b_{\varepsilon 1}}} & \dots & 0 & 0\\
	0 & 0  & \sqrt{\frac{b}{b_{\varepsilon 2}}} & \dots & 0\\
	\vdots & \vdots &0 &  \ddots & \vdots\\ 
	0 & 0 & \dots & \dots & \sqrt{\frac{b}{b_{\varepsilon^*}}} \\
	\end{pmatrix}.
\end{align*}	
Note that we never need to compute $M$, but rather $M^TM$ so that we do not need to worry about precision. In this case, 
\begin{align*}
M^TM &= b_{\varepsilon_1}\cdots b_{\varepsilon_r}\begin{pmatrix}
	A^TD^2A & 0 & \dots & 0 & 0\\
	0 & \frac{b}{b_{\varepsilon 1}} & \dots & 0 & 0\\
	0 & 0  & \frac{b}{b_{\varepsilon 2}} & \dots & 0\\
	\vdots & \vdots &0 &  \ddots & \vdots\\ 
	0 & 0 & \dots & \dots & \frac{b}{b_{\varepsilon^*}} \\
	\end{pmatrix}\\
	& = \begin{pmatrix}
	(b_{\varepsilon_1}\cdots b_{\varepsilon_r})A^TD^2A & 0 & \dots & 0 & 0\\
	0 & b b_{\varepsilon_2}\cdots b_{\varepsilon_r} & \dots & 0 & 0\\
	0 & 0  & bb_{\varepsilon_1}b_{\varepsilon_3}\cdots b_{\varepsilon_r} & \dots & 0\\
	\vdots & \vdots &0 &  \ddots & \vdots\\ 
	0 & 0 & \dots & \dots & bb_{\varepsilon_1}\cdots b_{\varepsilon_{r-1}} \\
	\end{pmatrix}.
\end{align*}	


%---------------------------------------------------------------------------------------------------------------------------------------------%

\subsection{The non-Archimedean ellipsoid}
\label{subsec:nonArchEllipsoid}

We now restrict our attention to those $p_v \in \{p_1, \dots, p_{\nu}\}$ and define the corresponding ellipsoid. As before, let $\mathbf{m} = (n_1, \dots, n_{\nu}, a_1, \dots, a_r) \in \mathbb{R}^{r + \nu}$ be any solution of \eqref{eq:EfficientSunit} with corresponding vector $\mathbf{n} = (n_1, \dots, n_{\nu})$. Take $\mathbf{l},\mathbf{h}\in\mathbb{R}^{\nu+m}$ such that $\mathbf{0} \leq \mathbf{l} \leq \mathbf{h}$ and suppose $h_v(z)\leq h_v$ for all $v\in S^*$. 

Now, Lemma~\ref{lem:boundqf} and Lemma~\ref{lem:mepsbound} still hold here. In particular, we let $b$, $b_{\eps_l}$ be defined as in \eqref{def:bbound} and \eqref{def:bepsbound}, respectively, where $l = 1, \dots, r$. We do not distinguish any $\varepsilon_l^*$. Instead, we will see later that the condition $h_{v}(z) \geq l_{v}$ corresponding to the set $\Sigma_v(\mathbf{l}, \mathbf{h})$ will be used elsewhere. 

We define the ellipsoid $\mathcal{E}_v \subseteq \mathbb{R}^{\nu + r}$ by 
\begin{align}\label{def:ellp}
& \mathcal{E}_v=\{q_l(\mathbf{x})\leq (1 + r)(bb_{\varepsilon_1}\cdots b_{\varepsilon_r}); \ \mathbf{x}\in\mathbb{R}^{r+\nu}\},\end{align}
where
\[q_v\mathbf{x})= (b_{\varepsilon_1}\cdots b_{\varepsilon_r})\left( q_f(x_1, \dots, x_{\nu}) + \sum_{i = 1}^r\frac{b}{b_{\varepsilon_i}}x_{\varepsilon_i}^2\right)\]
and
\[q_f(\mathbf{y}) = (A\mathbf{y})^{\text{T}}D^2A\mathbf{y}.\]



%---------------------------------------------------------------------------------------------------------------------------------------------%

\section{Archimedean sieve: the real case}

Let $\tau:L\to\mathbb{C}$ be an embedding. We take $l,h \in \mathbb{R}^{m+\nu}$ with $0\leq l\leq h$ and $l_\tau\geq \log 2$. Then we consider the 
translated lattice $\Gamma_{\tau}\subset \mathbb{Z}^{r + \nu}$ defined by 
\[\Gamma_{\tau} =\Phi_{\tau}(\mathbb{Z}^{r+\nu})+w\]
where $w=(0,\dotsc,0,\alpha_0)^{\text{T}}$ for $c$ a constant of the size $e^{l_\tau}$ and where $\Phi_\tau$ is a linear transformation which is the identity on $\mathbb{Z}^{u+r - 1}$ and which sends 
\[(0, \dots, 0, 1) \mapsto (\alpha_{\gamma 1}, \dots, \alpha_{\gamma {\nu}}, \alpha_{\varepsilon 1}, \dots, \alpha_{\varepsilon {r}}).\]
That is, 
\[\left( \left[c\log\left(\tau\left(\frac{\gamma_1^{(k)}}{\gamma_1^{(j)}}\right)\right)\right], \dots, \left[c\log\left(\tau\left(\frac{\gamma_{\nu}^{(k)}}{\gamma_{\nu}^{(j)}}\right)\right)\right], \left[c\log\left(\tau\left(\frac{\varepsilon_1^{(k)}}{\varepsilon_1^{(j)}}\right)\right)\right], \dots, \left[c\log\left(\tau\left(\frac{\varepsilon_r^{(k)}}{\varepsilon_r^{(j)}}\right)\right)\right] \right).\]
The matrix associated to this lattice is therefore
\[\Gamma_{\tau} = \begin{pmatrix}
	1 & 0 & \dots &  \dots & 0 & 0\\ 
	0 & 1	& \dots & \dots & 0 & 0\\
	\vdots & \vdots & \ddots & \dots & \vdots & \vdots \\ 
	0 & 0 & \dots &  \dots & 1 & 0\\ 
	\alpha_{\gamma 1} & \dots &\alpha_{\gamma {\nu}} & \alpha_{\varepsilon 1} & \dots & \alpha_{\varepsilon {r}}
\end{pmatrix}.\]

%for some fixed $\epsilon^*\in\unit_\infty$ and whose row indexed by $\epsilon^*$ is given by $(\alpha_u)\in\ZZ^\unit$. 
Let $\mathcal E_{\tau}=\mathcal E_{\tau}(h,l_{\tau})$ be the ellipsoid constructed in \eqref{def:ellreal}. Let ${\mathbf{m} = (n_1, \dots, n_{\nu}, a_1, \dots, a_r) \in \mathbb{R}^{r + \nu}}$ be any solution of \eqref{Eq:main3}. We say that $\mathbf{m}$ is determined by some $\mathbf{y} \in \Gamma_{\tau}$ if 
\[\mathbf{y} = (y_1, \dots, y_{r+ \nu}) = \left(n_1, \dots, n_{\nu}, a_1, \dots, a_{r-1}, \alpha_0+\sum_{i = 1}^r a_i \alpha_{\varepsilon i} + \sum_{i = 1}^{\nu} n_i \alpha_{\gamma i}\right)\]
where the missing element $a_{i}$ corresponds to $\varepsilon^*$.

\begin{lemma}\label{lem:archsieve}
Let ${\mathbf{m} = (n_1, \dots, n_{\nu}, a_1, \dots, a_r) \in \mathbb{R}^{r + \nu}}$ be any solution of \eqref{Eq:main3} which lies in $\Sigma_\tau(l,h)$. Then $\mathbf{m}$ is determined by some $\gamma\in \Gamma_\tau\cap\mathcal E_\tau.$
\end{lemma}

%\begin{proof}
%We define $\gamma\in \ZZ^\unit$ with $\gamma_u=m_u$ for each $u\in\unit\setminus \epsilon^*$ and $\gamma_{\epsilon^*}=\alpha_0+\sum_{u\in\unit} m_u \alpha_u$. Then $\gamma\in \Gamma_\tau$ and \eqref{def:bepsstarbound} implies that $\gamma_{\epsilon^*}^2\leq b_{\epsilon^*}$. Further \eqref{def:bbound} gives that $q_f(\gamma_\delta)\leq b$, and  \eqref{def:bepsbound} provides that $\gamma_\epsilon^2\leq b_\epsilon$ for each $\epsilon\in\unit_\infty$ with $\epsilon\neq \epsilon^*$. It follows that $$q_\tau(\gamma)\leq \frac{1}{|\unit|}(|\unit_S|b+\sum_{\epsilon\in\unit_\infty}\frac{b}{b_\epsilon}b_\epsilon)\leq b.$$
%This proves that $\gamma\in\mathcal E_\tau$ and hence the statement follows.
%\end{proof}
%
Suppose that $\gamma\in \Gamma_\tau\cap \mathcal E_\tau$. Let $M=M_\tau$ be the matrix defining the ellipsoid 
\[\mathcal E_\tau: z^tM^tMz\leq (1 + r)(bb_{\varepsilon_1}\cdots b_{\varepsilon_r}),\]
that is,  
\begin{align*}
M &=\sqrt{b_{\varepsilon_1}\cdots b_{\varepsilon_r}}\begin{pmatrix}
	DA & 0 & \dots & 0 & 0\\
	0 & \sqrt{\frac{b}{b_{\varepsilon 1}}} & \dots & 0 & 0\\
	0 & 0  & \sqrt{\frac{b}{b_{\varepsilon 2}}} & \dots & 0\\
	\vdots & \vdots &0 &  \ddots & \vdots\\ 
	0 & 0 & \dots & \dots & \sqrt{\frac{b}{b_{\varepsilon^*}}} \\
	\end{pmatrix}.
\end{align*}	
Note that we never need to compute $M$, but rather $M^TM$ so that we do not need to worry about precision. In this case, 
\begin{align*}
M^TM &= b_{\varepsilon_1}\cdots b_{\varepsilon_r}\begin{pmatrix}
	A^TD^2A & 0 & \dots & 0 & 0\\
	0 & \frac{b}{b_{\varepsilon 1}} & \dots & 0 & 0\\
	0 & 0  & \frac{b}{b_{\varepsilon 2}} & \dots & 0\\
	\vdots & \vdots &0 &  \ddots & \vdots\\ 
	0 & 0 & \dots & \dots & \frac{b}{b_{\varepsilon^*}} \\
	\end{pmatrix}\\
	& = \begin{pmatrix}
	(b_{\varepsilon_1}\cdots b_{\varepsilon_r})A^TD^2A & 0 & \dots & 0 & 0\\
	0 & b b_{\varepsilon_2}\cdots b_{\varepsilon_r} & \dots & 0 & 0\\
	0 & 0  & bb_{\varepsilon_1}b_{\varepsilon_3}\cdots b_{\varepsilon_r} & \dots & 0\\
	\vdots & \vdots &0 &  \ddots & \vdots\\ 
	0 & 0 & \dots & \dots & bb_{\varepsilon_1}\cdots b_{\varepsilon_{r-1}} \\
	\end{pmatrix}.
\end{align*}	
	
Since $\gamma\in \Gamma_\tau\cap \mathcal E_\tau$, there exists $x\in \mathbb{R}^{r + \nu}$ such that $\gamma=\Gamma_\tau x+w$ and ${\gamma^tM^tM\gamma\leq (1 + r)(bb_{\varepsilon_1}\cdots b_{\varepsilon_r})}$. We thus have
\[(\Gamma_\tau x+w)^tM^tM(\Gamma_\tau x+w) \leq (1 + r)(bb_{\varepsilon_1}\cdots b_{\varepsilon_r}).\]
As $\Gamma_{\tau}$ is clearly invertible, with matrix inverse
\[\Gamma_{\tau}^{-1} = \begin{pmatrix}
	1 & 0 & \dots &  \dots & 0 & 0\\ 
	0 & 1	& \dots & \dots & 0 & 0\\
	\vdots & \vdots & \ddots & \dots & \vdots & \vdots \\ 
	0 & 0 & \dots &  \dots & 1 & 0\\ 
	-\frac{\alpha_{\gamma 1}}{\alpha_{\varepsilon {r}}} & \dots &-\frac{\alpha_{\gamma {\nu}}}{\alpha_{\varepsilon {r}}} & -\frac{\alpha_{\varepsilon 1}}{\alpha_{\varepsilon {r}}} & \dots & \frac{1}{\alpha_{\varepsilon {r}}}
\end{pmatrix},\]
we can find a vector $c$ such that $\Gamma_{\tau}c = -w$. Indeed, this vector is $c = \Gamma_{\tau}^{-1}(-w)$, where
\[c = \Gamma_{\tau}^{-1}w = \begin{pmatrix}
	1 & 0 & \dots &  \dots & 0 & 0\\ 
	0 & 1	& \dots & \dots & 0 & 0\\
	\vdots & \vdots & \ddots & \dots & \vdots & \vdots \\ 
	0 & 0 & \dots &  \dots & 1 & 0\\ 
	-\frac{\alpha_{\gamma 1}}{\alpha_{\varepsilon {r}}} & \dots &-\frac{\alpha_{\gamma {\nu}}}{\alpha_{\varepsilon {r}}} & -\frac{\alpha_{\varepsilon 1}}{\alpha_{\varepsilon r}} & \dots & \frac{1}{\alpha_{\varepsilon {r}}}
\end{pmatrix}
	\begin{pmatrix}
	0 \\ 0 \\ \vdots \\ 0 \\ -\alpha_0
	\end{pmatrix}
	= \begin{pmatrix}
	0 \\ 0 \\ \vdots \\ 0 \\ -\frac{\alpha_0}{\alpha_{\varepsilon r}}
\end{pmatrix}.\]
Now, 
\begin{align*}
(1 + r)(bb_{\varepsilon_1}\cdots b_{\varepsilon_r})
	& \geq (\Gamma_\tau x+w)^tM^tM(\Gamma_\tau x+w) \\
	& =  (\Gamma_\tau x- (-w))^tM^tM(\Gamma_\tau x-(-w)) \\
	& = (\Gamma_\tau x-\Gamma_{\tau}c)^tM^tM(\Gamma_\tau x-\Gamma_{\tau}c)\\
	& = (\Gamma_\tau (x-c))^tM^tM(\Gamma_\tau (x-c))\\
	& = (x-c)^t(M\Gamma_{\tau})^tM\Gamma_\tau(x-c)\\
	& = (x-c)^tB^tB(x-c)
\end{align*}
where $B = M\Gamma_\tau$. That is, we are left to solve
\[(x-c)B^tB(x-c) \leq (1 + r)(bb_{\varepsilon_1}\cdots b_{\varepsilon_r}).\]
%Here we observe that the vector $c$ is likely not an integral vector; that is, $-\frac{\alpha_0}{\alpha_{\varepsilon r}} \notin \mathbb{Z}$. To amend this, we modify the bound as follows
%\begin{align*}
%\alpha_{\varepsilon r}^2(x-c)^tB^tB(x-c) & \leq \alpha_{\varepsilon r}^2(1 + r)(bb_{\varepsilon_1}\cdots b_{\varepsilon_r})\\
%(\alpha_{\varepsilon r}x-\alpha_{\varepsilon r}c)^tB^tB(\alpha_{\varepsilon r}x-\alpha_{\varepsilon r}c) & \leq \alpha_{\varepsilon r}^2(1 + r)(bb_{\varepsilon_1}\cdots b_{\varepsilon_r})\\
%(z-d)^tB^tB(z-d) & \leq \alpha_{\varepsilon r}^2(1 + r)(bb_{\varepsilon_1}\cdots b_{\varepsilon_r})
%\end{align*}
%where $d = (0 \dots 0 -\alpha_0)$ and $z = \alpha_{\varepsilon r}x$. 
%Is this necessary? Our thing should work over the rationals regardless


%---------------------------------------------------------------------------------------------------------------------------------------------%
%---------------------------------------------------------------------------------------------------------------------------------------------%

%
%
%
%
%
%
% Since
%\[M\gamma=M(\Gamma_\tau x+w) = M\Gamma_{\tau} x + Mw.\]
%Here, it is clear that $M$ is invertible, with inverse
%
%
%
%, it is also invertible. Hence we can find a vector $c$ such that $Bc = v$. 
%
%\begin{align*}
%B = &M\Gamma_{\tau} =\sqrt{b_{\varepsilon_1}\cdots b_{\varepsilon_r}}\begin{pmatrix}
%	DA & 0 & \dots & 0 & 0\\
%	0 & \sqrt{\frac{b}{b_{\varepsilon 1}}} & \dots & 0 & 0\\
%	0 & 0  & \sqrt{\frac{b}{b_{\varepsilon 2}}} & \dots & 0\\
%	\vdots & \vdots &0 &  \ddots & \vdots\\ 
%	0 & 0 & \dots & \dots & \sqrt{\frac{b}{b_{\varepsilon^*}}} \\
%	\end{pmatrix}
%	\begin{pmatrix}
%	1 & 0 & \dots &  \dots & \dots & 0 & 0\\ 
%	0 & 1	& \dots & \dots & \dots & 0 & 0\\
%	\vdots & \vdots & \ddots & \dots & \dots & \vdots & \vdots \\ 
%	0 & 0 & \dots &  \dots & \dots & 1 & 0\\ 
%	\alpha_0 & \alpha_{\gamma 1} & \dots &\alpha_{\gamma {\nu}} & \alpha_{\varepsilon 1} & \dots & \alpha_{\varepsilon^*}
%	\end{pmatrix}\\
%& = \sqrt{b_{\varepsilon_1}\cdots b_{\varepsilon_r}}\begin{pmatrix}
%	DA & \dots & \dots & \dots& 0 & 0\\
%	 0  & \sqrt{\frac{b}{b_{\varepsilon 1}}}& \dots & \dots & \vdots & \vdots \\
%	\vdots & \vdots &\vdots & \ddots & \vdots & \vdots\\ 
%	0 & 0  & \dots & \dots & \sqrt{\frac{b}{b_{\varepsilon r-1}}} &0\\
%	\alpha_0 \sqrt{\frac{b}{b_{\varepsilon^*}}}  & \dots & \dots & \alpha_{\varepsilon 1}\sqrt{\frac{b}{b_{\varepsilon^*}}} & \dots & \alpha_{\varepsilon^*}\sqrt{\frac{b}{b_{\varepsilon^*}}}
%	\end{pmatrix}\\
%	& =\begin{pmatrix}
%	 \sqrt{b_{\varepsilon_1}\cdots b_{\varepsilon_r}}DA & \dots & \dots & \dots& 0 & 0\\
%	 0  &  \sqrt{bb_{\varepsilon_2}\cdots b_{\varepsilon_r}}& \dots & \dots & \vdots & \vdots \\
%	\vdots & \vdots &\vdots & \ddots & \vdots & \vdots\\ 
%	0 & 0  & \dots & \dots &  \sqrt{bb_{\varepsilon_1}\cdots b_{\varepsilon_{r-2}}b_{\varepsilon_{r}}} &0\\
%	\alpha_0 \sqrt{bb_{\varepsilon_1}\cdots b_{\varepsilon_{r-1}}}  & \dots & \dots & \alpha_{\varepsilon 1}\sqrt{bb_{\varepsilon_1}\cdots b_{\varepsilon_{r-1}}} & \dots & \alpha_{\varepsilon^*}\sqrt{bb_{\varepsilon_1}\cdots b_{\varepsilon_{r-1}}}
%	\end{pmatrix}.
%\end{align*}
%
%It follows that
%\begin{align*}
%(1 + r)(bb_{\varepsilon_1}\cdots b_{\varepsilon_r})	
%	& \geq \gamma^tM^tM\gamma \\
%	& = (M\gamma)^tM\gamma \\
%	& = (Bx + v)^t(Bx + v).
%%	& = ((Bx)^t+v^t)(Bx+v)\\
%%	& = (x^tB^t + v^t)(Bx+v)\\
%%	& = x^tB^tBx + (Bx)^tv+v^tBx+v^tv.
%\end{align*}
%Now, since $B$ is positive-definite, it is also invertible. Hence we can find a vector $c$ such that $Bc = v$. 
%
%Now, 
%\[(Bx)^tv = (M\gamma - v)^tv = (M\gamma)^tv -v^tv\]
%and
%\[v^tBx = v^t(M\gamma - v) = v^t(M\gamma) -v^tv\]
%and thus
%\begin{align*}
%(1 + r)(bb_{\varepsilon_1}\cdots b_{\varepsilon_r})
%	& \geq x^tB^tBx + (Bx)^tv+v^tBx+v^tv\\
%	& = x^tB^tBx + (M\gamma)^tv -v^tv + v^t(M\gamma) -v^tv +v^tv\\
%	& = x^tB^tBx + (M\gamma)^tv + v^t(M\gamma) -v^tv \\
%	& = x^tB^tBx + 2(v^t(M\gamma)) -v^tv.
%\end{align*}
%Next, we observe that
%\begin{align*}
%|v^t(M\gamma)|
%	& = |(Mw)^t(M\gamma)|\\
%	& = |wM^TM\gamma|\\
%	& = |\alpha_0 bb_{\varepsilon_1}\cdots b_{\varepsilon_{r-1}} \gamma_n|\\
%	& \leq |\alpha_0| bb_{\varepsilon_1}\cdots b_{\varepsilon_{r-1}} \sqrt{b_{\varepsilon}^*},
%\end{align*}
%%
%%
%%\[|v^t(M\gamma)| \leq \frac{b}{b_{\epsilon^*}^{1/2}}|\alpha_0|\]
%and thus
%\[x^tB^tBx\leq (1 + r)(bb_{\varepsilon_1}\cdots b_{\varepsilon_r})+v^tv - 2v^t(M\gamma)\]
%so that 
%\begin{align*}
%|x^tB^tBx| 
%	& \leq |(1 + r)(bb_{\varepsilon_1}\cdots b_{\varepsilon_r})+v^tv - 2v^t(M\gamma)|\\
% 	& \leq |(1 + r)(bb_{\varepsilon_1}\cdots b_{\varepsilon_r})+ v^tv| + 2|v^t(M\gamma)|\\
%	& \leq \left|(1 + r)(bb_{\varepsilon_1}\cdots b_{\varepsilon_r})+ v^tv\right| + 2 |\alpha_0| bb_{\varepsilon_1}\cdots b_{\varepsilon_{r-1}} \sqrt{b_{\varepsilon}^*}.
%\end{align*}
%
%Now, following the steps of the Fincke-Pohst algorithm, we first generate the matrix $R$ via Cholesky decomposition applied to $B^{\text{T}}B$. In particular, this means that we don't need to know $B$ explicitly, but rather $B^TB$, where
%\[B^TB = (M\Gamma_{\tau})^T(M\Gamma_{\tau}) = \Gamma_{\tau}^T(M^TM)\Gamma_{\tau}\]
%where both $M^TM$ and $\Gamma_{\tau}$ are integral matrices. 
%
%
%\subsection{Non-Archimedean sieve}
%%Let $v\in T$. To simplify our exposition, we slightly abuse notation and we write $v=\ord_p:\bar{\QQ}_p\to\QQ$. We take $l,h\in\RR^{S^*}$ with $0\leq l\leq h$ and $l_v/\log p\geq \max\bigl(\tfrac{1}{p-1},v(\mu_0)\bigl)-v(\lambda_0)$, and then we consider the 
%%translated lattice $\Gamma_v\subset \ZZ^\unit$ defined below.  We say that $(x,y)\in\Sigma$ with $m\in\ZZ^{\unit}$ is determined by some $\gamma\in \Gamma_v$ if the entries of $\gamma$ are a (fixed) permutation of the entries of $m$. Let $\mathcal E_v\subseteq \RR^\unit$ be the ellipsoid constructed in \eqref{def:ellnonarch}.
%%\begin{lemma}\label{lem:nonarchsieve}
%%Any $(x,y)\in\Sigma_v(l,h)$ is determined by some $\gamma\in \Gamma_v\cap\mathcal E_v.$
%%\end{lemma}
%%
%%In the remaining of this section we prove this lemma.
%
%---------------------------------------------------------------------------------------------------------------------------------------------%
%---------------------------------------------------------------------------------------------------------------------------------------------%

\subsection{Archimedean Real Case Summary}
If $(n_1, \dots, n_{\nu}, a_1, \dots, a_r) \in \mathbb{R}^{r+\nu}$ is a solution which lies in $\Sigma_{\tau}(l,h)$, then, by definition, it corresponds to a solution $(x,y)$ satisfying
\[\Sigma_{\tau}(l,h) = \{(x,y) \in \Sigma \ | \ (h_v(z))\leq h \text{ and }  (h_v(z))\nleq 0 \text{ and } h_{\tau}(z)>l_{\tau}\}.\]
Here, $l_{\tau}$ is defined as some constant such that 
\[l_{\tau} > c_{\tau}.\]
By the computations above (see page 99), it follows that
\[\left|\alpha_0+\sum_{i = 1}^r a_i \alpha_{\varepsilon i} + \sum_{i = 1}^{\nu} n_i \alpha_{\gamma i}\right|
 \leq \frac{1}{2}\left(\frac{1}{[K:\mathbb{Q}]}\sum_{l = 1}^{\nu}w_l h_l + \frac{1}{[L:\mathbb{Q}]}\sum_{\sigma :L \to \mathbb{C}} w_{\sigma}h_{\sigma} \right) + \frac{1}{2} + c\kappa_{\tau}e^{-l_{\tau}}.\]

Now, consider the vector 
\[\gamma = (n_1, \dots, n_{\nu}, a_1, \dots, a_{r-1}, \alpha_0+\sum_{i = 1}^r a_i \alpha_{\varepsilon i} + \sum_{i = 1}^{\nu} n_i \alpha_{\gamma i})\]
and the lattice defined by $\Gamma_{\tau}x + w$
\[\begin{pmatrix}
	1 & 0 & \dots &  \dots & 0 & 0\\ 
	0 & 1	& \dots & \dots & 0 & 0\\
	\vdots & \vdots & \ddots & \dots & \vdots & \vdots \\ 
	0 & 0 & \dots &  \dots & 1 & 0\\ 
	\alpha_{\gamma 1} & \dots &\alpha_{\gamma {\nu}} & \alpha_{\varepsilon 1} & \dots & \alpha_{\varepsilon {r}}
\end{pmatrix}
\begin{pmatrix}
	x_1 \\ x_2 \\ \vdots \\ x_{\nu+r - 1} \\ x_{\nu+r}\\ 
\end{pmatrix}+
\begin{pmatrix}
	0 \\ 0 \\ \vdots \\ 0 \\ \alpha_0\\ 
\end{pmatrix}\]
for some vector $(x_1, \dots, x_{\nu+r}) \in \mathbb{Z}^{\nu+r}$.
If  $(x_1, \dots, x_{\nu+r}) = (n_1, \dots, n_{\nu}, a_1, \dots, a_{r})$, then a quick computation shows that 
\[\Gamma_{\tau}x + w = 
\begin{pmatrix}
	x_1 \\ x_2 \\ \vdots \\ x_{\nu+r - 1} \\ \alpha_0+\sum_{i = 1}^r x_i \alpha_{\varepsilon i} + \sum_{i = 1}^{\nu} x_i \alpha_{\gamma i}\\ 
\end{pmatrix} = 
\begin{pmatrix}
	n_1 \\ n_2 \\ \vdots \\ a_{r - 1} \\ \alpha_0+\sum_{i = 1}^r a_i \alpha_{\varepsilon i} + \sum_{i = 1}^{\nu} n_i \alpha_{\gamma i}\\ 
\end{pmatrix} = \gamma^T.\]
Hence, $\gamma$ is in the lattice $\Gamma$.

Now, consider the ellipsoid $\mathcal{E}_\tau$. We claim that $\gamma \in \mathcal{E}_{\tau}$. Indeed, this means that 
\[\gamma^tM^tM\gamma  \leq (1 + r)(bb_{\varepsilon_1}\cdots b_{\varepsilon_r}).\]
In particular
\begin{align*}
& \gamma^TM^TM\gamma \\ 
	& = \begin{pmatrix}
	n_1 \\ n_2 \\ \vdots \\ a_{r - 1} \\ \alpha_0+\sum_{i = 1}^r a_i \alpha_{\varepsilon i} + \sum_{i = 1}^{\nu} n_i 	\alpha_{\gamma i}\\ 
	\end{pmatrix}
	\begin{pmatrix}
	(b_{\varepsilon_1}\cdots b_{\varepsilon_r})A^TD^2A & 0 & \dots & 0\\
	0 & b b_{\varepsilon_2}\cdots b_{\varepsilon_r} & \dots & 0 \\
	\vdots & \vdots  &  \ddots & \vdots\\ 
	0 & 0 & \dots & bb_{\varepsilon_1}\cdots b_{\varepsilon_{r-1}} \\
	\end{pmatrix}
	\gamma\\
	& = \begin{pmatrix}
		\begin{pmatrix} 
		n_1 \\ n_2 \\ \vdots \\ n_{\nu}
		\end{pmatrix}(b_{\varepsilon_1}\cdots b_{\varepsilon_r})A^TD^2A\\
		a_1 b b_{\varepsilon_2}\cdots b_{\varepsilon_r} \\ 
		\vdots \\
		a_{r - 1} b b_{\varepsilon_1}\cdots b_{\varepsilon_{r-2}}b_{\varepsilon_r} \\ 
		\left(\alpha_0+\sum_{i = 1}^r a_i \alpha_{\varepsilon i} + \sum_{i = 1}^{\nu} n_i \alpha_{\gamma i}\right)b b_{\varepsilon_1}\cdots b_{\varepsilon_{r-1}}\\ 
	\end{pmatrix}
	\begin{pmatrix}
	 n_1 & \dots & a_{r-1} & \alpha_0+\sum_{i = 1}^r a_i \alpha_{\varepsilon i} + \sum_{i = 1}^{\nu} n_i \alpha_{\gamma i}
	 \end{pmatrix}\\
	 & = \begin{pmatrix} 
		n_1 \\ n_2 \\ \vdots \\ n_{\nu}
		\end{pmatrix}(b_{\varepsilon_1}\cdots b_{\varepsilon_r})A^TD^2A
		\begin{pmatrix}
	 n_1 & \dots & n_{\nu} \end{pmatrix} + a_1^2 b b_{\varepsilon_2}\cdots b_{\varepsilon_r} + \cdots + a_{r - 1}^2 b b_{\varepsilon_1}\cdots b_{\varepsilon_{r-2}}b_{\varepsilon_r} + \\
	 & \quad \quad \quad + \left(\alpha_0+\sum_{i = 1}^r a_i \alpha_{\varepsilon i} + \sum_{i = 1}^{\nu} n_i \alpha_{\gamma i}\right)^2b b_{\varepsilon_1}\cdots b_{\varepsilon_{r-1}}.
\end{align*}	
Now, by definition, we have 
\begin{align*}
\gamma^TM^TM\gamma 
	& = \begin{pmatrix} 
		n_1 \\ n_2 \\ \vdots \\ n_{\nu}
		\end{pmatrix}(b_{\varepsilon_1}\cdots b_{\varepsilon_r})A^TD^2A
		\begin{pmatrix}
	 n_1 & \dots & n_{\nu} \end{pmatrix} + a_1^2 b b_{\varepsilon_2}\cdots b_{\varepsilon_r} + \cdots + a_{r - 1}^2 b b_{\varepsilon_1}\cdots b_{\varepsilon_{r-2}}b_{\varepsilon_r} + \\
	 & \quad \quad \quad + \left(\alpha_0+\sum_{i = 1}^r a_i \alpha_{\varepsilon i} + \sum_{i = 1}^{\nu} n_i \alpha_{\gamma i}\right)^2b b_{\varepsilon_1}\cdots b_{\varepsilon_{r-1}}\\
& \leq (b_{\varepsilon_1}\cdots b_{\varepsilon_r}) q_f(n_1 \dots n_{\nu}) +					 			b_{\varepsilon_1} b b_{\varepsilon_2}\cdots b_{\varepsilon_r} + \cdots + b_{\varepsilon_{r - 1}} b 		b_{\varepsilon_1}\cdots b_{\varepsilon_{r-2}}b_{\varepsilon_r} + b_{\varepsilon_r}b 					b_{\varepsilon_1}\cdots b_{\varepsilon_{r-2}}b_{\varepsilon_{r-1}}\\
& \leq (b_{\varepsilon_1}\cdots b_{\varepsilon_r})b + r(bb_{\varepsilon_1}\cdots b_{\varepsilon_r})\\
& = (1+r)(bb_{\varepsilon_1}\cdots b_{\varepsilon_r}).
\end{align*}
Thus $\gamma \in \mathcal{E}_{\tau}$. Thus, if we assume that our solution lies in $\Sigma_{\tau}(l,h)$, it follows that $\gamma \in \Gamma \cap \mathcal{E}_{\tau}$. Now, by Rafael, 
\[\Sigma=\Sigma\left(h_{0}\right), \quad \Sigma(h)=\Sigma(l, h) \cup \Sigma(l) \quad \text { and } \quad \Sigma(l, h)=\cup_{v \in S^{*}} \Sigma_{v}(l, h).\]
Thus, we collect all solutions from $\Sigma_{\tau}(l,h)$ and continue to generate all solutions at the other places. 

MAYBE COULD USE MORE DETAIL HERE
%---------------------------------------------------------------------------------------------------------------------------------------------%
%---------------------------------------------------------------------------------------------------------------------------------------------%

\section{Non-Archimedean Case}

\subsection{Non-Archimedean sieve}

Note that in this section we might use $v$ and $l$ interchangeably. Eventually this will be fixed to be consistent...

Let $v \in \{1, \dots, \nu\}$. We take vectors $l,h \in \mathbb{R}^{\nu+r}$ with $0 \leq l \leq h$ and 
\[\frac{l_v}{\log(p)} \geq \max\left( \frac{1}{p-1}, \ord_{p_v}(\delta_1)\right) - \ord_{p_v}(\delta_2)\]
and then consider the translated lattice $\Gamma_v \subseteq \mathbb{Z}^{\nu + r}$ defined below. We say that $(x,y) \in \Sigma$ with ${\mathbf{m} = (n_1, \dots, n_{\nu}, a_1, \dots, a_r) \in \mathbb{R}^{r + \nu}}$ is determined by some $\gamma \in \Gamma_v$ if the entries of $\gamma$ are a (fixed) permutation of the entries of $\mathbf{m}$. Let $\mathcal{E}_v$ be the ellipsoid constructed in \eqref{def:ellp}. 

\begin{lemma}
And $(x,y) \in \Sigma_v(l,h)$ is determined by some $\gamma \in \Gamma_v \cap \mathcal{E}_v$. 
\end{lemma}

In the remainder of this section, we prove this lemma. 

\subsubsection{Computing $u_l - r_l = \sum_{i = 1}^{\nu} n_ia_{li}$}

Recall that $z \in \mathbb{C}_p$ having $\ord_p(z) = 0$ is called a $p$-adic unit. 

Let $l \in \{1, \dots, \nu \}$ and consider the prime $p = p_l$. For every $i \in \{1, \dots, r\}$, part (ii) of the Corollary of Lemma 2 of Tzanakis-de Weger tells us that $\frac{\varepsilon_1^{(i_0)}}{\varepsilon_1^{(j)}}$ and $\frac{\varepsilon_i^{(k)}}{\varepsilon_i^{(j)}}$ (for $i = 1, \dots, r$) are $p_l$-adic units. 

From now on we make the following choice for the index $i_0$. Let $g_l(t)$ be the irreducible factor of $g(t)$ in $\mathbb{Q}_{p_l}[t]$ corresponding to the prime ideal $\mathfrak{p}_l$. Since $\mathfrak{p}_l$ has ramification index and residue degree equal to $1$, $\deg(g_l[t]) = 1$. We choose $i_0 \in \{1,2,3\}$ so that $\theta^{(i_0)}$ is the root of $g_l(t)$. The indices of $j,k$ are fixed, but arbitrary. 

\begin{lemma} \label{Lem:units} \
\begin{enumerate}
\item[(i)] Let $i \in \{1, \dots, \nu\}$. Then $\frac{\gamma_i^{(k)}}{\gamma_i^{(j)}}$ are $p_l$-adic units. 
\item[(ii)] Let $i \in \{1, \dots, \nu\}$. Then $\ord_{p_l}\left(\frac{\gamma_i^{(i_0)}}{\gamma_i^{(j)}}\right) = a_{li}$, where $\mathbf{a_i} = (a_{1i}, \dots, a_{\nu i})$. 
\end{enumerate}
\end{lemma}

\begin{proof}
Consider the factorization of $g(t)$ in $\mathbb{Q}_{p_l}[t]: g(t) = g_1(t) \cdots g_m(t)$. Note $\theta^{(j)}$ is a root of some $g_h(t) \neq g_l(t)$. Let $\mathfrak{p}_h$ be the corresponding prime ideal above $p_l$ and $e_h$ be its ramification index. Then $\mathfrak{p} \neq \mathfrak{p}_l$ and since 
\[(\gamma_i)\mathcal{O}_K = \mathfrak{p}_1^{a_{1i}} \cdots \mathfrak{p}_{\nu}^{a_{\nu i}},\]
we have 
\[\ord_{p_l}(\gamma_i^{(j)}) = \frac{1}{e_h}\ord_{\mathfrak{p}_h}(\gamma_i) = 0.\]
An analogous argument gives $\ord_{p_l}(\gamma_i^{(k)}) = 0$. On the other hand, 
\[\ord_{p_l}(\gamma_i^{(i_0)}) = \frac{1}{e_l}\ord_{\mathfrak{p}_l}(\gamma_i) = \ord_{\mathfrak{p}_l}(\mathfrak{p}_1^{a_{1i}} \cdots \mathfrak{p}_{\nu}^{a_{\nu i}}) = a_{li}.\]

\end{proof}

We consider the form
\[\Lambda_{p_l} = \log_{p_l}(\delta_1) + \sum_{i=1}^r a_i\log_{p_l}\left( \frac{\varepsilon_i^{(k)}}{\varepsilon_i^{(j)}}\right) + \sum_{i=1}^{\nu} n_i \log_{p_l} \left( \frac{\gamma_i^{(k)}}{\gamma_i^{(j)}}\right).\]
To simplify our exposition, we introduce the following notation. 
\[b_1 = 1, \quad b_{1+i} = n_i \ \text{ for } i \in \{1, \dots, \nu\},\]
and
\[ b_{1 + \nu+i} = a_i \ \text{ for } i \in \{1, \dots, r\}.\]
Put
\[\alpha_1 = \log_{p_l} \delta_1, \quad \alpha_{1+i} = \log_{p_l}\left( \frac{\gamma_i^{(k)}}{\gamma_i^{(l)}}\right)  \ \text{ for } i \in \{1, \dots, \nu\},\]
and
\[\alpha_{1+ \nu+i} = \log_{p_l}\left( \frac{\varepsilon_i^{(k)}}{\varepsilon_i^{(l)}}\right)
\ \text{ for } i \in \{1, \dots, r\}.\]
Now, 
\[\Lambda_{p_l} = \log_{p_l}(\delta_1) + \sum_{i=1}^r a_i\log_{p_l}\left( \frac{\varepsilon_i^{(k)}}{\varepsilon_i^{(j)}}\right) + \sum_{i=1}^{\nu} n_i \log_{p_l} \left( \frac{\gamma_i^{(k)}}{\gamma_i^{(j)}}\right) = \sum_{i = 1}^{1 + \nu + r} b_i\alpha_i.\]


The next lemma deals with a special case in which the $n_l$ can be computed directly. 

\begin{lemma}
Let $l \in \{1, \dots, \nu \}$. If $\ord_{p_l}(\delta_1) \neq 0$, then 
\[ \sum_{i = 1}^{\nu} n_ia_{li} = \min\{\ord_{p_l}(\delta_1), 0\} - \ord_{p_l}(\delta_2).\]
\end{lemma}

%From here it follows that
%\[z_l = u_l + t_l = \sum_{j = 1}^{\nu}n_ja_{lj} + r_l + t_l.\]

\begin{proof}
Apply the Corollary of Lemma $2$ of Tzanakis-de Weger and Lemma~\ref{Lem:units} to both expressions of $\lambda$ in \eqref{Eq:Sunit}. On the one hand, we obtain $\ord_{p_l}(\lambda) = \min\{\ord_{p_l}(\delta_1), 0\}$, and on the other hand, we obtain 
\[\begin{split}
\ord_{p_l}(\lambda)
& = \ord_{p_l}(\delta_2) + \sum_{i = 1}^{\nu} \ord_{p_l}\left( \frac{\gamma_i^{(i_0)}}{\gamma_i^{(j)}}\right)^{n_i}\\
& = \ord_{p_l}(\delta_2) + \sum_{i = 1}^{\nu} n_ia_{li}.
\end{split}\]
\end{proof}

That is, 
\[\sum_{i = 1}^{\nu} n_ia_{li} = \begin{cases}
- \ord_{p_l}(\delta_2) & \text{ if } \ord_{p_l}(\delta_1) > 0 \\
 \ord_{p_l}(\delta_1)- \ord_{p_l}(\delta_2) = \ord_{p_l}(\delta_1/\delta_2) & \text{ if } \ord_{p_l}(\delta_1) < 0 \\
\end{cases}\]


From here, we will need to assume that $\ord_{p_l}(\delta_1) = 0$. We first recall the following result of the $p_l$-adic logarithm:

\begin{lemma}\label{Lem:padic}
Let $z_1, \dots, z_m \in \overline{\mathbb{Q}}_p$ be $p$-adic units and let $b_1, \dots, b_m \in \mathbb{Z}$. If
\[\ord_p(z_1^{b_1}\cdots z_m^{b_m} - 1) > \frac{1}{p-1}\]
then
\[\ord_p(b_1\log_p z_1 + \cdots + b_m \log_p z_m) = \ord_p(z_1^{b_1}\cdots z_m^{b_m} - 1) \]
\end{lemma}

NOT CLEAR TO ME IF THE BELOW IS AN EFFICIENT COMPUTATION TO MAKE, OR THAT IT HAPPENS OFTEN ENOUGH TO TEST.

For $l \in \{1, \dots \nu\}$, we identify conditions in which $n_l$ can be bounded by a small explicit constant.

Let $L$ be a finite extension of $\mathbb{Q}_{p_l}$ containing $\delta_1$, $\frac{\gamma_i^{(k)}}{\gamma_i^{(l)}}$ (for $i = 1, \dots, \nu$), and $ \frac{\varepsilon_i^{(k)}}{\varepsilon_i^{(l)}}$ (for $i = 1, \dots, r$). Since finite $p$-adic fields are complete, $\alpha_i \in L$ for $i = 1, \dots, 1+ \nu +r$ as well. Choose $\phi \in \overline{\mathbb{Q}_{p_l}}$ such that $L = \mathbb{Q}_{p_l}(\phi)$ and $\ord_{p_l}(\phi) > 0 $. Let $G(t)$ be the minimal polynomial of $\phi$ over $\mathbb{Q}_{p_l}$ and let $S$ be its degree. For $i = 1, \dots, 1 + \nu + r$ write
\[\alpha_i = \sum_{h = 1}^S \alpha_{ih}\phi^{h - 1}, \quad \alpha_{ih} \in \mathbb{Q}_{p_l}.\]
Then
\begin{equation} \label{Eq:lambdalh}
\Lambda_l = \sum_{h = 1}^S \Lambda_{lh}\phi^{h-1},
\end{equation}
with
\[\Lambda_{lh} = \sum_{i = 1}^{1 + \nu + r } b_i \alpha_{ih}\]
for $h = 1, \dots, S$. 

\begin{lemma}\label{Lem:discG}
For every $h \in \{1, \dots, S\}$, we have
\[\ord_{p_l}(\Lambda_{lh}) > \ord_{p_l}(\Lambda_l) - \frac{1}{2}\ord_{p_l}(\text{Disc}(G(t))).\]
\end{lemma}

\begin{proof}
Taking the images of \eqref{Eq:lambdalh} under conjugation $\phi \mapsto \phi^{(h)}$ ($h = 1, \dots, S$) gives
\[\begin{bmatrix}
\Lambda_l^{(1)} \\
\vdots \\
\Lambda_l^{(S)}	\\
\end{bmatrix}
=
\begin{bmatrix}
1 		& \phi^{(1)} 	& \cdots 	& \phi^{(1)S-1}\\
\vdots 	& \vdots 		& 		& \vdots \\
1 		& \phi^{(S)} 	& \cdots  	& \phi^{(S)S-1}\\
\end{bmatrix}
\begin{bmatrix}
\Lambda_{l1}\\
\vdots \\
\Lambda_{lS}\\
\end{bmatrix}\]
The $s \times s$ matrix $(\phi^{(h)i-1})$ above is invertible, with inverse
\[\frac{1}{\displaystyle \prod_{1\leq j<k\leq S} (\phi^{(k)} - \phi^{(j)})}
\begin{bmatrix}
\gamma_{11} 	& \cdots 	& \gamma_{1S}\\
\vdots 		& 		& \vdots\\
\gamma_{s1} 	& \cdots 	& \gamma_{SS}\\
\end{bmatrix},\]
where $\gamma_{jk}$ is a polynomial in the entries of $(\phi^{(h)i-1})$ having integer coefficients. Since $\ord_{p_l}(\phi) > 0$ and since $\ord_{p_l}(\phi^{(h)}) = \ord_{p_l}(\phi)$ for all $h = 1, \dots, S$, it follows that $\ord_{p_l}(\gamma_{jk}) > 0 $ for every $\gamma_{jk}$. Therefore, since 
\[\Lambda_{lh} = \frac{1}{\displaystyle \prod_{1\leq j<k\leq S}(\phi^{(k)} - \phi^{(j)})}\sum_{i = 1}^S \gamma_{hi}\Lambda_l^{(i)},\]
we have 
\[\begin{split}
\ord_{p_l}(\Lambda_{lh}) 
	& = \min_{1 \leq i \leq S} \left\{\ord_{p_l}(\gamma_{hi}) + \ord_{p_l}(\Lambda_l^{(i)})\right\} -\frac{1}{2}\ord_{p_l}(\text{Disc}(G(t)))\\
	& \geq \min_{1 \leq i \leq S} \ord_{p_l}(\Lambda_l^{(i)}) +  \min_{1 \leq i \leq S} \ord_{p_l}(\gamma_{hi}) - \frac{1}{2}\ord_{p_l}(\text{Disc}(G(t)))\\
	& = \ord_{p_l}\Lambda_l + \min_{1 \leq i \leq S} \ord_{p_l}(\gamma_{hi}) - \frac{1}{2}\ord_{p_l}(\text{Disc}(G(t)))
\end{split}\]
for every $h \in \{1, \dots, S\}$. 
%\min_{1 \leq i \leq s} \left\{\ord_{p_l}(\gamma_{hi}) + \ord_{p_l}(\Lambda_l^{(i)}) -\frac{1}{2}\ord_{p_l}(\text{Disc}(G(t)))\right\}\]
\end{proof}

\textbf{Remark.} The above can made more precise using the $\gamma_{jk}$, possibly giving us a tighter subsequent bound on the $n_l$. 

\begin{lemma} \label{Lem:Lambda}
If $\ord_{p_l}(\delta_1) = 0$ and 
\[\sum_{i = 1}^{\nu} n_{i}a_{li} > \frac{1}{p_l-1} - \ord_{p_l}(\delta_2),\]
then
\[\ord_{p_l}(\Lambda_l) = \sum_{i = 1}^{\nu} n_{i}a_{li} + \ord_{p_l}(\delta_2).\]
\end{lemma}

\begin{proof}
Immediate from Lemma~\ref{Lem:padic}.
\end{proof}
Said another way, this means
\[ \sum_{i = 1}^{\nu} n_{i}a_{li} = \ord_{p_l}(\Lambda_l) - \ord_{p_l}(\delta_2) =  \ord_{p_l}(\Lambda_l/\delta_2).\]

\begin{lemma} \label{Lem:specialcase} \
Suppose $\ord_{p_l}(\delta_1) = 0$. 
\begin{enumerate}
\item[(i)] If $\ord_{p_l}(\alpha_1) < \displaystyle \min_{2 \leq i \leq 1 + \nu + r} \ord_{p_l}(\alpha_i)$, then
\[\sum_{i = 1}^{\nu} n_i a_{li} \leq \max \left\{ \bigg\lfloor{\frac{1}{p-1} - \ord_{p_l}(\delta_2)}\bigg\rfloor,  \bigg \lceil\displaystyle \min_{2 \leq i \leq 1 + \nu + r} \ord_{p_l}(\alpha_{i}) - \ord_{p_l}(\delta_2) \bigg \rceil - 1 \right\}\]

\item[(ii)] For all $h \in \{1, \dots, S\}$, if $\ord_{p_l}(\alpha_{1h}) < \displaystyle \min_{2 \leq i \leq 1+ \nu + r} \ord_{p_l}(\alpha_{ih})$, then
\[\sum_{i = 1}^{\nu} n_i a_{li} \leq \max \left\{ \bigg\lfloor{\frac{1}{p-1} - \ord_{p_l}(\delta_2)}\bigg\rfloor, \bigg \lceil \displaystyle \min_{2 \leq i \leq 1+\nu+r} \ord_{p_l}(\alpha_{ih})- \ord_{p_l}(\delta_2) + \nu_l \bigg \rceil - 1\right\},\]
where 
\[\nu_l = \frac{1}{2}\ord_{p_l}(\text{Disc}(G(t)))\]
\end{enumerate}
\end{lemma}

AGAIN, IT'S NOT CLEAR HOW EFFICIENT THIS COMPUTATION IS... WE SHALL SEE. PART (1) IS ACTUALLY NEEDED IN THE REST OF THE COMPUTATIONS, BUT PART (2) MIGHT BE THE INEFFICIENT PART OF THIS. 

\begin{proof} \
\begin{enumerate}
\item[(i)] We prove the contrapositive. Suppose
\[\sum_{i = 1}^{\nu} n_i a_{li} > \frac{1}{p-1} - \ord_{p_l}(\delta_2), \]
and
\[\sum_{i = 1}^{\nu} n_i a_{li}  \geq \displaystyle \min_{2 \leq i \leq 1 +\nu + r} \ord_{p_l}(\alpha_{i}) - \ord_{p_l}(\delta_2).\]
Observe that
\[\begin{split}
\ord_{p_l}(\alpha_{1}) 	
	& = \ord_{p_l}\left( \Lambda_{l} - \sum_{i = 2}^{1+\nu +r}b_i\alpha_{i}\right) \\
	& \geq \min\left\{ \ord_{p_l}(\Lambda_{l}), \min_{2 \leq i \leq 1+\nu+r} \ord_{p_l}(b_i\alpha_{i})\right\}.
\end{split}\]
Therefore, it suffices to show that 
\[\ord_{p_l}(\Lambda_{l}) \geq \min_{2 \leq i \leq 1 + \nu + r} \ord_{p_l}(b_i\alpha_{i}).\]
By Lemma~\ref{Lem:padic}, the first inequality implies $\ord_{p_l}(\Lambda_{l}) = \displaystyle \sum_{i = 1}^{\nu} n_ia_{li} + \ord_{p_l}(\delta_2)$, from which the result follows. 

\item[(ii)] We prove the contrapositive. Let $h \in \{1, \dots, S\}$ and suppose
\[\sum_{i = 1}^{\nu} n_i a_{li} > \frac{1}{p-1} - \ord_{p_l}(\delta_2), \]
and
\[\sum_{i = 1}^{\nu} n_i a_{li}  \geq \nu_l + \displaystyle \min_{2 \leq i \leq 1+\nu+r} \ord_{p_l}(\alpha_{ih}) - \ord_{p_l}(\delta_2).\]
Observe that 
\[\begin{split}
\ord_{p_l}(\alpha_{1h}) 	
	& = \ord_{p_l}\left( \Lambda_{lh} - \sum_{i = 2}^{1+\nu+r}b_i\alpha_{ih}\right) \\
	& \geq \min\left\{ \ord_{p_l}(\Lambda_{lh}), \min_{2 \leq i \leq 1+\nu+r} \ord_{p_l}(b_i\alpha_{ih})\right\}
\end{split}\]
Therefore, it suffices to show that 
\[\ord_{p_l}(\Lambda_{lh}) \geq \min_{2 \leq i \leq 1+\nu+r} \ord_{p_l}(b_i\alpha_{ih}).\]
By Lemma~\ref{Lem:padic}, the first inequality implies $\ord_{p_l}(\Lambda_{l}) = \displaystyle \sum_{i = 1}^{\nu}n_ia_{li} + \ord_{p_l}(\delta_2)$. Combining this with Lemma~\ref{Lem:discG} yields
\[\ord_{p_l}(\Lambda_{lh}) \geq \displaystyle \sum_{i = 1}^{\nu} n_ia_{li} + \ord_{p_l}(\delta_2) - \nu_l.\]
The results now follow from our second assumption. 
\end{enumerate}
\end{proof}


We now set some notation and give some preliminaries for the $p_l$-adic reduction procedures. Consider a fixed index $l \in \{1, \dots, v\}$. Following Lemma \ref{Lem:specialcase}, we have
\[\ord_{p_l}(\alpha_1) \geq \min_{2\leq i\leq 1+ \nu+ r} \ord_{p_l}(\alpha_i) \quad \text{ and } \quad \ord_{p_l}(\alpha_{1h}) \geq \min_{2\leq i\leq 1+ \nu+ r}(\alpha_{ih}) \quad h = (1, \dots, s).\]
and 
\[\ord_{p_l}(\delta_1) = 0.\]

Let $I$ be the set of all indices $i' \in \{2, \dots, 1+ \nu + r\}$ for which
\[\ord_{p_l}(\alpha_{i'}) = \min_{2\leq i\leq 1+ \nu+ r} \ord_{p_l}(\alpha_i).\]
We will identify two cases, the \textit{special case} and the \textit{general case}. The special case occurs when there is some index $i' \in I$ such that $\alpha_i/\alpha_{i'} \in \mathbb{Q}_{p_l}$ for $i = 1, \dots, 1+ \nu+ r$. The general case is when there is no such index. 

We now assume that our Thue-Mahler equation has degree $3$ to assure that our linear form in $p$-adic logs has coefficients in $\mathbb{Q}_p$. DETAILS NEEDED HERE [p51 of HAMBROOK]. This means that we are indeed always in the Special Case of TdW/Hambrook. 

Thus, let $\hat{i}$ be an arbitrary index in $I$ for which $\alpha_i/\alpha_{\hat{i}} \in \mathbb{Q}_{p_l}$ for every $i = 1, \dots, 1+ \nu+ r$. We further define
\[\beta_i = - \frac{\alpha_i}{\alpha_{\hat{i}}} \quad i = 1, \dots, 1+ \nu+ r,\]
and 
\[\Lambda'_l = \frac{1}{\alpha_{\hat{i}}}\Lambda_l = \sum_{i = 1}^{1+ \nu+ r} b_i(-\beta_i).\]
Now, we have $\beta_i \in \mathbb{Z}_{p_l}$ for $i = 1, \dots, 1+ \nu+ r$. 

\begin{lemma} \label{Lem:19.1}
Suppose $\ord_{p_l}(\delta_1) = 0$ and 
\[\sum_{i = 1}^v n_{i}a_{li} > \frac{1}{p_l-1} - \ord_{p_l}(\delta_2).\]
Then
\[\ord_{p_l}(\Lambda_l') = \sum_{i = 1}^v n_{i}a_{li} + \ord_{p_l}(\delta_2) - \ord_{p_l}(\alpha_{\hat{i}}).\]
\end{lemma}

\begin{proof}
Immediate from Lemma \ref{Lem:discG} and Lemma \ref{Lem:Lambda}. 
\end{proof}

We now describe the $p_l$-adic reduction procedure. Recall that $l_v$ is a constant such that
\[\frac{l_v}{\log(p)} \geq \max\left( \frac{1}{p-1}, \ord_{p_v}(\delta_1)\right) - \ord_{p_v}(\delta_2).\]
Now, let $l_v'$ (denoted $\mu$ in BeGhKr and $m$ in TdW) be the largest element of $\mathbb{Z}_{\geq 0}$ at most
\[l_v' \leq \frac{l_v}{\log(p)} - \ord_{p_l}(\alpha_{\hat{i}}) + \ord_{p_l}(\delta_2).\]
We will use the notation $l_v'$ and $\mu$ interchangeably. Eventually we should use consistent notation here, but we will just use $\mu$ for now in place of $l_v'$. 

For each $x \in \mathbb{Z}_{p_l}$, let $x^{\{\mu\}}$ denote the unique rational integer in $[0,p_l^{\mu} - 1]$ such that $\ord_{p_l}(x - x^{\mu}) \geq \mu$ (ie. $x \equiv x^{\{\mu\}} \pmod{p_l^{\mu}}$). That is, 
\[x \equiv x^{\{\mu\}} \pmod{p_l^{\mu}} \implies x - x^{\{\mu\}} =\alpha p_l^{\mu}\]
for some $\alpha \in \mathbb{Z}$. Hence $x \equiv x^{\{\mu\}} \pmod{p_l^{j}}$ for $j =1, \dots, \mu$. In other words, we must have
\[x = a_0+ a_1 p + \cdots + a_n p^n + \cdots \quad \text{ and } \quad x^{\{\mu\}} = a_0+ a_1 p + \cdots + a_{\mu - 1}p^{\mu - 1}.\]
Then 
\[x - x^{\{\mu\}} = a_{\mu}p^{\mu} + \cdots + a_n p^n + \cdots \implies x - x^{\{\mu\}} \equiv 0 \pmod{p^{\mu}}\]
so that the highest power dividing $x - x^{\{\mu\}}$ is at least $\mu$. Recall, the order is the first non-zero term appearing in the series expansion of $x - x^{\{\mu\}}$, and thus $a_{\mu}$ may or may not be the first non-zero term, hence the order is at least $\mu$, though can be larger.

Let $\Gamma_{\mu}$ be the $(\nu+r)$-dimensional translated lattice $A_{\mu}x + w$, where $A_{\mu}$ is the diagonal matrix having $\hat{i}^{\text{th}}$ row 
\[\left(\beta_2^{\{\mu\}}, \cdots, \beta_{\hat{i} - 1}^{\{\mu\}}, p_l^{\mu}, \beta_{\hat{i} + 1}^{\{\mu\}}, \cdots, \beta_{1+ \nu+ r}^{\{\mu\}}\right) \in \mathbb{Z}^{\nu+r}.\]
Here, $p_l^{\mu}$ is the $(\hat{i},\hat{i})$ entry of $A_{\mu}$. That is, 
\[A_{\mu} = 
\begin{pmatrix}
1	& 		&		&		&		&		&	\\
	& \ddots	& 		&		& 0		& 		&	\\
	&		& 1		&		&		&		&	\\
	\beta_2^{\{\mu\}}& \cdots & \beta_{\hat{i} - 1}^{\{\mu\}} & p_l^{\mu} & \beta_{\hat{i} + 1}^{\{\mu\}}& \cdots &\beta_{1+ \nu+ r}^{\{\mu\}}\\
	& 		& 		& 		& 1		&		&	\\	
	& 0		& 		& 		&		& \ddots	&	\\	
	& 		& 		& 		&		& 		& 1	\\	
\end{pmatrix}.\]
Additionally, $w$ is the vector whose only non-zero entry is the $\hat{i}^{\text{th}}$ element, $ \beta_1^{\{\mu\}}$,
\[w = (0, \dots 0, \beta_1^{\{\mu\}},0, \dots, 0)^T \in \mathbb{Z}^{\nu + r}.\]

Of course, we must compute the $\beta_i$ to $p_l$-adic precision at least $\mu$ in order to avoid errors here. 
Let $\gamma = (n_1, \dots, n_{\nu}, a_1, \dots, a_r) \in \mathbb{R}^{\nu + r}$ be a solution to our $S$-unit equation. 

\begin{lemma}
Suppose $\ord_{p_l}(\delta_1) = 0$ and 
\[\sum_{i = 1}^v n_{i}a_{li} > \frac{1}{p_l-1} - \ord_{p_l}(\delta_2).\]
Then the following equivalence holds: 
\begin{align*}
\sum_{i = 1}^v n_{i}a_{li}  \geq \mu - \ord_{p_l}(\delta_2) + \ord_{p_l}(\alpha_{\hat{i}}) 
	& \quad \text{ if and only if } \quad \ord_{p_l}(\Lambda_l') \geq \mu \\
	& \quad \text{ if and only if } \quad \gamma \in\Gamma_v.
\end{align*}
\end{lemma} 

\begin{remark}
Note that the conditions $\ord_{p_l}(\delta_1) = 0$ and 
\[\sum_{i = 1}^v n_{i}a_{li} > \frac{1}{p_l-1} - \ord_{p_l}(\delta_2)\]
are equivalent to 
\[\sum_{i = 1}^v n_{i}a_{li} > \max\left\{\frac{1}{p_l-1}, \ord_{p_l}(\delta_1)\right\} - \ord_{p_l}(\delta_2).\]
\end{remark}

\begin{proof}
By Lemma \ref{Lem:19.1}, the assumption means that 
\[\ord_{p_l}(\Lambda_l') = \sum_{i = 1}^v n_{i}a_{li} + \ord_{p_l}(\delta_2) - \ord_{p_l}(\alpha_{\hat{i}}).\]

Now, suppose 
\[\sum_{i = 1}^v n_{i}a_{li}  \geq \mu - \ord_{p_l}(\delta_2) + \ord_{p_l}(\alpha_{\hat{i}}).\]
We thus have
\begin{align*}
\ord_{p_l}(\Lambda_l')	
	& = \sum_{i = 1}^v n_{i}a_{li} + \ord_{p_l}(\delta_2) - \ord_{p_l}(\alpha_{\hat{i}})\\
	& \geq  \mu - \ord_{p_l}(\delta_2) + \ord_{p_l}(\alpha_{\hat{i}}) + \ord_{p_l}(\delta_2) - \ord_{p_l}(\alpha_{\hat{i}})\\	
	& = \mu.
\end{align*}
Conversely, suppose $\ord_{p_l}(\Lambda_l') \geq \mu$. Then
\[\mu \leq \ord_{p_l}(\Lambda_l') = \sum_{i = 1}^v n_{i}a_{li} + \ord_{p_l}(\delta_2) - \ord_{p_l}(\alpha_{\hat{i}}).\]
That is, 
\[\sum_{i = 1}^v n_{i}a_{li} \geq \mu - \ord_{p_l}(\delta_2) + \ord_{p_l}(\alpha_{\hat{i}}).\]
Hence, it follows that $\displaystyle \sum_{i = 1}^v n_{i}a_{li} \geq \mu - \ord_{p_l}(\delta_2) + \ord_{p_l}(\alpha_{\hat{i}})$ if and only if $\ord_{p_l}(\Lambda_l') \geq \mu$.

Now, suppose $\gamma = (n_1, \dots, n_{\nu}, a_1, \dots, a_r) \in \mathbb{R}^{\nu + r}$ is a solution to our $S$-unit equation. Suppose further that $\displaystyle \sum_{i = 1}^v n_{i}a_{li} \geq \mu - \ord_{p_l}(\delta_2) + \ord_{p_l}(\alpha_{\hat{i}})$ so that $\ord_{p_l}(\Lambda_l') \geq \mu$. Let
\[\lambda = \frac{1}{p^{\mu}}\sum_{i = 1}^{\nu + r + 1}b_i(-\beta_i^{\{\mu\}})\]
and consider the $(\nu + r)$-dimensional vector
\[x= (n_1, \dots, n_{\hat{i} -1}, \lambda, n_{\hat{i} +1}, \dots, n_{\nu}, a_1, \dots, a_r)^T.\]
We claim $x \in \mathbb{Z}^{\nu + r}$. That is, $\lambda \in \mathbb{Z}$, meaning that $\sum_{i = 1}^{\nu + r + 1}b_i(-\beta_i^{\{\mu\}})$ is divisible by $p^{\mu}$, or equivalently, 
\[\ord_p\left(\sum_{i = 1}^{\nu + r + 1}b_i(-\beta_i^{\{\mu\}})\right) \geq \mu.\]
Indeed, since 
\[\ord_{p_{l}}\left(\beta_{i}^{\{\mu\}}-\beta_{i}\right) \geq \mu \quad \text { for } i=1, \dots, 1+\nu+r,\]
by definition, it follows that $\beta_{i}^{\{\mu\}}$ and $\beta_{i}$ share the first $\mu - 1$ terms and thus $\ord_p(\beta_i) = \ord_p(\beta_i^{\{\mu\}})$.
Now, to compute this order, we only need to concern ourselves with the first non-zero term in the series expansion of $\sum_{i = 1}^{\nu + r + 1}b_i(-\beta_i^{\{\mu\}})$. Since $\beta_{i}^{\{\mu\}}$ and $\beta_{i}$ share the first $\mu - 1$ terms, it follows that showing
\[\ord_p\left(\sum_{i = 1}^{\nu + r + 1}b_i(-\beta_i^{\{\mu\}})\right) \geq \mu\]
is equivalent to showing that 
\[\ord_p\left(\sum_{i = 1}^{\nu + r + 1}b_i(-\beta_i)\right) \geq \mu \implies \ord_{p_l}(\Lambda_l') \geq \mu.\]
This latter inequality is true by assumption. Thus $\lambda \in \mathbb{Z}$. 

Then, computing $A_{\mu}x + w$ yields
\begin{align*}
A_{\mu}x + w & = 
\begin{pmatrix}
1	& 		&		&		&		&		&	\\
	& \ddots	& 		&		& 0		& 		&	\\
	&		& 1		&		&		&		&	\\
	\beta_2^{\{\mu\}}& \cdots & \beta_{\hat{i} - 1}^{\{\mu\}} & p_l^{\mu} & \beta_{\hat{i} + 1}^{\{\mu\}}& \cdots &\beta_{1+ \nu+ r}^{\{\mu\}}\\
	& 		& 		& 		& 1		&		&	\\	
	& 0		& 		& 		&		& \ddots	&	\\	
	& 		& 		& 		&		& 		& 1	\\	
\end{pmatrix}
\begin{pmatrix}
n_1 \\ \vdots \\ n_{\hat{i} -1} \\ \lambda \\ n_{\hat{i} +1} \\ \vdots \\ n_{\nu} \\ a_1 \\ \vdots \\ a_r \end{pmatrix}
+ \begin{pmatrix}
0 \\ \vdots \\ 0 \\ \beta_1^{\{\mu\}} \\ 0\\ \vdots \\ \vdots \\ \vdots \\ 0
\end{pmatrix}\\
&= \begin{pmatrix}
1	& 		&		&		&		&		&	\\
	& \ddots	& 		&		& 0		& 		&	\\
	&		& 1		&		&		&		&	\\
	\beta_2^{\{\mu\}}& \cdots & \beta_{\hat{i} - 1}^{\{\mu\}} & p_l^{\mu} & \beta_{\hat{i} + 1}^{\{\mu\}}& \cdots &\beta_{1+ \nu+ r}^{\{\mu\}}\\
	& 		& 		& 		& 1		&		&	\\	
	& 0		& 		& 		&		& \ddots	&	\\	
	& 		& 		& 		&		& 		& 1	\\	
\end{pmatrix}\begin{pmatrix}
b_2 \\ \vdots \\ b_{\hat{i} -1} \\ \lambda \\ b_{\hat{i} +1} \\ \vdots \\ b_{\nu + r + 1} 
\end{pmatrix}
+ \begin{pmatrix}
0 \\ \vdots \\ 0 \\ \beta_1^{\{\mu\}} \\ 0 \\ \vdots \\ 0 
\end{pmatrix}\\
& = \begin{pmatrix}
b_2 \\ \vdots \\ b_{\hat{i} -1} \\ 
 b_2\beta_2^{\{\mu\}} + \cdots + b_{\hat{i} - 1}\beta_{\hat{i} - 1}^{\{\mu\}} + \lambda p_l^{\mu} + b_{\hat{i} + 1}\beta_{\hat{i} + 1}^{\{\mu\}} + \cdots + b_{\nu+r+ 1}\beta_{1+\nu+r}^{\{\mu\}} + \beta_1^{\{\mu\}} \\
b_{\hat{i} +1} \\ \vdots \\ b_{\nu+r+1} \\
\end{pmatrix}\\
\end{align*}
Now, 
\[ \lambda p_l^{\mu} = p^{\mu}\frac{1}{p^{\mu}}\sum_{i = 1}^{\nu + r + 1}b_i(-\beta_i^{\{\mu\}}) = \sum_{i = 1}^{\nu + r + 1}b_i(-\beta_i^{\{\mu\}}),\]
hence
\begin{align*}
& b_2\beta_2^{\{\mu\}} + \cdots + b_{\hat{i} - 1}\beta_{\hat{i} - 1}^{\{\mu\}} + b_{\hat{i} + 1}\beta_{\hat{i} + 1}^{\{\mu\}} + \cdots + b_{\nu+r+ 1}\beta_{1+\nu+r}^{\{\mu\}} + \lambda p_l^{\mu} + \beta_1^{\{\mu\}}\\
& = b_1 \beta_1^{\{\mu\}} + b_2\beta_2^{\{\mu\}} + \cdots + b_{\hat{i} - 1}\beta_{\hat{i} - 1}^{\{\mu\}} + b_{\hat{i} + 1}\beta_{\hat{i} + 1}^{\{\mu\}} + \cdots + b_{\nu+r+ 1}\beta_{1+\nu+r}^{\{\mu\}} + \sum_{i = 1}^{\nu + r + 1}b_i(-\beta_i^{\{\mu\}}) \\
& = b_{\hat{i}}(-\beta_{\hat{i}}^{\{\mu\}})\\
& = b_{\hat{i}}
\end{align*}
where the last equality follows from the fact that 
\[-\beta_i = \frac{\alpha_{\hat{i}}}{\alpha_{\hat{i}}} =1.\]
Thus, 
\[A_{\mu}x + w = \begin{pmatrix}
b_2 \\ \vdots \\ b_{\hat{i} -1} \\ b_{\hat{i}} \\ b_{\hat{i} +1} \\ \vdots \\ b_{\nu+r+1}
\end{pmatrix} = 
\begin{pmatrix}
n_1 \\ \vdots \\ n_{\nu} \\ a_1 \\ \vdots \\ a_r \end{pmatrix} = \gamma.\]
Thus, it follows that $\gamma \in \Gamma_v$. 
CONVERSELY STILL NEED TO SHOW THE CONVERSE, THAT IS 
\[m'\in\Gamma_v \quad \text{ implies } \quad \sum_{i = 1}^v n_{i}a_{li}  \geq \mu - \ord_{p_l}(\delta_2) + \ord_{p_l}(\alpha_{\hat{i}}).\]


%Our assumption gives $n_p-a_p\geq \max(0,\ord_p(\mu_0))-\ord_p(\lambda_0)$ and then Lemma~\ref{lem:padiccomp} implies that $\ord_p(\mu_0)=0$. Therefore, on using again our assumption which assures that $n_p-a_p> \tfrac{1}{p-1}-\ord_p(\lambda_0)$, we see that an application of Lemma~\ref{lem:padiccomp}  gives  $$(n_p-a_p)+\ord_p(\lambda_0)=\ord_p(\Lambda_p)=\ord_p(\Lambda'_p)+\ord_p(\xi_p).$$
%Thus $n_p-a_p\geq l_v'-\ord_p(\xi_p/\lambda_0)$ if and only if $\ord_p(\Lambda'_p)\geq l_v'$. Further, the TdW arguments show that $\ord_p(\Lambda'_p)\geq l_v'$ if and only if $m'\in\Gamma_v$. On combining we deduce the lemma.
\end{proof}

\begin{remark}
In the case $n=3$, the construction of $\Lambda'_p$ is simpler since there are only two cases to consider (either $g_p$ splits completely over $\mathbb{Q}_p$, or it has square factor).
\end{remark}


%
%
%
%
%
%
%
%
%
%
%
%Put
%\[Q = \sum_{i = 2}^{v+2} W_i^2 B_i^2.\]
%
%\begin{lem} \label{lem:LLL}
%If $\ell(\Gamma_{\mu},\mathbf{y}) > Q^{1/2}$ then
%\[\sum_{i = 1}^v n_{i}a_{li} \leq \max\left\{ \frac{1}{p_l-1} - \ord_{p_l}(\delta_2), \mu - d_l - 1,0\right\}\]
%\end{lem}
%
%\begin{proof}
%We prove the contrapositive. Assume 
%\[\sum_{i = 1}^v n_{i}a_{li} > \frac{1}{p_l-1} - \ord_{p_l}(\delta_2), \quad \sum_{i = 1}^v n_{i}a_{li} > \mu - d_l 
%\quad \text{ and } \quad \sum_{i = 1}^v n_{i}a_{li} > 0.\]
%Consider the vector
%\[\mathbf{x} = A_{\mu}
%\begin{pmatrix}
%b_2\\
%\vdots\\
%b_{\hat{i}-1}\\
%b_{\hat{i}+1}\\
%\vdots\\
%b_{v+2}\\
%\lambda
%\end{pmatrix}
%= 
%\begin{pmatrix}
%W_2b_2\\
%\vdots\\
%W_{\hat{i}-1}b_{\hat{i}-1}\\
%W_{\hat{i}+1}b_{\hat{i}+1}\\
%\vdots\\
%W_{v+2}b_{v+2}\\
%-W_{\hat{i}}b_{\hat{i}}
%\end{pmatrix}
%+ \mathbf{y}.\]
%By Lemma~\ref{Lem:19.1},  
%\[\ord_{p_l}\left( \sum_{i=1}^{v+2}b_i(-\beta_i)\right) = \ord_{p_l}(\Lambda_l') \geq\sum_{i = 1}^v n_{i}a_{li} + d_l \geq \mu.\]
%Since $\ord_{p_l}(\beta_i^{\{\mu\}} - \beta_i) \geq \mu$ for $i = 1, \dots, v+2$, it follows that
%\[\ord_{p_l}\left( \sum_{i=1}^{v+2}b_i(-\beta_i^{\{\mu\}})\right) \geq \mu,\]
%so that $\lambda \in \mathbb{Z}$. Hence $\mathbf{x} \in \Gamma_{\mu}$. Now $\sum_{i = 1}^v n_{i}a_{li} > 0$ so that there exists some $i$ such that $n_ia_{li} \neq 0$, and in particular, $b_{1+i} = n_i \neq 0$. Thus we cannot have $\mathbf{x} = \mathbf{y}$. Therefore, 
%\[\ell(\Gamma_{\mu}, \mathbf{y})^2 \leq |\mathbf{x} - \mathbf{y}|^2 = \sum_{i = 2}^{v+2}W_i^2 b_i^2
%\leq  \sum_{i = 2}^{v+2}W_i^2 |b_i|^2 \leq  \sum_{i = 2}^{v+2}W_i^2 B_i^2 = Q.\]
%\end{proof}

%The reduction procedure works as follows. Taking $A_{\mu}$ as input, we first compute an LLL-reduced basis for $\Gamma_{\mu}$. Then, we find a lower bound for $\ell(\Gamma_{\mu}, \mathbf{y})$. If the lower bound is not greater than $Q^{1/2}$ so that Lemma \ref{lem:LLL} does not give a new upper bound, we increase $\mu$ and try the procedure again. If we find that several increases of $\mu$ have failed to yield a new upper bound $N_l$ and that the value of $\mu$ has become significantly larger than it was initially, we move onto the next $l \in \{1, \dots, v\}$.
%
%If the lower bound is greater than $Q^{1/2}$, Lemma \ref{lem:LLL} gives a new upper bound $N_l$ for $\sum_{i = 1}^v n_{i}a_{li}$ and hence for $m$
%\[m = \frac{\sum_{j = 1}^{v}n_ja_{lj} + r_l + t_l}{\alpha_l} < \frac{N_l+ r_l + t_l}{\alpha_l} = M.\]
%If $M < 3000$, we exit the algorithm and enter the brute force search. Otherwise, we update the bounds $N_1, \dots, N_{l-1}, N_{l+1}, \dots, N_v$ via
%\[\sum_{j=1}^v n_ja_{ij} = m\alpha_i - r_i - t_i \leq M\alpha_i - r_i - t_i = N_i.\]
%Then using 
%\[|n_l| \leq \max_{1 \leq i \leq v}|n_i| \leq ||A^{-1}||_{\infty}\max_{1 \leq i\leq v}\sum_{j = 1}^v n_j a_{ij}
%\leq ||A^{-1}||_{\infty} \max_{1 \leq i\leq v}(N_i) = B_{l+1}.\]
%we update the $B_i$ and repeat the above procedure until $M < 3000$ or until no further improvement can be made on the $B_i$, in which case we move onto the next $l \in \{1, \dots, v\}$.

We define 
\[c_p=\log p\left(\max\left(\tfrac{1}{p-1},\ord_{p_l}(\delta_1)\right)-\ord_{p_l}(\delta_2)\right).\]

\begin{corollary}
Assume that $h_{p_l}(z)>\max(0,c_p)$. Then the following equivalence holds: 
\[h_{p_l}(z)\geq \log{p_l}\left(\mu - \ord_{p_l}(\delta_2) + \ord_{p_l}(\alpha_{\hat{i}})\right) \quad \text{ if and only if } \quad \gamma \in\Gamma_v.\]
\end{corollary}
\begin{proof}
Recall from Proposition~\ref{prop:heightdecomp} that 
\[h_{p_l}(z) = 
\begin{cases}
\log(p_l)|u_l - r_l| \\
0
\end{cases}.\]
Since $h_{p_l}(z) > 0$, it follows that $h_{p_l}(z) = \log(p_l)|u_l - r_l|$. Hence the assumption becomes
\begin{align*}
&\log(p_l)|u_l - r_l| = h_{p_l}(z) > \log p\left(\max\left(\tfrac{1}{p-1},\ord_{p_l}(\delta_1)\right)-\ord_{p_l}(\delta_2)\right)\\
&|u_l - r_l| = h_{p_l}(z) > \left(\max\left(\tfrac{1}{p-1},\ord_{p_l}(\delta_1)\right)-\ord_{p_l}(\delta_2)\right) \\
& \sum_{j = 1}^{\nu}n_ja_{lj} > \left(\max\left(\tfrac{1}{p-1},\ord_{p_l}(\delta_1)\right)-\ord_{p_l}(\delta_2)\right)
\end{align*}
with conclusion
\begin{align*}
h_{p_l}(z)\geq \log{p_l}\left(\mu - \ord_{p_l}(\delta_2) + \ord_{p_l}(\alpha_{\hat{i}})\right) 
	& \quad \text{ if and only if } \quad \gamma \in\Gamma_v\\
\log(p_l)|u_l - r_l| \geq \log{p_l}\left(\mu - \ord_{p_l}(\delta_2) + \ord_{p_l}(\alpha_{\hat{i}})\right) 
	&\quad \text{ if and only if } \quad \gamma \in\Gamma_v\\
|u_l - r_l| \geq \left(\mu - \ord_{p_l}(\delta_2) + \ord_{p_l}(\alpha_{\hat{i}})\right) 
	& \quad \text{ if and only if } \quad \gamma \in\Gamma_v\\
\sum_{j = 1}^{\nu}n_ja_{lj} \geq \left(\mu - \ord_{p_l}(\delta_2) + \ord_{p_l}(\alpha_{\hat{i}})\right) 
	&\quad \text{ if and only if } \quad \gamma \in\Gamma_v,
\end{align*}
which is the previous lemma. 

%This follows from the above lemma, since Proposition~\ref{prop:heightdecomp} together with $h_p(z)>0$ implies that $h_p(z)=\log p(n_p-a_p)$.
\end{proof}

Recall that we wish to prove the following lemma:
\begin{lemma}\label{lem:nonarchsieve}
Any $(x,y) \in \Sigma_v(l,h)$ is determined by some $\gamma \in \Gamma_v \cap \mathcal{E}_v$. 
\end{lemma}

\begin{proof}[Proof of Lemma~\ref{lem:nonarchsieve}]
If $(n_1, \dots, n_{\nu}, a_1, \dots, a_r) \in \mathbb{R}^{r+\nu}$ is a solution which lies in $\Sigma_v(l,h)$, then, by definition, it corresponds to a solution $(x,y)$ satisfying
\[\Sigma_v(l,h) = \{(x,y) \in \Sigma \ | \ (h_w(z))\leq h \text{ and }  (h_w(z))\nleq 0 \text{ and } h_v(z)>l_v\}.\]
Hence $h_v(z)>l_v$, where $l_v$ is a constant such that
\[\frac{l_v}{\log(p)} \geq \max\left( \frac{1}{p-1}, \ord_{p_v}(\delta_1)\right) - \ord_{p_v}(\delta_2).\]
That is, 
\[h_v(z) > l_v \geq \log(p)\left(\max\left( \frac{1}{p-1}, \ord_{p_v}(\delta_1)\right) - \ord_{p_v}(\delta_2)\right) = c_p.\]
Now, recall that $l \geq 0$ so that $l_v \geq 0$. It thus follows that 
\[h_v(z) > l_v \geq
\begin{cases}
0\\
c_p
\end{cases}
\implies h_v(z) >\max(0,c_p).\]
In other words, the condition of the previous corollary is satisfied. 

Now, recall that $l_v'$ (sometimes denoted $\mu$) is the largest element of $\mathbb{Z}_{\geq 0}$ at most
\[l_v' \leq \frac{l_v}{\log(p)} - \ord_{p_l}(\alpha_{\hat{i}}) + \ord_{p_l}(\delta_2).\]
That is
\[\frac{l_v}{\log(p)} \geq l_v' + \ord_{p_l}(\alpha_{\hat{i}}) - \ord_{p_l}(\delta_2)\]
so that
\[h_v(z) > l_v \geq \log(p)\left(\l_v' + \ord_{p_l}(\alpha_{\hat{i}}) - \ord_{p_l}(\delta_2)\right).\]
Now, by the previous corollary, we must have $\gamma \in \Gamma_v$. This shows that $(x,y)$ is determined by $\gamma=m'\in\Gamma_v$, which proves Lemma~\ref{lem:nonarchsieve}.
%
%If $h_v(z)>l_v$ then $h_v(z)>\max(0,c_p)$ since $l\geq 0$ and $l_v\geq c_p$ by assumption. Further, it holds that $l_v/\log p\geq l_v'-v(\xi_p/\lambda_0)$ by the definition of $l_v'$. Hence $h_v(z)\geq \log p(l_v'-v(\xi_p/\lambda_0))$ and then the above corollary implies that $m'\in\Gamma_v$. This shows that $(x,y)$ is determined by $\gamma=m'\in\Gamma_v$, which proves Lemma~\ref{lem:nonarchsieve}.
\end{proof}


\subsection{Non-Archimedean ellipsoid.} 
Recall that
\[h\left(\frac{\delta_2}{\lambda}\right) =  \frac{1}{[K:\mathbb{Q}]}\sum_{l = 1}^{\nu} \log(p_l)|u_l - r_l| + \frac{1}{[L:\mathbb{Q}]}\sum_{w :L \to \mathbb{C}} \log \max \left\{ \left|w\left(\frac{\delta_2}{\lambda}\right)\right|, 1\right\}.\]

We now restrict our attention to those $p \in \{p_1, \dots, p_{\nu}\}$ and study the $p$-adic valuations of the numbers appearing in \eqref{Eq:Sunit}. Let $l \in \{1, \dots, \nu\}$, corresponding to $p_l \in \{p_1, \dots, p_{\nu}\}$. Take $\mathbf{h}\in\mathbb{R}^{r+\nu}$ such that $\mathbf{h}\geq \mathbf{0}$. Let
\[b = \frac{1}{\log(2)^2}\sum_{k = 1}^{\nu} h_k^2\]
where
\[\log(2)^2q_f(\mathbf{n}) = \log(2)^2\sum_{k = 1}^{\nu}\left\lfloor\frac{\log(p_k)^2}{\log(2)^2}\right\rfloor|u_k-r_k|^2 \leq \sum_{k = 1}^{\nu} \log(p_k)^2|u_k -r_k|^2 \leq \sum_{k = 1}^{\nu} h_k^2.\]

%\[q_f(\mathbf{n}) = \sum_{l = 1}^{\nu} \lfloor\log(p_l)^2\rfloor|u_l -r_l|^2 \leq \sum_{l = 1}^{\nu} \log(p_l)^2|u_l -r_l|^2 \leq \sum_{l = 1}^{\nu} h_k^2:= b \]

%\[q_f(\mathbf{n}) = \frac{1}{[K:\mathbb{Q}]}\sum_{k = 1}^{\nu} \log(p_k)^2|u_k -r_k|^2 \leq \frac{1}{[K:\mathbb{Q}]}\sum_{k = 1}^{\nu} h_k^2 = b.\]
For each $\varepsilon_l$ in $\{\varepsilon_1, \dots, \varepsilon_r\}$, we define
%\[b_{\varepsilon_l} = 
%\begin{cases}
%\left( \frac{2}{[L:\mathbb{Q}]}\max_{\sigma:L\to \mathbb{C}} w_{\varepsilon \sigma}h_{\sigma}  + \frac{1}{[K:\mathbb{Q}]}\sum_{k = 1}^{\nu} w_{\gamma k}h_k\right)^2 & \text{ if } \sqrt{\Delta}\notin\mathbb{Q} \\
%\left( \frac{1}{[L:\mathbb{Q}]}\max_{\sigma:L\to \mathbb{C}} w_{\varepsilon \sigma}h_{\sigma} + \frac{1}{[K:\mathbb{Q}]}\sum_{k = 1}^{\nu} w_{\gamma k}h_k\right)^2 & \text{ if } \sqrt{\Delta}\in\mathbb{Q}, \\
%\end{cases}\]
%where
\[|a_l|^2 \leq \left( \frac{1}{[K:\mathbb{Q}]}\sum_{k = 1}^{\nu} w_{\gamma l k}h_k + \frac{1}{[L:\mathbb{Q}]}\sum_{\sigma:L\to \mathbb{C}} w_{\varepsilon l \sigma}h_{\sigma}\right)^2=:b_{\varepsilon_l}.\]
Note that here, we do not distinguish any $\varepsilon_l^*$. 

We define the ellipsoid $\mathcal{E}_l \subseteq \mathbb{R}^{\nu + r}$ by 
\begin{align}\label{def:ellp}
& \mathcal{E}_l=\{q_l(\mathbf{x})\leq (1 + r)(bb_{\varepsilon_1}\cdots b_{\varepsilon_r}); \ \mathbf{x}\in\mathbb{R}^{r+\nu}\}, \quad \text{ where }\\
&\quad \quad \quad q_l(\mathbf{x})= (b_{\varepsilon_1}\cdots b_{\varepsilon_r})\left( q_f(x_1, \dots, x_{\nu}) + \sum_{i = 1}^r\frac{b}{b_{\varepsilon_i}}x_{\varepsilon_i}^2\right)\\
&\quad \quad \quad q_l(\mathbf{x})=\left((b_{\varepsilon_1}\cdots b_{\varepsilon_r})\cdot q_f(x_1, \dots, x_{\nu}) + (b_{\varepsilon_1}\cdots b_{\varepsilon_r})\sum_{i = 1}^r\frac{b}{b_{\varepsilon_i}}x_{\varepsilon_i}^2\right)\\
&\quad \quad \quad q_l(\mathbf{x})=\left((b_{\varepsilon_1}\cdots b_{\varepsilon_r})\cdot q_f(x_1, \dots, x_{\nu}) + \sum_{i = 1}^rb(b_{\varepsilon_1}\cdots b_{\varepsilon_{i-1}}b_{\varepsilon_{i+1}}b_{\varepsilon_r})x_{\varepsilon_i}^2\right)
\end{align}
where
\[q_f(\mathbf{y}) = (A\mathbf{y})^{\text{T}}D^2A\mathbf{y}.\]

To generate the matrix for this ellipsoid, recall that $I$ is the set of all indices $i' \in \{2, \dots, 1+ \nu + r\}$ for which
\[\ord_{p_l}(\alpha_{i'}) = \min_{2\leq i\leq 1+ \nu+ r} \ord_{p_l}(\alpha_i).\]
We note that we are always in the so-called \textit{special case}, where there is some index $i' \in I$ such that $\alpha_i/\alpha_{i'} \in \mathbb{Q}_{p_l}$ for $i = 1, \dots, 1+ \nu+ r$. 

Now we state several relatively-easy-to-check conditions that each imply that we are always in the special case for degree $3$ Thue-Mahler equations. Moreover, each condition implies that we have $\frac{\alpha_{i_1}}{\alpha_{i_2}} \in \mathbb{Q}_p$ for every $i_1, i_2 \in\{1, \dots, 1+\nu+r\}$.

\begin{enumerate}[(a)]
\item $\alpha_1, \dots, \alpha_{1+\nu+r} \in \mathbb{Q}_p$
\item $g(t)$ has three or more linear factors in $\mathbb{Q}_p[t]$ and $\theta^{(i_0)}, \theta^{(j)}, \theta^{(k)}$ are roots of such polynomials. 
\item $g(t)$ has an irreducible factor in $\mathbb{Q}_p[t]$ of degree two, and $\theta^{(j)}, \theta^{(k)}$ are roots of this
\item $g(t)$ has a non-linear irreducible factor in $\mathbb{Q}_p[t]$ that splits completely in the extension of $\mathbb{Q}_p$ that it generates and $\theta^{(j)}, \theta^{(k)}$ are roots of this factor
\end{enumerate}

\begin{proof}
It is obvious that (a) implies $\alpha_{i_1}/\alpha_{i_2} \in \mathbb{Q}_p$ for every $i_1, i_2 \in\{1, \dots, 1+\nu+r\}$. If (b) holds, then $\delta_1, \gamma_{i}^{(k)} / \gamma_{i}^{(j)} (i=1, \ldots, \nu), \varepsilon_{i}^{(k)} / \varepsilon_{i}^{(j)} (i=1, \ldots, r)$ all belong to $\mathbb{Q}_p$, which, since $\mathbb{Q}_p$ is complete, implies (a). Now, (c) implies (d). We claim that (d) implies $\alpha_{i_1}/\alpha_{i_2} \in \mathbb{Q}_p$ for every $i_1, i_2 \in\{1, \dots, 1+\nu+r\}$. To see this, assume (d), let $L$ be the extension of $\mathbb{Q}_p$ generated by the factor of $g(t)$ in question, and consider any $\alpha, \beta \in L$. The automorphisms on $L$ that maps $\theta^{(j)}$ to $\theta^{(k)}$ multiplies the logarithms $\log _{p_{l}}\left(\alpha^{(k)} / \alpha^{(j)}\right)$ and $\log _{p_{l}}\left(\beta^{(k)} / \beta^{(j)}\right)$ by $-1$ and hence fixes the quotient
\begin{equation}\label{logs}
\frac{\log _{p_{l}}\left(\alpha^{(k)} / \alpha^{(j)}\right)}{\log _{p_{l}}\left(\beta^{(k)} / \beta^{(j)}\right)}.
\end{equation}
Therefore, since $L$ is Galois, this quotient belongs to $\mathbb{Q}_p$. Since $\alpha_{i_1}/\alpha_{i_2}$ is of the form \eqref{logs} for every $i_1, i_2 \in\{1, \dots, 1+\nu+r\}$, the claim is proved. 
\end{proof}

Now, recall that if our Thue-Mahler is only of degree $3$, it follows that $g(t)$ can only split in 3 ways in $\mathbb{Q}_p$. 
\begin{enumerate}[(a)]
\item $g(t) = g_1(t)$, where $\deg(g_1(t)) = 3$
\item $g(t) = g_1(t)g_2(t)$ where $\deg(g_1(t)) = 1$ and $\deg(g_2(t)) = 2$ (without loss of generality) 
\item $g(t) = g_1(t)g_2(t)g_3(t)$ where $\deg(g_i(t)) = 3$ for $i = 1, 2, 3$. 
\end{enumerate}
In the event that that $g(t)$ is irreducible (a), the corresponding prime ideal $\mathfrak{p}$ in $K$ has $ef = 3$, and is therefore bounded. That is, it does not appear in the set of unbounded ideals $\{\mathfrak{p}_1, \dots, \mathfrak{p}_{\nu}\}$, and we can ignore this case. The other 2 cases appear in the list above, therefore guaranteeing that $\alpha_{i_1}/\alpha_{i_2} \in \mathbb{Q}_p$ for every $i_1, i_2 \in\{1, \dots, 1+\nu+r\}$. 


% and take $h\in\mathbb{R}^{}$ with $h\geq 0$. Let $b=b(h)$ be the real number defined in \eqref{def:bbound} and for any $\epsilon\in\unit_\infty$ we define $b_\epsilon=b_\epsilon(h)$ as in \eqref{def:bepsbound}. Then we define the ellipsoid $\mathcal E_v\subseteq \RR^{\unit}$ by
%\begin{equation}\label{def:ellnonarch}
%\mathcal E_v=\{q_v(x)\leq b; \ x\in\RR^\unit\}, \quad q_v(x)=\frac{1}{|\unit|}\left(|\unit_S|q_f(x_\delta)+\sum_{\epsilon\in\unit_\infty}\frac{b}{b_\epsilon}x_{\epsilon}^2\right)
%\end{equation}
%where $x=(x_\delta)\oplus(x_\epsilon)$ with $(x_\delta)\in\RR^{\unit_S}$ and $(x_\epsilon)\in \RR^{\unit_\infty}$.
%
%

Finally, suppose that $\gamma\in \Gamma_v\cap \mathcal E_v$. Let $M=M_v$ be the matrix defining the ellipsoid 
\[\mathcal E_\tau: z^tM^tMz\leq (1 + r)(bb_{\varepsilon_1}\cdots b_{\varepsilon_r}),\]
that is,  
\begin{align*}
M &=\sqrt{b_{\varepsilon_1}\cdots b_{\varepsilon_r}}\begin{pmatrix}
	DA & 0 & \dots & 0 & 0\\
	0 & \sqrt{\frac{b}{b_{\varepsilon 1}}} & \dots & 0 & 0\\
	0 & 0  & \sqrt{\frac{b}{b_{\varepsilon 2}}} & \dots & 0\\
	\vdots & \vdots &0 &  \ddots & \vdots\\ 
	0 & 0 & \dots & \dots & \sqrt{\frac{b}{b_{\varepsilon}}} \\
	\end{pmatrix}.
\end{align*}	
Note that we never need to compute $M$, but rather $M^TM$ so that we do not need to worry about precision. In this case, 
\begin{align*}
M^TM &= b_{\varepsilon_1}\cdots b_{\varepsilon_r}\begin{pmatrix}
	A^TD^2A & 0 & \dots & 0 & 0\\
	0 & \frac{b}{b_{\varepsilon 1}} & \dots & 0 & 0\\
	0 & 0  & \frac{b}{b_{\varepsilon 2}} & \dots & 0\\
	\vdots & \vdots &0 &  \ddots & \vdots\\ 
	0 & 0 & \dots & \dots & \frac{b}{b_{\varepsilon}} \\
	\end{pmatrix}\\
	& = \begin{pmatrix}
	(b_{\varepsilon_1}\cdots b_{\varepsilon_r})A^TD^2A & 0 & \dots & 0 & 0\\
	0 & b b_{\varepsilon_2}\cdots b_{\varepsilon_r} & \dots & 0 & 0\\
	0 & 0  & bb_{\varepsilon_1}b_{\varepsilon_3}\cdots b_{\varepsilon_r} & \dots & 0\\
	\vdots & \vdots &0 &  \ddots & \vdots\\ 
	0 & 0 & \dots & \dots & bb_{\varepsilon_1}\cdots b_{\varepsilon_{r-1}} \\
	\end{pmatrix}.
\end{align*}	
	
Recall that 
\[A_{\mu}= \begin{pmatrix}
1	& 		&		&		&		&		&	\\
	& \ddots	& 		&		& 0		& 		&	\\
	&		& 1		&		&		&		&	\\
	\beta_2^{\{\mu\}}& \cdots & \beta_{\hat{i} - 1}^{\{\mu\}} & p_l^{\mu} & \beta_{\hat{i} + 1}^{\{\mu\}}& \cdots &\beta_{1+ \nu+ r}^{\{\mu\}}\\
	& 		& 		& 		& 1		&		&	\\	
	& 0		& 		& 		&		& \ddots	&	\\	
	& 		& 		& 		&		& 		& 1	\\	
\end{pmatrix}\]
$A_{\mu}x + w$ define the lattice $\Gamma_v$ where $\gamma\in \Gamma_v\cap \mathcal E_v$. In particular, since $\gamma\in \Gamma_v\cap \mathcal E_v$, there exists $x\in \mathbb{R}^{r + \nu}$ such that $\gamma=\Gamma_v x+w$ and ${\gamma^tM^tM\gamma\leq (1 + r)(bb_{\varepsilon_1}\cdots b_{\varepsilon_r})}$. We thus have
\[(\Gamma_v x+w)^tM^tM(\Gamma_\tau x+w) \leq (1 + r)(bb_{\varepsilon_1}\cdots b_{\varepsilon_r}).\]
As $A_{\tau}$ is clearly invertible, with matrix inverse
\[A_{\tau}^{-1} = \begin{pmatrix}
1	& 		&		&		&		&		&	\\
	& \ddots	& 		&		& 0		& 		&	\\
	&		& 1		&		&		&		&	\\
	-\frac{\beta_2^{\{\mu\}}}{p_l^{\mu}} & \cdots & -\frac{\beta_{\hat{i} - 1}^{\{\mu\}}}{p_l^{\mu}} & \frac{1}{p_l^{\mu}} & -\frac{\beta_{\hat{i} + 1}^{\{\mu\}}}{p_l^{\mu}}& \cdots &-\frac{\beta_{1+ \nu+ r}^{\{\mu\}}}{p_l^{\mu}}\\
	& 		& 		& 		& 1		&		&	\\	
	& 0		& 		& 		&		& \ddots	&	\\	
	& 		& 		& 		&		& 		& 1	\\
	\end{pmatrix},\]
we can find a vector $c$ such that $A_{\tau}c = -w$. Indeed, this vector is $c = A_{\tau}^{-1}(-w)$, where
\[c = A_{\tau}^{-1}w =  \begin{pmatrix}
1	& 		&		&		&		&		&	\\
	& \ddots	& 		&		& 0		& 		&	\\
	&		& 1		&		&		&		&	\\
	-\frac{\beta_2^{\{\mu\}}}{p_l^{\mu}} & \cdots & -\frac{\beta_{\hat{i} - 1}^{\{\mu\}}}{p_l^{\mu}} & \frac{1}{p_l^{\mu}} & -\frac{\beta_{\hat{i} + 1}^{\{\mu\}}}{p_l^{\mu}}& \cdots &-\frac{\beta_{1+ \nu+ r}^{\{\mu\}}}{p_l^{\mu}}\\
	& 		& 		& 		& 1		&		&	\\	
	& 0		& 		& 		&		& \ddots	&	\\	
	& 		& 		& 		&		& 		& 1	\\
	\end{pmatrix}\begin{pmatrix}
	0 \\ \vdots \\ 0 \\ -\beta_1^{\{\mu\}} \\ 0 \\ \vdots \\ 0
	\end{pmatrix}
	= \begin{pmatrix}
	0 \\ \vdots \\ 0 \\-\frac{\beta_1^{\{\mu\}}}{p^{\{\mu\}}} \\ 0 \\ \vdots \\ 0
\end{pmatrix}.\]
Now, 
\begin{align*}
(1 + r)(bb_{\varepsilon_1}\cdots b_{\varepsilon_r})
	& \geq (\Gamma_\tau x+w)^tM^tM(\Gamma_\tau x+w) \\
	& =  (\Gamma_\tau x- (-w))^tM^tM(\Gamma_\tau x-(-w)) \\
	& = (\Gamma_\tau x-\Gamma_{\tau}c)^tM^tM(\Gamma_\tau x-\Gamma_{\tau}c)\\
	& = (\Gamma_\tau (x-c))^tM^tM(\Gamma_\tau (x-c))\\
	& = (x-c)^t(M\Gamma_{\tau})^tM\Gamma_\tau(x-c)\\
	& = (x-c)^tB^tB(x-c)
\end{align*}
where $B = M\Gamma_\tau$. That is, we are left to solve
\[(x-c)B^tB(x-c) \leq (1 + r)(bb_{\varepsilon_1}\cdots b_{\varepsilon_r}).\]




Recall that $\gamma \in \Gamma_v$ means
\[A_{\mu}x + w = \begin{pmatrix}
b_2 \\ \vdots \\ b_{\hat{i} -1} \\ b_{\hat{i} +1} \\ \vdots \\ b_{\nu+r+1} \\ b_{\hat{i}}
\end{pmatrix} = \gamma.\]


%---------------------------------------------------------------------------------------------------------------------------------------------%

\endinput

Any text after an \endinput is ignored.
You could put scraps here or things in progress.

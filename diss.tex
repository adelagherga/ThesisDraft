%--------------------------------------------------------------------------------------------------------------------------------------------%
% Template for a UBC-compliant dissertation
% At the minimum, you will need to change the information found
% after the "Document meta-data"
%
%!TEX TS-program = pdflatex
%!TEX encoding = UTF-8 Unicode

%% The ubcdiss class provides several options:
%%   gpscopy (aka fogscopy)
%%       set parameters to exactly how GPS specifies
%%         * single-sided
%%         * page-numbering starts from title page
%%         * the lists of figures and tables have each entry prefixed
%%           with 'Figure' or 'Table'
%%       This can be tested by `\ifgpscopy ... \else ... \fi'
%%   10pt, 11pt, 12pt
%%       set default font size
%%   oneside, twoside
%%       whether to format for single-sided or double-sided printing
%%   balanced
%%       when double-sided, ensure page content is centred
%%       rather than slightly offset (the default)
%%   singlespacing, onehalfspacing, doublespacing
%%       set default inter-line text spacing; the ubcdiss class
%%       provides \textspacing to revert to this configured spacing
%%   draft
%%       disable more intensive processing, such as including
%%       graphics, etc.
%%

% For submission to GPS
\documentclass[gpscopy,onehalfspacing,11pt]{ubcdiss}

% For your own copies (looks nicer)
% \documentclass[balanced,twoside,11pt]{ubcdiss}

%--------------------------------------------------------------------------------------------------------------------------------------------%%--------------------------------------------------------------------------------------------------------------------------------------------%
%%
%% FONTS:
%% 
%% The defaults below configures Times Roman for the serif font,
%% Helvetica for the sans serif font, and Courier for the
%% typewriter-style font.  Configuring fonts can be time
%% consuming; we recommend skipping to END FONTS!
%% 
%% If you're feeling brave, have lots of time, and wish to use one
%% your platform's native fonts, see the commented out bits below for
%% XeTeX/XeLaTeX.  This is not for the faint at heart. 
%% (And shouldn't you be writing? :-)
%%

%% NFSS font specification (New Font Selection Scheme)
\usepackage{times,courier}
\usepackage[scaled=.92]{helvet}

%% Math or theory people may want to include the handy AMS macros
%\usepackage{amssymb}
%\usepackage{amsmath}
%\usepackage{amsfonts}

%% The pifont package provides access to the elements in the dingbat font.   
%% Use \ding{##} for a particular dingbat (see p7 of psnfss2e.pdf)
%%   Useful:
%%     51,52 different forms of a checkmark
%%     54,55,56 different forms of a cross (saltyre)
%%     172-181 are 1-10 in open circle (serif)
%%     182-191 are 1-10 black circle (serif)
%%     192-201 are 1-10 in open circle (sans serif)
%%     202-211 are 1-10 in black circle (sans serif)
%% \begin{dinglist}{##}\item... or dingautolist (which auto-increments)
%% to create a bullet list with the provided character.
\usepackage{pifont}

%--------------------------------------------------------------------------------------------------------------------------------------------%
%% Configure fonts for XeTeX / XeLaTeX using the fontspec package.
%% Be sure to check out the fontspec documentation.
%\usepackage{fontspec,xltxtra,xunicode}	% required
%\defaultfontfeatures{Mapping=tex-text}	% recommended
%% Minion Pro and Myriad Pro are shipped with some versions of
%% Adobe Reader.  Adobe representatives have commented that these
%% fonts can be used outside of Adobe Reader.
%\setromanfont[Numbers=OldStyle]{Minion Pro}
%\setsansfont[Numbers=OldStyle,Scale=MatchLowercase]{Myriad Pro}
%\setmonofont[Scale=MatchLowercase]{Andale Mono}

%% Other alternatives:
%\setromanfont[Mapping=tex-text]{Adobe Caslon}
%\setsansfont[Scale=MatchLowercase]{Gill Sans}
%\setsansfont[Scale=MatchLowercase,Mapping=tex-text]{Futura}
%\setmonofont[Scale=MatchLowercase]{Andale Mono}
%\newfontfamily{\SYM}[Scale=0.9]{Zapf Dingbats}
%% END FONTS
%--------------------------------------------------------------------------------------------------------------------------------------------%%--------------------------------------------------------------------------------------------------------------------------------------------%



%--------------------------------------------------------------------------------------------------------------------------------------------%%--------------------------------------------------------------------------------------------------------------------------------------------%
%%
%% Recommended packages
%%
\usepackage{checkend}	% better error messages on left-open environments
\usepackage{graphicx}	% for incorporating external images

%% booktabs: provides some special commands for typesetting tables as used
%% in excellent journals.  Ignore the examples in the Lamport book!
\usepackage{booktabs}

%% listings: useful support for including source code listings, with
%% optional special keyword formatting.  The \lstset{} causes
%% the text to be typeset in a smaller sans serif font, with
%% proportional spacing.
\usepackage{listings}
\lstset{basicstyle=\sffamily\scriptsize,showstringspaces=false,fontadjust}

%% The acronym package provides support for defining acronyms, providing
%% their expansion when first used, and building glossaries.  See the
%% example in glossary.tex and the example usage throughout the example
%% document.
%% NOTE: to use \MakeTextLowercase in the \acsfont command below,
%%   we *must* use the `nohyperlinks' option -- it causes errors with
%%   hyperref otherwise.  See Section 5.2 in the ``LaTeX 2e for Class
%%   and Package Writers Guide'' (clsguide.pdf) for details.
\usepackage[printonlyused,nohyperlinks]{acronym}
%% The ubcdiss.cls loads the `textcase' package which provides commands
%% for upper-casing and lower-casing text.  The following causes
%% the acronym package to typeset acronyms in small-caps
%% as recommended by Bringhurst.
\renewcommand{\acsfont}[1]{{\scshape \MakeTextLowercase{#1}}}

%% color: add support for expressing colour models.  Grey can be used
%% to great effect to emphasize other parts of a graphic or text.
%% For an excellent set of examples, see Tufte's "Visual Display of
%% Quantitative Information" or "Envisioning Information".
\usepackage{color}
\definecolor{greytext}{gray}{0.5}

%% comment: provides a new {comment} environment: all text inside the
%% environment is ignored.
%%   \begin{comment} ignored text ... \end{comment}
\usepackage{comment}

%% The natbib package provides more sophisticated citing commands
%% such as \citeauthor{} to provide the author names of a work,
%% \citet{} to produce an author-and-reference citation,
%% \citep{} to produce a parenthetical citation.
%% We use \citeeg{} to provide examples
\usepackage[numbers,sort&compress]{natbib}
\newcommand{\citeeg}[1]{\citep[e.g.,][]{#1}}

%% The titlesec package provides commands to vary how chapter and
%% section titles are typeset.  The following uses more compact
%% spacings above and below the title.  The titleformat that follow
%% ensure chapter/section titles are set in singlespace.
\usepackage[compact]{titlesec}
\titleformat*{\section}{\singlespacing\raggedright\bfseries\Large}
\titleformat*{\subsection}{\singlespacing\raggedright\bfseries\large}
\titleformat*{\subsubsection}{\singlespacing\raggedright\bfseries}
\titleformat*{\paragraph}{\singlespacing\raggedright\itshape}

%% The caption package provides support for varying how table and
%% figure captions are typeset.
\usepackage[format=hang,indention=-1cm,labelfont={bf},margin=1em]{caption}

%% url: for typesetting URLs and smart(er) hyphenation.
%% \url{http://...} 
\usepackage{url}
\urlstyle{sf}	% typeset urls in sans-serif


%--------------------------------------------------------------------------------------------------------------------------------------------%%--------------------------------------------------------------------------------------------------------------------------------------------%
%%
%% Possibly useful packages: you may need to explicitly install
%% these from CTAN if they aren't part of your distribution;
%% teTeX seems to ship with a smaller base than MikTeX and MacTeX.
%%
%\usepackage{pdfpages}	% insert pages from other PDF files
%\usepackage{longtable}	% provide tables spanning multiple pages
%\usepackage{chngpage}	% support changing the page widths on demand
%\usepackage{tabularx}	% an enhanced tabular environment

%% enumitem: support pausing and resuming enumerate environments.
%\usepackage{enumitem}

%% rotating: provides two environments, sidewaystable and sidewaysfigure,
%% for typesetting tables and figures in landscape mode.  
%\usepackage{rotating}

%% subfig: provides for including subfigures within a figure,
%% and includes being able to separately reference the subfigures.
%\usepackage{subfig}

%% ragged2e: provides several new new commands \Centering, \RaggedLeft,
%% \RaggedRight and \justifying and new environments Center, FlushLeft,
%% FlushRight and justify, which set ragged text and are easily
%% configurable to allow hyphenation.
%\usepackage{ragged2e}

%% The ulem package provides a \sout{} for striking out text and
%% \xout for crossing out text.  The normalem and normalbf are
%% necessary as the package messes with the emphasis and bold fonts
%% otherwise.
%\usepackage[normalem,normalbf]{ulem}    % for \sout

%--------------------------------------------------------------------------------------------------------------------------------------------%
%% HYPERREF:
%% The hyperref package provides for embedding hyperlinks into your
%% document.  By default the table of contents, references, citations,
%% and footnotes are hyperlinked.
%%
%% Hyperref provides a very handy command for doing cross-references:
%% \autoref{}.  This is similar to \ref{} and \pageref{} except that
%% it automagically puts in the *type* of reference.  For example,
%% referencing a figure's label will put the text `Figure 3.4'.
%% And the text will be hyperlinked to the appropriate place in the
%% document.
%%
%% Generally hyperref should appear after most other packages

%% The following puts hyperlinks in very faint grey boxes.
%% The `pagebackref' causes the references in the bibliography to have
%% back-references to the citing page; `backref' puts the citing section
%% number.  See further below for other examples of using hyperref.
%% 2009/12/09: now use `linktocpage' (Jacek Kisynski): GPS now prefers
%%   that the ToC, LoF, LoT place the hyperlink on the page number,
%%   rather than the entry text.
\usepackage[bookmarks,bookmarksnumbered,%
    allbordercolors={0.8 0.8 0.8},%
    pagebackref,linktocpage%
    ]{hyperref}
%% The following change how the the back-references text is typeset in a
%% bibliography when `backref' or `pagebackref' are used
%%
%% Change \nocitations if you'd like some text shown where there
%% are no citations found (e.g., pulled in with \nocite{xxx})
\newcommand{\nocitations}{\relax}
%%\newcommand{\nocitations}{No citations}
%%
%\renewcommand*{\backref}[1]{}% necessary for backref < 1.33
\renewcommand*{\backrefsep}{,~}%
\renewcommand*{\backreftwosep}{,~}% ', and~'
\renewcommand*{\backreflastsep}{,~}% ' and~'
\renewcommand*{\backrefalt}[4]{%
\textcolor{greytext}{\ifcase #1%
\nocitations%
\or
\(\rightarrow\) page #2%
\else
\(\rightarrow\) pages #2%
\fi}}


%% The following uses most defaults, which causes hyperlinks to be
%% surrounded by colourful boxes; the colours are only visible in
%% PDFs and don't show up when printed:
%\usepackage[bookmarks,bookmarksnumbered]{hyperref}

%% The following disables the colourful boxes around hyperlinks.
%\usepackage[bookmarks,bookmarksnumbered,pdfborder={0 0 0}]{hyperref}

%% The following disables all hyperlinking, but still enabled use of
%% \autoref{}
%\usepackage[draft]{hyperref}

%% The following commands causes chapter and section references to
%% uppercase the part name.
\renewcommand{\chapterautorefname}{Chapter}
\renewcommand{\sectionautorefname}{Section}
\renewcommand{\subsectionautorefname}{Section}
\renewcommand{\subsubsectionautorefname}{Section}

%% If you have long page numbers (e.g., roman numbers in the 
%% preliminary pages for page 28 = xxviii), you might need to
%% uncomment the following and tweak the \@pnumwidth length
%% (default: 1.55em).  See the tocloft documentation at
%% http://www.ctan.org/tex-archive/macros/latex/contrib/tocloft/
% \makeatletter
% \renewcommand{\@pnumwidth}{3em}
% \makeatother

%--------------------------------------------------------------------------------------------------------------------------------------------%%--------------------------------------------------------------------------------------------------------------------------------------------%
%%
%% Some special settings that controls how text is typeset
%%
% \raggedbottom		% pages don't have to line up nicely on the last line
% \sloppy		% be a bit more relaxed in inter-word spacing
% \clubpenalty=10000	% try harder to avoid orphans
% \widowpenalty=10000	% try harder to avoid widows
% \tolerance=1000

%% And include some of our own useful macros
% This file provides examples of some useful macros for typesetting
% dissertations.  None of the macros defined here are necessary beyond
% for the template documentation, so feel free to change, remove, and add
% your own definitions.

% We recommend that you define macros to separate the semantics
% of the things you write from how they are presented.  For example,
% you'll see definitions below for a macro \file{}: by using
% \file{} consistently in the text, we can change how filenames
% are typeset simply by changing the definition of \file{} in
% this file.

%% The following is a directive for TeXShop to indicate the main file
%%!TEX root = diss.tex

% relevant packages
\usepackage[parfill]{parskip} 	% Activate to begin paragraphs with an empty line rather than an indent
\usepackage{graphicx}
\usepackage{amssymb}
\usepackage{amsmath}
\usepackage{mathrsfs }
\usepackage{amsthm}
\usepackage{epstopdf}
\usepackage{enumerate}
\usepackage{tikz}
\usetikzlibrary{matrix}
\usepackage{listings}
\usepackage{color}
\usepackage[all]{xy}
\usepackage[english]{babel}
\usepackage{setspace}

% relevant environments
\newtheorem{theorem}{Theorem}[section]
\newtheorem{conjecture}[theorem]{Conjecture}
\newtheorem{corollary}[theorem]{Corollary}
\newtheorem{lemma}[theorem]{Lemma}
\newtheorem{properties}[theorem]{Properties}
\newtheorem{proposition}[theorem]{Proposition}
\newtheorem{problem}[theorem]{Problem}
\newtheorem{question}[theorem]{Question}

\theoremstyle{definition}
\newtheorem{Algorithm}[theorem]{Algorithm}
\newtheorem{definition}[theorem]{Definition}
\newtheorem{example}[theorem]{Example}
\newtheorem{remark}[theorem]{Remark}

% math operators
\DeclareMathOperator{\ord}{ord}
\DeclareMathOperator{\sgn}{sgn}
\DeclareMathOperator{\Cl}{Cl}
\DeclareMathOperator{\Gal}{Gal}
\newcommand{\eps}{\varepsilon}

\newcommand{\NA}{\textsc{n/a}}	% for "not applicable"
\newcommand{\eg}{e.g.,\ }	% proper form of examples (\eg a, b, c)
\newcommand{\ie}{i.e.,\ }	% proper form for that is (\ie a, b, c)
\newcommand{\etal}{\emph{et al}}

% macros for typesetting terms.
\newcommand{\file}[1]{\texttt{#1}}
\newcommand{\class}[1]{\texttt{#1}}
\newcommand{\latexpackage}[1]{\href{http://www.ctan.org/macros/latex/contrib/#1}{\texttt{#1}}}
\newcommand{\latexmiscpackage}[1]{\href{http://www.ctan.org/macros/latex/contrib/misc/#1.sty}{\texttt{#1}}}
\newcommand{\env}[1]{\texttt{#1}}
\newcommand{\BibTeX}{Bib\TeX}

% draft edit commands
\usepackage{soul}
\newcommand{\aaron}[1]{\textcolor{purple}{\footnotesize #1}}
\newcommand{\strike}[1]{\textcolor{red}{\st{#1}}}
% The following definition will also output a warning to the console
\newcommand{\edit}[1]{\typeout{**EDIT** #1}{\textcolor{blue}{#1}}}


% Define a command \doi{} to typeset a digital object identifier (DOI).
% Note: if the following definition raise an error, then you likely
% have an ancient version of url.sty.  Either find a more recent version
% (3.1 or later work fine) and simply copy it into this directory,  or
% comment out the following two lines and uncomment the third.
\DeclareUrlCommand\DOI{}
\newcommand{\doi}[1]{\href{http://dx.doi.org/#1}{\DOI{doi:#1}}}
%\newcommand{\doi}[1]{\href{http://dx.doi.org/#1}{doi:#1}}

% Useful macro to reference an online document with a hyperlink
% as well with the URL explicitly listed in a footnote
% #1: the URL
% #2: the anchoring text
\newcommand{\webref}[2]{\href{#1}{#2}\footnote{\url{#1}}}

% END


%--------------------------------------------------------------------------------------------------------------------------------------------%%--------------------------------------------------------------------------------------------------------------------------------------------%
%%
%% Document meta-data: be sure to also change the \hypersetup information
%%

\title{Computing elliptic curves over $\mathbb{Q}$ via Thue-Mahler equations and related problems}
%\subtitle{If you want a subtitle}

\author{Adela Gherga}
\previousdegree{B.Sc. Mathematics, McMaster University, 2006}
\previousdegree{M.Sc. Mathematics, McMaster University, 2010}

% What is this dissertation for?
\degreetitle{Doctor of Philosophy}

\institution{The University of British Columbia}
\campus{Vancouver}

\faculty{The Faculty of Graduate and Postdoctoral Studies}
\department{Mathematics}
\submissionmonth{July}
\submissionyear{2019}

% details of your examining committee
%\examiningcommittee{John Smith, Materials Engineering}{Supervisor}
%\examiningcommittee{Mary Maker, Materials Engineering}%
%    {Supervisory Committee Member}
%\examiningcommittee{Nebulous Name, Department}{Supervisory Committee Member}
%\examiningcommittee{Magnus Monolith, Other Department}{Additional Examiner}
%
%% details of your supervisory committee
%\supervisorycommittee{Ira Crater, Materials Engineering}%
%    {Supervisory Committee Member}
%\supervisorycommittee{Adeline Long, \textsc{CEO} of Aerial Machine
%    Transportation, Inc.}{Supervisory Committee Member}

%% hyperref package provides support for embedding meta-data in .PDF
%% files
\hypersetup{
  pdftitle={Change this title!  (DRAFT: \today)},
  pdfauthor={Adela Gherga},
  pdfkeywords={Your keywords here}
}

%--------------------------------------------------------------------------------------------------------------------------------------------%%--------------------------------------------------------------------------------------------------------------------------------------------%
%% 
%% The document content
%%

%% LaTeX's \includeonly commands causes any uses of \include{} to only
%% include files that are in the list.  This is helpful to produce
%% subsets of your thesis (e.g., for committee members who want to see
%% the dissertation chapter by chapter).  It also saves time by 
%% avoiding reprocessing the entire file.
%\includeonly{intro,conclusions}
%\includeonly{discussion}

\begin{document}

%--------------------------------------------------------------------------------------------------------------------------------------------%
%% From Thesis Components: Tradtional Thesis
%% <http://www.grad.ubc.ca/current-students/dissertation-thesis-preparation/order-components>

% Preliminary Pages (numbered in lower case Roman numerals)
%    1. Title page (mandatory)
\maketitle

%    2. Committee page (mandatory): lists supervisory committee and,
%    if applicable, the examining committee
%\makecommitteepage

%    3. Abstract (mandatory - maximum 350 words)
%% The following is a directive for TeXShop to indicate the main file
%%!TEX root = diss.tex

\chapter{Abstract}

We present a practical and efficient algorithm for solving an arbitrary Thue-Mahler equation. This algorithm uses explicit height bounds with refined sieves, combining Diophantine approximation techniques of Tzanakis-de Weger with new geometric ideas. We begin by using methods of algebraic number theory to reduce the problem of solving the Thue-Mahler equation to the problem of solving a finite collection of related Diophantine equations. In the first part of this thesis, we establish the key results which allow us to drastically reduce the number of such Diophantine equations and subsequently reduce the running time. 

In the second part of this thesis, we show that if $n \geq 3$ is a fixed integer, then there exists an effectively computable constant $c (n)$ such that if $x, y$ and $m$ are integers satisfying
$$
\frac{x^m-1}{x-1} = \frac{y^n-1}{y-1}, \; \; y>x>1, \; m > n,
$$
with $\gcd(m-1,n-1)>1$,
then $\max \{ x, y, m \} < c (n)$. In case $n \in \{ 3, 4, 5 \}$, we solve the equation completely, subject to this non-coprimality condition.  In case $n=5$, our resulting computations require a variety of innovations for solving Ramanujan-Nagell equations of the shape $f(x)=y^n$, where $f(x)$ is a given polynomial with integer coefficients (and degree at least two), and $y$ is a fixed integer. In particular, we modify our Thue-Mahler algorithm and specialize our refinements to the case of Ramanujan-Nagell equations, enabling us to fully resolve the $n = 5$ case. 

In the third part, we discuss an algorithm for finding all elliptic curves over $\mathbb{Q}$ with a given conductor. Though based on 
classical ideas derived from reducing the problem to one of solving associated Thue-Mahler equations,  our approach, in many cases at least, appears to be reasonably efficient  computationally. We provide 
details of the output derived from running the algorithm, concentrating on the cases of conductor $p$ or $p^2$, for $p$ prime, with comparisons to existing 
data. 

Finally, we specialize the Thue-Mahler algorithm to degree $3$, applying an analogue of Matshke-von K\:anel's elliptic logarithm sieve to construct a global sieve, leading to reduced search spaces. The algorithm is implemented in the Magma computer algebra system, and is part of an on-going collaborative project. 



%--------------------------------------------------------------------------------------------------------------------------------------------%

\endinput

This document provides brief instructions for using the \class{ubcdiss}
class to write a \acs{UBC}-conformant dissertation in \LaTeX.  This
document is itself written using the \class{ubcdiss} class and is
intended to serve as an example of writing a dissertation in \LaTeX.
This document has embedded \acp{URL} and is intended to be viewed
using a computer-based \ac{PDF} reader.

Note: Abstracts should generally try to avoid using acronyms.

Note: at \ac{UBC}, both the \ac{GPS} Ph.D. defence programme and the
Library's online submission system restricts abstracts to 350
words.

% Consider placing version information if you circulate multiple drafts
%\vfill
%\begin{center}
%\begin{sf}
%\fbox{Revision: \today}
%\end{sf}
%\end{center}

Any text after an \endinput is ignored.
You could put scraps here or things in progress.
\cleardoublepage

%    4. Lay Summary (Effective May 2017, mandatory - maximum 150 words)
%% The following is a directive for TeXShop to indicate the main file
%%!TEX root = diss.tex

%% https://www.grad.ubc.ca/current-students/dissertation-thesis-preparation/preliminary-pages
%% 
%% LAY SUMMARY Effective May 2017, all theses and dissertations must
%% include a lay summary.  The lay or public summary explains the key
%% goals and contributions of the research/scholarly work in terms that
%% can be understood by the general public. It must not exceed 150
%% words in length.

\chapter{Lay Summary}

Consider any collection of prime numbers $\{p_1, \dots, p_v\}$ and any collection of integers $c, c_0, \dots, c_n$. Our main result involves the \textit{Thue-Mahler} equation
\[F(x,y) = c_0x^n + c_1x^{n-1}y + \cdots + c_{n-1}xy^{n-1} + c_ny^n = cp_1^{z_1}\cdots p_v^{z_v},\]
where the values $x, y$, and $z_1, \dots, z_v$ are unknown. In particular, for any such equation, we know that there are only finitely many values of $x,y$, and $z_1, \dots, z_n$ which satisfy it. In our work, we develop an algorithm to find all of these solutions for any given collection of primes and coefficients $c_i$. The solutions to these Thue-Mahler have many important mathematical applications, and we modify and refine our algorithm for use in those applications.


\endinput

The lay or public summary explains the key goals and contributions of
the research\slash{}scholarly work in terms that can be understood by the
general public. It must not exceed 150 words in length.

\cleardoublepage

%    5. Preface
%% The following is a directive for TeXShop to indicate the main file
%%!TEX root = diss.tex

\chapter{Preface}

The work presented in \autoref{ch:Goormaghtigh} is joint work with Dr. M. Bennett and Dr. D. Kreso and has been submitted for publication \cite{BeGhKr}. I was responsible for modifying and specializing the Thue-Mahler algorithm to resolve the remaining cases, $0 \leq x \leq 720$, for $n = 5$. I implemented the resulting algorithm in Magma and performed the tests on each remaining case, as well as wrote \autoref{TM}. The remainder of the work submitted for publication was originally drafted by M. Bennett and D. Kreso. 

\autoref{ch:EllipticCurves} is work completed in collaboration with Dr. M. Bennett and Dr. A. Rechnitzer. A version of this chapter has been published and appears in M. A. Bennett, A. Gherga and A. Rechnitzer, \emph{Computing elliptic curves over $\mathbb{Q}$}, Math. Comp. 88 (2019), no. 317, 1341-1390. In this work, I modified and implemented all of the code needed to resolve the reducible and irreducible forms. Furthermore, I was responsible for running this code to generate all of the solutions and resulting elliptic curves to the forms in the section ``Examples''. I drafted the majority of this section, while the remainder of the paper was originally drafted by M. Bennett and A. Rechnitzer.

\autoref{ch:AlgorithmsForTM} and \autoref{ch:EfficientTMSolver} is part of an ongoing collaborative project, currently in preparation \cite{GhKaMaSi} with Dr. B. Matshke, Dr. R. von K\"anel, and Dr. S. Siksek. The ideas presented in \autoref{sec:PIRL} and \autoref{subsec:FactorizationTMwithoutOK} are attributed to S. Siksek. The work in \autoref{ch:EfficientTMSolver} is joint work with R. von K\"anel, to whom the new ideas are attributed. Here, I helped to develop the theory and details behind these ideas, as well as implemented and tested the algorithm presented in both chapters.



\cleardoublepage

%    6. Table of contents (mandatory - list all items in the preliminary pages
%    starting with the abstract, followed by chapter headings and
%    subheadings, bibliographies and appendices)
\tableofcontents
\cleardoublepage	% required by tocloft package

%    7. List of tables (mandatory if thesis has tables)
\listoftables
\cleardoublepage	% required by tocloft package

%    8. List of figures (mandatory if thesis has figures)
\listoffigures
\cleardoublepage	% required by tocloft package

%    9. List of illustrations (mandatory if thesis has illustrations)
%   10. Lists of symbols, abbreviations or other (optional)

%   11. Glossary (optional)
%% The following is a directive for TeXShop to indicate the main file
%%!TEX root = diss.tex

\chapter{Glossary}

This glossary uses the handy \latexpackage{acroynym} package to automatically
maintain the glossary.  It uses the package's \texttt{printonlyused}
option to include only those acronyms explicitly referenced in the
\LaTeX\ source.

% use \acrodef to define an acronym, but no listing
\acrodef{UI}{user interface}
\acrodef{UBC}{University of British Columbia}

% The acronym environment will typeset only those acronyms that were
% *actually used* in the course of the document
\begin{acronym}[ANOVA]
\acro{ANOVA}[ANOVA]{Analysis of Variance\acroextra{, a set of
  statistical techniques to identify sources of variability between groups}}
\acro{API}{application programming interface}
\acro{CTAN}{\acroextra{The }Common \TeX\ Archive Network}
\acro{DOI}{Document Object Identifier\acroextra{ (see
    \url{http://doi.org})}}
\acro{GPS}[GPS]{Graduate and Postdoctoral Studies}
\acro{PDF}{Portable Document Format}
\acro{RCS}[RCS]{Revision control system\acroextra{, a software
    tool for tracking changes to a set of files}}
\acro{TLX}[TLX]{Task Load Index\acroextra{, an instrument for gauging
  the subjective mental workload experienced by a human in performing
  a task}}
\acro{UML}{Unified Modelling Language\acroextra{, a visual language
    for modelling the structure of software artefacts}}
\acro{URL}{Unique Resource Locator\acroextra{, used to describe a
    means for obtaining some resource on the world wide web}}
\acro{W3C}[W3C]{\acroextra{the }World Wide Web Consortium\acroextra{,
    the standards body for web technologies}}
\acro{XML}{Extensible Markup Language}
\end{acronym}

% You can also use \newacro{}{} to only define acronyms
% but without explictly creating a glossary
% 
% \newacro{ANOVA}[ANOVA]{Analysis of Variance\acroextra{, a set of
%   statistical techniques to identify sources of variability between groups.}}
% \newacro{API}[API]{application programming interface}
% \newacro{GOMS}[GOMS]{Goals, Operators, Methods, and Selection\acroextra{,
%   a framework for usability analysis.}}
% \newacro{TLX}[TLX]{Task Load Index\acroextra{, an instrument for gauging
%   the subjective mental workload experienced by a human in performing
%   a task.}}
% \newacro{UI}[UI]{user interface}
% \newacro{UML}[UML]{Unified Modelling Language}
% \newacro{W3C}[W3C]{World Wide Web Consortium}
% \newacro{XML}[XML]{Extensible Markup Language}
	% always input, since other macros may rely on it

\textspacing		% begin one-half or double spacing

%   12. Acknowledgements (optional)
%% The following is a directive for TeXShop to indicate the main file
%%!TEX root = diss.tex

\chapter{Acknowledgments}

I am indebted to Dr. Michael A. Bennett for suggesting to me the line of research on which this thesis is based and for numerous comments and suggestions that helped me to improve this thesis. 

This research was funded in part by a National Sciences and Engineering Research Council Postgraduate Scholarship.

\endinput

Thank those people who helped you. 

Don't forget your parents or loved ones.

You may wish to acknowledge your funding sources.


%   13. Dedication (optional)

% Body of Thesis (not all sections may apply)
\mainmatter

\acresetall	% reset all acronyms used so far

%    1. Introduction
%% The following is a directive for TeXShop to indicate the main file
%%!TEX root = diss.tex

\chapter{Introduction}
\label{ch:Introduction}

\aaron{
i mean, the beginning is the part you're not comfortable writing, right? the longer it went on, the better it flowed. at that point you're quoting and weaving results you know well, referencing the little mental web you have woven. it seems cohesive, but also i don't understand it. 
the beginning bit seems thrown together � like Mike told you to include bits about DEs and so you begrudgingly injected something ?? 
like the very very beginning bit
anyway, i'll e-mail you back the tex file and the pdf. 
i know you're not asking for this advice but it's coming from ozgur and yaniv and they are very smart and i trust them lots:}

\aaron{
\begin{enumerate}
\item be very careful about whether you're using colloquial language, and how it might be interpreted. e.g. be careful not to insult people's work, and try to not to flip flop on how hand wavey you are being. I think I have a couple of notes in the file pertaining to each of these points
\item when citing work, either use the author names every time or don't. don't mix and match unless appropriate. why would you deny some the respect of appearing in your work, but not others? 
\item if you're going to write notes to yourself in your thesis/papers, you must have a way of ensuring that you'll see them later before you send it off. caps lock is not sufficient and yaniv and ozgur can provide examples if you need. I included a little \edit{} command for you so that you can just Cmd+F (or C-s ?? ) for all appearances of \edit in the .tex file if you use it. Has the added advantage of making PDF text blue so that everyone reading too knows that it doesn't belong. 
\end{enumerate}}

\aaron{
also my disclaimer for edits: 
1. for some reason my brain is tired today;
2. I don't know the culture of your field nor some of the very elementary things you're presenting
3. Because of 1, I tried to communicate what I wanted to say using the best language I could, but may not have always succeeded at clarity/intent/approachability ?? 
So basically, remember that it's possible that my edits deserve to be treated with a grain of salt. ??}




\aaron{This start feels outside the realm of where you're going --- it seems at once abrupt and off-topic. It would be nice to have an introductory sentence or two to get the reader on track before discussing the ``required background'' material.}
A Diophantine equation is a polynomial equation in several variables defined over the integers. The term \textit{Diophantine} refers to the Greek mathematician Diophantus of Alexandria, who studied such equations in the 3rd century A.D.  \aaron{why the history lesson? maybe you could use this as one way of motivating/introducing DEs: ``look at these things. look how long they've been studied. here's why, and here are the ways people study them. \ldots or something\ldots}

\aaron{remove separate paragraph if it's the same thought --- $f$ is a DE right? If so, then these next lines are providing additional information to what was given above, not starting a new thread.}
Let $f(x_1, \dots, x_n)$ be a polynomial with integer coefficients. We wish to study the set of solutions $(x_1, \dots, x_n) \in \mathbb{Z}^n$ to the equation
\begin{equation}\label{Introduction:Diophantine}
f(x_1, \dots, x_n) = 0.
\end{equation}
There are several different approaches for doing so, arising from three basic
problems concerning Diophantine equations. The first such problem is to
determine whether \strike{or not} \eqref{Introduction:Diophantine} has any
solutions \strike{at all} \aaron{(too colloquial imo)}. Indeed, one of the most
famous theorems in mathematics, Fermat's Last Theorem, proven by Wiles in 1995,
states that for $f(x,y,z) = x^n + y^n - z^n$, where $n \geq 3$, there are no
solutions in the positive integers $x,y,z$ \aaron{(there are so many commas in this sentence. You can remove at least 2 of 7 by splicing and/or rearranging)}. Qualitative questions of this type
are often studied using algebraic methods.

Suppose now that \eqref{Introduction:Diophantine} is solvable, that is, has at least one solution. The second basic problem is to determine whether the number of solutions is finite or infinite. For example, consider the \textit{Thue equation}, 
\begin{equation}\label{Introduction:Thue}
f(x,y) = a,
\end{equation}
where $f(x,y)$ is an integral binary form of degree $n \geq 3$ \aaron{(feels
  like you really jump into the language here. you spelled out what a DE was,
  but now assume the reader knows the definition of an integral binary
  form. Personally, I knew the former but the latter reads like domain-specific
  jargon to me)} and $a$ is a fixed nonzero rational integer. In 1909, Thue
[\textcolor{red}{REF}] proved that this equation has only finitely many
solutions. This result followed from a sharpening of Liouville's inequality, an
observation that algebraic numbers do not admit very strong approximation by
rational numbers. That is, if $\alpha$ is a real algebraic number of degree
$n \geq 2$ and $p,q$ are integers, Liouville's ([\textcolor{red}{REF}])
observation states that
\begin{equation}\label{Introduction:Liouville}
\left|\alpha - \frac{p}{q}\right| > \frac{c_1}{q^{n}},
\end{equation}
where $c_1 >0 $ is a value depending explicitly on $\alpha$. The finitude of the number of solutions to \eqref{Introduction:Thue} follows directly from a sharpening of \eqref{Introduction:Liouville} of the type
\begin{equation}\label{Introduction:Sharpening}
\left|\alpha - \frac{p}{q}\right| > \frac{\lambda(q)}{q^{n}}, \quad \lambda(q) \to \infty.
\end{equation}
\aaron{what is the limit $\lambda \to \infty$ with respect to?}  Indeed, if
$\alpha$ is a real root of $f(x,1)$ and $\alpha^{(i)}$, $i = 1, \dots, n$ are
its conjugates, it follows from \eqref{Introduction:Thue} that
\[\prod_{i=1}^n\left|\alpha^{(i)}-\frac{x}{y}\right| = \frac{a}{|a_0||y|^n}\]
where $a_0$ is the leading coefficient of the polynomial $f(x,1)$. If the Thue equation has integer solutions with arbitrarily large $|y|$, the product $\prod_{i=1}^n|\alpha^{(i)}-x/y|$ must take arbitrarily small values for solutions $x,y$ of \eqref{Introduction:Thue}. As all the $\alpha^{(i)}$ are different, $x/y$ must be correspondingly close to one of the real numbers $\alpha^{(i)}$, say $\alpha$. Thus we obtain
\[\left|\alpha - \frac{x}{y}\right| < \frac{c_2}{|y|^n}\]
where $c_2$ depends only on $a_0$, $n$, and the conjugates $\alpha^{(i)}$. Comparison of this inequality with 
\eqref{Introduction:Sharpening} shows that $|y|$ cannot be arbitrarily large, and so the number of solutions of the Thue equation is finite. Using this argument, an explicit bound can be constructed on the solutions of \eqref{Introduction:Thue} provided that an effective \aaron{(descriptive? explicit? tight? tractable?)} inequality \eqref{Introduction:Sharpening} is known. The sharpening of the Liouville inequality however, especially in effective form, proved to be very difficult. \aaron{REF? also ``very difficult'' seems a subjective qualification; is that okay for your audience?}

In [\textcolor{red}{REF:THUE}], Thue published a proof that 
\[\left|\alpha - \frac{p}{q}\right| < \frac{1}{q^{\frac{n}{2} + 1 + \varepsilon}}\]
has only finitely many solutions in integers $p,q > 0$ for all algebraic
numbers $\alpha$ of degree $n \geq 3$ and any $\varepsilon > 0$. In essence, he
obtained the inequality \eqref{Introduction:Sharpening} with
$\lambda(q) = c_3q^{\frac{1}{2}n - 1 - \varepsilon}$ \aaron{this function does
  not match the one appearing above in displaymath. is that supposed to be the
  case? might have something to do with the $<$ not matching the $>$ in
  \eqref{Introduction:Sharpening}? it is not clear to me, but hopefully it will
  be to typical reader}, where $c_3 > 0$ depends on $\alpha$ and $\varepsilon$,
thereby confirming that all Thue equations have only finitely many
solutions. Unfortunately, Thue's arguments do not allow one to find the
explicit dependence of $c_3$ on $\alpha$ and $\varepsilon$, and so the bound
for the number of solutions of the Thue equation cannot be given in explicit
form either. That is, Thue's proof is ineffective, meaning that it
  provides no means to \strike{actually} find the solutions to \eqref{Introduction:Thue}. \aaron{I feel like I would dance more carefully around calling someone's proof ineffective.}

Nonetheless, the investigation of Thue's equation and its generalizations was central to the development of the theory of Diophantine equations in the early 20th century when it was discovered that many Diophantine equations in two unknowns could be reduced to it. In particular, the thorough development and enrichment of Thue's method led Siegel to his theorem on the finitude of the number of integral points on an algebraic curve of genus greater than zero \aaron{[REF?]}. However, as Siegel's result relies on Thue's rational approximation to algebraic numbers, it too is ineffective \aaron{in the above sense}. 

Shortly following Thue's result, Goormaghtigh conjectured that the only non-trivial integer solutions of the exponential Diophantine equation
\begin{equation} \label{Introduction:Goormaghtigh}
\frac{x^m-1}{x-1} = \frac{y^n-1}{y-1}
\end{equation}
satisfying $x > y > 1$ and $n,m > 2$ are
\[31 = \frac{2^{5}-1}{2-1} = \frac{5^{3}-1}{5-1} \quad \text{ and } \quad 8191 = \frac{2^{13}-1}{2-1}=\frac{90^{3}-1}{90-1}.\]
These correspond to the known solutions $(x,y,m,n)=(2,5,5,3)$ and $(2,90,13,3)$ to what is nowadays termed {\it Goormaghtigh's equation}. The Diophantine equation \eqref{Introduction:Goormaghtigh} asks for integers having all digits equal to one with respect to two distinct bases, yet whether it has finitely many solutions is still unknown. By fixing the exponents $m$ and $n$ however, Davenport, Lewis, and Schinzel ([REF]) were able to prove that \eqref{Introduction:Goormaghtigh} has only finitely many solutions. Unfortunately, this result rests on Siegel's aforementioned finiteness theorem, and is therefore ineffective. 

In 1933, Mahler [\textcolor{red}{REF}] published a paper on the investigation of the Diophantine equation
\[f(x,y) = p_1^{z_1}\cdots p_v^{z_v}, \quad (x,y) = 1,\] in which
$S= \{p_1, \dots, p_v\}$ denote\aaron{s} a fixed set of prime numbers,
$x,y,z_i \geq 0$, $i = 1, \dots, v$ are unknown integers, and $f(x,y)$ is an
integral irreducible binary form of degree $n \geq 3$. Generalizing the
classical result of Thue, Mahler proved that this equation has only finitely
many solutions. Unfortunately, like Thue, Mahler's argument is also
ineffective \aaron{each time I read this, I believe more strongly that a different word should be used to describe their work. ineffective seems like an attack, and a broad stroke that misses the precise critique you're looking to discuss}.

This leads us to the third basic problem regarding Diophantine equations and the main focus of this thesis: given a solvable Diophantine equation, determine all of its solutions. Until long after Thue's work, no method was known for the construction of bounds for the number of solutions of a Thue equation in terms of the parameters of the equation. Only in 1968 was such a method introduced by Baker [REF], based on his theory of bounds for linear forms in the logarithms of algebraic numbers. Generalizing Baker's ground-breaking result to the $p$-adic case, Sprind\u zuk and Vinogradov [CITE] and Coates [CITE] proved that the solutions of any \textit{Thue-Mahler equation},
\begin{equation}\label{Introduction:ThueMahler}
f(x,y) = ap_1^{z_1}\cdots p_v^{z_v}, \quad (x,y) = 1,
\end{equation}
where $a$ is a fixed integer, could, at least in principal, be effectively determined.  The first practical method for solving the general Thue-Mahler equation \eqref{Introduction:ThueMahler} over $\mathbb{Z}$ is attributed to Tzanakis and de Weger [CITE], whose ideas were inspired in part by the method of Agrawal, Coates, Hunt, and van der Poorten [CITE] in their work to solve the specific Thue-Mahler equation
\[x^3 - x^2y + xy^2 + y^3 = \pm 11^{z_1}.\]
Using optimized bounds arising from the theory of linear forms in logarithms, a refined, automated version of this explicit method has since been implemented by Hambrook as a MAGMA package \aaron{[REF?]}. 

As for Goormaghtigh's equation, when $m$ and $n$ are fixed and 
\begin{equation}\label{Introduction:GoormaghtighCondition}
\gcd(m-1,n-1) > 1,
\end{equation}
Davenport, Lewis, and Schinzel ([REF]) were able to replace Siegel's result by an effective argument due to Runge. This result was improved by Nesterenko and Shorey ([REF]) and Bugeaud and Shorey ([REF]) using Baker's theory of linear forms in logarithms. In either case, in order to deduce effectively computable bounds \aaron{(I like this use of effectively)} upon the polynomial variables $x$ and $y$, one must impose the constraints upon $m$ and $n$ that either $m=n+1$, or that the assumption \eqref{Introduction:GoormaghtighCondition} holds. In the extensive literature on this problem, there are a number of striking results that go well beyond what we have mentioned here. By way of example, work of Balasubramanian and Shorey ([REF]) shows that equation \eqref{Introduction:Goormaghtigh} has at most finitely many solutions if we fix only the set of prime divisors of $x$ and $y$, while Bugeaud and Shorey ([REF]) prove an analogous finiteness result, under the additional assumption of \eqref{Introduction:GoormaghtighCondition}, provided the quotient $(m-1)/(n-1)$ is bounded above. Additional results on special cases of equation \eqref{Introduction:Goormaghtigh} are available in, for example, \cite{HeTo}, \cite{Le1}, \cite{Le2} and \cite{Le3}.  An excellent overview of results on this problem  can be found in the survey of Shorey \cite{ShoSur}.

%---------------------------------------------------------------------------------------------------------------------------------------------%

\section{Statement of the results}

The novel contributions of this thesis concern the development and implementation of efficient algorithms to determine all solutions of certain Goormaghtigh equations and Thue-Mahler equations. In particular, we follow [REF: BeGhKr] to prove that, in fact, under assumption \eqref{Introduction:GoormaghtighCondition}, equation \eqref{Introduction:Goormaghtigh} has at most finitely many solutions which may be found effectively, even if we fix only a single exponent. \\

\begin{theorem}[BeGhKr]\label{IntroductionTheorem:Goormaghtigh1}
If there is a solution in integers $x,y, n$ and $m$ to equation \eqref{Introduction:Goormaghtigh}, satisfying \eqref{Introduction:GoormaghtighCondition}, then
\begin{equation} \label{IntroductionTheorem:Goormaghtigh1Eq}
x <  (3d)^{4n/d} \leq  36^n.
\end{equation}
In particular, if $n$ is fixed, there is an effectively computable constant $c=c(n)$ such that
$\max \{ x, y, m \} < c$.
\end{theorem}
We note that the latter conclusion here follows immediately from \eqref{IntroductionTheorem:Goormaghtigh1Eq}, in conjunction with, for example, work of Baker ([REF]). The constants present in our upper bound \eqref{IntroductionTheorem:Goormaghtigh1Eq} may be sharpened somewhat at the cost of increasing the complexity of our argument. By refining our approach, in conjunction with some new results from computational Diophantine approximation, we are able to achieve the complete solution of equation \eqref{Introduction:Goormaghtigh}, subject to condition \eqref{Introduction:GoormaghtighCondition}, for small fixed values of $n$. \\

\begin{theorem}[BeGhKr] \label{IntroductionTheorem:Goormaghtigh2Eq}
If there is a solution in integers $x,y$ and $m$ to equation \eqref{Introduction:Goormaghtigh}, with $n \in \{ 3, 4, 5 \}$ and satisfying \eqref{Introduction:GoormaghtighCondition}, then
\[(x,y,m,n) = (2,5,5,3)  \; \mbox{ and } \; (2,90,13,3).\]
\end{theorem}

In the case $n = 5$ of Theorem \eqref{IntroductionTheorem:Goormaghtigh2Eq} ``off-the-shelf'' techniques for finding integral points on models of elliptic curves or for solving {\it Ramanujan-Nagell} equations of the shape $F(x)=z^n$ (where $F$ is a polynomial and $z$ a fixed integer) do not apparently permit the full resolution of this problem in a reasonable amount of time. Instead, we sharpen the existing techniques of [TdW] and [Hambrook] for solving Thue-Mahler equations and specialize them to this problem. 

A direct consequence and primary motivation for developing an efficient Thue-Mahler algorithm is the computation of elliptic curves over $\mathbb{Q}$. Let $S$ be a finite set of rational primes. In $1963$, Shafarevich [CITE] proved that there are at most finitely many $\mathbb{Q}$-isomorphism classes of elliptic curves defined over $\mathbb{Q}$ having good reduction outside $S$. The first effective proof of this statement was provided by Coates [CITE] in 1970 for the case $K = \mathbb{Q}$ and $S = \{2,3\}$ using bounds for linear forms in $p$-adic and complex logarithms. Early attempts to make these results explicit for fixed sets of small primes overlap with the arguments of [COATES], in that they reduce the problem to that of solving a number of degree $3$ Thue-Mahler equations of the form
\[F(x,y) = au,\]
where $u$ is an integer whose prime factors all lie in $S$. 

In the $1950$'s and $1960$'s, Taniyama and Weil asked whether all elliptic curves over $\mathbb{Q}$ of a given conductor $N$ are related to modular functions. While this conjecture is now known as the Modularity Theorem, until its proof in $2001$ \cite{Breuil}, attempts to verify it sparked a large effort to tabulate all elliptic curves over $\mathbb{Q}$ of given conductor $N$. In 1966, Ogg (\cite{Ogg1}, \cite{Ogg2}) determined all elliptic curves defined over $\mathbb{Q}$ with conductor of the form $2^a$. Coghlan, in his dissertation \cite{Coghlan}, studied the curves of conductor $2^a3^b$ independently of Ogg, while Setzer \cite{Setzer} computed all $\mathbb{Q}$-isomorphism classes of elliptic curves of conductor $p$ for certain small primes $p$. Each of these examples corresponds, via the [BR] approach, to cases with reducible forms. The first analysis on irreducible forms in \eqref{Eq:tmEquation} was carried out by Agrawal, Coates, Hunt and van der Poorten \cite{Agrawal}, who determined all elliptic curves of conductor $11$ defined over $\mathbb{Q}$ to verify the (then) conjecture of Taniyama-Weil.

There are very few, if any, subsequent attempts in the literature to find elliptic curves of given conductor via Thue-Mahler equations. Instead, many of the approaches involve a completely different method to the problem, using modular forms. This method relies upon the Modularity Theorem of Breuil, Conrad, Diamond and Taylor \cite{Breuil}, which was still a conjecture (under various guises) when these ideas were first implemented. Much of the success of this approach can be attributed to Cremona (see e.g. \cite{Cremona1}, \cite{Cremona2}) and his collaborators, who have devoted decades of work to it. In fact, using this method, all elliptic curves over $\mathbb{Q}$ of conductor $N$ have been determined for values of $N$ as follows
\begin{itemize} \itemsep0em
\item Antwerp IV ($1972$): $N \leq 200$
\item Tingley ($1975$): $N \leq 320$
\item Cremona ($1988$): $N \leq 600$
\item Cremona ($1990$): $N \leq 1000$
\item Cremona ($1997$): $N \leq 5077$
\item Cremona ($2001$): $N \leq 10000$
\item Cremona ($2005$): $N \leq 130000$
\item Cremona ($2014$): $N \leq 350000$
\item Cremona ($2015$): $N \leq 364000$
\item Cremona ($2016$): $N \leq 390000$.
\end{itemize}

In this thesis, we follow [BeGhRe] wherein we return to techniques based upon solving Thue-Mahler equations, using a number of results from classical invariant theory. In particular, we illustrate the connection between elliptic curves over $\mathbb{Q}$ and cubic forms and subsequently describe an effective algorithm for determining all elliptic curves over $\mathbb{Q}$ having good reduction outside $S$. This result can be summarized as follows. If we wish to find an elliptic curves $E$ of conductor $N = p_1^{a_1}\cdots p_v^{a_v}$ for some $a_i \in \mathbb{N}$, by Theorem 1 of [BeGhRe], there exists an integral binary cubic form $F$ of discriminant $N_0 \mid 12 N$ and relatively prime integers $u,v$ satisfying
\[F(u,v) = w_0u^3 + w_1u^2v + w_2uv^2 + w_3v^3 = 2^{\alpha_1}3^{\beta_1}\prod_{p|N_0}p^{\kappa_p}\]
for some $\alpha_1, \beta_1, \kappa_p$. Then $E$ is isomorphic over $\mathbb{Q}$ to the elliptic curve $E_{\mathcal{D}}$, where $E_{\mathcal{D}}$ is determined by the form $F$ and $(u,v)$. It is worth noting that Theorem 1 of [BeGhRe] very explicitly describes how to generate $E_{\mathcal{D}}$; once a solution $(u,v)$ to the Thue-Mahler equation $F$ is known, a quick computation of the Hessian and Jacobian discriminant of $F$ evaluated at $(u,v)$ yields the coefficients of $E_{\mathcal{D}}$. Using this theorem, all $E/\mathbb{Q}$ of conductor $N$ may be computed by generating all of the relevant binary cubic forms, solving the corresponding Thue-Mahler equations, and outputting the elliptic curves that arise. The first and last steps of this process are straightforward. Indeed, Bennett and Rechnitzer describe an efficient algorithm for carrying out the first step \aaron{REF}. In fact, they having carried out a one-time computation of all irreducible forms that can arise in Theorem 1 of absolute discriminant bounded by $10^{10}$. The bulk of the work is therefore concentrated in step 2, solving a large number of degree 3 Thue-Mahler equations. 

Unfortunately, despite many refinements, [Hambrook's] MAGMA implementation of a Thue-Mahler solver encounters a multitude of bottlenecks which often yield unavoidable timing and memory problems, even when parallelization is considered. As our aim is to use the results of [BeGhRe] to generate all elliptic curves over $\mathbb{Q}$ of conductor $N < 10^6$, in its current state, the Hambrook algorithm is inefficient for this task, and in many cases, simply unusable due to its memory requirements. The main novel contribution of this thesis is therefore the efficient resolution of an arbitrary degree $3$ Thue-Mahler equation and the implementation of this algorithm as a MAGMA package. This work is based on ideas of Matshke, von Kanel [CITE], and Siksek and is summarized in the following steps.



% OLD INTRO

%A Thue-Mahler equation is a Diophantine equation of the form
%\begin{equation} \label{Eq:tmEquation}
%F(x,y) = ap_1^{z_1}\cdots p_v^{z_v},
%\end{equation}
%where
%\[F(x,y) = f_0x^n + f_1x^{n-1}y + \cdots + f_{n-1}xy^{n-1} + f_ny^n\]
%is an irreducible binary form of degree at least $3$, $a$ is a nonzero integer, and $p_1, \dots, p_v$ are rational primes. The number of solutions in relatively prime integers $x$ and $y$ is known to be finite via the work of Mahler [CITE]. Though this proof is ineffective, it generalizes a classical result of Thue [CITE], who had proved an analogous statement for equations of the form $F(x,y) = a$. In the mid-1960's, Baker [CITE] proved his ground-breaking results on effective lower bounds for linear forms in logarithms of algebraic numbers. 
%
%Generalizing Baker's result to the $p$-adic case, Sprind\u zuk and Vinogradov [CITE] and Coates [CITE] proved that the solutions of any Thue-Mahler equation could, at least in principal, be effectively determined. The first practical method for solving the general Thue-Mahler equation over $\mathbb{Z}$ is attributed to Tzanakis and de Weger [CITE], whose ideas were inspired in part by the method of Agrawal, Coates, Hunt, and van der Poorten [CITE] in their work to solve the specific Thue-Mahler equation
%\[x^3 - x^2y + xy^2 + y^3 = \pm 11^{z_1}.\]

%In 2011, using optimized bounds arising from the theory of linear forms in logarithms, Hambrook [CITE] implemented a refined, automated version of the explicit method of [TdW] as a MAGMA package. Similar to [TdW], this algorithm begins by reducing the problem to one of solving a collection of finitely many $S$-unit equations in a certain algebraic number field $K$. Of course, by an $S$-unit, we mean an integer whose prime factors all lie in $S$. For each such equation, a very large upper bound on the solutions is generated using the theory of linear forms in logarithms. This bound is then reduced via Diophantine approximation techniques. Finally, the algorithm searches below this reduced bound using a combination of clever sieves and brute force. 
%
%Unfortunately, despite Hambrook's refinements, this algorithm encounters a multitude of bottlenecks which often yield unavoidable timing and memory problems, even when parallelization is considered. As we will outline shortly, one of our primary aims in this thesis is to solve a very large number of Thue-Mahler equations. In its current state, the Hambrook algorithm is inefficient for this task, and in many cases, simply unusable due to its memory requirements. The main novel contribution of this thesis is therefore the efficient resolution of an arbitrary Thue-Mahler equation and the implementation of this algorithm as a MAGMA package. This work is based on ideas of Matshke, von Kanel [CITE], and Siksek and is summarized in the following steps.

\textbf{Step 1.}
Following [TdW] and [Hambrook], we reduce the problem of solving the given Thue-Mahler equation to the problem of solving a collection of finitely many $S$-unit equations in a certain algebraic number field $K$. These are equations of the form
\begin{equation}\label{Introduction:SUnit}
\mu_0 y - \lambda_0 x = 1
\end{equation}
for some $\mu_0, \lambda_0 \in K$ and unknowns $x,y$. The collection of forms is such that if we know the solutions of each equation in the collection, then we can easily derive all of the solutions of the Thue-Mahler equation. This reduction is performed in two steps. First, \eqref{Introduction:ThueMahler} is reduced to a finite number of ideal equations over $K$. Here, we employ new results by Siksek [Cite?] to significantly reduce the number of ideal equations to consider. Next, we reduce each ideal equation to a number of certain $S$-unit equations \eqref{Introduction:SUnit} via a finite number of principalization tests. The method of [TdW] reduces \eqref{Introduction:ThueMahler} to $(m/2) h^v$ $S$-unit equations, where $m$ is the number of roots of unity of $K$, $h$ is the class number, and $v$ is the number of rational primes $p_1, \dots, p_v$. The method of Siksek that we employ gives only $m/2$ $S$-unit equations. The principle computational work here consists of computing an integral basis, a system of fundamental units, and a splitting field of $K$, as well as computing the class group of $K$ and the factorizations of the primes $p_1,\dots,p_v$ into prime ideals in the ring of integers of $K$. 

The remaining steps are performed for each of the $S$-unit equations in our collection. 

\textbf{Step 2.}
In place of the logarithmic sieves used in [TdW] to derive a large upper bound, we work with the global logarithmic Weil height
\[h: \mathbb{G}_m(\overline{\mathbb{Q}}) \to \mathbb{R}_{\geq 0}.\]
For a given \eqref{Introduction:SUnit}, we show that the height $h(1/x)$ admits a decomposition into local heights at each place of $K$ appearing in the $S$-unit equation. Using [CITE : Matshke, von Kanel], we generate a very large upper bound on the height $h(1/x)$, and subsequently, on the local heights. This step is a straightforward computation, whereas the analogous step in Hambrook and TdW is a complex and lengthy derivation which involves factoring rational primes into prime ideals in a splitting field of $K$ and computing heights of certain elements of the splitting field. 

\textbf{Step 3.}
For each place of $K$ appearing in \eqref{Introduction:SUnit}, we drastically reduce the upper bounds derived in Step 2 by using computational Diophantine approximation techniques applied to the intersection of a certain ellipsoid and translated lattice. This technique involves using the Finke-Pohst algorithm to enumerate all short vectors in the intersection. Here, working with the Weil height $h(1/x)$ has the advantage that it leads to ellipsoids whose volumes are smaller than the ellipsoids implicitly used in [TdW] by a factor of $\sim r^{r/2}$ for $r$ the number of places of $K$ appearing in our $S$-unit equation. In this way, we reduce the number of short vectors appearing from the Fincke-Pohst algorithm, and consequently reduce our running time and memory requirements. 

\textbf{Step 4.}
Samir's sieve - this may not be done in time as we only just received Samir's writeup and explanation as pertaining to Thue-Mahler equations.

\textbf{Step 5.}
Finally, we use a sieving procedure to find all the solutions of the Diophantine equation that live in the box defined by the bounds derived in the previous steps. To carry out this step, we run through all the possible solutions in the box and �sieve� out the vast majority of non-solutions. This is done via certain low-cost congruence tests. The candidate solutions passing this test are then verified directly against \eqref{Introduction:SUnit}. Though we expect the bounds defining the box to be small, there can still be a very large number of possible solutions to check, especially if the number of rational primes involved in the Thue-Mahler equation is large. The computations performed on each individual candidate solution are relatively simple, but the sheer number of candidates often makes this step the computational bottleneck of the entire algorithm. 

\textbf{Step 6.}
Having performed Steps 2-5 for each $S$-unit equation in our collection, we now have all the solutions of each such equation, and we use this knowledge to determine all the solutions of the Thue-Mahler equation. 

The reader will notice several parallels between this refined algorithm and the aforementioned Goormaghtigh equation solver in the case $n=5$. In particular, both algorithms share the same setup and refinements of the [TdW] and [Hambrook] solver. For \eqref{Introduction:Goormaghtigh}, however, we are left to solve
\[f(y) = x^m,\]
a Thue-Mahler-like equation of degree $4$ in explicit values of $x$ and unknown integers $y$ and $m$. In this case, we are permitted simplifications which allow us to omit the Fincke-Pohst algorithm and final congruence sieves. Instead, for each $x$, we rely on only a few iterations of the LLL algorithm to reduce our initial bound on the exponents before entering a naive search to complete our computation. Of course, this algorithm can be refined further for efficiency, however, in the context of [BeGhKr], such improvements are not needed. 

The outline of this thesis is as follows. ADD


%--------------------------------------------------------------------------------------------------------------------------------------------%
%--------------------------------------------------------------------------------------------------------------------------------------------%

\endinput

Any text after an \endinput is ignored.
You could put scraps here or things in progress.



This document provides a quick set of instructions for using the
\class{ubcdiss} class to write a dissertation in \LaTeX. 
Unfortunately this document cannot provide an introduction to using
\LaTeX.  The classic reference for learning \LaTeX\ is
\citeauthor{lamport-1994-ladps}'s
book~\cite{lamport-1994-ladps}.  There are also many freely-available
tutorials online;
\webref{http://www.andy-roberts.net/misc/latex/}{Andy Roberts' online
    \LaTeX\ tutorials}
seems to be excellent.
The source code for this docment, however, is intended to serve as
an example for creating a \LaTeX\ version of your dissertation.

We start by discussing organizational issues, such as splitting
your dissertation into multiple files, in
\autoref{sec:SuggestedThesisOrganization}.
We then cover the ease of managing cross-references in \LaTeX\ in
\autoref{sec:CrossReferences}.
We cover managing and using bibliographies with \BibTeX\ in
\autoref{sec:BibTeX}. 
We briefly describe typesetting attractive tables in
\autoref{sec:TypesettingTables}.
We briefly describe including external figures in
\autoref{sec:Graphics}, and using special characters and symbols
in \autoref{sec:SpecialSymbols}.
As it is often useful to track different versions of your dissertation,
we discuss revision control further in
\autoref{sec:DissertationRevisionControl}. 
We conclude with pointers to additional sources of information in
\autoref{sec:Conclusions}.

%--------------------------------------------------------------------------------------------------------------------------------------------%
%--------------------------------------------------------------------------------------------------------------------------------------------%
\section{Suggested Thesis Organization}
\label{sec:SuggestedThesisOrganization}

The \acs{UBC} \acf{GPS} specifies a particular arrangement of the
components forming a thesis.\footnote{See
    \url{http://www.grad.ubc.ca/current-students/dissertation-thesis-preparation/order-components}}
This template reflects that arrangement.

In terms of writing your thesis, the recommended best practice for
organizing large documents in \LaTeX\ is to place each chapter in
a separate file.  These chapters are then included from the main
file through the use of \verb+\include{file}+.  A thesis might
be described as six files such as \file{intro.tex},
\file{relwork.tex}, \file{model.tex}, \file{eval.tex},
\file{discuss.tex}, and \file{concl.tex}.

We also encourage you to use macros for separating how something
will be typeset (\eg bold, or italics) from the meaning of that
something. 
For example, if you look at \file{intro.tex}, you will see repeated
uses of a macro \verb+\file{}+ to indicate file names.
The \verb+\file{}+ macro is defined in the file \file{macros.tex}.
The consistent use of \verb+\file{}+ throughout the text not only
indicates that the argument to the macro represents a file (providing
meaning or semantics), but also allows easily changing how
file names are typeset simply by changing the definition of the
\verb+\file{}+ macro.
\file{macros.tex} contains other useful macros for properly typesetting
things like the proper uses of the latinate \emph{exempli grati\={a}}
and \emph{id est} (\ie \verb+\eg+ and \verb+\ie+), 
web references with a footnoted \acs{URL} (\verb+\webref{url}{text}+),
as well as definitions specific to this documentation
(\verb+\latexpackage{}+).

%--------------------------------------------------------------------------------------------------------------------------------------------%
\section{Making Cross-References}
\label{sec:CrossReferences}

\LaTeX\ make managing cross-references easy, and the \latexpackage{hyperref}
package's\ \verb+\autoref{}+ command\footnote{%
    The \latexpackage{hyperref} package is included by default in this
    template.}
makes it easier still. 

A thing to be cross-referenced, such as a section, figure, or equation,
is \emph{labelled} using a unique, user-provided identifier, defined
using the \verb+\label{}+ command.  
The thing is referenced elsewhere using the \verb+\autoref{}+ command.
For example, this section was defined using:
\begin{lstlisting}
    \section{Making Cross-References}
    \label{sec:CrossReferences}
\end{lstlisting}
References to this section are made as follows:
\begin{lstlisting}
    We then cover the ease of managing cross-references in \LaTeX\
    in \autoref{sec:CrossReferences}.
\end{lstlisting}
\verb+\autoref{}+ takes care of determining the \emph{type} of the 
thing being referenced, so the example above is rendered as
\begin{quote}
    We then cover the ease of managing cross-references in \LaTeX\
    in \autoref{sec:CrossReferences}.
\end{quote}

The label is any simple sequence of characters, numbers, digits,
and some punctuation marks such as ``:'' and ``--''; there should
be no spaces.  Try to use a consistent key format: this simplifies
remembering how to make references.  This document uses a prefix
to indicate the type of the thing being referenced, such as \texttt{sec}
for sections, \texttt{fig} for figures, \texttt{tbl} for tables,
and \texttt{eqn} for equations.

For details on defining the text used to describe the type
of \emph{thing}, search \file{diss.tex} and the \latexpackage{hyperref}
documentation for \texttt{autorefname}.


%--------------------------------------------------------------------------------------------------------------------------------------------%
\section{Managing Bibliographies with \BibTeX}
\label{sec:BibTeX}

One of the primary benefits of using \LaTeX\ is its companion program,
\BibTeX, for managing bibliographies and citations.  Managing
bibliographies has three parts: (i) describing references,
(ii)~citing references, and (iii)~formatting cited references.

\subsection{Describing References}

\BibTeX\ defines a standard format for recording details about a
reference.  These references are recorded in a file with a
\file{.bib} extension.  \BibTeX\ supports a broad range of
references, such as books, articles, items in a conference proceedings,
chapters, technical reports, manuals, dissertations, and unpublished
manuscripts. 
A reference may include attributes such as the authors,
the title, the page numbers, the \ac{DOI}, or a \ac{URL}.  A reference
can also be augmented with personal attributes, such as a rating,
notes, or keywords.

Each reference must be described by a unique \emph{key}.\footnote{%
    Note that the citation keys are different from the reference
    identifiers as described in \autoref{sec:CrossReferences}.}
A key is a simple sequence of characters, numbers, digits, and some
punctuation marks such as ``:'' and ``--''; there should be no spaces. 
A consistent key format simiplifies remembering how to make references. 
For example:
\begin{quote}
   \fbox{\emph{last-name}}\texttt{-}\fbox{\emph{year}}\texttt{-}\fbox{\emph{contracted-title}}
\end{quote}
where \emph{last-name} represents the last name for the first author,
and \emph{contracted-title} is some meaningful contraction of the
title.  Then \citeauthor{kiczales-1997-aop}'s seminal article on
aspect-oriented programming~\cite{kiczales-1997-aop} (published in
\citeyear{kiczales-1997-aop}) might be given the key
\texttt{kiczales-1997-aop}.

An example of a \BibTeX\ \file{.bib} file is included as
\file{biblio.bib}.  A description of the format a \file{.bib}
file is beyond the scope of this document.  We instead encourage
you to use one of the several reference managers that support the
\BibTeX\ format such as
\webref{http://jabref.sourceforge.net}{JabRef} (multiple platforms) or
\webref{http://bibdesk.sourceforge.net}{BibDesk} (MacOS\,X only). 
These front ends are similar to reference manages such as
EndNote or RefWorks.


\subsection{Citing References}

Having described some references, we then need to cite them.  We
do this using a form of the \verb+\cite+ command.  For example:
\begin{lstlisting}
    \citet{kiczales-1997-aop} present examples of crosscutting 
    from programs written in several languages.
\end{lstlisting}
When processed, the \verb+\citet+ will cause the paper's authors
and a standardized reference to the paper to be inserted in the
document, and will also include a formatted citation for the paper
in the bibliography.  For example:
\begin{quote}
    \citet{kiczales-1997-aop} present examples of crosscutting 
    from programs written in several languages.
\end{quote}
There are several forms of the \verb+\cite+ command (provided
by the \latexpackage{natbib} package), as demonstrated in
\autoref{tbl:natbib:cite}.
Note that the form of the citation (numeric or author-year) depends
on the bibliography style (described in the next section).
The \verb+\citet+ variant is used when the author names form
an object in the sentence, whereas the \verb+\citep+ variant
is used for parenthetic references, more like an end-note.
Use \verb+\nocite+ to include a citation in the bibliography
but without an actual reference.
\nocite{rowling-1997-hpps}
\begin{table}
\caption{Available \texttt{cite} variants; the exact citation style
    depends on whether the bibliography style is numeric or author-year.}
\label{tbl:natbib:cite}
\centering
\begin{tabular}{lp{3.25in}}\toprule
Variant & Result \\
\midrule
% We cheat here to simulate the cite/citep/citet for APA-like styles
\verb+\cite+ & Parenthetical citation (\eg ``\cite{kiczales-1997-aop}''
    or ``(\citeauthor{kiczales-1997-aop} \citeyear{kiczales-1997-aop})'') \\
\verb+\citet+ & Textual citation: includes author (\eg
    ``\citet{kiczales-1997-aop}'' or
    or ``\citeauthor{kiczales-1997-aop} (\citeyear{kiczales-1997-aop})'') \\
\verb+\citet*+ & Textual citation with unabbreviated author list \\
\verb+\citealt+ & Like \verb+\citet+ but without parentheses \\
\verb+\citep+ & Parenthetical citation (\eg ``\cite{kiczales-1997-aop}''
    or ``(\citeauthor{kiczales-1997-aop} \citeyear{kiczales-1997-aop})'') \\
\verb+\citep*+ & Parenthetical citation with unabbreviated author list \\
\verb+\citealp+ & Like \verb+\citep+ but without parentheses \\
\verb+\citeauthor+ & Author only (\eg ``\citeauthor{kiczales-1997-aop}'') \\
\verb+\citeauthor*+ & Unabbreviated authors list 
    (\eg ``\citeauthor*{kiczales-1997-aop}'') \\
\verb+\citeyear+ & Year of citation (\eg ``\citeyear{kiczales-1997-aop}'') \\
\bottomrule
\end{tabular}
\end{table}

\subsection{Formatting Cited References}

\BibTeX\ separates the citing of a reference from how the cited
reference is formatted for a bibliography, specified with the
\verb+\bibliographystyle+ command. 
There are many varieties, such as \texttt{plainnat}, \texttt{abbrvnat},
\texttt{unsrtnat}, and \texttt{vancouver}.
This document was formatted with \texttt{abbrvnat}.
Look through your \TeX\ distribution for \file{.bst} files. 
Note that use of some \file{.bst} files do not emit all the information
necessary to properly use \verb+\citet{}+, \verb+\citep{}+,
\verb+\citeyear{}+, and \verb+\citeauthor{}+.

There are also packages available to place citations on a per-chapter
basis (\latexpackage{bibunits}), as footnotes (\latexpackage{footbib}),
and inline (\latexpackage{bibentry}).
Those who wish to exert maximum control over their bibliography
style should see the amazing \latexpackage{custom-bib} package.

%--------------------------------------------------------------------------------------------------------------------------------------------%
\section{Typesetting Tables}
\label{sec:TypesettingTables}

\citet{lamport-1994-ladps} made one grievous mistake
in \LaTeX: his suggested manner for typesetting tables produces
typographic abominations.  These suggestions have unfortunately
been replicated in most \LaTeX\ tutorials.  These
abominations are easily avoided simply by ignoring his examples
illustrating the use of horizontal and vertical rules (specifically
the use of \verb+\hline+ and \verb+|+) and using the
\latexpackage{booktabs} package instead.

The \latexpackage{booktabs} package helps produce tables in the form
used by most professionally-edited journals through the use of
three new types of dividing lines, or \emph{rules}.
% There are times that you don't want to use \autoref{}
Tables~\ref{tbl:natbib:cite} and~\ref{tbl:LaTeX:Symbols} are two
examples of tables typeset with the \latexpackage{booktabs} package.
The \latexpackage{booktabs} package provides three new commands
for producing rules:
\verb+\toprule+ for the rule to appear at the top of the table,
\verb+\midrule+ for the middle rule following the table header,
and \verb+\bottomrule+ for the bottom-most at the end of the table.
These rules differ by their weight (thickness) and the spacing before
and after.
A table is typeset in the following manner:
\begin{lstlisting}
    \begin{table}
    \caption{The caption for the table}
    \label{tbl:label}
    \centering
    \begin{tabular}{cc}
    \toprule
    Header & Elements \\
    \midrule
    Row 1 & Row 1 \\
    Row 2 & Row 2 \\
    % ... and on and on ...
    Row N & Row N \\
    \bottomrule
    \end{tabular}
    \end{table}
\end{lstlisting}
See the \latexpackage{booktabs} documentation for advice in dealing with
special cases, such as subheading rules, introducing extra space
for divisions, and interior rules.

%--------------------------------------------------------------------------------------------------------------------------------------------%
\section{Figures, Graphics, and Special Characters}
\label{sec:Graphics}

Most \LaTeX\ beginners find figures to be one of the more challenging
topics.  In \LaTeX, a figure is a \emph{floating element}, to be
placed where it best fits.
The user is not expected to concern him/herself with the placement
of the figure.  The figure should instead be labelled, and where
the figure is used, the text should use \verb+\autoref+ to reference
the figure's label.
\autoref{fig:latex-affirmation} is an example of a figure.
\begin{figure}
    \centering
    % For the sake of this example, we'll just use text
    %\includegraphics[width=3in]{file}
    \Huge{\textsf{\LaTeX\ Rocks!}}
    \caption{Proof of \LaTeX's amazing abilities}
    \label{fig:latex-affirmation}   % label should change
\end{figure}
A figure is generally included as follows:
\begin{lstlisting}
    \begin{figure}
    \centering
    \includegraphics[width=3in]{file}
    \caption{A useful caption}
    \label{fig:fig-label}   % label should change
    \end{figure}
\end{lstlisting}
There are three items of note:
\begin{enumerate}
\item External files are included using the \verb+\includegraphics+
    command.  This command is defined by the \latexpackage{graphicx} package
    and can often natively import graphics from a variety of formats.
    The set of formats supported depends on your \TeX\ command processor.
    Both \texttt{pdflatex} and \texttt{xelatex}, for example, can
    import \textsc{gif}, \textsc{jpg}, and \textsc{pdf}.  The plain
    version of \texttt{latex} only supports \textsc{eps} files.

\item The \verb+\caption+ provides a caption to the figure. 
    This caption is normally listed in the List of Figures; you
    can provide an alternative caption for the LoF by providing
    an optional argument to the \verb+\caption+ like so:
    \begin{lstlisting}
    \caption[nice shortened caption for LoF]{%
	longer detailed caption used for the figure}
    \end{lstlisting}
    \ac{GPS} generally prefers shortened single-line captions
    in the LoF: multiple-line captions are a bit unwieldy.

\item The \verb+\label+ command provides for associating a unique, user-defined,
    and descriptive identifier to the figure.  The figure can be
    can be referenced elsewhere in the text with this identifier
    as described in \autoref{sec:CrossReferences}.
\end{enumerate}
See Keith Reckdahl’s excellent guide for more details,
\webref{http://www.ctan.org/tex-archive/info/epslatex.pdf}{\emph{Using
imported graphics in LaTeX2e}}.

\section{Special Characters and Symbols}
\label{sec:SpecialSymbols}

\LaTeX\ appropriates many common symbols for its own purposes,
with some used for commands (\ie \verb+\{}&%+) and
mathematics (\ie \verb+$^_+), and others are automagically transformed
into typographically-preferred forms (\ie \verb+-`'+) or to
completely different forms (\ie \verb+<>+).
\autoref{tbl:LaTeX:Symbols} presents a list of common symbols and
their corresponding \LaTeX\ commands.  A much more comprehensive list 
of symbols and accented characters is available at:
\url{http://www.ctan.org/tex-archive/info/symbols/comprehensive/}
\begin{table}
\caption{Useful \LaTeX\ symbols}\label{tbl:LaTeX:Symbols}
\centering\begin{tabular}{ccp{0.5cm}cc}\toprule
\LaTeX & Result && \LaTeX & Result \\
\midrule
    \verb+\texttrademark+ & \texttrademark && \verb+\&+ & \& \\
    \verb+\textcopyright+ & \textcopyright && \verb+\{ \}+ & \{ \} \\
    \verb+\textregistered+ & \textregistered && \verb+\%+ & \% \\
    \verb+\textsection+ & \textsection && \verb+\verb!~!+ & \verb!~! \\
    \verb+\textdagger+ & \textdagger && \verb+\$+ & \$ \\
    \verb+\textdaggerdbl+ & \textdaggerdbl && \verb+\^{}+ & \^{} \\
    \verb+\textless+ & \textless && \verb+\_+ & \_ \\
    \verb+\textgreater+ & \textgreater && \\
\bottomrule
\end{tabular}
\end{table}

%--------------------------------------------------------------------------------------------------------------------------------------------%\section{Changing Page Widths and Heights}

The \class{ubcdiss} class is based on the standard \LaTeX\ \class{book}
class~\cite{lamport-1994-ladps} that selects a line-width to carry
approximately 66~characters per line.  This character density is
claimed to have a pleasing appearance and also supports more rapid
reading~\cite{bringhurst-2002-teots}.  I would recommend that you
not change the line-widths!

\subsection{The \texttt{geometry} Package}

Some students are unfortunately saddled with misguided supervisors
or committee members whom believe that documents should have the
narrowest margins possible.  The \latexpackage{geometry} package is
helpful in such cases.  Using this package is as simple as:
\begin{lstlisting}
    \usepackage[margin=1.25in,top=1.25in,bottom=1.25in]{geometry}
\end{lstlisting}
You should check the package's documentation for more complex uses.

\subsection{Changing Page Layout Values By Hand}

There are some miserable students with requirements for page layouts
that vary throughout the document.  Unfortunately the
\latexpackage{geometry} can only be specified once, in the document's
preamble.  Such miserable students must set \LaTeX's layout parameters
by hand:
\begin{lstlisting}
    \setlength{\topmargin}{-.75in}
    \setlength{\headsep}{0.25in}
    \setlength{\headheight}{15pt}
    \setlength{\textheight}{9in}
    \setlength{\footskip}{0.25in}
    \setlength{\footheight}{15pt}

    % The *sidemargin values are relative to 1in; so the following
    % results in a 0.75 inch margin
    \setlength{\oddsidemargin}{-0.25in}
    \setlength{\evensidemargin}{-0.25in}
    \setlength{\textwidth}{7in}       % 1.1in margins (8.5-2*0.75)
\end{lstlisting}
These settings necessarily require assuming a particular page height
and width; in the above, the setting for \verb+\textwidth+ assumes
a \textsc{US} Letter with an 8.5'' width.
The \latexpackage{geometry} package simply uses the page height and
other specified values to derive the other layout values.
The
\href{http://tug.ctan.org/tex-archive/macros/latex/required/tools/layout.pdf}{\texttt{layout}}
package provides a
handy \verb+\layout+ command to show the current page layout
parameters. 


\subsection{Making Temporary Changes to Page Layout}

There are occasions where it becomes necessary to make temporary
changes to the page width, such as to accomodate a larger formula. 
The \latexmiscpackage{chngpage} package provides an \env{adjustwidth}
environment that does just this.  For example:
\begin{lstlisting}
    % Expand left and right margins by 0.75in
    \begin{adjustwidth}{-0.75in}{-0.75in}
    % Must adjust the perceived column width for LaTeX to get with it.
    \addtolength{\columnwidth}{1.5in}
    \[ an extra long math formula \]
    \end{adjustwidth}
\end{lstlisting}


%--------------------------------------------------------------------------------------------------------------------------------------------%
\section{Keeping Track of Versions with Revision Control}
\label{sec:DissertationRevisionControl}

Software engineers have used \acf{RCS} to track changes to their
software systems for decades.  These systems record the changes to
the source code along with context as to why the change was required.
These systems also support examining and reverting to particular
revisions from their system's past.

An \ac{RCS} can be used to keep track of changes to things other
than source code, such as your dissertation.  For example, it can
be useful to know exactly which revision of your dissertation was
sent to a particular committee member.  Or to recover an accidentally
deleted file, or a badly modified image.  With a revision control
system, you can tag or annotate the revision of your dissertation
that was sent to your committee, or when you incorporated changes
from your supervisor.

Unfortunately current revision control packages are not yet targetted
to non-developers.  But the Subversion project's
\webref{http://tortoisesvn.net/docs/release/TortoiseSVN_en/}{TortoiseSVN}
has greatly simplified using the Subversion revision control system
for Windows users.  You should consult your local geek.

A simpler alternative strategy is to create a GoogleMail account
and periodically mail yourself zipped copies of your dissertation.

%--------------------------------------------------------------------------------------------------------------------------------------------%
\section{Recommended Packages}

The real strength to \LaTeX\ is found in the myriad of free add-on
packages available for handling special formatting requirements.
In this section we list some helpful packages.

\subsection{Typesetting}

\begin{description}
\item[\latexpackage{enumitem}:]
    Supports pausing and resuming enumerate environments.

\item[\latexpackage{ulem}:]
    Provides two new commands for striking out and crossing out text
    (\verb+\sout{text}+ and \verb+\xout{text}+ respectively)
    The package should likely
    be used as follows:
    \begin{verbatim}
    \usepackage[normalem,normalbf]{ulem}
    \end{verbatim}
    to prevent the package from redefining the emphasis and bold fonts.

\item[\latexpackage{chngpage}:]
    Support changing the page widths on demand.

\item[\latexpackage{mhchem}:] 
    Support for typesetting chemical formulae and reaction equations.

\end{description}

Although not a package, the
\webref{http://www.ctan.org/tex-archive/support/latexdiff/}{\texttt{latexdiff}}
command is very useful for creating changebar'd versions of your
dissertation.


\subsection{Figures, Tables, and Document Extracts}

\begin{description}
\item[\latexpackage{pdfpages}:]
    Insert pages from other PDF files.  Allows referencing the extracted
    pages in the list of figures, adding labels to reference the page
    from elsewhere, and add borders to the pages.

\item[\latexpackage{subfig}:]
    Provides for including subfigures within a figure, and includes
    being able to separately reference the subfigures.  This is a
    replacement for the older \texttt{subfigure} environment.

\item[\latexpackage{rotating}:]
    Provides two environments, sidewaystable and sidewaysfigure,
    for typesetting tables and figures in landscape mode.  

\item[\latexpackage{longtable}:]
    Support for long tables that span multiple pages.

\item[\latexpackage{tabularx}:]
    Provides an enhanced tabular environment with auto-sizing columns.

\item[\latexpackage{ragged2e}:]
    Provides several new commands for setting ragged text (\eg forms
    of centered or flushed text) that can be used in tabular
    environments and that support hyphenation.

\end{description}


\subsection{Bibliography Related Packages}

\begin{description}
\item[\latexpackage{bibunits}:]
    Support having per-chapter bibliographies.

\item[\latexpackage{footbib}:]
    Cause cited works to be rendered using footnotes.

\item[\latexpackage{bibentry}:] 
    Support placing the details of a cited work in-line.

\item[\latexpackage{custom-bib}:]
    Generate a custom style for your bibliography.

\end{description}


%--------------------------------------------------------------------------------------------------------------------------------------------%
\section{Moving On}
\label{sec:Conclusions}

At this point, you should be ready to go.  Other handy web resources:
\begin{itemize}
\item \webref{http://www.ctan.org}{\ac{CTAN}} is \emph{the} comprehensive
    archive site for all things related to \TeX\ and \LaTeX. 
    Should you have some particular requirement, somebody else is
    almost certainly to have had the same requirement before you,
    and the solution will be found on \ac{CTAN}.  The links to
    various packages in this document are all to \ac{CTAN}.

\item An online
    \webref{http://www.ctan.org/get/info/latex2e-help-texinfo/latex2e.html}{%
	reference to \LaTeX\ commands} provides a handy quick-reference
    to the standard \LaTeX\ commands.

\item The list of 
    \webref{http://www.tex.ac.uk/cgi-bin/texfaq2html?label=interruptlist}{%
	Frequently Asked Questions about \TeX\ and \LaTeX}
    can save you a huge amount of time in finding solutions to
    common problems.

\item The \webref{http://www.tug.org/tetex/tetex-texmfdist/doc/}{te\TeX\
    documentation guide} features a very handy list of the most useful
    packages for \LaTeX\ as found in \ac{CTAN}.

\item The
\webref{http://www.ctan.org/tex-archive/macros/latex/required/graphics/grfguide.pdf}{\texttt{color}}
    package, part of the graphics bundle, provides handy commands
    for changing text and background colours.  Simply changing
    text to various levels of grey can have a very 
    \textcolor{greytext}{dramatic effect}.


\item If you're really keen, you might want to join the
    \webref{http://www.tug.org}{\TeX\ Users Group}.

\end{itemize}




%    2. Main body
% Generally recommended to put each chapter into a separate file
%\include{relatedwork}
%\include{model}
%\include{impl}
%\include{discussion}
%\include{conclusions}

%% The following is a directive for TeXShop to indicate the main file
%%!TEX root = diss.tex

\chapter{Preliminaries}
\label{ch:Preliminaries}

%--------------------------------------------------------------------------------------------------------------------------------------------%
%--------------------------------------------------------------------------------------------------------------------------------------------%

\section{Algebraic number theory} 
\label{sec:AlgebraicNumberTheory}

In this section we recall some basic results from algebraic number theory that we use throughout the remaining chapters. We refer to \cite{Mar} and \cite{Neuk} for full details. 

Let $K$ be a finite algebraic extension of $\mathbb{Q}$ of degree $n = [K:\mathbb{Q}]$. There are $n$ embeddings $\sigma: K \to \mathbb{C}$. These embeddings can be described by writing $K = \mathbb{Q}(\theta)$ for some $\theta \in \mathbb{C}$ and observing that $\theta$ can be sent to any one of its conjugates. 
Let $s$ denote the number of real embeddings of $K$ and let $t$ denote the number of conjugate pairs of complex embeddings of $K$, where $n = s + 2t$. By Dirichlet's Unit Theorem, the group of units of $K$ is the direct product of a finite cyclic group consisting of the roots of unity in $K$ and a free abelian group of rank $r = s + t -1$. Equivalently, there exists a system of $r$ independent units $\eps_1, \dots, \eps_r$ such that the group of units of $K$ is given by 
\[\left\{\zeta \cdot \eps_1^{a_1}\cdots \eps_r^{a_r} \ : \ \zeta \text{ a root of unity}, a_i \in \mathbb{Z} \text{ for } i = 1, \dots, r\right\}.\]
Any set of independent units that generate the torsion-free part of the unit group is called a system of \textit{fundamental units}. 

An element $\alpha \in K$ is called an \textit{algebraic integer} if its minimal polynomial over $\mathbb{Z}$ is monic. The set of algebraic integers in $K$ forms a ring, denoted $\mathcal{O}_K$. We refer to this ring as the \textit{ring of integers} or \textit{number ring} corresponding to the number field $K$. For any $\alpha \in K$, we define the \textit{norm} of $\alpha$ as 
\[N_{K/\mathbb{Q}}(\alpha) = \prod_{\sigma:K \to \mathbb{C}} \sigma(\alpha)\]
where the product is taken over all embeddings $\sigma$ of $K$. For algebraic integers, $N_{K/\mathbb{Q}}(\alpha) \in \mathbb{Z}$. The units are precisely the elements of norm $\pm 1$. Two elements $\alpha, \beta$ of $K$ are called \textit{associates} if there exists a unit $\eps$ such that $\alpha = \eps \beta$. Let $(\alpha)\mathcal{O}_K$ denote the ideal generated by $\alpha$. Associated elements generate the same ideal, and distinct generators of an ideal are associated. There exist only finitely many non-associated algebraic integers in $K$ with given norm. 

Any element of the ring of integers can be written as a product of \textit{irreducible} elements. These are non-zero non-unit elements of $\mathcal{O}_K$ which have no integral divisors but their own associates. Unfortunately, number rings are not alway unique factorization domains: this decomposition into irreducible elements may not be unique. However, every number ring is a Dedekind domain. This means that every ideal can be decomposed into a product of prime ideals and this decomposition is unique. A \textit{principal} ideal is an ideal generated by a single element $\alpha$. Two fractional ideals are called equivalent if their quotient is principal. It is well known that there are only finitely many equivalence classes of fractional ideals and the set of all such classes forms a finite abelian group called the \textit{ideal class group}, $\Cl(K)$. The number of ideal classes, $\#\Cl(K)$, is called the \textit{class number} of $\mathcal{O}_K$ and is denoted by $h_K$. For an ideal $\mathfrak{a}$, it is always true that $\mathfrak{a}^{h_K}$ is principal. The norm of the (integral) ideal $\mathfrak{a}$ is defined by $N_{K/\mathbb{Q}}(\mathfrak{a}) = \#\left(\mathcal{O}_K/\mathfrak{a}\right)$. If $\mathfrak{a} = (\alpha) \mathcal{O}_K$ is a principal ideal, then $N_{K/\mathbb{Q}}(\mathfrak{a}) = \left|N_{K/\mathbb{Q}}(\alpha)\right|$. 

Let $L$ be a finite field extension of $K$ with ring of integers $\mathcal{O}_L$. Every prime ideal $\mathfrak{P}$ of $\mathcal{O}_L$ \textit{lies over} a unique prime ideal $\mathfrak{p}$ in $\mathcal{O}_K$. That is, $\mathfrak{P}$ divides $\mathfrak{p}$. The \textit{ramification index} $e({\mathfrak{P}}|\mathfrak{p})$ is the largest power to which $\mathfrak{P}$ divides $\mathfrak{p}$. The field $\mathcal{O}_L/\mathfrak{P}$ is an extension of finite degree $f(\mathfrak{P}|\mathfrak{p})$ over $\mathcal{O}_K/\mathfrak{p}$. We call $f(\mathfrak{P}|\mathfrak{p})$ the \textit{inertial degree} of $\mathfrak{P}$ over $\mathfrak{p}$. For $\mathfrak{p}$ lying over the rational prime $p$, this is the integer such that 
\[N_{K/\mathbb{Q}}(\mathfrak{p}) = p^{f(\mathfrak{p}|p)}.\]
The ramification index and inertial degree are multiplicative in a tower of fields. In particular, if $\mathfrak{P}$ lies over $\mathfrak{p}$ which lies over the rational prime $p$, then
\[e({\mathfrak{P}}|p) = e({\mathfrak{P}}|\mathfrak{p})e({\mathfrak{p}}|p) \quad \text{ and } \quad f({\mathfrak{P}}|p) = f({\mathfrak{P}}|\mathfrak{p})f({\mathfrak{p}}|p).\]
Let $\mathfrak{P}_1, \dots, \mathfrak{P}_m$ be the primes of $\mathcal{O}_L$ lying over a prime ideal $\mathfrak{p}$ of $\mathcal{O}_K$. Denote by $e({\mathfrak{P}}_1|\mathfrak{p}),\dots, e({\mathfrak{P}}_m|\mathfrak{p})$ and $f({\mathfrak{P}}_1|\mathfrak{p}), \dots, f({\mathfrak{P}}_m|\mathfrak{p})$ the corresponding ramification indices and inertial degrees. Then
\[\sum_{i=1}^m e({\mathfrak{P}}_i|\mathfrak{p})f({\mathfrak{P}}_i|\mathfrak{p}) = [L:K].\]

If $L$ is normal over $K$ and $\mathfrak{P}_i$ and $\mathfrak{P}_j$ are two prime ideals lying over $\mathfrak{p}$, then $e({\mathfrak{P}}_i|\mathfrak{p}) = e({\mathfrak{P}}_j|\mathfrak{p})$ and $f({\mathfrak{P}}_i|\mathfrak{p}) = f({\mathfrak{P}}_j|\mathfrak{p})$. In this case, $\mathfrak{p}$ factors as
\[\mathfrak{p}\mathcal{O}_L = \left( \mathfrak{P}_1 \cdots \mathfrak{P}_m\right)^e\]
in $\mathcal{O}_L$, where the $\mathfrak{P}_i$ are distinct prime ideals all having the same ramification degree $e$ and inertial degree $f$ over $\mathfrak{p}$. It follows that 
\[mef = [L:K].\]

%--------------------------------------------------------------------------------------------------------------------------------------------%
%--------------------------------------------------------------------------------------------------------------------------------------------%

\section{$p$-adic valuations}
\label{sec:pAdicValuations}

In this section we give a concise exposition of $p$-adic valuations. As references for this material we give \cite{BS} (especially Theorem 3 in Chapter 4, Section 2), \cite{Ca} (especially Lemma 2.1 in Chapter 9), \cite{Has2} (especially Chapter 18), \cite{Ko} (especially Chapter 3, Section 2), and \cite{Nark} (especially Theorem 6.1).

We denote the algebraic closure of $\mathbb{Q}_p$ by $\overline{\mathbb{Q}}_p$. The completion of $\overline{\mathbb{Q}}_p$ with respect to the absolute value of $\overline{\mathbb{Q}}_p$ is denoted by $\mathbb{C}_p$.

Let $K$ be an arbitrary number field. A homomorphism $v: K^* \to \mathbb{R}_{\geq 0}$ of the multiplicative group of $K$ into the group of positive real numbers is called a \textit{valuation} if it satisfies the condition
\[v(x+y) \leq v(x) + v(y).\]
This definition may be extended to all of $K$ by setting $v(0) = 0$. If
\[v(x+y) \leq \max(v(x),v(y))\]
holds for all $x,y \in K$, then $v$ is called a \textit{non-Archimedean valuation}. All remaining valuations on $K$ are called \textit{Archimedean}. 

Every valuation $v$ induces on $K$ the structure of a metric topological space which may or may not be complete. We say that two valuations are \textit{equivalent} if they define the same topology and we call an equivalence class of absolute values a \textit{place} of $K$. It is an elementary result of topology that every metric space may be embedded in a complete metric space, and this can be done in an essentially unique way. For the field $K$, the resulting complete metric space may be given a field structure. Equivalently, there exists a field $L$ with a valuation $w$ such that $L$ is complete in the topology induced by $w$. The field $K$ is contained in $L$ and the valuations $v$ and $w$ coincide in $K$. Moreover, the completion $L$ of $K$ is unique up to topological isomorphism.

For any non-zero prime ideal $\mathfrak{p}$ of $\mathcal{O}_K$, let $\ord_{\mathfrak{p}}(\mathfrak{a})$ denote the exact power to which $\mathfrak{p}$ divides the ideal $\mathfrak{a}$. For fractional ideals $\mathfrak{a}$ this number may be negative. For $\alpha \in K$, we write $\ord_{\mathfrak{p}}(\alpha)$ for $\ord_{\mathfrak{p}}\left((\alpha)\mathcal{O}_K\right)$. Every prime ideal defines a discrete non-Archimedean valuation on $K$ via
\[v(x):= \left(\frac{1}{N_{K/\mathbb{Q}}(\mathfrak{p})}\right)^{\ord_{\mathfrak{p}}(x)}.\]
Furthermore, every embedding of $K$ into the complex field defines an Archimedean valuation. Conversely, every discrete valuation on $K$ arises in this way by a prime ideal of $\mathcal{O}_K$, while every Archimedean valuation of $K$ is equivalent to $|\sigma(x)|$, where $\sigma$ is an embedding of $K$ into $\mathbb{C}$. Valuations defined by different prime ideals are non-equivalent, and two valuations defined by different embeddings of $K$ into $\mathbb{C}$ are equivalent if and only if those embeddings are complex conjugated. The topology induced in $K$ by a prime ideal $\mathfrak{p}$ of $\mathcal{O}_K$ is called the \textit{$\mathfrak{p}$-adic topology}. The completion of $K$ under this valuation is denoted by $K_{\mathfrak{p}}$ or $K_v$ and called the \textit{$\mathfrak{p}$-adic field}. Let $V$ be the set of all valuations of an algebraic number field $K$. Then for every non-zero element $\alpha \in K$ we have 
\[\prod_{v \in V} v(\alpha) = 1.\]

In the ring of integers of $\mathbb{Q}$, the prime ideals are generated by the rational primes $p$, and the resulting topology in the field $\mathbb{Q}$ is called the \textit{$p$-adic topology}. The completion of $\mathbb{Q}$ under this valuation is denoted by $\mathbb{Q}_p$. If $v(x)$ is a non-trivial valuation of $\mathbb{Q}$, then either $v(x)$ is equivalent to the ordinary absolute value $|x|$, or it is equivalent to one of the $p$-adic valuations induced by rational primes. Analogous to $\ord_{\mathfrak{p}}$, for any prime $p$ we define the $p$-adic order of $x \in \mathbb{Q}$ as the largest exponent of $p$ dividing $x$. Then, the $p$-adic valuation $v$ is defined as
\[v(x) = p^{-\ord_p(x)}.\]
If $K_{\mathfrak{p}}$ is a $\mathfrak{p}$-adic field, it is necessarily a finite extension of a certain $\mathbb{Q}_p$. 

Consider now $K/\mathbb{Q}$ where $n = [K:\mathbb{Q}]$ and let $g(t)$ denote the minimal polynomial of $K$ over $\mathbb{Q}$. Suppose $p$ is a rational prime and let $g(t) = g_1(t) \cdots g_m(t)$ be the decomposition of $g(t)$ into irreducible polynomials $g_i(t) \in \mathbb{Q}_p[t]$ of degree $n_i = \deg g_i(t)$. The prime ideals in $K$ dividing $p$ are in one-to-one correspondence with $g_1(t), \dots, g_m(t)$. More precisely, we have in $K$ the following decomposition of $(p)\mathcal{O}_K$
\[(p)\mathcal{O}_K = \mathfrak{p}_1^{e(\mathfrak{p}_1|p)} \cdots \mathfrak{p}_m^{e(\mathfrak{p}_m|p)},\]
with $\mathfrak{p}_1, \dots, \mathfrak{p}_m$ distinct prime ideals and ramification indices $e(\mathfrak{p}_1 | p), \dots, e(\mathfrak{p}_m | p) \in \mathbb{N}$. For $i = 1, \dots, m$ the inertial degree of $\mathfrak{p}_i$ is denoted by $f(\mathfrak{p}_i|p)$. Then $n_i = e(\mathfrak{p}_i | p)f(\mathfrak{p}_i | p)$ and $K_{\mathfrak{p}_i} \simeq \mathbb{Q}_p(\theta_i)$, where $g(\theta_i) = 0$. 

By $\overline{\mathbb{Q}_p}$ we denote the algebraic closure of $\mathbb{Q}_p$. There are $n$ embeddings of $K$ into $\overline{\mathbb{Q}_p}$, and each one fixes $\mathbb{Q}$ and maps $\theta$ to a root of $g$ in $\overline{\mathbb{Q}_p}$. Let $\theta_i^{(1)}, \dots, \theta_i^{(n_i)}$ denote the roots of $g_i(t)$ in $\overline{\mathbb{Q}_p}$. For $i = 1, \dots, m$ and $j = 1, \dots, n_i$, let $\sigma_{ij}$ be the embedding of $K$ into $\mathbb{Q}_p(\theta_i^{(j)})$ defined by $\theta \mapsto \theta_i^{(j)}$. The $m$ classes of conjugate embeddings are $\{\sigma_{i1}, \dots, \sigma_{in_i}\}$ for $i = 1, \dots, m$. Note that $\sigma_{ij}$ coincides with the embedding $K \hookrightarrow K_{\mathfrak{p}_i} \simeq \mathbb{Q}_p(\theta_i) \simeq \mathbb{Q}_p(\theta_i^{(j)})$. 

For any finite extension $L$ of $\mathbb{Q}_p$, the $p$-adic valuation $v$ of $\mathbb{Q}_p$ extends uniquely to $L$ as 
\[v(x) = |N_{L/\mathbb{Q}_p}(x)|^{1/[L:\mathbb{Q}_p]}.\]
Here, we define the $p$-adic order of $x \in L$ by
\[\ord_p(x) = \frac{1}{[L:\mathbb{Q}_p]}\ord_p(N_{L/\mathbb{Q}_p}(x)).\]
This definition is independent of the field $L$ containing $x$. So, since each element of $\overline{\mathbb{Q}_p}$ is by definition contained in some finite extension of $\mathbb{Q}_p$, this definition can be used to define the $p$-adic valuation $v$ of any $x \in \overline{\mathbb{Q}_p}$. Every finite extension of $\mathbb{Q}_p$ is complete with respect to $v$, but $\overline{\mathbb{Q}_p}$ is not. The completion of $\overline{\mathbb{Q}_p}$ with respect to $v$ is denoted by $\mathbb{C}_p$. 

The $m$ extensions of the $p$-adic valuation on $\mathbb{Q}$ to $K$ are just multiples of the $\mathfrak{p}_i$-adic valuation on $K$:
\[\ord_p(x) = \frac{1}{e_i}\ord_{\mathfrak{p}_i}(x) \quad \text{ for } i = 1, \dots, m.\]
We also view these extensions as arising from various embeddings of $K$ into $\overline{\mathbb{Q}_p}$. Indeed, the extension to $\mathbb{Q}_p(\theta_i^{(j)})$ of the $p$-adic valuation on $\mathbb{Q}_p$ induces a $p$-adic valuation on $K$ via the embedding $\sigma_{ij}$ as 
\[v(x) = |N_{K_{\mathfrak{p}_i}/\mathbb{Q}_p}(\sigma_{ij}(x))|^{1/n_i}.\]
Here, as before, $n_i = \deg g_i(t) = [K_{\mathfrak{p}_i} : \mathbb{Q}_p]$. Furthermore, 
\[\ord_p(x) = \ord_p(\sigma_{ij}(x)),\]
and we have
\[\ord_p(\sigma_{ij}(x)) =  \frac{1}{e_i}\ord_{\mathfrak{p}_i}(x) \quad \text{ for } i = 1, \dots, m,\ j = 1, \dots, n_i.\]

Of course, in the special case $x \in \mathbb{Q}_p$, we can write
\[x = \sum_{i=k}^{\infty} u_ip^i\]
where $k = \ord_p(x)$ and the $p$-adic digits $u_i$ are in $\{0, \dots, p-1\}$ with $u_k \neq 0$. If $\ord_p(x) \geq 0$ then $x$ is called a $p$-adic \textit{integer}. The set of $p$-adic integers is denoted $\mathbb{Z}_p$. A $p$-adic \textit{unit} is an $x \in \mathbb{Q}_p$ with $\ord_p(x) = 0$. For any $p$-adic integer $\alpha$ and $\mu \in \mathbb{N}_0$ there exists a unique rational integer $x^{(\mu)} = \sum_{i=0}^{\mu-1}u_ip^i$ such that 
\[\ord_p(x-x^{(\mu)}) \geq \mu, \quad \text{ and } \quad 0 \leq x^{(\mu)} \leq p^{\mu} - 1.\]
For $\ord_p(x) \geq k$ we also write $x \equiv 0 \mod{p^k}$.

%--------------------------------------------------------------------------------------------------------------------------------------------%
%--------------------------------------------------------------------------------------------------------------------------------------------%

\section{$p$-adic logarithms}
\label{sec:pAdicLogarithms}

We have seen how to extend $p$-adic valuations to algebraic extensions of $\mathbb{Q}$. For any $z \in \mathbb{C}_p$ with $\ord_p(z-1) > 0$, we can also define the $p$-adic logarithm of $z$ by
\[\log_p(z) = -\sum_{i=1}^{\infty} \frac{(1-z)^i}{i}.\]
By the $n^{\text{th}}$ term test, this series converges precisely in the region where ${\ord_p(z-1) > 0}$. Three important properties of the $p$-adic logarithm are
\begin{enumerate}
\item $\log_p(xy) = \log_p(x) + \log_p(y)$ whenever $\ord_p(x-1) > 0$ and $\ord_p(y-1) > 0$.
\item $\log_p(z^k) = k \log(p)$ whenever $\ord_p(z-1) > 0$ and $k \in \mathbb{Z}$. 
\item $\ord_p(\log_p(z)) = \ord_p(z-1)$ whenever $\ord_p(z-1) > 1/(p-1)$.
\end{enumerate}
Proofs of the first and last property can be found in \cite{Has2} (pp. 264-265). The second property follows from the first.

We will use the following lemma to extend the definition of the $p$-adic logarithm to all $p$-adic units in $\overline{\mathbb{Q}_p}$. 
\begin{lemma} \label{lem: pAdicLogarithms}
Let $z$ be a $p$-adic unit belonging to a finite extensions $L$ of $\mathbb{Q}_p$. Let $e$ and $f$ be the ramification index and inertial degree of $L$. 
\begin{enumerate}[(a)]
\item There is a positive integer $r$ such that $\ord_p(z^r-1) >0$.
\item If $r$ is the smallest positive integer having $\ord_p(z^r-1) >0$, then $r$ divides $p^f-1$, and an integer $q$ satisfies $\ord_p(z^q-1) >0$ if and only if it is a multiple of $r$.
\item If $r$ is a nonzero integer with $\ord_p(z^r-1) >0$, and if $k$ is an integer with $p^k(p-1) > e$, then
\[\ord_p(z^{rp^k}-1) >\frac{1}{p-1}.\]
\end{enumerate}
\end{lemma}

For $z$ a $p$-adic unit in $\overline{\mathbb{Q}_p}$ we define
\[\log_p{z} = \frac{1}{q}\log_p{z^q},\]
where $q$ is an arbitrary non-zero integer such that $\ord_p(z^q-1) >0$. To see that this definition is independent of $q$, let $r$ be the smallest positive integer with $\ord_p(z^r-1) >0$, note that $q/r$ is an integer, and use the second property of $p$-adic logarithms above to write
\[\frac{1}{q}\log_p{z^q} = \frac{1}{r(q/r)}\log_p{z^{r(q/r)}} = \frac{1}{r}\log_p{z^r}.\]
Choosing $q$ such that $\ord_p(z^q-1) > 1/(p-1)$ helps to speed up and control the convergence of the series defining $\log_p$ (cf. \cite{Sm} (pp. 28-30) and \cite{Coh2} (pp. 263-265)).

It is straightforward to see that Properties 1 and 2 above extend to the case where $x,y,z$ are $p$-adic units. Combining this with Property 3, we obtain
\begin{lemma}\label{lem:pAdicLogarithms2}
Let $z_1, \dots, z_m \in \overline{\mathbb{Q}_p}$ be $p$-adic units and let $b_1, \dots, b_m \in \mathbb{Z}$. If 
\[\ord_p(z_1^{b_1}\cdots z_m^{b_m} - 1) > \frac{1}{p-1}\]
then 
\[\ord_p(b_1\log_p{z_1} + \cdots + b_m \log_p{z_m}) = \ord_p(z_1^{b_1}\cdots z_m^{b_m} - 1).\]
\end{lemma}

%--------------------------------------------------------------------------------------------------------------------------------------------%
%--------------------------------------------------------------------------------------------------------------------------------------------%

\section{The Weil height}
\label{sec:WeilHeight}

Let $K$ be a number field and at each place $v$ of $K$, let $K_v$ denote the completion of $K$ at $v$. Then
\[\sum_{v|p} [K_v:\mathbb{Q}_v] = [K:\mathbb{Q}]\]
for all places $p$ of $\mathbb{Q}$. We will use two absolute values $| \cdot |_v$ and $\| \cdot \|_v$ on $K$ which we now define. If $v|\infty$, then $\| \cdot \|_v$ restricted to $\mathbb{Q}$ is the usual Archimedean absolute value; if $v|p$ for a rational prime $p$, then $\| \cdot \|_v$ restricted to $\mathbb{Q}$ is the usual $p$-adic valuation. We then set
\[ | \cdot |_v = \| \cdot \|_v^{[K_v:\mathbb{Q}_v]/[K:\mathbb{Q}]}.\]
Let $x \in K$ and let $\log^+(\cdot)$ denote the real-valued function $\max\{\log(\cdot),0\}$ on $\mathbb{R}_{\geq 0}$. We define the \textit{logarithmic Weil height} $h(x)$ by 
\[h(x) = \frac{1}{[K:\mathbb{Q}]}\sum_v \log^+|x|_v,\]
where the sum is take over all places $v$ of $K$. If $x$ is an algebraic unit, then $|x|_v = 1$ for all non-Archimedean places $v$, and therefore $h(x)$ can be taken over the Archimedean places only. 
In particular, if $x \in \mathbb{Q}$, then with $x = p/q$ for $p,q \in \mathbb{Z}$ with $\gcd(p,q) = 1$, we have $h(x) = \log\max\{|p|,|q|\}$, and if $x \in \mathbb{Z}$ then $h(x) = \log|x|$. 

%reference: BoGu

%--------------------------------------------------------------------------------------------------------------------------------------------%
%--------------------------------------------------------------------------------------------------------------------------------------------%

\section{Elliptic curves}
\label{sec:EllipticCurves}

Let $K$ be a field of characteristic $\text{char}(K) \neq 2,3$. An \textit{elliptic curve} $E$ over $K$ is a nonsingular curve of the form 
\begin{equation} \label{eq:EllipticCurve}
E: y^2 + a_1xy + a_3y = x^3 + a_2x^2 + a_4x + a_6
\end{equation}
with $a_i \in K$ having a specified base point, $\mathcal{O}\in E$. An equation of the form (\ref{eq:EllipticCurve}) is called a \textit{Weierstrass equation}. This equation is unique up to a coordinate transformation of the form
\[x = u^2x' + r, \quad\quad y = u^3y' + su^2x' + t, \]
with $r,s,t,u \in K, u\neq 0$. 
Applying several linear changes of variables and writing 
\[b_2 = a_1^2 + 4a_2, \quad b_4 = a_1a_3 + 2a_4, \quad b_6 = a_3^2 + 4a_6,\]
\[b_8 = a_1^2a_6 + 4a_2a_6 - a_1a_3a_4 + a_2a_3^2 - a_4^2,\]
\[ c_4 = b_2^2 - 24b_4, \quad \text{ and } \quad c_6 = -b_2^3 + 36b_2b_4 + 9b_2b_4b_6,\]
$E$ can be written as
\[E: y^2 = x^3 - 27c_4x - 54c_6.\]
Associated to this curve are the quantities 
\[\Delta = -b_2^2b_8 - 8b_4^3 - 27b_6^2 + 9b_2b_4b_6 \quad \text{ and } \quad j = c_4^3/\Delta,\]
where $\Delta$ is called the \textit{discriminant} of the Weierstrass equation and the quantity $j$ is called the \textit{j-invariant} of the elliptic curve. The condition of being nonsingular is equivalent to $\Delta$ being non-zero. Two elliptic curves are isomorphic over $\bar{K}$, the algebraic closure of $K$, if and only if they both have the same $j$-invariant.

When $K = \mathbb{Q}$, the Weierstrass model (\ref{eq:EllipticCurve}) can be chosen so that $\Delta$ has minimal $p$-adic order for each rational prime $p$ and $a_i \in \mathbb{Z}$. Suppose (\ref{eq:EllipticCurve}) is such a global minimal model for an elliptic curve $E$ over $\mathbb{Q}$. Reducing the coefficients modulo a rational prime $p$ yields a (possibly singular) curve over $\mathbb{F}_p$
\begin{equation}
\tilde{E}: y^2 + \tilde{a_1}xy + \tilde{a_3}y = x^3 + \tilde{a_2}x^2 + \tilde{a_4}x + \tilde{a_6},
\end{equation}
where $\tilde{a_i} \in \mathbb{F}_p$. This ``reduced" curve $\tilde{E}/\mathbb{F}_p$ is called the \textit{reduction of $E$ modulo} $p$. It is nonsingular provided that $\Delta \not \equiv 0 \mod{p}$, in which case it is an elliptic curve defined over $\mathbb{F}_p$. The curve $E$ is said to have \textit{good reduction} modulo $p$ if $\tilde{E}/\mathbb{F}_p$ is nonsingular, otherwise, we say $E$ has \textit{bad reduction} modulo $p$. 

The reduction type of $E$ at a rational prime $p$ is measured by the \textit{conductor}, 
\[N = \prod_{p}p^{f_p}\]
where the product runs over all primes $p$ and $f_p = 0$ for all but finitely many primes. In particular, $f_p \neq 0$ if $p$ does not divide $\Delta$. Equivalently, $E$ has bad reduction at $p$ if and only if $p \mid N$. Suppose $E$ has bad reduction at $p$ so that $f_p \neq 0$. The reduction type of $E$ at $p$ is said to be \textit{multiplicative} ($E$ has a node over $\mathbb{F}_p$) or \textit{additive} ($E$ has a cusp over $\mathbb{R}_p$) depending on whether $f_p = 1$ or $f_p \geq 2$, respectively. The $f_p$, hence the conductor, are invariant under isogeny. 

%--------------------------------------------------------------------------------------------------------------------------------------------%
%--------------------------------------------------------------------------------------------------------------------------------------------%

\section{Cubic forms}
\label{sec:CubicForms}

Let $a,b,c$ and $d$ be integers and consider the binary cubic form
\[F(x,y) = ax^3 + bx^2y + cxy^2 + dy^3.\]
Two such forms $F_1$ and $F_2$ are called \textit{equivalent} if they are equivalent under the $GL_{2}(\mathbb{Z})$-action. That is, if there exist integers $a_1, a_2, a_3$, and $a_4$ such that 
\[F_1(a_1x + a_2y, a_3x + a_4y) = F_2(x,y)\]
for all $x,y$, where $a_1a_4 - a_2a_3 = \pm 1$. In this case, we write $F_1 \sim F_2$. The \textit{discriminant} $D_F$ of such a form is given by 
\[D_F = -27a^2d^2 + b^2c^2 + 18abcd - 4ac^3 - 4b^3d = a^4 \prod_{i < j} (\alpha_i - \alpha_j)^2,\]
where $\alpha_1, \alpha_2$ and $\alpha_3$ are the roots of the polynomial $F(x,1)$. We observe that if $F_1 \sim F_2$, then $D_{F_1} = D_{F_2}$. 

Associated to $F$ is the Hessian $H_F(x,y)$, given by
\begin{align*}
H_F(x,y) & = -\frac{1}{4}\left( \frac{\partial^2F}{\partial x^2} \frac{\partial^2F}{\partial y^2} - \left(\frac{\partial^2F}{\partial x \partial y}\right)^2\right)\\
& = (b^2 - 3ac)x^2 + (bc - 9ad)xy + (c^2 - 3bd)y^2,
\end{align*}
and the Jacobian determinant of $F$ and $H_F$, a cubic form $G_F(x,y)$ defined by
\begin{align*}
G_F(x,y) &= \frac{\partial F}{\partial x} \frac{\partial H_F}{\partial y} - \frac{\partial F}{\partial y} \frac{\partial H_F}{\partial x} \\
& =  (-27a^2d + 9abc -2b^3)x^3 + (-3b^2c - 27abd + 18ac^2)x^2y +  \\
& \quad \quad + (3bc^2 - 18b^2d + 27acd)xy^2 + (-9bcd + 2c^3 + 27ad^2)y^3.
\end{align*}

%--------------------------------------------------------------------------------------------------------------------------------------------%
%--------------------------------------------------------------------------------------------------------------------------------------------%

\section{Lattices}
\label{sec:Lattices}

An $n$-dimensional lattice is a discrete subgroup of $\mathbb{R}^n$ of the form
\[\Gamma = \left\{ \sum_{i=1}^n x_i \mathbf{b}_i \ : \ x_i \in \mathbb{Z} \right\},\]
where $\mathbf{b}_1, \dots, \mathbf{b_n}$ are vectors forming a basis for $\mathbb{R}^n$. We say that the vectors $\mathbf{b}_1, \dots, \mathbf{b_n}$ form a \textit{basis} for $\Gamma$, or that they generate $\Gamma$. Let $B$ denote the matrix whose columns are the vectors $\mathbf{b}_1, \dots, \mathbf{b_n}$. Any lattice element $\mathbf{v}$ may be expressed as $\mathbf{v} = B\mathbf{x}$ for some $\mathbf{x} \in \mathbb{Z}^n$. We call $\mathbf{v}$ the \textit{embedded vector} and $\mathbf{x}$ the \textit{coordinate vector}.

A \textit{bilinear form} on a lattice $\Gamma$ is a function $\Phi: \Gamma \times \Gamma \to \mathbb{Z}$ satisfying
\begin{enumerate}
\item $\Phi(\mathbf{u}, \mathbf{v}+\mathbf{w}) = \Phi(\mathbf{u},\mathbf{v}) + \Phi(\mathbf{u},\mathbf{w})$
\item $\Phi(\mathbf{u}+\mathbf{v}, \mathbf{w}) = \Phi(\mathbf{u},\mathbf{w}) + \Phi(\mathbf{v},\mathbf{w})$
\item $\Phi(a\mathbf{u}, \mathbf{w}) = a\Phi(\mathbf{u},\mathbf{w})$
\item $\Phi(\mathbf{u}, a\mathbf{w}) = a\Phi(\mathbf{u},\mathbf{w})$
\end{enumerate}
for all $\mathbf{u}, \mathbf{v}$, and $\mathbf{w}$ in $\Gamma$ and any $a \in \mathbb{Z}$. 

Given a basis, we can define a specific bilinear form on our lattice $\Gamma$ as part of its structure. This form describes a kind of distance between elements $\mathbf{u}$ and $\mathbf{v}$ and we say the lattice is \textit{defined} by $\Phi$. Associated to this bilinear form is a quadratic form $Q: \Gamma \to \mathbb{Z}$ defined by $Q(\mathbf{v}) = \Phi(\mathbf{v}, \mathbf{v})$. A lattice is called \textit{positive definite} if its quadratic form is positive definite. 

The bilinear forms (and their associated quadratic forms) that we will be using come from the usual inner product on vectors in $\mathbb{R}^n$. This is simply the dot product $\Phi(\mathbf{u},\mathbf{v}) = \mathbf{u} \cdot \mathbf{v}$ for embedded vectors, $\mathbf{u},\mathbf{v}$. For the coordinate vectors $\mathbf{x},\mathbf{y}$ associated to these vectors, this translates to multiplication with the basis matrix. Precisely, if $\mathbf{u} = B\mathbf{x}$ and $\mathbf{v} = B\mathbf{y}$, we have $\Phi(\mathbf{u},\mathbf{v}) = \mathbf{x}^TB^TB\mathbf{y}$. 

If $\mathbf{v} = B\mathbf{x}$, the \textit{norm} of the vector $\mathbf{v} \in \Gamma$ is defined to be the inner product $\Phi(\mathbf{v},\mathbf{v})$. In terms of the corresponding coordinate vector $\mathbf{x}$, this is
\[\mathbf{v}^T\mathbf{v} = \mathbf{x}^TB^TB\mathbf{x}.\]
Equivalently, we write $\mathbf{x}^TA\mathbf{x}$ where $A = B^TB$ is the Gram matrix of $\Gamma$ with basis $B$ and bilinear form $\Phi$. The entries of the matrix $A$ are $a_{ij} = \Phi(\mathbf{b}_i,\mathbf{b}_j)$.

Two basis matrices $B_1$ and $B_2$ define the same lattice $\Gamma$ if and only if there is a unimodular matrix $U$ such that $B_1U = B_2$. The bilinear form on $\Gamma$ can be written with respect to either embedded or coordinate vectors. Using another basis to express the lattice elements is possible, and sometimes preferable. However, the Gram matrix is specific to the bilinear form on the lattice and should not change when operating on embedded vectors. If it is operating on coordinate vectors, the change of basis must be accounted for. 

%--------------------------------------------------------------------------------------------------------------------------------------------%

\endinput

Any text after an \endinput is ignored.
You could put scraps here or things in progress.
%% The following is a directive for TeXShop to indicate the main file
%%!TEX root = diss.tex

\chapter{Algorithms for Thue-Mahler Equations}
\label{ch:AlgorithmsForTM}

In this chapter, we give some of primary algorithms needed to solve an arbitrary Thue-Mahler equation. The methods presented here follow somewhat \cite{Ham} and \cite{TW3}, with new results and modifications from \cite{GhKaMaSi}. 

%--------------------------------------------------------------------------------------------------------------------------------------------%
%--------------------------------------------------------------------------------------------------------------------------------------------%

\section{First steps}
\label{sec:FirstSteps}

Fix a nonzero integer $c$ and let $S=\{p_1,\dotsc,p_v\}$ be a set of rational primes. Let
\[F(X,Y) = c_0 X^n + c_1 X^{n-1}Y + \cdots + c_{n-1}XY^{n-1} + c_nY^n\]
be an irreducible binary form over $\mathbb{Z}$ of degree $n \geq 3$. We want to solve the Thue--Mahler equation
\begin{equation} \label{eq:ThueMahler}
F(X,Y) = c p_1^{Z_1}\cdots p_v^{Z_v}
\end{equation}
for unknowns $X,Y, Z_1, \dots, Z_v$ with $\gcd(X,Y) = 1$ and $Z_i \geq 0$ for $i = 1,\dots, v$. To do so, we first reduce \eqref{eq:ThueMahler} to the special case where $c_0 = 1$ and $\gcd(c,p_i) = 1$ for $i = 1, \dots, v$, loosely following \cite{Ham}. 

As $F$ is irreducible by assumption, at least one of the coefficients $c_0$ and $c_n$ is nonzero. Hence, we may transform the given Thue--Mahler equation to one with $c_0 \neq 0$ by interchanging $X$ and $Y$ and by renaming the coefficients $c_i$ appropriately. In particular, solving \eqref{eq:ThueMahler} is equivalent to solving 
\[ c_0' \overline{X}^n + c_1' \overline{X}^{n-1}\overline{Y} + \cdots + c_{n-1}'\overline{X}\overline{Y}^{n-1} + c_n'\overline{Y}^n = c p_1^{Z_1}\cdots p_v^{Z_v},\]
where $c_i' = c_{n-1}$ for $i = 0, \dots, n$, $\overline{X} = Y$, and $\overline{Y} = X$. 

Denote by $\mathcal{D}$ the set of all positive rational integers $m$ dividing $c_0$ such that ${\ord_p(m)\leq \ord_p(c)}$ for each rational prime $p\notin S$. Equivalently, $\mathcal{D}$ is precisely the set of all possible integers $d$ such that $d = \gcd(c_0,Y)$. To see this, let $q_1, \dots, q_{w}$ denote the distinct prime divisors of $a$ not contained in $S$. Then 
\[c = \prod_{i=1}^w q_i^{b_i}\cdot \prod_{i=1}^v p_i^{\ord_{p_i}(c)}\]
for some integers $b_i >0$. If $(X,Y,Z_1, \dots, Z_v)$ is a solution of the Thue-Mahler equation in question, it follows that
\[F(X,Y) = cp_1^{Z_1}\dots p_v^{Z_v} =  \prod_{i=1}^w q_i^{b_i}\cdot \prod_{i=1}^v p_i^{\ord_{p_i}(c) + Z_i}.\]
Suppose $\gcd(c_0,Y) = d$. Since $d$ divides $F(X,Y)$, it necessarily divides 
\[{\prod_{i=1}^w q_i^{b_i}\cdot \prod_{i=1}^v p_i^{\ord_{p_i}(c) + Z_i}}.\] 
In particular, 
\[d = \prod_{i=1}^w q_i^{s_i}\cdot \prod_{i=1}^v p_i^{t_i}\]
for some non-negative integers $s_1, \dots, s_w, t_1, \dots, t_v$ such that 
\[s_i \leq \min\{\ord_{q_i}(c), \ord_{q_i}(c_0)\} \quad \text{ and } \quad 
	t_i \leq \min\{\ord_{p_i}(c) + Z_i, \ord_{p_i}(c_0)\}.\] 
From here, it is easy to see that ${\ord_p(d)\leq \ord_p(c)}$ for each rational prime $p\notin S$ so that $d \in \mathcal{D}$. 

Conversely, suppose $d \in \mathcal{D}$ so that $\ord_{p}(d) \leq \ord_{p}(c)$ for all $p \notin S$. That is, the right-hand side of 
\[\ord_{p}(d) \leq \ord_{p}(c) = 
\ord_p\left(\prod_{i=1}^w q_i^{b_i}\cdot \prod_{i=1}^v p_i^{\ord_{p_i}(c)}\right)\]
is non-trivial only at the primes $\{q_1, \dots, q_w\}$. In particular, 
\[d = \prod_{i=1}^w q_i^{s_i}\cdot \prod_{i=1}^v p_i^{t_i}\]
for non-negative integers $s_1, \dots, s_w, t_1, \dots, t_v$ such that 
\[s_i \leq \min\{\ord_{q_i}(c), \ord_{q_i}(c_0)\} \quad \text{ and } \quad 
	t_i \leq \ord_{p_i}(c_0).\] 
It follows that $d = \gcd(c_0,Y)$ for some solution $(X,Y,Z_1, \dots, Z_v)$ of equation~\eqref{eq:ThueMahler}. 

For any $d\in \mathcal{D}$, we define the rational numbers 
\[u_d = c_0^{n-1}/d^n \quad \textnormal{and}\quad c_d = \sgn(u_dc)\prod_{p\notin S} p^{\ord_p(u_dc)}.\]
On using that $d\in \mathcal{D}$, we see that the rational number $c_d$ is in fact an integer coprime to $S$. 

Suppose $(X,Y,Z_1, \dots, Z_v)$ is a solution of \eqref{eq:ThueMahler} with ${\gcd(X,Y) = 1}$ and $d = \gcd(c_0,Y)$. Define the homogeneous polynomial $f(x,y) \in \mathbb{Z}[x,y]$ of degree $n$ by
\[f(x,y) = x^n + C_1 x^{n-1}y + \dots + C_{n-1}xy^{n-1} + C_ny^n,\]
where
\[x=\tfrac{c_0X}{d},\quad y=\tfrac{Y}{d} \quad \text{ and } \quad C_i = c_ic_0^{i-1} \quad \text{ for } i = 1, \dots, n.\]
Since $\gcd(X,Y) = 1$, the numbers $x$ and $y$ are also coprime integers by definition of $d$. We observe that 
\[f(x,y) = u_dF(X,Y) = u_dc \prod_{i = 1}^v p_i^{Z_i} = c_d\prod_{p \in S}p^{Z_i + \ord_p(u_dc)}.\]
Setting $z_i = Z_i + \ord_p(u_dc)$ for all $i \in \{1, \dots, v\}$, we obtain
\begin{equation} \label{eq:ThueMahler2}
f(x,y) = x^n + C_1 x^{n-1}y + \dots + C_{n-1}xy^{n-1} + C_ny^n = c_d p_1^{z_1}\cdots p_v^{z_v}, 
\end{equation}
where $\gcd(x,y) = 1$ and $\gcd(c_d,p_i) = 1$ for all $i = 1, \dots, v$. 

Since there are only finitely many choices for $d = \gcd(c_0, Y)$, there are only finitely many choices for $\{c_d,u_d,d\}$. Then, solving \eqref{eq:ThueMahler} is equivalent to solving the finitely many Thue-Mahler equations \eqref{eq:ThueMahler2} for each choice of $\{c_d,u_d,d\}$.  For each such choice, the solution $\{x,y,z_1, \dots, z_v\}$ is related to $\{X,Y, Z_1, \dots, Z_v\}$ via
\[X = \frac{dx}{c_0},\quad Y=dy \quad \text{ and } \quad Z_i = z_i - \ord_p(u_dc).\]

Lastly, we observe that the polynomial $f(x,y)$ of \eqref{eq:ThueMahler2} remains the same for any choice of $\{c_d,u_d,d\}$. Thus, to solve the family of equations \eqref{eq:ThueMahler2}, we need only to enumerate over every possible $c_d$. Now, if $\mathcal{C}$ denotes the set of all $\{c_d,u_d,d\}$ and $d_1, d_2 \in \mathcal{D}$, we may have $\{c_{d_1},u_{d_1}, d_1\}, \{c_{d_2},u_{d_2}, d_2\} \in \mathcal{C}$ where $c_{d_1} = c_{d_2}$. That is, $d_1, d_2$ may yield the same value of $c_d$, reiterating that we need only solve \eqref{eq:ThueMahler2} for each distinct $c_d$. 

%--------------------------------------------------------------------------------------------------------------------------------------------%
%--------------------------------------------------------------------------------------------------------------------------------------------%

\section{The relevant algebraic number field}
\label{sec:RelevantAlgNumField}

For the remainder of this chapter, we consider the Thue-Mahler equation
\begin{equation} \label{eq:ThueMahler3}
f(x,y) = x^n + C_1 x^{n-1}y + \dots + C_{n-1}xy^{n-1} + C_ny^n = c p_1^{z_1} \cdots p_v^{z_v}
\end{equation}
where $\gcd(x,y) = 1$ and $\gcd(c,p_i) = 1$ for $i = 1, \dots, p_v$.

Following \cite{TW3}, put
\[g(t) = f(t,1) = t^n + C_1 t^{n-1} + \dots + C_{n-1}t + C_n\]
and note that $g(t)$ is irreducible in $\mathbb{Z}[t]$. Let $K = \mathbb{Q}(\theta)$ with $g(\theta) = 0$. Now \eqref{eq:ThueMahler3} is equivalent to the norm equation
\begin{equation} \label{eq:normTM}
N_{K/\mathbb{Q}}(x-y\theta) = cp_1^{z_1}\dots p_v^{z_v}.
\end{equation}

Let $p_i$ be any rational prime and let 
\[(p_i)\mathcal{O}_K = \prod_{j = 1}^{m_i} \mathfrak{p}_{ij}^{e(\mathfrak{p}_{ij}|p_i)}\]
be the factorization of $p_i$ into prime ideals in the ring of integers $\mathcal{O}_K$ of $K$. Let $f(\mathfrak{p}_{ij}|p_i)$ be the inertial degree of $\mathfrak{p}_{ij}$ over $p_i$. Since $N(\mathfrak{p}_{ij}) = p_i^{f_{ij}}$, \eqref{eq:normTM} leads to finitely many ideal equations of the form
\begin{equation} \label{eq:idealTM}
(x-y\theta)\mathcal{O}_K = \mathfrak{a} \prod_{j = 1}^{m_1} \mathfrak{p}_{1j}^{z_{1j}} \cdots \prod_{j = 1}^{m_v} \mathfrak{p}_{vj}^{z_{vj}}
\end{equation}
where $\mathfrak{a}$ is an ideal of norm $|c|$ and the $z_{ij}$ are unknown integers related to $z_i$ by 
\[\sum_{j = 1}^{m_i} f(\mathfrak{p}_{ij}|p_i)z_{ij} = z_i\]
for $i \in \{1, \dots, v\}$.

Our first task is to cut down the number of variables appearing in \eqref{eq:idealTM}. We will do this by showing that only a few prime ideals can divide $(x-y\theta)\mathcal{O}_K$ to a large power. 

%--------------------------------------------------------------------------------------------------------------------------------------------%
%--------------------------------------------------------------------------------------------------------------------------------------------%

\section{The prime ideal removing lemma}
\label{sec:PIRL}

In this section, we establish some key results that will allow us to cut down the number of prime ideals that can appear to a large power in the factorization of $(x-y\theta)\mathcal{O}_K$. It is of particular importance to note that we do not appeal to the Prime Ideal Removing Lemma of Tzanakis and de Weger (\cite{TW3}) here and instead apply the following results of \cite{GhKaMaSi}. 

Let $p \in \{p_1, \dots, p_v\}$. We will produce the following two finite lists $L_p$ and $M_p$. The list $L_p$ will
consist of certain ideals $\mathfrak{b}$ of $\mathcal{O}_K$ supported at the prime ideals above $p$. The list $M_p$ will consist of certain pairs $(\mathfrak{b},\mathfrak{p})$ where $\mathfrak{b}$ is supported at the prime ideals above $p$ and $\mathfrak{p}$ is a prime ideal lying over $p$ satisfying $e(\mathfrak{p}|p)=f(\mathfrak{p}|p)=1$. These lists will satisfy the following property: if $(x,y,z_1,\dots,z_v)$ is a solution to the Thue-Mahler equation \eqref{eq:ThueMahler3} then
\begin{enumerate}[(i)]
\item either there is some $\mathfrak{b} \in L_p$
such that
\begin{equation} \label{eq:Lp}
\mathfrak{b} \mid (x-y\theta )\mathcal{O}_K, \qquad \text{$(x-y\theta)\mathcal{O}_K/\mathfrak{b}$ is coprime to $(p)\mathcal{O}_K$};
\end{equation}
\item or there is a pair $(\mathfrak{b},\mathfrak{p}) \in M_p$ and a non-negative integer $v_p$ such that
\begin{equation} \label{eq:Mp}
(\mathfrak{b} \mathfrak{p}^{v_p}) \mid (x-y\theta)\mathcal{O}_K, \qquad \text{$(x-y\theta)\mathcal{O}_K/(\mathfrak{b} \mathfrak{p}^{v_p})$ is coprime to $(p)\mathcal{O}_K$}.
\end{equation}
\end{enumerate}

To generate the lists $M_p$, $L_p$ we consider two affine patches, $p \nmid y$ and $p \mid y$. We begin with the following lemmata.

\begin{lemma} \label{lem:AffinePatch1}
Let $(x,y,z_1, \dots, z_v)$ be a solution of \eqref{eq:ThueMahler3} with $p \nmid y$, let $t$ be a positive integer, and suppose $x/y \equiv u \pmod{p^t}$, where ${u \in \{0,1,2,\dotsc,p^{t}-1\}}$. If $\mathfrak{q}$ is a prime ideal of $\mathcal{O}_K$ lying over $p$, then
\[\ord_{\mathfrak{q}}(x-y\theta)\ge \min\{\ord_{\mathfrak{q}}(u-\theta), t \cdot e(\mathfrak{q}|p)\}.\]
Moreover, if $\ord_{\mathfrak{q}}(u-\theta) < t \cdot e(\mathfrak{q}|p)$, then
\[\ord_\mathfrak{q}(x-y\theta) = \ord_{\mathfrak{q}}(u-\theta).\]
\end{lemma}

\begin{lemma} \label{lem:AffinePatch2}
Let $(x,y,z_1, \dots, z_v)$ be a solution of \eqref{eq:ThueMahler3} with $p \mid y$ (and thus $p \nmid x$), let $t$ be a positive integer, and suppose $y/x \equiv u \pmod{p^t}$, where $u \in \{0,1,2,\dotsc,p^{t}-1\}$. If $\mathfrak{q}$ is a prime ideal of $\mathcal{O}_K$ lying over $p$, then
\[\ord_{\mathfrak{q}}(x-y\theta)\ge \min\{\ord_{\mathfrak{q}}(1-\theta u), t \cdot e(\mathfrak{q}|p)\}.\]
Moreover, if $\ord_{\mathfrak{q}}(1-\theta u) < t \cdot e(\mathfrak{q}|p)$, then
\[\ord_\mathfrak{q}(x-y\theta) = \ord_{\mathfrak{q}}(1 - \theta u).\]
\end{lemma}

\begin{proof}[Proof of Lemmas~\ref{lem:AffinePatch1} and \ref{lem:AffinePatch2}]
Suppose $p \nmid y$. Thus $\ord_{\mathfrak{q}}(y) = 0$ and hence 
\[\ord_{\mathfrak{q}}(x-y\theta) = \ord_{\mathfrak{q}}(x/y - \theta).\]
Since $x/y-\theta = u - \theta + x/y - u,$ we have
\[\begin{array}{ll}
\ord_\mathfrak{q}(x/y-\theta)	& = \ord_{\mathfrak{q}}(u - \theta + x/y - u) \\
						& \geq \min\{\ord_{\mathfrak{q}}(u - \theta), \ord_{\mathfrak{q}}(x/y - u)\}. 
\end{array}\]
By assumption, 
\[\ord_{\mathfrak{q}}(x/y-u) \geq \ord_{\mathfrak{q}}(p^t) = t \cdot e(\mathfrak{q}|p),\]
 %Thus $\ord_\fq(x-\theta)=\ord_\fq(u-\theta)$,
completing the proof of Lemma~\ref{lem:AffinePatch1}. The proof of Lemma~\ref{lem:AffinePatch2} is similar. 
\end{proof}

The following algorithm computes the lists $L_p$ and $M_p$ that come from the first patch $p \nmid y$. We denote these respectively by $\mathcal{L}_p$ and $\mathcal{M}_p$. 

\begin{algorithm} \label{alg:AffinePatch1}
To compute
$\mathcal{L}_p$ and $\mathcal{M}_p$:

\begin{enumerate}[Step (1)]
\item Let 
\[\mathcal{L}_p \leftarrow \emptyset, \qquad \mathcal{M}_p \leftarrow \emptyset,\]
\[ t \leftarrow 1, \quad \mathcal{U} \leftarrow \{w : w \in \{0,1,\dots,p-1\} \}.\]
\item Let
\[\mathcal{U}^\prime \leftarrow \emptyset.\]
Loop through the elements $u \in \mathcal{U}$. Let 
\[\mathcal{P}_u= \{\mathfrak{q} \text{ lying above } p \ : \ \ord_{\mathfrak{q}}(u-\theta) \geq t \cdot e(\mathfrak{q}|p)\}\]
and
\[ \mathfrak{b}_u 	= \prod_{\mathfrak{q} \mid p} \mathfrak{q}^{\min\{\ord_\mathfrak{q}(u-\theta), t \cdot e(\mathfrak{q}|p)\}} 
				= (u-\theta) \mathcal{O}_K+p^t \mathcal{O}_K.\]
\begin{enumerate}[(i)]
\item If $\mathcal{P}_u = \emptyset$ then
\[\mathcal{L}_p \leftarrow \mathcal{L}_p \cup \{\mathfrak{b}_u\}.\]
\item Else if $\mathcal{P}_u = \{\mathfrak{p}\}$ with $e(\mathfrak{p}|p)=f(\mathfrak{p}|p)=1$ and there is at least one $\mathbb{Z}_p$-root $\alpha$ of $g(t)$ satisfying $\alpha \equiv u \pmod{p^t}$, then
\[\mathcal{M}_p \leftarrow \mathcal{M}_p \cup \{ (\mathfrak{b}_u,\mathfrak{p})\}.\]
\item Else 
\[\mathcal{U}^\prime \leftarrow \mathcal{U} \cup \{ u+p^{t}w : w \in \{0,\dots,p-1\} \}.\]
\end{enumerate}

\item If $\mathcal{U}^\prime \ne \emptyset$ then let
\[t \leftarrow t+1, \qquad \mathcal{U} \leftarrow \mathcal{U}^{\prime},\]
and return to Step (2). Else output $\mathcal{L}_p$, $\mathcal{M}_p$.
\end{enumerate}
\end{algorithm}

\begin{lemma}
Algorithm~\ref{alg:AffinePatch1} terminates.
\end{lemma}

\begin{proof}
Suppose otherwise. Write $t_0=1$ and $t_i=t_0+i$ for $i=1,2,3,\dots$. Then there is an infinite sequence of congruence classes $u_i \mod{p^{t_i}}$ such that ${u_{i+1} \equiv u_i \mod{p^{t_i}}}$, and such that the $u_i$ fail the hypotheses of both (i) and (ii). This means that $\mathcal{P}_{u_i}$ is non-empty for every $i \in \mathbb{N}_{>0}$. By the pigeon-hole principle, some prime ideal $\mathfrak{p}$ of $\mathcal{O}_K$ appears in infinitely many of the $\mathcal{P}_{u_i}$. Thus ${\ord_{\mathfrak{p}}(u_i-\theta) \ge t_i\cdot e(\mathfrak{p}|p)}$ infinitely often. However, the sequence $\{u_i\}_{i=1}^{\infty}$ converges to some $\alpha \in \mathbb{Z}_p$ so that $\alpha=\theta$ in $K_\mathfrak{p}$. This forces $e(\mathfrak{p}|p)=f(\mathfrak{p}|p)=1$ and $\alpha$ to be a $\mathbb{Z}_p$-root of $g(t)$. In this case, $\mathfrak{p}$ corresponds to the factor $(t-\alpha)$ in the $p$-adic factorisation of $g(t)$. There can be at most one such $\mathfrak{p}$, forcing $\mathcal{P}_{u_i}=\{\mathfrak{p}\}$ for all $i$. In particular, the hypothesis of (ii) are satisfied and we reach a contradiction.
\end{proof}

\begin{lemma}\label{lem:AffinePatch1Check}
Let $p \in \{p_1, \dots, p_v\}$ and let $\mathcal{L}_p$, $\mathcal{M}_p$ be as given by Algorithm~\ref{alg:AffinePatch1}. Let $(x,y,z_1,\dots, z_v)$ be a solution to \eqref{eq:ThueMahler3}. Then
\begin{itemize}
\item either there is some $\mathfrak{b} \in \mathcal{L}_p$ such that \eqref{eq:Lp} is satisfied; 
\item or there is some $(\mathfrak{b},\mathfrak{p}) \in \mathcal{M}_p$ with $e(\mathfrak{p}|p)=f(\mathfrak{p}|p)=1$ and integer $v_p \geq 0$ such that \eqref{eq:Mp} is satisfied.
\end{itemize}
\end{lemma}

\begin{proof}
Let 
\[t_0 = 1 \quad \text{ and } \quad \mathcal{U}_0=\{w \; :\;  w \in \{0,1,\dots,p-1\}\}\]
be the initial values for $t$ and $\mathcal{U}$ in the algorithm. Then $x/y \equiv u_0 \pmod{p^{t_0}}$ for some $u_0 \in \mathcal{U}_0$. Write $\mathcal{U}_i$ for the value of $\mathcal{U}$ after $i$ iterations of the algorithm  and let $t_i=t_0+i$. As the algorithm terminates, $\mathcal{U}_i = \emptyset$ for some sufficiently large $i$. Hence there is some $i$ such that $x/y \equiv u_i \mod{p^{t_i}}$ where $u_i \in \mathcal{U}_i$, but there is no element in $\mathcal{U}_{i+1}$ congruent to $x/y$ modulo $p^{t_{i+1}}$. In other words, $u_i$ must satisfy the hypotheses of either step (i) or (ii) of algorithm~\ref{alg:AffinePatch1}. Write $u=u_i$ and $t=t_i$ for $x/y \equiv u \mod{p^t}$ and consider the ideal $\mathfrak{b}_u$ generated in this step. By Lemma~\ref{lem:AffinePatch1}, $\mathfrak{b}_u$ divides $(x-y\theta) \mathcal{O}_K$. Furthermore, by definition of $\mathcal{P}_u$, if $\mathfrak{q}$ is a prime ideal of $\mathcal{O}_K$ not contained in $\mathcal{P}_u$, then $(x-y\theta)\mathcal{O}_K/\mathfrak{b}_u$ is not divisible by $\mathfrak{q}$. 

Suppose first that the hypothesis of (i) is satisfied: $\mathcal{P}_u = \emptyset$. The algorithm adds $\mathfrak{b}_u$ to the set $\mathcal{L}_p$, with the above remarks ensuring that \eqref{eq:Lp} is satisfied.

Suppose next that the hypothesis of (ii) is satisfied: $\mathcal{P}_u=\{\mathfrak{p}\}$ where ${e(\mathfrak{p}|p)=f(\mathfrak{p}|p)=1}$ and there is a unique $\mathbb{Z}_p$ root $\alpha$ of $g(t)$ such that $\alpha \equiv u \mod{p^t}$. The algorithm adds $(\mathfrak{b}_u,\mathfrak{p})$ to the set $\mathcal{M}_p$. By the above, $(x-y\theta)\mathcal{O}_K/\mathfrak{b}_u$ is an integral ideal, not divisible by any prime ideal $\mathfrak{q} \neq \mathfrak{p}$ lying over $p$. Thus there is some positive integer $v_p \geq 0$ such that \eqref{eq:Mp} is satisfied, concluding the proof. 
\end{proof}

Having computed the lists arising from the affine patch $p \nmid y$, we initialize $L_p$ and $M_p$ as $\mathcal{L}_p$ and $\mathcal{M}_p$, respectively, and append to these lists the elements from the second patch, $p \mid y$, using the following algorithm.  

\begin{algorithm}\label{alg:AffinePatch2}
To compute $L_p$ and $M_p$.

\begin{enumerate}[Step (1)]
\item Let 
\[ L_p \leftarrow \mathcal{L}_p, \qquad M_p \leftarrow \mathcal{M}_p,\]
where $\mathcal{L}_p$, $\mathcal{M}_p$ are computed by Algorithm~\ref{alg:AffinePatch1}.
\item Let
\[ t \leftarrow 2, \qquad \mathcal{U} \leftarrow \{pw \; : \; w \in \{0,1,\dots,p-1\} \}.\]
\item Let
\[ \mathcal{U}^{\prime} \leftarrow \emptyset.\]
Loop through the elements $u \in \mathcal{U}$. Let 
\[\mathcal{P}_u=\{\mathfrak{q} \text{ lying above } p \ : \ \ord_{\mathfrak{q}}(1-u\theta ) \ge t \cdot e(\mathfrak{q}|p)\},\]
and
\[ \mathfrak{b}_u=\prod_{\mathfrak{q} \mid p} \mathfrak{q}^{\min\{\ord_{\mathfrak{q}}(1-u\theta ), t \cdot e(\mathfrak{q}|p)\}} =(1-u\theta) \mathcal{O}_K+p^t \mathcal{O}_K.\]

\begin{enumerate}[(i)]
\item If $\mathcal{P}_u=\emptyset$  
%$\ord_p(\Norm(u-\theta)) \ge (n-1) c_0$ 
then
\[L_p \leftarrow L_p \cup \{\mathfrak{b}_u\}.\]
%\item[(ii)] Else if $\cP_u=\{\fp\}$ with
%$e(\fp/p)=f(\fp/p)=1$, 
%and
%$\ord_p(\Norm(u-\theta)) \ge (n-1) c_0$,
%and there is at least on $\Z_p$-root $\alpha$ of 
%$f$ satisfying $\alpha \equiv u \pmod{p^t}$,
%then
%\[
%\cM_p \leftarrow \cM_p \cup \{ (\fb_u,\fp)\}.
%\]
\item Else 
\[\mathcal{U}^\prime \leftarrow \mathcal{U}^\prime \cup \{ u+p^{t}w : w \in \{0,\dotsc,p-1\} \}.\]
\end{enumerate}

\item If $\mathcal{U}^\prime \ne \emptyset$ then let
\[t \leftarrow t+1, \qquad \mathcal{U} \leftarrow \mathcal{U}^\prime,\]
and return to Step (3). Else output $L_p$, $M_p$.
\end{enumerate}
%\noindent \textbf{Output:} $\cL_p$, $\cM_p$.
\end{algorithm}

\begin{lemma} 
Algorithm~\ref{alg:AffinePatch2} terminates. 
\end{lemma}

\begin{proof}
Suppose that the algorithm does not terminate. Let $t_0=2$ and $t_i=t_0+i$ for $i \in \mathbb{N}$. Then there is an infinite sequence of congruence classes $\{u_i\}_{i = 0}^{\infty}$ and corresponding sets $\{\mathcal{P}_{u_i}\}_{i=0}^{\infty}$ such that $u_{i+1} \equiv u_i \mod{t_i}$ and $\mathcal{P}_{u_i} \ne \emptyset$ for all $i$. Moreover, $p \mid u_0$. Let $\alpha$ be the limit of $\{u_i\}_{i=0}^{\infty}$ in $\mathbb{Z}_p$. By the pigeon-hole principle, there is some ideal $\mathfrak{q}$ in $\mathcal{O}_K$ above $p$ which appears in infinitely many sets $\mathcal{P}_{u_i}$. It follows that $\ord_{\mathfrak{q}}(1 -u_i \theta) \ge t_i \cdot e(\mathfrak{q}|p)$ and thus $1-\alpha \theta=0$ in $K_{\mathfrak{q}}$. But as $p \mid u_0$, we have $\ord_p(\alpha) \ge 1$, and so $\ord_{\mathfrak{q}}(\theta)<0$. This contradicts the fact that $\theta$ is an algebraic integer. Therefore the algorithm must terminate.
\end{proof}

\begin{lemma}\label{lem:AffinePatch2Check}
Let $p \in \{p_1, \dots, p_v\}$ and let $L_p$, $M_p$ be as given by Algorithm~\ref{alg:AffinePatch2}. Let $(x,y,z_1,\dots, z_v)$ be a solution to \eqref{eq:ThueMahler3}. Then
\begin{itemize}
\item either there is some $\mathfrak{b} \in L_p$ such that \eqref{eq:Lp} is satisfied; 
\item or there is some $(\mathfrak{b},\mathfrak{p}) \in M_p$ with $e(\mathfrak{p}|p)=f(\mathfrak{p}|p)=1$ and integer $v_p \geq 0$ such that \eqref{eq:Mp} is satisfied.
\end{itemize}
\end{lemma}

\begin{proof}
Let $(x,y,z_1,\dots, z_v)$ be a solution to \eqref{eq:ThueMahler3}. In view of Lemma~\ref{lem:AffinePatch1Check} we may suppose $p \mid y$. Then $\ord_{\mathfrak{q}}(x) = 0$ and $\ord_{\mathfrak{q}}(x-y\theta)=\ord_{\mathfrak{q}}(1 - (y/x) \theta)$ for any prime ideal $\mathfrak{q}$ lying over $p$. The remainder of the proof is analogous to the proof of Lemma~\ref{lem:AffinePatch1Check}.
\end{proof}

%--------------------------------------------------------------------------------------------------------------------------------------------%

\subsection{Computational remarks and refinements}
\label{subsec:PIRLRemarks}

In implementing Algorithms~\ref{alg:AffinePatch1} and \ref{alg:AffinePatch2}, we reduce the number of prime ideals appearing in the factorization of $(x-y\theta)\mathcal{O}_K$ to a large power. The Prime Ideal Removing Lemma, as originally stated in Tzanakis - de Weger outlines a similar process by comparing the valuations of $(x-y\theta)\mathcal{O}_K$ at two prime ideals $\mathfrak{p}_1$ and $\mathfrak{p}_2$ above $p$. Of course if $\mathfrak{p}_1 \mid (x-y\theta)\mathcal{O}_K$, we restrict the possible values for $x$ and $y$ modulo $p$. However any choice of $x$ and $y$ modulo $p$ affects the valuations of $(x-y\theta)\mathcal{O}_K$ at all prime ideals above $p$. In the present refinement outlined by Lemma~\ref{lem:AffinePatch1} and Lemma~\ref{lem:AffinePatch2}, we instead study the valuations of $(x-y\theta)\mathcal{O}_K$ at all prime ideals above $p$ simultaneously. This presents us with considerably less ideal equations of the form \ref{eq:idealTM} to resolve. 

Moreover, this variant of the Prime Ideal Removing Lemma permits the following additional refinements:
\begin{itemize}
\item Let $\mathfrak{b} \in L_p$. If there exists a pair $(\mathfrak{b}^\prime,\mathfrak{p}) \in M_p$ with $\mathfrak{b}^\prime \mid \mathfrak{b}$ and $\mathfrak{b}/\mathfrak{b}^\prime=\mathfrak{p}^w$
for some $w \ge 0$, then we may delete $\mathfrak{b}$ from $L_p$. In doing so, the conclusion to Lemma~\ref{lem:AffinePatch2Check} continues to hold.
\item Suppose $(\mathfrak{b},\mathfrak{p})$, $(\mathfrak{b}^\prime,\mathfrak{p}) \in M_p$ with $\mathfrak{b}^{\prime} \mid \mathfrak{b}$, and $\mathfrak{b}/\mathfrak{b}^{\prime}=\mathfrak{p}^w$ for some ${w \geq 0}$. Then, we may delete $(\mathfrak{b},\mathfrak{p})$ from $M_p$ without affecting the conclusion to Lemma~\ref{lem:AffinePatch2Check}. 
\end{itemize}

%--------------------------------------------------------------------------------------------------------------------------------------------%
%--------------------------------------------------------------------------------------------------------------------------------------------%

\section{Factorization of the Thue-Mahler equation}
\label{sec:FactorizationTM}

After applying Algorithm~\ref{alg:AffinePatch1} and Algorithm~\ref{alg:AffinePatch2}, we are reduced to solving finitely many ideal equations of the form
\begin{equation}\label{eq:TMfactored}
(x-y\theta)\mathcal{O}_K=\mathfrak{a} \mathfrak{p}_1^{u_1}\cdots \mathfrak{p}_{\nu}^{u_{\nu}}
\end{equation}
in integer variables $x,y,u_1, \dots, u_{\nu}$ with $u_i \geq 0$ for $i = 1, \dots, \nu$, where ${0 \leq \nu \leq v}$. Here
\begin{itemize}
\item for $i \in \{1, \dots, \nu\}$, $\mathfrak{p}_i$ is a prime ideal of $\mathcal{O}_K$ arising from Algorithm~\ref{alg:AffinePatch1} and Algorithm~\ref{alg:AffinePatch2} applied to $p \in \{p_1, \dots, p_v\}$, such that $(\mathfrak{b}, \mathfrak{p}_i) \in M_p$ for some ideal $\mathfrak{b}$;
\item for $i \in \{\nu+1, \dots, v\}$, the corresponding rational prime $p_i \in S$ yields $M_{p_i} = \emptyset$, in which case we set $u_i = 0$;
\item $\mathfrak{a}$ is an ideal of $\mathcal{O}_K$ of norm $|c|\cdot p_1^{t_1} \cdots p_v^{t_v}$ such that
$u_i + t_i =  z_i$. 
\end{itemize}

For each choice of $\mathfrak{a}$ and prime ideals $\mathfrak{p}_1, \dots, \mathfrak{p}_{\nu}$, we reduce equation~\eqref{eq:TMfactored} to a number of so-called ``$S$-unit equations''. We present two different algorithms for doing so and outline the advantages and disadvantages of each. In practicality, we do not know a priori which of these two options is more efficient. Instead, we implement and use both algorithms simultaneously and selecting the most computationally efficient option as it appear. 

%--------------------------------------------------------------------------------------------------------------------------------------------%

\subsection{Avoiding the class group $\Cl(K)$}
\label{subsec:FactorizationTMwithoutOK}

For $i = 1, \dots, {\nu}$ let $h_i$ be the smallest positive integer for which $\mathfrak{p}_i^{h_i}$ is principal and let 
$r_i$ be a positive integer satisfying $0 \leq r_i < h_i$. Let
\[\mathbf{a}_i = (a_{1i}, \dots, a_{{\nu}i}).\]
where $a_{ii} = h_i$ and $a_{ji} = 0$ for $j \neq i$. We let $A$ be the matrix with columns $\mathbf{a}_1, \dots, \mathbf{a}_{\nu}$. Hence $A$ is a $\nu \times \nu$ diagonal matrix over $\mathbb{Z}$ with diagonal entries $h_i$. Now, if \eqref{eq:TMfactored} has a solution $\mathbf{u} = (u_1, \dots, u_{\nu})$, it necessarily must be of the form $\mathbf{u} = A\mathbf{n} + \mathbf{r}$, where $\mathbf{n} = (n_1, \dots, n_{\nu})$ and $\mathbf{r} = (r_1, \dots, r_{\nu})$. The vector $\mathbf{n}$ is comprised of integers $n_i$ which we solve for. The vector $\mathbf{r}$ is comprised of the values $r_i$ satisfying $0 \leq r_i < h_i$ for $i = 1, \dots, \nu$. 

Using the above notation, we let
\[\mathfrak{c}_i = \tilde{\mathfrak{p}}^{\mathbf{a}_i}=\mathfrak{p}_1^{a_{1i}}\cdot \mathfrak{p}_2^{a_{2i}} \cdots \mathfrak{p}_{\nu}^{a_{{\nu}i}} = \mathfrak{p}_i^{h_i} \]
for all $i \in \{1, \dots, {\nu}\}$.

Thus, we can write \eqref{eq:TMfactored} as
\[ (x-y\theta) \mathcal{O}_K = \mathfrak{a} \tilde{\mathfrak{p}}^{\mathbf{u}}  = (\mathfrak{a} \cdot \tilde{\mathfrak{p}}^\mathbf{r}) \cdot \mathfrak{c}_1^{n_1}\cdots \mathfrak{c}_{\nu}^{n_{\nu}}.\]

By definition of $h_i$, each $i \in \{1, \dots, {\nu}\}$ yields an element $\gamma_i \in K^*$ such that 
\[\mathfrak{c}_i = (\gamma_i) \mathcal{O}_K.\]
Furthermore, if $\mathbf{u}$ is a solution of \eqref{eq:TMfactored} with corresponding vectors $\mathbf{n}, \mathbf{r}$, there exists some $\alpha \in K^*$ such that 
\[\mathfrak{a} \cdot \tilde{\mathfrak{p}}^\mathbf{r}= (\alpha)\mathcal{O}_K.\]

%--------------------------------------------------------------------------------------------------------------------------------------------%

\subsection{Using the class group $\Cl(K)$}
\label{subsec:FactorizationTMwithOK}

Let $\mathbf{u}=(u_1,\dots, u_{\nu})$ be a solution of \eqref{eq:TMfactored} and consider the map
\[\phi : \mathbb{Z}^{\nu} \rightarrow \text{Cl}(K), \qquad (x_1,\dots ,x_{\nu}) \mapsto [\mathfrak{p}_1]^{x_1}\cdots [\mathfrak{p}_{\nu}]^{x_{\nu}},\]
where $[ \mathfrak{q} ]$ denotes the equivalence class of the fractional ideal $\mathfrak{q}$. 
Since the product of $\mathfrak{a}$ and $\mathfrak{p}_1^{u_1}\cdots \mathfrak{p}_{\nu}^{u_{\nu}}$ defines a principal ideal, the map $\phi$ implies
\[\phi(\mathbf{u})=[\mathfrak{a}]^{-1}.\]
In particular, if $[\mathfrak{a}]^{-1}$ does not belong to the image of $\phi$ then \eqref{eq:TMfactored} has no solutions. We therefore suppose that $[\mathfrak{a}]^{-1}$ belongs to the image. Let $\mathbf{r}=(r_1,\dotsc,r_{\nu})$ denote a preimage of $[\mathfrak{a}]^{-1}$ and observe that $\mathbf{u} - \mathbf{r}$ belongs to the kernel of $\phi$. The kernel is a subgroup of $\mathbb{Z}^v$ of rank $\nu$. Let $\mathbf{a}_1,\dots,\mathbf{a}_{\nu}$ be a basis for the kernel, where
\[\mathbf{a}_i = (a_{1i}, \dots, a_{\nu i}) \quad \text{ for } i = 1, \dots, \nu.\]
Let
\[\mathbf{u}-\mathbf{r}=n_1 \mathbf{a}_1+\cdots + n_{\nu} \mathbf{a}_{\nu}\]
for some integers $n_i \in \mathbb{Z}$ and let $A$ denote the $\nu \times \nu$ matrix over $\mathbb{Z}$ with columns $\mathbf{a}_1,\dots,\mathbf{a}_{\nu}$. It follows that $\mathbf{u}= A\mathbf{n}+\mathbf{r}$ where $\mathbf{n} = (n_1,\dots,n_{\nu})$.

For $\mathbf{a}_i=(a_{1i},\dotsc,a_{\nu i}) \in \mathbb{Z}^{\nu}$, we adopt the notation 
\[\tilde{\mathfrak{p}}^\mathbf{a} :=\mathfrak{p}_1^{a_{1i}}\cdot \mathfrak{p}_2^{a_{2i}} \cdots \mathfrak{p}_{\nu}^{a_{\nu i}}.\]
Let
\[\mathfrak{c}_1= \tilde{\mathfrak{p}}^{\mathbf{a}_1},\dotsc,\mathfrak{c}_{\nu}= \tilde{\mathfrak{p}}^{\mathbf{a}_{\nu}}.\]
Thus, we can rewrite \eqref{eq:TMfactored} as
\[(x-y\theta) \mathcal{O}_K = \mathfrak{a} \tilde{\mathfrak{p}}^{\mathbf{u}} = (\mathfrak{a} \cdot \tilde{\mathfrak{p}}^\mathbf{r}) \cdot \mathfrak{c}_1^{n_1}\cdots \mathfrak{c}_{\nu}^{n_{\nu}}.\]

Consider the ideal equivalence class of $(\mathfrak{a} \cdot \tilde{\mathfrak{p}}^\mathbf{r})$ in $\Cl(K)$ and note that
\[[\mathfrak{a} \cdot \tilde{\mathfrak{p}}^\mathbf{r}] 
	= [\mathfrak{a}] \cdot [\mathfrak{p}_1]^{r_1}\cdots [\mathfrak{p}_{\nu}]^{r_{\nu}} 
	= [\mathfrak{a}]\cdot \phi(r_1,\dotsc,r_{\nu})=[1]\]
as $\phi(r_1,\dotsc,r_{\nu})=[\mathfrak{a}]^{-1}$ by construction. This means 
\[\mathfrak{a} \cdot \tilde{\mathfrak{p}}^\mathbf{r}= (\alpha) \mathcal{O}_K\]
for some $\alpha \in K^*$. Furthermore, 
\[[\mathfrak{c}_i] = [\tilde{\mathfrak{p}}^{\mathbf{a}_i}] = \phi(\mathbf{a}_i) = [1] \quad \text{ for } i = 1, \dots, \nu,\]
as the $\mathbf{a}_i$ are a basis for the kernel of $\phi$. For all $i \in \{1, \dots, {\nu}\}$, we therefore have
\[\mathfrak{c}_i= (\gamma_i) \mathcal{O}_K\]
for some $\gamma_i \in K^*$.

%--------------------------------------------------------------------------------------------------------------------------------------------%

\subsection{The $S$-unit equation}
\label{subsec:SUnitEquation}

\autoref{subsec:FactorizationTMwithoutOK} and  \autoref{subsec:FactorizationTMwithOK} outline two different algorithms to reduce the ideal equation~\eqref{eq:TMfactored} to a number of certain ``$S$-unit equations'', which we define shortly. Regardless of which method we use, under both algorithms outlined above, equation~\eqref{eq:TMfactored} becomes
\begin{equation} \label{eq:TMprincipal}
(x-y\theta) \mathcal{O}_K= (\alpha \cdot \gamma_1^{n_1} \cdots \gamma_{\nu}^{n_{\nu}}) \mathcal{O}_K
\end{equation}
for some vector $\mathbf{n} = (n_1, \dots, n_{\nu}) \in \mathbb{Z}^{\nu}$. The ideal generated by $\alpha$ in $K$ has norm 
\[|c|\cdot p_1^{t_1 + r_1} \cdots p_{\nu}^{t_{\nu} + r_{\nu}}p_{\nu +1}^{t_{\nu +1}} \cdots p_v^{t_v}\]
and the $n_i$ are related to the $z_i$ via
\[z_i = u_i + t_i = \sum_{j = 1}^{\nu}n_ja_{ij} + r_i + t_i \quad \text{ for } i =1, \dots, v.\]
where $u_i = r_i = 0$ for all $i \in \{\nu + 1, \dots, v\}$. 

Fix a complete set of fundamental units $\{\eps_1, \dots, \eps_r\}$ of $\mathcal{O}_K$. Here $r = s + t -1$, where $s$ denotes the number of real embeddings of $K$ into $\mathbb{C}$ and $t$ denotes the number of complex conjugate pairs of non-real embeddings of $K$ into $\mathbb{C}$. Then, under either method, equation~\eqref{eq:TMfactored} reduces to a finite number of equations in $K$ of the form
\begin{equation} \label{eq:TMinK}
x-y\theta = \alpha \zeta \varepsilon_1^{a_1} \cdots \varepsilon_r^{a_r}\gamma_1^{n_1}\cdots \gamma_{\nu}^{n_{\nu}}
\end{equation}
with unknowns $a_i \in \mathbb{Z}$, $n_i \in \mathbb{Z}$, and $\zeta$ in the set $T$ of roots of unity in $\mathcal{O}_K$. Since $T$ is finite, we treat $\zeta$ as another parameter. 

Let $p \in \{p_1, \dots, p_v, \infty\}$. Recall that $g(t)$ is an irreducible polynomial in $\mathbb{Z}[t]$ arising from \eqref{eq:ThueMahler3} such that
\[g(t) = f(t,1) = t^n + C_1 t^{n-1} + \dots + C_{n-1}t + C_n.\]
Denote the roots of $g(t)$ in $\overline{\mathbb{Q}_p}$ (where $\overline{\mathbb{Q}_{\infty}} = \overline{\mathbb{R}} = \mathbb{C}$) by $\theta^{(1)}, \dots, \theta^{(n)}$. Let $i_0, j, k \in \{1,\dots, n\}$ be distinct indices and consider the three embeddings of $K$ into $\overline{\mathbb{Q}_p}$ defined by $\theta \mapsto \theta^{(i_0)}, \theta^{(j)}, \theta^{(k)}$. We use $z^{(i)}$ to denote the image of $z$ under the embedding $\theta \mapsto \theta^{(i)}$. From the Siegel identity
\[(\theta^{(i_0)} - \theta^{(j)})(x-y\theta^{(k)}) + (\theta^{(j)} - \theta^{(k)})(x-y\theta^{(i_0)}) + (\theta^{(k)} - \theta^{(i_0)})(x-y\theta^{(j)}) = 0,\]
applying the embeddings to $\beta = x-y\theta$ yields the so-called ``$S$-unit equation''
\begin{equation} \label{eq:Sunit}
\delta_1 \prod_{i = 1}^r\left( \frac{\varepsilon_i^{(k)}}{\varepsilon_i^{(j)}}\right)^{a_i}\prod_{i = 1}^{\nu} \left( \frac{\gamma_i^{(k)}}{\gamma_i^{(j)}}\right)^{n_i} - 1 = \delta_2 \prod_{i = 1}^{r}\left( \frac{\varepsilon_i^{(i_0)}}{\varepsilon_i^{(j)}}\right)^{a_i} \prod_{i = 1}^{\nu} \left( \frac{\gamma_i^{(i_0)}}{\gamma_i^{(j)}}\right)^{n_i},
\end{equation}
where
\[\delta_1 = \frac{\theta^{(i_0)} - \theta^{(j)}}{\theta^{(i_0)} - \theta^{(k)}}\cdot\frac{\alpha^{(k)}\zeta^{(k)}}{\alpha^{(j)}\zeta^{(j)}}, \quad \delta_2 = \frac{\theta^{(j)} - \theta^{(k)}}{\theta^{(k)} - \theta^{(i_0)}}\cdot \frac{\alpha^{(i_0)}\zeta^{(i_0)}}{\alpha^{(j)}\zeta^{(j)}}\]
are constants. 

To summarize, our original problem of solving \eqref{eq:ThueMahler3} for $(x,y,z_1,\dots, z_v)$ has been reduced to solving finitely many equations of the form \eqref{eq:Sunit} for the variables $(x,y, n_1, \dots, n_{\nu},a_1,\dots,a_r)$.

%--------------------------------------------------------------------------------------------------------------------------------------------%

\subsection{Computational remarks and comparisons}
\label{subsec:FactorizationRemarks}

In \autoref{subsec:FactorizationTMwithoutOK}, we follow closely the method of \cite{TW3} to reduce the ideal equation~\eqref{eq:TMfactored} to the $S$-unit equation~\eqref{eq:Sunit}. To implement this reduction, we begin by computing all $h_i$ for which $\mathfrak{p}_i^{h_i}$ is principal for $i = 1, \dots, \nu$. In doing so, we generate all possible values for $r_i$, the non-negative integer satisfying $0 \leq r_i < h_i$. We then generate every possible vector $\mathbf{r} = (r_1, \dots, r_{\nu})$ and test the corresponding ideal product $\mathfrak{a} \cdot \tilde{\mathfrak{p}}^{\mathbf{r}}$ for principality. Those vectors which pass this test yield an $S$-unit equation~\eqref{eq:Sunit}. In the worst case scenario, this method reduces to $h_K^{\nu}$ such equations, where $h_K$ is the class number of $K$. Moreover, this process needs to be applied to every ideal equation~\eqref{eq:TMfactored}, yielding what may be a very large number of principalization tests and subsequent large number of $S$-unit equations to solve. 

In contrast, the method in \autoref{subsec:FactorizationTMwithOK} reduces \eqref{eq:TMfactored} to only $\#T/2$ $S$-unit equations to solve, where $T$ is the set of roots of unity in $K$. In particular, the sum total of $S$-unit equations does not drastically increase. If $[\mathbf{a}]^{-1}$ is not in the image of $\phi$, we reach a contradiction. If $[\mathbf{a}]^{-1}$ is in the image of $\phi$ then we obtain only $\#T/2$ corresponding equations \eqref{eq:Sunit}. In particular, the number of principalization tests in this method is limited by the number of ideal equations~\eqref{eq:TMfactored}, where each such equation yields only $(1+\nu)$ tests. 

However, when generating the vectors $\mathbf{r} = (r_1, \dots, r_{\nu})$ using the class group, we observe that some of the integers $r_i$ may be negative, so we do not expect $\alpha$ to be an algebraic integer in general. This can be problematic later in the algorithm when we compute the embedding of $K$ into our $p$-adic fields. In those instances, the precision on our $p$-adic fields may not be high enough, and as a result, some non-zero elements of $K$ may be erroneously mapped to $0$. To avoid this, we force the $r_i$ to be positive by adding sufficiently many multiples of the class number. 

In most cases, the method described in \autoref{subsec:FactorizationTMwithOK} is far more efficient than that of \autoref{subsec:FactorizationTMwithoutOK}. However, computing the class group may be a very costly computation. Indeed, for some Thue-Mahler equations, this may be the bottle-neck of the algorithm. In this case, it may happen that computing the class group will take longer than directly checking each potential $S$-unit equation arising from the alternative method. Unfortunately, we cannot know a piori how long computing $\Cl(K)$ will take in so much that we cannot know a priori how long solving all $S$-unit equations from the other algorithm will take. In practicality, generating the class group in Magma is a process which cannot be terminated without exiting the program. For this reason, we cannot simply apply a timeout in Magma if computing $\Cl(K)$ is exceeding what we deem a reasonable amount of time. Adding to this, Magma does not support parallelization, so we cannot implement both algorithms simultaneously. Our compromise to solve a single Thue-Mahler equation is to run two separate instances of Magma in parallel, each generating the $S$-unit equations using the two aforementioned algorithms. When one of these instances finishes, the other is forced to terminate. In this way, though far from ideal, we are able to select the most computationally efficient option. 

%--------------------------------------------------------------------------------------------------------------------------------------------%
%--------------------------------------------------------------------------------------------------------------------------------------------%

\section{A small upper bound for $u_l$ in a special case}
\label{sec:SmallBoundForSpecialCase}

We now restrict our attention to those $p \in \{p_1, \dots, p_{\nu}\}$ and study the $p$-adic valuations of the numbers appearing in \eqref{eq:Sunit}. In particular, for $l \in \{1, \dots, \nu\}$, we identify conditions in which $\sum_{j = 1}^{\nu} n_ja_{lj}$ can be bounded by a small explicit constant, where $a_{lj}$ is the $(l,j)^{\text{th}}$ entry of the matrix $A$ derived in either \autoref{subsec:FactorizationTMwithoutOK} or \autoref{subsec:FactorizationTMwithOK}. Recall that $u_l + r_l = \sum_{j = 1}^{\nu} n_ja_{lj}$, where $r_l$ is known, so that a bound on $\sum_{j = 1}^{\nu} n_ja_{lj}$ yields a bound on the exponent $u_l$ in \eqref{eq:TMfactored}.

Fix a rational prime $p_l \in \{p_1, \dots, p_{\nu}\}$ and recall that $z \in \mathbb{C}_{p_l}$ having $\ord_{p_l}(z) = 0$ is called a $p_l$-adic unit. Part (i) of the Corollary of Lemma 7.2 of \cite{TW3} tells us that $\frac{\eps_1^{(i_0)}}{\varepsilon_1^{(j)}}, \dots, \frac{\eps_r^{(i_0)}}{\varepsilon_r^{(j)}}$ and $\frac{\varepsilon_1^{(k)}}{\varepsilon_1^{(j)}}, \dots, \frac{\varepsilon_r^{(k)}}{\varepsilon_r^{(j)}}$ are $p_l$-adic units. 

Let $g_l(t)$ be the irreducible factor of $g(t)$ in $\mathbb{Q}_{p_l}[t]$ corresponding to the prime ideal $\mathfrak{p}_l$. Since $\mathfrak{p}_l$ has ramification index and residue degree equal to $1$, $\deg(g_l(t)) = 1$. We now choose $i_0 \in \{1, \dots, 4\}$ so that $\theta^{(i_0)}$ is the root of $g_l(t)$. We fix this choice of index $i_0$ for the remainder of this chapter. The indices of $j,k$ are fixed, but arbitrary. 

\begin{lemma} \label{lem:SunitUnits} \
\begin{enumerate}
\item[(i)] Let $i \in \{1, \dots, \nu\}$. Then $\frac{\gamma_i^{(k)}}{\gamma_i^{(j)}}$ are $p_l$-adic units. 
\item[(ii)] Let $i \in \{1, \dots, \nu\}$. Then $\ord_{p_l}\left(\frac{\gamma_i^{(i_0)}}{\gamma_i^{(j)}}\right) = a_{li}$, where $\mathbf{a_i} = (a_{1i}, \dots, a_{vi})$ is the $i^{\text{th}}$ column of the matrix $A$ of either \autoref{subsec:FactorizationTMwithoutOK} or \autoref{subsec:FactorizationTMwithOK}. 
\end{enumerate}
\end{lemma}

\begin{proof}
Consider the factorization $g(t) = g_1(t) \cdots g_m(t)$ of $g(t)$ in $\mathbb{Q}_{p_l}[t]$. Note that $\theta^{(j)}$ is a root of some $g_h(t) \neq g_l(t)$. Let $\mathfrak{p}_h$ be the corresponding prime ideal above $p_l$ and $e(\mathfrak{p}_h|p_l)$ be its ramification index. Then $\mathfrak{p} \neq \mathfrak{p}_l$ and since 
\[(\gamma_i)\mathcal{O}_K = \mathfrak{p}_1^{a_{1i}} \cdots \mathfrak{p}_v^{a_{vi}},\]
we have 
\[\ord_{p_l}(\gamma_i^{(j)}) = \frac{1}{e(\mathfrak{p}_h|p_l)}\ord_{\mathfrak{p}_h}(\gamma_i) = 0.\]
An analogous argument gives $\ord_{p_l}(\gamma_i^{(k)}) = 0$. On the other hand, 
\[\ord_{p_l}(\gamma_i^{(i_0)}) = \frac{1}{e(\mathfrak{p}_l|p_l)}\ord_{\mathfrak{p}_l}(\gamma_i) = \ord_{\mathfrak{p}_l}(\mathfrak{p}_1^{a_{1i}} \cdots \mathfrak{p}_v^{a_{vi}}) = a_{li}.\]
\end{proof}

The next lemma deals with a special case in which the sum $\sum_{j = 1}^{\nu} n_ja_{lj}$ can be computed directly. This lemma is analogous to Lemma 7.3 of \cite{TW3}.

Recall the constants
\[\delta_1 = \frac{\theta^{(i_0)} - \theta^{(j)}}{\theta^{(i_0)} - \theta^{(k)}}\cdot\frac{\alpha^{(k)}\zeta^{(k)}}{\alpha^{(j)}\zeta^{(j)}}, \quad \delta_2 = \frac{\theta^{(j)} - \theta^{(k)}}{\theta^{(k)} - \theta^{(i_0)}}\cdot \frac{\alpha^{(i_0)}\zeta^{(i_0)}}{\alpha^{(j)}\zeta^{(j)}}\]
of \eqref{eq:Sunit}.
\begin{lemma}\label{lem:Delta1}
Let $l \in \{1, \dots, v\}$. If $\ord_{p_l}(\delta_1) \neq 0$, then 
\[ \sum_{i = 1}^{\nu} n_ia_{li} = \min\{\ord_{p_l}(\delta_1), 0\} - \ord_{p_l}(\delta_2).\]
\end{lemma}

\begin{proof}
Apply the Corollary of Lemma $7.2$ of \cite{TW3} and Lemma~\ref{lem:SunitUnits} to both expressions of $\lambda$ in \eqref{eq:Sunit}. On the one hand, we obtain that $\ord_{p_l}(\lambda) = \min\{\ord_{p_l}(\delta_1), 0\}$, and on the other hand, 
\begin{align*}
\ord_{p_l}(\lambda)
& = \ord_{p_l}(\delta_2) + \sum_{i = 1}^{\nu} \ord_{p_l}\left( \frac{\gamma_i^{(i_0)}}{\gamma_i^{(j)}}\right)^{n_i}\\
& = \ord_{p_l}(\delta_2) + \sum_{i = 1}^{\nu} n_ia_{li}.
\end{align*}
\end{proof}

For the remainder of this section, we assume $\ord_{p_l}(\delta_1) = 0$. Here, it is convenient to use the notation
\[b_1 = 1, \quad b_{1+i} = n_i \ \text{ for } i \in \{1, \dots, \nu\},\] 
and
\[ b_{1+{\nu}+i} = a_i \ \text{ for } i  \in \{1, \dots, r\}.\]
Put
\[\alpha_1 = \log_{p_l} \delta_1, \quad \alpha_{1+i} = \log_{p_l}\left( \frac{\gamma_i^{(k)}}{\gamma_i^{(j)}}\right)  \ \text{ for } i \in \{1, \dots, \nu\},\]
and
\[\alpha_{1+\nu+i} = \log_{p_l}\left( \frac{\varepsilon_i^{(k)}}{\varepsilon_i^{(j)}}\right) \ \text{ for } i  \in \{1, \dots, r\}.\]
Define
\[\Lambda_l = \sum_{i = 1}^{1+\nu+r} b_i\alpha_i.\]

Let $L$ be a finite extension of $\mathbb{Q}_{p_l}$ containing $\delta_1$, $\frac{\gamma_1^{(k)}}{\gamma_1^{(j)}}, \dots, \frac{\gamma_{\nu}^{(k)}}{\gamma_{\nu}^{(j)}}$, and $\frac{\varepsilon_1^{(k)}}{\varepsilon_1^{(j)}}, \dots, \frac{\varepsilon_r^{(k)}}{\varepsilon_r^{(j)}}$. Since finite $p_l$-adic fields are complete, $\alpha_i \in L$ for $i = 1, \dots, 1+\nu+r$ as well. Choose $\phi \in \overline{\mathbb{Q}_{p_l}}$ such that $L = \mathbb{Q}_{p_l}(\phi)$ and $\ord_{p_l}(\phi) > 0 $. Let $G(t)$ be the minimal polynomial of $\phi$ over $\mathbb{Q}_{p_l}$ and let $s$ be its degree. For $i = 1, \dots, 1+\nu+r$ write
\[\alpha_i = \sum_{h = 1}^s \alpha_{ih}\phi^{h - 1}, \quad \alpha_{ih} \in \mathbb{Q}_{p_l}.\]
Then
\begin{equation} \label{eq:LambdaL}
\Lambda_l = \sum_{h = 1}^s \Lambda_{lh}\phi^{h-1},
\end{equation}
with
\[\Lambda_{lh} = \sum_{i = 1}^{1+\nu+r} b_i \alpha_{ih}\]
for $h = 1, \dots, s$. 

\begin{lemma}\label{lem:DiscG}
For every $h \in \{1, \dots, s\}$, we have
\[\ord_{p_l}(\Lambda_{lh}) > \ord_{p_l}(\Lambda_l) - \frac{1}{2}\ord_{p_l}(\text{Disc}(G(t))).\]
\end{lemma}

\begin{proof}
Taking the images of \eqref{eq:LambdaL} under conjugation $\phi \mapsto \phi^{(h)}$ ($h = 1, \dots, s$) yields
\[\begin{bmatrix}
\Lambda_l^{(1)} \\
\vdots \\
\Lambda_l^{(s)}	\\
\end{bmatrix}
=
\begin{bmatrix}
1 		& \phi^{(1)} 	& \cdots 	& \phi^{(1)s-1}\\
\vdots 	& \vdots 		& 		& \vdots \\
1 		& \phi^{(s)} 	& \cdots  	& \phi^{(s)s-1}\\
\end{bmatrix}
\begin{bmatrix}
\Lambda_{l1}\\
\vdots \\
\Lambda_{ls}\\
\end{bmatrix}\]
The $s \times s$ matrix $(\phi^{(h)i-1})$ above is invertible, with inverse
\[\frac{1}{\displaystyle \prod_{1\leq j<k\leq s} (\phi^{(k)} - \phi^{(j)})}
\begin{bmatrix}
\gamma_{11} 	& \cdots 	& \gamma_{1s}\\
\vdots 		& 		& \vdots\\
\gamma_{s1} 	& \cdots 	& \gamma_{ss}\\
\end{bmatrix},\]
where $\gamma_{jk}$ is an integral polynomial in the entries of $(\phi^{(h)i-1})$. Since $\ord_{p_l}(\phi) > 0$ and $\ord_{p_l}(\phi^{(h)}) = \ord_{p_l}(\phi)$ for all $h = 1, \dots, s$, it follows that $\ord_{p_l}(\gamma_{jk}) > 0 $ for every $\gamma_{jk}$. Therefore, as 
\[\Lambda_{lh} = \frac{1}{\displaystyle \prod_{1\leq j<k\leq s}(\phi^{(k)} - \phi^{(j)})}\sum_{i = 1}^s \gamma_{hi}\Lambda_l^{(i)},\]
we have 
\begin{align*}
\ord_{p_l}(\Lambda_{lh}) 
	& = \min_{1 \leq i \leq s} \left\{\ord_{p_l}(\gamma_{hi}) + \ord_{p_l}(\Lambda_l^{(i)})\right\} -\frac{1}{2}\ord_{p_l}(\text{Disc}(G(t)))\\
	& \geq \min_{1 \leq i \leq s} \ord_{p_l}(\Lambda_l^{(i)}) +  \min_{1 \leq i \leq s} \ord_{p_l}(\gamma_{hi}) - \frac{1}{2}\ord_{p_l}(\text{Disc}(G(t)))\\
	& = \ord_{p_l}\Lambda_l + \min_{1 \leq i \leq s} \ord_{p_l}(\gamma_{hi}) - \frac{1}{2}\ord_{p_l}(\text{Disc}(G(t)))
\end{align*}
for every $h \in \{1, \dots, s\}$. 
%\min_{1 \leq i \leq s} \left\{\ord_{p_l}(\gamma_{hi}) + \ord_{p_l}(\Lambda_l^{(i)}) -\frac{1}{2}\ord_{p_l}(\text{Disc}(G(t)))\right\}\]
\end{proof}

\begin{lemma} \label{lem:Lambda}
If 
\[\sum_{i = 1}^{\nu} n_{i}a_{li} > \frac{1}{p_l-1} - \ord_{p_l}(\delta_2),\]
then
\[\ord_{p_l}(\Lambda_l) = \sum_{i = 1}^{\nu} n_{i}a_{li} + \ord_{p_l}(\delta_2).\]
\end{lemma}

\begin{proof}
Immediate from Lemma~\ref{lem:pAdicLogarithms2}.
\end{proof}

\begin{lemma} \label{lem:specialcase} \
\begin{enumerate}[(i)]
\item If $\ord_{p_l}(\alpha_1) < \displaystyle \min_{2 \leq i \leq 1+\nu+r} \ord_{p_l}(\alpha_i)$, then
\[\sum_{i = 1}^{\nu} n_i a_{li} \leq \max \left\{ \bigg\lfloor{\frac{1}{p_l-1} - \ord_{p_l}(\delta_2)}\bigg\rfloor,  \bigg \lceil\displaystyle \min_{2 \leq i \leq 1+\nu+r} \ord_{p_l}(\alpha_{i}) - \ord_{p_l}(\delta_2) \bigg \rceil - 1 \right\}\]
\item For all $h \in \{1, \dots, s\}$, if $\ord_{p_l}(\alpha_{1h}) < \displaystyle \min_{2 \leq i \leq 1+\nu+r} \ord_{p_l}(\alpha_{ih})$, then
\[\sum_{i = 1}^{\nu} n_i a_{li} \leq \max \left\{ \bigg\lfloor{\frac{1}{p_l-1} - \ord_{p_l}(\delta_2)}\bigg\rfloor, \bigg \lceil \displaystyle \min_{2 \leq i \leq 1+\nu+r} \ord_{p_l}(\alpha_{ih})- \ord_{p_l}(\delta_2) + w_l \bigg \rceil - 1\right\},\]
where 
\[w_l = \frac{1}{2}\ord_{p_l}(\text{Disc}(G(t))).\]
\end{enumerate}
\end{lemma}

\begin{proof} \
\begin{enumerate}[(i)]
\item We prove the contrapositive. Suppose
\[\sum_{i = 1}^{\nu} n_i a_{li} > \frac{1}{p_l-1} - \ord_{p_l}(\delta_2), \]
and
\[\sum_{i = 1}^{\nu} n_i a_{li}  \geq \displaystyle \min_{2 \leq i \leq 1+\nu+r} \ord_{p_l}(\alpha_{i}) - \ord_{p_l}(\delta_2).\]
Observe that
\begin{align*}
\ord_{p_l}(\alpha_{1}) 	
	& = \ord_{p_l}\left( \Lambda_{l} - \sum_{i = 2}^{1+\nu+r}b_i\alpha_{i}\right) \\
	& \geq \min\left\{ \ord_{p_l}(\Lambda_{l}), \min_{2 \leq i \leq 1+\nu+r} \ord_{p_l}(b_i\alpha_{i})\right\}.
\end{align*}
Therefore, it suffices to show that 
\[\ord_{p_l}(\Lambda_{l}) \geq \min_{2 \leq i \leq 1+\nu+r} \ord_{p_l}(b_i\alpha_{i}).\]
By Lemma~\ref{lem:pAdicLogarithms2}, the first inequality implies ${\ord_{p_l}(\Lambda_{l}) = \displaystyle \sum_{i = 1}^{\nu} n_ia_{li} + \ord_{p_l}(\delta_2)}$, from which the result follows. 

\item Similar to the proof of (i).
%\item[(ii)] We prove the contrapositive. Let $h \in \{1, \dots, s\}$ and suppose
%\[\sum_{i = 1}^v n_i a_{li} > \frac{1}{p-1} - \ord_{p_l}(\delta_2), \]
%and
%\[\sum_{i = 1}^v n_i a_{li}  \geq \nu_l + \displaystyle \min_{2 \leq i \leq v+2} \ord_{p_l}(\alpha_{ih}) - \ord_{p_l}(\delta_2).\]
%Observe that 
%\[\begin{split}
%\ord_{p_l}(\alpha_{1h}) 	
%	& = \ord_{p_l}\left( \Lambda_{lh} - \sum_{i = 2}^{v+2}b_i\alpha_{ih}\right) \\
%	& \geq \min\left\{ \ord_{p_l}(\Lambda_{lh}), \min_{2 \leq i \leq v+2} \ord_{p_l}(b_i\alpha_{ih})\right\}
%\end{split}\]
%Therefore, it suffices to show that 
%\[\ord_{p_l}(\Lambda_{lh}) \geq \min_{2 \leq i \leq v+2} \ord_{p_l}(b_i\alpha_{ih}).\]
%By Lemma~\ref{Lem:padic}, the first inequality implies $\ord_{p_l}(\Lambda_{l}) = \displaystyle \sum_{i = 1}^v n_ia_{li} + \ord_{p_l}(\delta_2)$. Combining this with Lemma~\ref{Lem:discG} yields
%\[\ord_{p_l}(\Lambda_{lh}) \geq \displaystyle \sum_{i = 1}^v n_ia_{li} + \ord_{p_l}(\delta_2) - \nu_l.\]
%The results now follow from our second assumption. 
\end{enumerate}
\end{proof}

%--------------------------------------------------------------------------------------------------------------------------------------------%
%--------------------------------------------------------------------------------------------------------------------------------------------%

\section{Lattice-Based Reduction}
\label{sec:LatticeReduction}

At this point in solving the Thue-Mahler equation, we proceed to solve each $S$-unit equation~\eqref{eq:Sunit} for the exponents $(n_1, \dots, n_{\nu}, a_1, \dots, a_r)$. To do so, we generate a very large upper bound on the exponents and reduce this bound via Diophantine approximation computations. The specific details of this process are described in \autoref{ch:EfficientTMSolver} and \autoref{ch:Goormaghtigh}. In general, from each $S$-unit equation, we generate several linear forms in logarithms to which we associate an integral lattice $\Gamma$. It will be important in this reduction process to enumerate all short vectors in these lattices. In this section, we describe two algorithms used in the short vector enumeration process. 

%--------------------------------------------------------------------------------------------------------------------------------------------%

\subsection{The $L^3$-lattice basis reduction algorithm}
\label{subsec:LLL}

Let $\Gamma$ be an $n$-dimensional lattice with basis vectors $\mathbf{b}_1, \dots, \mathbf{b}_n$ equipped with a bilinear form $\Phi: \Gamma \times \Gamma \to \mathbb{Z}$. Recall that $\Phi$ defines a norm on $\Gamma$ via the usual inner product on $\mathbb{R}^n$. For $i = 1, \dots, n$, define the vectors $\mathbf{b}_i^*$ inductively by
\[\mathbf{b}_i^* = \mathbf{b}_i - \sum_{j=1}^{i-1}\mu_{ij}\mathbf{b}_j^*, \quad \mu_{ij} = \frac{\Phi(\mathbf{b}_i,\mathbf{b}_j^*)}{\Phi(\mathbf{b}_j^*,\mathbf{b}_j)},\]
where $\mu_{ij} \in \mathbb{R}$ for $1\leq j < i \leq n$. This is the usual Gram-Schmidt process. The basis $\mathbf{b}_1,\dots, \mathbf{b}_n$ is called \textit{LLL-reduced} if
\[|\mu_{ij}| \leq \frac{1}{2} \quad \text{ for } 1\leq j < i \leq n, \]
\[\frac{3}{4}|\mathbf{b}_{i-1}^*|^2 \leq |\mathbf{b}_i^* + \mu_{ii-1}\mathbf{b}_{i-1}^*|^2 \quad \text{ for } 1 <i \leq n,\]
where $| \cdot |$ is the usual Euclidean norm in $\mathbb{R}^n$, 
\[|\mathbf{v}| = \Phi(\mathbf{v},\mathbf{v}) = \mathbf{v}^{T}\mathbf{v}.\]

These properties imply that an LLL-reduced basis is approximately orthogonal, and that, generically, its constituent vectors are roughly of the same length. Every $n$-dimensional lattice has an LLL-reduced basis and such a basis can be computed very quickly using the so-called LLL algorithm (\cite{LLL}). This algorithm takes as input an arbitrary basis for a lattice and outputs an LLL-reduced basis. The algorithm is typically modified to additionally output a unimodular matrix $U$ such that $A = BU$, where $B$ is the matrix whose column-vectors are the input basis and $A$ is the matrix whose column-vectors are the LLL-reduced output basis. Several versions of this algorithm are implemented in Magma, including de Weger's exact integer version. (\cite{Weg0}).

We remark that a lattice may have more than one reduced basis, and that the ordering of the basis vectors is not arbitrary. The properties of reduced bases that are of most interest to us are the following. Let $\mathbf{v}$ a vector in $\mathbb{R}^n$ and denote by $l(\Gamma,\mathbf{v})$ the distance from $\mathbf{v}$ to the nearest point in the lattice $\Gamma$, viz.
\[l(\Gamma,\mathbf{v}) = \min_{\mathbf{u} \in \Gamma \backslash\{\mathbf{v}\}} |\mathbf{u} - \mathbf{v}|.\]
From an LLL-reduced basis for $\Gamma$, we can compute lower bounds for $l(\Gamma,\mathbf{v})$, according to the following results. 

\begin{lemma} \label{lem:LLL}
Let $\Gamma$ be a lattice with LLL-reduced basis $\mathbf{c}_1, \dots, \mathbf{c}_n$ and let $\mathbf{v}$ be a vector in $\mathbb{R}^n$. 
\begin{enumerate}[(a)]
\item If $\mathbf{v} = \mathbf{0}$, then $l(\Gamma,\mathbf{v}) \geq 2^{-(n-1)/2}|\mathbf{c}_1|$.
\item Assume $\mathbf{v} = s_1\mathbf{c}_1 + \cdots + s_n \mathbf{c}_n$, where $s_1, \dots, s_n \in \mathbb{R}$ with not all $s_i \in \mathbb{Z}$. Put 
\[J = \{j \in \{1, \dots, n\} \ : \ s_j \notin \mathbb{Z} \}.\]
For $j \in J$, set 
\[\delta(j) = 
\begin{cases}
\max_{i > j} \|s_i \| |\mathbf{c}_i| 	& \text{ if } j < n\\
0 							& \text{ if } j = n,
\end{cases}\]
where $\| \cdot \|$ denotes the distance to the nearest integer. We have
\[l(\Gamma,\mathbf{v}) \geq \max_{j \in J}\left(2^{-(n-1)/2}\| s_j\| |\mathbf{c}_1| - (n-j)\delta(j)\right).\]
\end{enumerate}
\end{lemma}
Lemma~\ref{lem:LLL} (a) is Proposition 1.11 in \cite{LLL}; proofs can be found in \cite{LLL}, \cite{Weg0} (Section 3.4), or \cite{Sm} (Section V.3). Lemma~\ref{lem:LLL} (b) is a combination of Lemmas 3.5 and 3.6 in \cite{Weg0}. Note that the assumption in Lemma~\ref{lem:LLL} (b) is equivalent to ${\mathbf{v} \notin \Gamma}$. 

We see that the vector $\mathbf{c}_1$ in a reduced basis is, in a very precise sense, not too far from being the shortest non-zero vector of $\Gamma$. As has already been mentioned, what makes this result so valuable is that there is a very simple and efficient algorithm to find a reduced basis in a lattice, namely the LLL algorithm.

%--------------------------------------------------------------------------------------------------------------------------------------------%

\subsection{The Fincke-Pohst algorithm}
\label{subsec:FinckePohst}

Sometimes it is not sufficient to have a lower bound for $l(\Gamma,\mathbf{v})$ only. It may be useful to know exactly all vectors $\mathbf{u} \in \Gamma$ such that $|\mathbf{u}|  = \Phi(\mathbf{u}, \mathbf{u}) \leq C$ for a given constant $C$. This can be done efficiently using an algorithm of Fincke-Pohst (cf. \cite{FP}, \cite{Coh1}). A version of this algorithm with some improvements due to Stehl\'e is implemented in Magma. As input this algorithm takes a matrix $B$, whose columns span the lattice $\Gamma$, and a constant $C > 0$. The output is a list of all lattice points $\mathbf{u} \in \Gamma$ with $|\mathbf{u}| \leq C$, apart from $\mathbf{u} = \mathbf{0}$. In this section, we outline the main steps in this algorithm. 

We begin by letting $B$ denote the basis matrix associated to the lattice $\Gamma$, with corresponding bilinear form $\Phi$. We call a vector $\mathbf{u} \in \Gamma$ \textit{small} if its norm $\Phi(\mathbf{u}, \mathbf{u})$ is less than a constant $C$. As an element of the lattice, $\mathbf{u} = B\mathbf{x}$ for some coordinate vector $\mathbf{x} \in \mathbb{Z}^n$. Let $Q$ be the quadratic form associated to $\Phi$ and let $A=B^TB$. Now finding the short vectors $\mathbf{u} \in \Gamma$ is equivalent to solving 
\begin{equation} \label{eq:ShortVector}
Q(\mathbf{x}) = \mathbf{x}^TA\mathbf{x} \leq C.
\end{equation}

Let $\mathbf{x} = (x_1, \dots, x_n)$. To solve this inequality, we first rearrange the terms of the quadratic form via quadratic completion. Here we assume that $\Gamma$ is positive definite so that every nonzero element of the lattice has a positive norm. With this, we find the Cholesky decomposition $A = R^TR$, where $R$ is an upper triangular matrix, and express $Q$ as
\[ Q(\mathbf{x}) = \sum_{i=1}^n q_{ii}\left( x_i + \sum_{j=i+1}^n q_{ij}x_j\right)^2.\]
The coefficients $q_{ij}$ are defined from $R$ and stored in a matrix $\tilde{Q}$ for convenience. In particular, 
\begin{equation} \label{eq:CholeskyCoeffs}
q_{ij} =
\begin{cases}
\frac{r_{ij}}{r_{ii}} & \text{ if } i < j\\
r_{ii}^2 & \text{ if } i = j.
\end{cases}
\end{equation}
Since $R$ is upper triangular, the matrix $\tilde{Q}$ is as well. This yields the following reformulation of \eqref{eq:ShortVector}
\[ \sum_{i=1}^n q_{ii}\left( x_i + \sum_{j=i+1}^n q_{ij}x_j\right)^2 \leq C.\]
From here we observe that the individual term $q_{nn}x_n^2$ must also be less than $C$. Specifically, 
\[x_n^2 \leq \frac{C}{q_{nn}}\]
so that $x_n$ is bounded above by $\sqrt{C/q_{nn}}$ and below by $-\sqrt{C/q_{nn}}$. This illustrates the first step in establishing bounds on a specific entry $x_i$. Adding more terms from the outer sum to this sequence, a pattern emerges. Let
\[U_k = \sum_{j = k+1}^n q_{kj}x_j,\]
where $U_n = 0$, and rewrite $Q(\mathbf{x})$ as 
\[Q(\mathbf{x}) = \sum_{i=1}^n q_{ii}\left( x_i + \sum_{j=i+1}^n q_{ij}x_j\right)^2 = \sum_{i=1}^n q_{ii}\left( x_i + U_i\right)^2.\]
In general, 
\[q_{kk}(x_k + U_k)^2 \leq C - \sum_{i = k+1}^n q_{ii}(x_i + U_i)^2.\]
Let $T_k$ denote the bound on the right-hand side, 
\[T_k = C - \sum_{i = k+1}^n q_{ii}(x_i + U_i)^2.\]
We set $T_n = C$ and find each subsequent $T_k$ by subtracting the next term from the outer summand,
\[T_k = T_{k+1} - q_{k+1,k+1}(x_{k+1} + U_{k+1})^2.\]
This yields the upper bound
\[q_{kk}(x_k + U_k)^2 \leq T_k\]
so that $x_k$ is bounded above by $\sqrt{T_k/q_{kk}} - U_k$ and below by ${-\sqrt{T_k/q_{kk}} - U_k}$. In this way, we iteratively enumerate all vectors $\mathbf{x}$ satisfying $Q(\mathbf{x}) \leq C$, beginning with the entry $x_n$ of $\mathbf{x}$ and working down towards $x_1$.  

%--------------------------------------------------------------------------------------------------------------------------------------------%

\subsection{Computational remarks and translated lattices}
\label{subsec:FinckePohstRemarks}

Recall that the Cholesky decomposition of $A = B^TB$ yields the upper triangular matrix $R$ where $A = R^TR$. It is noted in the \cite{FP} that if we label the columns of $R$ by $\mathbf{r}_i$ and the rows of $R^{-1}$ by $\mathbf{r}'_i$, then 
\[x_k^2 = \left( \mathbf{r}'^{\ T}_k \cdot \sum_{i=1}^n x_i \mathbf{r}_i \right)^2 \leq \mathbf{r}'^{\ T}_k \mathbf{r}_k (\mathbf{x}^TR^TR\mathbf{x}) \leq | \mathbf{r}'_k |^2C.\]
To reduce the search space, it is thus beneficial to reduce the rows of $R^{-1}$. Furthermore, rearranging the columns of $R$ so that the shortest column vector is first helps reduce the total running time of the Fincke-Pohst algorithm. In particular, doing so leads to progressively smaller intervals in which $x_k$ may exist. 

We express this reduction with a unimodular matrix $V^{-1}$ so that $R_1^{-1} = V^{-1}R^{-1}$. Applying an appropriate permutation matrix $P$, we then reorder the columns of $R_1$. Since $R_1 = RV$, this yields $R_2 = (RV)P$. Finally, we compute the solutions $\mathbf{y}$ to $\mathbf{y}^TR_2^TR_2\mathbf{y}\leq C$ and recover the short vectors $\mathbf{x}$ satisfying the original inequality \eqref{eq:ShortVector} via $\mathbf{x} = VP\mathbf{y}$. 

As before, let $\Gamma$ be an $n$-dimensional lattice with basis matrix $B$, quadratic form $\Phi$, and associated bilinear form $Q$. In \autoref{subsec:FinckePohst}, it is noted that an implementation of the Fincke-Pohst algorithm is available in Magma. Unfortunately, this implementation does not support \textit{translated} lattices, a variant of the Fincke-Pohst algorithm which we will need in \autoref{ch:EfficientTMSolver}. By a translated lattice, we mean the discrete subgroup of $\mathbb{R}^n$ of the form
\[\Gamma + \mathbf{w} = \left\{ \sum_{i=1}^n x_i \mathbf{b}_i + \mathbf{w}\ : \ x_i \in \mathbb{Z} \right\},\]
where $\mathbf{b}_1, \dots, \mathbf{b}_n$ form the columns of $B$ and $\mathbf{w} \in \mathbb{R}^n$. In the remainder of this section, we describe how to modify the Fincke-Pohst algorithm and its refinements to support translated lattices. 

Analogous to the non-translated case, any embedded vector $\mathbf{u}$ of $\Gamma + \mathbf{w}$ may be expressed as $\mathbf{u} = B\mathbf{x} + \mathbf{w}$ for a corresponding coordinate vector $\mathbf{x}$. In this case, we call the vector $\mathbf{u} \in \Gamma + \mathbf{w}$ \textit{small} if 
\begin{equation} \label{eq:TransShortVector}
(\mathbf{x}-\mathbf{c})^TB^TB(\mathbf{x}-\mathbf{c}) \leq C
\end{equation}
for some $C \geq 0$, where $\mathbf{c} = -\mathbf{w}$. 

As in the usual short vectors process, we begin by applying Cholesky decomposition to the positive definite matrix $A=B^TB$ to obtain an upper triangular matrix $R$ satisfying $A = R^TR$. We then generate the matrices $R_1, R_2, V,$ and $P$ described earlier in this section. This allows us to write $A = U^TGU$ for a unimodular matrix $U$ and Gram matrix $G$ given by
\[U = P^{-1}V^{-1} \quad \text{ and } \quad G = R_2^TR_2.\]
Thus the inequality~\eqref{eq:TransShortVector} becomes
\begin{equation} \label{eq:TransShortVector2}
(\mathbf{y}-\mathbf{d})^TG(\mathbf{y}-\mathbf{d}) \leq C
\end{equation}
where
\[\mathbf{y} = U\mathbf{x} \quad \text{ and } \quad \mathbf{d} = U\mathbf{c}.\]
To enumerate the vectors $\mathbf{y}$ which satisfy this inequality, we consider the bilinear form $Q$ associated to the lattice $\Gamma$. We express this form as
\[ Q(\mathbf{y}-\mathbf{d}) = \sum_{i=1}^n q_{ii}\left( y_i - d_i + \sum_{j=i+1}^n q_{ij}(y_j - d_j)\right)^2.\]
As in the usual Fincke-Pohst algorithm, the coefficients $q_{ij}$ are defined from the matrix $R$ via equation~\eqref{eq:CholeskyCoeffs}. Let
\[U_k = -d_k + \sum_{j = k+1}^n q_{kj}(y_j - d_j),\]
where $U_n = -d_n$, and rewrite $Q(\mathbf{y}-\mathbf{d})$ as
\[ Q(\mathbf{y}-\mathbf{d}) = \sum_{i=1}^n q_{ii}\left( y_i - d_i + \sum_{j=i+1}^n q_{ij}(y_j - d_j)\right)^2 = \sum_{i=1}^n q_{ii}\left( y_i + U_i\right)^2.\]
From here, we proceed as in the usual Fincke-Pohst algorithm described in \autoref{subsec:FinckePohst}. Once we compute all vectors $\mathbf{y}$ which satisfy \eqref{eq:TransShortVector2}, we recover $\mathbf{x}$ using $\mathbf{x} = U^{-1}\mathbf{y}$. 

As a final remark about Fincke-Pohst for translated lattices, it is worth noting that one could use the variant implemented in Magma simply by increasing the dimension of the lattice $\Gamma$ and appropriately redefining the basis vectors $\mathbf{b}_i$. This is highly ill-advised as it increases the search space and subsequent running time of the algorithm.  

Generally speaking, the use of Fincke-Pohst in our applications poses one of the main bottlenecks in solving Thue-Mahler and Thue-Mahler-like equations. Specifically, this algorithm often yields upwards of hundreds of millions of short vectors, each one needing to be stored and, in our case, appropriately manipulated. This creates both timing and memory problems, often leading to gigabytes of data usage. Deleting these vectors does not release the memory and, as with the class group function, Magma's built-in Fincke-Pohst process cannot be terminated without exiting the program. The primary advantage of implementing and using our own version of Fincke-Pohst, as described in this section, is therefore the ability to add a fail-stop should the number of vectors found become too large. 

%--------------------------------------------------------------------------------------------------------------------------------------------%
%--------------------------------------------------------------------------------------------------------------------------------------------%

\endinput

Any text after an \endinput is ignored.
You could put scraps here or things in progress.



%% The following is a directive for TeXShop to indicate the main file
%%!TEX root = diss.tex

\chapter{Goormaghtigh Equations}
\label{ch:Goormaghtigh} 

Let $m$ and $n$ be integers such that $m > n > 2$, where either $m=n+1$ or
 \begin{equation} \label{condition}
 \gcd (m-1,n-1) = d > 1m
 \end{equation}
 and consider the Goormaghtigh's equation
\begin{equation} \label{eq-main}
\frac{x^m-1}{x-1} = \frac{y^n-1}{y-1}, \; \; y>x>1, \; m > n > 2.
\end{equation}
 
In this chapter, we prove that, in fact, under assumption (\ref{condition}), equation (\ref{eq-main}) has at most finitely many solutions which may be found effectively, even if we fix only a single exponent.
\begin{theorem} \label{main-thm}
If there is a solution in integers $x,y, n$ and $m$ to equation (\ref{eq-main}), satisfying (\ref{condition}), then
\begin{equation} \label{hive}
x <  (3d)^{4n/d} \leq  36^n.
\end{equation}
In particular, if $n$ is fixed, there is an effectively computable constant $c=c(n)$ such that
$\max \{ x, y, m \} < c$.
\end{theorem}
We note that the latter conclusion here follows immediately from (\ref{hive}), in conjunction with, for example, work of Baker \cite{Bak}.
The constants present in our upper bound (\ref{hive}) may be sharpened somewhat at the cost of increasing the complexity of our argument. By refining our approach, in conjunction with some new results from computational Diophantine approximation, we are able to achieve the complete solution of equation (\ref{eq-main}), subject to condition (\ref{condition}),  for small fixed values of $n$.

\begin{theorem} \label{main-thm2}
If there is a solution in integers $x,y$ and $m$ to equation (\ref{eq-main}), with $n \in \{ 3, 4, 5 \}$ and satisfying (\ref{condition}), then
$$
(x,y,m,n) = (2,5,5,3)  \; \mbox{ and } \; (2,90,13,3).
$$
\end{theorem}

Essentially half of the current chapter is concerned with developing Diophantine approximation machinery for the case $n=5$ in Theorem \ref{main-thm2}. Here, ``off-the-shelf'' techniques for finding integral points on models of elliptic curves or for solving {\it Ramanujan-Nagell} equations of the shape $F(x)=z^n$ (where $F$ is a polynomial and $z$ a fixed integer) do not apparently permit the full resolution of this problem in a reasonable amount of time. The new ideas introduced here are explored more fully in the general setting of {\it Thue-Mahler} equations in the forthcoming paper  \cite{GhKaMaSi}. These are polynomial-exponential equations of the form $F(x,y)=p_1^{\alpha_1} \cdots p_k^{\alpha_k}$ where $F$ is a binary form of degree three or greater and $p_1, \ldots, p_k$ are fixed rational primes. Here, we take this opportunity to specialize these refinements to the case of Ramanujan-Nagell equations, and to introduce some further sharpenings which enable us to complete the proof of Theorem \ref{main-thm2}.

%--------------------------------------------------------------
\section{Rational approximations} \label{proof}
%--------------------------------------------------------------

In what follows, we will always assume that $x, y, m$ and $n$ are integers satisfying (\ref{eq-main}) with (\ref{condition}), and write
\begin{equation} \label{m0n0}
 m-1 = d m_0 \; \; \mbox{ and } \; \; n-1 = d n_0.
 \end{equation}
 We note, for future use, that  an appeal to Th\'eor\`eme II of Karanicoloff \cite{Ka} (which, in our notation, states that the only solution to (\ref{eq-main}) with $n_0=1$ and $m_0=2$ in (\ref{m0n0}) is given by $(x,y,m,n) = (2,5,5,3)$) allows us to suppose that either $(x,y,m,n) = (2,5,5,3)$, or that  $m_0 \geq 3$ and $n_0 \geq 1$.

Our starting point, as in, for example, \cite{BuSh} and \cite{NeSh}, is the observation that the existence of a solution to (\ref{eq-main}) with (\ref{condition}) implies a number of unusually good rational approximations to certain irrational algebraic numbers. One such approximation arises from rewriting (\ref{eq-main}) as 
$$
x \, \frac{x^{d m_0}}{x-1} - y \, \frac{y^{d n_0}}{y-1} = \frac{1}{x-1} - \frac{1}{y-1},
$$
whereby
\begin{equation} \label{good}
\left| \sqrt[d]{\frac{y(x-1)}{x(y-1)}} - \frac{x^{m_0}}{y^{n_0}} \right| < \frac{1}{y^{d n_0}}.
\end{equation}
The latter inequality was used, in conjunction with lower bounds for linear forms in logarithms (in \cite{NeSh}) and with machinery based upon Pad\'e approximation to binomial functions (in \cite{BuSh}), to derive a number of strong restrictions upon $x, y$ and $d$ satisfying equation (\ref{eq-main}).

Our argument will be somewhat different, as we consider instead a rational approximation to 
$\sqrt[d]{(x-1)/x}$ that is, on the surface, much less impressive than that to $\sqrt[d]{\frac{y(x-1)}{x(y-1)}}$ afforded by (\ref{good}). The key additional idea is that we are able to take advantage of the arithmetic structure of our approximations  to obtain very strong lower bounds for how well they can approximate $\sqrt[d]{(x-1)/x}$. This argument has its genesis in work of Beukers \cite{Beu1}, \cite{Beu2}.

For the remainder of this section, we will always assume that $x \geq 40$. From
 \[\frac{y^n-1}{y-1} = y^{dn_0} \left( 1 + \frac{1}{y} +  \cdots + \frac{1}{y^{dn_0}} \right)\]
 and
 \[\frac{x^m-1}{x-1} = x^{dm_0} \left( 1 + \frac{1}{x} + \cdots + \frac{1}{x^{dm_0}} \right),\]

 we thus have
 $$
y^{dn_0}  <  \frac{y^n-1}{y-1} =  \frac{x^m-1}{x-1}  < \frac{x}{x-1} \,   x^{dm_0}
$$
and
$$
\frac{y}{y-1} \, x^{dm_0} \leq \frac{x+1}{x}  \, x^{dm_0} <  \frac{x^m-1}{x-1}  =  \frac{y^n-1}{y-1} 
< \frac{y}{y-1} \, y^{dn_0},
$$
so that 
\begin{equation} \label{goop}
x^{m_0} < y^{n_0} <\left( \frac{x}{x-1} \right)^{1/d} \, x^{m_0} \leq \sqrt{40/39} \, x^{m_0} < 1.013 \, x^{m_0}.
\end{equation}


We will rewrite  (\ref{eq-main})
 as 
 $$
 x^{d m_0} -  \frac{(x-1)}{x} \;   \sum_{j=0}^{dn_0} y^j = \frac{1}{x}.
 $$
 From this equation, we will show that $\sqrt[d]{(x-1)/x}$ is well approximated by a rational number whose numerator is divisible by $x^{m_0}$.
 
If we define, as in Nesterenko and Shorey \cite{NeSh},  $A_k (d)$ via
% $$
% \left( 1 + \frac{1}{X} + \cdots + \frac{1}{X^{\mu-1}} \right)^{1/d} = \sum_{k=0}^\infty \frac{A_k (\mu,d)}{ X^{k}}
% $$
% and
 $$
  \left( 1 - \frac{1}{X} \right)^{-1/d} = \sum_{k=0}^\infty A_k (d) \, X^{-k} = \sum_{k=0}^\infty
  \frac{d^{-1} ( d^{-1}+1) \cdots (d^{-1} + k-1)}{k!} \, X^{-k},
 $$
 then we can write 
 $$
 \sum_{j=0}^{dn_0} y^j = \left( \sum_{k = 0}^{n_0} A_k (d) y^{n_0-k} \right)^d + \sum_{j=0}^{(d-1)n_0-1} B_j(d) y^j.
 $$
Here, the $B_j$ are positive, monotone increasing in $j$, and satisfy
 $$
 B_{(d-1)n_0-1} (d) = \frac{n}{n_0+1} A_{n_0} (d),
 $$
 while, for the $A_k(d)$, we have the inequalities
 $$
 \frac{d+1}{k d^2} \leq A_k (d) \leq  \frac{d+1}{2 d^2},
 $$
valid  provided $k \geq 2$ (see displayed equation (14) of \cite{NeSh}). 
%Note that
% $$
% A_{k+1} (d) = \frac{d^{-1}+k}{1+k} \, A_k (d).
 %$$
 
 We thus have
\begin{equation} \label{careful}
 x^{d m_0} -  \frac{(x-1)}{x} \; \left( \sum_{k = 0}^{n_0} A_k (d) y^{n_0-k} \right)^d = \frac{1}{x} + \frac{x-1}{x} \, \sum_{j=0}^{(d-1)n_0-1} B_j(d) y^j
 \end{equation}
 and so
 \begin{equation} \label{careful-upper}
 0 <  x^{d m_0} -  \frac{(x-1)}{x} \; \left( \sum_{k = 0}^{n_0} A_k (d) y^{n_0-k} \right)^d <  \frac{(dn_0+1)(d+1)}{2 (n_0+1) d^2} \, \frac{y}{y-1} \, y^{(d-1) n_0-1}.
 \end{equation}
 Since
 $$
 \frac{(dn_0+1)(d+1)}{2 (n_0+1) d^2} < \frac{d+1}{2d} \leq \frac{3}{4},
 $$
from the fact that $n_0 \geq 1$ and $d \geq 2$, and since $y > x \geq 40$, we may conclude that
  \begin{equation} \label{careful-upper2}
 0 <  x^{d m_0} -  \frac{(x-1)}{x} \; \left( \sum_{k = 0}^{n_0} A_k (d) y^{n_0-k} \right)^d < 0.769 \, y^{(d-1) n_0-1}.
 \end{equation}
 Applying the Mean Value Theorem,
  \begin{equation} \label{careful-upper3}
 0 <   x^{m_0} -  \sqrt[d]{\frac{x-1}{x}} \, \sum_{k = 0}^{n_0} A_k (d) y^{n_0-k}  <  0.769 \,  \frac{y^{(d-1) n_0-1}}{d Y^{d-1}},
  \end{equation}
  where $Y$ lies in the interval
  $$
  \left( \sqrt[d]{\frac{x-1}{x}} \, \sum_{k = 0}^{n_0} A_k (d) y^{n_0-k}, x^{m_0} \right).
  $$
 We thus have
 $$
 Y^{d-1} >  \left( \frac{x-1}{x} \right)^{(d-1)/d} \, y^{(d-1) n_0}
 $$
 and so, from (\ref{careful-upper3}) and the fact that $d \geq 2$ and $x \geq 40$, 
   \begin{equation} \label{careful-upper4}
 0 <   x^{m_0} -  \sqrt[d]{\frac{x-1}{x}} \, \sum_{k = 0}^{n_0} A_k (d) y^{n_0-k}  <  \frac{0.779}{d y}.
  \end{equation}
  
Let us define
 $$
 C(k,d) = d^k \prod_{p \mid d} p^{\mbox{ord}_p (k!)},
 $$
 where by $\mbox{ord}_p(z)$ we mean the largest power of $p$ that divides a nonzero integer $z$. Here,
 $k$  and $d$ positive integers with $d \geq 2$. Then we have
$$
C(k,d) =  d^k \prod_{p \mid d} p^{\left[ \frac{k}{p} \right] + \left[ \frac{k}{p^2} \right] + \cdots}
$$
and hence it follows that
\begin{equation} \label{C-upper}
C(k,d) < \left( d \,  \prod_{p \mid d} p^{1/(p-1)} \right)^k.
\end{equation}
Further (see displayed equation (18) of Nesterenko and Shorey \cite{NeSh}), and critically for our purposes, $C(k,d) A_k (d)$ is an integer. Multiplying equation (\ref{careful}) by $C(n_0,d)$ and setting 
\begin{equation} \label{P-Q-def}
 P = C(n_0,d) \,  x^{m_0} \; \; \mbox{ and } \; \; Q = C(n_0,d) \,  \sum_{k = 0}^{n_0} A_k (d) y^{n_0-k},
\end{equation}
 then $P$ and $Q$ are integers  and, defining
\begin{equation} \label{ep-def}
\epsilon = P - \sqrt[d]{\frac{x-1}{x}} Q,
\end{equation}
 we thus have, from (\ref{careful-upper4}), that the following result holds.
 \begin{proposition} \label{upper-ep}
 Suppose that $(x,y,m,n)$ is a solution in integers to equation (\ref{eq-main}), with (\ref{condition}) and $x \geq 40$. If we define $\epsilon$ via (\ref{ep-def}),  then
\begin{equation} \label{start}
0 < \epsilon < \frac{0.779 \, C(n_0,d)}{dy}.
\end{equation}
 \end{proposition}
 
Our next goal will be to construct a second linear form $\delta$, in $1$ and $\sqrt[d]{(x-1)/x}$, with the property that a particular linear combination of $\epsilon$ and $\delta$ is a (relatively large) nonzero integer, a fact we will use to derive a lower bound on $\epsilon$. This argument, which will employ off-diagonal Pad\'e approximants to the binomial function $\sqrt[d]{1+z}$, follows work of Beukers \cite{Beu1}, \cite{Beu2}.

To apply Proposition \ref{upper-ep} and for our future arguments, we will have use of bounds upon the quantity $C(k,d)$. 
 \begin{proposition} \label{Cee}
If $k$ is a positive integer, then
$$
2^k \leq C(k,2) < 4^k
$$
and
$$
d^k \leq C(k,d) < (2 d \log d)^k,
$$
for $d > 2$.
\end{proposition}
We will postpone the proof of this result until Section \ref{Cee-proof}; the upper bound here for large $d$ may be sharpened somewhat, but this is unimportant for our purposes.

%---------------------------------------------------------
\section{Pad\'e approximants} \label{Pade}
%--------------------------------------------------------


In this section, we will define Pad\'e approximants to $(1+z)^{1/d}$, for $d \geq 2$.
Suppose that $m_1$ and $m_2$ are nonnegative integers, and set
$$
I_{m_1,m_2}(z) = \frac{1}{2 \pi i} \, \int_{\gamma} ~
\frac{(1+zv)^{m_2}(1+zv)^{1/d}}{v^{m_1+1} (1-v)^{m_2+1}} \, dv,
$$
where $\gamma$ is a closed, counter-clockwise contour, containing $v=0$ and $v=1$. Applying Cauchy's residue theorem, we may write $I_{m_1,m_2}(z)$ as $R_0+R_1$, where
$$
R_i = \mbox{Res}_{v=i} \left( \frac{(1+zv)^{m_2}(1+zv)^{1/d}}{v^{m_1+1} (1-v)^{m_2+1}} \right). 
$$
Now
$$
R_0=
\frac{1}{m_1!} \, \lim_{v \rightarrow 0} \frac{d^{m_1}}{dv^{m_1}} \, \frac{(1+zv)^{m_2}(1+zv)^{1/d}}{(1-v)^{m_2+1}} = P_{m_1,m_2} (z) 
$$
and
$$
R_1=
\frac{1}{m_2!} \, \lim_{v \rightarrow 1} \frac{d^{m_2}}{dv^{m_2}} \, \frac{(1+zv)^{m_2}(1+zv)^{1/d}}{v^{m_1+1}} =  - Q_{m_1,m_2} (z) \, (1+z)^{1/d},
$$
where
\begin{equation}  
P_{m_1,m_2} (z) = \sum_{k=0}^{m_1} \binom{m_2 + 1/d}{k} \binom{m_1+m_2-k}{m_2} z^k
\end{equation}
and
\begin{equation} \label{queu}
Q_{m_1,m_2} (z) = \sum_{k=0}^{m_2} \binom{m_1 - 1/d}{k} \binom{m_1+m_2-k}{m_1} z^k.
\end{equation}
Note that there are typographical errors in the analogous statement given in displayed equation (2.3) of \cite{BaBe}.
We take $z=-1/x$. Arguing as in the proof of Lemma 4.1 of \cite{BaBe}, we find that
\begin{equation} \label{del0}
|I_{m_1,m_2}(-1/x)| = \frac{\sin (\pi/d)}{\pi \,  x^{m_1+m_2+1}} \;  \int^{1}_{0} ~
\frac{v^{m_2+1/d} (1-v)^{m_1-1/d} dv}{(1-(1-v)/x)^{m_2+1}}.
\end{equation}
Upon multiplying the identity
$$
P_{m_1, m_2}(-1/x)-Q_{m_1, m_2}(-1/x)   \sqrt[d]{\frac{x-1}{x}} =I_{m_1,m_2}(-1/x) 
$$
through by $x^{m_2} C(m_2,d)$,
and setting
$$
\delta = C_0 P_1 -  \sqrt[d]{\frac{x-1}{x}} \; Q_1,
$$
where we write $m_0=m_2-m_1$,
$$
C_0 = x^{m_0} C(m_2,d)/C(m_1,d), \; \; P_1 = x^{m_1} C(m_1,d) P_{m_1,m_2}(-1/x)
$$
and
\begin{equation} \label{Q1-def}
Q_1=x^{m_2} C(m_2,d) Q_{m_1,m_2}(-1/x),
\end{equation}
it follows, from Lemma 3.1 of Chudnovsky \cite{Chud},
that $C_0, P_1$ and $Q_1$ are integers. Further, from (\ref{del0}), 
\begin{equation} \label{del}
|\delta| = \frac{\sin (\pi/d) \, C(m_2,d)}{\pi \,  x^{m_1+1}} \;  \int^{1}_{0} ~
\frac{v^{m_2+1/d} (1-v)^{m_1-1/d} dv}{(1-(1-v)/x)^{m_2+1}}.
\end{equation}

%Then (see e.g. ), we have that
%\begin{equation} \label{aye}
% P_{m_1,m_2} (z) - \left( 1+z \right)^{1/d} \; Q_{m_1,m_2} (z) = z^{m_1+m_2+1} \, E_{m_1,m_2} (z),
%\end{equation}
%where 
%\begin{equation} \label{frump}
%E_{m_1,m_2} (z) =   \frac{(-1)^{m_2} \, \Gamma (m_2+(d+1)/d)}{\Gamma(-m_1+1/d) \Gamma(m_1+m_2+2)}  F(m_1 + (d-1)/d,m_2+1, m_1+m_2+2,-z),
%\end{equation}
%for $F$ the hypergeometric function given by
%$$
%F(a,b,c,-z) = 1 - \frac{a \cdot b}{1 \cdot c} z + \frac{a \cdot (a+1) \cdot b \cdot (b+1)}{1 \cdot 2 \cdot c \cdot (c+1)} z^2 - \cdots.
%$$

Recall that $P$ and $Q$ are defined as in (\ref{P-Q-def}). Here and henceforth, we will assume that 
\begin{equation} \label{sturdy}
m_2-m_1=m_0. 
\end{equation}
We have
\begin{lemma} \label{pumpkin}
If  $m_1$ and $m_2$ are nonnegative integers satisfying (\ref{sturdy}),  then it follows that $PQ_1 \neq C_0 P_1 Q$.
\end{lemma}

\begin{proof}
Let $p$ be a prime with $p \mid d$. 
Then
$$
\mbox{ord}_p (P) = n_0 \, \mbox{ord}_p (d)  + \mbox{ord}_p (n_0!) + m_0 \, \mbox{ord}_p (x),
$$
$$
\mbox{ord}_p (P_1) = \mbox{ord}_p (Q_1) = \mbox{ord}_p (Q) = 0
$$
and
$$
\mbox{ord}_p (C_0) = m_0 \, \mbox{ord}_p (d)  + \mbox{ord}_p (m_2!)  - \mbox{ord}_p (m_1!) + m_0 \, \mbox{ord}_p (x).
$$
Since $m_2 - m_1 = m_0 > n_0$, 
we have
$$
\mbox{ord}_p \left( \frac{C_0P_1 Q}{P Q_1} \right) = (m_0-n_0)  \, \mbox{ord}_p (d)  + \mbox{ord}_p \left( \frac{m_2!}{m_1! n_0!} \right) > 0
$$
so that 
$$
\mbox{ord}_p (P Q_1 - C_0 P_1 Q) = \mbox{ord}_p (P Q_1) = n_0 \, \mbox{ord}_p (d)  + \mbox{ord}_p (n_0!) + m_0 \, \mbox{ord}_p (x)
$$
and, in particular, $P Q_1 - C_0 P_1 Q \neq 0$.

\end{proof}

%Appealing twice to  (\ref{aye}) and  (\ref{frump}) with, in the second instance,  $m_2$ replaced by $m_2+1$, and eliminating $(1+z)^{1/q}$,  we find that $P_{m_1,m_2+1}(z)Q_{m_1,m_2}(z)-P_{m_1,m_2}(z)Q_{m_1,m_2+1}(z)$ is a polynomial of degree $m_1+m_2+1$ with a zero at $z=0$ of order $m_1+m_2+1$ (and hence monomial). It follows that we may write
%\begin{equation} \label{zero}
%P_{m_1,m_2+1}(z)Q_{m_1,m_2}(z)-P_{m_1,m_2}(z)Q_{m_1,m_2+1}(z) = c z^{m_1+m_2+1},
%\end{equation}
%with, a short calculation reveals, $c \neq 0$.

It follows from Lemma \ref{pumpkin}
and its proof  that $P Q_1 - C_0 P_1 Q$ is a nonzero integer multiple of 
$C(n_0,d) \, x^{m_0}$, so that, from the definitions of $\epsilon$ and $\delta$, 
\begin{equation} \label{key}
|\epsilon Q_1 - \delta Q| = |P Q_1 - C_0 P_1 Q| \geq  C(n_0,d) \, x^{m_0}.
\end{equation}

Now 
$$
Q = C(n_0,d) \,  \sum_{k = 0}^{n_0} A_k (d) y^{n_0-k} < \frac{y}{y-1} \, C(n_0,d) \, y^{n_0} \leq 1.025 \, C(n_0,d) \, y^{n_0},
$$
since $y > x \geq 40$, and hence,
from (\ref{goop}), 
\begin{equation} \label{Q-upper}
Q < 1.039  \, C(n_0,d) \, x^{m_0}.
\end{equation}

Combining (\ref{goop}), (\ref{start}), (\ref{key}) and (\ref{Q-upper}), we thus have
 \begin{proposition} \label{upper-cool}
 Suppose that $(x,y,m,n)$ is a solution in integers to equation (\ref{eq-main}), with (\ref{condition}) and $x \geq 40$. If $m_0, n_0$ and $d$ are defined as in (\ref{m0n0}), and $m_1$ and $m_2$ are nonnegative integers satisfying (\ref{sturdy}), then for $Q_1$ and $|\delta|$ as given  in (\ref{Q1-def}) and (\ref{del}), we may conclude that
 \begin{equation} \label{imp}
  |Q_1| > 1.28 \, d \, (1-1.039 |\delta|) \, x^{m_0+m_0/n_0}.
\end{equation}
 \end{proposition}

In the other direction, we will deduce two upper bounds upon $|Q_1|$; we will use one or the other depending on whether or not $m_1$ is ``large'', relative to $x$. The first result is valid for all choices of $x$.
\begin{proposition} \label{shazam2}
If $m_1, m_2$ and $x$ are integers with $m_2 > m_1 \geq 1$ and $x \geq 2$, define $\alpha = m_2/m_1$ and $|\delta|$ as in (\ref{del}).
Then
\begin{equation} \label{munchkin}
|Q_1| < \sqrt[d]{\frac{x}{x-1}} \; \left( \frac{(\alpha+1)^2}{\alpha} \, (e \, (\alpha + 1))^{m_1}  \, x^{m_2} \, C(m_2,d)
+ |\delta| \right).
\end{equation}
\end{proposition}


If $x \geq m_1$, we will have use of the following slightly sharper bound.
 \begin{proposition} \label{shazam}
If $m_1$ and $m_2$ are integers with $m_2 > m_1 \geq 0$ and $x \geq \frac{m_1m_2}{m_1+m_2}$, then
$$
|Q_1| < \frac{x}{x-1} \, \binom{m_1+m_2}{m_1}  \, C(m_2,d) \, x^{m_2}.
$$
\end{proposition}

\begin{proof}[Proof of Proposition \ref{shazam2}]
Let us write $\alpha = m_2/m_1>1$
and define
\begin{equation} \label{arrr}
r (\alpha,  u) =
\frac{1}{2u} \left( (\alpha + 1) - (\alpha - 1) u - \sqrt{\left((\alpha + 1) - (\alpha - 1) u \right)^2 - 4 u} \right),
\end{equation}
and
\begin{equation} \label{emm}
M(\alpha,x) =  \frac{(1-r(\alpha,1/x)/x)^{\alpha}}{(1-r(\alpha,1/x))^{\alpha} r(\alpha,1/x)}.
\end{equation}

Via the Mean Value Theorem, 
\begin{equation} \label{upper-al}
 \frac{1}{\alpha +1}  < r(\alpha,1/x) < \frac{x}{(x-1)(\alpha +1)}
\end{equation}
and so, from calculus,
\begin{equation} \label{upper-emm}
M(\alpha,x) < \left( \frac{(x-1) \, (\alpha+1)-1}{(x-1) (\alpha +1)-x} \right)^\alpha \cdot (\alpha+1) < e \, (\alpha +1)
\end{equation}
and
\begin{equation} \label{M-lower}
M(\alpha,x) > \left( 1 + \frac{x-1}{x \alpha} \right)^\alpha \, \left( \frac{x-1}{x} \right) \; (\alpha+1).
\end{equation}

Arguing as in the proof of Lemma 3.1 of \cite{BaBe}, we find that 
$$
|C_0 P_1| \leq \frac{\left( 1-r(\alpha,1/x)/x \right)^{1/d}}{r(\alpha, 1/x) (1 - r(\alpha,1/x))} \, M(\alpha,x)^{m_1} \, x^{m_2} \, C(m_2,d),
$$
whereby inequalities (\ref{upper-al}) and  (\ref{upper-emm}) imply that
$$
|C_0 P_1| < \frac{(\alpha+1)^2}{\alpha} \, (e \, (\alpha + 1))^{m_1}  \, x^{m_2} \, C(m_2,d).
$$
Since $C_0 P_1 = \sqrt[d]{\frac{x-1}{x}} \; Q_1 + \delta$, we conclude as desired.
\end{proof}

\begin{proof}[Proof of Proposition \ref{shazam}]
To bound $Q_1$ from above, we begin by noting that
\begin{equation} \label{naslund}
x^{m_2} \left| Q_{m_1,m_2} (-1/x) \right| = \left| \sum_{k=0}^{m_2} \binom{m_1 - 1/d}{k} \binom{m_1+m_2-k}{m_1} (-1)^k x^{m_2-k} \right|.
\end{equation}
Defining
$$
f (k) = \binom{m_1 - 1/d}{k} \binom{m_1+m_2-k}{m_1},
$$
it follows that, for $0 \leq k \leq m_2-1$,
$$
f (k+1)/f(k) = \frac{(m_1 -1/d-k)(m_2-k)}{(k+1)(m_1+m_2-k)}.
$$
If $k \leq m_1-1$, we thus have that
\begin{equation} \label{bound1}
0 < f (k+1)/f(k) < \frac{(m_1 -k)(m_2-k)}{(k+1)(m_1+m_2-k)} \leq \frac{m_1m_2}{m_1+m_2}.
\end{equation}
If instead $k \geq m_1$, 
\begin{equation} \label{bound2}
\frac{(m_1 -k-1)(m_2-k)}{(k+1)(m_1+m_2-k)}  < f (k+1)/f(k) < 0.
\end{equation}
It follows via calculus, in this case, that
$$
|f(k+1)/f(k)| < \frac{(m_2-m_1+1)^2}{(m_2+m_1+1)^2}.
$$
We thus have that $x^{m_2} \left| Q_{m_1,m_2} (-1/x) \right|$ is bounded above by
$$
 \binom{m_1+m_2}{m_1}  x^{m_2} +  \left| \binom{m_1 - 1/d}{m_1} \right| \binom{m_2}{m_1}  \sum_{k=m_1+1}^{m_2}  x^{m_2-k}
$$
which implies the desired result. 
\end{proof}


%------------------------------------------------------
\section{Proof of Theorem \ref{main-thm} } \label{sec-main-thm}
%------------------------------------------------------

To prove Theorem \ref{main-thm}, we will work with Pad\'e approximants to $(1+z)^{1/d}$, as in Section \ref{Pade},  of degrees $m_1$ and $m_2$ where we choose
\begin{equation} \label{choice}
m_1 = \left[ \frac{m_0}{2 n_0} \right] \; \; \mbox{ and } \; \; m_2 = m_0 + \left[ \frac{m_0}{2 n_0} \right],
\end{equation}
for $m_0, n_0$ and $d$ as given in (\ref{m0n0}). Here $[x]$ denotes the greatest integer less than or equal to $x$.
Let us assume further that $x \geq (3d)^{4n/d} \geq 6^6$. 
We will make somewhat different choices later, when we prove Theorem \ref{main-thm2}. 

Our strategy will be as follows. We begin by showing that 
 $\delta$ as given in (\ref{del}) satisfies $|\delta| < \frac{1}{1.039}$, so that the lower bound upon $|Q_1|$ in Proposition \ref{upper-cool} is nontrivial. From there, we will appeal to Proposition \ref{shazam2}  to contradict Proposition \ref{upper-cool}.
 
%----------------------------------
\subsection{Bounding $\delta$}
%----------------------------------

From the aforementioned Th\'eor\`eme II of Karanicoloff \cite{Ka}, we may suppose that $m_0 \geq 3$ and hence, arguing crudely, since $m_2 \geq m_0 \geq 3$ and $m_1 \geq 0$, we have
$$
 \int^{1}_{0} ~
\frac{v^{m_2+1/d} (1-v)^{m_1-1/d} dv}{(1-(1-v)/x)^{m_2+1}} < 1
$$
and hence, from (\ref{del}), 
\begin{equation} \label{frog}
|\delta| < \frac{\sin (\pi/d) \, C(m_2,d)}{\pi \,  x^{m_1+1}} \leq \frac{C(m_2,d)}{\pi \,  x^{m_1+1}}.
\end{equation}
 
From (\ref{choice}),  $m_1+1 > \frac{m_0}{2 n_0}$ and so, the  assumption that $x \geq (3d)^{4n/d}$ yields the inequality 
$$
x^{m_1+1} >  (3d)^{2 m_0}. 
$$
Applying Proposition \ref{Cee}, if $d=2$, it follows from $m_1 \leq \frac{m_0}{2n_0}$ that 
$$
|\delta| <  \frac{1}{\pi} \, 4^{m_1} \, 3^{-2m_0} \leq \frac{8}{729 \pi} < 0.01,
$$
since $m_0 \geq 3$ and $n_0 \geq 1$.
Similarly, if $d \geq 3$, 
$$
|\delta| <  \frac{(2 d \log d)^{m_0+m_1}}{(3d)^{2m_0}} \leq \frac{(2 d \log d)^{m_0+\frac{m_0}{2n_0}}}{(3d)^{2m_0}}
= \left( \frac{(2 d \log d)^{1+\frac{1}{2n_0}}}{9d^{2}} \right)^{m_0} < 0.01,
$$
again from $m_0 \geq 3$ and $n_0 \geq 1$.
Appealing to Proposition \ref{upper-cool}, we thus have, in either case,
\begin{equation} \label{moving}
 |Q_1| > 1.25 \, d \, x^{m_0+m_0/n_0}.
\end{equation}


%-------------------------------------------------------------
\subsection{Applying Proposition \ref{shazam2}}
%-------------------------------------------------------------

We will next apply Proposition \ref{shazam2} to deduce an upper bound upon $|Q_1|$. To use this result, we must first separately treat the case when $m_1=0$. In this situation, 
Proposition \ref{shazam} implies that
$$
|Q_1| < \frac{x}{x-1}   \, C(m_0,d) \, x^{m_0}.
$$
Inequality (\ref{moving}) and $x \geq (3d)^{4n/d} > (3d)^{4n_0}$ thus lead to the inequalities 
$$
C(m_0,d) > d \, x^{m_0/n_0} > (3d)^{4 m_0},
$$
contradicting Proposition \ref{Cee} in all cases.

Assuming now that $m_1 \geq 1$, combining  Proposition \ref{shazam2} with (\ref{moving}), $d \geq 2$ and the fact that $\alpha = 1+m_0/m_1 \geq 3$, implies that
$$
x^{\frac{m_0}{n_0} - m_1} <  \alpha \, C(m_2,d) \, (e \, (\alpha+1))^{m_1}.
$$
Since $m_1 \leq m_0/2n_0$, $x \geq (3d)^{4n/d} > (3d)^{4n_0}$ and $\alpha = 1+m_0/m_1$, it follows that
$$
(3d)^{2m_0} < (1 + m_0/m_1) \,  C(m_0+m_1,d) \,   (e \, (2 + m_0/m_1))^{m_1}
$$
and so
\begin{equation} \label{fish}
9 d^2 < (1 + m_0/m_1)^{1/m_0}  \,  C(m_0+m_1,d)^{1/m_0} \,   (e \, (2 + m_0/m_1))^{m_1/m_0}.
\end{equation}
If $d=2$, Proposition \ref{Cee} yields
\begin{equation} \label{fleece}
36 < (1 + m_0/m_1)^{1/m_0}  \,  4^{1+m_1/m_0} \,   (e \, (2 + m_0/m_1))^{m_1/m_0},
\end{equation}
contradicting  the fact that $m_0 \geq \max \{ 3, 2m_1 \}$.

If $d \geq 3$, (\ref{fish}) and Proposition \ref{Cee}  lead to the inequality
$$
9 d^2 < (1 + m_0/m_1)^{1/m_0}  \,  (2 d \log d)^{1+m_1/m_0} \,   (e \, (2 + m_0/m_1))^{m_1/m_0},
$$
whence
\begin{equation} \label{bluff}
2.744 < \frac{9 \sqrt{d}}{2 \sqrt{2} (\log d)^{3/2}} < (1 + m_0/m_1)^{1/m_0}  \,  (e \, (2 + m_0/m_1))^{m_1/m_0}.
\end{equation}
If $n_0 \geq 3$, then $m_0 \geq 6 m_1$ and hence 
$$
(1 + m_0/m_1)^{1/m_0}  \,  (e \, (2 + m_0/m_1))^{m_1/m_0} < 2.4,
$$
a contradiction, while, from the second inequality in (\ref{bluff}), we find that  $d \leq 1112$ or $d \leq 64$, if $n_0=1$ or $n_0=2$, respectively. 

For these remaining values, we will argue somewhat more carefully. From (\ref{C-upper}) and (\ref{fish}), 
\begin{equation} \label{flag}
9 d^2 < (1 + m_0/m_1)^{1/m_0}  \,  \left( d \,  \prod_{p \mid d} p^{1/(p-1)} \right)^{1+m_1/m_0} \,   (e \, (2 + m_0/m_1))^{m_1/m_0}.
\end{equation}
If $n_0=2$ (so that $m_0 \geq 4 m_1$), we thus have
$$
d^{3/4} <  0.34 \, \left( \prod_{p \mid d} p^{1/(p-1)} \right)^{5/4},
$$
and hence, for $3 \leq d \leq 64$, a contradiction. Similarly, if $n_0=1$, we have from $m_0 \geq 3$ that either $(m_0,m_1)=(3,1)$ or $m_0 \geq 4$. In the first case,
$$
d^{2/3} <  0.43 \, \left( \prod_{p \mid d} p^{1/(p-1)} \right)^{4/3},
$$
contradicting the fact that $d \leq 1112$. If $m_0 \geq 4$ (so that $m_1 \geq 2$), then (\ref{flag}) implies the inequality
$$
d^{1/2} <  \frac{e^{1/2} \cdot 2 \cdot 3^{1/2m_1}}{9}  \, \left( \prod_{p \mid d} p^{1/(p-1)} \right)^{3/2}
$$
and hence, after a short computation and using that $d \le 1112$, either $d=6$, $m_0=2m_1$ and $m_1 \leq 15$, or $d=30$ and $(m_0,m_1)=(4,2)$.
In this last case, 
$$
x^6 Q_{2,6} (-1/x) = \sum_{k=0}^{6} \binom{2 - 1/30}{k} \binom{8-k}{2} (-x)^{6-k}
$$
and so $x^6 Q_{2,6} (-1/x)$ is equal to 
$$
28 x^6-\frac{413}{10} x^5+ \frac{1711}{120} x^4+ \frac{1711}{16200} x^3+\frac{53041}{3240000} x^2+\frac{3235501}{972000000} x + \frac{294430591}{524880000000}
< 28 x^6,
$$
since $x \geq 6^6$. 
From $C(6,30)=52488000000$, we have that
$$
|Q_1| < 1.47 \cdot 10^{13} \, x^6.
$$
On the other hand,  (\ref{moving}) implies that $|Q_1| > 37.5 \cdot x^8$, so that $x < 6.3 \cdot 10^5$,
contradicting $x \geq (3d)^{4n/d} > 90^{4}$.

For $d=6$, $2 \leq m_1 \leq 15$ and $m_0=2m_1$, we argue in a similar fashion, explicitly computing $Q_{m_1,m_2} (z)$ and finding that
$$
|Q_1| < \kappa_{m_1} x^{3m_1},
$$
where
$$
\begin{array}{cc|cc|cc}
m_1 & \kappa_{m_1} & m_1 & \kappa_{m_1}  &  m_1 & \kappa_{m_1} \\ \hline
2 & 1.89 \cdot 10^8 & 7 & 1.35 \cdot 10^{32} & 12 & 1.60 \cdot 10^{57}  \\
3 & 2.30 \cdot 10^{13} & 8 & 1.24 \cdot 10^{37} &  13 & 1.89 \cdot 10^{61} \\ 
4 & 9.86 \cdot 10^{17} & 9 & 1.29 \cdot 10^{42} & 14 & 1.79 \cdot 10^{66}  \\ 
5 & 1.09 \cdot 10^{22} & 10 & 6.02 \cdot 10^{46}  & 15 & 1.28 \cdot 10^{71} \\
6 & 5.88 \cdot 10^{27} & 11 &  1.13 \cdot 10^{52} & &  \\
\end{array}
$$
With (\ref{moving}), we thus have
$$
x^{m_1} <  \frac{2}{15} \, \kappa_{m_1},
$$
and so
$$
x < \left( \frac{2}{15} \, \kappa_{m_1} \right)^{1/m_1} < 5.5 \cdot 10^4,
$$
contradicting our assumption that $x \geq 18^{2n/3} \geq 18^{14/3} > 7.2 \cdot 10^5$. This completes the proof of Theorem \ref{main-thm}.


%------------------------------------------------------
\section{Proof of Theorem \ref{main-thm2} for $x$ of moderate size }  \label{sec-main-thm2}
%------------------------------------------------------

As can be observed from the proof of Theorem \ref{main-thm}, the upper bound $x <  (3d)^{4n/d}$ may, for fixed values of $n$ (and hence $d$), be improved with a somewhat more careful argument. By way of example, for small choices of $n$, we may derive bounds of the shape $x < x_0(n)$, provided we assume that $m \geq m_0 (n)$ for effectively computable $m_0$, where we have
$$
\begin{array}{cc|cc|cc|cc} \hline
n & x_0(n) & n & x_0(n) & n & x_0(n) & n & x_0(n) \\ \hline
3 & 38 & 5 & 676 & 7 & 11647 & 9 & 195712 \\
4 & 80  & 6 & 230 & 8 & 492 & 10 &  72043. \\
\end{array}
$$
To prove Theorem \ref{main-thm2}, we will begin by deducing slightly weaker versions of these bounds, for $n \in \{ 3, 4, 5 \}$, where the corresponding values $m_0$ are amenable to explicit computation. Our arguments will closely resemble those of the preceding section, with slightly different choices of $m_1$ and $m_2$, and with a certain amount of additional care. Note that, from Theorem \ref{main-thm}, we may assume that we are in one of the following cases
\begin{enumerate}
\item $n=3, \; d=2, \;  n_0=1, \; 2 \leq x \leq 46655$,
\item $n=4, \; d=3, \; n_0=1, \; 2 \leq x \leq 122826$,
\item $n=5, \; d=2, \; n_0=2, \; 2 \leq x \leq 60466175$,
\item $n=5, \; d=4, \; n_0=1, \; 2 \leq x \leq 248831$.
\end{enumerate}
Initially, we will suppose that $x \geq 40$ and, in all cases, that $m_1$ and $m_2$ are nonnegative integers satisfying (\ref{sturdy}). We will always, in fact, choose $m_1$ positive. Again setting $m_2=\alpha m_1$, 
via calculus, we may bound the integral 
\[\int^{1}_{0} ~
\frac{v^{m_2+1/d} (1-v)^{m_1-1/d} dv}{(1-(1-v)/x)^{m_2+1}}\]
in (\ref{del}) by
$$
\left( \max_{v \in [0,1]} \frac{v^{(\alpha+1)/d}}{ \left( 1 - (1-v)/x \right)^{(\alpha+d)/d}} \right) 
\; M(\alpha,x)^{1/d-m_1} < M(\alpha,x)^{1/d-m_1}.
$$
From (\ref{del}), it thus follows that
\begin{equation} \label{del-upper}
|\delta| < \frac{\sin (\pi/d) \, C(m_2,d)}{\pi \,  x^{m_1+1}} \;  M(\alpha,x)^{1/d-m_1}.
\end{equation}


%-------------------------------------------------------------------------------
\subsection{Case (1) : $n=3, \; d=2, \;  n_0=1, \; x \geq 40$}
%-------------------------------------------------------------------------------

In this case, we will take
$$
m_1 = \Big\lceil \frac{2m_0}{7} \Big\rceil \; \; \mbox{ and } \; \; m_2 = m_0+  \Big\lceil \frac{2m_0}{7} \Big\rceil,
$$
where by $\lceil x \rceil$ we mean the least integer that is $\geq x$, so that $m_1 \geq 2 m_2/9$, i.e. $\alpha \leq 9/2$. 
From (\ref{del-upper}) and   Proposition \ref{Cee}, 
$$
|\delta| < \frac{M(\alpha,x)^{1/2}}{\pi x} \; \left( \frac{4^{\alpha} }{x \, M(\alpha,x)} \right)^{m_1}.
$$
Appealing to (\ref{M-lower}), since $x \geq 40$ and $\alpha \leq 9/2$, it follows that 
$$
\frac{4^{\alpha} }{x \, M(\alpha,x)} \leq \frac{4^{\alpha} }{\left( 1 + \frac{39}{40 \alpha} \right)^\alpha \, 39 \; (\alpha+1)} < 1,
$$
whence, from (\ref{upper-emm}), 
$$
|\delta| < \frac{M(\alpha,x)^{1/2}}{\pi x} < \frac{(e \, (\alpha+1))^{1/2}}{\pi x}  < 0.031.
$$
We may therefore apply Proposition \ref{upper-cool} to conclude that
\begin{equation} \label{fly}
 |Q_1| > 2.477 \, x^{2m_0}.
\end{equation}
From (\ref{munchkin}), Proposition \ref{Cee}, $\alpha \leq 9/2$ and $x \geq 40$, we have
$$
|Q_1| < 6.81  \cdot 14.951^{m_1}  \, (4x)^{m_0+m_1} 
$$
and so
\begin{equation} \label{kite}
x < \left( 2.75 \cdot 14.951^{m_1}  \, 4^{m_0+m_1} \right)^{\frac{1}{m_0-m_1}}.
\end{equation}
We may check that $m_0 > 3.4 m_1$ (so that $\alpha > 4.4$) whenever $m_0 \geq 96$ and hence, since the right hand side of (\ref{kite}) is monotone decreasing in $m_0$, may conclude that $x < 40$, a contradiction. 

For $m_0 \leq 95$, we note that
$$
\frac{m_1m_2}{m_1+m_2} \leq m_1 = \Big\lceil \frac{2m_0}{7} \Big\rceil \leq   \Big\lceil \frac{2 \cdot 95}{7} \Big\rceil = 28 < x
$$
and hence may
appeal to Proposition \ref{shazam}.
It follows from (\ref{fly}) and $x \geq 40$ that
$$
x < \left( \frac{C(m_2,2)}{2.415} \, \binom{m_1+m_2}{m_1} \right)^{\frac{1}{m_0-m_1}}.
$$
A short computation leads to the conclusion that $x < 40$, unless $m_0 =4$ (in which case $x \leq 108$) or $m_0=18$ (whence $x \leq 40$). In the last case, we therefore have $x=40$ and $m=37$, and we may easily check that there are no corresponding solutions to equation (\ref{eq-main}).
 If $m_0=4$ (so that $m=9$) and $40 \leq x \leq 108$, there are, similarly, no solutions to (\ref{eq-main}) with $n=3$. 

%--------------------------------------------------------------------------------
\subsection{Case (2) : $n=4, \; d=3, \;  n_0=1, \; x \geq 85$}
%-------------------------------------------------------------------------------

We argue similarly in this case, choosing
$$
m_1 = \Big\lceil \frac{m_0}{3.23} \Big\rceil \; \; \mbox{ and } \; \; m_2 = m_0+  \Big\lceil \frac{m_0}{3.23} \Big\rceil,
$$
so that $\alpha \leq 4.23$.
From (\ref{del-upper}) and   Proposition \ref{Cee}, 
$$
|\delta| < \frac{\sqrt{3} \, M(\alpha,x)^{1/3}}{2 \, \pi x} \; \left( \frac{3^{3\alpha/2} }{x \, M(\alpha,x)} \right)^{m_1}.
$$
Applying (\ref{M-lower}), $x \geq 85$ and $\alpha \leq 4.23$, 
$$
 \frac{3^{3\alpha/2} }{x \, M(\alpha,x)} \leq \frac{3^{3\alpha/2} }{\left( 1 + \frac{84}{85 \alpha} \right)^\alpha \, 84 \; (\alpha+1)} < 1
$$
and so 
$$
|\delta| < \frac{\sqrt{3} \, M(\alpha,x)^{1/3}}{2 \, \pi x}  < 
\frac{\sqrt{3} \,(e \, (\alpha+1))^{1/3}}{2 \,\pi x}  < 0.008.
$$
Proposition \ref{upper-cool} thus implies 
\begin{equation} \label{fowl}
|Q_1| > 3.808 \, x^{2 m_0}
\end{equation}
while (\ref{munchkin}), Proposition \ref{Cee}, $\alpha \leq 4.23$ and $x \geq 85$ give
$$
|Q_1| < 6.5 \cdot 14.217^{m_1}  \, (3 \sqrt{3} \, x)^{m_0+m_1}.
$$
 It follows that
\begin{equation} \label{kite3}
x < \left( 1.707 \cdot 14.217^{m_1}  \, (3 \sqrt{3})^{m_0+m_1} \right)^{\frac{1}{m_0-m_1}}.
\end{equation}
 We may check that $m_0 \geq 3.14 m_1$, for all $m_0 \geq 98$ (and $m_1 \geq 31$) and hence, for these $m_0$, we have $\alpha \geq 4.14$ and so
 $$
 x < 1.707^{1/67} \cdot 14.217^{1/2.14} \cdot (3 \sqrt{3})^{4.14/2.14},
 $$
 which contradicts $x \geq 85$.
 
 For $m_0 \leq 97$, we again find that
 $$
\frac{m_1m_2}{m_1+m_2} \leq m_1 = \Big\lceil \frac{m_0}{3.23} \Big\rceil \leq   \Big\lceil \frac{97}{3.23} \Big\rceil = 31 < x
$$
 and hence, from Proposition \ref{shazam}, (\ref{fowl}) and $x \geq 85$, 
$$
x < \left( \frac{C(m_2, 3)}{3.763} \, \binom{m_1+m_2}{m_1} \right)^{\frac{1}{m_0-m_1}},
$$
contradicting $x \geq 85$, unless we have $m_0=4$ and $x \leq 220$, or $m_0=7$ and $x \leq 138$, or $m_0=10$ and $x \leq 99$, or $m_0=13$ and $x \leq 110$,
or $m_0=20$ and $x \leq 87$. In each case, we may verify that there are no solutions to equation (\ref{eq-main}). By way of example, if $m_0=4$, then $m=13$ and a short computation reveals that, for $85 \leq x \leq 220$, there are no corresponding solutions to (\ref{eq-main}).

 
%--------------------------------------------------------------------------------
\subsection{Case (3) : $n=5, \; d=2, \;  n_0=2, \; x \geq 720$}
%-------------------------------------------------------------------------------

In this case, we will take
$$
m_1 = \Big\lceil \frac{m_0}{5.906} \Big\rceil \; \; \mbox{ and } \; \; m_2 = m_0+  \Big\lceil \frac{m_0}{5.906} \Big\rceil,
$$
so that $\alpha \leq 6.906$. 
From (\ref{del-upper}) and   Proposition \ref{Cee}, 
$$
|\delta| < \frac{M(\alpha,x)^{1/2}}{\pi x} \; \left( \frac{4^{\alpha} }{x \, M(\alpha,x)} \right)^{m_1}.
$$
Appealing to (\ref{M-lower}), since $x \geq 720$ and $\alpha \leq 6.906$, it follows that 
$$
\frac{4^{\alpha} }{x \, M(\alpha,x)} \leq \frac{4^{\alpha} }{\left( 1 + \frac{719}{720 \alpha} \right)^\alpha \, 719 \; (\alpha+1)} < 1,
$$
whence, from (\ref{upper-emm}), 
$$
|\delta| < \frac{M(\alpha,x)^{1/2}}{\pi x} < \frac{(e \, (\alpha+1))^{1/2}}{\pi x}  < 0.003.
$$
We may therefore apply Proposition \ref{upper-cool} to conclude that
\begin{equation}\label{fowl511}
 |Q_1| > 2.552 \, x^{\frac{3}{2}m_0}.
\end{equation}
On the other hand, from (\ref{munchkin}), Proposition \ref{Cee}, $\alpha \leq 6.906$ and $x \geq 720$ we have
$$
|Q_1| < 9.058 \cdot 21.491^{m_1}  \, (4x)^{m_0+m_1}.
$$
 It follows that
\[
x < \left( 3.550 \cdot 21.491^{m_1}  \, 4^{m_0+m_1} \right)^{\frac{2}{m_0-2m_1}}.
\]
 We may check that $m_0>5.809 m_1$ (so that $\alpha>6.809$), for all $m_0 \geq 332$ and hence, for these $m_0$, we have 
 $$
 x < 3.550^{1/108} \cdot 21.491^{2/3.809} \cdot 4^{2+6/3.809}
 $$
 which contradicts  $x \geq 720$. For $m_0\leq 331$, 
$$
\frac{m_1m_2}{m_1+m_2} \leq m_1 = \Big\lceil \frac{m_0}{5.906} \Big\rceil \leq   \Big\lceil \frac{331}{5.906} \Big\rceil= 57 < x
$$
 and hence Proposition \ref{shazam}, (\ref{fowl511}) and $x \geq 720$ imply that
$$
x < \left( \frac{C(m_2, 2)}{2.548} \, \binom{m_1+m_2}{m_1} \right)^{\frac{2}{m_0-2m_1}},
$$
contradicting $x \geq 720$, unless  we have $m_0$ and $720 \leq x \leq x_0$ as follows :
$$
\begin{array}{cc|cc|cc|cc|cc} 
m_0 & x_0 & m_0 & x_0 & m_0 & x_0 & m_0 & x_0 & m_0 & x_0\\ \hline
3 & 63090 & 12 & 2780 & 19 & 992 & 31 & 834 & 54 & 836 \\
6 & 578712 & 13 & 2531 & 20 & 909 & 36 & 859 & 55 & 723 \\
7 & 12601 & 14 & 1177 & 24 & 1101 & 37 & 777 & 65 & 765\\
8 & 2605 & 15 & 755 & 25 & 847 & 42 & 849 & 71 & 768\\
9 & 762 & 18 & 1667 & 30 & 1103 & 48 & 767 & 83 & 734\\
\end{array}
$$
Since we are assuming that $m_0$ is odd, because $\gcd(m-1, n-1)=2$,  this table reduces to the following:

$$
\begin{array}{cc|cc|cc|cc|cc} 
m_0 & x_0 & m_0 & x_0 & m_0 & x_0 & m_0 & x_0 & m_0 & x_0\\ \hline
3 & 63090 & 13 & 2531 & 25 & 847 & 55 & 723 & 83 & 734\\
7 & 12601 & 15 & 755 & 31 & 834 & 65 & 765\\
9 & 762 & 19 & 992 & 37 & 777 & 71 & 768\\

\end{array}
$$

 
For these remaining triples $(x,n,m)=(x,5,2m_0+1)$, with $720 \leq x \leq x_0$, just as in the cases $n=3$ and $n=4$, we reach a contradiction  upon explicitly verifying that there are no integers $y$ satisfying equation~\eqref{eq-main}. 

%--------------------------------------------------------------------------------
\subsection{Case (4) : $n=5, \; d=4, \;  n_0=1, \; x \geq 300$}
%-------------------------------------------------------------------------------

In this case, we will take
$$
m_1 = \Big\lceil \frac{m_0}{2.93} \Big\rceil \; \; \mbox{ and } \; \; m_2 = m_0+  \Big\lceil \frac{m_0}{2.93} \Big\rceil,
$$
so that $\alpha \leq 3.93$. 
From (\ref{del-upper}) and   Proposition \ref{Cee}, 
$$
|\delta| < \frac{\sqrt{2}M(\alpha,x)^{1/4}}{2\pi x} \; \left( \frac{8^{\alpha} }{x \, M(\alpha,x)} \right)^{m_1}.
$$
Appealing to (\ref{M-lower}), since $x \geq 300$ and $\alpha \leq 3.93$, it follows that 
$$
\frac{8^{\alpha} }{x \, M(\alpha,x)} \leq \frac{8^{\alpha} }{\left( 1 + \frac{299}{300 \alpha} \right)^\alpha \, 299 \; (\alpha+1)} < 1,
$$
whence, from (\ref{upper-emm}), 
$$
|\delta| < \frac{\sqrt{2}M(\alpha,x)^{1/4}}{2\pi x} < \frac{\sqrt{2}(e \, (\alpha+1))^{1/4}}{2\pi x}  < 0.002.
$$
We may therefore apply Proposition \ref{upper-cool} to conclude that
\begin{equation} \label{cliff2}
 |Q_1| > 5.109 \, x^{2m_0}.
\end{equation}
On the other hand, from (\ref{munchkin}), Proposition \ref{Cee}, $\alpha \leq 3.93$ and $x \geq 300$ we have
$$
|Q_1| < 6.19 \cdot 13.402^{m_1}  \, (8x)^{m_0+m_1}.
$$
 It follows that
\[
x < \left(1.212 \cdot 13.402^{m_1}  \, 8^{m_0+m_1} \right)^{\frac{1}{m_0-m_1}}.
\]
 We may check that $m_0 \geq 2.87 m_1$ (so that $\alpha\geq 3.87$) for all $m_0 \geq 133$ (and hence for $m_1\geq 46$) and hence, for these $m_0$, we have 
 $$
 x < 1.212^{1/87} \cdot 13.402^{1/1.87} \cdot 8^{3.87/1.87}
 $$
 which contradicts  $x\geq 300$. 
 
For $m_0 \leq 132$, 
$$
\frac{m_1m_2}{m_1+m_2} \leq m_1 = \Big\lceil \frac{m_0}{2.93} \Big\rceil \leq   \Big\lceil \frac{132}{2.93} \Big\rceil = 46 < x
$$
 and hence Proposition \ref{shazam}, (\ref{cliff2}) and $x \geq 300$ imply that
$$
x < \left( \frac{C(m_2, 4)}{5.091} \, \binom{m_1+m_2}{m_1} \right)^{\frac{1}{m_0-m_1}}.
$$
 A short computation leads to the conclusion that $x<300$ for all $m_0\leq 132$, unless we have $m_0$ and $x \leq x_0$ as follows :
 
 $$
\begin{array}{cc|cc|cc} 
m_0 & x_0 & m_0 & x_0 & m_0 & x_0 \\ \hline
3 & 33791 & 7 & 350 & 15 & 343 \\
4 & 600 & 9 & 502 & 18 & 315 \\
6 & 1131 &12 & 434 & & \\
\end{array}
$$
In the remaining cases, we again reach a contradiction  upon explicitly verifying that there are no integers $y$ satisfying equation~\eqref{eq-main} (assuming thereby $x\geq 300$). 

%------------------------------------------------------
 \subsection{Treating the remaining small values of $x$ for $n \in \{ 3, 4 \}$}
%-----------------------------------------------------

To deal with the remaining pairs $(x,n)$ for $n \in \{ 3, 4, 5 \}$, we can, in each case, reduce the problem to finding ``integral points'' on particular models of genus one curves. Such a reduction is not apparently available for larger values of $n$. In case $n \in \{ 3, 4 \}$, this approach enables us to complete the proof of Theorem \ref{main-thm2}. When $n=5$ (where we are left to treat values $2 \leq x < 720$), the resulting computations are much more involved. To complete them, we must work rather harder; we postpone the details to the next section.

%------------------------------------------------------
 \subsubsection{Small values of $x$ for $n=3$}
%-----------------------------------------------------

To complete the proof of Theorem \ref{main-thm2} for $n=3$, it remains to solve equation (\ref{eq-main}) with $2 \leq x \leq 39$.
In this case,  (\ref{eq-main}) becomes
\begin{equation} \label{eq-three}
y^2+y+1 = \frac{x^m-1}{x-1},
\end{equation}
whereby
$$
\left( 4 (x-1)^2 (2y+1) \right)^2 = 64 (x-1)^3  x^m - 16 (3x+1)(x-1)^3.
$$
Writing $m = 3 \kappa + \delta$ for $\kappa \in \mathbb{Z}$ and $\delta \in \{ 0, 1, 2 \}$, we thus have
\begin{equation} \label{Mordell}
Y^2 = X^3 - k,
\end{equation}
for 
$$
X = 4 (x-1) x^{\kappa+\delta}, \; \; Y = 4 (x-1)^2 (2y+1) x^\delta \; \mbox{ and } \; k = 16 (3x+1)(x-1)^3 x^{2 \delta}.
$$

We solve equation (\ref{Mordell}) for the values of $k$ arising from $2 \leq x \leq 39$ and $0 \leq \delta \leq 2$ rather quickly using Magma's {\it IntegralPoints} routine (see \cite{magma}). The only solutions we find with the property that $4 (x-1) x^2 \mid X$ are those coming from trivial solutions corresponding to $m =2$, together with $(x,\delta,X,|Y|)$ equal to one of
$$
\begin{array}{c}
(2,1,128,1448), (2,2,32,176), (5,2,800,22400), (8,2,3584,213248),  \\
(19,2,389880,243441072), (26,2,11897600,41038270000) \mbox{ or } \\
(27,2,227448,108416880). \\
\end{array}
$$
Of these, only  $(x,\delta,X,|Y|)=(2,1,128,1448)$ and $(2,2,32,176)$ have the property that $X= 4 (x-1) x^{t}$ for $t$ an integer, corresponding to  the solutions $(x,y,m)=(2,90,13)$ and $(2,5,5)$ to equation (\ref{eq-three}), respectively.


%------------------------------------------------------
 \subsubsection{Small values of $x$ for $n=4$}
%-----------------------------------------------------

 If $n=4$ and we write $m=2 \kappa + \delta$, for $\kappa \in\mathbb{Z}$ and $\delta \in \{ 0, 1 \}$, then (\ref{eq-main}) becomes
 $$
 x^\delta (x^\kappa)^2 = (x-1) ( y^3+y^2+y+1) + 1,
 $$
 whereby
 $$
 Y^2 = X^3 + x^\delta (x-1)X^2 + x^{2 \delta} (x-1)^2 X + x^{1+3 \delta} \, (x-1)^2,
 $$
for
 $$
 X=(x-1) x^\delta y \; \; \mbox{ and } \; \;  Y=(x-1) \, x^{\kappa+2 \delta}.
 $$
 Once again applying Magma's {\it IntegralPoints} routine, we find that the only points for $2 \leq x \leq 84$ and $\delta \in \{0, 1 \}$, and having $(x-1) x^2 \mid Y$ correspond to either trivial solutions to (\ref{eq-main}) with either $y=0$ or $m=4$, or have $\delta=1$ and  $(x,X,|Y|)$ among
 $$
 \begin{array}{c}
 (4,48,384), (9,648,17496), (16,3840,245760), (21,1680,79380), \\
(21,465360,317599380), (25,15000,1875000), (36,45360,9797760), \\
(41,33620,6320560), (49,115248,39530064), (64,258048,132120576), \\
(65,10400,1352000), (81,524880,382637520). \\
\end{array}
 $$
None of these triples lead to nontrivial solutions to (\ref{eq-main}) with $n=4$.
 
 %------------------------------------------------------
 \section{Small values of $x$ for $n=5$} \label{TM}
%-----------------------------------------------------
 
 In case $n=5$, solving equation (\ref{eq-main}) can, for a fixed choice of $x$, also be reduced to a question of finding integral points on a particular model of a genus $1$ curve.
 Generally, for $m$ odd, say $m= 2 \kappa+1$, we can rewrite  (\ref{eq-main}) as
 $$
 x \left( x^{\kappa} \right)^2 = (x-1) \left( y^4+y^3+y^2+y+1 \right) + 1,
 $$
 so that 
 $$
 \left( x^{\kappa+1} \right)^2 = (x^2-x) \left( y^4+y^3+y^2+y \right) + x^2.
 $$
 Applying Magma's {\it IntegralQuarticPoints} routine, 
 %$([x^2-x,x^2-x,x^2-x,x^2-x,x^2])$, 
 we may find solutions to the more general Diophantine equation
 \begin{equation} \label{quartic}
 Y^2 = (x^2-x) \left( y^4+y^3+y^2+y \right) + x^2;
 \end{equation}
 note that we always have, for each $x$, solutions $(y,Y) = (0, \pm x), (-1, \pm x)$ and $(x, \pm x^3)$. 
 
 Unfortunately, it does not appear that this approach is computationally efficient enough to solve equation (\ref{quartic}) in a reasonable time for all values of $x$ with  $2 \leq x < 720$ (though it does work somewhat quickly for $2 \leq x \leq 59$ and various other $x < 720$). The elliptic curve defined by (\ref{quartic}) has, in each case, rank at least $2$ (the solutions corresponding to $(y,Y) = (0, x)$ and $(-1, x)$ are independent non-torsion points). Magma's {\it IntegralQuarticPoints} routine is based on bounds for  linear forms in elliptic logarithms and hence requires detailed knowledge of the generators of the Mordell-Weil group. Thus, when the rank is much larger than $2$, Magma's   {\it IntegralQuarticPoints} routine can, in practice, work very slowly. This is the case, for example, when $x=60$ (where the corresponding elliptic curve has rank $5$ over $\mathbb{Q}$).  

 
%Unfortunately, it does not appear that this approach is computationally efficient enough to solve equation (\ref{quartic}) in a reasonable time for all values of $x$ with  $2 \leq x < 720$ (though it does work somewhat quickly for $2 \leq x \leq 59$ and various other $x < 720$). The elliptic curve defined by (\ref{quartic}) has, in each case, rank at least $2$ (the solutions corresponding to $(y,Y) = (0, x)$ and $(-1, x)$ are independent non-torsion points). When the rank is much larger than this, Magma's   {\it IntegralQuarticPoints} routine (which is based on bounds for  linear forms in elliptic logarithms and hence requires detailed knowledge of the generators of the Mordell-Weil group) can, in practice, work very slowly. This is the case, for example, when $x=60$ (where the corresponding elliptic curve has rank $5$ over $\mathbb{Q}$). 
 
%    For $2 \leq x \leq 100$, we find additional solutions with $(x, y,|Y|)$ among
% $$
% \begin{array}{c}
% (3, -28, 1887), (4, 32, 3604), (5,-7, 205), (5,-3, 35), (5,-2, 15), (5,2, 25), (14,6, 532), \\
%  (14, 7,  714), (15,5,405), (15,9,1245),  (20, 4, 360),   (20, 15, 4540),  (25, 1, 55),  \\
%(25, 132, 428425), (27,-3,207), (40,2,220), (45,-8,2685), (45,7,2355), \\
 %  (49,-2,161), (64,-3,496), (72,-5,1632),
%  \end{array}
% $$
% (actually, this computation is still running and seems to be having trouble with $x=45, 62, 78, 86$ -- possible rank problem).
 
% Done for the following values of $x$ (if one trusts magma, stuck on $x=60$) :
 %$$
% 2 \leq x \leq 59, \; 61 \leq x  \leq  100.
% $$
 
 Instead, we will argue somewhat differently. We write (\ref{eq-main}) as
 
 \begin{equation} \label{TM-start}
 F_x(y,1) = x^m,
 \end{equation}
 where
 $$
 F_x(y,z) = (x-1)(y^4 + y^3z + y^2z^2 + yz^3) + xz^4.
 $$
For the remainder of this section, we consider the homogeneous quartic form \eqref{TM-start} for fixed $x$. Notably, we observe that this equation is a special case of the Thue-Mahler equation~\eqref{eq:ThueMahler}. In particular, if $x = p_1^{\alpha_1}\cdots p_v^{\alpha_v}$ is the prime factorization of $x$ with $\alpha_i \geq 0$, then equation (\ref{TM-start}) becomes 
\begin{equation}\label{Eq:main}
F_x(y,1) =  p_1^{Z_1}\dots p_v^{Z_v}
\end{equation}
where $Z_i = m\alpha_i$. 

To find all solutions to this equation, we will use linear forms in $p$-adic logarithms to generate a very large upper bound on $m$. Then, applying several instances of the LLL lattice basis reduction algorithm, we will reduce the bound on $m$ until it is sufficiently small enough that we may perform a brute force search efficiently. The remainder of this section is devoted to the details of this approach.

%--------------------------------------------------------------------------------------------------------------------------------------------%

\subsection{First steps and small bounds}
\label{subsec:FirstStepsSmallBoundsGE}

Following arguments of \autoref{ch:AlgorithmsForTM} for solving  Thue-Mahler equations, put $S = \{p_1, \dots, p_v\}$. This is the set of all distinct rational primes dividing $x$. As we seek only those solutions $(y,z, Z_1, \dots, Z_v)$ to \eqref{Eq:main} for which $z = 1$, here and henceforth we write, for concision, $F(y)=F_x(y,1)$. 

Recall in  \autoref{sec:FirstSteps} of \autoref{ch:AlgorithmsForTM} the set $\mathcal{D}$. This set consists of all positive rational integers $m$ dividing $(x-1)$ such that $\ord_{p}(m) \leq \ord_p(c)$ for all primes $p \notin S$. In our case, $c = 1$ so that $\mathcal{D} = \{1\}$. Thus the only possible values for $u_d,c_d$ are
\[u_d = (x-1)^{3} \quad \text{ and } \quad c_d = (x-1)^3.\]
Under the appropriate change of variables associated to $u_d,c_d$, this yields
$$
g(t) = (x-1)^3F\left(\frac{t}{x-1}\right) = t^4 + (x-1)t^3 + (x-1)^2t^2 + (x-1)^3t + x(x-1)^3.
$$
Note that $g(t)$ is irreducible in $\mathbb{Z}[t]$. Writing $K = \mathbb{Q}(\theta)$ with $g(\theta) = 0$, it follows that
 \eqref{Eq:main} is equivalent to
\begin{equation} \label{Eq:norm}
N_{K/\mathbb{Q}}((x-1)y-\theta) =  (x-1)^{3}p_1^{Z_1}\dots p_v^{Z_v}.
\end{equation}

Let 
\[(p_i)\mathcal{O}_K = \prod_{j = 1}^{m_i} \mathfrak{p}_{ij}^{e(\mathfrak{p}_{ij}|p_i)}\]
be the factorization of $p_i$ into prime ideals in the ring of integers $\mathcal{O}_K$ of $K$. In this decomposition, $e(\mathfrak{p}_{ij}|p_i)$ and $f(\mathfrak{p}_{ij}|p_i)$ denote the ramification index and residue degree of $\mathfrak{p}_{ij}$ respectively. Then, since $N(\mathfrak{p}_{ij}) = p_i^{f(\mathfrak{p}_{ij}|p_i)}$, equation \eqref{Eq:norm} leads to finitely many ideal equations of the form
\begin{equation} \label{Eq:ideals}
((x-1)y-\theta)\mathcal{O}_K = \mathfrak{a} \prod_{j = 1}^{m_1} \mathfrak{p}_{1j}^{z_{1j}} \cdots \prod_{j = 1}^{m_v} \mathfrak{p}_{vj}^{z_{vj}}
\end{equation}
where $\mathfrak{a}$ is an ideal of norm $(x-1)^3$ and the $z_{ij}$ are unknown integers related to $m$ by $\sum_{j = 1}^{m_i} f(\mathfrak{p}_{ij}|p_i)z_{ij} = Z_i = m \alpha_i$. 
Applying Algorithms~\ref{alg:AffinePatch1} and \ref{alg:AffinePatch2}, we reduce the number of prime ideals appearing to a large power in this equation. In doing so, we are reduced to solving finitely many equations of the form
\begin{equation}\label{Eq:TMfactored}
((x-1) y- \theta)\mathcal{O}_K=\mathfrak{a} \mathfrak{p}_1^{u_1}\cdots \mathfrak{p}_v^{u_v}
\end{equation}
in integer variables $y,u_1, \dots, u_v$ with $u_i \geq 0$ for $i = 1, \dots, v$. 
Here
\begin{itemize}
\item for $i \in \{1, \dots, v\}$, $\mathfrak{p}_i$ is a prime ideal of $\mathcal{O}_K$ arising from 
Algorithm~\ref{alg:AffinePatch1} and Algorithm~\ref{alg:AffinePatch2} applied to $p \in \{p_1, \dots, p_v\}$, such that $(\mathfrak{b}, \mathfrak{p}_i) \in M_p$ for some ideal $\mathfrak{b}$;
\item for any $p_i \in S$ such that $M_{p_i} = \emptyset$, $\mathfrak{p}_i$ denotes the trivial ideal $\mathfrak{p}_i = (1)\mathcal{O}_K$;
\item $\mathfrak{a}$ is an ideal of $\mathcal{O}_K$ of norm $(x-1)^3\cdot p_1^{t_1} \cdots p_v^{t_v}$ such that
$u_i + t_i =  Z_i = m\alpha_i$.
\end{itemize}

\begin{remark}\label{rem:uv}
Unlike in \cite{TW3} and \cite{GhKaMaSi}, if, after applying Algorithm~\ref{alg:AffinePatch1} and Algorithm~\ref{alg:AffinePatch2}, we are in the situation that $u_i = 0$ for some $i$ in $\{1, \dots, v\}$, it follows that
\[m = \frac{Z_i}{\alpha_i } =  \frac{u_i + t_i}{\alpha_i } = \frac{t_i}{\alpha_i }.\]
We iterate this computation over all $i \in \{1, \dots, v\}$ such that $u_i = 0$ and take the smallest $m$ as our bound. For all of the values of $x$ that we are interested in, this bound on $m$ is small enough that we may go directly to the final brute force search for solutions.
\end{remark}

Following Remark \ref{rem:uv}, for the remainder of this paper, we assume that $u_i \neq 0$ for all $i = 1, \dots, v$. Fix a complete set of fundamental units $\{\eps_1, \dots, \eps_r\}$ of $\mathcal{O}_K$. Here $r = s + t -1$, where $s$ denotes the number of real embeddings of $K$ into $\mathbb{C}$ and $t$ denotes the number of complex conjugate pairs of non-real embeddings of $K$ into $\mathbb{C}$. A quick computation in Maple shows that  
\[g(t) = t^4 + (x-1)t^3 + (x-1)^2t^2 + (x-1)^3t + x(x-1)^3\]
has only complex roots for $x \geq 2$. It follows that we have no real embeddings of $K$ into $\mathbb{R}$, two pairs of complex conjugate embeddings, and hence only one fundamental unit, $\varepsilon_1$. 

Now, for each choice of $\mathfrak{a}$ and prime ideals $\mathfrak{p}_1, \dots, \mathfrak{p}_v$, we reduce each equation~\eqref{Eq:TMfactored} to a number of so-called ``$S$-unit equations'' via either procedure outlined in \autoref{subsec:FactorizationTMwithoutOK} and \autoref{subsec:FactorizationTMwithOK} of \autoref{ch:AlgorithmsForTM}. Regardless of which of these principalization methods is used, we arrive at finitely many equations of the form
\begin{equation} \label{Eq:main2}
(x-1)y - \theta = \alpha \zeta \eps_1^{a_1}\gamma_1^{n_1}\cdots \gamma_v^{n_v}
\end{equation}
with unknowns $a_1 \in \mathbb{Z}$, $n_i \in \mathbb{Z}_{\geq 0}$, and $\zeta$ in the set $T$ of roots of unity in $\mathcal{O}_K$. Since $T$ is also finite, we will treat $\zeta$ as another parameter. Moreover, we note that the ideal generated by $\alpha$ has norm
\begin{equation} \label{Eq:main3}
(x-1)^3\cdot p_1^{t_1 + r_1} \cdots p_v^{t_v + r_v},
\end{equation}
and the $n_i$ are related to $m$ via
\[m \alpha_i = Z_i = u_i + t_i = \sum_{j = 1}^{v}n_ja_{ij} + r_i + t_i.\]
To summarize, our original problem of solving \eqref{Eq:main} is now reduced to the problem of solving finitely many equations of the form \eqref{Eq:main3} for the variables 
\[y, a_1, n_1, \dots, n_v.\] 

From here, we follow the arguments of \autoref{subsec:SUnitEquation} to deduce a so-called $S$-unit equation. In doing so, we eliminate the variable $y$ and set ourselves the task of bounding the exponents $a_1, n_1, \dots, n_v$. 

In particular, let $p \in \{p_1, \dots, p_v, \infty\}$. Denote the roots of $g(t)$ in $\overline{\mathbb{Q}_p}$ (where $\overline{\mathbb{Q}_{\infty}} = \overline{\mathbb{R}} = \mathbb{C}$) by $\theta^{(1)}, \dots, \theta^{(4)}$. Let $i_0, j, k \in \{1, \dots, 4\}$ be distinct indices and consider the three embeddings of $K$ into $\overline{\mathbb{Q}_p}$ defined by $\theta \mapsto \theta^{(i_0)}, \theta^{(j)}, \theta^{(k)}$. We use $z^{(i)}$ to denote the image of $z$ under the embedding $\theta \mapsto \theta^{(i)}$. Applying these embeddings to $\beta = (x-1)y - \theta$ yields
\begin{equation}\label{Eq:Sunit}
\lambda = \delta_1 \left( \frac{\varepsilon_1^{(k)}}{\varepsilon_1^{(j)}}\right)^{a_1}\prod_{i = 1}^v \left( \frac{\gamma_i^{(k)}}{\gamma_i^{(j)}}\right)^{n_i} - 1 = \delta_2 \left( \frac{\varepsilon_1^{(i_0)}}{\varepsilon_1^{(j)}}\right)^{a_1} \prod_{i = 1}^v \left( \frac{\gamma_i^{(i_0)}}{\gamma_i^{(j)}}\right)^{n_i},
\end{equation}
where
\[\delta_1 = \frac{\theta^{(i_0)} - \theta^{(j)}}{\theta^{(i_0)} - \theta^{(k)}}\cdot\frac{\alpha^{(k)}\zeta^{(k)}}{\alpha^{(j)}\zeta^{(j)}}, \quad \delta_2 = \frac{\theta^{(j)} - \theta^{(k)}}{\theta^{(k)} - \theta^{(i_0)}}\cdot \frac{\alpha^{(i_0)}\zeta^{(i_0)}}{\alpha^{(j)}\zeta^{(j)}}\]
are constants. 

Note that $\delta_1$ and $\delta_2$ are constants, in the sense that they do not depend upon $y,a_1,n_1, \dots, n_v.$

Let $l \in \{1, \dots, v\}$ and consider the prime $p = p_l$. From now on we make the following choice for the index $i_0$. Let $g_l(t)$ be the irreducible factor of $g(t)$ in $\mathbb{Q}_{p_l}[t]$ corresponding to the prime ideal $\mathfrak{p}_l$. Since $\mathfrak{p}_l$ has ramification index and residue degree equal to $1$, $\deg(g_l[t]) = 1$. We choose $i_0 \in \{1, \dots, 4\}$ so that $\theta^{(i_0)}$ is the root of $g_l(t)$. The indices of $j,k$ are fixed, but arbitrary. 

By Lemma~\ref{lem:Delta1}, if $\ord_{p_l}(\delta_1) \neq 0$ for any $l \in \{1, \dots, v\}$, then 
\[ \sum_{i = 1}^v n_ia_{li} = \min\{\ord_{p_l}(\delta_1), 0\} - \ord_{p_l}(\delta_2).\]
For us, if this bound holds for any prime $p_l \in S$, it follows that
\[m = \frac{\sum_{j = 1}^{v}n_ja_{lj} + r_l + t_l}{\alpha_l} = \frac{\min\{\ord_{p_l}(\delta_1), 0\} - \ord_{p_l}(\delta_2) + r_l + t_l}{\alpha_l}. \]
In particular, we iterate this computation over all $i \in \{1, \dots, v \}$ for which Lemma~\ref{lem:Delta1} holds and take the smallest $m$ as our bound on the solutions. We then compute all solutions below this bound using a simple brute force search. 

For the remainder of this chapter, we may assume that $\ord_{p_l}(\delta_1) = 0$, since otherwise a reasonable bound is afforded by Lemma \ref{lem:Delta1}. 

Following the notation of \autoref{sec:SmallBoundForSpecialCase}, we let
\[b_1 = 1, \quad b_{1+i} = n_i \ \text{ for } i \in \{1, \dots, v\},\]
and
\[ b_{v+2} = a_1.\]
Put
\[\alpha_1 = \log_{p_l} \delta_1, \quad \alpha_{1+i} = \log_{p_l}\left( \frac{\gamma_i^{(k)}}{\gamma_i^{(l)}}\right)  \ \text{ for } i \in \{1, \dots, v\},\]
and
\[\alpha_{v+2} = \log_{p_l}\left( \frac{\varepsilon_1^{(k)}}{\varepsilon_1^{(l)}}\right).\]
Define
\[\Lambda_l = \sum_{i = 1}^{v+2} b_i\alpha_i.\]

Let $L$ be a finite extension of $\mathbb{Q}_{p_l}$ containing $\delta_1$, $\frac{\gamma_i^{(k)}}{\gamma_i^{(l)}}$ (for $i = 1, \dots, v$), and $ \frac{\varepsilon_1^{(k)}}{\varepsilon_1^{(l)}}$. Since finite $p$-adic fields are complete, $\alpha_i \in L$ for $i = 1, \dots, v+2$ as well. Choose $\phi \in \overline{\mathbb{Q}_{p_l}}$ such that $L = \mathbb{Q}_{p_l}(\phi)$ and $\ord_{p_l}(\phi) > 0 $. Let $G(t)$ be the minimal polynomial of $\phi$ over $\mathbb{Q}_{p_l}$ and let $s$ be its degree. For $i = 1, \dots, v+2$ write
\[\alpha_i = \sum_{h = 1}^s \alpha_{ih}\phi^{h - 1}, \quad \alpha_{ih} \in \mathbb{Q}_{p_l}.\]
Then
\begin{equation} \label{Eq:lambdalh}
\Lambda_l = \sum_{h = 1}^s \Lambda_{lh}\phi^{h-1},
\end{equation}
with
\[\Lambda_{lh} = \sum_{i = 1}^{v+2} b_i \alpha_{ih}\]
for $h = 1, \dots, s$. 

We recall several important lemmata from \autoref{sec:SmallBoundForSpecialCase} which we restate here.
\begin{lemma}\label{Lem:discG}
For every $h \in \{1, \dots, s\}$, we have
\[\ord_{p_l}(\Lambda_{lh}) > \ord_{p_l}(\Lambda_l) - \frac{1}{2}\ord_{p_l}(\text{Disc}(G(t))).\]
\end{lemma}

\begin{lemma} \label{Lem:Lambda}
If 
\[\sum_{i = 1}^v n_{i}a_{li} > \frac{1}{p_l-1} - \ord_{p_l}(\delta_2),\]
then
\[\ord_{p_l}(\Lambda_l) = \sum_{i = 1}^v n_{i}a_{li} + \ord_{p_l}(\delta_2).\]
\end{lemma}

\begin{lemma} \label{Lem:specialcase} \
Let
\[w_l = \bigg\lfloor{\frac{1}{p_l-1} - \ord_{p_l}(\delta_2)}\bigg\rfloor.\]
\begin{enumerate}
\item[(i)] If $\ord_{p_l}(\alpha_1) < \displaystyle \min_{2 \leq i \leq v+2} \ord_{p_l}(\alpha_i)$, then
\[\sum_{i = 1}^v n_i a_{li} \leq \max \left\{w_l, \bigg \lceil\displaystyle \min_{2 \leq i \leq v+2} \ord_{p_l}(\alpha_{i}) - \ord_{p_l}(\delta_2) \bigg \rceil - 1 \right\}\]

\item[(ii)] For all $h \in \{1, \dots, s\}$, if $\ord_{p_l}(\alpha_{1h}) < \displaystyle \min_{2 \leq i \leq v+2} \ord_{p_l}(\alpha_{ih})$, then
\[\sum_{i = 1}^v n_i a_{li} \leq \max \left\{ w_l, \bigg \lceil \displaystyle \min_{2 \leq i \leq v+2} \ord_{p_l}(\alpha_{ih})- \ord_{p_l}(\delta_2) + \nu_l \bigg \rceil - 1\right\},\]
where 
\[\nu_l = \frac{1}{2}\ord_{p_l}(\text{Disc}(G(t))).\]
\end{enumerate}
\end{lemma}

Similar to Lemma~\ref{lem:Delta1}, if Lemma \ref{Lem:specialcase} holds for $p_l$ giving
\[\sum_{i = 1}^v n_i a_{li} \leq B_l\]
for some bound $B_l$ as in the lemma, it follows that
\[m = \frac{\sum_{j = 1}^{v}n_ja_{lj} + r_l + t_l}{\alpha_l} \leq \frac{B_l + r_l + t_l}{\alpha_l}. \]
Again, we iterate this computation over all $l \in \{1, \dots, v \}$ for which Lemma \ref{Lem:specialcase} holds and take the smallest $m$ as our bound on the solutions. We then compute all solutions below this bound using a simple naive search. 

%---------------------------------------------------------------------------------------------------------------------------------------------%

\subsection{Bounding the $\sum_{j = 1}^v n_ja_{ij}$}

At this point, similar to \cite{TW3}, a very large upper bound for 
\[\left(|a_1|, \sum_{j = 1}^v n_ja_{1j}, \dots, \sum_{j = 1}^v n_ja_{vj}\right)\]
is derived using the theory of linear forms in logarithms. In practice, however, this requires that we compute the absolute logarithmic height of all terms of our so-called $S$-unit equation, \eqref{Eq:Sunit}. More often than not, this proves to be a computational bottleneck, and is best avoided whenever possible. In particular, the approach of Tzanakis and de Weger \cite{TW3}  requires the computation of the absolute logarithmic height of each algebraic number in the product of \eqref{Eq:Sunit}. Unfortunately, in many such instances, the fundamental units may be very large, with each coefficient having over $10^5$ digits in their representation. Similarly, the generators of our principal ideals may also be very large, making elementary operations on them (such as division) a very time-consuming process. In the particular instance of $x = 60$, by way of example, each coefficient of $\alpha$ has in excess of 20,000 digits. As a result of this, computing the absolute logarithmic height of these elements, a process which must be done for each choice of parameters $\zeta, \mathfrak{a}, \mathfrak{p}_1, \dots, \mathfrak{p}_v,$ is  computationally painful. Instead of this approach, we appeal to results of Bugeaud and Gy\H{o}ry  \cite{BugeaudGyory} to generate a (very large) upper bound for these quantities, which, while not sharp, will nevertheless prove adequate for our purposes. Following the notation of \cite{BugeaudGyory}, we now describe this bound.  

Arguing as in \cite{BugeaudGyory}, put $Z_i = 4U_i + V_i$ with $U_i, V_i \in \mathbb{Z}$, $0 \leq V_i < 4$ for $i = 1, \dots, v$ and let $R_K$ and $h_K$ be the regulator and class number of $K$, respectively. Let $T$ be the set of all extensions to $K$ of the places of $\{p_1, \dots, p_v\}$. Let $P$ denote $\max\{p_1, \dots, p_v\}$, and let $R_T$ denote the $T$-regulator of $K$. Further, let $H$ be an upper bound for the maximum absolute value of the coefficients of $F$, namely $H = |x| = x$. Let $B = 3$, let $\log^*{a}$ denote $\max(\log(a), 1)$, and let
\[C_8 = \text{exp}\left\{c_{24}P^N R_T (\log^*R_T)\left(\frac{\log^*(PR_T)}{\log^*P}\right)(R_{K} + vh_{K} + \log(HB'))\right\},\]
where $N = 24$, $B' \leq BHP^{4v} = 2xP^{4v}$, and 
\[\begin{array}{cc}
c_{24}  & = 3^{v+1 +25}(v+1)^{5(v+1) +12} N^{3(v+1)+16} \\
	& = 3^{v + 26}(v+1)^{5v+17}N^{3v + 19}.
\end{array}\]
Then, \cite{BugeaudGyory} shows that $p_i^{U_i} \leq C_8$. Now, ${\log^*(PR_T)/\log^*P \leq 2\log^*R_T}$, so that 
\[C_8 \leq \text{exp}\left\{c_{24}P^N R_T 2(\log^*R_T)^2(R_{K} + vh_{K} + \log(HB'))\right\}.\] 
Lastly, we have, by \cite{BugeaudGyory} $R_T \leq R_Kh_K(4\log^*P)^{4v}$. We note that the fundamental units of $K$ may be very large, and so computing the regulator of $K$ can be a very costly computation. To avoid this, we simply appeal to the upper bound of \cite{BugeaudGyory}, namely
\[R_K < \frac{|\text{Disc}(K)|^{1/2}(\log|\text{Disc}(K)|)^{3}}{3!h_K}.\]

Now we have all of the components necessary to explicitly compute an upper bound on $C_8$, denoted $C_9$ in \cite{BugeaudGyory}, from which it follows that
\[U_i \leq \frac{\log(C_9)}{\log{p_i}}\]
and hence
\[m\alpha_i = Z_i = 4U_i + V_i < \frac{4\log(C_9)}{\log(p_i)} + V_i < \frac{4\log(C_9)}{\log(p_i)} + 4.\] 
We thus obtain the inequality  
\[m < \frac{4\log(C_9)}{\alpha_i\log(p_i)} + \frac{4}{\alpha_i} = C_{10};\]
we compute this for all $p_i \in \{1, \dots, v\}$ and select the smallest value of $C_{10}$ as our bound on $m$. 
 
From \eqref{Eq:main3}, it follows that
\[0 \leq \sum_{j=1}^v n_ja_{ij} = m\alpha_i - r_i - t_i \leq C_{10}\alpha_i - r_i - t_i.\]
At this point, converting this bound to a bound on $m$ would yield far too large of an exponent to apply our brute force search. Instead, we must argue somewhat more carefully. 
Note that 
\[||\mathbf{n}||_{\infty} = ||A^{-1}(\mathbf{u} - \mathbf{r})||_{\infty} \leq ||\mathbf{u} - \mathbf{r}||_{\infty}||A^{-1}||_{\infty},\]
and so
\[\max_{1 \leq i \leq v}|n_i| \leq ||A^{-1}||_{\infty}\max_{1 \leq i\leq v}\sum_{j = 1}^v n_j a_{ij}
\leq ||A^{-1}||_{\infty} \max_{1 \leq i\leq v}(C_{10}\alpha_i - r_i - t_i) = C_{11}.\]

%---------------------------------------------------------------------------------------------------------------------------------------------%

\subsection{A bound for $|a_1|$}

In this subsection, we establish an upper bound for $|a_1|$ by considering two cases separately. Our argument is based loosely on \cite{TW3} but differs substantially in order to accommodate our new $S$-unit equation, which, unlike in \cite{TW3}, may now have negative exponents, $n_i$. In this subsection, $\theta^{(1)}, \dots, \theta^{(4)}$ will denote the roots of $g(t)$ in $\mathbb{C}$. We order the roots of $g(t)$ in $\mathbb{C}$ so that
\[\theta^{(1)} = \overline{\theta^{3}} \quad \text{ and } \quad \theta^{(2)} = \overline{\theta^{4}} \in \mathbb{C}.\]



Put
\[C_{12} = \left|\log \frac{(x-1)^3}{\displaystyle \min_{1\leq i \leq 4} |\alpha^{(i)}\zeta^{(i)}|} + C_{10} \log{x}\right|\]
and 
\[C_{13} = \sum_{j = 1}^v \max_{1\leq i \leq 4} |\log|\gamma_j^{(i)}||\]

%\[{c_8'} = \log \frac{p^{r_1 + t_1} \cdots p^{r_v + t_v}}{\displaystyle \min_{1\leq i \leq n} |\alpha^{(i)}\zeta^{(i)}|}, 
%\quad \quad
%{c_9'} = \log \frac{p_1^{\sum_{j = 1}^v a_{j1}} \cdots p_v^{\sum_{j = 1}^va_{jv}}}{\displaystyle \min_{1\leq i \leq n} |\gamma_1^{(i)} \cdots \gamma_v^{(i)}|},\]
%\[{c_8''} = \log {\displaystyle \max_{1\leq i \leq n} |\alpha^{(i)}\zeta^{(i)}|}, 
%\quad \quad
%{c_9''} = \log {\displaystyle \max_{1\leq i \leq n} |\gamma_1^{(i)} \cdots \gamma_v^{(i)}|}.\]

Set 
\[C_{14} = \min \left( |\log|\varepsilon_1^{(1)}||, |\log|\varepsilon_1^{(2)}||\right)\]
and let $C_{15}$ be any number satisfying $0 < C_{15} < \frac{C_{14}}{3}$.
So we have 
\[C_{14} - C_{15} > C_{14} - 3C_{15} > 0.\]

\begin{lemma}
If $\displaystyle \min_{1\leq i \leq 4}|(x-1)y-\theta^{(i)}| > e^{-C_{15}|a_1|}$, we have
\[|a_1|<  \frac{C_{12} + C_{11}C_{13}}{C_{14} - 3C_{15}} .\]
\end{lemma}

\begin{proof}
Let $k \in \{1,2\}$ be an index such that 
\[C_{14} = \min \left( |\log|\varepsilon_1^{(1)}||, |\log|\varepsilon_1^{(2)}||\right) = |\log|\varepsilon_1^{(k)}||. \]
By \eqref{Eq:norm}, 
\[|\beta^{(k)}| \cdot \prod_{i \neq k}|\beta^{(i)}| = (x-1)^3\cdot p_1^{Z_1}\cdots p_v^{Z_v}, \]
therefore
\[|(x-1)y-\theta^{(k)}| = |\beta^{(k)}| < (x-1)^3\cdot x^{C_{10}} \cdot e^{3C_{15}|a_1|}.\]
%Here,
%\[Z_i = m\alpha_i \leq C_{10}\alpha_i.\]
%and so
%\[|(x-1)X - \theta^{(k)}| = |\beta^{(k)}| < |(x-1)^3|x^{C_{10}} e^{(n-1)c_{11}|a_1|}.\]
Now, 
\[|\varepsilon_1^{(k)a_1}|
= \frac{|(x-1)y-\theta^{(k)}|}{|\alpha^{(k)}\zeta^{(k)}||\gamma_1^{(k)}|^{n_1}\cdots |\gamma_v^{(k)}|^{n_v}}
< \frac{(x-1)^3\cdot x^{C_{10}} \cdot e^{3C_{15}|a_1|}}{\displaystyle \min_{1\leq i \leq 4}|\alpha^{(i)}\zeta^{(i)}| 		\cdot |\gamma_1^{(k)}|^{n_1}\cdots |\gamma_v^{(k)}|^{n_v}}\]
from which it follows that
\[\log|\varepsilon_1^{(k)a_1}| 
< \log{\frac{(x-1)^3}{\displaystyle \min_{1\leq i \leq 4}|\alpha^{(i)}\zeta^{(i)}|}} +  C_{10}\log{x} + 3C_{15}|a_1|- \sum_{j = 1}^v n_j \log|\gamma_j^{(k)}|.\]
Taking absolute values yields
\[|a_1|C_{14} = |a_1| |\log|\varepsilon_1^{(k)}| < C_{12} + 3C_{15}|a_1| + \sum_{j = 1}^v |n_j|\log{\gamma_j^{(k)}|}|.\]
Now
\[\begin{split}
|a_1| 
& < \frac{C_{12} +\displaystyle\sum_{j = 1}^v |n_j||\log|\gamma_j^{(k)}||}{C_{14} - 3C_{15}}\\
& < \frac{C_{12} +C_{11}\displaystyle\sum_{j = 1}^v |\log|\gamma_j^{(k)}||}{C_{14} - 3C_{15}}\\
& < \frac{C_{12} +C_{11}C_{13}}{C_{14} - 3C_{15}}.\\
\end{split}\]

%
%\[|\varepsilon_1^{(k)a_1}| = e^{Ac_{10}} \quad \text{ or } \quad |\varepsilon_1^{(k)a_1}| = e^{-Ac_{10}},\]
%where $A = |a_1|$. 
%
%Now, 
%\[|(x-1)X - \theta^{(k)}| = p_1^{Z_1}\cdots p_v^{Z_v}\prod_{i \neq k} |(x-1)X-\theta^{(i)}|^{-1}<
%p_1^{Z_1}\cdots p_v^{Z_v} e^{(n-1)c_{11}A}.\]
%So $|\varepsilon_1^{(k)a_1}| = e^{Ac_{10}}$ implies
%\[\begin{split}
% e^{Ac_{10}} = |\varepsilon_1^{(k)a_1}|	
%	& = \frac{|(x-1)X-\theta^{(k)}|}{|\alpha^{(k)}\zeta^{(k)}||\gamma_1^{(k)}|^{n_1}\cdots |\gamma_1^{(k)}|^{n_v}}\\	& < \frac{p_1^{Z_1}\cdots p_v^{Z_v} e^{(n-1)c_{11}A}}{|\alpha^{(k)}\zeta^{(k)}||\gamma_1^{(k)}|^{n_1}\cdots |\gamma_1^{(k)}|^{n_v}}\\
%	& = \frac{p_1^{r_1 + t_1}\cdots p_v^{r_v + t_v} e^{(n-1)c_{11}A}}{|\alpha^{(k)}\zeta^{(k)}|}\cdot
%		\frac{p_1^{\sum_{j=1}^vn_ja_{j1}}\cdots p_v^{\sum_{j=1}^vn_ja_{jv}}}{|\gamma_1^{(k)}|^{n_1}\cdots |\gamma_1^{(k)}|^{n_v}}\\
%	& \leq \frac{p_1^{r_1 + t_1}\cdots p_v^{r_v + t_v} e^{(n-1)c_{11}A}}{\displaystyle \min_{1\leq i \leq n}|\alpha^{(i)}\zeta^{(i)}|}\cdot
%		\left(\frac{p_1^{a_{11}}\cdots p_v^{a_{v1}}}{|\gamma_1^{(k)}|}\right)^{n_1} \cdots
%		\left(\frac{p_1^{a_{v1}}\cdots p_v^{a_{vv}}}{|\gamma_v^{(k)}|}\right)^{n_v} \\
%	& \leq \frac{p_1^{r_1 + t_1}\cdots p_v^{r_v + t_v} e^{(n-1)c_{11}A}}{\displaystyle \min_{1\leq i \leq n}|\alpha^{(i)}\zeta^{(i)}|}\cdot
%		\left(\frac{p_1^{a_{11}}\cdots p_v^{a_{v1}}}{|\gamma_1^{(k)}|} \cdots
%		\frac{p_1^{a_{v1}}\cdots p_v^{a_{vv}}}{|\gamma_v^{(k)}|}\right)^{N} \\
%	& \leq \frac{p_1^{r_1 + t_1}\cdots p_v^{r_v + t_v} e^{(n-1)c_{11}A}}{\displaystyle \min_{1\leq i \leq n}|\alpha^{(i)}\zeta^{(i)}|}\cdot
%		\left(\frac{p_1^{\sum_{j = 1}^v a_{j1}} \cdots p_v^{\sum_{j = 1}^va_{jv}}}{\displaystyle \min_{1\leq i \leq n} |\gamma_1^{(i)} \cdots \gamma_v^{(i)}|}\right)^N \\
%	& = \text{exp}(c_8 + c_9N + (n-1)c_{11}A).
%\end{split}\]
%Note that we used the fact that the matrix $A$ is positive to assert that for each $i \in \{1, \dots, v\}$,
%\[n_i \leq \sum_{j = 1}^v n_ja_{jl} < N.\]
%From here, we deduce that
%\[A < \frac{c_8 + c_9N}{c_{10}-(n-1)c_{11}}.\]
%
%Conversely, if $|\varepsilon_1^{(k)a_1}| = e^{-Ac_{10}}$, we have
%\[\begin{split}
% e^{-Ac_{10}} = |\varepsilon_1^{(k)a_1}|	
%	& = \frac{|(x-1)X-\theta^{(k)}|}{|\alpha^{(k)}\zeta^{(k)}||\gamma_1^{(k)}|^{n_1}\cdots |\gamma_1^{(k)}|^{n_v}}\\	& \geq \frac{e^{-c11A}}{|\alpha^{(k)}\zeta^{(k)}||\gamma_1^{(k)}|^{n_1}\cdots |\gamma_1^{(k)}|^{n_v}}\\
%	& \geq \frac{e^{-c_{11}A}}{\displaystyle \max_{1\leq i \leq n}|\alpha^{(i)}\zeta^{(i)}|\displaystyle \max_{1\leq i \leq n} |\gamma_1^{(i)} \cdots \gamma_v^{(i)}|^N}\\
%	& = \text{exp}(-c_{11}A - c_8'' -c_9''N),
%\end{split}\]
%from which it follows that 
%\[A < \frac{c_8'' + c_9''}{c_{10} -c_{11}}.\]
\end{proof}

%\noindent \textbf{Remark.} To avoid computing 
%\[|\log|\gamma_1^{(i)} \cdots \gamma_v^{(i)}||,\]
%we could just as well compute
%\[|\gamma_1^{(i)} \cdots \gamma_v^{(i)}| - 1\]
%if $|\gamma_1^{(i)} \cdots \gamma_v^{(i)}| > 1$, and 
%\[\frac{1}{|\gamma_1^{(i)} \cdots \gamma_v^{(i)}|}-1\]
%if $|\gamma_1^{(i)} \cdots \gamma_v^{(i)}| < 1$.

Now, put 
$$
C_{16} = \left\lfloor{-\frac{1}{C_{15}} \log \min_{1 \leq j \leq t} |\text{Im} (\theta^{(j)})|}\right\rfloor.
$$
\begin{lemma}
If $\displaystyle \min_{1\leq i\leq n}|(x-1)y-\theta^{(i)}| \leq e^{-C_{15}|a_1|}$, then 
\[|a_1| \leq C_{16}.\]
\end{lemma}

\begin{proof}
\[e^{-C_{15}|a_1|} \geq |(x-1)y-\theta^{(i)}| \geq |\text{Im}(\theta^{(i)})| \geq \min_{1 \leq j \leq t} |\text{Im} (\theta^{(j)})|,\]
hence $|a_1| \leq C_{16}$.
\end{proof}

It follows that 
\[|a_1| \leq \max\left\{\frac{C_{12} + C_{11}C_{13}}{C_{14} - 3C_{15}}, C_{16}\right\}.\] 

%---------------------------------------------------------------------------------------------------------------------------------------------%
\subsection{The reduction strategy} 

The upper bounds on 
\[\left(|a_1|, \sum_{j = 1}^v n_ja_{1j}, \dots, \sum_{j = 1}^v n_ja_{vj}\right)\]
are expected to be very large. Enumeration of the solutions by a naive search at this stage would be prohibitively expensive computationally. Instead, following the methods of \cite{TW3}, we reduce the above bound considerably by applying the LLL-algorithm to approximation lattices associated to the linear forms in logarithms obtained from \eqref{Eq:Sunit}. 

In the standard algorithm for Thue-Mahler equations, this procedure is applied repeatedly to the real/complex and $p$-adic linear forms in logarithms until no further improvement on the bound is possible. The search space for solutions below this reduced bound can then be narrowed further using the Fincke-Pohst algorithm applied to the real/complex and $p$-adic linear forms in logarithms. Lastly, a sieving process and final enumeration of possibilities determines all solutions of the Thue-Mahler equation. In our situation however, after obtaining the above bounds, we apply the LLL algorithm for the $p$-adic linear forms in logarithms only. 

In each step, we let $N_l$ denote the current best upper bound on $\sum_{j = 1}^v n_ja_{lj}$, let $A_0$ denote the current best upper bound on $|a_1|$, and let $M$ denote the current best upper bound on $m$. We will use the notation 
\[b_1 = 1, \quad b_{1+i} = n_i \ \text{ for } i \in \{1, \dots, v\},\]
and
\[ b_{v+2} = a_1\]
of Section \autoref{subsec:FirstStepsSmallBoundsGE} frequently. It will therefore be convenient to let $B_l$ denote the current best upper bound for $|b_l|$ for $l = 1, \dots, v+2$. Then
\[B_1 = 1 \quad \text{ and } \quad B_{v+2} = A_0.\]
For $l = 1, \dots, v$, using that
\[ \sum_{j = 1}^v n_ja_{lj} < N_l, \quad \text{ for } l = 1, \dots, v,\]
we compute
\[|n_l| \leq \max_{1 \leq i \leq v}|n_i| \leq ||A^{-1}||_{\infty}\max_{1 \leq i\leq v}\sum_{j = 1}^v n_j a_{ij}
\leq ||A^{-1}||_{\infty} \max_{1 \leq i\leq v}(N_i) = B_{l+1}.\]

For each $l \in \{1, \dots, v\}$, our expectation is that the LLL algorithm will reduce the upper bound $N_l$ to roughly $\log{N_l}$. Note that we expect the original upper bounds to be of size $10^{120}$ and hence a single application of our $p_l$-adic reduction procedure should yield a new bound $N_l$ that is hopefully much smaller than $3000$. Then we would have
$$
m = \frac{\sum_{j = 1}^{v}n_ja_{lj} + r_l + t_l}{\alpha_l} < \frac{N_l+ r_l + t_l}{\alpha_l} = M < 3000
$$
at which point we could simply search naively (i.e. by brute force)  for all solutions arising from this $S$-unit equation. Of course, if this does not occur, we use our new upper bound on $m$, $M$, to reduce the bounds $N_1, \dots, N_{l-1}, N_{l+1}, \dots, N_v$ via
\[\sum_{j=1}^v n_ja_{ij} = m\alpha_i - r_i - t_i \leq M\alpha_i - r_i - t_i = N_i.\]
We then repeat this procedure with $p_{l+1}$ until $M < 3000$. We note that for all $x$ with $2 \leq x \leq 719$, the bound $m < 3000$ is, in each case, attained in $1$ or $2$ iterations of LLL.

Note also that if a bound on $\sum_{j = 1}^v n_ja_{ij}$ is obtained via Lemma~\ref{Lem:specialcase}, then we similarly compute the bound $M$ on $m$ and enter the final search. We may do so because this bound always furnishes a bound on $m$ that is smaller than $3000$ for $x$ with $2 \leq x \leq 719$. 

Lastly, rather than testing each possible tuple $(|a_1|,|n_1|, \dots, |n_v|)$ as in \cite{TW3}, our brute force search simply checks for solutions of \eqref{TM-start} using the smallest bound obtained on $m$. Because of this, we may omit the reduction procedures on the real/complex linear forms in logarithms, and furthermore, we need only to reduce the bounds on $\sum_{j = 1}^v n_ja_{ij}$ so that $M < 3000$. 

%---------------------------------------------------------------------------------------------------------------------------------------------%

\subsection{The $p_l$-adic reduction procedure}

In this section, we set some notation and give some preliminaries for the $p_l$-adic reduction procedures. Consider a fixed index $l \in \{1, \dots, v\}$. Following Section \ref{subsec:FirstStepsSmallBoundsGE}, we have
\[\ord_{p_l}(\alpha_1) \geq \min_{2\leq i\leq v+2} \ord_{p_l}(\alpha_i) \  \text{ and } \ \ord_{p_l}(\alpha_{1h}) \geq \min_{2\leq i\leq v+2}(\alpha_{ih}) \quad h = (1, \dots, s).\]

Let $I$ be the set of all indices $i' \in \{2, \dots, v+2\}$ for which
\[\ord_{p_l}(\alpha_{i'}) = \min_{2\leq i\leq v+2} \ord_{p_l}(\alpha_i).\]
We will identify two cases, the \textit{special case} and the \textit{general case}. The special case occurs when there is some index $i' \in I$ such that $\alpha_i/\alpha_{i'} \in \mathbb{Q}_{p_l}$ for $i = 1, \dots, v+2$. The general case is when there is no such index. 

In the special case, let $\hat{i}$ be an arbitrary index in $I$ for which $\alpha_i/\alpha_{\hat{i}} \in \mathbb{Q}_{p_l}$ for every $i = 1, \dots, v+2$. We further define
\[\beta_i = - \frac{\alpha_i}{\alpha_{\hat{i}}} \quad i = 1, \dots, v+2,\]
and 
\[\Lambda'_l = \frac{1}{\alpha_{\hat{i}}}\Lambda_l = \sum_{i = 1}^{v+2} b_i(-\beta_i).\]

In the general case, we fix an $h \in \{1, \dots, s\}$ arbitrarily. Then we let $\hat{i}$ be an index in $\{2, \dots, v+2\}$ such that 
\[ \ord_{p_l}(\alpha_{\hat{i}h}) = \min_{2\leq i\leq v+2}(\alpha_{ih}),\]
and define
\[\beta_i = - \frac{\alpha_{ih}}{\alpha_{\hat{i}h}} \quad i = 1, \dots, v+2,\]
and 
\[\Lambda'_l = \frac{1}{\alpha_{\hat{i}h}}\Lambda_{lh} = \sum_{i = 1}^{v+2} b_i(-\beta_i).\]
Now in both cases we have $\beta_i \in \mathbb{Z}_{p_l}$ for $i = 1, \dots, v+2$. 

\begin{lemma} \label{Lem:19.1}
Suppose
\[\sum_{i = 1}^v n_{i}a_{li} > \frac{1}{p_l-1} - \ord_{p_l}(\delta_2).\]
In the special case, we have 
\[\ord_{p_l}(\Lambda_l') = \sum_{i = 1}^v n_{i}a_{li} + d_l\]
with
\[d_l = \ord_{p_l}(\delta_2) - \ord_{p_l}(\alpha_{\hat{i}}).\]
In the general case we have
\[\ord_{p_l}(\Lambda_{l}') \geq \sum_{i = 1}^v n_{i}a_{li} + d_l\]
with 
\[d_l = \ord_{p_l}(\delta_2) - \frac{1}{2}\ord_{p_l}(\text{Disc}(G(t))) - \ord_{p_l}(\alpha_{\hat{i}h}).\]
\end{lemma}

\begin{proof}
Immediate from Lemma \ref{Lem:discG} and Lemma \ref{Lem:Lambda}. 
\end{proof}

We now describe the $p_l$-adic reduction procedure. Let $\mu, W_2, \dots, W_{v+2}$ denote positive integers. These are parameters that we will need to balance in order to obtain a good reduction for the upper bound of 
$\sum_{i = 1}^v n_{i}a_{li}$. We will discuss how to choose these parameters later in this section. For each $x \in \mathbb{Z}_{p_l}$, let $x^{\{\mu\}}$ denote the unique rational integer in $[0,p_l^{\mu} - 1]$ such that $\ord_{p_l}(x - x^{\mu}) \geq \mu$ (ie. $x \equiv x^{\{\mu\}} \pmod{p_l^{\mu}}$). Let $\Gamma_{\mu}$ be the $(v+1)$-dimensional lattice generated by the column vectors of the matrix
\[A_{\mu} = 
\begin{pmatrix}
W_2 & 		&				&				&		&	&	\\
	& \ddots	& 				&				& 0		& 	&	\\
	&		& W_{\hat{i} - 1}	&				&		&	&	\\
	& 		& 				& W_{\hat{i} + 1}	&		&	&	\\	
	& 0		& 				& 				&\ddots	&	&	\\
W_{\hat{i}}\beta_2^{\{\mu\}}& \cdots & W_{\hat{i}}\beta_{\hat{i} - 1}^{\{\mu\}} & W_{\hat{i}}\beta_{\hat{i} + 1}^{\{\mu\}}& \cdots &W_{\hat{i}}\beta_{v+2}^{\{\mu\}}& W_{\hat{i}}p_l^{\mu}\\	
\end{pmatrix}.\]
Put
\[ \lambda = \frac{1}{p_l^{\mu}} \sum_{i = 1}^{v+2} b_i\left(-\beta_i^{\{\mu\}}\right)\]
and
\[\mathbf{y} = 
\begin{pmatrix}
0 \\
\vdots \\
0 \\
-W_{\hat{i}}\beta_1^{\mu}
\end{pmatrix}
\in \mathbb{Z}^{v+1}.\]
Of course, we must compute the $\beta_i$ to $p_l$-adic precision at least $\mu$ in order to avoid errors here. We observe that $\mathbf{y} \in \Gamma_{\mu}$ if and only if $\mathbf{y}= \mathbf{0}$. To see that this is true, note that $\mathbf{y} \in \Gamma_{\mu}$ means there are integers $z_1, \dots, z_{v+1}$ such that $\mathbf{y}=A_{\mu}[z_1, \dots, z_{v+1}]^{T}$. The last equation of this equivalence forces $z_1 = \dots = z_{v} = 0$ and $-\beta_1^{\{\mu\}} = z_{v+1}p_l^{m}$. Since $\beta_1^{\{\mu\}} \in [0, p_l^m - 1]$, we must then have $z_{v+1} = 0$ also. Hence $\mathbf{y} = \mathbf{0}$. 

Put
\[Q = \sum_{i = 2}^{v+2} W_i^2 B_i^2.\]

\begin{lemma} \label{lem:LLL}
If $\ell(\Gamma_{\mu},\mathbf{y}) > Q^{1/2}$ then
\[\sum_{i = 1}^v n_{i}a_{li} \leq \max\left\{ \frac{1}{p_l-1} - \ord_{p_l}(\delta_2), \mu - d_l - 1,0\right\}\]
\end{lemma}

\begin{proof}
We prove the contrapositive. Assume 
\[\sum_{i = 1}^v n_{i}a_{li} > \frac{1}{p_l-1} - \ord_{p_l}(\delta_2), \quad \sum_{i = 1}^v n_{i}a_{li} > \mu - d_l 
\quad \text{ and } \quad \sum_{i = 1}^v n_{i}a_{li} > 0.\]
Consider the vector
\[\mathbf{x} = A_{\mu}
\begin{pmatrix}
b_2\\
\vdots\\
b_{\hat{i}-1}\\
b_{\hat{i}+1}\\
\vdots\\
b_{v+2}\\
\lambda
\end{pmatrix}
= 
\begin{pmatrix}
W_2b_2\\
\vdots\\
W_{\hat{i}-1}b_{\hat{i}-1}\\
W_{\hat{i}+1}b_{\hat{i}+1}\\
\vdots\\
W_{v+2}b_{v+2}\\
-W_{\hat{i}}b_{\hat{i}}
\end{pmatrix}
+ \mathbf{y}.\]
By Lemma~\ref{Lem:19.1},  
\[\ord_{p_l}\left( \sum_{i=1}^{v+2}b_i(-\beta_i)\right) = \ord_{p_l}(\Lambda_l') \geq\sum_{i = 1}^v n_{i}a_{li} + d_l \geq \mu.\]
Since $\ord_{p_l}(\beta_i^{\{\mu\}} - \beta_i) \geq \mu$ for $i = 1, \dots, v+2$, it follows that
\[\ord_{p_l}\left( \sum_{i=1}^{v+2}b_i(-\beta_i^{\{\mu\}})\right) \geq \mu,\]
so that $\lambda \in \mathbb{Z}$. Hence $\mathbf{x} \in \Gamma_{\mu}$. Now $\sum_{i = 1}^v n_{i}a_{li} > 0$ so that there exists some $i$ such that $n_ia_{li} \neq 0$, and in particular, $b_{1+i} = n_i \neq 0$. Thus we cannot have $\mathbf{x} = \mathbf{y}$. Therefore, 
\[\ell(\Gamma_{\mu}, \mathbf{y})^2 \leq |\mathbf{x} - \mathbf{y}|^2 = \sum_{i = 2}^{v+2}W_i^2 b_i^2
\leq  \sum_{i = 2}^{v+2}W_i^2 |b_i|^2 \leq  \sum_{i = 2}^{v+2}W_i^2 B_i^2 = Q.\]
\end{proof}

The reduction procedure works as follows. Taking $A_{\mu}$ as input, we first compute an LLL-reduced basis for $\Gamma_{\mu}$. Then, we find a lower bound for $\ell(\Gamma_{\mu}, \mathbf{y})$. If the lower bound is not greater than $Q^{1/2}$ so that Lemma \ref{lem:LLL} does not give a new upper bound, we increase $\mu$ and try the procedure again. If we find that several increases of $\mu$ have failed to yield a new upper bound $N_l$ and that the value of $\mu$ has become significantly larger than it was initially, we move onto the next $l \in \{1, \dots, v\}$.

If the lower bound is greater than $Q^{1/2}$, Lemma \ref{lem:LLL} gives a new upper bound $N_l$ for $\sum_{i = 1}^v n_{i}a_{li}$ and hence for $m$
\[m = \frac{\sum_{j = 1}^{v}n_ja_{lj} + r_l + t_l}{\alpha_l} < \frac{N_l+ r_l + t_l}{\alpha_l} = M.\]
If $M < 3000$, we exit the algorithm and enter the brute force search. Otherwise, we update the bounds $N_1, \dots, N_{l-1}, N_{l+1}, \dots, N_v$ via
\[\sum_{j=1}^v n_ja_{ij} = m\alpha_i - r_i - t_i \leq M\alpha_i - r_i - t_i = N_i.\]
Then using 
\[|n_l| \leq \max_{1 \leq i \leq v}|n_i| \leq ||A^{-1}||_{\infty}\max_{1 \leq i\leq v}\sum_{j = 1}^v n_j a_{ij}
\leq ||A^{-1}||_{\infty} \max_{1 \leq i\leq v}(N_i) = B_{l+1}.\]
we update the $B_i$ and repeat the above procedure until $M < 3000$ or until no further improvement can be made on the $B_i$, in which case we move onto the next $l \in \{1, \dots, v\}$.

%---------------------------------------------------------------------------------------------------------------------------------------------%

\subsection{Computational conclusions}

Bottlenecks for this computation are generating the class group, generating the ring of integers of the splitting field of $K$ (this is entirely because of a Magma issue and cannot be avoided) and generating the unit group. 

%(it may be possible to avoid generating the unit group, however, at least according to Mike Jacobson, using the map from $\mathbb{Z}^v$ to the class group, though it's not so clear that we can still do all the computations that we will need to do using this approach. For the time being, we use Magma's IndependentUnits function for this.)


%
%Now, we use LLL to reduce this bound. Each time we do so, we compute $m$, choose the lowest possible value, and then update all $n_i$. We repeat this process until $m < 1000$, after which point we just enter a brute force search. 
%
%
%
%
%\texbf{Old TM Writeup}
%
%To solve these equations, we will argue as in Tzanakis and de Weger \cite{TW3}. Other than to establish notation, we will, for the most part, restrict ourselves to commenting about where our approach to solving the special equation (\ref{TM-start}) differs from that used to treat a more general Thue-Mahler equation. Fix $x$ with $2 \leq x < 750$.
%Let $\xi$ be a root of $F_x(y,1)=0$ and put $K = \mathbb{Q}(\xi)$. Now \eqref{Eq:main} is equivalent to
%\begin{equation}
%(x-1)N_{K/\mathbb{Q}}(y-\xi) =  p_1^{\alpha_1}\dots p_v^{\alpha_v}.
%\end{equation}
%Put $\tilde{y} = (x-1)y$ and $\theta = (x-1)\xi$ so that \eqref{Eq:main} is equivalent to
%\begin{equation} \label{Eq:norm}
%N_{K/\mathbb{Q}}(\tilde{y}-\theta) =  (x-1)^{3}p_1^{\alpha_1}\dots p_v^{\alpha_v}.
%\end{equation}
%Note that $K = \mathbb{Q}(\theta)$, $[K:\mathbb{Q}] = 4$ and the minimal polynomial $g(t)$ of $\theta$ is monic, i.e.
%\[g(t) = t^4 + (x-1)t^3 + (x-1)^2t^2 + (x-1)^3t + x(x-1)^3,\]
%so that $\theta$ is an algebraic integer. 
%
%For each $i$, let 
%\[(p_i) = \prod_{j = 1}^{m_i} \mathfrak{p}_{ij}^{e_{ij}}\]
%be the factorization of $p_i$ into prime ideals in the ring of integers $\mathcal{O}_K$ of $K$, and let $f_{ij}$ be the residue degree of $\mathfrak{p}_{ij}$ over $p_i$. Then, since $N(\mathfrak{p}_{ij}) = p_i^{f_{ij}}$, \eqref{Eq:norm} implies finitely many ideal equations of the form
%\begin{equation} \label{Eq:ideals}
%(\tilde{y}-\theta) = \mathfrak{a} \prod_{j = 1}^{m_1} \mathfrak{p}_{1j}^{e_{1j}} \cdots \prod_{j = 1}^{m_v} \mathfrak{p}_{vj}^{e_{vj}}
%\end{equation}
%where $\mathfrak{a}$ is an ideal of norm $|(x-1)^3|$ and the $z_{ij}$ are unknown integers related to $m$ by $\sum_{j = 1}^{m_i} f_{ij}z_{ij} = \alpha_i = ma_i$. 
%
%By applying the Prime Ideal Removing Lemma (Lemma 1 of Tzanakis and de Weger \cite{TW3}), the task of solving the finite set of equations represented by \eqref{Eq:ideals} is reduced to solving the set of equations of the form
%\begin{equation} \label{Eq:PIRL}
%(\tilde{y}-\theta) = \mathfrak{a}\mathfrak{b} \mathfrak{p}_1^{u_1} \cdots \mathfrak{p}_v^{u_v}
%\end{equation}
%in integer variables $x,u_1, \dots, u_v$ with $u_i \geq 0$ for $i = 1, \dots, v$. Here
%\begin{itemize}
%\item $\mathfrak{a}$ is an ideal of $\mathcal{O}_K$ of norm $|(x-1)^3|$.
%\item $\mathfrak{p}_i$ is a prime ideal $\mathcal{O}_K$ above $p_i$ with ramification index and residue degree both equal to 1. Note that such an ideal must exist since $g(t)$ always has a linear factor over $\mathbb{Q}_{p_i}[t]$.
%\item $\mathfrak{b}$ is an ideal of $\mathcal{O}_K$ whose prime ideal factors are those that divide one of the $p_i$, but are not equal to one of the $\mathfrak{p}_i$
%\item $u_i + \ord_{p_i}(N(\mathfrak{b})) = \alpha_i = ma_i$.
%\end{itemize}
%
%Consider a particular instance of \eqref{Eq:PIRL}, i.e. fix choices for $\mathfrak{a}, \mathfrak{b}, \mathfrak{p}_1, \dots, \mathfrak{p}_v$. Fix a complete set of fundamental units of $\mathcal{O}_K: \varepsilon_1, \dots, \varepsilon_r$. Here $r = s + t -1$, where $s$ denotes the number of real embeddings of $K$ into $\mathbb{C}$ and $t$ denotes the number of complex conjugate pairs of non-real embeddings of $K$ into $\mathbb{C}$. For $i = 1, \dots, v$ let $h_i$ be the smallest positive integer for which $\mathfrak{p}_i^{h_i}$ is principal and let $\pi_i \in \mathcal{O}_K$ be a generator for $\mathfrak{p}_i^{h_i}$. Then
%\[(\tilde{y}-\theta) = \mathfrak{a}\mathfrak{b} \mathfrak{p}_1^{s_1} \cdots \mathfrak{p}_v^{s_v}(\pi_1^{n_1})\cdots (\pi_v^{n_v})\]
%where the unknown integers $s_1, \dots, s_v, n_1, \dots, n_v$ satisfy 
%\[u_i = h_in_i + s_i, \quad n_i \geq 0, \quad 0 \leq s_i < h_i.\]
%
%Since the $s_i$ vary in a finite set, we treat them as parameters. Now fix values for $s_1, \dots, s_v$. The ideal $\mathfrak{a}\mathfrak{b} \mathfrak{p}_1^{s_1} \cdots \mathfrak{p}_v^{s_v}$ is necessarily principal, and we fix $\alpha \in \mathcal{O}_K$ such that 
%\[(\alpha) = \mathfrak{a}\mathfrak{b} \mathfrak{p}_1^{s_1} \cdots \mathfrak{p}_v^{s_v}.\]
%This is an equality of principal ideals and since generators of principal ideals are determined only up to multiplication by units, we are led to the equation
%\begin{equation} \label{Eq:main2}
%\tilde{y} - \theta = \alpha \zeta \varepsilon_1^{a_1} \cdots \varepsilon_r^{a_r}\pi_1^{n_1}\cdots \pi_v^{n_v}
%\end{equation}
%with unknowns $a_i \in \mathbb{Z}$, $n_i \in \mathbb{Z}_{\geq 0}$, and $\zeta$ in the set $T$ of roots of unity in $\mathcal{O}_K$. Since $T$ is also finite, we will treat $\zeta$ as another parameter. 
%
%To summarize, our original problem of solving \eqref{Eq:main} has been reduced to the problem of solving finitely many equations of the form \eqref{Eq:main2} for the variables 
%$$
%\tilde{y},a_1, \dots, a_r, n_1, \dots, n_v. 
%$$
%Since the $\pi_i$ are determined by the $\mathfrak{p}_i$, and since $\alpha$ is a fixed ideal generator, the parameters in \eqref{Eq:main2} are $\zeta, \mathfrak{a}, \mathfrak{b}, \mathfrak{p}_1, \dots, \mathfrak{p}_v, s_1, \dots, s_v$. We must solve \eqref{Eq:main2} for each combination of possible values of these parameters. Note that each solution $(y, \alpha_1, \dots, \alpha_v)$ of \eqref{Eq:main} corresponds to a solution $(\tilde{y}, a_1, \dots, a_r, n_1, \dots, n_v)$. 
%
%From here, we deduce a so-called $S$-unit equation. In doing so, we eliminate the variable $
%\tilde{y}$ and set ourselves up to bound the exponents $a_1, \dots, a_r, n_1, \dots, n_v$ where
%\begin{equation} \label{Eq:main3}
%ma_i = \alpha_i = n_ih_i + s_i + t_i,
%\end{equation}
%and $t_i = \ord_{p_i}(N(\mathfrak{b}))$.
%
%Let $p \in \{p_1, \dots, p_v, \infty\}$. Denote the roots of $g(t)$ in $\overline{\mathbb{Q}_p}$ (where $\overline{\mathbb{Q}_{\infty}} = \overline{\mathbb{R}} = \mathbb{C}$) by $\theta^{(1)}, \dots, \theta^{(n)}$. Let $i_0, j, k \in \{1, \dots, 4\}$ be distinct indices and consider the three embeddings of $K$ into $\overline{\mathbb{Q}_p}$ defined by $\theta \mapsto \theta^{(i_0)}, \theta^{(j)}, \theta^{(k)}$. We use $z^{(i)}$ to denote the image of $z$ under the embedding $\theta \mapsto \theta^{(i)}$. Applying these embeddings to $\beta = \tilde{y} - \theta$ yields
%\begin{equation} \label{Eq:SUnit}
%\lambda = \delta_1 \prod_{i = 1}^v \left( \frac{\pi_i^{(k)}}{\pi_i^{(j)}}\right)^{n_i} \prod_{i = 1}^r \left( \frac{\varepsilon_i^{(k)}}{\varepsilon_i^{(j)}}\right)^{a_i} - 1 = \delta_2  \prod_{i = 1}^v \left( \frac{\pi_i^{(i_0)}}{\pi_i^{(j)}}\right)^{n_i} \prod_{i = 1}^r \left( \frac{\varepsilon_i^{(i_0)}}{\varepsilon_i^{(j)}}\right)^{a_i},
%\end{equation}
%where
%\[\delta_1 = \frac{\theta^{(i_0)} - \theta^{(j)}}{\theta^{(i_0)} - \theta^{(k)}}\cdot\frac{\alpha^{(k)}}{\alpha^{(j)}}, \quad \delta_2 = \frac{\theta^{(j)} - \theta^{(k)}}{\theta^{(k)} - \theta^{(i_0)}}\cdot \frac{\alpha^{(i_0)}}{\alpha^{(j)}}\]
%are constants. 
%
%\end{theorem}
%Let $l \in \{1, \dots, v\}$. If $\ord_{p_l} \neq 0$, then 
%\[n_l = \frac{1}{h_l}(\min\{\ord_{p_l}(\delta_1), 0\} - \ord_{p_l}(\delta_2)).\]
%\end{lemma}
%In this case, it follows from \eqref{Eq:main3} that
%\[m = \frac{n_l h_l + s_l + t_l}{a_l}\]
%so that this $S$-unit equation is solved completely once we test whether this value of $m$ yields a solution of \eqref{Eq:main}. If not, we proceed as follows. 
%
%At this point, following \cite{TW3}, a very large upper bound for ${(a_1, \dots, a_r, n_1, \dots, n_v)}$ is derived using the theory of linear forms in logarithms. However, in practice this requires that we compute the absolute logarithmic height of all terms of our so-called $S$-unit equation, \eqref{Eq:SUnit}. More often than not, this is a computational bottleneck, and so it is best to avoid it overall. In particular, Tzanakis-de Weger's approach requires that we compute the absolute logarithmic height of each algebraic number in the product of \eqref{Eq:SUnit}. Unfortunately, in many such instances, the fundamental units may be very large, with each coefficient having over $100,000$ digits in their representation. Similarly, the generators of our principal ideals may also be very large, making elementary operations on them (such as division) a very time-consuming process. In one particular instance of $x = 60$, for example, each coefficient of $\alpha$ was found to have over $20,000$ digits. As a result of this, computing the absolute logarithmic height of these elements, a process which must be done for each choice of parameters $\zeta, \mathfrak{a}, \mathfrak{b}, \mathfrak{p}_1, \dots, \mathfrak{p}_v, s_1, \dots, s_v$, is a computational bottleneck which is best avoided altogether. Instead of this approach, we appeal to \cite{BugeaudGyory} to generate a much larger upper bound. Largely following the notation of \cite{BugeaudGyory}, we now describe this bound.  
%
%Put $\alpha_i = 4\mu_i + \nu_i$ with $\mu_i, \nu_i \in \mathbb{Z}$, $0 \leq \nu_i < 4$ for $i = 1, \dots, v$ and let $R_K$ and $h_K$ be the regulator and class number of $K$, respectively. Let $T$ be the set of all extensions to $K$ of the places in $\{p_1, \dots, p_v\}$. Let $P$ denote $\max\{p_1, \dots, p_v\}$, and let $R_T$ denote the $T$-regulator of $K$. Further, let $H$ be an upper bound for the maximum absolute value of the coefficients of $F$; that is, let $H = \max\{3, x\}$. In the notation of \cite{BugeaudGyory}, let $\log^*(a)$ denote $\max(\log(a), 1)$ and let
%\[C_8 = \text{exp}\left\{c_{24}P^{24} R_T (\log^*R_T)\left(\frac{\log^*(PR_T)}{\log^*P}\right)(R_{K} + vh_{K} + \log(HB'))\right\},\]
%where $B' \leq 3HP^{4v} = 3\max\{3,x\}P^{4v}$ and 
%\[c_{24} = 3^{v + 26}\cdot (v+1)^{5v+17}\cdot 24^{3v + 19}.\]
%From \cite{BugeaudGyory}, we now have $p_i^{\mu_i} \leq C_8$. Additionally, it is shown in \cite{BugeaudGyory} that 
%\[\frac{\log^*(PR_T)}{\log^*P} \leq 2\log^*R_T,\] 
%so that 
%\[C_8 \leq \text{exp}\left\{2c_{24}P^{24} R_T (\log^*R_T)^2(R_{K} + vh_{K} + \log(HB'))\right\}.\] 
%Lastly, by \cite{BugeaudGyory}, we have that $R_T \leq R_Kh_K(4\log^*P)^{4v}$, where $R_K$ is easily calculated in Magma. In some instances, $h_K$ may be computationally costly, and so instead we simply appeal to the upper bound of \cite{BugeaudGyory}, namely
%\[h_K < \frac{|\text{Disc}(K)|^{1/2}(\log|\text{Disc}(K)|)^{3}}{3!R_M}.\]
%
%Now we have all of the components necessary to explicitly compute an upper bound on $C_8$, denoted $C_9$, from which it follows that
%\[\mu_i \leq \frac{\log(C_9)}{\log{p_i}}\]
%and hence
%\[ma_i = \alpha_i = 4\mu_i + \nu_i < \frac{4\log(C_9)}{\log(p_i)} + \nu_i < \frac{4\log(C_9)}{\log(p_i)} + 4.\] 
%We thus have  
%\[m < \frac{4\log(C_9)}{a_v\log(p_i)} + \frac{4}{a_v} = C_{10}\]
%where $a_v$ denotes the exponent on the largest prime $P$ of $\{p_1, \dots, p_v\}$. 
%From \eqref{Eq:main3}, it follows that
%\begin{equation} \label{Eq:nBound}
%n_i = \frac{ma_i - s_i - t_i}{h_i} \leq \frac{C_{10}a_i - s_i - t_i}{h_i}.
%\end{equation}
%
%Now, returning to the method of \cite{TW3}, for each $n_i$ where $i = 1, \dots v$, we reduce the above bound considerably by applying the LLL-algorithm to approximation lattices associated to the linear forms in logarithms obtained from \eqref{Eq:SUnit}. In the standard algorithm for Thue-Mahler equations, this procedure is applied repeatedly to the real/complex and $p$-adic linear forms in logarithms until no further improvement in the bound is possible. The solutions below this reduced bound can then be reduced further using the Fincke-Pohst algorithm applied to the real/complex and $p$-adic linear forms in logarithms. Finally, a sieving process and final enumeration of possibilities determines all solutions of the Thue-Mahler equation.
%
%Unlike in \cite{TW3}, after obtaining the bound \eqref{Eq:nBound} on each $n_i$, we apply the LLL algorithm for the $p$-adic linear forms in logarithms only. With each iteration of this technique, we obtain a new upper bound on, say $n_i$, from which we compute the new upper bound on $m$. From this single iteration, we then use \eqref{Eq:main3} to generate new upper bounds on the remaining $n_1, \dots, n_{i-1}, n_{i+1}, \dots, n_v$. We repeat this process until $m < 1000$, after which point we simply enter a brute force search. In the rare cases where this does not occur, we implement the Fincke-Pohst technique of the Thue-Mahler solver for $p$-adic linear forms, again reducing the $n_i$ simultaneously until $m < 1000$. We note that for all $x \in [2,750]$, the bound $m < 1000$ is always attained by this stage. [CAN MAKE THIS FASTER ACTUALLY (ie. change 1000 to 3000)- WILL EDIT ONCE IMPLEMENTED]
%
%Lastly, rather than testing each possible tuple $(a_1, \dots, a_r, n_1, \dots, n_v)$ as in \cite{TW3}, our brute force search simply checks for solutions of \eqref{TM-start} using the smallest bound obtained on $m$. Because of this, we may omit the reduction procedures on the real/complex linear forms in logarithms, and furthermore, we need only to reduce the bounds on $(n_1, \dots, n_v)$ so that $m < 1000$. In fact, this final computation is fast enough that we may even use $m < 3000$ as our cutoff. 

An implementation of this algorithm is available at
\begin{center}
\url{http://www.nt.math.ubc.ca/BeGhKr/GESolverCode}.
\end{center}

As before, we have, for each $x$, solutions $(x,y,m) = (x,-1,1), (x,0,1),$ and $(x,x,5)$. For $x$ with $2 \leq x \leq 719$, we find additional solutions $(x,y,m)$ among 
$$
\begin{array}{c}
(4,1,2), (5,2,3), (10,-2,2), (10,-6,4), (30,2,2), (60,-3,2), \\
(120, 3,2), (204,-4,2), (340, 4,2), (520,-5,2).\\
\end{array}
$$

Altogether, this computation took 3 weeks on a $16$-core 2013 vintage MacPro, with the case $x=710$ being the most time-consuming, taking roughly $5$ days and $16$ hours on a single core. This is the better timing attained for this value of $x$ from our two approaches, computed using the class group to generate the $S$-unit equations. The most time-consuming job when computing the class group was $x = 719$, which took $10$ days and $8$ hours. However, using our alternate code, the better timing for $x = 719$ was only $2$ hours. Without computing the class group, the most time-consuming process was $x = 654$, which took $2$ days and $7$ hours. However, this is the faster timing that was attained for this value of $x$, as computing the class group took roughly $4$ days and $8$ hours. 

We list below some timings for our computation. These times are listed in seconds, with the second column indicating the algorithm requiring the computation of the class group, and the third column indicating the time taken by the algorithm which avoids the class group. In implementing these two algorithms, we terminated the latter algorithm if the program ran longer than its class group counterpart took. From these timings, it is clear that it is not always easy to predict which algorithm will prove faster. 

\[\begin{array}{c | c| c|c}
x & \text{Timing with } \Cl(K) & \text{Timing without } \Cl(K) & \text{Solutions} \\ \hline \hline
689 & 647.269 	& \text{Terminated} & [ -1, 1 ],[ 0, 1 ],[ 689, 5 ]  \\
690  & 215306.420 & \text{Terminated} & [ -1, 1 ],[ 0, 1 ],[ 690, 5 ] \\
691  & 456194.210 &   1821.049 & [ -1, 1 ],[ 0, 1 ],[ 691, 5 ] \\
692  & 152385.640 & \text{Terminated} & [ -1, 1 ],[ 0, 1 ],[ 692, 5 ] \\
693  & 36922.540  &    1908.230  & [ -1, 1 ],[ 0, 1 ],[ 693, 5 ]  \\
694  & 8288.190     & \text{Terminated}    &   [ -1, 1 ],[ 0, 1 ],[ 694, 5 ]   \\
695  & 362453.820  &   9786.649  &  [ -1, 1 ],[ 0, 1 ],[ 695, 5 ]  \\
696  & 76273.470 & \text{Terminated}  &    [ -1, 1 ],[ 0, 1 ],[ 696, 5 ] \\
697  & 14537.219  &    725.340  & [ -1, 1 ],[ 0, 1 ],[ 697, 5 ]  \\
698  & 451700.650 &    2708.920 &    [ -1, 1 ],[ 0, 1 ],[ 698, 5 ] \\
\end{array}\]
 
Full computational details are available at
\begin{center}
\url{http://www.nt.math.ubc.ca/BeGhKr/GESolverData},
\end{center}
including the timings obtained for each value of $x$, under both iterations of the algorithm. 

This completes the proof of Theorem \ref{main-thm2}.


%-------------------------------------------------------------------------------------------------------------
\section{Bounding $C(k,d)$ : the proof of Proposition \ref{Cee}} \label{Cee-proof}
%-------------------------------------------------------------------------------------------------------------


To complete the proof of Proposition \ref{Cee}, from (\ref{C-upper}), it remains to show that $\prod_{p \mid d} p^{1/(p-1)}  < 2 \log d$, provided $d > 2$. 
We verify this by explicit calculation for all $d \leq d_0 = 10^5$. 

Since $\log p/ (p-1)$ is decreasing in $p$, if we denote by $\omega (d)$ the number of distinct prime divisors of $d$, we have
\begin{equation} \label{ome}
\sum_{p\mid d} \frac{\log p}{p-1} \leq \sum_{p \leq p_{\omega (d)}} \frac{\log p}{p-1},
\end{equation}
where $p_k$ denotes the $k$th smallest prime. Since we have
$$
 \sum_{p \leq p_{10}} \frac{\log p}{p-1} < \log (2 \log (d_0)),
 $$
 we may thus suppose that $\omega (d) \geq 11$, whereby 
 $$
 d \geq d_1 = 2 \cdot 3 \cdot 5 \cdot 7 \cdot 11 \cdot 13 \cdot 17 \cdot 19 \cdot 23 \cdot 29 \cdot 31 = 200560490130.
$$
The fact that
$$
 \sum_{p \leq p_{21}} \frac{\log p}{p-1} < \log (2 \log (d_1))
 $$
 thus implies that $\omega (d) \geq 22$ and
 $$
 d \geq d_2 = \prod_{1 \leq i \leq 22} p_i > 3 \cdot 10^{30}.
 $$
 We iterate this argument, finding that
 $$
 \sum_{p \leq p_{\kappa (j)}} \frac{\log p}{p-1} < \log (2 \log (d_j)),
 $$
 so that 
 $$
 d \geq d_{j+1} = \prod_{1 \leq i \leq \kappa (j) + 1} p_i,
 $$
for $j=0, 1, 2, 3, 4$ and $5$, where
 $$
 \kappa (0)=10, \; \kappa (1) = 21, \; \kappa (2) = 50, \, \kappa (3) = 130, \; \kappa (4) = 361 \mbox{ and } \kappa (5) = 1055.
 $$
We thus have that $\omega (d) \geq 1056$ and 
$$
d \geq \prod_{1 \leq i \leq 1056} p_i > e^{8316}.
$$
We may thus apply Th\'eor\`eme 12 of  of Robin \cite{Ro} to conclude that
$$
\omega (d) \leq \frac{\log d}{\log \log d} + 1.4573 \, \frac{\log d}{(\log \log d)^2}  <  \frac{7 \, \log d}{6 \, \log \log d},
$$
while the Corollary to Theorem 3 of Rosser-Schoenfeld yields 
$$
p_n < n (\log n + \log \log n) < \frac{10}{9} n \log n.
$$

It follows that
$$
p_{\omega (d)} < \frac{35}{27} \, \frac{\log d}{\log \log d} \log \left(  \frac{7 \, \log d}{6 \, \log \log d} \right) <  \frac{35}{27} \, \log d.
$$


By  Theorem 6 of Rosser-Schoenfeld, we have
\begin{equation} \label{RoSc}
\sum_{p < x} \frac{\log p}{p} < \log x - 1.33258 + \frac{1}{2 \log x},
\end{equation}
for all $x \geq 319$. Also, if $j \geq 2$,
\begin{equation} \label{inty}
\int_k^\infty \frac{\log u}{u^j} du = \frac{(j-1) \log (k) +1}{(j-1)^2 k^{j-1}}.
\end{equation}
For $2 \leq j \leq 10$, we have
$$
 \sum_{p < x} \frac{\log p}{p^j} <  \sum_{p < 10^6} \frac{\log p}{p^j} +  \sum_{p > 10^6} \frac{\log p}{p^j} <  \sum_{p < 10^6} \frac{\log p}{p^j}
 + \int_{10^6}^\infty \frac{\log u}{u^j} du,
 $$
 whereby
 \begin{equation} \label{little}
 \sum_{p < x} \frac{\log p}{p^j} <  \sum_{p < 10^6} \frac{\log p}{p^j} + \frac{(j-1) \log (10^6) +1}{(j-1)^2 10^{6(j-1)}}.
 \end{equation}
 By explicit computation, from (\ref{little}), we find that
 \begin{equation} \label{quip}
 \sum_{j=2}^{10}  \sum_{p < x} \frac{\log p}{p^j} < 0.755,
 \end{equation}
while, from (\ref{inty}), 
 \begin{equation} \label{quip2}
 \sum_{j \geq 11}  \sum_{p < x} \frac{\log p}{p^j} <  \sum_{j \geq 11}   \frac{(j-1) \log (2) +1}{(j-1)^2 2^{j-1}} < \sum_{j \geq 11} \frac{1}{(j-1)2^{j-1}}.
 \end{equation}
Evaluating this last sum explicitly, it follows that 
$$
\sum_{j \geq 2} \sum_{p < x} \frac{\log p}{p^j} < 0.755 + \log (2)- \frac{447047}{645120} < 0.756,
$$
whereby, from (\ref{RoSc}), if $x \geq 319$, 
$$
\sum_{p <  x} \frac{\log p}{p-1} < \log x - 0.489.
$$

Applying this last inequality with $x = \frac{35}{27} \log d > \frac{35}{27} \cdot 8316 = 10780$, we conclude from our earlier arguments that
$$
\sum_{p\mid d} \frac{\log p}{p-1} < \log \log d.
$$
This completes the  proof of Proposition \ref{Cee}.


%-------------------------------------------------------------------------------------------------------------
\section{Concluding remarks} \label{conclude}
%-------------------------------------------------------------------------------------------------------------

The techniques employed in this chapter may be used, with very minor modifications, to treat equation (\ref{eq-main}), subject to condition (\ref{condition}), with the variables $x$ and $y$ integers (rather than just positive integers). Since 
$$
\frac{(-a-1)^3-1}{(-a-1)-1} = \frac{a^3-1}{a-1},
$$
in addition to the known solutions $(x,y,m,n)=(2,5,5,3)$ and $(2,90,13,3)$ to (\ref{eq-main}),
we also find  $(x,y,m,n)=(2,-6,5,3)$ and $(2,-91,13,3)$, where we have assumed that $|y|>|x|>1$.
Beyond these, a short computer search uncovers only three more integer tuples $(x,y,m,n)$ satisfying
$$
\frac{x^m-1}{x-1} = \frac{y^n-1}{y-1}, \; \; m > n \geq 3, \; \; |y| > |x| >1,
$$
namely
$$
(x,y,m,n)=(-2,-7,7,3), (-2,6,7,3) \mbox{ and }  (-6,10,5,4).
$$
Perhaps there are no others; we can prove this to be the case if, for example, $n=3$, subject to (\ref{condition}). This result was obtained earlier as Corollary 4.1 of Yuan \cite{Yu0}, though the statement there overlooks the solutions $(x,y,m,n)= (-2,6,7,3), (2,-6,5,3)$ and $(2,-91,13,3)$.


%---------------------------------------------------------------------------------------------------------------------------------------------%

\endinput

\edit{some added stuff:}
\edit{FROM BEGHKR STARTS HERE}


. In the worst case scenario, the method in \cite{TW3} would reduce to $h^v$ such equations, where $h$ is the class number of $K$. This becomes computationally inefficient when the class number is large since, as we will see in the next section, we will need to apply a principal ideal test to each such case. Instead of this, we apply the method of \cite{GhKaMaSi}, which gives only $\kappa/2$ $S$-unit equations, where $\kappa$ is the number of roots of unity in $K$ (typically this means only one $S$-unit equation). We now describe this second method. 

In many cases, the method described above is far more efficient than that of Tzanakis-de Weger \cite{TW3}. However, there are values of $x$ where computing the class group may prove very costly. In fact, for these values of $x$, it may happen that class group computations  take longer than directly running a  principal ideal test on each ideal equation. In such cases, we return to the method of \cite{TW3}, which we now describe. 
\edit{end of added stuff}

I'd like to thank the Elysian Room and Pallet Coffee for not kicking me out of their establishments during the arduous writing process, and my lovely lounge chair for (literally) supporting me after closing hours. 


Any text after an \endinput is ignored.
You could put scraps here or things in progress.

%%% The following is a directive for TeXShop to indicate the main file
%%!TEX root = diss.tex

\chapter{Computing Elliptic Curves over $\mathbb{Q}$}
\label{ch:EllipticCurves} 

In the chapter at hand, we outline an algorithm to compute elliptic curves over $\mathbb{Q}$, based upon techniques of solving Thue-Mahler equations. Our aim is to give a straightforward demonstration of the link 
between the conductors of the elliptic curves in question and the corresponding equations, and to make the Diophantine approximation problem 
that follows as easy to tackle as possible. It is worth noting here that these connections are quite straightforward for primes $p > 3$, but require 
careful analysis at the primes $2$ and $3$. We will demonstrate our approach for a number of specific conductors and sets $S$, and then focus 
our main computational efforts on curves with bad reduction at a single prime (i.e. curves of conductor $p$ or $p^2$ for 
$p$ prime).  In these cases, the computations simplify significantly and we 
are able to find all curves of prime conductor up to $2 \times 10^9$ ($10^{10}$ in the case of curves of positive 
discriminant) and conductor $p^2$ for $p \leq 5\times 10^5$. We then extend these computations in the case of conductor 
$p$,  for prime $p \leq 2 \times 10^{13}$, and conductor $p^2$ for prime $p \leq 10^{10}$. We are not, however, able to 
guarantee completeness for these extended computations (we will discuss this further in what follows). 

%We are in the process of making our data more easily available through the LMFDB\footnote{The $L$-function and modular 
%form database, currently accessible at \url{http://www.lmfdb.org/}}. Until this is completed, we invite 
%the interested reader to contact the authors.


%-----------------------------------------------
\section{Elliptic curves} \label{elliptic}
%-----------------------------------------------

Our basic problem is to find a model for each isomorphism class of elliptic curves over $\mathbb{Q}$ with a given 
conductor. Let $S=\{ p_1, p_2, \ldots, p_k \}$ where the $p_i$ are distinct primes, and fix a conductor $N= p_1^{\eta_1} \cdots p_k^{\eta_k}$ for
$\eta_i \in \mathbb{N}$.  Any curve of conductor $N$ has a minimal 
model
\begin{align*}
E:&\phantom{=} y^2 + a_1 xy + a_3 y = x^3 + a_2 x^2 + a_4 x + a_6
\end{align*}
with the $a_i$ integral and discriminant 
\begin{align*}
\Delta_E &= (-1)^\delta p_1^{\gamma_1} \cdots p_k^{\gamma_k},
\end{align*}
where the $\gamma_i$ are positive integers satisfying $\gamma_i \geq \eta_i$, for each $i = 1, 2, \ldots, k$, and $\delta \in \{ 0, 1 \}$. 

Writing
$$
b_2 = a_1^2+4a_2, \; \; b_4 = a_1 a_3 + 2 a_4, \; \; 
b_6 = a_3^2+4a_6, \; \; 
c_4 = b_2^2-24 b_4 
$$
and
$$
c_6 = -b_2^3+ 36 b_2 b_4 -216 b_6,
$$
we have
$1728 \Delta_E = c_4^3-c_6^2$ and
$j_E = c_4^3/\Delta_E$.
It follows that
\begin{align} \label{first}
c_6^2 &= c_4^3 + (-1)^{\delta +1} 2^6 \cdot 3^3 \cdot p_1^{\gamma_1} \cdots p_k^{\gamma_k}.
\end{align}
In fact, it is equation~\eqref{first} that lies at the heart of our method (see also Cremona and 
Lingham \cite{CrLi} for an approach to the problem that takes as its starting point equation~\eqref{first}, but 
subsequently heads in a rather different direction).

Let $\nu_p(x)$ be the largest power of a prime $p$ dividing a nonzero integer $x$. Since our model is minimal, we 
may suppose  (via Tate's algorithm; see, for example, Papadopoulos \cite{Pap}) that 
$$
\min \{ 3 \nu_p (c_4), 2 \nu_p (c_6) \} < 12 + 12 \nu_p(2) + 6 \nu_p(3),
$$
for each prime $p$, while
$$
\nu_p (N_E) \leq 2 + \nu_p (1728).
$$ 
For future use, it will be helpful to have a somewhat more precise determination of the possible 
values of $\nu_p(c_4)$ and $\nu_p(c_6)$ we encounter. We compile this data from Papadopoulos~\cite{Pap} and
summarize it in Tables~\ref{tab nu2},~\ref{tab nu3} and~\ref{tab nup}.

\begin{table}[h]
$$
\begin{array}{|r|r|r|r|}
\hline
\nu_2 (c_4) & \nu_2 (c_6) & \nu_2 (\Delta_E) & \nu_2 (N)  \\ \hline
0 & 0 & \geq 0  & \min \{ 1, \nu_2 (\Delta_E)  \}  \\
\geq 4 &  3 & 0 & 0  \\
\geq 4 & 5 & 4 & 2, 3 \mbox{ or } 4  \\
\geq 4 & \geq 6 & 6 & 5 \mbox{ or } 6  \\
4 & 6 & 7 & 7  \\
4 & 6 & 8 & 2, 3 \mbox{ or } 4  \\
4 & 6 & 9 &  5   \\
4 & 6 & 10 \mbox{ or } 11 & 3 \mbox{ or } 4   \\
4 & 6 & \geq 12 &  4  \\
5 & 7 & 8 & 7  \\
\geq 6 & 7 & 8 & 2, 3 \mbox{ or } 4  \\
\hline
\end{array}
\quad
\begin{array}{|r|r|r|r|}
\hline
\nu_2 (c_4) & \nu_2 (c_6) & \nu_2 (\Delta_E) & \nu_2 (N)  \\ \hline
 5 & \geq 8 & 9 & 8 \\
  \geq 6 & 8 & 10 & 6 \\
 6 & \geq 9 & 12 & 5 \mbox{ or } 6 \\
 6 & 9 & \geq 14 &  6 \\
 7 & 9 & 12 & 5 \\
  \geq 8 & 9 & 12 & 4 \\
 6 & 9 & 13 & 7 \\
 7 & 10 & 14 & 7 \\
 7 & \geq 11 & 15 & 8 \\
  \geq 8 &  10 & 14 & 6 \\
  & & & \\
\hline
\end{array}
$$


\caption{The possible values of $\nu_2(c_4), \nu_2(c_6), \nu_2(\Delta_E)$ and $\nu_2(N)$.}
\label{tab nu2}
\end{table}

\begin{table}[h]
$$
\begin{array}{|r|r|r|r|}
\hline
\nu_3 (c_4) & \nu_3 (c_6) & \nu_3 (\Delta_E) & \nu_3 (N)  \\ \hline
0 & 0 & \geq 0  & \min \{ 1, \nu_3 (\Delta_E)  \}  \\
1 & \geq 3 & 0 & 0  \\
\geq 2 & 3 & 3 & 2 \mbox{ or } 3  \\
2 & 4 & 3 & 3  \\
2 & \geq 5 & 3 & 2  \\
2 & 3 & 4 & 4  \\
2 & 3 & 5 & 3  \\
2 & 3 & \geq 6 & 2  \\
\geq 3 & 4 & 5 & 5  \\
3 & 5 & 6 & 4  \\
\hline
\end{array}
\quad
\begin{array}{|r|r|r|r|}
\hline
\nu_3 (c_4) & \nu_3 (c_6) & \nu_3 (\Delta_E) & \nu_3 (N)  \\ \hline
 3 & \geq 6 & 6 & 2 \\
 \geq 4 & 5 & 7 & 5 \\
 \geq 4 & 6 & 9 & 2 \mbox{ or } 3 \\
 4 & 7 & 9 & 3 \\
 4 & \geq 8 & 9 & 2 \\
 4 & 6 & 10 & 4 \\
 4 & 6 & 11 & 3 \\
 \geq 5 & 7 & 11 & 5 \\
 5 & 8 & 12 & 4 \\
 \geq 6 & 8 & 13 & 5 \\
\hline
\end{array}
$$

\caption{The possible values of $\nu_3(c_4), \nu_3(c_6), \nu_3(\Delta_E)$ and $\nu_3(N)$.}
\label{tab nu3}
\end{table}

\begin{table}[h]
$$
\begin{array}{|r|r|r|r|}
\hline
\nu_p (c_4) & \nu_p (c_6) & \nu_p (\Delta_E) & \nu_p (N)  \\ \hline
0 & 0 & \geq 1  & 1  \\
\geq 1 & 1 & 2 & 2  \\
1 & \geq 2 & 3 & 2  \\
\geq 2 & 2 & 4 & 2  \\
\geq 2 & \geq 3 & 6 & 2 \\
\hline
\end{array}
\quad
\begin{array}{|r|r|r|r|}
\hline
\nu_p (c_4) & \nu_p (c_6) & \nu_p (\Delta_E) & \nu_p (N)  \\ \hline
 2 & 3 & \geq 7 & 2 \\
 \geq 3 & 4 & 8 & 2 \\
 3 & \geq 5 & 9 & 2 \\
 \geq 4 & 5 & 10 & 2 \\
  &  &  &  \\
\hline
\end{array}
$$

\caption{The possible values of $\nu_p(c_4), \nu_p(c_6), \nu_p(\Delta_E)$ and $\nu_p(N)$ when $p > 3$ is prime and 
$p\mid \Delta_E$.}
\label{tab nup}
\end{table}


%-----------------------------------------------------------------------------------------
\section{Cubic forms : the main theorem and algorithm} \label{forms}
%-----------------------------------------------------------------------------------------

Having introduced the notation we require for elliptic curves, we now turn our attention to cubic forms and our main result. Fix integers $a, b, 
c$ and $d$, and consider the binary cubic form 
\begin{align} \label{form0}
F(x,y)&=ax^3+bx^2y+cxy^2+dy^3,
\end{align}
with discriminant
\begin{equation} \label{claire-bear}
D_F = -27 a^2 d^2 + b^2 c^2 + 18 abcd -4 ac^3 -4 b^3 d.
\end{equation}
To any such form, we can associate a pair of covariants, the Hessian  $H=H_F$:
\begin{align*}
H=  H_F (x,y)=  - \frac{1}{4} \left(\frac{\partial^2 F}{\partial x^2} \frac{\partial^2 F}{\partial y^2} - 
\left(\frac{\partial^2 F}{\partial x \partial y}\right)^2 \right) 
\end{align*}
and the Jacobian determinant of $F$ and $H$,  a cubic form $G=G_F$ defined by
\begin{align*}
G&=G_F (x,y)=\frac{\partial F}{\partial x}\frac{\partial H}{\partial y}-  \frac{\partial F}{\partial y} \frac{\partial 
H}{\partial x}.
\end{align*}
A quick computation reveals that, explicitly, 
$$
H= (b^2-3ac) x^2 + (bc-9ad) xy + (c^2-3bd) y^2  
$$
and
$$
\arraycolsep=1.4pt\def\arraystretch{1.4}
\begin{array}{ll}
G = & (-27 a^2d+9abc-2b^3)  x^3 + (-3b^2c-27 abd+18ac^2) x^2 y \\
   & + (3bc^2-18b^2d+27acd)  x y^2 + (-9bcd+2c^3+27ad^2) y^3.\\
\end{array}
$$
These satisfy the syzygy
\begin{align} \label{syz}
4H(x,y)^3 &= G(x,y)^2+27D_F F(x,y)^2
\end{align}
as well as the resultant identities:
\begin{equation} \label{resultant}
\mbox{Res} (F,G) = -8 D_F^3 \; \; \mbox{ and } \; \; 
 \mbox{Res} (F,H) = D_F^2.
\end{equation}
Note here that we could just as readily work with $-G$ instead of $G$ here (corresponding to taking the Jacobian determinant of $H$ and $F$, rather than of $F$ and $H$). Indeed, as we shall observe in Section \ref{note}, for our applications we will, in some sense, need to consider both possibilities.

Notice that if we set $(x,y)=(1,0)$ and multiply through by $\mathcal{D}^6/4$ (for any rational $\mathcal{D}$), then 
this syzygy can be rewritten as
\begin{align*}
  ( \mathcal{D}^2 H(1,0))^3 - \left( \frac{\mathcal{D}^3}{2} G(1,0) \right)^2 
  &=  1728 \cdot \frac{\mathcal{D}^6 D_F}{256} F(1,0)^2.
\end{align*}
Given an elliptic curve with corresponding invariants $c_4, c_6$ and $\Delta_E$, we will show that it is always possible to construct a binary 
cubic form $F$, with corresponding  $\mathcal{D}$ for which 
$$
\mathcal{D}^2 H(1,0) = c_4, \; \; -\frac{1}{2} \mathcal{D}^3 G(1,0) = c_6 \; \mbox{ and } \; \Delta_E =  \frac{\mathcal{D}^6  D_F F(1,0)^2}{256}
$$
(and hence equation (\ref{first}) is satisfied). This is the basis of the proof of our main result, which provides an algorithm for computing all isomorphism classes of elliptic curves $E/\mathbb{Q}$ with conductor a fixed positive integer $N$. Though we state our result for curves with $j_E \neq 0$, the case $j_E=0$ is easy to treat separately (see Section \ref{Mordell}). 

\begin{theorem} \label{fisk}
Let $E/\mathbb{Q}$ be an elliptic curve of conductor $N=2^\alpha 3^\beta N_0$, where $N_0$ is coprime to $6$ and $0 \leq \alpha \leq 8$, $0 \leq \beta \leq 5$. Suppose further that $j_E \neq 0$.
Then there exists an integral binary cubic form $F$ of discriminant 
\begin{align*}
D_F &= \text{sign}(\Delta_E) 2^{\alpha_0} 3^{\beta_0} N_1,
\end{align*}
and relatively prime integers $u$ and $v$ with 
\begin{equation} \label{TM-eq}
F(u,v) =  \omega_0 u^3 + \omega_1 u^2v + \omega_2 uv^2 + \omega_3 v^3 = 2^{\alpha_1} \cdot 3^{\beta_1} \cdot \prod_{p \mid N_0} p^{\kappa_p},
\end{equation}
such that $E$ is isomorphic over $\mathbb{Q}$ to $E_{\mathcal{D} }$, where
\begin{equation} \label{curvey}
E_{\mathcal{D}} \; \; : \; \;  3^{[\beta_0/3]} y^2 = x^3 -27 \mathcal{D}^2 H_F(u,v) x +27 \mathcal{D}^3 G_F(u,v)
\end{equation}
and, for $[r]$ the greatest integer not exceeding a real number $r$,
\begin{align} \label{Dee}
\mathcal{D} &= \prod_{p \mid \gcd (c_4(E), c_6(E))} p^{\min \{ [\nu_p (c_4(E))/2], [\nu_p (c_6(E))/3] \}}.
\end{align}
The $\alpha_0$, $\alpha_1$, $\beta_0$, $\beta_1$ and $N_1$ are nonnegative integers satisfying  $N_1 \mid N_0$, 
\begin{align*}
(\alpha_0, \alpha_1) &=
\begin{cases}
(2, 0)  \mbox{ or } (2,3)  
    & \mbox{ if }  \alpha =0, \\
(3,\geq 3) \mbox{ or } (2,\geq 4)  
    & \mbox{ if }  \alpha =1, \\
(2,1), (4,0) \mbox{ or }  (4,1)  
    & \mbox{ if }  \alpha =2, \\
(2,1), (2,2), (3,2), (4,0)  \mbox{ or }  (4,1)  
    & \mbox{ if }  \alpha =3, \\
(2, \geq 0), (3, \geq 2), (4,0)  \mbox{ or }  (4, 1) 
    & \mbox{ if }  \alpha =4,  \\
(2, 0) \mbox{ or } (3,1) 
    & \mbox{ if }  \alpha =5, \\
(2, \geq 0), (3, \geq 1), (4,0) \mbox{ or }  (4, 1) 
    & \mbox{ if }  \alpha =6, \\
(3,0) \mbox{ or } (4,0) 
    & \mbox{ if }  \alpha =7,  \\
(3, 1) 
    & \mbox{ if }  \alpha =8
\end{cases}
\intertext{and}
(\beta_0, \beta_1) &=
\begin{cases}
(0, 0) 
    & \mbox{ if } \beta =0, \\
(0, \geq 1) \mbox{ or } (1, \geq 0) 
    & \mbox{ if } \beta =1, \\
(3,0), (0, \geq 0) \mbox{ or } (1, \geq 0) 
    & \mbox{ if } \beta =2, \\
(\beta, 0) \mbox{ or } (\beta,1) 
    & \mbox{ if } \beta \geq 3.
\end{cases}
\end{align*}
% $$
% \; \; \; \; \; \; \; \; \; \; \; 
% (\alpha_0, \alpha_1)  = \left\{
% \begin{array}{l}
% (2, 0)  \mbox{ or } (2,3)  \; \mbox{ if } \; \alpha =0, \\
% (3,\geq 3) \mbox{ or } (2,\geq 4)  \; \mbox{ if } \; \alpha =1, \\
% (2,1), (4,0) \mbox{ or }  (4,1)  \; \mbox{ if } \; \alpha =2, \\
% (2,1), (2,2), (3,2), (4,0)  \mbox{ or }  (4,1)  \; \mbox{ if } \; \alpha =3, \\
% (2, \geq 0), (3, \geq 2), (4,0)  \mbox{ or }  (4, 1) \; \mbox{ if } \; \alpha =4,  \\
% (2, 0) \mbox{ or } (3,1)  \; \mbox{ if } \; \alpha =5, \\
% (2, \geq 0), (3, \geq 1), (4,0) \mbox{ or }  (4, 1) \; \mbox{ if } \; \alpha =6, \\
% (3,0) \mbox{ or } (4,0)  \; \mbox{ if } \; \alpha =7, \\
% (3, 1)  \; \mbox{ if } \; \alpha =8, \\
% \end{array}
% \right.
% $$
% $$
% (\beta_0, \beta_1)  = \left\{
% \begin{array}{l}
% (0, 0) \; \mbox{ if } \; \beta =0, \\
% (0, \geq 1) \mbox{ or } (1, \geq 0) \; \mbox{ if } \; \beta =1, \\
% (3,0), (0, \geq 0) \mbox{ or } (1, \geq 0) \; \mbox{ if } \; \beta =2, \\
% (\beta, 0) \mbox{ or } (\beta,1) \; \mbox{ if } \; \beta \geq 3, \\
% \end{array}
% \right.
% $$
The  $\kappa_p$ are nonnegative integers with
\begin{equation} \label{term0}
\nu_p (\Delta_E)  =
\left\{
\begin{array}{lc}
 \nu_p (D_F) + 2 \kappa_p & \mbox{ if }  p \nmid \mathcal{D}, \\
\nu_p (D_F) + 2 \kappa_p + 6 & \mbox{ if }  p \mid \mathcal{D} \\
\end{array}
\right.
\end{equation}
and
\begin{equation} \label{term1}
\kappa_p \in \{ 0, 1 \} \; \; \mbox{ whenever } \; \; p^2 \mid N_1.
\end{equation}
Further, we have
\begin{equation} \label{term2}
 \mbox{ if } \; \; \beta_0 \geq 3, \; \mbox{ then } \; 3 \mid \omega_1 \mbox{ and } 3 \mid \omega_2,
\end{equation}
and
\begin{equation} \label{term3}
 \mbox{ if } \nu_p(N)=1, \mbox{ for } p \geq 3, \mbox{ then } p \mid D_F F(u,v).
\end{equation}
\end{theorem}

Here, as we shall make explicit in the next subsection, the form $F$ corresponding to the curve $E$
in Theorem \ref{fisk} determines the $2$-division field of $E$. This connection was noted by Rubin and Silverberg \cite{RuSi} in a somewhat different context -- they proved that if $K$ is a field of characteristic $\neq 2, 3$,  $F(u,v)$ is a binary cubic form defined over $K$,  $E$ is an elliptic curve defined by $y^2=F(x,1)$, and $E_0$ is another elliptic curve over $K$ with the property that
$E[2] \cong E_0[2]$ (as Galois modules), then $E_0$ is isomorphic to the curve
$$
y^2 = x^3 - 3 H_F(u,v) x + G_F(u,v),
$$
for some $u, v \in K$. 

%--------------------------------------------------------------
\subsection{Remarks}
%---------------------------------------------------------------
Before we proceed, there are a number of observations we should make regarding Theorem \ref{fisk}. 

\subsubsection{Historical comments}
Theorem \ref{fisk} is based upon a generalization of classical work of Mordell \cite{Mor1} (see also Theorem 3 of 
Chapter 24 of Mordell \cite{Mor}), in which the Diophantine equation 
$$
X^2+kY^2 = Z^3
$$
is treated through reduction to binary cubic forms\index{cubic forms} and their covariants, under the assumption that 
$X$ and $Z$ are coprime. That this last restriction can, with some care, be eliminated, was noted by Sprindzuk (see 
Chapter VI of \cite{Spri}). A similar approach to this problem can be made through the invariant theory of binary quartic forms, where one is led to solve, instead, equations of the shape
$$
X^2 + k Y^3 = Z^3.
$$
We will not carry out the analogous analysis here.

\subsubsection{$2$-division fields and reducible forms}
It might happen that the form $F$ whose existence is guaranteed by Theorem \ref{fisk} is reducible over $\mathbb{Z}[x,y]$. This occurs precisely when the elliptic curve $E$ has a nontrivial rational $2$-torsion point. This follows from the more general fact that the cubic form $F(u,v) =  \omega_0 u^3 + \omega_1 u^2v + \omega_2 uv^2 + \omega_3 v^3$ corresponding to an elliptic curve $E$ has the property that the splitting field of $F(u,1)$ is isomorphic to the $2$-division field of $E$. This is almost immediate from the identity
$$
\arraycolsep=1.4pt\def\arraystretch{1.4}
\begin{array}{ll}
3^3 \, \omega_0^2 \, F \left( \frac{x-\omega_1}{3 \omega_0},1 \right) & = x^3+(9 \omega_0 \omega_2-3 \omega_1^2) x+27 \omega_0^2 \omega_3-9 \omega_0 \omega_1 \omega_2+2 \omega_1^3 \\
 & = x^3 - 3 H_F(1,0) x  + G_F (1,0). \\
\end{array}
$$
Indeed, from (\ref{curvey}), the elliptic curve defined by the equation
$y^2=x^3 - 3 H_F(1,0) x  + G_F (1,0)$
is a quadratic twist of that given by the model $y^2 = x^3 -27 c_4(E) x -54 c_6(E)$, and hence also of $E$ (whereby they have the same $2$-division field).

\subsubsection{Imprimitive forms}
It is also the case that the cubic forms arising need not be primitive (in the sense that $\gcd 
(\omega_0,\omega_1,\omega_2,\omega_3)=1$). This situation can occur if each of 
the coefficients of $F$ is divisible by some integer $g \in \{ 2, 3, 6 \}$. Since the discriminant is a quartic form in the coefficients of $F$, for this to take place one requires that
$$
D_F \equiv 0 \mod{g^4}.
$$
This is a necessary but not sufficient condition for the form $F$ to be imprimitive. It follows, if we wish to restrict attention to primitive forms in Theorem \ref{fisk}, that the possible values for $\nu_p (D_F)$ that can arise are
\begin{align} \label{lumpy}
& \nu_2 (D_F) \in \{ 0, 2, 3, 4 \}, \; \;  \nu_3 (D_F) \in \{ 0, 1, 3, 4, 5 \} \\
& \mbox{ and } \nu_p (D_F) \in \{ 0, 1, 2 \}, \; \mbox{ for } p > 3.
\end{align}


\subsubsection{Possible twists}
We note that necessarily
\begin{align} \label{froggie}
\mathcal{D} &\mid 2^3 \cdot 3^2 \cdot \prod_{p \mid N_0} p,
\end{align}
so that, given $N$, there is a finite set of $E_{\mathcal{D}}$ to consider (we can restrict our attention to quadratic twists of the curve defined via
$y^2=x^3 - 3 H_F(1,0) x  + G_F (1,0)$,
by squarefree divisors of $6N$). In  case we are dealing with squarefree conductor $N$ (i.e. for semistable curves $E$), then, from Tables \ref{tab nu2}, \ref{tab nu3} and \ref{tab nup}, it follows that $\mathcal{D} \in \{ 1, 2 \}$.


\subsubsection{Necessity, but not sufficiency}
If we search for elliptic curves of conductor $N$, say, there may exist a cubic form $F$ for which the corresponding Thue-Mahler equation \eqref{TM-eq} \index{Thue-Mahler equations} has a 
solution, where all of the conditions of Theorem \ref{fisk} are satisfied, but for which the corresponding  $E_{\mathcal{D}}$ has conductor $N_{E_{\mathcal{D}}} \neq N$ for all possible $\mathcal{D}$. This can happen when 
certain local conditions at primes dividing $6N$ are not met; these local conditions are, in practice, easy to check and only a minor issue when 
performing computations. Indeed, when producing tables of elliptic curves of conductor up to some given bound, we will, in many cases,  apply Theorem \ref{fisk} to find all curves with good reduction outside a fixed set of primes -- in effect, working with 
multiple conductors simultaneously. For such a computation, the conductor of every twist $E_{\mathcal{D}}$ we encounter will be of interest to 
us. 

\subsubsection{Special binary cubic forms} \label{dahlia}
If, for a given binary form $F(x,y)=a x^3 + b x^2 y + c xy^2 + d y^3$, 3 divides both the coefficients $b$ and $c$ (say $b = 3 b_0$ and 
$c=3 c_0$), then  $27 \mid D_F$ and, consequently, we can write $D_F=27 \widetilde{D}_F$, where
\begin{align*}
\widetilde{D}_F &= -a^2d^2+6ab_0c_0d+3b_0^2c_0^2-4ac_0^3-4b_0^3d.
\end{align*}
One can show that the set of binary cubic forms with $b \equiv c \equiv 0 \mod{3}$ is closed within the larger set of all binary 
cubic forms in $\mathbb{Z}[x,y]$, under the action of either $\mbox{SL}_2 ( \mathbb{Z})$ or $\mbox{GL}_2 ( \mathbb{Z})$. 
Also note that for such forms we have
$$
 \widetilde{H}_F(x,y) = \frac{H_F(x,y)}{9}= (b_0^2-ac_0) x^2 + (b_0c_0-ad) xy + (c_0^2-b_0d) y^2 
$$
and $\widetilde{G}_F (x,y) =  G_F(x,y)/27$, so that
$$
\arraycolsep=1.4pt\def\arraystretch{1.4}
\begin{array}{cl} 
\widetilde{G}_F (x,y) = & (-a^2d+3ab_0c_0-2b_0^3) x^3 + 3 (-b_0^2c_0-ab_0d+2ac_0^2) x^2 y \\
 & + 3 (b_0c_0^2-2b_0^2d+ac_0d) x y^2 + (-3b_0c_0d+2c_0^3+ad^2) y^3. \\
 \end{array}
$$
The syzygy now becomes
\begin{align} \label{syz2}
4\widetilde{H}_F (x,y)^3 &=\widetilde{G}_F(x,y)^2+\widetilde{D}_F F(x,y)^2.
\end{align}
We note, from Theorem \ref{fisk}, that we will be working exclusively with forms of this shape whenever we wish to treat elliptic curves of conductor $N \equiv 0 \mod{3^3}$.

\subsubsection{The case $j_E = 0$} \label{Mordell}
This case is treated over a general number field in Proposition 4.1 of Cremona and Lingham \cite{CrLi}.
The elliptic curves $E/\mathbb{Q}$ with $j_E=0$ and a given conductor $N$ are particularly easy to determine, since a curve with this property is necessarily isomorphic over $\mathbb{Q}$ to a {\it Mordell} curve with a model of the shape $Y^2 = X^3  - 54 c_6$ where $c_6=c_6(E)$. Such a model is minimal except possibly at $2$ and $3$ and has discriminant 
$-2^6 \cdot 3^9 \cdot c_6^2$ (whereby any primes $p > 2$ which divide $c_6$  necessarily also divide $N$). Here, without loss of generality, we may suppose that $c_6$ is sixth-power-free.
Further, from Tables \ref{tab nu2}, \ref{tab nu3}, and \ref{tab nup}, we have that $\nu_2(N) \in \{ 0, 2, 3, 4, 6 \}$, that $\nu_3 (N) \in \{ 2, 3, 5 \}$, and that $\nu_p(N)=2$ whenever $p \mid N$ for $p > 3$. Given a positive integer $N$ satisfying these constraints, it is therefore a simple matter to check to see if there are elliptic curves $E/\mathbb{Q}$ with conductor $N$ and $j$-invariant $0$. One needs only to compute the conductors of the curves given by $Y^2 = X^3  - 54 c_6$ for each sixth-power-free integer (positive or negative) $c_6$ dividing $64 N^3$.


\subsection{The algorithm}
It is straightforward to convert Theorem~\ref{fisk} into an algorithm for finding all $E/\mathbb{Q}$ of conductor $N$. We can proceed as follows.
\begin{enumerate}
\item Begin by finding all $E/\mathbb{Q}$ of conductor $N$ with $j_E=0$, as outlined in Section \ref{Mordell}.
\item Next, compute $\mbox{GL}_2(\mathbb{Z})$-representatives for every binary form $F$ with discriminant
\begin{align*}
\Delta_F &= \pm 2^{\alpha_0} 3^{\beta_0} N_1
\end{align*}
for each divisor $N_1$ of $N_0$, and each possible pair $(\alpha_0,\beta_0)$ given in the statement of Theorem 
\ref{fisk} (see (\ref{lumpy}) for specifics). We describe an algorithm for listing these forms in Section~\ref{rep}. 
\item Solve the corresponding Thue-Mahler equations\index{Thue-Mahler equations}, finding pairs of integers $(u,v)$ such that
$F(u,v)$ is an $S$-unit, where $S = \{ p \mbox{ prime } : p \mid N \} \cup \{2 \}$ and $F(u,v)$ satisfies the additional conditions given in the statement of Theorem 
\ref{fisk}.
\item For each cubic form $F$ and pair of integers $(u,v)$, consider the elliptic curve
$$
E_1 \; \; : \; \; y^2 = x^3 -27 H_F(u,v) x +27 G_F(u,v)
$$
and all its quadratic twists by squarefree divisors of $6N$. Output those curves with conductor $N$ (if any).
\end{enumerate}
The first, second and fourth steps here are straightforward; the first and second can be done efficiently, while the fourth is essentially 
trivial. The main bottleneck is step~(3). While there is a deterministic procedure for carrying this out (see Tzanakis and de 
Weger \cite{TW2}, \cite{TW3}), it is both involved and, often, computationally taxing. An earlier  implementation of this method in Magma due to Hambrook \cite{Ham} has subsequently been refined by the second author \cite{GhKaMaSi}; the most up-to-date version of this code (which we will reference here and henceforth as UBC-TM) is available at 
\begin{center}
\url{http://www.nt.math.ubc.ca/BeGhRe/Code/UBC-TMCode}
\end{center}

We give a number of examples of this general procedure in Section~\ref{examples}. In Section~\ref{primes}, we show that  in the 
special cases where the conductor is prime or the square of a prime, the Thue-Mahler equations~\eqref{TM-eq} (happily) 
reduce to Thue equations (i.e. the exponents on the right hand side of~\eqref{TM-eq} are absolutely bounded).  This situation occurs because, for such elliptic curves, a very strong form of Szpiro's conjecture (bounding the minimal discriminant of an elliptic curve from above in terms of its conductor) is known to hold. Thue equations 
can be solved by routines that are computationally much easier than is currently the case for Thue-Mahler equations; such procedures have been implemented in Pari/GP \cite{PARI2}  and Magma 
\cite{magma}. Further, in this situation,  it is possible to apply a much more computationally efficient argument to find all such elliptic 
curves heuristically but not, perhaps, completely (see Section~\ref{compy}).
 

%------------------------------------------------------------
\section{Proof of Theorem \ref{fisk}} \label{BigProof}
%------------------------------------------------------------

\begin{proof}
Given an elliptic curve $E/\mathbb{Q}$ of conductor $N=2^\alpha 3^\beta N_0$ and invariants $c_4= c_4(E) \neq 0$ and 
$c_6=c_6(E)$, we will construct a corresponding cubic form $F$ explicitly. In fact, our form $F$ will have the property that its leading coefficient will be supported on the primes dividing $6N$, i.e. that
$$
F(1,0) = 2^{\alpha_1} \cdot 3^{\beta_1} \cdot \prod_{p \mid N_0} p^{\kappa_p}.
$$
Define $\mathcal{D}$ as in~\eqref{Dee}, i.e. take 
$\mathcal{D}$ to be the largest integer whose square divides $c_4$ and whose cube divides $c_6$. We then set 
$$
X = c_4/\mathcal{D}^2 \; \; \mbox{ and } \; \; Y = c_6/\mathcal{D}^3,
$$
whereby, from (\ref{first}),
\begin{align} \label{first2}
Y^2 &= X^3 + (-1)^{\delta +1}  M,
\end{align}
for
\begin{align*}
M &=\mathcal{D}^{-6} \cdot 2^6 \cdot 3^3 \cdot |\Delta_E|.
\end{align*}
Note that the assumption that $c_4(E) \neq 0$ ensures that both the $j$-invariant $j_E \neq 0$ and that $X \neq 0$.

It will prove useful to us later to understand precisely the possible common factors among $X, Y, \mathcal{D}$ and $M$.
For any $p>3$, we have $\nu_p(N) \leq 2$. When 
$\nu_p(N)=1$, from Table \ref{tab nup} we find that
\begin{equation} \label{super-1}
( \nu_p (\mathcal{D}), \nu_p (X), \nu_p (Y), \nu_p (M)) = (0,0,0, \geq 1),
\end{equation}
while, if $\nu_p (N)=2$, then either 
\begin{equation} \label{super0}
\nu_p (\mathcal{D}) = 1 \mbox{ and }  \min \{ \nu_p (X), \nu_p (Y) \} = 0, \; \nu_p (M)=0 ,
\end{equation}
or
%
\begin{align} \label{super}
\nu_p (\mathcal{D}) \leq 1, \;
&(\nu_p (X), \nu_p (Y), \nu_p (M) ) = (0,0, \geq 1), (\geq 1, 1, 2), (1, \geq 2, 3) \\
&\mbox{ or } (\geq 2, 2, 4).
\end{align}
Things are rather more complicated for the primes $2$ and $3$; we summarize this in Tables~\ref{tab nu2 nxym} and~\ref{tab nu3 
nxym} (which are, in turn, compiled from the data in Tables~\ref{tab nu2} and ~\ref{tab nu3}). 

\begin{table}[h]
$$
\begin{array}{|c|l|} \hline
\nu_2 (N) & (\nu_2 (X), \nu_2 (Y), \nu_2 (M), \nu_2 (\mathcal{D}) ) \\ \hline
0 &  (\geq 2, 0, 0, 1)  \mbox{ or } (0,0,6, 0)  \\
1 & (0,0,\geq 7, 0)  \\
2 & (\geq 2, 2, 4, 1), (\geq 2, 1, 2, 2) \mbox{ or } (0,0,2, 2) \\
3 & (\geq 2, 2, 4, 1), (\geq 2, 1, 2, 2)  \mbox{ or }  (0,0,t, 2), t= 2, 4 \mbox{ or } 5  \\
4 &  (\geq 2, 2, 4, 1), (\geq 2, 1, 2, 2), (\geq 2, 0, 0, 3)  \mbox{ or }  (0,0,t, 2), t= 2 \mbox{ or } t \geq 4  \\
5 & (\geq 0, \geq 0, 0, 2), (0, \geq 0, 0, 3), (0,0,3, 2) \mbox{ or } (1,0,0, 3)  \\
6 & (\geq 0, \geq 0, 0, 2), (0, \geq 0, 0, 3), (\geq 2, 2, 4, 2), (\geq 2, 1, 2, 3) \mbox{ or }  (0,0,\geq 2, 3) \\
7 & (0,0,1,2), (0,0,1,3), (1,1,2,2) \mbox{ or } (1,1,2,3) \\
8 & (1, \geq 2, 3, 2) \mbox{ or } (1, \geq 2, 3, 3). \\
 \hline
\end{array}
$$
\caption{The possible values of $\nu_2(N), \nu_2 (X), \nu_2 (Y), \nu_2(M)$ and $\nu_2 (D)$}
\label{tab nu2 nxym}
\end{table}

\begin{table}[h]
$$
\begin{array}{|c|l|} \hline
\nu_3 (N) & (\nu_3 (X), \nu_3 (Y), \nu_3 (M), \nu_3 (\mathcal{D}) ) \\ \hline
0 & (1, \geq 3, 3, 0)  \mbox{ or } (0,0,3, 0)  \\
1 & (0,0,\geq 4, 0)  \\
2 & (\geq 0, 0, 0, 1), (0, \geq 2, 0, 1), (0,0, \geq 3, 1), (1, \geq 3, 3, 1), (\geq 0,0,0,2)  \\
 & \mbox{ or } (0,\geq 2, 0, 2) \\
3 & (\geq 0, 0, 0, 1),  (\geq 0, 0, 0, 2), (0,1,0, 1), (0,1,0, 2), (0,0,2,1)\\
 &  \mbox{ or } (0,0,2, 2) \\
4 &  (0,0,1,1), (0,0,1,2), (1,2,3,1) \mbox{ or } (1, 2, 3, 2) \\
5 &   (\geq 1, 1, 2, 1), (\geq 1, 1, 2, 2), (\geq 2, 2, 4, 1) \mbox{ or }  (\geq 2, 2, 4,2). \\
 \hline
\end{array}
$$
\caption{The possible values of $\nu_3(N), \nu_3(X), \nu_3 (Y), \nu_3(M)$ and $\nu_3 (D)$}
\label{tab nu3 nxym}
\end{table}

We will construct a cubic form
$$
 F_1(x,y) = ax^3 + 3b_0 x^2y + 3c_0 xy^2 + dy^3,
$$
one coefficient at a time; our main challenge will be to ensure that the $a, b_0, c_0$ and $d$ we produce are actually integral rather than just rational. The form $F$ whose existence is asserted
in the statement of Theorem \ref{fisk} will turn out to be either $F_1$ or $F_1/3$. 

Let us write
\begin{align*}
  M = M_1 \cdot M_2
\end{align*}
where $M_2$ is the largest integer divisor of $M$ that is coprime to $X$, so that
$$
M_1 = \prod_{p \, \mid  \, X} p^{\nu_p (M)} \; \; \mbox{ and } \; \;  M_2 = \prod_{p \, \nmid \, X} p^{\nu_p (M)}.
$$
We define
\begin{equation} \label{a1}
a_1 = \prod_{p \mid M_1} p^{\left[ \frac{\nu_p (M)-1}{2} \right]}
\end{equation}
and set
\begin{equation} \label{a2}
a_2 = \left\{
\begin{array}{cl}
3^{-1} \, \prod_{p \mid M_2} p^{\left[ \frac{\nu_p (M)}{2} \right]} & \mbox{ if }  \nu_3 (X)=0, \, \nu_3(M) =2t, \, t \in \mathbb{Z}, t \geq 2,  \\
\prod_{p \mid M_2} p^{\left[ \frac{\nu_p (M)}{2} \right]} & \mbox{ otherwise. } \\
\end{array}
\right.
\end{equation}
Define $a = a_1 \cdot a_2$.
It follows that $a_1^2 \mid M_1$ and, from (\ref{super-1}), (\ref{super0}), (\ref{super}), and Tables \ref{tab nu2 nxym} and \ref{tab nu3 nxym}, that both
$$
a_1 \mid X \; \; \mbox{ and } \; \;  a_1^2 \mid Y.
$$
We write $X = a_1 \cdot X_1$ and observe that $a_2^2 \mid M_2$. Note that $a_2$ is coprime to $X$ and hence to $a_1$. 
Since $a^2 \mid M$, we may thus define a positive integer $K$ via $K = M/a^2$, so that (\ref{first2}) becomes
\begin{align*}
Y^2-X^3 &= (-1)^{\delta+1} Ka^2.
\end{align*}
From the fact that $\gcd (a_2,X)=1$ and $X \neq 0$, we may choose $B$ so that
\begin{align*}
  a_2 B &\equiv -Y/a_1 \mod{X^3},
\end{align*}
whereby 
\begin {equation} \label{truth}
a B+Y \equiv 0 \mod{a_1X^3}.
\end{equation}  
Note that, since $a_1^2 \mid Y$ and $a_1 \mid X$, it follows that $a_1 \mid B$.
Let us define
\begin{equation} \label{definite}
b_0 = \frac{aB+Y}{X}, \; \; 
c_0 = \frac{b_0^2-X}{a} \; \; \mbox{ and } \; \; 
d = \frac{b_0c_0 - 2B}{a}.
\end{equation}
We now demonstrate that these are all integers. That $b_0 \in \mathbb{Z}$ is immediate from (\ref{truth}).
Since $b_0 X - Y = aB$, we know that $b_0 X \equiv Y \mod{a}$. Squaring both sides 
thus gives 
$$
b_0^2 X^2 \equiv Y^2 \equiv X^3 + (-1)^{\delta+1} K a^2  \equiv X^3 \mod{a_1 \cdot a_2},
$$
and, since $\gcd (a_2,X)=1$,
$$
 b_0^2 \equiv X \mod{a_2}.
$$
From (\ref{truth}), we have $b_0 \equiv 0 \mod{a_1 X^2}$, whereby, since $a_1 \mid X$, 
$$
b_0^2 \equiv X \equiv 0 \mod{a_1}.
$$
The fact that $\gcd(a_1,a_2)=1$ thus allows us to conclude that $b_0^2 \equiv X \mod{a}$ and hence that $c_0 \in \mathbb{Z}$.

It remains to show that $d$ is an integer.
Let us rewrite $ad$ as
$$
ad = b_0c_0-2B = \left( \frac{aB+Y}{aX} \right) \left( \left( \frac{aB+Y}{X} \right)^2 - X \right) - 2B,
$$
so that
$$
ad =\left( \frac{aB+Y}{aX} \right) \left( \frac{(-1)^{\delta+1} K a^2 + 2 a B Y + a^2 B^2}{X^2} \right) - 2B.
$$
Expanding, we find that
\begin{equation} \label{three}
X^3 d = (-1)^{\delta+1} KY+ 3 Y B^2 + a B^3 + (-1)^{\delta+1} 3 KaB.
\end{equation}
We wish to show that
$$
(-1)^{\delta+1} KY+ 3 Y B^2 + a B^3 + (-1)^{\delta+1} 3 KaB \equiv 0 \mod{X^3}.
$$
From (\ref{truth}), we have that
$$
(-1)^{\delta+1} KY+ 3 Y B^2 + a B^3 + (-1)^{\delta+1} 3 KaB \equiv 2Y \left( B^2 + (-1)^\delta K \right) \mod{a_1 X^3}.
$$
Multiplying congruence (\ref{truth}) by $aB-Y$ (which, from our prior discussion, is divisible by $a_1^2$), 
we find that 
$$
a^2 B^2 \equiv Y^2 \equiv X^3 + (-1)^{\delta+1} K a^2 \mod{a_1^3 X^3}
$$
and hence, dividing through by $a_1^2$,
$$
a_2^2 B^2 \equiv  a_1 X_1^3 + (-1)^{\delta+1} K a_2^2 \mod{a_1 X^3}.
$$
It follows that
\begin{equation} \label{three-2}
B^2 + (-1)^\delta K \equiv a_2^{-2} a_1 X_1^3  \mod{a_1 X^3},
\end{equation}
and so, since $a_1^2 \mid Y$, 
$$
Y \left( B^2 + (-1)^\delta K \right) \equiv 0 \mod{X^3},
$$
whence we conclude that $d$ is an integer, as desired.

With these values of $a, b_0, c_0$ and $d$, we can then confirm (with a quick computation) that the cubic form
\begin{align*}
  F_1(x,y) &= ax^3 + 3b_0 x^2y + 3c_0 xy^2 + dy^3
\end{align*}
has discriminant
\begin{align*}
  D_{F_1} &= \frac{108}{a^2} (X^3-Y^2) = (-1)^\delta \cdot 2^2 \cdot 3^3 \cdot K
\end{align*}
We also note that 
$$
F_1(1,0) = a, \; \; \widetilde{H}_{F_1}(1,0) = b_0^2 -a c_0 = X
$$
and
$$
 -\frac{1}{2} \widetilde{G}_{F_1}(1,0) = \frac{1}{2}(a^2d -3ab_0c_0+2b_0^3) = Y,
$$
where $\widetilde{G}_{F}$ and $\widetilde{H}_{F}$ are as in Section \ref{dahlia}.

Summarizing Table \ref{tab nu3 nxym},  we find that we are in one of the following four cases :

\begin{enumerate}
\item[(i)] $\nu_3 (X)=1, \;  \nu_3(Y) =2,  \; \nu_3(M)=3 \; \mbox{ and } \;  \nu_3(N)=4,$
\item[(ii)] $\nu_3(X) \geq 2, \;  \nu_3(Y) =2, \; \nu_3(M) =4, \; \nu_3(N)=5,$
\item[(iii)] $\nu_3(M) \leq 2 $ and $ \nu_3(N) \geq 2$, or
\item[(iv)] $\nu_3 (M)  \geq 3$ and either $\nu_3 (XY)=0 $ or $\nu_3(X)=1, \,  \nu_3(Y) \geq 3$.
\end{enumerate}
In cases (i), (ii), and (iii), we choose $F=F_1$, i.e.
\begin{align*}
(\omega_0,\omega_1,\omega_2,\omega_3) = (a, 3 b_0, 3 c_0, d),
\end{align*}
so that
\[F(1,0)=a, \; \; D_{F} = (-1)^\delta 2^2 \cdot 3^3 \cdot K, \; \; 
c_4 = \mathcal{D}^2 \widetilde{H}_{F} (1,0)\]
and 
\[c_6 = - \frac{1}{2}  \mathcal{D}^3 \widetilde{G}_{F} (1,0).\]
It follows that $E$ is isomorphic over $\mathbb{Q}$ to the curve
$$
y^2 = x^3 -27 c_4 x -54 c_6 = x^3 - 3 \mathcal{D}^2 H_{F}(1,0) x + \mathcal{D}^3 G_{F}(1,0).
$$

In  case (iv), observe that, from definitions (\ref{a1}) and (\ref{a2}),
\begin{equation} \label{foster}
\nu_3(a) = \left[ \frac{\nu_3(M)-1}{2} \right] \; \; \mbox{ and } \; \; 
\nu_3(K) = \nu_3(M) - 2 \nu_3 (a),
\end{equation}
so that $3 \mid a$ and $3 \mid K$. From equation (\ref{three}),  $3 \mid X^3 d$.
If $\nu_3(X)=0$ this implies that $3 \mid d$. On the other hand, 
if $\nu_3(X)=1$, then, from (\ref{three-2}), we may conclude that $3 \mid B$. Since each of $a, B$ and $K$ is divisible by $3$, while $\nu_3 (X) =1$ and $\nu_3 (Y) \geq 3$,  equation (\ref{three})
once again implies that $3 \mid d$.
In this case, we can therefore write $a = 3a_0$ and $d=3d_0$, for integers $a_0$ and $d_0$ and 
set $F = F_1/3$, i.e. take
\begin{align*}
(\omega_0,\omega_1,\omega_2,\omega_3) = (a_0, b_0, c_0, d_0).
\end{align*}
We have
\[F(1,0) =a/3, \; \;  D_{F}  = (-1)^\delta 2^2  \cdot K/3, \; \; 
c_4 = \mathcal{D}^2 H_{F} (1,0)\]
and 
\[c_6 = - \frac{1}{2}  \mathcal{D}^3 G_{F} (1,0).\]
The curve $E$ is now isomorphic over $\mathbb{Q}$ to the model
$$
y^2 = x^3 -27 c_4 x -54 c_6 = x^3 - 27 \mathcal{D}^2 H_{F}(1,0) x + 27 \mathcal{D}^3 G_{F}(1,0).
$$

Since $|D_F|/D_F = (-1)^\delta$ and $a^2 K \mid 1728 \Delta_E$, we may write
$$
F(1,0) = 2^{\alpha_1} \cdot 3^{\beta_1} \cdot \prod_{p \mid N_0} p^{\kappa_p}
\; \; \mbox{ and } \; \; D_F = (|\Delta_E|/\Delta_E) 2^{\alpha_0} 3^{\beta_0} N_1,
$$
for nonnegative integers $\alpha_0, \alpha_1, \beta_0, \beta_1, \kappa_p$ and a positive integer $N_1$,  divisible only by primes dividing $N_0$. More explicitly, we have 
$$
\alpha_0 = \nu_2 (K) + 2\; \; \mbox{ and } \;  \; \beta_0 =\nu_3 (K) + 
\left\{ 
\begin{array}{rl}
3 & \mbox{ in case (i), (ii) or (iii), or } \\
-1& \mbox{ in case (iv),} \\
\end{array}
\right.
$$
and
$$
\alpha_1 = \nu_2 (a) \; \; \mbox{ and } \;  \; \beta_1 =\nu_3 (a) + 
\left\{ 
\begin{array}{rl}
0 & \mbox{ in case (i), (ii) or (iii), or } \\
-1& \mbox{ in case (iv). } \\
\end{array}
\right.
$$
It remains for us to prove that these integers satisfy the conditions listed in the statement of the theorem. It is straightforward to check this,  considering in turn each possible triple $(X,Y,M)$ from 
(\ref{super-1}), (\ref{super0}), (\ref{super}), and Tables~\ref{tab nu2 nxym} and ~\ref{tab nu3 nxym}, and using the fact that $K = M/a^2$. 
%, where
%$$
%a = a_3 \mbox{ or } a_3/3, \mbox{ according to (\ref{a2}), } \; \mbox{and } \; 
%a_3 = \prod_{p \mid M} p^{\left[ \frac{\nu_p (M)-\delta_p}{2} \right]} \; \mbox{ for } \; \; 
%\delta_p =
%\left\{
%\begin{array}{ll}
%1 & \mbox{ if } p \mid X, \\
%0 & \mbox{ if } p \nmid X. \\
%\end{array}
%\right.
%$$

In particular, if $p > 3$, we have $\nu_p(\Delta_E) = 6 \nu_p(\mathcal{D}) + \nu_p (D_F) + 2 \kappa_p$. From Table \ref{tab nup} and (\ref{Dee}), we have $\nu_p(\mathcal{D}) \leq 1$, whereby (\ref{term0}) follows. 
Further,
\begin{equation} \label{toothbrush}
\nu_p(a) = 
\left\{\begin{array}{ll}
\left[ \frac{\nu_p(M)-1}{2} \right] & \mbox{ if } p \mid X, \\
\left[ \frac{\nu_p(M)}{2} \right] & \mbox{ if } p \nmid X, \\
\end{array}
\right.
\end{equation}
and so, if $p \nmid X$, 
$$
\nu_p(M)-2 \nu_p(a) \leq 1.
$$
Since $a^2K=M$, if $p^2 \mid D_F$, then $\nu_p(N)=2$ and it follows that we are in case (\ref{super}), with $p \mid X$. We may thus conclude that $\nu_p (M) \in \{ 2, 3, 4 \}$ and hence, from (\ref{toothbrush}), that $\nu_p(a) \leq 1$. This proves (\ref{term1}).

For (\ref{term2}), note that, in cases (i), (ii) and (iii), we clearly have that  $3 \mid \omega_1$ and $3 \mid \omega_2$.  In case (iv), from (\ref{foster}),
$$
\beta_0 = \nu_3 (D_F) = \nu_3(K) -1 = \nu_3(M) - 2  \left[ \frac{\nu_3(M)-1}{2} \right] - 1 \in \{ 0, 1 \}.
$$
Finally, to see (\ref{term3}), note that if $\nu_p(N)=1$, for  $p > 3$, then we have (\ref{super-1}) and hence
$$
\nu_p(D_F) + 2 \nu_p (F(u,v)) = \nu_p(M) \geq 1,
$$
whereby $p \mid D_F$ or $p \mid F(u,v)$.
We may also readily check that the same conclusion obtains for $p=3$ (since, equivalently, $\beta_0+\beta_1 \geq 1$). This completes the proof of Theorem \ref{fisk}.


\end{proof}

To illustrate this argument, suppose we consider the elliptic curve (denoted 109a1 in Cremona's database) defined via
$$
E \; \; : \; \; y^2 + xy = x^3 - x^2 -8 x -7,
$$
with $\Delta_E=-109$.
We have
$$
c_4(E) = 393 \; \; \mbox{ and } \; \; c_6(E)=7803,
$$
so that $\gcd (c_4(E), c_6(E))=3$. It follows that
$$
\mathcal{D} = 1, \; X=393, \; Y=7803, \; \delta=1, \; M = 2^6 \cdot 3^3 \cdot 109,
$$
and hence we have
$$
M_1 = 3^3, \; M_2=2^6 \cdot 109, \; a_1 = 3, \; a_2 = 2^3, \; a=2^3 \cdot 3 \; \mbox{ and } \; K = 3 \cdot 109.
$$
We solve the congruence $8B \equiv -2601 \mod{393^3}$ to find that we may choose $B=7586982$, so that
$$
b_0 =463347, \; \; 
c_0 = 8945435084 \; \; \mbox{ and } \; \; 
d =172701687278841.
$$
We are in case (iv) and thus set
$$
F(x,y) = 8 x^3 + 463347 x^2 y + 8945435084 x y^2 + 57567229092947 y^3,
$$
with discriminant $D_F = -4 \cdot 109$,
$$
G_F(1,0) = -15606 = -2 c_6(E) \; \; \mbox{ and } \; \; H_F(1,0)=393 = c_4(E).
$$
The curve $E$ is thus isomorphic to the model
\begin{equation} \label{flag}
E_{\mathcal{D}} \; \; : \; \; y^2 = x^3 - 27 \mathcal{D}^2 H_{F}(1,0) x + 27 \mathcal{D}^3 G_{F}(1,0) = x^3 - 10611x-421362.
\end{equation}

We observe that the form $F$ is $\mbox{GL}_2(\mathbb{Z})$-equivalent to a ``reduced'' form (see Section \ref{rep} for details),  given by
$$
\tilde{F} (x,y) = x^3 + 3 x^2y + 4 x y^2 + 6 y^3.
$$
In fact, this is the only form (up to $\mbox{GL}_2(\mathbb{Z})$-equivalence) of discriminant $\pm 4 \cdot 109$. We can check that the solutions to the Thue equation
$\tilde{F}(u,v)=8$ are given by $(u,v)=(2,0)$ and $(u,v)=(-7,3)$. 
The minimal quadratic twist of 
$$
y^2 = x^3 -27 H_{\tilde{F}}(2,0) x +27 G_{\tilde{F}}(2,0)
$$
has conductor $2^5 \cdot 109$ and hence cannot correspond to $E$. For the solution $(u,v)=(-7,3)$, we find that the curve given by the model
$$
y^2 = x^3 -27 H_{\tilde{F}}(-7,3) x +27 G_{\tilde{F}}(-7,3) =x^3 - 10611x+421362,
$$
is the quadratic twist by $-1$ of the curve (\ref{flag}). This situation arises from the fact that $G_F$ is an $\mbox{SL}_2(\mathbb{Z})$-covariant, but not a $\mbox{GL}_2(\mathbb{Z})$-covariant of $F$ (we will discuss this more in the next section).

%---------------------------------------------------------------
\section{Finding representative forms} \label{rep}
%---------------------------------------------------------------

As Theorem \ref{fisk} illustrates, we are able to tabulate elliptic curves over $\mathbb{Q}$ with good reduction outside a given set of primes, 
by finding a set of representatives for $\mbox{GL}_2 ( \mathbb{Z})$-equivalence classes of binary cubic forms with 
certain discriminants, and then solving a number of Thue-Mahler equations. In this section, we will provide a brief description of techniques to find 
distinguished \emph{reduced} representatives for equivalence classes of cubic forms over a given range of 
discriminants. For both positive and negative discriminants, the notion of \emph{reduction} arises from associating a 
particular definite quadratic form to a given cubic form. 
%We do not make these quadratic forms explicit, but rather 
%state our definitions of reduction solely in terms of the coefficients of the cubic forms.

%-------------------------------------------------------
\subsection{Irreducible Forms}
%-------------------------------------------------------
For forms of positive discriminant, there is a well developed classical theory of reduction dating back to work of
Hermite \cite{Her1}, \cite{Her2} and, later, Davenport (see e.g. \cite{Dav}, \cite{Dav2} and \cite{DaHe}).  We can actually apply 
this method to both reducible and irreducible forms. Initially, though, we will assume the forms are irreducible, since we will treat 
the elliptic curves corresponding to reducible forms by a somewhat different approach (see Section~\ref{ssec reducible}). 
Note that when one speaks of ``irreducible, reduced forms'',
as Davenport observes,
``the terminology is unfortunate, but can hardly be avoided'' (\cite{Dav3}, page 184).

In each of Belabas \cite{Be}, Belabas and Cohen \cite{BeCo} and Cremona 
\cite{Cr}, we find very efficient algorithms for computing cubic forms of both positive and negative discriminant, refining classical work of Hermite, Berwick 
and Mathews \cite{BeMa}, and Julia \cite{Ju}. These are  readily translated into computer code to loop over valid $(a,b,c,d)$-values (with corresponding forms $ax^3+bx^2y+cxy^2+dy^3$).  The running time in each case is linear in the upper bound $X$.
Realistically, this step (finding representatives for our cubic forms) is highly unlikely to be the bottleneck in our computations.
  
%----------------------------------------------------------------------
\subsection{Reducible forms} \label{ssec reducible}
%----------------------------------------------------------------------
One can make similar definitions of reduction for reducible forms (see \cite{BeGh} for example). However, for our 
purposes, it is sufficient to note that a reducible form is equivalent to 
$$
F(x,y) = b x^2 y + c x y^2 + d y^3 \; \; \mbox{ with }  \; \; 0 \leq d \leq c,
$$
which has discriminant
\begin{align*}
\Delta_F &= b^2 (c^2 - 4 b d).
\end{align*}

To find all elliptic curves with good reduction outside $S = \{ p_1, p_2, \ldots, p_k \}$, corresponding to 
reducible cubics in Theorem~\ref{fisk} (i.e. those $E$ with at least one rational $2$-torsion point), it is enough to find all such triples $(b,c,d)$ for which there exist integers $x$ and $y$ so that both
$$
b^2(c^2-4bd) \; \; \mbox{ and } \; \;  b x^2 y + c x y^2 + d y^3
$$
are $S^*$-units (with $S^* = S \cup \{ 2 \}$). For this to be true, it is necessary that each of the integers
$$
b, \;  \; c^2 - 4 bd,\;  \; y \; \; \mbox{ and } \; \;  \mu = bx^2 + cxy + dy^2
$$
is an $S^*$-unit. Taking the discriminant of $\mu$ as a function of $x$, we thus require that 
\begin{align} \label{weger}
(c^2 - 4 bd )  y^2  + 4 b \mu &=Z^2,
\end{align}
for some integer $Z$. This is an equation of the shape 
\begin{align} \label{S-square}
 X+Y &=Z^2
\end{align}
in $S^*$-units $X$ and $Y$. 

An algorithm for solving such equations is described in detail in Chapter~7 of de~Weger \cite{Weg0} (see also 
\cite{Weg}); it relies on bounds for linear forms in $p$-adic and complex logarithms and various reduction 
techniques from Diophantine approximation. An implementation of this is available at 
\begin{center}
\url{http://www.nt.math.ubc.ca/BeGhRe/Code/UBC-TMCode}.
\end{center}

While \emph{a priori} equation~\eqref{S-square} arises as only a necessary condition for the existence 
of an elliptic curve of the desired form, given any solution to~\eqref{S-square} in $S^*$-units $X$ and $Y$ and integer $Z$, the curves 
$$
E_1(X,Y) \; \; : \; \; y^2 = x^3 + Z x^2 + \frac{X}{4} x
$$
and
$$
E_2(X,Y) \; \; : \; \; y^2 = x^3 + Z x^2 + \frac{Y}{4} x
$$
have nontrivial rational $2$-torsion (i.e. the point corresponding to $(x,y)=(0,0)$) and discriminant $X^2Y$ and $XY^2$, respectively (and hence good reduction at all primes outside $S^*$). 

Though a detailed analysis of running times for solving equations of the shape (\ref{S-square}), or for solving more general cubic Thue-Mahler equations, has not to our knowledge been carried out, our experience from carrying out such computations for several thousand sets $S$ is that, typically,  the former can be done significantly faster than the latter. By way of example, solving (\ref{S-square}) for $S=\{ 2, 3, 5, 7, 11 \}$ takes only a few hours on a laptop, while treating the analogous problem of determining all elliptic curves over $\mathbb{Q}$ with trivial rational $2$-torsion and good reduction outside $S$ (see Section \ref{gummy-bear}) requires many thousand machine-hours.

%--------------------------------------------------------------------
\subsection{Computing forms of fixed discriminant} \label{pickles}
%--------------------------------------------------------------------

For our purposes, we will typically compute and tabulate a large list of irreducible forms of absolute discriminant bounded by a given positive number $X$ (of size up to $10^{12}$ of so, beyond which storage becomes problematical). In certain situations, however, we will want to compute all forms of a given fixed, larger discriminant (perhaps up to size $10^{15}$). To carry this out and find desired forms of the shape $ax^3+bx^2y+cxy^2+dy^3$, we can argue as in, for example, Cremona \cite{Cr}, to restrict our attention to $O(X^{3/4})$ triples $(a,b,c)$.
From (\ref{claire-bear}), the definition of $D_F$, we have that
$$
d = \frac{9abc-2b^3 \pm \sqrt{4 (b^2-3ac)^3-27 a^2 D_F}}{27 a^2}
$$
and hence it remains to check that the quantity $4 (b^2-3ac)^3-27 a^2 D_F$ is an integer square, that the relevant conditions modulo $27a^2$ are satisfied, and that a variety of further inequalities from \cite{Cr} are satisfied. The running time for finding forms with discriminants of absolute value of size $X$ via this approach is of order $X^{3/4}$.

%----------------------------------------------------------------------------------------------------------------
\subsection{$\mbox{GL}_2(\mathbb{Z})$ vs $\mbox{SL}_2(\mathbb{Z})$} \label{note}
%----------------------------------------------------------------------------------------------------------------

One last observation which is very important to make before we proceed, is that while $G_F^2$ is $\mbox{GL}_2(\mathbb{Z})$-covariant\index{covariants}, the same is not actually true for $G_F$ (it is, however, an $\mbox{SL}_2(\mathbb{Z})$-covariant). This may seem like a subtle point, but what it means for us in practice is that, having found our $\mbox{GL}_2(\mathbb{Z})$-representative forms $F$ and corresponding curves of the shape $E_{\mathcal{D}}$ from Theorem \ref{fisk}, we need, in every case, to also check to see if  
$$
\tilde{E}_{\mathcal{D}} \; \; : \; \; 3^{[\beta_0/3]} y^2 = x^3 -27 \mathcal{D} ^2 H_F(u,v) x -27 \mathcal{D} ^3 G_F(u,v),
$$
the quadratic twist of $E_{\mathcal{D}}$ by $-1$, yields a curve of the desired conductor\index{conductor}.

%--------------------------------------------------------------------------------------------
\section{Examples}\label{examples}
%--------------------------------------------------------------------------------------------

In this section, we will describe a few applications of Theorem \ref{fisk} to computing all elliptic curves of a fixed conductor $N$, or all curves with good reduction outside
a given set of primes $S$. We restrict our attention to examples with composite conductors, since the case of conductors $p$ and $p^2$, for $p$ prime, 
will be treated at length in Section \ref{primes} (and subsequently). For the examples in Sections \ref{lapel}, \ref{exe1}, \ref{exe2} and \ref{exe3}, since the conductors under discussion are not ``square-full'', there are necessarily no curves $E$ encountered with $j_E=0$.

In our computations in this section, we executed all jobs in parallel via the shell tool \cite{Tange2011a}. We note that our Magma code lends itself easily to parallelization, and we made full use of this fact throughout.

We carried out a one-time computation of all irreducible cubic forms that can arise in Theorem \ref{fisk}, of absolute discriminant bounded by $10^{10}$. This computation took slightly more than $3$ hours on a cluster of $40$ cores; roughly half this time was taken up with sorting and organizing output files.  There are $996198693$  classes of  irreducible cubic forms of positive discriminant and $3079102475$ of negative discriminant in the range in question; storing them requires roughly $120$ gigabytes. We could also have tabulated and stored representatives for each class of reducible form of absolute discriminant up to $10^{10}$, but chose not to since our approach to solving equation (\ref{S-square}) does not require them.

%--------------------------------------------------------------------------------------------
\subsection{Cases without irreducible forms} \label{lapel}
%--------------------------------------------------------------------------------------------

We begin by noting an obvious corollary to Theorem \ref{fisk} that, in many cases, makes it a relatively routine matter to determine all elliptic curves of a given conductor, provided  we can show the nonexistence of certain corresponding cubic forms.
\begin{corollary} \label{impulse}
Let $N$ be a square-free positive integer with $\gcd(N,6)=1$ and suppose that there do not exist irreducible binary cubic forms in $\mathbb{Z}[x,y]$ of discriminant $\pm 4N_1$, for each positive integer $N_1 \mid N$. Then every elliptic curve over $\mathbb{Q}$ of conductor $N_1$,  for each  $N_1 \mid N$, has nontrivial rational $2$-torsion.
\end{corollary}

We will apply this result to  a pair of examples (chosen somewhat  arbitrarily). Currently, such an approach is feasible for forms of absolute discriminant (and hence potentially conductors) up to roughly $10^{15}$. 
We observe that, among the positive integers $N < 10^8$ satisfying
$$
\nu_2(N) \leq 8, \; \; \nu_3 (N) \leq 5 \; \; \mbox{ and } \; \; \nu_p (N) \leq 2 \; \mbox{ for } \; p > 3,
$$
i.e. those for which there might actually exist elliptic curves $E/\mathbb{Q}$ of conductor $N$, we find that $708639$ satisfy the hypotheses of Corollary \ref{impulse}. 

It is somewhat harder to modify the statement of Corollary \ref{impulse} to include reducible forms (with corresponding elliptic curves having nontrivial rational $2$-torsion). One of the difficulties one encounters is that there actually do exist reducible forms of, by way of example, discriminant $4p$ for every $p \equiv 1 \mod{8}$; writing $p = 8k+1$, for instance, the form
$$
F(x,y) = 2 x^2y+xy^2-ky^3
$$
has this property.

%--------------------------------------------------------------------------------------------
\subsubsection{Conductor $2655632887 = 31 \cdot 9007 \cdot 9511$}
%--------------------------------------------------------------------------------------------

In the notation of Theorem \ref{fisk}, we have $\alpha=\beta=0$ and hence $\alpha_0 =2$ and $\beta_0 =0$, so that, in order for there to be an elliptic curve with trivial rational $2$-torsion and this conductor, we require the existence of an irreducible cubic form of discriminant $4 N_1$ where 
$N_1 \mid  31 \cdot 9007 \cdot 9511$,
i.e. discriminant $\pm 4 \cdot 31^{\delta_1} \cdot 9007^{\delta_2} \cdot 9511^{\delta_3}$, for $\delta_i \in \{ 0, 1 \}$. We check that there are no such forms, directly from our table of forms, except for the possibility of $D_F = \pm 4 \cdot 31 \cdot 9007 \cdot 9511$, which exceeds $10^{10}$ in absolute value. For these latter possibilities, we argue as in Section \ref{pickles} to show that no such forms exist. We may thus appeal to Corollary \ref{impulse}.

For the possible cases with rational $2$-torsion, we solve $X+Y=Z^2$ with $X$ and $Y$ $S$-units for $S = \{ 2, 31, 9007, 9511 \}$. The solutions to this equation with $X \geq Y$, $Z > 0$ and $\gcd (X,Y)$ squarefree are precisely those with
$$
\begin{array}{lll}
(X,Y) &= &  (2,-1), (2,2), (8,1), (32,-31), (62,2), (256,-31), (961,128), \\
& & (992,-31), (3968,1), (76088,-9007), (294841, 8)\\
& & \mbox{ and } (492032,-9007).  \\
\end{array}
$$
A short calculation confirms that each elliptic curve arising from these solutions via quadratic twist has bad reduction at the prime $2$ (and, in particular, cannot have conductor $2655632887$). There are thus no elliptic curves over $\mathbb{Q}$ with conductor $2655632887$.
Observe that these calculations in fact ensure that there do not exist elliptic curves over $\mathbb{Q}$ with conductor dividing $2655632887$. 

Full computational details are available at
\begin{center}
\url{http://www.nt.math.ubc.ca/BeGhRe/Examples/2655632887-data}.
\end{center}
We observe that it is much more challenging computationally to try to extend this argument to tabulate curves $E$ with good reduction outside 
$S = \{ 31, 9007, 9511 \}$. To do this, we would have to first determine whether or not there exist irreducible cubic forms of discriminant, say,
$D_F = \pm 4 \cdot 31^2 \cdot 9007^2 \cdot 9511^2 > 2.8 \times 10^{19}$. This appears to be at or beyond current computational limits.

%--------------------------------------------------------------------------------------------
\subsubsection{Conductor $3305354359 = 41 \cdot 409 \cdot 439 \cdot 449$}
%--------------------------------------------------------------------------------------------

For there to exist an elliptic curve with trivial rational $2$-torsion and conductor $3305354359$, we require the existence of an irreducible cubic form of discriminant $\pm 4  \cdot 41^{\delta_1} \cdot 409^{\delta_2} \cdot 439^{\delta_3} \cdot 449^{\delta_4}$, with $\delta_i \in \{ 0, 1 \}$. We check that, again, there are no such forms (once more employing a short auxiliary computation in the case $D_F =\pm 4  \cdot 41 \cdot 409 \cdot 439 \cdot 449$). If we solve $X+Y=Z^2$ with $X$ and $Y$ $S$-units for $S = \{ 2, 41, 409, 439, 449 \}$, we find that the solutions to this equation with $X \geq Y$, $Z > 0$ and $\gcd (X,Y)$ squarefree are precisely 
$$
\begin{array}{l}
(X,Y) =  (2,-1), (2,2), (8,1), (41,-16),  (41,-32), (41,8), (82,-1), \\
 \hskip9.6ex  (128,41), (409, -328), (409, 32), (439, 2), (449, -328), (449, -8),\\
\hskip9.6ex   (512, 449),(818, 82), (898, 2), (3272, 449), (3362, 2), (7184, 41),\\
 \hskip9.6ex  (16769, -128), (16769, -14368),  (18409, -16384), \\
\hskip9.6ex   (33538, -18409),(36818, 818), (41984, 41), (68921, -57472), \\
\hskip9.6ex   (183641, -1312),(183641, -56192), (183641, 41984), \\
\hskip9.6ex  (359102, 898), (403202, -33538),(403202, -359102), \\
\hskip9.6ex (403202, 17999), (737959, 183641), (754769, -6544), \\
\hskip9.6ex (6858521, -919552), (8265641, -16) \\
\hskip9.6ex \mbox{and } (7095601778, -5610270178). \\
\end{array}
$$
Once again, a short calculation confirms that each elliptic curve arising from these solutions via twists has even conductor. There are thus no elliptic curves over $\mathbb{Q}$ with conductor $3305354359$.

Full computational details are available at
\begin{center}
\url{http://www.nt.math.ubc.ca/BeGhRe/Examples/3305354359-data}.
\end{center}

%------------------------------------------------------------------------------------------------------
\subsection{Cases with fixed conductor (and corresponding irreducible forms)}
%------------------------------------------------------------------------------------------------------

%--------------------------------------------------------------------------------------------
\subsubsection{Conductor $399993 = 3 \cdot 11 \cdot 17 \cdot 23 \cdot 31$} \label{exe1}
%--------------------------------------------------------------------------------------------

We next choose an example where full data is already available for comparison in the LMFDB \cite{LMFDB}. In particular, there are precisely $10$ isogeny classes of curves of this conductor (labelled $399993a$ to $399993j$ in the LMFDB), containing a total of $21$ isomorphism classes. Of these, $7$ isogeny classes (and $18$ isomorphism classes) have nontrivial rational $2$-torsion.

According to Theorem \ref{fisk}, the curves arise from  consideration of cubic forms of discriminant discriminant $\pm 4  K$, where $K \mid 3 \cdot 11 \cdot 17 \cdot 23 \cdot 31$. 
The (reduced)  irreducible cubic forms $F(u,v)$ of these discriminants are as follows, where $F(u,v) =  \omega_0 u^3 + \omega_1 u^2v + \omega_2 uv^2 + \omega_3 v^3$.

$$
\begin{array}{cc|cc} 
(\omega_0,\omega_1,\omega_2,\omega_3) & D_F & (\omega_0,\omega_1,\omega_2,\omega_3)  & D_F \\ \hline
(1,1,1,3) & -4 \cdot 3 \cdot 17 & (2, 4, -6, -3) & 4 \cdot 3 \cdot 17 \cdot 23 \\
(1,2,2,2) & - 4 \cdot 11 & (2, 5, 2, 6) & -4 \cdot 3 \cdot 17 \cdot 23 \\
(1,2,2,6) & -4 \cdot 11 \cdot 17 & (3, 3, -8, -2) & 4 \cdot 3 \cdot 23 \cdot 31 \\
(1, 4, -16, -2) & 4 \cdot 11 \cdot 17 \cdot 31 & (3, 3, 44, 66) &  -4 \cdot 3 \cdot 11 \cdot 17 \cdot 23 \cdot 31 \\
(1, 8, -2, 42) & -4 \cdot 3 \cdot 17 \cdot 23 \cdot 31 & (3, 4, 10, 14) & -4  \cdot 11 \cdot 23 \cdot 31 \\
(1,11,-12,-6) & 4 \cdot 3 \cdot 11 \cdot 17 \cdot 31 & (3, 7, 5, 7) & -4 \cdot 3 \cdot 23 \cdot 31 \\
(2,0,7,1) & -4 \cdot  23 \cdot 31  & (4, 17, 10, 28) & -4 \cdot 11 \cdot 17 \cdot 23 \cdot 31 \\ 
(2, 1, 14, -2) & -4 \cdot 11 \cdot 17 \cdot 31 & & \\
\end{array}
$$


\vskip1.2ex
In each case, we are thus led to solve the Thue-Mahler equation
\begin{equation} \label{lonely}
F(u,v) = 2^{3 \delta} 3^{\beta_1} 11^{\kappa_{11}} 17^{\kappa_{17}} 23^{\kappa_{23}} 31^{\kappa_{31}},
\end{equation}
where $\gcd(u,v) =1$, $\delta \in \{ 0, 1 \}$ and $\beta_1$, $\kappa_{11}$, $\kappa_{17}$, $\kappa_{23}$ and $\kappa_{31}$ are arbitrary nonnegative integers. Applying (\ref{term3}), in order to find a curve of conductor $399993$, we require additionally that, for a corresponding solution to (\ref{lonely}), 
\begin{equation} \label{mary}
F(u,v) \, D_F \equiv 0 \mod{ 3 \cdot 11 \cdot 17 \cdot 23 \cdot 31}.
\end{equation}
We readily check that the congruence $F(u,v) \equiv 0 \mod{p}$ has only the solution $u \equiv v \equiv 0 \mod{p}$ for the following forms $F$ and primes $p$ (whereby (\ref{mary}) cannot be satisfied by coprime integers $u$ and $v$ for these forms) :


$$
\begin{array}{cc|cc} 
(\omega_0,\omega_1,\omega_2,\omega_3) & p & (\omega_0,\omega_1,\omega_2,\omega_3)  & p \\ \hline
(1,1,1,3) & 11,23 & (2,0,7,1) & 3, 17 \\
(1,2,2,2) & 3, 23, 31 & (2, 5, 2, 6) & 11, 31 \\
(1, 4, -16, -2) & 3, 23 & (3,3,-8,-2) & 11  \\
(1, 8, -2, 42) & 11 & (4, 17, 10, 28) & 3 \\
(1,11,-12,-6) & 23 & & \\
\end{array}
$$


For the remaining $6$ forms under consideration, we appeal to UBC-TM. The only solutions we find satisfying (\ref{mary})
are as follows.
$$
\begin{array}{c|c} 
(\omega_0,\omega_1,\omega_2,\omega_3) & (u,v) \\ \hline
(1, 2, 2, 6) & (-1851, 892), (14133, -3790) \\
(2, 1, 14, -2) & (13, -5), (-29, -923) \\
(2, 4, -6, -3) & (10,-3), (64, 49), (-95, 199), (-3395, 1189), \\
& (3677, -1069), (5158, 4045), (-23546, 57259), \\
& (-77755, 30999) \\
(3,3,44,66) & (1,0), (1,2), (-3,4), (3,-2), (-11,9), (25,-3), \\
& (231,2), (-317,240), (489,61), (1263, -878), (6853, -4119) \\
(3, 7, 5, 7) & (1, 12), (-29, 26), (78, 1),  (423, -160) \\
(3, 4, 10, 14) & (-41, 84), (95, -69), (307, 90) \\
\end{array}
$$ 
From these, we compute the conductors of $E_{\mathcal{D}}$ in (\ref{curvey}), where $\mathcal{D} \in \{ 1, 2 \}$, together with their twists by $-1$.
The only curves with conductor $399993$ arise from the form $F$ with $(\omega_0,\omega_1,\omega_2,\omega_3)=(2,4,-6,-3)$ and the solutions
$$
(u,v) \in \left\{ (10,-3), (5158,4045),  (-23546, 57259) \right\}.
$$
In each case, $\mathcal{D}=2$. The solution $(u,v)=(10,-3)$ corresponds to, in the notation of the LMFBD, curve 399993.j1, $(u,v)=(5158,4045)$ to 399993.i1, and $(u,v)=(-23546, 57259)$ to 399993.h1. Note that every form and solution we consider leads to elliptic curves with good reduction outside $\{ 2, 3, 11, 17, 23, 31 \}$, just not necessarily of  conductor $399993$. By way of example, if $(\omega_0,\omega_1,\omega_2,\omega_3) =(2, 4, -6, -3)$ and $(u,v)=(-77755, 30999)$, we find curves with minimal quadratic twists of conductor 
$$
2^5 \cdot 3 \cdot 11 \cdot 17 \cdot 23 \cdot 31 = 2^5 \cdot 399993.
$$

To determine the curves of conductor $399993$ with nontrivial rational $2$-torsion, we are led to solve the equation $X+Y=Z^2$ in $S$-units $X$ and $Y$, and integers $Z$, where $S=\{2,3,11,17,23,31 \}$.
We employ Magma code available at
\begin{center}
\url{http://nt.math.ubc.ca/BeGhRe/Code/UBC-TMCode}
\end{center}
to find precisely $2858$ solutions with $X \geq |Y|$ and  $\gcd (X,Y)$ squarefree (this computation took slightly less than $2$ hours). Of these, $1397$ have $Z > 0$, with $Z$ largest for the solution corresponding to the identity
$$
48539191572432 - 40649300451407 = 2^4 \cdot 3^4 \cdot 11 \cdot 23^7 - 17^5 \cdot 31^5 = 2808895^2.
$$
As in subsection \ref{ssec reducible}, we attach to each solution a pair of elliptic curves $E_1(X,Y)$ and $E_2(X,Y)$.
Of these, the only twists we find to have conductor $399993$ are the quadratic twists by $t$ of $E_i(X,Y)$  given in the following table. Note that there is some duplication -- the curve labelled 399993.f2 in the LMFDB, for example, arises from three distinct solutions to $X+Y=Z^2$.

$$
\begin{array}{c|c|c|c|c}
X & Y & E_i & t & \mbox{ LMFDB} \\ \hline
16192 & -4743 & E_1 & -1 & 399993.g2 \\
16192 & -4743 & E_2 & 2 & 399993.g1 \\
23529 & 18496 & E_1 & -2 & 399993.f2 \\
23529 & 18496 & E_2 & 1 & 399993.f3 \\ 
116281 & -75072 & E_1 & 2 & 399993.f4 \\
116281 & -75072 & E_2 & -1 & 399993.f2 \\
371008 & 4761 & E_1 & 1 & 399993.f2 \\
371008 & 4761 & E_2 & -2 & 399993.f1 \\
519777 & -131648 & E_1 & 2 & 399993.d2   \\
519777 & -131648 & E_2 & -1 & 399993.d1 \\
534336 & -506447 & E_1 & -1 & 399993.e2 \\
534336 & -506447 & E_2 & 2 & 399993.e1 \\
1311552 & -527 & E_1 & 1 & 399993.a2 \\
1311552 & -527 & E_2 & -2 & 399993.a1\\
1414017 & -1045568 & E_1 & 2 & 399993.b2 \\
1414017 & -1045568 & E_2 & -1 & 399993.b1 \\
6305121 & 3027904 & E_1 & 2 & 399993.c1  \\
6305121 & 3027904 & E_2 & -1 & 399993.c2 \\
6988113 & 18496 & E_1 & 2 & 399993.c2 \\
6988113 & 18496 & E_2 & -1 & 399993.c3\\
7745089 & -2731968 & E_1 & 2 & 399993.c4\\
7745089 & -2731968 & E_2 & -1 & 399993.c2 \\
\end{array}
$$

Full computational details are available at
\begin{center}
\url{http://www.nt.math.ubc.ca/BeGhRe/Examples/399993-data}.
\end{center}

%--------------------------------------------------------------------------------------------
\subsubsection{Conductor $10^6-1$}  \label{exe2}
%--------------------------------------------------------------------------------------------

We next treat a slightly larger conductor, which is not available in the LMFDB currently (but probably within computational range).
We have 
$$
10^6-1 = 3^3 \cdot 7 \cdot 11 \cdot 13 \cdot 37.
$$
From Theorem \ref{fisk}, we thus need to consider binary cubic forms $F(u,v) =  \omega_0 u^3 + \omega_1 u^2v + \omega_2 uv^2 + \omega_3 v^3$ of discriminant $D_F = \pm 108 N_1$, where $N_1 \mid 7 \cdot 11 \cdot 13 \cdot 37$ and $\omega_1 \equiv \omega_2 \equiv 0 \mod{3}$. The irreducible forms of this shape are as follows.

$$
\begin{array}{ccc} 
(\omega_0,\omega_1,\omega_2,\omega_3) & D_F & p \\ \hline
(1,0,-6,-2) & 108 \cdot 7 & 37 \\
(1,0,21,16) & -108 \cdot 11 \cdot 37 & 7, 13 \\
(1,0,30,2) & -108 \cdot 7 \cdot 11 \cdot 13 & \mbox{ none } \\
(1, 3, 3, 3) & -108 & 7, 13, 37  \\
(1, 3, 6, 16) & -108 \cdot 37 & 7 \\
 %(2,10,-26,-13) & 108 \cdot 7 \cdot 13 \cdot 37 & 11  \\
(1,3,12,26) & -108 \cdot 7 \cdot 13 & \mbox{ none } \\
(1,3,33,117) & -108 \cdot 7 \cdot 11 \cdot 37 & \mbox{ none }  \\ 
(1,6,-36,-34) & 108 \cdot 7 \cdot 13 \cdot 37 & 11 \\
(1,6,3,6) & -108 \cdot 37 & 7 \\
(1,6,9,26) & -108 \cdot 11 \cdot 13 & \mbox{ none } \\
(1,9,0,74) & -108 \cdot 7 \cdot 13 \cdot 37 & \mbox{ none } \\
(1,12,12,14) & -108 \cdot 13 \cdot 37 & 11  \\
%(6,32,-35,-7) & 108 \cdot 7 \cdot 11 \cdot 13 \cdot 37 & \mbox{ none } \\
(2, 0, -18, -5) & 108 \cdot 11 \cdot 37 & 13  \\
(2,0,3,3) & -108 \cdot 11 & 7, 37   \\
(2,0,15,3) & -108 \cdot 7 \cdot 37 & 11, 13 \\
(2,0,18,7) & -108 \cdot 13 \cdot 37 & 11 \\
 (2,3,-78,-26) & 108 \cdot 7 \cdot 11 \cdot 13 \cdot 37 & \mbox{ none }\\
(2,3,6,3) & -108 \cdot 7 & 11, 37\\
  (2,3,6,8) & -108 \cdot 37 & 7 \\
 (2, 6, -12, 1) & 108 \cdot 11 \cdot 13 & 7\\
(2,6,21,88) & -108 \cdot 11 \cdot 13 \cdot 37 &  \mbox{ none } \\
(2,12,0,13) & -108 \cdot 7 \cdot 11 \cdot 13 &  \mbox{ none }\\
(2,21,-6,80) &  -108 \cdot 7 \cdot 11 \cdot 13 \cdot 37 &  \mbox{ none } \\
(3,3,18,20) &  -108 \cdot 7 \cdot 11 \cdot 13 &  \mbox{ none } \\
(4,6,15,14) & -108 \cdot 13 \cdot  37 & 11 \\
(5,6,27,14) & -108 \cdot 7 \cdot 11 \cdot 37 &  \mbox{ none } \\
 (5,9,3,21) & -108 \cdot 7 \cdot 11 \cdot 37 &  \mbox{ none } \\
(7,0,12,14) &  -108 \cdot 7 \cdot 11 \cdot 37 &  \mbox{ none } \\
(10,3,42,-16) & -108 \cdot 7 \cdot 11 \cdot 13 \cdot 37 &  \mbox{ none }\\
(10,6,12,3) & -108 \cdot 13 \cdot  37 &  \mbox{ none }\\
 (11,6,12,6) & -108 \cdot 7 \cdot 11 \cdot 13  &  \mbox{ none }\\
(21,12,27,20) &  -108 \cdot 7 \cdot 11 \cdot 13 \cdot 37 &  \mbox{ none }  \\
\end{array}
$$

Here, we list primes $p$ for which a local obstruction exists modulo $p$ to finding coprime integers  $u$ and $v$ satisfying (\ref{term3}). 
It is worth noting at this point that the restriction to forms with $\omega_1 \equiv \omega_2 \equiv 0 \mod{3}$ that follows from the fact that we are considering a conductor divisible by $3^3$ is a helpful one. There certainly can and do exist irreducible forms $F$ with $108 \mid D_F$ that fail to satisfy $\omega_1 \equiv \omega_2 \equiv 0 \mod{3}$.

We are thus left to treat $17$ Thue-Mahler equations which we solve using UBC-TM; see 
\begin{center}
\url{http://www.nt.math.ubc.ca/BeGhRe/Examples/999999-data}
\end{center}
 for computational details. From (\ref{term3}), we require that
$D_F F(u,v) \equiv 0 \mod{7 \cdot 11 \cdot 13 \cdot 37}$; the only solutions we find satisfying this constraint are as follows.

$$
\begin{array}{c|c} 
(\omega_0,\omega_1,\omega_2,\omega_3) & (u,v) \\ \hline
(1, 0, 30, 2) & (-1,21), (1,16), (27,25) \\
(1,3,33,117) & (26,-7) \\
(1,9,0,74) &  (-19,2) \\
(2,3,-78,-26) & (-1,3), (-3,2), (-5,-1), (9,-1), (13,2), (-17,-58), \\
&    (-39,-61),(-57,-10), (-59,9), (65,-6), (79,-330), \\
& (159,-23) \\
(2,6,21,88) & (3,1), (165,-43) \\
(2,12,0,13) & (-1,9), (18,23) \\
(2,21,-6,80) & (1, -10), (2,1), (4, -3), (4,-1), (17, 1),\\
& 	(19, -5),(21, -2 ),(138, -11 ),(1356, -127) \\
(3,3,18,20) & (9,13), (97,-12) \\
(5,6,27,14) & (14,1),  (19,6),  (-21,44) \\
(5,9,3,21) & (-1,2), (6,1), (8,-3), (-649,284), (1077,-464)  \\
% (6,32,-35,-7)  & (0,-1), (2,-1), (2,-3),(16,3), (17,-3), (28,25), (95, -13), (859,-5009) \\
(7,0,12,14) & (-1,5), (-7,9), (301,-62), (-459,553)  \\
(10,3,42,-16) & (1,1), (1,2), (2,-1), (3,1), (4,-17), (20,19), (-22,-69), \\
& (127,339) \\
(10,6,12,3) & (2,-1), (5,-13), (-12,83), (-24,89), (81,-107), \\
&    (125,-437) \\
(11,6,12,6) & (-1,22), (47,-72), (223,-429) \\
(21,12,27,20) & (1, -3 ),(1, 0 ),(1, 5 ),(4, -9),(4, 3 ),(9, -29 ),\\
& 	(19, -15 ),(29, -40 ),(316, -455 ),(551, -805)\\
\end{array}
$$ 


The only ones of these for which we find an $E_{\mathcal{D}}$ in (\ref{curvey})  of conductor $999999$ are as follows, where $E_{\mathcal{D}}$ is isomorphic over $\mathbb{Q}$ to a curve with model
$$
y^2 + a_1 xy + a_3 y = x^3 + a_2 x^2 + a_4 x + a_6.
$$

$$
\begin{array}{c|c|c|c|c|c|c|c} 
(\omega_0,\omega_1,\omega_2,\omega_3) & (u,v) & \mathcal{D} & a_1 & a_2 & a_3 & a_4 & a_6 \\ \hline
(1,0,30,2) & (27,25) & 6 & 0 & 0 & 1 & -40395 & 5402579  \\
(1,0,30,2) & (27,25) & -2 & 0 & 0 & 1 & -363555 & -145869640  \\
(5,6,27,14) & (14,1) & 1 & 1 & -1 & 0 & 14700 & 55223 \\
(5,6,27,14) & (14,1) & -3 & 1 & -1 & 1 & 1633 & -2590 \\
(5,9,3,21) & (-1,2) & 6 & 0 & 0 & 1 & 30 & 2254 \\
(5,9,3,21) & (-1,2) & -2 & 0 & 0 & 1 & 270 & -60865 \\
(10,6,12,3) & (125,-437) & 2 & 0 & 0 & 1 & -17205345 & -27554570341 \\
(10,6,12,3) & (125,-437) & -6 & 0 & 0 & 1 & -1911705 & 1020539642 \\
(21,12,27,20) & (4,3) & -1 & 1 & -1 & 0 & 12432 & -164125 \\
(21,12,27,20) & (4,3) & 3 & 1 & -1 & 1 & 1381 & 5618 \\
\end{array}
$$ 

Each of these listed curves has trivial rational $2$-torsion. To search for curves of conductor $999999$ with nontrivial rational $2$-torsion, we solve the equation $X+Y=Z^2$ in $S$-units $X$ and $Y$, and integers $Z$, where $S=\{2,3,7, 11,13, 37 \}$.
We find that there are precisely $4336$ solutions with $X \geq |Y|$ and  $\gcd (X,Y)$ squarefree. Of these, $2136$ have $Z > 0$, with $Z$ largest for the solution corresponding to the identity
$$
103934571636753 - 68209863326528 = 3^{15} \cdot 11 \cdot 13 \cdot 37^3 - 2^6 \cdot 7^{13} \cdot 11 =   5977015^2.
$$
Once again, we attach to each solution a pair of elliptic curves $E_1(X,Y)$ and $E_2(X,Y)$. We find $505270$ isomorphism classes of $E/\mathbb{Q}$ with good reduction outside of $\{ 2, 3, 7, 11, 13, 37 \}$ and nontrivial rational $2$-torsion. None of them have  conductor $999999$, whereby we conclude that there are precisely  $10$ isomorphism classes of elliptic curves over $\mathbb{Q}$ with conductor $10^6-1$. Checking that these curves each have distinct traces of Frobenius $a_{47}$ shows that they are nonisogenous.


%--------------------------------------------------------------------------------------------
\subsubsection{Conductor $10^9-1$}  \label{exe3}
%--------------------------------------------------------------------------------------------
 
This example is chosen to be somewhat beyond the current scope of the LMFDB.
We have
$$
10^9-1 = 3^4 \cdot 37 \cdot 333667
$$
and so, applying Theorem \ref{fisk}, we are led to consider binary cubic forms of discriminant $\pm 4 \cdot 3^4 \cdot 37^{\delta_1} \cdot 333667^{\delta_2}$, where $\delta_i \in \{ 0, 1 \}$. These include imprimitive forms with the property that each of its coefficients $\omega_i$ is divisible by $3$. For such forms, from Theorem \ref{fisk}, we necessarily have $\beta_1 \in \{ 0, 1 \}$ and hence $\beta_1=1$. Dividing through by $3$, we may thus 
restrict our attention to primitive forms of discriminant $\pm 4 \cdot 3^\kappa \cdot 37^{\delta_1} \cdot 333667^{\delta_2}$, where $\delta_i \in \{ 0, 1 \}$
 and $\kappa \in \{ 0, 4 \}$.
For the irreducible forms, we have, by slight abuse of notation (since, for the $F$ listed here with $D_F \not\equiv 0 \mod{3}$, the form whose existence is guaranteed by Theorem \ref{fisk} is actually $3F$), the following.

$$
\begin{array}{ccc} 
(\omega_0,\omega_1,\omega_2,\omega_3) & D_F & p \\ \hline
(1,1,-3,-1) & 4 \cdot 37 & 333667 \\
(1,4,52,250) & -4 \cdot 333667 & 37 \\
(1,9,37,279) & -4 \cdot 333667 &   \mbox{ none }\\
(1,21,117,2135) & -4 \cdot 3^4 \cdot 333667 & 37  \\
(2,0,3,1) & -4 \cdot 3^4 & 37  \\
(2,17,-26,-31) & 4 \cdot 333667 & 37  \\
(4,30,117,665) & -4 \cdot 3^4 \cdot 333667 & 37 \\
(4,35,14,216) & -4 \cdot 37 \cdot 333667 & \mbox{ none }\\
(5,6,9,6) & -4 \cdot 3^4 \cdot 37 & \mbox{ none } \\ 
(5,7,19,51) & -4 \cdot 333667 & 37  \\
 (5,14,19,54) &  -4 \cdot 333667 & 37\\
 (6,18,168,323) & -4 \cdot 3^4 \cdot 333667 & 37\\
 (6,27,42,356) & -4 \cdot 3^4 \cdot 333667 & 37 \\
 (6,54,-48,115) & -4 \cdot 3^4 \cdot 333667 & 37\\
 (10,18,96,229) &  -4 \cdot 3^4 \cdot 333667 & 37\\
 (26,9,102,4) & -4 \cdot 3^4 \cdot 333667 & \mbox{ none }\\
  (27,7,70,32) & -4 \cdot 37 \cdot 333667 & \mbox{ none }\\
  (31,9,87,-25) & -4 \cdot 3^4 \cdot 333667 & \mbox{ none } \\
 (49,51,63,55) & -4 \cdot 3^4 \cdot 333667  & \mbox{ none } \\
(52,55,72,37) & -4 \cdot 37 \cdot 333667 & \mbox{ none }\\
\end{array}
$$
Once again, we list primes $p$ for which a local obstruction exists modulo $p$ to finding coprime integers  $u$ and $v$ satisfying (\ref{term3}). There are thus $8$ Thue-Mahler equations left to solve.
In the (four) cases where  $D_F \not\equiv 0 \mod{3}$, these take the shape
$$
F(u,v) = 2^{3 \delta_1} \cdot 37^{\gamma_1} \cdot 333667^{\gamma_2},
$$
where $\delta_1 \in \{ 0, 1 \}$, $\gamma_1$ and $\gamma_2$ are nonnegative integers, and $u$ and $v$ are coprime integers. For the remaining $F$, the analogous equation is
$$
F(u,v) = 2^{3 \delta_1} \cdot 3^{\delta_2} \cdot 37^{\gamma_1} \cdot 333667^{\gamma_2},
$$
where $\delta_i \in \{ 0, 1 \}$, $\gamma_1, \gamma_2 \in \mathbb{Z}^+$ and $u, v \in \mathbb{Z}$ with $\gcd(u,v)=1$. We solve these equations using the UBC-TM Thue-Mahler solver. The only cases where we find  that
$$
D_F F(u,v) \equiv 0 \mod{37 \cdot 333667}
$$
occur for $(\omega_0,\omega_1,\omega_2,\omega_3)=(4,35,14,216)$ and $(u,v)=(-8,1)$ or $(u,v)=(-2,1)$, for
 $(\omega_0,\omega_1,\omega_2,\omega_3)=(27,7,70,32)$ and $(u,v)=(1,-2)$ or $(2,-1)$, and for $(\omega_0,\omega_1,\omega_2,\omega_3)=(52,55,72,37)$ and 
 $(u,v)=(0,1)$ or $(-3,5)$. In each case, all resulting  twists have bad reduction at $2$ (and hence cannot have conductor $10^9-1$).
 
 To search for curves with nontrivial rational $2$-torsion and conductor $10^9-1$, we solve the equation $X+Y=Z^2$ in $S$-units $X$ and $Y$, and integers $Z$, where 
 $S=\{2,3, 37, 333667 \}$. There are precisely $98$ solutions with $X \geq |Y|$ and  $\gcd (X,Y)$ squarefree. Of these, $41$ have $Z > 0$, with $Z$ largest for the solution coming from  the identity
$$
27027027 -101306 = 3^4 \cdot 333667 -  2 \cdot 37^3 =  5189^2.
$$
These correspond via twists to elliptic curves of conductor as large as $2^8 \cdot 3^2 \cdot 37^2 \cdot 333667^2$, but none of conductor $10^9-1$. There thus exist no curves $E/\mathbb{Q}$  of conductor $10^9-1$.

Full computational details are available at
\begin{center}
\url{http://www.nt.math.ubc.ca/BeGhRe/Examples/999999999-data}.
\end{center}



%----------------------------------------------------------------------------------------------------------------------
\subsection{Curves with good reduction outside $\{ 2, 3, 23 \}$ : an example of Koutsianis and of von Kanel and Matchke}
%----------------------------------------------------------------------------------------------------------------------

This case was worked out by Koutsianis \cite{Kou} (and also by von Kanel and Matschke \cite{KanMat}, who actually computed $E/\mathbb{Q}$ with good reduction outside $\{ 2, 3, p \}$ for all prime $p \leq 163$), by rather different methods from those employed here. We include it here to provide an example where we determine all $E/\mathbb{Q}$ with good reduction outside a specific set $S$, which is of somewhat manageable size in terms of the set of cubic forms encountered. Our data agrees with that  of \cite{KanMat} and \cite{Kou}.

To begin, we observe that the elliptic curves with good reduction outside $\{ 2, 3, 23 \}$ and $j$-invariant $0$ are precisely those with models of the shape
$$
E \; \; : \; \; Y^2 = X^3 \pm 2^a 3^b 23^c, \; \; \mbox{ where } \; 0 \leq a, b, c \leq 5.
$$
Appealing to (\ref{lumpy}), we next  look through our precomputed list to find  all the irreducible primitive cubic forms of discriminant $\pm 2^{\alpha} 3^\beta 23^\gamma$,
where 
$$
\alpha \in \{ 0, 2, 3, 4 \}, \; \; \beta \in \{ 0, 1, 3, 4, 5 \} \; \; \mbox{ and } \; \; \gamma \in \{ 0, 1, 2 \}. 
$$
The imprimitive forms we need consider correspond to primitive
forms $F$ with either $\nu_2 (D_F) = 0$ or $\nu_3 (D_F) \in \{ 0, 1 \}$. We find precisely $95$ irreducible, primitive cubic forms of the desired discriminants.

\begin{landscape}
$$
\begin{array}{cc|cc|cc} 
(\omega_0,\omega_1,\omega_2,\omega_3) & D_F & (\omega_0,\omega_1,\omega_2,\omega_3)  & D_F & (\omega_0,\omega_1,\omega_2,\omega_3)  & D_F \\ \hline
(1,0,-18,-6) & 2^2 \cdot 3^5 \cdot 23 & (2, 0, 3, 4) & -2^3 \cdot 3^5  & (4, 9, 24, 29) & -2^2 \cdot 3^4 \cdot 23^2 \\
(1,0,-3,-1) & 3^4 &   (2, 3, 6, 4) & -2^2 \cdot 3^5  & (4,12,12,27) & -2^4 \cdot 3^3 \cdot 23^2 \\
(1,0,3,2) & -2^3 \cdot 3^3 & (2, 3, 12, 8) & -2^4 \cdot 3^3 \cdot 23 & (4,12,12,73) & -2^4 \cdot 3^5 \cdot 23^2 \\
(1,0,6,2) & -2^2 \cdot 3^5 & (2, 3, 36, 29) & -2^3 \cdot 3^4 \cdot 23^2  & (4, 18, 9, 24) & -2^2 \cdot 3^5 \cdot 23^2  \\
(1,0,6,4) & -2^4 \cdot 3^4 & (2, 3, 36, 98) & -2^3 \cdot 3^5 \cdot 23^2  & (4, 18, 27, 48) & -2^2 \cdot 3^5 \cdot 23^2 \\
(1,0,9,6) & -2^4 \cdot 3^5 & (2, 5, 8, 15) & -2^3 \cdot 3 \cdot 23^2  & (5, 6, 7, 4) & -2^3 \cdot  23^2 \\
(1,0,33,32) & -2^2 \cdot 3^4 \cdot 23^2 & (2, 6, -12, -1) & 2^2 \cdot 3^5 \cdot 23 & (5, 6, 15, 8) & -2^3 \cdot 3^5 \cdot 23 \\
(1,1,2,1) & -23 & (2, 6, 6, 5) & -2^2 \cdot 3^5  & (5, 9, 12, 10) & -2^2 \cdot 3^5 \cdot 23 \\
(1,1,8,6) & -2^2 \cdot 23^2 & (2, 6, 6, 25) & -2^2 \cdot 3^3 \cdot 23^2  & (5, 12, 18, 20) & -2^4 \cdot 3^5 \cdot 23 \\
(1, 3, -9, -4) & 3^5 \cdot 23 & (2, 6, 27, 117) & -2^3 \cdot 3^5 \cdot 23^2 & (5, 18, 30, 46) & -2^2 \cdot 3^5 \cdot 23^2 \\
(1,3,-6,-4) & 2^2 \cdot 3^3 \cdot 23 & (2, 9, -6, -4) & 2^2 \cdot 3^5 \cdot 23 & (5,24,-3,26) & -2^4 \cdot 3^5 \cdot 23^2  \\
(1,3,-3,-2) & 3^3\cdot 23 & (2, 9, 0, -4) & 2^4 \cdot 3^3 \cdot 23 & (6, 3, 12, -7) & -2^3 \cdot 3^3 \cdot 23^2 \\
(1,3,-6,-2) & 2^3 \cdot 3^5 & (2,9,48,185) & -2^4 \cdot 3^5 \cdot 23^2 & (6, 3, 12, 16) & -2^4 \cdot 3^3 \cdot 23^2 \\
(1,3,3,3) & -2^2 \cdot 3^3 & (2, 12, 24, 85) & -2^2 \cdot 3^5 \cdot 23^2 & (6, 6, 9, 13) & -2^3 \cdot 3^3 \cdot 23^2 \\
(1,3,3,5) & -2^4 \cdot 3^3 & (2, 18, -15, 31) & -2^3 \cdot 3^5 \cdot 23^2 & (6, 9, 12, 23) & -2^3 \cdot 3^4 \cdot 23^2 
\end{array}
$$
$$
\begin{array}{cc|cc|cc} 
(\omega_0,\omega_1,\omega_2,\omega_3) & D_F & (\omega_0,\omega_1,\omega_2,\omega_3)  & D_F & (\omega_0,\omega_1,\omega_2,\omega_3)  & D_F \\ \hline
(1,3,3,7) & -2^2 \cdot 3^5 & (3, 0, 3, 2) & -2^4 \cdot 3^4 & (6, 18, 18, 29) & -2^2 \cdot 3^5 \cdot 23^2 \\
(1,3,3,13) & -2^4 \cdot 3^5 & (3, 4, 12, 12) & -2^4 \cdot 3 \cdot 23^2 & (7, 6, 9, 4) & -2^3 \cdot 3^4 \cdot 23 \\
(1, 3, 18, 50) & -2^3 \cdot 3^5 \cdot 23 & (3, 6, 4, 6) & -2^2 \cdot 3 \cdot 23^2  & (7, 15, 3, 17) & -2^2 \cdot 3^5 \cdot 23^2 \\
( 1, 6, -24, -4) & 2^4 \cdot 3^5 \cdot 23 & (3, 6, 9, 8) & -2^3 \cdot 3^3 \cdot 23 & (8, 9, 12, 13) & -2^2 \cdot 3^4 \cdot 23^2 \\
(1, 6, 3, 32) & -2^3 \cdot 3^5 \cdot 23 & (3, 9, 9, 7) & -2^4 \cdot 3^5 & (8, 15, 18, 21) & -2^3 \cdot 3^4 \cdot 23^2 \\
(1, 6, 6, 16) & -2^4 \cdot 3^3 \cdot 23 & (3, 9, 9, 49) & -2^2 \cdot 3^5 \cdot 23^2 & (9,9,3,31) & -2^4 \cdot 3^5 \cdot 23^2 \\
(1,6,12,54) & -2^2 \cdot 3^3 \cdot 23^2 & (3,18,36,116) & -2^4 \cdot 3^5 \cdot 23^2 & (10, 6, 15, 1) & -2^3 \cdot 3^3 \cdot 23^2 \\
(1,6,12,100) & -2^4 \cdot 3^3 \cdot 23^2 & (3,27,9,29) & -2^4 \cdot 3^5 \cdot 23^2 & (11, 6, 12, 2) & -2^2 \cdot 3^3 \cdot 23^2 \\
( 1, 9, -12, -16) & 2^4 \cdot 3^5 \cdot 23 & ( 4, 0, -18, -3) & 2^4 \cdot 3^5 \cdot 23 & (11, 15, 15, 17) & -2^2 \cdot 3^5 \cdot 23^2 \\
(1,9,-9,-3) & 2^2 \cdot 3^5 \cdot 23 & (4, 0, 6, 1) & -2^4 \cdot 3^5  & (12,9,36,16) & -2^4 \cdot 3^5 \cdot 23^2 \\
(1,9,27,165) & -2^2 \cdot 3^5 \cdot 23^2 & (4, 2, 8, 3) & -2^4  \cdot 23^2 & (12,36,36,35) & -2^4 \cdot 3^5 \cdot 23^2 \\
(1,9,27,303) & -2^4 \cdot 3^5 \cdot 23^2 & (4, 3, 6, 2) & -2^2 \cdot 3^3 \cdot 23 & (13, 9, 18, 12) & -2^2 \cdot 3^5 \cdot 23^2 \\
(1, 12, 9, 18) & -2^4 \cdot 3^5 \cdot 23 & (4, 3, 12, 10) & -2^3 \cdot 3^5 \cdot 23 & (13, 15, 27, 7) & -2^2 \cdot 3^5 \cdot 23^2 \\
(1,12,12,44) & -2^4 \cdot 3^3 \cdot 23^2 & (4, 3, 18, 13) & -2^3 \cdot 3^3 \cdot 23^2  & (21,9,27,11) & -2^4 \cdot 3^5 \cdot 23^2 \\
( 1, 15, 3, -7) & 2^4 \cdot 3^5 \cdot 23 & (4, 3, 18, 36) & -2^2 \cdot 3^5 \cdot 23^2 & (23,30,36,20) & -2^4 \cdot 3^5 \cdot 23^2  \\
(2, 0, 3, 1) & -2^2 \cdot 3^4 & (4, 4, 9, 1) & -2^4 \cdot 23^2 & (24,27,36,16) & -2^4 \cdot 3^5 \cdot 23^2 \\
(2, 0, 3, 2) & -2^3 \cdot 3^4 & (4, 6, 3, 12) & -2^2 \cdot 3^3 \cdot 23^2 & & \\
\end{array}
$$
\end{landscape}

In each case, we solve the corresponding Thue-Mahler equation specified by Theorem \ref{fisk}. For example, if $D_F = \pm 2^4 \cdot 3^t \cdot 23^2$, with $t \geq 3$, then we actually need only solve the (eight) Thue equations of the shape
$$
F(u,v) = 2^{\delta_1} 3^{\delta_2} 23^{\delta_3}, \; \; \mbox{ where } \; \; \delta_i \in \{ 0, 1 \}.
$$
For all other discriminants, we must treat ``genuine'' Thue-Mahler equations (where at least one of the exponents on the right-hand-side of equation (\ref{TM-eq}) is, {\it a priori}, unconstrained). Details of this computation are available at 
\begin{center}
\url{http://www.nt.math.ubc.ca/BeGhRe/Examples/2-3-23-data}.
\end{center}
In total, we found precisely $730$ solutions to these equations, leading, after twisting, to $3856$ isomorphism classes of $E/\mathbb{Q}$ with good reduction outside $\{ 2, 3, 23 \}$ and trivial rational $2$-torsion.

Once again, to find the curves with nontrivial rational $2$-torsion, we solved $X+Y=Z^2$ in $S$-units $X$ and $Y$, and integers $Z$, where 
 $S=\{2,3, 23 \}$. There are precisely $118$ solutions with $X \geq |Y|$ and  $\gcd (X,Y)$ squarefree (this computation took less than 1 hour). Of these, $55$ have $Z > 0$, with $Z$ largest for the solution coming from  the identity
$$
89424 - 23 = 2^4\cdot 3^5\cdot 23 - 23 = 299^2.
$$
These correspond via twists to elliptic curves of conductor as large as $2^8\cdot 3^2 \cdot 23^2$, a total of $1664$ isomorphism classes. There thus exist a total of $5520$ isomorphism classes (in $3968$ isogeny classes) of elliptic curves $E/\mathbb{Q}$ with good reduction outside $\{ 2, 3, 23 \}$. Note that $432 = 2 \times 6^3$ of these have $j_E=0$.

%MAYBE NOTE ANY OTHER TORSION! Theorem 5.2 of Hadano \cite{??}
%Manuscripta finds $43$ isomorphism classes of curves with nontrivial torsion (all types) and good reduction outside $\{ 3, 5 \}$.

%----------------------------------------------------------------------------------------------------------------------------------------------------
\subsection{Curves with good reduction outside $\{ 2, 3, 5, 7, 11 \}$ : an example of von Kanel and Matschke} \label{gummy-bear}
%----------------------------------------------------------------------------------------------------------------------------------------------------

This is the largest computation carried out along these lines by von Kanel and Matschke \cite{KanMat} (and also a very substantial computation via our approach, 
taking many thousand machine hours on $80$ cores).

As in the preceding example, note that
the curves with models of the shape
$$
E \; \; : \; \; Y^2 = X^3 \pm 2^a 3^b 5^c 7^d 11^e, \; \; 0 \leq a, b, c, d, e \leq 5
$$
are precisely the $E/\mathbb{Q}$  with good reduction outside $\{ 2, 3, 5, 7, 11 \}$ and $j$-invariant $0$. We next proceed  by searching our precomputed list for all irreducible primitive cubic forms of discriminant $2^{\alpha} 3^\beta M$,
where 
$$
\alpha \in \{ 0, 2, 3, 4 \}, \; \; \beta \in \{ 0, 1, 3, 4, 5 \} \; \; \mbox{ and } \; \; M \mid 5^2 \cdot 7^2 \cdot 11^2. 
$$
The imprimitive forms we need consider again correspond to primitive 
forms $F$ with either $\nu_2 (D_F) = 0$ or $\nu_3 (D_F) \in \{ 0, 1 \}$.
We encounter $1796$ irreducible cubic forms, which we tabulate at 
\begin{center}
\url{http://www.nt.math.ubc.ca/BeGhRe/Examples/2-3-5-7-11-data}
\end{center}
where details on the resulting Thue-Mahler computation may also be found. Confirming the results of von Kanel and Matschke \cite{KanMat}, we find that there exist a total of $592192$ isomorphism classes (in $453632$ isogeny classes) of elliptic curves $E/\mathbb{Q}$ with good reduction outside $\{ 2, 3, 5, 7, 11\}$, including $15552=2 \times 6^5$ with $j_E=0$.


%------------------------------------------------------------------------------------
\section{Good reduction outside a single prime} \label{primes}
%------------------------------------------------------------------------------------

For the remainder of this chapter, we will focus our attention on the case of elliptic curves with bad reduction at a single prime, i.e. curves of conductor $p$ or $p^2$, for $p$ prime.
In this case, our approach simplifies considerably and 
rather than being required to solve Thue-Mahler equations, the problem reduces to one of solving \emph{Thue} equations, i.e. equations of the shape $F(x,y)=m$, where $F$ is a form and $m$ is a fixed integer. While, once again, we do not have a detailed computational complexity analysis of either algorithms for solving Thue equations or more general algorithms for solving Thue-Mahler equations, computations to date strongly support the contention that the former is, usually, much, much faster than the latter, particularly if the set of primes $S$ considered for the Thue-Mahler equations is anything other than tiny. Since none of these conductors are divisible by $9$, we may always suppose that $j_E \neq 0$. We note that the data we have produced in these cases totals several terabytes. As a result, we have not yet determined how best to make it publicly available; interested readers should contact the authors for further details.

\subsection{Conductor $N=p$}
Suppose that $E$ is a curve with conductor $N=p$ prime with invariants $c_4$ and $c_6$. From Tables \ref{tab nu2}, \ref{tab nu3} and \ref{tab nup}, we 
necessarily have one of 
\begin{align*}
(\nu_2 (c_4), \nu_2(c_6)) &= (0,0) \mbox{ or } (\geq 4, 3), \mbox{ and } \nu_2 (\Delta_E) = 0,  \; \mbox{ or} \\
(\nu_3 (c_4), \nu_3 (c_6)) &= (0,0) \mbox{ or } (1, \geq 3), \mbox{ and } \nu_3 (\Delta_E) = 0, \; \mbox{ or} \\
(\nu_p(c_4), \nu_p(c_6)) &= (0, 0) \mbox{ and } \nu_p(\Delta_E) \geq 1.
\end{align*}
From this we see that $\mathcal{D}= 1$ or $2$. Theorem \ref{fisk} then implies that there is a cubic form of 
discriminant\index{discriminant} $\pm 4$ or $\pm 4p$, and integers $u, v$, with
$$
F(u,v) = p^{\kappa_p} \mbox{ or } 8 p^{\kappa_p}, \; \; 
c_4 = \mathcal{D}^2 H_F(u,v)  \; \; \mbox{ and } \; \; 
c_6 = - \frac{1}{2} \mathcal{D}^3 G_F (u,v),
$$
for $\mathcal{D} \in \{1,2\}$ and $\kappa_p$ a nonnegative integer. Note that, while the smallest absolute discriminant for an irreducible cubic form in $\mathbb{Z}[x,y]$ is $23$, there do exist reducible cubic forms of discriminants $4$ and $-4$ which we must consider.

Appealing to Th\'eor\`eme~2 of Mestre and Oesterl\'e \cite{MO} (and using \cite{BCDT}), we can actually restrict the choices for $n$ dramatically. In fact, we 
have 3 possibilities --  either  $p \in \{ 11, 17, 19, 37 \}$, or $p=t^2+64$ for some integer $t$, or, in all other cases, $\Delta_E = \pm p$. There are precisely $14$ isomorphism classes of $E/\mathbb{Q}$ with conductor in $\{ 11, 17, 19, 37 \}$; one may consult Cremona \cite{Cre1} for details. If we can write $p=t^2+64$ for an integer $t$ (which we may, without loss of generality, assume to satisfy $t \equiv 1\mod{4}$),  then the ($2$-isogenous) curves defined by 
$$
 y^2 + xy = x^3 + \frac{t-1}{4} \cdot x^2  - x
$$
and
$$
 y^2 + xy = x^3 + \frac{t-1}{4} \cdot x^2 + 4x + t
$$
have  rational points of order $2$ given by $(x,y)=(0,0)$ and $(x,y) = (-t/4,t/8)$, respectively, and discriminants $t^2+64$ and  $-(t^2+64)^2$, respectively. In the 
final case (in which $\Delta_E=\pm p$), we have (using the notation of Section~\ref{forms} and, in particular, appealing to (\ref{term0}) which, in this case yields the equation $1 = \nu_p(\Delta_E) =  \nu_p (D_F) + 2 \kappa_p$)
$$
\alpha_0 =2, \;  \alpha_1 \in \{ 0, 3 \}, \;  \beta_0=\beta_1=0, \;
\kappa_p = 0 \; \; \mbox{ and } \; \;  N_1  \in \{1,p\}.
$$
Theorem~\ref{fisk} thus tells us that to determine the elliptic curves of conductor $p$, we are led to 
to find all binary cubic forms (reducible and  irreducible) $F$ of discriminant $\pm 4$ and $\pm 4p$ and then solve the 
Thue equations
$$
F(x,y) =1 \; \; \mbox{ and } \; \; F(x,y)=8.
$$
Since for any solution $(x,y)$ to the equation $F(x,y)=1$, we have $F(2x,2y)=8$, we may thus restrict our attention to the equation  $F(x,y)=8$ (where we assume that $\gcd(x,y) \mid 2$). 

\subsection{Conductor $N=p^2$}

In case $E$ has conductor $N=p^2$, we have that either $E$ is a either a quadratic twist of a curve of conductor $p$, or we have $\nu_p (\Delta_E) \in \{ 2, 3, 4 \}$.
To see this, note that, via  Table~\ref{tab nup}, $p \mid c_4$, $p \mid c_6$ and 
$\mathcal{D} \mid 2p$, and we may suppose that 
$(\nu_p (c_4(E)),  \nu_p (c_6(E)), \nu_p (\Delta_E) )$ is one of
$$
(\geq 1, 1, 2),   (1, \geq 2, 3),   (\geq 2, 2,4),
(\geq 2, \geq 3, 6),  (2,3,\geq 7),  (\geq 3, 4, 8), 
(3, \geq 5, 9 ), $$
or $(\geq 4, 5, 10).$ In each case with $\nu_p (c_6(E)) \geq 3$, denote by $E_1$ the quadratic twist of $E$ by $(-1)^{(p-1)/2} p$. For curves $E$ with 
$$
(\nu_p (c_4(E)),  \nu_p (c_6(E)), \nu_p (\Delta_E) )=(\geq 2, \geq 3, 6), 
$$
one may verify that $E_1$ has good 
reduction at $p$ and hence conductor $1$, a contradiction. If we have
$$
(\nu_p (c_4(E)),  \nu_p (c_6(E)), \nu_p (\Delta_E) )=(2, 3, \geq 7),
$$
then
$$
(\nu_p (c_4(E_1)),  \nu_p (c_6(E_1)), \nu_p (\Delta_{E_1}) ) = (0, 0, \nu_p (\Delta_E)-6)
$$
and so $E_1$ has conductor $p$. In the remaining cases, where
$$
(\nu_p (c_4(E)),  \nu_p (c_6(E)), \nu_p (\Delta_E) ) \in \{ (\geq 3, 4, 8), (3, \geq 5, 9 ), (\geq 4, 5, 10) \},
$$
we check that
$$
(\nu_p (c_4(E_1)),  \nu_p (c_6(E_1)), \nu_p (\Delta_{E_1}) ) \in \{ (\geq 1, 1, 2),   (1, \geq 2, 3),   (\geq 2, 2,4) \}.
$$

It follows that, in order to determine all isomorphism classes of $E/\mathbb{Q}$ of conductor $p^2$, it suffices to carry out the following program.
\begin{itemize}
\item Find all curves of conductor $p$.
\item Find $E/\mathbb{Q}$ with minimal discriminant\index{discriminant}
$\Delta_E \in \{ \pm p^2, \pm p^3, \pm p^4 \}$, and then
\item consider all appropriate quadratic twists of these curves.
\end{itemize}
The fact that we can essentially restrict attention to $E/\mathbb{Q}$ with minimal discriminant\index{discriminant}
\begin{equation} \label{submarine}
\Delta_E \in \{ \pm p^2, \pm p^3, \pm p^4 \}
\end{equation}
(once we have all curves of conductor $p$) was noted by Edixhoven, de Groot and Top in Lemma~1 of \cite{EGT}. To find the $E$ satisfying (\ref{submarine}), Theorem~\ref{fisk} (with specific appeal to (\ref{term0}))  leads us 
to consider Thue equations of the shape
\begin{equation} \label{fish1}
F(x,y) = 8 \; \mbox{ for  $F$ a form of discriminant }  \pm 4 p^2,
\end{equation}
\begin{equation} \label{fish2}
F(x,y) = 8p \; \mbox{ for  $F$ a form of discriminant }  \pm 4 p
\end{equation}
and
\begin{equation} \label{fish3}
F(x,y) = 8p \; \mbox{ for  $F$ a form of discriminant }  \pm 4 p^2,
\end{equation}
corresponding to $\Delta_E = \pm p^2$, $\pm p^3$ and $\pm p^4$, respectively.



%----------------------------------------------------
\subsection{Reducible forms}
%----------------------------------------------------

To find all elliptic curves $E/\mathbb{Q}$ with conductor $p$ or $p^2$ arising from reducible forms, 
via Theorem~\ref{fisk} we are led to solve equations
\begin{equation} \label{loaf}
 F(x,y)=8 p^n \; \; \mbox{ with } \; \;  n \in \mathbb{Z} \; \; \mbox{ and } \; \; \gcd(x,y) \mid 2,
\end{equation}
where $F$ is a reducible binary cubic form of discriminant\index{discriminant} $\pm 4$, $\pm 4p$ and $\pm 4 p^2$. 
This is an essentially elementary, though rather painful, exercise. Alternatively, we may observe that curves of 
conductor $p$ or $p^2$ arising from reducible cubic forms are exactly those with at least one rational $2$-torsion 
point. We can then use Theorem~I of Hadano \cite{Had} to show that the only such $p$ are $p=7, 17$ and $p=t^2+64$ for 
integer $t$.
In any case, after some rather tedious but straightforward work, we can show that the elliptic curves  of conductor $p$ 
or $p^2$ corresponding to reducible forms, are precisely those given in Table~\ref{tab red p curves} (up to quadratic 
twists by $\pm p$). 

\begin{table}
$$
% \begin{array}{|l|l||l|c|c|} \hline
%   c_4 & c_6 & p & \Delta_E & N_E \\ \hline
%   273 & 4455 & 17 & 17^2 & 17 \\
%  33 & 12015 & 17 & -17^4 & 17 \\
%  p-256 &  -t(p+512) & t^2+64 & -p^2 &  p \\
%  105 & 1323 & 7 & -7^3 & 7^2 \\
%  1785 & 75411 & 7 & 7^3 & 7^2 \\
%  33 & -81 & 17 & 17^3 & 17 \\
%  4353 & 287199 & 17 & 17 & 17 \\
%  p-16 & -t (p+8) & t^2+64 & p &  p \\ \hline
% \end{array}
\begin{array}{|c|c||c|c|c|} \hline
  c_4 & c_6 & p & \Delta_E & N_E \\ \hline
 4353 & 287199 & 17 & 17 & 17 \\
 33 & -81 & 17 & 17 & 17 \\
 t^2+48 & -t (t^2+72) & t^2+64 & t^2+64 &  t^2+64 \\ \hline
  273 & 4455 & 17 & 17^2 & 17 \\
 t^2-192 &  -t(t^2+576) & t^2+64 & -(t^2+64)^2 &  t^2+64 \\
\hline
 1785 & 75411 & 7 & 7^3 & 7^2 \\
 105 & 1323 & 7 & -7^3 & 7^2 \\
 \hline
 33 & 12015 & 17 & -17^4 & 17 \\
\hline
\end{array}
$$
\caption{All curves of conductor $p$ and $p^2$, for $p$ prime,  corresponding to reducible forms (i.e. with nontrivial rational $2$-torsion). Note 
that $t$ is any integer so that $t^2+64$ is prime. For the sake of brevity, we have omitted curves that are quadratic 
twists by $\pm p$ of curves of conductor $p$. }\label{tab red p 
curves}
\end{table}



%----------------------------------------------------
\subsection{Irreducible forms : conductor $p$}
%----------------------------------------------------
A quick search demonstrates that there are no irreducible cubic forms of discriminant $\pm 4$. Consequently if we wish 
to find  elliptic curves of conductor $p$ coming from irreducible cubics in Theorem \ref{fisk}, we need to solve equations of the shape 
$F(x,y)=8$ for all cubic forms of discriminant $\pm 4p$. An almost immediate consequence of this is the following.
\begin{proposition} \label{Setzer-prop}
Let $p > 17$ be prime. If there exists an elliptic curve $E/\mathbb{Q}$ of conductor $p$, then either $p=t^2+64$ for some integer $t$, or there exists an irreducible binary cubic form of discriminant $\pm 4p$.
\end{proposition}

On the other hand, if we denote by $h(K)$  the class number of a number field $K$, classical results of Hasse \cite{Has} imply the following.

\begin{proposition} \label{Setzer-prop2}
Let $p \equiv \pm 1 \mod{8}$ be prime and $\delta \in \{ 0, 1 \}$. If there exists an irreducible cubic form of discriminant $(-1)^\delta 4 p$, then 
$$
h \left( \mathbb{Q} (\sqrt{(-1)^\delta p}) \right) \equiv 0 \mod{3}.
$$
\end{proposition}

Combining Propositions \ref{Setzer-prop} and \ref{Setzer-prop2}, we thus have
\begin{corollary}[Theorem 1 of Setzer \cite{Set}] \label{CorSet}
Let $p \equiv \pm 1 \mod{8}$ be prime. If there exists an elliptic curve $E/\mathbb{Q}$ of conductor $p$, then either $p=t^2+64$ for some integer $t$, or we have
$$
h \left( \mathbb{Q} (\sqrt{p}) \right) \cdot h \left( \mathbb{Q} (\sqrt{-p}) \right) \equiv 0 \mod{3}.
$$
\end{corollary}

We remark that Proposition \ref{Setzer-prop} is actually a rather stronger criterion for guaranteeing the non-existence of elliptic curves of conductor $p$ than Corollary \ref{CorSet}. Indeed, by way of example, we may readily check that there are no irreducible cubic forms of discriminant $\pm 4p$ for 
$$
p \in \{ 23, 31, 199, 239, 257, 367, 439 \},
$$
(and hence no elliptic curves of conductor $p$ for these primes)
while, in each case, we have that
$ 3 \mid h \left( \mathbb{Q} (\sqrt{p}) \right) \cdot h \left( \mathbb{Q} (\sqrt{-p}) \right)$.

%----------------------------------------------------
\subsection{Irreducible forms : conductor $p^2$}
%---------------------------------------------------- 

As noted earlier, to determine all  elliptic curves of conductor $p^2$ for $p$ prime, arising via Theorem \ref{fisk} from irreducible cubics, it suffices to find those 
of conductor $p$  and those of conductor $p^2$ with $\Delta_F = \pm p^2, \pm p^3$ and $\pm p^4$ (and subsequently twist 
them). We explore these cases below.

%--------------------------------------------------------------------
\subsubsection{Elliptic curves of discriminant $\pm p^3$} \label{winky}
%--------------------------------------------------------------------
To find elliptic curves of discriminant $\pm p^3$, we need to solve Thue equations of the shape $F(x,y) = 8p$, where $F$ runs over all  cubic forms of discriminant $\Delta_F = 
\pm 4p$. These forms are already required to compute curves of conductor $p$. Now, we can either proceed directly to 
solve $F(x,y)=8p$ or transform the problem into one of solving a pair of new Thue equations of the shape $G_i(x,y)=8$. 
In practice, we used the former when solving rigorously and the latter when solving heuristically (see 	
Section~\ref{ssec heuristic}).

We now describe this transformation. Let $F(x,y)=ax^3+bx^2y+cxy^2+dy^3$ be a reduced form of discriminant $\pm 4p$. 
Since $p \mid \Delta_F$, we have
\begin{align*}
F(x,y) \equiv a (x-r_0 y)^2 (x-r_1y) \mod{p},
\end{align*}
where we must have that $p \nmid a$, since $F$ is a reduced form (which implies that $1 \leq a < p$). Comparing 
coefficients of $x$ shows that
$$
2 r_0 + r_1  \equiv -b/a \mod{p}, \; \; 
r_0^2 + 2 r_0 r_1 \equiv c/a \mod{p}
$$
and
$$
r_0^2 r_1 \equiv -d/a \mod{p}.
$$
Multiply the first two of these by $a$ and add them to get
\begin{align*}
3 a r_0^2 + 2 b r_0 + c  & \equiv 0 \mod{p}.
\end{align*}
We can solve this for $r_0$ and hence $r_1$:
\begin{align*}
(r_0,r_1) &\equiv (3a)^{-1} \left(-b \pm t, -b \mp 2t \right)  \mod{p},
\end{align*}
where we require that $t$ satisfies $t^2 \equiv b^2 - 3 ac \mod{p}$. Finding square roots modulo $p$ can be done efficiently via the Tonelli-Shanks 
algorithm, for example (see e.g. Shanks \cite{Shanks}), and almost trivially if, say, $p \equiv 3 \mod{4}$. Once we have these $(r_0,r_1$), we can 
readily check which pair satisfies $r_0^2 r_1 \equiv -d/a \mod{p}$.

Now if $F(x,y)=8p$ then we must have either
$$
 x \equiv r_0 y \mod{p} \; \; \mbox{ or } \; \; 
 x \equiv r_1 y \mod{p}.
$$
In either case, write $x=r_i y + p u$, which maps the equation $F(x,y)=8p$ to a pair of equations of the shape
$$
G_i(u,y) = 8,
$$
where
$$
G_i(u,y) = ap^2 u^3 + (3 a p r_i + bp) u^2 y + (3 a r_i^2 + 2 b r_i + c) u y^2 + \frac{1}{p} (a r_i^3 + b r_i^2 + c 
r_i + d) y^3.
$$
Notice that $\Delta_{G_i}= p^2 \Delta_F$. In practice, for our deterministic approach, we will actually solve the 
equation $F(x,y)=8p$ directly. For our heuristic approach (where a substantial increase in the size of the form's 
discriminant\index{discriminant} is not especially problematic), we will reduce to consideration of the equations 
$G_i(x,y)=8$.

\subsubsection{A (conjecturally infinite) family of forms and solutions}
We note that there are families of primes for which we can guarantee that the equation $F(x,y)=8p$ has solutions. By way of
example, define a quartic form $p_{r,s}$ via
\begin{align*}
p_{r,s}=r^4+9 r^2 s^2 + 27 s^4.
\end{align*}
Then for a given $r,s$ and $p=p_{r,s}$ the cubic form 
\begin{align*}
F(x,y) = s x^3 + r x^2 y - 3 s x y^2 -r y^3
\end{align*}
has discriminant $4 p$. Additionally one can readily verify the polynomial identities
$$
F(2r^2/s+6s,-2r) = 8p \; \; \mbox{ and } \; \;  F(6s,-18s^2/r-2r) = 8p.
$$
If we set $s \in \{ 1, 2 \}$ in the first of these, or $r \in \{ 1, 2 \}$ in the second, then we arrive at four one-parameter families 
of forms of discriminant $4p$ for which the equation $F(x,y)=8p$ has a solution, namely:
\begin{align*}
(p,x,y) &= (r^4+9r^2+27,2r^2+6,-2r), (r^4+36 r^2 + 432,r^2+12,-2r), \\
 &\phantom{=+}  (27 s^4+9 s^2+1,6s,-18s^2-2), (27s^4+36s^2+16,6s,-9s^2-4).
\end{align*}
Similarly, if we define 
\begin{align*}
p_{r,s}=r^4-9 r^2 s^2 + 27 s^4
\end{align*}
then the form
\begin{align*}
F(x,y) &= s x^3 + r x^2 y + 3 s x y^2 +r y^3
\end{align*}
has discriminant $-4 p$, and the equation $F(x,y)=8p$ has solutions
\begin{align*}
(x,y) &= (-2r^2/s + 6s, 2r)  \text{ and } (6s,-18s^2/r + 2r)
\end{align*}
and hence we again find  (one parameter) families  of primes corresponding to either $r\in \{ 1,2 \}$ or $s\in \{1,2 \}$:
\begin{align*}
(p,x,y)= & (r^4-9r^2+27,-2r^2+6,2r), (r^4-36 r^2 + 432,-r^2+12,2r), \\
 &\quad (27 s^4-9 s^2+1,6s,-18s^2+2), (27s^4-36s^2+16,6s,-9s^2+4).
\end{align*}
We expect that each of the quartic families described here attains infinitely many prime values, but proving this is 
beyond current technology.

%------------------------------------------------------------------------------------
\subsubsection{Elliptic curves of discriminant $p^2$ and $p^4$} \label{pinky}
%------------------------------------------------------------------------------------
To find elliptic curves of discriminant $p^2$ and $p^4$ via Theorem~\ref{fisk}, we need to 
solve Thue equations $F(x,y)=8$ and $F(x,y)=8p$, respectively, for cubic forms $F$ of discriminant 
$4p^2$. Such forms are quite special and it turns out that they form a $2$-parameter family.

Indeed, in order for there to exist a cubic form of discriminant $4 p^2$, it is necessary and 
sufficient that we are able to write $p=r^2+27 s^2$ for positive integers $r$ and $s$, whereby $F$ 
is equivalent to the form
\begin{align*}
F_{r,s} (x,y) &= s x^3 + r x^2y - 9 s x y^2 - r y^3.
\end{align*}
To see this, note that the existence of an irreducible cubic form of discriminant $4p^2$ for prime $p$ necessarily implies that of a (cyclic) cubic field of discriminant $p^2$ and field index $2$. From Silvester, Spearman and Williams \cite{SSW}, there is a unique such field up to isomorphism, which exists precisely when the prime $p$ can be represented by the quadratic form $r^2+27 s^2$. We conclude as desired upon observing that
$$
D_{F_{r,s}} = 4 \left( r^2+27 s^2 \right)^2.
$$

It remains, then, to solve the Thue equations
$$
F_{r,s} (x,y) =8 \; \; \mbox{ and } \; \;   F_{r,s} (x,y)=8 p.
$$
We can transform the problem of solving the second of these equations to one of solving a related Thue 
equation of the form $G_{r,s}(x,y)=8$. This transformation is quite similar to the one described in 
the previous subsection. 

First note that we may assume that $p \nmid y$, since otherwise, we would require that $p \mid sx$, contradicting the facts that $s < \sqrt{p}$ and $p^2 \nmid F$. Since $p^2 \mid \Delta_F$, it follows that the congruence
\begin{align*}
  su^3 +ru^2 -9su - r &\equiv 0 \mod{p}
\end{align*}
has a unique solution modulo $p$; one may readily check that this satisfies $u \equiv 9s/r \mod{p}$:
\begin{align*}
  su^3 +ru^2 -9su - r &\equiv -r^{-3} \cdot (r^2-27s^2)(r^2+27s^2) \equiv 0 \mod{p}.
\end{align*}
Consequently, we know that $x \equiv uy \mod{p}$. Substituting $x = uy + vp$ into $F$ gives
\begin{align*}
F_{r,s}(uy+vp,y) &= a_0 v^3 + b_0 v^2 y + c_0 v y^2 + d_0 y^3
\end{align*}
so, with a quick renaming of variables, we obtain
\begin{align*}
G_{r,s}(x,y) &= a_0 x^3 + b_0 x^2 y + c_0 x y^2 + d_0 y^3 =8,
\end{align*} 
where
$$
a_0 = s p^2, \; \; b_0 = (3 u s+r) p, \; \; 
c_0 = 3 u^2 s+2 r u - 9s
$$
and $d_0 = (u^3 s+r u^2 - 9 u s - r)/p$. A little algebra confirms that 
\begin{align*}
\Delta_{G_{r,s}} = 4 p^4.
\end{align*}
As noted in the previous subsection, we have solved $F_{r,s}(x,y)=8p$ directly in our deterministic 
approach, while we solved equation $G_{r,s}(x,y)=8$ for our heuristic method.

%------------------------------------------------------------------------------------
\subsubsection{Elliptic curves of discriminant $-p^2$ and $-p^4$} \label{pinky2}
%------------------------------------------------------------------------------------
Elliptic curves of discriminant $-p^2$ and $-p^4$ can be found through Theorem~\ref{fisk} by solving 
the Thue equations $F(x,y)=8$ and $F(x,y)=8p$, respectively, this time  for cubic forms $F$ of 
discriminant $-4p^2$. As in the cases treated in the preceding subsection, these forms can be described as a $2$-parameter family.
Specifically, such forms arise precisely when there exist integers $r$ and $s$ such that
$p = | r^2 - 27 s^2|$, in which case the form $F$ is equivalent to 
 \begin{align*}
F_{r,s}(x,y)&=sx^3+r x^2y + 9s x y^2 + r y^3.
\end{align*}
The primes $p$ for which we can write  $p = | r^2 - 27 s^2|$ are those  with $p \equiv \pm 1 \mod{12}$.
To see this, note first that if $p \equiv 1 \mod{3}$ and $p = | r^2 - 27 s^2|$, then necessarily $p=r^2-27s^2$, so that $p \equiv 1 \mod{4}$, while, if $p \equiv -1 \mod{3}$ and
 $p = | r^2 - 27 s^2|$, then $p=27s^2-r^2$ so that $p \equiv -1 \mod{4}$. It follows that necessarily $p \equiv \pm 1 \mod{12}$. To show that this is sufficient to have $p =|r^2-27s^2|$ for integers $r$ and $s$, we appeal
 to the following.
\begin{proposition} \label{neg-pee2}
If $p \equiv 1 \mod{12}$ is prime, there exist positive integers $r$ and $s$ such that
$$
r^2-27 s^2 = p, \; \mbox{ with } \; r < \frac{3}{2} \sqrt{6p} \; \; \mbox{ and } \; \;  s < \frac{5}{18} \sqrt{6p}.
$$
If $p \equiv -1 \mod{12}$ is prime, there exist positive integers $r$ and $s$ such that
$$
r^2-27 s^2 = -p,  \; \mbox{ with } \; r < \frac{5}{2} \sqrt{2p}  \; \; \mbox{ and } \; \; s < \frac{1}{2} \sqrt{2p}.
$$
\end{proposition}
This result is, in fact, an almost  direct consequence of the following.
\begin{theorem}[Theorem~112 from Nagell \cite{Nag2}]\label{Nag}
If $p \equiv 1 \mod{12}$ is prime, there exist positive integers $u$ and $v$ such that
$$
p = u^2-3 v^2, \; \; u  < \sqrt{3p/2}  \; \; \mbox{ and } \; \; v < \sqrt{p/6}.
$$
If $p \equiv -1 \mod{12}$ is prime, there exist positive integers $u$ and $v$ such that
$$
-p = u^2-3v^2, \; \; u < \sqrt{p/2} \; \; \mbox{ and } \; \;  v < \sqrt{p/2}.
$$
\end{theorem}
\begin{proof}[Proof of Proposition \ref{neg-pee2}]
If $p \equiv \pm 1 \mod{12}$, Theorem  \ref{Nag} guarantees the existence of integers $u$ and $v$ such that $p = |u^2-3v^2|$.
If $3\mid v$ then we set $r=u, s=v/3$. 
Clearly $3\nmid u$, so if $3 \nmid v$ then we have (replacing $v$ by $-v$ is necessary) that $u \equiv v \mod{3}$. If we now 
set $r=2u+3v$ and $s=(2v+u)/3$, then it follows that
\begin{align*}
|r^2 - 27 v^2| &= |(2u+3v)^2 - 3 (2v+u)^2| = |u^2 - 3 v^2|  = p
\end{align*}
and hence either 
$$
|r| \leq 2 \sqrt{3p/2} + 3 \sqrt{p/6} = \frac{3}{2} \sqrt{6 p}  \; \; \mbox{ and } \; \; 
|s| \leq \frac{1}{3} (2 \sqrt{p/6}  + \sqrt{3p/2} ) = \frac{5}{18} \sqrt{6p},
$$
or
$$
|r| \leq 2 \sqrt{p/2} + 3 \sqrt{p/2} = \frac{5}{2} \sqrt{2 p} \; \; \mbox{ and } \; \; 
|s| \leq \frac{1}{3} (2 \sqrt{p/2}  + \sqrt{p/2} ) = \frac{1}{2} \sqrt{2 p}.
$$
\end{proof}

Again, we are able to reduce the problem of solving $F_{r,s}(x,y)=8p$ to that of treating a related 
equation $G_{r,s}(x,y)=8$. As before, note that if $u \equiv -9s/r 
\mod{p}$, then
\begin{align*}
s u^3 + r u^2 + 9 s u + r  \equiv r^{-3} (r^2-27 s^2) (r^2+27s^2) \equiv 0 \mod{p}.
\end{align*}
Again, write $x=r_0y+vp$ so that, after renaming $v$, we have
\begin{align*}
G_{r,s}(x,y) &= a_0 x^3 + b_0 x^2 y + c_0 x y^2 + d_0 y^3 =8,
\end{align*}
where
$$
a_0 = s p^2, \; \; 
b_0 = (3 u s+r) p, \; \;
c_0 = 3 u^2 s+2 r u + 9s
$$
and $d_0 = (u^3 s+r u^2 + 9 u s + r)/p$.

Note that, in contrast to the case where $p =r^2 + 27s^2$, here $p$ is represented by an indefinite quadratic form and so the presence of infinitely many units in $\mathbb{Q}(\sqrt{3})$ implies that a given representation is not unique. 
If, however, we have two solutions to the equation $|r^2-27s^2|=p$, say $(r_1,s_1)$ and $(r_2,s_2)$, then the corresponding forms
$$
s_1x^3+r_1 x^2y + 9s_1 x y^2 + r_1 y^3 \; \mbox{ and } \; s_2x^3+r_2 x^2y + 9s_2 x y^2 + r_2 y^3
$$
may be shown to be $\mbox{GL}_2 ( \mathbb{Z})$-equivalent.

% 
% As a final comment, we note that if we have two solutions to the equation $|r^2-27s^2|=p$, say $(r_1,s_1)$ and $(r_2,s_2)$, then the corresponding forms
% $$
% s_1x^3+r_1 x^2y + 9s_1 x y^2 + r_1 y^3 \; \mbox{ and } \; s_2x^3+r_2 x^2y + 9s_2 x y^2 + r_2 y^3
% $$
% are readily seen to be $\mbox{GL}_2 ( \mathbb{Z})$-equivalent.

%-----------------------------------------------------------------
 \section{Computational details} \label{compy}
%-----------------------------------------------------------------

As noted earlier, the computations required to generate curves of prime conductor  $p$  (and subsequently conductor $p^2$) fall into a small number of 
distinct parts. 

%--------------------------------------------------------
\subsection{Generating the required forms}
%--------------------------------------------------------

To find the irreducible forms  potentially corresponding to elliptic curves of prime conductor $p \leq X$ for some fixed 
positive real $X$, arguing as in Section \ref{rep}, we tabulated all reduced forms $F(x,y)=ax^3+bx^2y+cxy^2+d$ with 
discriminants in $(0,4X]$ and $[-4X,0)$, separately. As each form was generated, we checked to see if it actually 
satisfied the desired definition of reduction. Of course, this does not only produce forms with discriminant $\pm 4p$ -- 
as each form was produced, we kept only those whose discriminant was in the appropriate range, and equal 
to $\pm 4p$ for some prime $p$. Checking primality was done using the Miller-Rabin primality test (see \cite{Mil}, 
\cite{Rab}; to make this deterministic for the range we require, we appeal to \cite{SW}). While it is straightforward to 
code the above in computer algebra packages such as \texttt{sage} \cite{Sage}, \texttt{maple} \cite{Maple} or Magma \cite{magma}, we 
instead implemented it in \texttt{c++} for speed. To avoid possible numerical overflows, we used the \texttt{CLN} 
library \cite{Hai} for \texttt{c++}. 


We computed forms of discriminant $\pm 4p$ in two separate runs --- first to $p\leq 10^{12}$ and then a second run to $p 
\leq 2\times 10^{13}$. In the first of these, we constructed all the forms and explicitly saved them to files. 
Constructing all the required positive discriminant forms took approximately 40 days of CPU time on a modern server, and 
about 300 gigabytes of disc space. Thankfully, the computation is easily parallelised and it only took about 1 day of 
real time. We split the jobs by running a manager which distributed $a$-values to the other cores. The output from each 
$a$-value was stored as a tab-delimited text file with one tuple of $p,a,b,c,d$ on each line. Generating all forms of 
negative discriminant took about 3 times longer and required about 900 gigabytes of disc space. The distribution of 
forms is heavily weighted to small values of $a$. To allow us to spread the load across many CPUs we actually split the 
task into 2 parts. We first ran $a \geq 3$, with the master node distributing $a$-values to the other cores. We then ran 
$a=1$ and $2$ with the master node distributing $b$-values to the other cores. The total CPU time was about three times longer 
than for the positive case (there being essentially three times as many forms), but more real-time was required due to 
these complications. Thus generating all forms took less than 1 week of real time but required about 1.2 terabytes of 
disc space. 

These forms were then sorted by discriminant while keeping positive and negative discriminant forms separated. Sorting 
a terabyte of data is a non-trivial task, and in practice we did this by first sorting\footnote{Using the standard unix 
\texttt{sort} command and taking advantage of multiple cores.} the forms for each $a$-value and then splitting them into 
files of discriminants in the ranges $[n \times 10^9, (n+1)\times 10^9)$ for $n\in[0,999]$. Finally, all the files of 
each discriminant range were sorted together. This process for positive and negative discriminant forms took around two days of real 
time. We found $9247369050$ forms of positive discriminant $4p$  and  $27938060315$ of negative discriminant $-4p$, with $p$
bounded by $10^{12}$. Of these, $475831852$ and $828238359$, respectively had $F(x,y)=8$ solvable (by the 
heuristic method described below), leading to $159552514$ and $276339267$ elliptic curves of positive and negative 
discriminant, respectively, with prime conductor up to $10^{12}$.

The second run to $p \leq 2\times 10^{13}$ required a different workflow due to space constraints. Saving all 
forms to disc was simply impractical --- we estimated it to require over 20 terabytes of space! Because of this we 
combined the form-generation code with the heuristic solution method (see below) and kept only those forms $F(x,y)$ 
for which solutions to $F(x,y)=8$ existed. Since only a small fraction of forms (asymptotically likely $0$)  have 
solutions, the disc space required was considerably less. Indeed to store all the required forms took about 250 and 400 
gigabytes for positive and negative forms respectively. This then translated into about 65 and 115 gigabytes of 
positive and negative discriminant curves, respectively, with prime conductor up to $2\times 10^{13}$. This 
second computation took roughly 20 times longer than the first, requiring about 4 months of real-time. This led to 
a final count of $1738595275$ and $3011354026$ (isomorphism classes of) curves of positive and negative discriminant, respectively, with prime 
conductor up to $2\times 10^{13}$.


\subsection{Complete solution of Thue equations : conductor $p$}
For each form encountered, we needed to solve the Thue equation\index{Thue equations}
\begin{align*}
  ax^3+bx^2y+cxy^2+dy^3 &= 8
\end{align*}
in integers $x$ and $y$ with $\gcd(x,y) \in\{ 1, 2 \}$.
We approached this in two distinct ways.

To solve the Thue equation rigorously, we appealed to by now well-known arguments of Tzanakis and de Weger \cite{TW}, based upon lower bounds for linear forms in complex logarithms, together with lattice basis reduction; these are implemented in several computer algebra packages, including  Magma \cite{magma} and Pari/GP \cite{PARI2}. The main computational bottleneck in this approach is typically that of computing the fundamental units in the corresponding cubic fields; for computations $p$ of size up to $10^9$ or so, we encountered no difficulties with any of the Thue equations arising (in particular, the fundamental units occurring can be certified without reliance upon the Generalized Riemann Hypothesis).

We ran this computation in Magma \cite{magma}, using  its 
built-in Thue equation solver. Due to memory consumption issues, we fed the forms into 
Magma in small batches, restarting Magma after each set. We saved the 
output as a tuple 
$$
p,a,b,c,d,n,\{(x_1,y_1),\dots,(x_n,y_n)\}, 
$$
where $p,a,b,c,d$ came 
from the form, $n$ counts the number of solutions of the Thue equation and $(x_i,y_i)$ the 
solutions. These solutions can then be converted into corresponding elliptic curves in minimal form using Theorem \ref{fisk} and standard techniques. 

For positive discriminant, this approach works without issue for $p < 10^{10}$. For forms of negative discriminant $-4p$, however, the fundamental unit $\epsilon_p$ in the associated cubic field can be extremely large (i.e. $\log |\epsilon_p|$ can be roughly of size  $\sqrt{p}$). For this reason, finding all negative discriminant curves with prime conductor exceeding $2 \cdot 10^9$ or so proves to be extremely time-consuming.
Consequently, for large $p$, we turned to a non-exhaustive method, which, 
though it finds solutions to the Thue equation, is not actually guaranteed to find them all.

%----------------------------------------------------------------------------------------
\subsection{Non-exhaustive, heuristic solution of Thue equations}\label{ssec heuristic}
%----------------------------------------------------------------------------------------
If we wish to find all ``small'' solutions to a Thue equation (which, subject to various well-accepted conjectures, might actually prove to be all solutions), there is an obvious and very  computationally efficient approach we can take, based upon the idea that, given any solution to the equation $F(x,y)=m$ for fixed integer $m$, we necessarily either have that $x$ and $y$ are (very) small, relative to $m$, or that $x/y$ is a convergent in the infinite simple continued fraction\index{continued fractions} expansion to a root of the equation $F(x,1)=0$.

Such techniques were developed in detail by Peth\H{o} \cite{Pet1}, \cite{Pet2}; in particular, he provides a precise and computationally efficient distinction  between ``large'' and ``small'' solutions. Following this, for each form $F$ under consideration,
we expanded the roots of $F(x,1)=0$ to high precision, again using the 
\texttt{CLN} library for \texttt{c++}. We then computed the continued fraction expansion for each real root, 
along with its associated convergents. Each convergent $x/y$ was then substituted into $F(x,y)$ and  
checked to see if $F(x,y)=\pm 1, \pm 8$. Replacing $(x,y)$ by one of $(-x,-y), (2x,2y)$ or $(-2x,-2y)$, if necessary, then provided the required solutions of $F(x,y)=8$. The precision was chosen so that we could 
compute convergents $x/y$ with $|x|,|y| \leq 2^{128} \approx 3.4\times10^{38}$. We then looked for solutions of small height using a 
brute force search over a relatively small range of values. 

To ``solve'' $F(x,y)=8$ by this method,  for all forms with discriminant $\pm 4p$ with $p\leq 10^{12}$, took about 1 
week of real time using 80 cores. The resulting solutions files (in which we stored also  forms with no corresponding 
solutions) required about 1.5 terabytes of disc space. Again, the files were split into files of absolute discriminant 
(or more precisely absolute discriminant divided by 4) in the ranges $[n \times 10^9, (n+1)\times 10^9)$ for 
$n\in[0,999]$. For the second computation run to $p \leq 2\times 10^{13}$, we combined the form-generation and 
heuristic-solutions steps, storing only forms which had solutions. This produced about $235$ and $405$ gigabytes of 
data for positive and negative discriminants, respectively. 

%---------------------------------------------
\subsection{Conversion to curves}
%---------------------------------------------
Once one has a tuple $(a,b,c,d,x,y)$, one then computes $G_F(x,y)$ and $H_F(x,y)$, appeals to Theorem \ref{fisk} and 
checks twists. This leaves us with a list of pairs $(c_4,c_6)$ corresponding to elliptic curves. It is now 
straightforward to derive $a_1,a_2$, $a_3,a_4$ and $a_6$ for a corresponding elliptic curve in minimal form (see e.g. 
Cremona \cite{Cre2}).  For each curve, we saved a tuple $p,a_1,a_2,a_3,a_4,a_6,\pm1$ with the last entry being the sign 
of the discriminant of the form used to generate the curve (which coincides with the sign of the discriminant of the 
curve). We then merged the curves with positive and negative discriminants and added the curves with prime conductor 
arising from reducible forms (i.e. of small conductor or for primes of the form $t^2+64$). After sorting by conductor, 
this formed a single file of about $17$ gigabytes for all curves with prime conductor $p < 10^{12}$ and about $180$ 
gigabytes for all curves with conductor $p<2\times10^{13}$.


%------------------------------------------------
\subsection{Conductor $p^2$}
%-----------------------------------------------
The conductor $p^2$ computation was quite similar, but was split further into parts. 
\subsubsection{Twisting conductor $p$}
The vast majority of curves of conductor $p^2$ that we encountered arose as quadratic twists of curves of conductor 
$p$. To compute these, we took all curves with conductor $p \leq 10^{10}$ and calculated the invariants 
$c_4$ and $c_6$. The twisted curve then has corresponding $c$-invariants
$$
  c_4' = p^2 c_4  \; \; \mbox{ and } \; \;  c_6' = (-1)^{(p-1)/2}  p^3 c_6.
$$
The minimal $a$-invariants were then computed as for curves of 
conductor $p$. 

We wrote a simple \texttt{c++} program to read curves of conductor $p$ and then twist 
them, recompute the $a$-invariants and output them as a tuple 
$p^2,a_1,a_2,a_3,a_4,a_6,\pm1$. The resulting code only took a few minutes to process the 
approximately $1.1\times10^7$ curves.

\subsubsection{Solving $F(x,y)=8p$ with $F$ of discriminant $\pm 4p$}
There was no need to retabulate forms for this computation; we reused the positive and negative forms of discriminant 
$\pm 4p$ with $p \leq 10^{10}$ from the conductor-$p$ computations. We subsequently rigorously solved the corresponding 
equations $F(x,y)=8p$ for $p \leq 10^8$. To solve the Thue equation $F(x,y)=8p$ for $10^8 < p \leq 10^{10}$, using the 
non-exhaustive, heuristic method, we first converted the equation to a pair of new Thue equations of the form 
$G_i(u,y)=8$ as described in Section \ref{winky} and then applied  Peth\H{o}'s solution search method (where we searched for solutions to these new equations with 
$|y|$ bounded by $2^{128}$ and $|u|=|(x-r_iy)/p|$ bounded in such way as to guarantee that our original $|x|$  is also bounded by $2^{128}$). 

The solutions were then processed into curves as for the conductor $p$ case 
above, and the resulting curves were twisted by $\pm p$ in order to obtain more 
curves of conductor $p^2$. 

\subsubsection{Solving $F(x,y) \in \{ 8, 8p \}$ with $F$ of  discriminant $\pm 4p^2$}
To find forms of discriminant $4 p^2$ with $p\leq 10^{10}$ we need only check to see which primes are of the form $p = r^2+27s^2$ in the desired range. To 
do so, we simply looped over $r$ and $s$ values and then again checked primality\index{primality testing} using 
Miller-Rabin. As each prime was found, the corresponding $p,r,s$ tuple was converted to a 
form  as in Section \ref{pinky}, and the Thue equations $F(x,y)=8$ and $F(x,y)=8p$ were solved,  using the rigorous
approach for $p < 10^6$ and  the 
non-exhaustive method described previously  for $10^6 < p \leq 10^{10}$. Again, in the latter situation, the equation $F(x,y)=8p$ was converted to a 
new equation $G(x,y)=8$ as described in Section \ref{pinky}. The process for forms of discriminant $-4p^2$ was very similar, 
excepting that more care is required with the range of $r$ and $s$ (appealing to Proposition \ref{neg-pee2}). The non-exhaustive 
method solving both $F(x,y)=8$ and $F(x,y)=8p$ for positive and negative forms took a total of 
approximately 5 days of real time on a smaller server of 20 cores. The rigorous approach, even restricted to prime $p < 10^6$ was much, much slower.

The solutions were then converted to curves as with the previous cases and each resulting 
curve was twisted by $\pm p$ to find other curves of conductor $p^2$. 

 %------------------------------------------
\section{Data} \label{data}
%-----------------------------------------

%------------------------------------------
\subsection{Previous work}
%------------------------------------------
The principal prior work on computing table of elliptic curves of prime conductor was carried out in two lengthy computations, by Brumer and McGuinness \cite{BrMc} in the late 1980s and by Stein and Watkins \cite{StWa} slightly more than ten years later. 
For the first of these computations, the authors  fixed the $a_1, a_2$ and $a_3$ invariants ($12$ possibilities) and looped over $a_4$ and $a_6$ chosen to make the corresponding discriminant small.
By this approach, they were able to find $311243$ curves of prime conductor $p < 10^8$ (representing approximately $99.6 \%$ of such curves). In the latter case, the authors looped instead over $c_4$ and $c_6$, subject to (necessary) local conditions. They obtained a large collection of elliptic curves of general conductor to $10^8$, and $11378912$ of those with prime conductor to $10^{10}$ (which we estimate to be slightly in excess of $99.8 \%$ of such curves).

%--------------------------------------------------
\subsection{Counts : conductor $p$}
%--------------------------------------------------

By way of comparison, we found the following numbers of isomorphism classes of
elliptic curves\index{elliptic curves}  over $\mathbb{Q}$ with prime conductor $p \leq X$:
\begin{landscape}
 $$
 \begin{array}{|c|c|c|c|c|c|c|} \hline
 X    & \Delta_E > 0  &   \Delta_E < 0 & \mbox{Ratio}^2 &  \mbox{Total}  & \mbox{Expected} & \mbox{Total / Expected} \\ 
\hline
10^3 &  33 &  51 &  2.3884 &  84 &  68 & 1.2353 \\
10^4 &  129 &  228 &  3.1239 &  357 &  321 & 1.1122 \\
10^5 &  624 &  1116 &  3.1986 &  1740 &  1669 & 1.0425 \\
10^6 &  3388 &  5912 &  3.0450 &  9300 &  9223 & 1.0084 \\
10^7 &  19605 &  34006 &  3.0087 &  53611 &  52916 & 1.0131\\
10^8 &  114452 &  198041 &  2.9941 &  312493 &  311587 & 1.0029\\
10^9 &  685278 &  1187686 &  3.0038 &  1872964 &  1869757 & 1.0017 \\
2\times10^9 &  1178204 &  2040736 &  3.0001 &  3218940 &  3216245 & 1.0008\\
\hline
10^{10} &  4171055 &  7226982 &  3.0021 &  11398037 &  11383665 & 1.0013\\
10^{11} &  25661634 &  44466339 &  3.0026 &  70127973 &  70107401 & 1.0003\\
10^{12} &  159552514 &  276341397 &  2.9997 &  435893911 &  435810488 & 1.0002\\
10^{13} &  999385394 &  1731017588 &  3.0001 &  2730402982 &  2730189484 & 1.00008 \\
2\times10^{13} &  1738595275 &  3011354026 &  3.0000 &  4749949301 &  4749609116 & 1.00007 \\
\hline
\end{array}
$$
\end{landscape}

The data above the line is rigorous; for positive discriminant, we actually have a rigorous 
result to $10^{10}$. For the positive forms this took about one week of real time using 80 cores. 
Unfortunately, the negative discriminant forms took significantly longer, roughly 2 months of real time using 80 cores. 
Heuristics given by Brumer and McGuinness \cite{BrMc} suggest that the number of elliptic curves of negative 
discriminant of absolute discriminant up to $X$ should be asymptotically $\sqrt{3}$ times as many as those of positive 
discriminant in the same range -- here we report the square of this ratio in the given ranges. The aforementioned 
heuristic count of Brumer and McGuinness suggests that the 
expected number of $E$ with prime $N_E \leq X$ 
should be 
  $$
  \frac{\sqrt{3}}{12} \, \left( \int_{1}^{\infty} \frac{1}{\sqrt{u^3-1}} du + \int_{-1}^{\infty} \frac{1}{\sqrt{u^3+1}} du  \right) \mbox{ Li} (X^{5/6}),
  $$
  which we list (after rounding) in the table above. It should not be surprising that this ``expected'' number of curves appears to slightly undercount the actual number, since it does not take into account the roughly $\sqrt{X}/\log X$ curves of conductor $p=n^2+64$ and discriminant $-p^2$ (counting only curves of discriminant $\pm p$).

\subsection{Counts : conductor $p^2$}
To compile the final list of curves of conductor $p^2$, we combined the five lists of 
curves: twists of curves of conductor $p$, curves from forms of discriminant $+4p$ and 
$-4p$, and curves from discriminant $+4p^2$ and $-4p^2$. The list was then sorted and any 
duplicates removed. The resulting list is approximately one gigabyte in size. The counts of curves 
are as follows; here we list numbers of isomorphism classes of curves of conductor $p^2$ for $p$ prime with $p \leq X$.
\vskip2ex

\begin{center}
 \begin{tabular}{|c|c|c|c|c|}
\hline
$X$ & $\Delta_E>0$ &$\Delta_E<0$ & Total &Ratio$^2$ \\
\hline
$10^3$  &  53  &  93  &  146  &  3.0790 \\
$10^4$  &  191  &  322  &  513  &  2.8421 \\
$10^5$  &  764  &  1304  &  2068  &  2.9132 \\
$10^6$  &  3764  &  6356  &  10120  &  2.8515 \\
$10^7$  &  20539  &  35096  &  55635  &  2.9198 \\
$10^8$  &  116894  &  200799  &  317693  &  2.9508 \\
$10^9$  &  691806  &  1195262  &  1887068  &  2.9851\\
$10^{10}$  &  4189445  &  7247980  &  11437425  &  2.9931\\

\hline
 \end{tabular}
\end{center}

\vskip2ex
Subsequently we decided that we should recompute the discriminants of these curves as a sanity check,  by reading the curves into \texttt{sage} and using its built-in 
elliptic curve \index{elliptic curves} routines to compute and then factor the discriminant. This took about one day 
on a single core.

The only curves of genuine interest are those that do not arise from twisting, i.e. those of discriminant $\pm p^2$, $\pm p^3$ and $\pm p^4$. In the last of these categories, we found only $5$ curves, of conductors $11^2$, $43^2$, $431^2$, $433^2$ and $33013^2$. The first four of these were noted by 
Edixhoven, de Groot and Top \cite{EGT} (and are of small enough conductor to now appear in Cremona's tables). The fifth, satisfying
$$
(a_1,a_2,a_3,a_4,a_6)=(1, -1, 1, -1294206576, 17920963598714),
$$
has discriminant $33013^4$. For discriminants $\pm p^2$ and $\pm p^3$, we found the following numbers of curves, for conductors $p^2$ with $p \leq X$ :
\vskip2ex

\begin{center}
 \begin{tabular}{|c|c|c|c|c|}
\hline
$X$ & $\Delta_E=-p^2$ & $\Delta_E=p^2$ & $\Delta_E=-p^3$ & $\Delta_E=p^3$  \\
\hline
$10^3$ & 12 & 4 & 7 & 4 \\
$10^4$ & 36 & 24 & 9 & 5 \\
$10^5$ & 80 & 58 & 12 & 9 \\
$10^6$ & 203 & 170 & 17 & 15 \\
$10^7$ & 519 & 441 & 24 & 23 \\
$10^8$ & 1345 & 1182 & 32 & 36 \\
$10^9$ & 3738 & 3203 & 48 & 58 \\
$10^{10}$ & 10437 & 9106 & 60 & 86 \\
\hline
 \end{tabular}
\end{center}

\vskip2ex
It is perhaps worth observing that the majority of these curves arise from, in the case of discriminant $\pm p^2$, forms with, in the notation of Sections \ref{pinky} and \ref{pinky2}, either $r$ or $s$ in $\{1, 8 \}$. Similarly, for $\Delta_E=\pm p^3$, most of the curves we found come from forms in the eight one-parameter families described in Section \ref{winky}. We are unaware of a heuristic predicting the number of curves of conductor $p^2$ up to $X$ that do not arise from twisting curves of conductor $p$.

%----------------------------------------
\subsection{Thue equations}
%----------------------------------------
 
  
It is noteworthy that all solutions we encountered to the Thue equations\index{Thue equations} $F(x,y)=8$ and $F(x,y)=8p$ under consideration satisfied $|x|, |y| < 2^{30}$. The ``largest'' such solution corresponded to the equation
$$
355 x^3 + 293 x^2 y -1310 x y^2-292 y^3=8,
$$
where we have
$$
(x,y)=(188455233,-82526573).
$$
This leads to the elliptic curve of conductor
$948762329069$,
$$
E \; \; : \; \; y^2+xy+y=x^2-2x^2+a_4x+a_6,
$$
with
$$
a_4=-1197791024934480813341
$$
and 
$$
a_6=15955840837175565243579564368641.
$$
Note that this curve does not actually correspond to a particularly impressive $abc$ or Hall conjecture (see  Section \ref{spartan} for the definition of this term) example.

In the following table, we collect data on the number of $\mbox{GL}_2 ( \mathbb{Z})$-equivalence classes of irreducible binary cubic forms of discriminant $4p$ or $-4p$ for $p$ in $[0,X]$, denoted $P_3(0,X)$ and $P_3(-X,0)$, respectively. We also provide counts for those forms where the corresponding equation $F(x,y)=8$ has at least one integer solution, denoted $P_3^*(0,X)$ and 
$P_3^*(-X,0)$ for positive and negative discriminant forms, respectively.

 $$
 \begin{array}{|c|c|c|c|c|} \hline
 X    & P_3(0,X)  & P_3^*(0,X)    &  P_3(-X,0) &  P_3^*(-X,0) \\ \hline
10^3   & 23  & 22  &   78 & 61    \\
10^4 & 204  &  163  &  740  & 453 \\
10^5 &  1851   &  1159 & 6104  & 2641 \\
10^6 & 16333  & 7668 &  53202  &  16079 \\
10^7 &  147653  & 49866 &  466601  & 97074  \\
10^8 & 1330934 & 314722 & 4126541 & 582792 \\
10^9 &  12050910 & 1966105 & 36979557 & 3530820 \\
2 \times 10^9 &  23418535 & 3408656 & 71676647  & 6080245 \\  \hline 
10^{10} &  109730653 & 12229663 & 334260481 & 21576585 \\    
10^{11} & 1004607003 & 76122366 & 3045402451 & 133115651 \\
10^{12} &  9247369050 & 475831852  & 27938060315  &   828238359 \\ \hline
\end{array}
$$
Due to space limitations we did not compute these statistics in the second large computational run.


\vskip2ex
Our expectation is that the number of forms for which the equation $F(x,y)=8$ has solutions with absolute discriminant up to $X$ is  $o(X)$ (i.e. this occurs for essentially ``zero'' percent of forms; a first step in proving something is this direction can be found in recent work of Akhtari and Bhargava \cite{AkBh}).


%----------------------------------------------------------------------------
\subsection{Elliptic curves with the same prime conductor}
%----------------------------------------------------------------------------

One might ask how many isomorphism classes of curves of a given prime conductor can occur. If one accepts recent 
heuristics that predict that the Mordell-Weil rank of $E/\mathbb{Q}$ is absolutely bounded (see e.g. \cite{PPVW} and \cite{Watetal}), then this number should also be so bounded. As noted by Brumer and Silverman \cite{BrSi}, 
there are $13$ curves of conductor $61263451$. Up to $p < 10^{12}$, the largest number we encountered was for $ p=530956036043$,
with $20$ isogeny classes, corresponding to $(a_1,a_2,a_3,a_4,a_6)$ as follows :
 $$
 \begin{array}{l}
 \left( 0, -1, 1, -1003, 37465 \right), \left( 0, -1, 1, -1775, 45957 \right), \\
 \left( 0, -1, 1, -38939, 2970729 \right), \left( 0, -1, 1, -659, -35439 \right), \\
 \left(0, -1, 1, 2011, 4311\right), \left(0, -2, 1, -27597, -1746656\right),\\
\left( 0, -2, 1, 57, 35020\right),  \left(1, -1, 0, -13337473,18751485796\right),\\
\left(0, 0, 1, -13921, 633170\right), \left( 0, 0, 1, -30292,-2029574\right),\\
\left(0, 0, 1, -6721, -214958\right), \left(0, 0, 1, -845710, -299350726\right), \\
\left( 0, 0, 1, -86411851, 309177638530\right),  \left(0,0, 1, -10717, 428466\right), \\
\left(1, -1, 0, -5632177, 5146137924\right), \left( 1, -1, 0, 878,33379\right),  \\
 \left(1, -1, 1, 1080, 32014\right), \left(1, -2, 1, -8117, -278943\right), \\
\left(1, -3, 0, -2879, 71732\right), \left(1, -3, 0, -30415, -2014316\right). \\
\end{array}
$$
% Of these $20$ curves, $2$ have rank $3$, $3$ have rank $2$, $9$ have rank $1$ and $6$ have rank $0$.
  All have discriminant $-p$.
 %The class group of $\mathbb{Q}(\sqrt{3 \cdot 530956036043})$ is isomorphic to
%$$
%\mathbb{Z}/3 \mathbb{Z} \oplus \mathbb{Z}/3 \mathbb{Z} \oplus \mathbb{Z}/3 \mathbb{Z},
%$$
%which, via a classical result of Hasse \cite{Has}, explains the existence of a large number of cubic forms of discriminant $-4p$.
 Elkies \cite{Elk1} found examples of rather larger conductor with more curves, including
 $21$ classes for $p=14425386253757$ and discriminant $p$, and $24$ classes for $p=998820191314747$ and discriminant $-p$. Our computations confirm, with high likelihood, that, for $p < 2 \times 10^{13}$, the number of isomorphism classes of elliptic curves of conductor a fixed prime $p$ is at most $21$.
 
 
 %-----------------------------------------------------
 \subsection{Rank and discriminant records}
 %-----------------------------------------------------
 
In the following table, we list the smallest prime conductor with a given Mordell-Weil rank\index{Mordell-Weil rank}. These were computed by running through our data, using Rubinstein's upper bounds for analytic ranks (as implemented in Sage) to search for candidate curves of ``large'' rank which were then checked using mwrank \cite{mwrank}.\index{rank records}
The last entry corresponds to a curve of rank $6$ with minimal positive prime  discriminant; we have not yet ruled out the existence of a rank $6$ curve with smaller absolute (negative) discriminant.

 $$
 \begin{array}{|c|c|c|c|} \hline
 N    & (a_1,a_2,a_3,a_4,a_6)  &  \mbox{sign}(\Delta_E) &  rk(E(\mathbb{Q})   \\ \hline
37  &   (0, 0, 1, -1, 0) & + & 1 \\
389 & (0, 1, 1, -2, 0) & + & 2 \\
5077 & (0, 0, 1, -7, 6) & + & 3  \\
501029 & (0, 1, 1, -72, 210)& + & 4  \\
19047851 & ( 0, 0, 1, -79, 342) & - & 5  \\
6756532597 & (0, 0, 1, -547, -2934) & + & 6 \\  \hline
\end{array}
$$
 \vskip1.3ex
 It is perhaps noteworthy that the curve listed here of rank $6$ has the smallest known minimal discriminant for such a curve (see Table 4 of Elkies and Watkins \cite{ElWa}).
 
If we are interested in similar records over all curves, including composite conductors, we have

$$
 \begin{array}{|c|c|c|c|} \hline
 N    & (a_1,a_2,a_3,a_4,a_6)  &  \mbox{sign}(\Delta_E) &  rk(E(\mathbb{Q})   \\ \hline
37  &   (0, 0, 1, -1, 0) & + & 1 \\
389 & (0, 1, 1, -2, 0] & + & 2 \\
5077 & (0, 0, 1, -7, 6) & + & 3  \\
234446 & (1, -1, 0, -79, 289) & + & 4  \\ \hline
19047851 & (0, 0, 1, -79, 342) & - & 5  \\
5187563742 & (1, 1, 0, -2582, 48720) & + & 6 \\  
382623908456 & (0,0,0,-10012,346900) & + & 7 \\ \hline
\end{array}
$$

\vskip2ex
Here, the curves listed above the line are proven to be those of smallest conductor with the given rank. Those listed below the line have the smallest known conductor for the corresponding rank. It is our belief that the techniques of this chapter should enable one to determine whether the  curve listed here of rank $5$ has the smallest conductor of any curve with this property.


%---------------------------------------------------
\section{Completeness of our data} \label{spartan}
%--------------------------------------------------

As a final result, we will present something that might, optimistically, be viewed as evidence that our ``heuristic'' approach, in practice, enables us to actually find all elliptic curves of prime conductor $p < 2 \times 10^{13}$.

A conjecture of Hall, admittedly one that without modification is widely disbelieved at present,  is that if $x$ and $y$ are integers for which $x^3-y^2$ is nonzero, then the {\it Hall ratio} 
$$
\mathcal{H}_{x,y}=\frac{|x|^{1/2}}{|x^3-y^2|}
$$
is absolutely bounded. The pair $(x,y)$ corresponding to the largest known Hall ratio comes from the identity
$$
5853886516781223^3 - 447884928428402042307918^2 = 1641843,
$$
noted by Elkies \cite{Elk2}, with $\mathcal{H}_{x,y} > 46.6$. All other examples known currently have $\mathcal{H}_{x,y} < 7$. We prove the following.

\begin{proposition} \label{total}
 If there is an elliptic curve $E$ with  conductor $p < 2 \times 10^{13}$, corresponding via Theorem \ref{fisk} to a cubic form $F$ and  $u, v \in \mathbb{Z}$, such that 
 $$
 F(u,v)=8 \; \; \mbox{ and  } \; \; \max \{ |u|, |v| \} \geq 2^{128},
 $$
 then
 \begin{equation} \label{trauma}
 \mathcal{H}_{c_4(E),c_6(E)} > 1.5 \times 10^6.
 \end{equation}
\end{proposition}

In other words, if there is an elliptic curve $E$  with  conductor $p < 2 \times 10^{13}$ that we have missed in our heuristic search, then we necessarily have inequality (\ref{trauma}) (and hence a record-setting Hall ratio).

\begin{proof}
The main idea behind our proof is that the roots of the Hessian $H_F(x,1)$ have no particularly good reason to be close to those of the polynomial $F(x,1)$. It follows that, if we have relatively large integers $u$ and $v$ satisfying the Thue equation $F(u,v)=8$ (so that $u/v$ is close to a root of $F(x,1)=0$), our expectation is that not only does $H_F(u,v)$ fail to be small, but, in fact, we should have inequalities of the order of
$$
H_F(u,v) \gg \left( \max\{|u|,|v|\} \right)^2 \; \; \mbox{ and } \; \; G_F(u,v) \gg \left( \max\{|u|,|v|\} \right)^3
$$
(where the Vinogradov symbol hides a possible dependence on $p$).
Since
$$
c_4(E) = \mathcal{D}^2 H_{F} (u,v) \mbox{ and }  c_6(E) = - \frac{1}{2}  \mathcal{D}^3 G_{F} (u,v),
$$
where $\mathcal{D} \in \{ 1, 2 \}$, these would imply that
$$
\mathcal{H}_{c_4(E),c_6(E)} \gg_p \frac{1}{p}  \max\{|u|,|v|\}.
$$
In fact, for forms (and curves) of positive discriminant, we can deduce inequalities of the shape
$$
\mathcal{H}_{c_4(E),c_6(E)} \gg_p p^{-3/4}  \min\{|u|,|v|\} \gg p^{-5/4}  \max\{|u|,|v|\},
$$
where the implicit constants are absolute. For curves of negative discriminant, we have a slightly weaker result :
$$
\mathcal{H}_{c_4(E),c_6(E)} \gg_p p^{-1}  \min\{|u|,|v|\} \gg p^{-3/2}  \max\{|u|,|v|\}.
$$

To make this argument precise, let us write, for concision, $c_4=c_4(E)$ and $c_6=c_6(E)$.  From the identity $|c_4^3-c_6^2|=1728p$, we have a Hall ratio 
$$
\mathcal{H}_{c_4,c_6} = \frac{|c_4|^{1/2}}{1728p} > \frac{|c_4|^{1/2}}{3.456 \times 10^{16}} \geq \frac{|H_F(u,v)|^{1/2}}{3.456 \times 10^{16}}.
$$
Our goal will thus be to obtain a lower bound upon $|H_{F} (u,v)|$ --
we claim that, in fact, $|H_{F} (u,v)| > 3 \times 10^{45}$, whereby this Hall ratio exceeds $1.5 \times 10^6$, as stated. Suppose that we have a cubic form $F$ and integers $u$ and $v$ with
 $D_F = \pm 4 p$ for $p$ prime,
\begin{equation} \label{size}
\max \{ |u|, |v| \} \geq 2^{128} \; \; \mbox{ and } \; \; 2 \times 10^{9} < p < 2 \times 10^{13}.
\end{equation}
Notice that $F(u,0) = \omega_0 u^3 = 8$ and hence (\ref{size}) implies that $v \neq 0$.

Write
$$
F(u,v) = \omega_0 (u -\alpha_1 v) (u-\alpha_2v) (u-\alpha_3v)
$$
and suppose that 
$$
|u - \alpha_1 v| = \min \{ |u-\alpha_iv|, \; i = 1, 2, 3 \}.
$$
We may further assume, without loss of generality, that the form $F$ is reduced. From (\ref{resultant}), we have 
\begin{equation} \label{rezzie}
\omega_0^2 \, \left|  H_F (\alpha_1,1) \, H_F (\alpha_2,1) \, H_F (\alpha_3,1) \right|=  16 \, p^2.
\end{equation}
For future use, we note that  the main result of Mahler \cite{Mah0} implies the inequality
\begin{equation} \label{Mahler}
|\omega_0| \prod_{i=1}^{3} \max \{ 1, |\alpha_i| \} \leq |\omega_0|+|\omega_1|+|\omega_2|+|\omega_3|.
\end{equation}

Let us assume first  that $D_F > 0$, whereby $H_F$ has negative discriminant ($D_{H_F}=-3 D_F$).  Since $F$ is reduced, we have
$$
| \omega_1 \omega_2 -9 \omega_0 \omega_3| \leq \omega_1^2-3\omega_0 \omega_2 \leq \omega_2^2 - 3 \omega_1 \omega_3,
$$
and hence the identity
\begin{equation} \label{bleep}
(\omega_1 \omega_2 -9 \omega_0 \omega_3)^2 - 4 (\omega_1^2-3\omega_0 \omega_2) (\omega_2^2 - 3 \omega_1 \omega_3) = - 3 D_F
\end{equation}
yields the inequalities
\begin{equation} \label{bleep2}
D_F \geq  (\omega_1^2-3\omega_0 \omega_2) (\omega_2^2 - 3 \omega_1 \omega_3) \geq (\omega_1^2-3\omega_0 \omega_2)^2.
\end{equation}

Since (\ref{bleep}) and $D_F > 0$ imply that $\omega_1^2-3\omega_0 \omega_2 \neq 0$,
we may write
\begin{align*}
& \frac{H_F (\alpha_1,1)}{ \omega_1^2-3\omega_0 \omega_2} \\
& \quad=  \left(\alpha_1 -  \frac{9 \omega_0 \omega_3 - \omega_1 \omega_2 + \sqrt{-3 D_F}}{2 (\omega_1^2-3\omega_0 \omega_2)} \right) \left(\alpha_1 -  \frac{9 \omega_0 \omega_3 - \omega_1 \omega_2 - \sqrt{-3 D_F}}{2 (\omega_1^2-3\omega_0 \omega_2)} \right).
\end{align*}
Defining
$$
\Gamma_1 = \alpha_1 -  \frac{9 \omega_0 \omega_3 - \omega_1 \omega_2 }{2 (\omega_1^2-3\omega_0 \omega_2)} \; \; \mbox{ and } \; \; 
\Gamma_2 = \frac{\sqrt{3 D_F}}{2 (\omega_1^2-3\omega_0 \omega_2)},
$$
we have
$$
H_F (\alpha_1,1) = \left( \omega_1^2-3\omega_0 \omega_2 \right) \left( \Gamma_1^2+\Gamma_2^2 \right)
$$
and so
\begin{equation} \label{pos-case}
|H_F(\alpha_1,1)| > \frac{3 D_F}{4 (\omega_1^2-3\omega_0 \omega_2)}.
\end{equation}

Since $\alpha_1$ is ``close'' to $u/v$, it follows that the same is true for $H_F(\alpha_1,1)$ and $H_F(u/v,1) = v^{-2} H_F(u,v)$. To make this precise, note that, 
via the Mean Value Theorem,
\begin{equation} \label{general}
\left| H_F (\alpha_1,1) - H_F (u/v,1) \right| = \left| 2 ( \omega_1^2-3\omega_0 \omega_2) y + \omega_1 \omega_2 -9 \omega_0 \omega_3 \right|
\left| \alpha_1 - \frac{u}{v} \right|,
\end{equation}
for some $y$ lying between $\alpha_1$ and $u/v$. We thus have
\begin{equation} \label{wobbly}
\left| H_F (\alpha_1,1) - H_F (u/v,1) \right| \leq  ( \omega_1^2-3\omega_0 \omega_2) \left( 2 \left( |\alpha_1| + \left| \alpha_1 - \frac{u}{v} \right| \right) +1 \right) \left| \alpha_1 - \frac{u}{v} \right|.
\end{equation}

To derive an upper bound upon $\left| \alpha_1 - \frac{u}{v} \right|$, we can argue as in the proof of Theorem 2 of Peth\H{o} \cite{Pet2} to obtain the inequality
\begin{equation} \label{tombstone}
\left| \alpha_1 - \frac{u}{v} \right| \leq 2^{7/3} D_F^{-1/6} v^{-2}.
\end{equation}
Since $|v| \geq 1$ and $D_F =4p > 8 \times 10^9$, we thus have that 
\begin{equation} \label{rince}
\left| \alpha_1 - \frac{u}{v} \right| < 0.12.
\end{equation}

We may suppose that $F$ is reduced,
$$
|\omega_0| \leq \frac{2D_F^{1/4}}{3 \sqrt{3}} \; \mbox{ and } \; |\omega_1| \leq  \frac{3\omega_0}{2}+  \left( \sqrt{D_F}-\frac{27\omega_0^2}{4} \right)^{1/2}
< \left( 1 + \frac{1}{\sqrt{3}} \right) D_F^{1/4}.
$$
From Proposition 5.5 of Belabas and Cohen \cite{BeCo}, 
$$
|\omega_2| \leq \left( \frac{35 + 13 \sqrt{13}}{216} \right)^{1/3} D_F^{1/3} \; \mbox{ and } \;
|\omega_3| \leq \frac{4}{27} D_F^{1/2},
$$
whence, after a little computation, we find that
$$
|\omega_0|+|\omega_1|+|\omega_2|+|\omega_3| < D_F^{1/2} = 2 p^{1/2}.
$$
From (\ref{Mahler}), it follows that
$$
|\alpha_1| \leq |\omega_0|+|\omega_1|+|\omega_2|+|\omega_3| < 2 p^{1/2},
$$
whereby inequalities (\ref{rince}) and (\ref{size}) thus yield 
$$
|u/v| < 2 p^{1/2} + 0.12 < 2^{23.1},
$$
and so, again appealing to (\ref{size}), $\min \{ |u|, |v| \} > 2^{104}$. Returning to inequality (\ref{wobbly}), we find that, after applying (\ref{bleep2}),
$$
\left| H_F (\alpha_1,1) - H_F (u/v,1) \right| \leq 2 p^{1/2}  \left( 4 p^{1/2} + 1.24 \right) 2^{7/3} (2 p)^{-1/6} v^{-2}.
$$
From $p < 2 \times 10^{13}$ and $|v| > 2^{104}$, it follows that
$$
\left| H_F (\alpha_1,1) - H_F (u/v,1) \right| < 10^{-50}.
$$
Combining this with (\ref{bleep2}) and (\ref{pos-case}) yields the inequality
$$
\left| H_F(u/v,1) \right| >  \frac{2 p}{|\omega_1^2-3\omega_0 \omega_2|},
$$
whence
$$
 |H_F(u,v)| = v^2 \left| H_F(u/v,1) \right| > \frac{2 v^2 p}{|\omega_1^2-3\omega_0 \omega_2|} \geq v^2 \sqrt{p},
$$
where the last inequality follows from  (\ref{bleep2}).
From  (\ref{size}) and the fact that $|v| > 2^{104}$, we conclude that
$$
 |H_F(u,v)|> 10^{67}.
$$

Next, suppose that $F$ has negative discriminant, so that $H_F$ has positive discriminant $D_{H_F}=-3 D_F$.  
If $\omega_1^2-3\omega_0 \omega_2=0$, then, from (\ref{bleep}), we have that
$$
3 p = - (\omega_1^2-3\omega_0 \omega_2) (\omega_2^2 - 3 \omega_1 \omega_3),
$$
which implies that
$$
\max \left\{ |\omega_1^2-3\omega_0 \omega_2|, |\omega_2^2 - 3 \omega_1 \omega_3| \right\} \geq p.
$$
On the other hand, from Lemma 6.4 of Belabas and Cohen \cite{BeCo}, we have
\begin{equation} \label{iota}
\arraycolsep=1.8pt\def\arraystretch{1.8}
\begin{array}{cc}
|\omega_0| \leq \frac{2^{3/2} p^{1/4}}{3^{3/4}}, \; 
|\omega_1| \leq \frac{2^{3/2} p^{1/4}}{3^{1/4}}, \; 
\max \{ |\omega_0 \omega_2^3|, |\omega_1^3 \omega_3| \} \leq \frac{(11+5 \sqrt{5}) p}{2}, \\
|\omega_1 \omega_2| \leq \frac{8 p^{1/2}}{3^{1/2}}
 \; \mbox{ and } \;
|\omega_0 \omega_3| \leq \frac{2 p^{1/2}}{3^{1/2}}, \\
\end{array}
\end{equation}
whereby a short calculation, together with the fact that $p > 2 \times 10^9$, yields a contradiction.
We may thus suppose that $\omega_1^2-3\omega_0 \omega_2 \neq 0$. We have
$$
H_F(\alpha_i,1) =  (\omega_1^2-3\omega_0 \omega_2)  \left(\alpha_i - \beta_1 \right) \left( \alpha_i - \beta_2 \right),
$$
where
$$
\beta_j = \frac{9 \omega_0 \omega_3 - \omega_1 \omega_2 + (-1)^j \sqrt{12 p}}{2 (\omega_1^2-3\omega_0 \omega_2)} \; \; \mbox{ for } \; \; j \in \{ 1, 2 \}.
$$
It follows that
$$
|\beta_j| \leq |\omega_1^2-3\omega_0 \omega_2|^{-1} 44 \cdot 3^{-1/2} p^{1/2}
$$
and, again from (\ref{Mahler}),
$$
|\omega_0 \alpha_i| \leq |\omega_0|+|\omega_1|+|\omega_2|+|\omega_3|,
$$
whereby
$$
|\omega_0 \alpha_i| \leq \frac{2^{3/2} p^{1/4}}{3^{3/4}}+\frac{2^{3/2} p^{1/4}}{3^{1/4}}+\frac{2^{2/3} \left(11+5 \sqrt{5} \right)^{1/3} p^{1/2}}{3^{1/2} |\omega_0|}+\frac{2 p^{1/2}}{3^{1/2}|\omega_0|},
$$
whence we find that
$$
|\alpha_i| \leq \frac{3.4 \, p^{1/4}}{|\omega_0|} + \frac{2.1 \, p^{1/2}}{|\omega_0|^2} < \frac{6.4 \, p^{1/2}}{|\omega_0|^2}.
$$
From (\ref{rezzie}), we thus have 
$$
\left|  H_F (\alpha_1,1)  \right| \geq  \omega_0^{-2} (\omega_1^2-3\omega_0 \omega_2)^{-2} \min \left\{ \frac{\omega_0^2}{3.2}, \frac{|\omega_1^2-3\omega_0 \omega_2|}{12.8} \right\}^{2}.
$$

If $|\omega_1^2-3\omega_0 \omega_2| > 4 \omega_0^2$, it follows that
$$
\left|  H_F (\alpha_1,1)  \right| \geq \frac{\omega_0^2}{10.24 \, (\omega_1^2-3\omega_0 \omega_2)^{2}}
$$
and so
$$
\left|  H_F (\alpha_1,1)  \right| \geq \frac{1}{10.24 \, (2^3 3^{-1/2} p^{1/2}+2^{2/3} 3^{1/2} \left(11+5 \sqrt{5} \right)^{1/3} p^{1/2})^{2}}
$$
which implies that
\begin{equation} \label{foggie}
\left|  H_F (\alpha_1,1)  \right| > \frac{1}{1561 \, p}.
\end{equation}
If, conversely, $|\omega_1^2-3\omega_0 \omega_2| \leq 4 \omega_0^2$, then
$$
\left|  H_F (\alpha_1,1)  \right| \geq \frac{1}{163.84 \, \omega_0^2} > \frac{1}{253 \sqrt{p}}
$$
and hence (\ref{foggie}) holds in either case.

Now if $\alpha_1 \not\in \mathbb{R}$, then, via Mahler \cite{Mah},
$$
\left| \mbox{Im} (\alpha_1) \right|  \geq \frac{1}{18} \left( |\omega_0|+|\omega_1|+|\omega_2|+|\omega_3| \right)^{-2} > \frac{\omega_0^2}{738 \, p},
$$
so that
$$
\left| \alpha_1 - \frac{u}{v} \right|> \frac{\omega_0^2}{738 \, p}
$$
and hence
$$
8 = |\omega_0| |v|^3 \left| \alpha_1 - \frac{u}{v} \right| \left| \alpha_2 - \frac{u}{v} \right| \left| \alpha_3 - \frac{u}{v} \right| > |\omega_0| |v|^3 \left( \frac{\omega_0^2}{738 \, p} \right)^3.
$$
It follows that
$$
|v| < 1476 p < 2.952 \times 10^{16},
$$
via (\ref{size}). Since $\max \{ |u|, |v| \} > 2^{128}$, we thus have
$$
\left| u/v \right| > 1.15 \times 10^{22}.
$$
From
$$
|\alpha_1| < 6.4 p^{1/2} < 6.4 \left( 2 \times 10^{13} \right)^{1/2} <3 \times 10^7, 
$$
we may thus conclude that
$$
\left| \alpha_1 - \frac{u}{v} \right|> 1.14 \times 10^{22}
$$
and so
$$
8 \geq \left( 1.14 \times 10^{22} \right)^3,
$$
an immediate contradiction.

We may thus suppose that $\alpha_1 \in \mathbb{R}$ (so that $\alpha_2, \alpha_3 \not\in \mathbb{R}$).  It follows from Mahler \cite{Mah} that
$$
\left| \alpha_i - \frac{u}{v} \right|> \frac{\omega_0^2}{738 \, p}, \; \; \mbox{ for } \; \; i \in \{ 2, 3 \},
$$
and so
\begin{equation} \label{frenchy}
\left| \alpha_1 - \frac{u}{v} \right| < \frac{8}{|\omega_0| |v|^3} \left( \frac{738 p }{\omega_0^2} \right)^2.
\end{equation}
Appealing to (\ref{size}) and the inequalities $|\alpha_1| < 3 \times 10^7$ and $|v| \geq 1$, we thus have that
$$
|u/v| <1.75 \times 10^{33} + 3 \times 10^7 < 1.76 \times 10^{33},
$$
and so, from $\max \{ |u|, |v| \} > 2^{128}$, $|v| > 1.9 \times 10^5$. Inequality (\ref{frenchy}) thus now implies 
$$
|u/v| < 2.6 \times 10^{17},
$$
whence $|v| > 1.3 \times 10^{21}$. Substituting this a third time into (\ref{frenchy}), 
$$
\left| \alpha_1 - \frac{u}{v} \right| <10^{-30},
$$
so that $|u/v| < 3.1 \times 10^7$ and $|v| > 10^{31}$. One final use of (\ref{frenchy}) thus yields the inequality
$$
\left| \alpha_1 - \frac{u}{v} \right| <10^{-59}.
$$
Appealing to (\ref{size}), (\ref{general}), (\ref{iota}), and the fact that $|\alpha_1| < 3 \times 10^7$, we thus have, after a little work,
$$
\left| H_F (\alpha_1,1) - H_F (u/v,1) \right| < 3.4 \times 10^{-44}.
$$
With (\ref{foggie}), this implies that
$$
\left|  H_F (u/v,1)  \right| > \frac{1}{1562 \, p}
$$
and so
$$
|H_F(u,v)| = v^2 \left| H_F(u/v,1) \right| > \frac{v^2 }{1562 p} > \frac{10^{62}}{3124 \times 10^{13}} > 3 \times 10^{45},
$$
as claimed.
\end{proof}


%We have
%$$
%H_F(u,v)= (\omega_1^2-3\omega_0 \omega_2) u^2 + (\omega_1 \omega_2-9\omega_0 \omega_3) uv + (\omega_2^2-3\omega_1\omega_3) v^2.
%$$

%$$
%\omega_3^2 - \omega_0^2 >  \omega_1 \omega_3-\omega_0 \omega_2, \; \; -(\omega_0-\omega_1)^2-\omega_0 \omega_2< \omega_0 \omega_3-\omega_1 \omega_2 < (\omega_0+\omega_1)^2+\omega_0 \omega_2.
%$$

%---------------------------------------------------
\section{Concluding remarks}
%--------------------------------------------------

Many of the techniques of this chapter can be generalized to potentially treat the problem of determining elliptic curves of a given conductor over a number field $K$. 
In case $K$ is an imaginary quadratic field of class number $1$, then, in fact, such an approach works without any especially new ingredients. 



%---------------------------------------------------------------------------------------------------------------------------------------------%

\endinput

Any text after an \endinput is ignored.
You could put scraps here or things in progress.

%% The following is a directive for TeXShop to indicate the main file
%%!TEX root = diss.tex

\chapter{Towards Efficient Resolution of Thue-Mahler Equations}
\label{ch:EfficientTMSolver}

Let $a$ denote a nonzero integer and let $S=\{p_1,\dotsc,p_v\}$ be a set of rational primes. In this section, we specialize the results of \autoref{ch:AlgorithmsForTM} to the degree $3$ Thue--Mahler equation
\begin{equation} \label{Eq:TM1}
F(X,Y) = c_0 X^3 + c_1 X^{2}Y + c_2XY^2 + c_3Y^3 = a p_1^{Z_1}\cdots p_v^{Z_v},
\end{equation}
where $(X,Y) \in \mathbb{Z}^2$, $\gcd(X,Y)=1$, and $Z_i \geq 0$ for $i = 1, \dots, v$. In particular, to enumerate the set of solutions $\{X,Y, Z_1, \dots, Z_v\}$ of this equation, we follow \autoref{sec:FactorizationTM} to reduce the problem of solving \eqref{Eq:TM1} to solving finitely many so-called ``$S$-unit'' equations
\begin{equation} \label{eq:EfficientSunit}
\lambda = \delta_1 \prod_{i = 1}^r\left( \frac{\varepsilon_i^{(k)}}{\varepsilon_i^{(j)}}\right)^{a_i}\prod_{i = 1}^{\nu} \left( \frac{\gamma_i^{(k)}}{\gamma_i^{(j)}}\right)^{n_i} - 1 = \delta_2 \prod_{i = 1}^{r}\left( \frac{\varepsilon_i^{(i_0)}}{\varepsilon_i^{(j)}}\right)^{a_i} \prod_{i = 1}^{\nu} \left( \frac{\gamma_i^{(i_0)}}{\gamma_i^{(j)}}\right)^{n_i},
\end{equation}
where
\[\delta_1 = \frac{\theta^{(i_0)} - \theta^{(j)}}{\theta^{(i_0)} - \theta^{(k)}}\cdot\frac{\alpha^{(k)}\zeta^{(k)}}{\alpha^{(j)}\zeta^{(j)}}, \quad \delta_2 = \frac{\theta^{(j)} - \theta^{(k)}}{\theta^{(k)} - \theta^{(i_0)}}\cdot \frac{\alpha^{(i_0)}\zeta^{(i_0)}}{\alpha^{(j)}\zeta^{(j)}}\]
are constants. Here, we adopt the notation of \autoref{ch:AlgorithmsForTM} and recall that 
\[g(t) = F(t,1) = t^3 + C_1t^2 + C_2t + C_3,\]
so that $K = \mathbb{Q}(\theta)$ with $g(\theta) = 0$. Recall further that $\zeta$ is a root of unity in $K$, while $\{\eps_1, \dots, \eps_r\}$ denotes the set of fundamental units of $\mathcal{O}_K$. In this case, as $K$ is a degree $3$ extension of $\mathbb{Q}$, we either have $3$ real embeddings of $K$ into $\mathbb{C}$, or there is one real embedding of $K$ into $\mathbb{C}$ and a pair of complex conjugate embeddings of $K$ into $\mathbb{C}$. Thus either $r = 1$ or $r = 2$. 

In this section, we describe new techniques to solve equation~\eqref{eq:EfficientSunit} via global sieves and new geometric ideas. This work is part of an on-going collaborative project with \edit{Rafael, Benjamin, Samir - how do I reference this?}

%---------------------------------------------------------------------------------------------------------------------------------------------%
%---------------------------------------------------------------------------------------------------------------------------------------------%
\section{Decomposition of the Weil height} 

The sieves of \cite{TW3} involving logarithms are of local nature. To obtain a global sieve, we work instead with the global logarithmic Weil height. This height is invariant under conjugation and admits a decomposition into local heights which can be related to complex and $p$-adic logarithms. 

Let $\mathbf{n} = (n_1, \dots, n_{\nu}, a_1, \dots, a_r)$ be a solution to \eqref{eq:EfficientSunit}, let
\[\frac{\lambda}{\delta_2}= \prod_{i = 1}^{r}\left( \frac{\varepsilon_i^{(i_0)}}{\varepsilon_i^{(j)}}\right)^{a_i} \prod_{i = 1}^{\nu} \left( \frac{\gamma_i^{(i_0)}}{\gamma_i^{(j)}}\right)^{n_i}\]
and consider the Weil height of $\frac{\delta_2}{\lambda}$, 
\[\frac{\delta_2}{\lambda}= \prod_{i = 1}^{r}\left( \frac{\varepsilon_i^{(j)}}{\varepsilon_i^{(i_0)}}\right)^{a_i} \prod_{i = 1}^{\nu} \left( \frac{\gamma_i^{(j)}}{\gamma_i^{(i_0)}}\right)^{n_i}.\]
Given the global Weil height of $\delta_2/\lambda$, or all the local heights of $\delta_2/\lambda$, we will construct several ellipsoids `containing' $\mathbf{n}$ such that the volume of the ellipsoids are as small as possible. We begin by computing the height of $\delta_2/\lambda$. 

Let $L$ be the splitting field of $K$, and note that for cubic extensions, the Galois group $\Gal(L/\mathbb{Q})$ is isomorphic to either the alternating group $A_3$ or the symmetric group $S_3$. 

\begin{lemma}\label{lem:cancellation}
Let $\mathfrak{P}$ be a prime ideal of $L$ and let $\mathfrak{P}^{(i_0)} = \sigma_{i_0}(\mathfrak{P})$ and $\mathfrak{P}^{(j)} = \sigma_{j}(\mathfrak{P})$ lying over $\mathfrak{p}^{(i_0)}$, $\mathfrak{p}^{(j)}$ respectively, where $\sigma_{i_0}: L \to L$, $\theta \mapsto \theta^{(i_0)}$ and $\sigma_{j}: L \to L$, $\theta \mapsto \theta^{(j)}$ are two automorphisms of $L$ such that $(i_0,j,k)$ form a subgroup of order $3$. For $i = 1, \dots, \nu$, 
\[\left( \frac{\gamma_i^{(j)}}{\gamma_i^{(i_0)}}\right)\mathcal{O}_L 
	 = \left(\prod_{\mathfrak{P}\mid\mathfrak{p}_1} \frac{\mathfrak{P}^{(j) \ e(\mathfrak{P}^{(j)}\mid\mathfrak{p}_1^{(j)})}}{\mathfrak{P}^{(i_0) \ e(\mathfrak{P}^{(i_0)}\mid\mathfrak{p}^{(i_0)}_1)}}\right)^{a_{1i}} \cdots \left(\prod_{\mathfrak{P}\mid\mathfrak{p}_{\nu}} \frac{\mathfrak{P}^{(j) \ e(\mathfrak{P}^{(j)}\mid\mathfrak{p}^{(j)}_{\nu})}}{\mathfrak{P}^{(i_0) \ e(\mathfrak{P}^{(i_0)}\mid\mathfrak{p}^{(i_0)}_{\nu})}}\right)^{a_{\nu i}}\]
where $\mathfrak{P}^{(j)} \neq \mathfrak{P}^{(i_0)}$ for all $\mathfrak{P}$ lying above $\mathfrak{p}$ in $K$. 
\end{lemma}

\begin{proof}
Since 
\[(\gamma_i)\mathcal{O}_K = \mathfrak{p}_1^{a_{1i}} \cdots \mathfrak{p}_{\nu}^{a_{\nu i}},\]
for $i = 1, \dots, \nu$, where
\[\mathfrak{p}_i\mathcal{O}_L=\prod_{\mathfrak{P}\mid\mathfrak{p}_i} \mathfrak{P}^{e(\mathfrak{P}\mid\mathfrak{p}_i)},\]
it holds that
\[(\gamma_i)\mathcal{O}_L = \left(\prod_{\mathfrak{P}\mid\mathfrak{p}_1} \mathfrak{P}^{e(\mathfrak{P}\mid\mathfrak{p}_1)}\right)^{a_{1i}} \cdots \left(\prod_{\mathfrak{P}\mid\mathfrak{p}_{\nu}} \mathfrak{P}^{e(\mathfrak{P}\mid\mathfrak{p}_{\nu})}\right)^{a_{\nu i}}.\]

Let $\mathfrak{P}^{(i_0)},\mathfrak{P}^{(j)}$ denote the ideal $\mathfrak{P}$ under the automorphisms of $L$
\[\sigma_{i_0}: L \to L, \quad \theta \mapsto \theta^{(i_0)} \quad \text{ and } \quad \sigma_{j}: L \to L, \quad \theta \mapsto \theta^{(j)},\]
respectively. That is, $\mathfrak{P}^{(i_0)} = \sigma_{i_0}(\mathfrak{P})$ and $\mathfrak{P}^{(j)} = \sigma_{j}(\mathfrak{P})$. Then
\[\left( \frac{\gamma_i^{(j)}}{\gamma_i^{(i_0)}}\right)\mathcal{O}_L 
	 = \left(\prod_{\mathfrak{P}\mid\mathfrak{p}_1} \frac{\mathfrak{P}^{(j) \ e(\mathfrak{P}^{(j)}\mid\mathfrak{p}_1^{(j)})}}{\mathfrak{P}^{(i_0) \ e(\mathfrak{P}^{(i_0)}\mid\mathfrak{p}^{(i_0)}_1)}}\right)^{a_{1i}} \cdots \left(\prod_{\mathfrak{P}\mid\mathfrak{p}_{\nu}} \frac{\mathfrak{P}^{(j) \ e(\mathfrak{P}^{(j)}\mid\mathfrak{p}^{(j)}_{\nu})}}{\mathfrak{P}^{(i_0) \ e(\mathfrak{P}^{(i_0)}\mid\mathfrak{p}^{(i_0)}_{\nu})}}\right)^{a_{\nu i}}.\]

Now, to show that $\mathfrak{P}^{(j)} \neq \mathfrak{P}^{(i_0)}$ for all $\mathfrak{P}$ lying above $\mathfrak{p}$ in $K$, we consider the decomposition group of $\mathfrak{P}$, 
\[D(\mathfrak{P}|p) = \{\sigma \in G \ : \ \sigma(\mathfrak{P}) = \mathfrak{P}\}.\]
If we consider all possible decompositions of $\mathfrak{p}$ in $L$, we conclude that $\mathfrak{P}^{(i_0)} \neq \mathfrak{P}^{(j)}$ whenever $D(\mathfrak{P}_i|p)$ does not have cardinality $2$. Thus if we choose $(i_0,j,k)$ so that it forms an order $3$ subgroup of $\Gal(L/\mathbb{Q})$, it cannot coincide with $D(\mathfrak{P}|p)$ and therefore cannot lead to $\mathfrak{P}^{(i_0)} = \mathfrak{P}^{(j)}$. 
\end{proof}

For the remainder of this paper, we assume that $(i_0,j,k)$ are automorphisms of $L$ selected as in Lemma~\ref{lem:cancellation}.

\begin{lemma}\label{lem:ordpz}
Let $\mathfrak{P}$ be a prime ideal of $L$ and let $\mathfrak{P}^{(i_0)} = \sigma_{i_0}(\mathfrak{P})$ and $\mathfrak{P}^{(j)} = \sigma_{j}(\mathfrak{P})$, where $\sigma_{i_0}: L \to L$, $\theta \mapsto \theta^{(i_0)}$ and $\sigma_{j}: L \to L$, $\theta \mapsto \theta^{(j)}$ are two automorphisms of $L$. For 
\[\frac{\delta_2}{\lambda}= \prod_{i = 1}^{r}\left( \frac{\varepsilon_i^{(j)}}{\varepsilon_i^{(i_0)}}\right)^{a_i} \prod_{i = 1}^{\nu} \left( \frac{\gamma_i^{(j)}}{\gamma_i^{(i_0)}}\right)^{n_i},\]
we have
\[\ord_{\mathfrak{P}}\left(\frac{\delta_2}{\lambda}\right)=
\begin{cases}
(u_l - r_l)e(\mathfrak{P}^{(j)}|\mathfrak{p}_l^{(j)})	
	& \textnormal{ if } \mathfrak{P}^{(j)} \mid p_l , \ p_l \in \{p_1,\dots, p_{\nu}\}\\
(r_l - u_l)e(\mathfrak{P}^{(i_0)}|\mathfrak{p}_l^{(i_0)})
	& \textnormal{ if } \mathfrak{P}^{(i_0)}\mid p_l, \ p_l \in \{p_1,\dots, p_{\nu}\}\\
0 	& \textnormal{ otherwise}.
\end{cases}\]
\end{lemma}
\begin{proof}

By Lemma~\ref{lem:cancellation}, we have 
\[\left( \frac{\gamma_i^{(j)}}{\gamma_i^{(i_0)}}\right)\mathcal{O}_L 
	 = \left(\prod_{\mathfrak{P}\mid\mathfrak{p}_1} \frac{\mathfrak{P}^{(j) \ e(\mathfrak{P}^{(j)}\mid\mathfrak{p}_1^{(j)})}}{\mathfrak{P}^{(i_0) \ e(\mathfrak{P}^{(i_0)}\mid\mathfrak{p}^{(i_0)}_1)}}\right)^{a_{1i}} \cdots \left(\prod_{\mathfrak{P}\mid\mathfrak{p}_{\nu}} \frac{\mathfrak{P}^{(j) \ e(\mathfrak{P}^{(j)}\mid\mathfrak{p}^{(j)}_{\nu})}}{\mathfrak{P}^{(i_0) \ e(\mathfrak{P}^{(i_0)}\mid\mathfrak{p}^{(i_0)}_{\nu})}}\right)^{a_{\nu i}}.\]
Hence
\begin{align*}
\left(\frac{\delta_2}{\lambda}\right)\mathcal{O}_L
	& = \left( \frac{\gamma_1^{(j)}}{\gamma_1^{(i_0)}}\right)^{n_1}\cdots \left( \frac{\gamma_{\nu}^{(j)}}{\gamma_{\nu}^{(i_0)}}\right)^{n_{\nu}} \mathcal{O}_L\\
	& = \left(\prod_{\mathfrak{P}\mid\mathfrak{p}_1} \frac{\mathfrak{P}^{(j) \ e(\mathfrak{P}^{(j)}\mid\mathfrak{p}_1^{(j)})}}{\mathfrak{P}^{(i_0) \ e(\mathfrak{P}^{(i_0)}\mid\mathfrak{p}^{(i_0)}_1)}}\right)^{\sum_{i = 1}^\nu n_ia_{1i}} \cdots \left(\prod_{\mathfrak{P}\mid\mathfrak{p}_{\nu}} \frac{\mathfrak{P}^{(j) \ e(\mathfrak{P}^{(j)}\mid\mathfrak{p}^{(j)}_{\nu})}}{\mathfrak{P}^{(i_0) \ e(\mathfrak{P}^{(i_0)}\mid\mathfrak{p}^{(i_0)}_{\nu})}}\right)^{\sum_{i=1}^{\nu} n_ia_{\nu i}}\\
	& = \left(\prod_{\mathfrak{P}\mid\mathfrak{p}_1} \frac{\mathfrak{P}^{(j) \ e(\mathfrak{P}^{(j)}\mid\mathfrak{p}_1^{(j)})}}{\mathfrak{P}^{(i_0) \ e(\mathfrak{P}^{(i_0)}\mid\mathfrak{p}^{(i_0)}_1)}}\right)^{u_1 - r_1} \cdots \left(\prod_{\mathfrak{P}\mid\mathfrak{p}_{\nu}} \frac{\mathfrak{P}^{(j) \ e(\mathfrak{P}^{(j)}\mid\mathfrak{p}^{(j)}_{\nu})}}{\mathfrak{P}^{(i_0) \ e(\mathfrak{P}^{(i_0)}\mid\mathfrak{p}^{(i_0)}_{\nu})}}\right)^{u_{\nu} - r_{\nu}}
\end{align*}
and so
\[\ord_{\mathfrak{P}}\left( \frac{\delta_2}{\lambda}\right)=
\begin{cases}
(u_l - r_l)e(\mathfrak{P}^{(j)}|\mathfrak{p}_l^{(j)})	
	& \textnormal{ if } \mathfrak{P}^{(j)} \mid p_l , \ p_l \in \{p_1,\dots, p_{\nu}\}\\
(r_l - u_l)e(\mathfrak{P}^{(i_0)}|\mathfrak{p}_l^{(i_0)})
	& \textnormal{ if } \mathfrak{P}^{(i_0)}\mid p_l, \ p_l \in \{p_1,\dots, p_{\nu}\}\\
0 	& \textnormal{ otherwise}.
\end{cases}\]
\end{proof}

Let $\log^+(\cdot)$ denote the real valued function $\max(\log(\cdot), 0)$ on $\mathbb{R}_{\geq 0}$. 

\begin{proposition}\label{prop:heightdecomp}
The height $h\left(\frac{\delta_2}{\lambda}\right)$ admits a decomposition
\[h\left(\frac{\delta_2}{\lambda}\right) = \frac{1}{[K:\mathbb{Q}]}\sum_{l = 1}^{\nu} \log(p_l)|u_l - r_l| + \frac{1}{[L:\mathbb{Q}]}\sum_{w :L \to \mathbb{C}} \log \max \left\{ \left|w\left(\frac{\delta_2}{\lambda}\right)\right|, 1\right\}.\]
Further, if $[K:\mathbb{Q}] = 3$, then
\[\sum_{w :L \to \mathbb{C}} \log \max \left\{ \left|w\left(\frac{\delta_2}{\lambda}\right)\right|, 1\right\} = 
\begin{cases}
2\max_{w:L\to \mathbb{C}} \log \max \left\{ \left|w\left(\frac{\delta_2}{\lambda}\right)\right|, 1\right\} & \text{ if } \sqrt{\Delta}\notin\mathbb{Q} \\
\max_{w:L\to \mathbb{C}}\sum_{w :L \to \mathbb{C}} \log \max \left\{ \left|w\left(\frac{\delta_2}{\lambda}\right)\right|, 1\right\} & \text{ if } \sqrt{\Delta}\in\mathbb{Q}
\end{cases}\]
when one can choose $(i_0),(j), (k) : L \to \mathbb{C}$ such that $\mathfrak{p}_{p}^{(j)} \neq \mathfrak{p}_{p}^{(i_0)}$ for all $p \in S$. 
\end{proposition}
Denote by $m$ the number of infinite places $w: L \to \mathbb{C}$. 

\begin{proof}[Proof of Proposition~\ref{prop:heightdecomp}]Since $\frac{\delta_2}{\lambda} \in L$, the definition of the absolute logarithmic Weil height gives
\[h\left(\frac{\delta_2}{\lambda}\right)=\frac{1}{[L:\mathbb{Q}]}\sum_{w \in M_L} \log \max \left\{ \left\|\frac{\delta_2}{\lambda}\right\|_{w}, 1\right\}\]
where $||z||_w$ are the usual norms and $M_L$ is a set of inequivalent absolute values on $L$. In particular, if $w: L \to \mathbb{C}$ is an infinite place, we obtain
\[ \log \max \left\{ \left\|\frac{\delta_2}{\lambda}\right\|_{w}, 1\right\} = \log \max \left\{ \left|w\left(\frac{\delta_2}{\lambda}\right)\right|, 1\right\}.\]

Now, for $z = \frac{\delta_2}{\lambda}$ and $w = \mathfrak{P}$ a finite place, we have
\[ \log \max \{ \|z\|_{w}, 1\} = \max \left\{ \log\left(\frac{1}{N(\mathfrak{P})^{\ord_{\mathfrak{P}}(z)}} \right), 0\right\}. \]
By Lemma~\ref{lem:ordpz}, 
\[\ord_{\mathfrak{P}}\left( \frac{\delta_2}{\lambda}\right)=
\begin{cases}
(u_l - r_l)e(\mathfrak{P}^{(j)}|\mathfrak{p}_l^{(j)})	
	& \textnormal{ if } \mathfrak{P}^{(j)} \mid p_l , \ p_l \in \{p_1,\dots, p_{\nu}\}\\
(r_l - u_l)e(\mathfrak{P}^{(i_0)}|\mathfrak{p}_l^{(i_0)})
	& \textnormal{ if } \mathfrak{P}^{(i_0)}\mid p_l, \ p_l \in \{p_1,\dots, p_{\nu}\}\\
0 	& \textnormal{ otherwise}.
\end{cases}\]
That is, for $\mathfrak{P}^{(j)}\mid p_l$ where $p_l \in \{p_1, \dots, p_{\nu}\}$, we have
\begin{align*}
 \log \max \{ ||z||_{w}, 1\}	
 	& = \max \left\{ \log\left(\frac{1}{N(\mathfrak{P})^{\ord_{\mathfrak{P}}(z)}} \right), 0\right\}\\
	& = \max \left\{ \log\left(\frac{1}{N(\mathfrak{P})^{(u_l - r_l)e(\mathfrak{P}^{(j)}|\mathfrak{p}_l^{(j)})}} \right), 0\right\}\\
	& = \max \left\{ \log\left(\frac{1}{p_l^{(u_l - r_l)f(\mathfrak{P}^{(j)}\mid p_l)e(\mathfrak{P}^{(j)}|\mathfrak{p}_l^{(j)})}} \right), 0\right\}\\
	& = \max \left\{ -(u_l - r_l)f(\mathfrak{P}^{(j)}\mid p_l)e(\mathfrak{P}^{(j)}|\mathfrak{p}_l^{(j)})\log(p_l), 0\right\}.
\end{align*}
For $p_l \in \{p_1, \dots, p_{\nu}\}$, there is 1 unique prime ideal $\mathfrak{p}_1$ in the ideal equation \eqref{Eq:TMfactored} lying above $p_l$ in $K$. Hence, each $\mathfrak{P}$ lying over $p_l$ must also lie over $\mathfrak{p}_l$. Now, 
\begin{align*}
\sum_{\mathfrak{P}^{(j)} \mid \mathfrak{p}_l^{(j)}} \log \max \left\{ \left\|\frac{\delta_2}{\lambda}\right\|_{w}, 1\right\}
	& = \sum_{\mathfrak{P}^{(j)} \mid \mathfrak{p}_l^{(j)}} \max \left\{ -(u_l - r_l)f(\mathfrak{P}^{(j)}\mid p_l)e(\mathfrak{P}^{(j)}|\mathfrak{p}_l^{(j)})\log(p_l), 0\right\}\\
	& = \max \left\{ (r_l - u_l)\log(p_l), 0\right\}\sum_{\mathfrak{P}^{(j)} \mid \mathfrak{p}_l^{(j)}}f(\mathfrak{P}^{(j)}\mid p_l)e(\mathfrak{P}^{(j)}|\mathfrak{p}_l^{(j)})\\
	& = \max \left\{ (r_l - u_l)\log(p_l), 0\right\}\sum_{\mathfrak{P}^{(j)} \mid \mathfrak{p}_l^{(j)}}f(\mathfrak{P}^{(j)}\mid \mathfrak{p}_l^{(j)})f(\mathfrak{p}_l^{(j)}\mid p_l)e(\mathfrak{P}^{(j)}|\mathfrak{p}_l^{(j)})\\
	& = \max \left\{ (r_l - u_l)\log(p_l), 0\right\}f(\mathfrak{p}_l^{(j)}\mid p_l)\sum_{\mathfrak{P}^{(j)} \mid \mathfrak{p}_l^{(j)}}f(\mathfrak{P}^{(j)}\mid \mathfrak{p}_l^{(j)})e(\mathfrak{P}^{(j)}|\mathfrak{p}_l^{(j)})\\
	& = \max \left\{ (r_l - u_l)\log(p_l), 0\right\}f(\mathfrak{p}_l^{(j)}\mid p_l)[L:\mathbb{Q}(\theta^{(j)})]\\
	& = \max \left\{ (r_l - u_l)\log(p_l), 0\right\}f(\mathfrak{p}_l^{(j)}\mid p_l)[L:K].
\end{align*}
where the last inequality follows from $K = \mathbb{Q}(\theta) \cong \mathbb{Q}(\theta^{(j)})$.

Similarly, for $\mathfrak{P}^{(i_0)}\mid p_l$ where $p_l \in \{p_1, \dots, p_{\nu}\}$, we have
\begin{align*}
 \log \max \{ \|z\|_{w}, 1\}	
 	& = \max \left\{ \log\left(\frac{1}{N(\mathfrak{P})^{\ord_{\mathfrak{P}}(z)}} \right), 0\right\}\\
	& = \max \left\{ \log\left(\frac{1}{N(\mathfrak{P})^{(r_l - u_l)e(\mathfrak{P}^{(i_0)}|\mathfrak{p}_l^{(i_0)})}} \right), 0\right\}\\
	& = \max \left\{ \log\left(\frac{1}{p_l^{(r_l - u_l)f(\mathfrak{P}^{(i_0)}\mid p_l)e(\mathfrak{P}^{(i_0)}|\mathfrak{p}_l^{(i_0)})}} \right), 0\right\}\\
	& = \max \left\{ -(r_l - u_l)f(\mathfrak{P}^{(i_0)}\mid p_l)e(\mathfrak{P}^{(i_0)}|\mathfrak{p}_l^{(i_0)})\log(p_l), 0\right\},
\end{align*}
and so
\begin{align*}
\sum_{\mathfrak{P}^{(i_0)} \mid \mathfrak{p}_l^{(i_0)}} \log \max \left\{ \left\|\frac{\delta_2}{\lambda}\right\|_{w}, 1\right\}
	& = \max \left\{ (u_l - r_l)\log(p_l), 0\right\}f(\mathfrak{p}_l^{(i_0)}\mid p_l)[L:K].
\end{align*}

Lastly, if $w = \mathfrak{P}$ such that $\mathfrak{P} \neq \mathfrak{P}^{(i_0)},  \mathfrak{P}^{(j)}$, we have
\begin{align*}
 \log \max \{ \|z\|_{w}, 1\}	
 	& = \max \left\{ \log\left(\frac{1}{N(\mathfrak{P})^{\ord_{\mathfrak{P}}(z)}} \right), 0\right\}\\
	& = \max \left\{ \log\left(\frac{1}{N(\mathfrak{P})^{0}} \right), 0\right\}\\
	& = 0.
\end{align*}

Now, we have 
\begin{align*}
h\left(\frac{\delta_2}{\lambda}\right)
	& =\frac{1}{[L:\mathbb{Q}]}\sum_{w \in M_L} \log \max \left\{ \left\|\frac{\delta_2}{\lambda}\right\|_{w}, 1\right\}\\
	& = \frac{1}{[L:\mathbb{Q}]}\sum_{w :L \to \mathbb{C}} \log \max \left\{ \left|w\left(\frac{\delta_2}{\lambda}\right)\right|, 1\right\} + \frac{1}{[L:\mathbb{Q}]}\sum_{\mathfrak{P} \in \mathcal{O}_L \text{ finite }} \log \max \left\{ \left\|\frac{\delta_2}{\lambda}\right\|_{\mathfrak{P}}, 1\right\},
\end{align*}
where
\begin{align*}
& \sum_{\mathfrak{P} \in \mathcal{O}_L \text{ finite }} \log \max \left\{ \left\|\frac{\delta_2}{\lambda}\right\|_{w}, 1\right\} \\
	& = \sum_{l = 1}^{\nu} \left(\sum_{\mathfrak{P}^{(j)} \mid \mathfrak{p}_l^{(j)}} \log \max \left\{ \left\|\frac{\delta_2}{\lambda}\right\|_{w}, 1\right\} + \sum_{\mathfrak{P}^{(i_0)} \mid \mathfrak{p}_l^{(i_0)}} \log \max \left\{ \left\|\frac{\delta_2}{\lambda}\right\|_{w}, 1\right\}\right)\\
	& = \sum_{l = 1}^{\nu} \left(\max \left\{ (r_l - u_l)\log(p_l), 0\right\}f(\mathfrak{p}_l^{(j)}\mid p_l)[L:K] + \max \left\{ (u_l - r_l)\log(p_l), 0\right\}f(\mathfrak{p}_l^{(i_0)}\mid p_l)[L:K]\right)\\
	& =  [L:K]\sum_{l = 1}^{\nu} \log(p_l)\left(\max \left\{ -(u_l - r_l), 0\right\}+ \max \left\{ (u_l - r_l), 0\right\}\right)\\
	& =  [L:K]\sum_{l = 1}^{\nu} \log(p_l)\max \left\{ -(u_l - r_l), (u_l - r_l)\right\}\\
	& =  [L:K]\sum_{l = 1}^{\nu} \log(p_l)|u_l - r_l|.
\end{align*}
Here, we recall that $K = \mathbb{Q}(\theta) \cong \mathbb{Q}(\theta^{(i_0)}) \cong \mathbb{Q}(\theta^{(j)})$ and therefore 
\[f(\mathfrak{p}_l^{(i_0)}\mid p_l) = f(\mathfrak{p}_l^{(j)}\mid p_l) = f(\mathfrak{p}_l\mid p_l) = 1.\]
Altogether, we have
\begin{align*}
h\left(\frac{\delta_2}{\lambda}\right)
	& =\frac{1}{[L:\mathbb{Q}]}\sum_{w \in M_L} \log \max \left\{ \left\|\frac{\delta_2}{\lambda}\right\|_{w}, 1\right\}\\
	& = \frac{1}{[L:\mathbb{Q}]}\sum_{w :L \to \mathbb{C}} \log \max \left\{ \left|w\left(\frac{\delta_2}{\lambda}\right)\right|, 1\right\} + \frac{1}{[L:\mathbb{Q}]}\sum_{\mathfrak{P} \in \mathcal{O}_L \text{ finite }} \log \max \left\{ \left\|\frac{\delta_2}{\lambda}\right\|_{\mathfrak{P}}, 1\right\}\\
	& = \frac{1}{[L:\mathbb{Q}]}\sum_{w :L \to \mathbb{C}} \log \max \left\{ \left|w\left(\frac{\delta_2}{\lambda}\right)\right|, 1\right\} + \frac{1}{[K:\mathbb{Q}]}\sum_{l = 1}^{\nu} \log(p_l)|u_l - r_l|\\
	& = \frac{1}{[L:\mathbb{Q}]}\sum_{w :L \to \mathbb{C}} \log \max \left\{ \left|w\left(\frac{\delta_2}{\lambda}\right)\right|, 1\right\} + \frac{1}{[K:\mathbb{Q}]}\log\left(p_1^{|u_1 - r_1|} \cdots p_{\nu}^{|u_{\nu} - r_{\nu}|}\right).
\end{align*}

To prove the last statement, we first assume that $\sqrt{\Delta}\in\mathbb{Q}$. Then $L=\mathbb{Q}(\theta, \sqrt{\Delta}) = \mathbb{Q}(\theta) = K$ and the Galois group of $L/\mathbb{Q}$ is the alternating group $A_3$. Hence the Galois group is generated by $\sigma$, which takes $\theta$ to one of the other roots of $g(t)$. In particular, 
\[\Gal(L/\mathbb{Q}) = \{ \text{id}_\text{L}, \sigma, \sigma^2\},\]
where
\begin{align*}
\text{id}_{\text{L}}: 
\begin{cases}
\theta_1 \mapsto \theta_1\\
\theta_2 \mapsto \theta_2\\
\theta_3 \mapsto \theta_3\\
\sqrt{D} \mapsto \sqrt{D}
\end{cases}
& \quad ,
\sigma: 
\begin{cases}
\theta_1 \mapsto \theta_2\\
\theta_2 \mapsto \theta_3\\
\theta_3 \mapsto \theta_1\\
\sqrt{D} \mapsto \sqrt{D}
\end{cases}
& \quad ,
\sigma^2: 
\begin{cases}
\theta_1 \mapsto \theta_3\\
\theta_2 \mapsto \theta_1\\
\theta_3 \mapsto \theta_2\\
\sqrt{D} \mapsto \sqrt{D}
\end{cases},
\end{align*}
Writing $j = 1, i_0 = 2$ and $k = 3$, the orbit of 
\[\frac{\delta_2}{\lambda}= \prod_{i = 1}^{r}\left( \frac{\varepsilon_i^{(1)}}{\varepsilon_i^{(2)}}\right)^{a_i} \prod_{i = 1}^{\nu} \left( \frac{\gamma_i^{(1)}}{\gamma_i^{(2)}}\right)^{n_i}\in L\]
is
\[\left\{ \prod_{i = 1}^{r}\left( \frac{\varepsilon_i^{(1)}}{\varepsilon_i^{(2)}}\right)^{a_i} \prod_{i = 1}^{\nu} \left( \frac{\gamma_i^{(1)}}{\gamma_i^{(2)}}\right)^{n_i}, 
	\prod_{i = 1}^{r}\left( \frac{\varepsilon_i^{(2)}}{\varepsilon_i^{(3)}}\right)^{a_i} \prod_{i = 1}^{\nu} \left( \frac{\gamma_i^{(2)}}{\gamma_i^{(3)}}\right)^{n_i}, 
	\prod_{i = 1}^{r}\left( \frac{\varepsilon_i^{(3)}}{\varepsilon_i^{(1)}}\right)^{a_i} \prod_{i = 1}^{\nu} \left( \frac{\gamma_i^{(3)}}{\gamma_i^{(1)}}\right)^{n_i}\right\}.\]
We choose $a,b,c\in\{1,2,3\}$ such that 
\[ \left|\prod_{i = 1}^{r}\left( \varepsilon_i^{(a)}\right)^{a_i} \prod_{i = 1}^{\nu} \left( \gamma_i^{(a)}\right)^{n_i}\right| \geq
	\left|\prod_{i = 1}^{r}\left( \varepsilon_i^{(b)}\right)^{a_i} \prod_{i = 1}^{\nu} \left( \gamma_i^{(b)}\right)^{n_i}\right| \geq
	\left|\prod_{i = 1}^{r}\left( \varepsilon_i^{(c)}\right)^{a_i} \prod_{i = 1}^{\nu} \left( \gamma_i^{(c)}\right)^{n_i}\right|.\]
Then we obtain
\begin{align*}
\sum_{w :L \to \mathbb{C}} \log \max \left\{ \left|w\left(\frac{\delta_2}{\lambda}\right)\right|, 1\right\}
	& = \log \max \left\{ \left|\text{id}_L\left(\frac{\delta_2}{\lambda}\right)\right|, 1\right\} 
		+ \log \max \left\{ \left|\sigma\left(\frac{\delta_2}{\lambda}\right)\right|,1\right\} \\
		& \quad+ \log \max \left\{ \left|\sigma^2\left(\frac{\delta_2}{\lambda}\right)\right|, 1\right\}\\
	& = \log \max \left\{ \left|\prod_{i = 1}^{r}\left( \frac{\varepsilon_i^{(a)}}{\varepsilon_i^{(b)}}\right)^{a_i} \prod_{i = 1}^{\nu} \left( \frac{\gamma_i^{(a)}}{\gamma_i^{(b)}}\right)^{n_i} \right|, 1\right\} \\
		& \quad + \log \max \left\{ \left|\ \prod_{i = 1}^{r}\left( \frac{\varepsilon_i^{(b)}}{\varepsilon_i^{(c)}}\right)^{a_i} \prod_{i = 1}^{\nu} \left( \frac{\gamma_i^{(b)}}{\gamma_i^{(c)}}\right)^{n_i} \right|,1\right\} \\
		& \quad+ \log \max \left\{ \left| \prod_{i = 1}^{r}\left( \frac{\varepsilon_i^{(c)}}{\varepsilon_i^{(a)}}\right)^{a_i} \prod_{i = 1}^{\nu} \left( \frac{\gamma_i^{(c)}}{\gamma_i^{(a)}}\right)^{n_i} \right|, 1\right\}\\
	& = \log  \left|\prod_{i = 1}^{r}\left( \frac{\varepsilon_i^{(a)}}{\varepsilon_i^{(b)}}\right)^{a_i} \prod_{i = 1}^{\nu} \left( \frac{\gamma_i^{(a)}}{\gamma_i^{(b)}}\right)^{n_i} \right| \\
		& \quad + \log \left|\ \prod_{i = 1}^{r}\left( \frac{\varepsilon_i^{(b)}}{\varepsilon_i^{(c)}}\right)^{a_i} \prod_{i = 1}^{\nu} \left( \frac{\gamma_i^{(b)}}{\gamma_i^{(c)}}\right)^{n_i} \right| \\
	& = \log  \left|\prod_{i = 1}^{r}\left( \frac{\varepsilon_i^{(a)}}{\varepsilon_i^{(c)}}\right)^{a_i} \prod_{i = 1}^{\nu} \left( \frac{\gamma_i^{(a)}}{\gamma_i^{(c)}}\right)^{n_i} \right|.
\end{align*}
Hence it follows that
\[\sum_{w :L \to \mathbb{C}} \log \max \left\{ \left|w\left(\frac{\delta_2}{\lambda}\right)\right|, 1\right\} = 
\max_{w:L\to \mathbb{C}}\sum_{w :L \to \mathbb{C}} \log \max \left\{ \left|w\left(\frac{\delta_2}{\lambda}\right)\right|, 1\right\}.\]

It remains to consider the case when $\sqrt{\Delta} \notin \mathbb{Q}$. Then the Galois group of $L/\mathbb{Q}$ is the symmetric group $S_3$. We now write $j = 1, i_0 = 2,$ and $k = 3$. The orbit of 
\[\frac{\delta_2}{\lambda}= \prod_{i = 1}^{r}\left( \frac{\varepsilon_i^{(1)}}{\varepsilon_i^{(2)}}\right)^{a_i} \prod_{i = 1}^{\nu} \left( \frac{\gamma_i^{(1)}}{\gamma_i^{(2)}}\right)^{n_i}\in L\]
is
\begin{align*}
& \left\{ \prod_{i = 1}^{r}\left( \frac{\varepsilon_i^{(1)}}{\varepsilon_i^{(2)}}\right)^{a_i} \prod_{i = 1}^{\nu} \left( \frac{\gamma_i^{(1)}}{\gamma_i^{(2)}}\right)^{n_i} , 
	\prod_{i = 1}^{r}\left( \frac{\varepsilon_i^{(2)}}{\varepsilon_i^{(3)}}\right)^{a_i} \prod_{i = 1}^{\nu} \left( \frac{\gamma_i^{(2)}}{\gamma_i^{(3)}}\right)^{n_i}, 
	\prod_{i = 1}^{r}\left( \frac{\varepsilon_i^{(3)}}{\varepsilon_i^{(1)}}\right)^{a_i} \prod_{i = 1}^{\nu} \left( \frac{\gamma_i^{(3)}}{\gamma_i^{(1)}}\right)^{n_i} \right. \\
	& \quad \left. \prod_{i = 1}^{r}\left( \frac{\varepsilon_i^{(3)}}{\varepsilon_i^{(2)}}\right)^{a_i} \prod_{i = 1}^{\nu} \left( \frac{\gamma_i^{(3)}}{\gamma_i^{(2)}}\right)^{n_i} , 
	\prod_{i = 1}^{r}\left( \frac{\varepsilon_i^{(1)}}{\varepsilon_i^{(3)}}\right)^{a_i} \prod_{i = 1}^{\nu} \left( \frac{\gamma_i^{(1)}}{\gamma_i^{(3)}}\right)^{n_i}, 
	\prod_{i = 1}^{r}\left( \frac{\varepsilon_i^{(2)}}{\varepsilon_i^{(1)}}\right)^{a_i} \prod_{i = 1}^{\nu} \left( \frac{\gamma_i^{(2)}}{\gamma_i^{(1)}}\right)^{n_i} \right\}.
\end{align*}
We choose $a,b,c\in\{1,2,3\}$ such that 
\[ \left|\prod_{i = 1}^{r}\left( \varepsilon_i^{(a)}\right)^{a_i} \prod_{i = 1}^{\nu} \left( \gamma_i^{(a)}\right)^{n_i}\right| \geq
	\left|\prod_{i = 1}^{r}\left( \varepsilon_i^{(b)}\right)^{a_i} \prod_{i = 1}^{\nu} \left( \gamma_i^{(b)}\right)^{n_i}\right| \geq
	\left|\prod_{i = 1}^{r}\left( \varepsilon_i^{(c)}\right)^{a_i} \prod_{i = 1}^{\nu} \left( \gamma_i^{(c)}\right)^{n_i}\right|.\]
Then we obtain
\begin{align*}
\sum_{w :L \to \mathbb{C}} \log \max \left\{ \left|w\left(\frac{\delta_2}{\lambda}\right)\right|, 1\right\}
	& = \log \max \left\{ \left|\text{id}_L\left(\frac{\delta_2}{\lambda}\right)\right|, 1\right\} 
		+ \log \max \left\{ \left|\sigma\left(\frac{\delta_2}{\lambda}\right)\right|,1\right\} \\
		& \quad+ \log \max \left\{ \left|\sigma^2\left(\frac{\delta_2}{\lambda}\right)\right|, 1\right\} + \log \max \left\{ \left|\tau\left(\frac{\delta_2}{\lambda}\right)\right|, 1\right\}\\
		& \quad+ \log \max \left\{ \left|\tau\sigma\left(\frac{\delta_2}{\lambda}\right)\right|, 1\right\} + \log \max \left\{ \left|\tau\sigma^2\left(\frac{\delta_2}{\lambda}\right)\right|, 1\right\}\\
	& = \log \left|\prod_{i = 1}^{r}\left( \frac{\varepsilon_i^{(a)}}{\varepsilon_i^{(b)}}\right)^{a_i} \prod_{i = 1}^{\nu} \left( \frac{\gamma_i^{(a)}}{\gamma_i^{(b)}}\right)^{n_i} \right|\\
		& \quad + \log \left|\ \prod_{i = 1}^{r}\left( \frac{\varepsilon_i^{(b)}}{\varepsilon_i^{(c)}}\right)^{a_i} \prod_{i = 1}^{\nu} \left( \frac{\gamma_i^{(b)}}{\gamma_i^{(c)}}\right)^{n_i} \right|\\
		& \quad+ \log \left| \prod_{i = 1}^{r}\left( \frac{\varepsilon_i^{(a)}}{\varepsilon_i^{(c)}}\right)^{a_i} \prod_{i = 1}^{\nu} \left( \frac{\gamma_i^{(a)}}{\gamma_i^{(c)}}\right)^{n_i} \right|\\
	& = 2\log  \left|\prod_{i = 1}^{r}\left( \frac{\varepsilon_i^{(a)}}{\varepsilon_i^{(c)}}\right)^{a_i} \prod_{i = 1}^{\nu} \left( \frac{\gamma_i^{(a)}}{\gamma_i^{(c)}}\right)^{n_i} \right|.
\end{align*}
Hence it follows that
\[\sum_{w :L \to \mathbb{C}} \log \max \left\{ \left|w\left(\frac{\delta_2}{\lambda}\right)\right|, 1\right\} = 
2\max_{w:L\to \mathbb{C}}\sum_{w :L \to \mathbb{C}} \log \max \left\{ \left|w\left(\frac{\delta_2}{\lambda}\right)\right|, 1\right\}.\]
\end{proof}

%---------------------------------------------------------------------------------------------------------------------------------------------%
\section{Initial height bounds}

Recall that we seek solutions to
\begin{equation*}
\lambda = \delta_1 \prod_{i = 1}^r\left( \frac{\varepsilon_i^{(k)}}{\varepsilon_i^{(j)}}\right)^{a_i}\prod_{i = 1}^{\nu} \left( \frac{\gamma_i^{(k)}}{\gamma_i^{(j)}}\right)^{n_i} - 1 = \delta_2 \prod_{i = 1}^{r}\left( \frac{\varepsilon_i^{(i_0)}}{\varepsilon_i^{(j)}}\right)^{a_i} \prod_{i = 1}^{\nu} \left( \frac{\gamma_i^{(i_0)}}{\gamma_i^{(j)}}\right)^{n_i},
\end{equation*}
where
\[\delta_1 = \frac{\theta^{(i_0)} - \theta^{(j)}}{\theta^{(i_0)} - \theta^{(k)}}\cdot\frac{\alpha^{(k)}\zeta^{(k)}}{\alpha^{(j)}\zeta^{(j)}}, \quad \delta_2 = \frac{\theta^{(j)} - \theta^{(k)}}{\theta^{(k)} - \theta^{(i_0)}}\cdot \frac{\alpha^{(i_0)}\zeta^{(i_0)}}{\alpha^{(j)}\zeta^{(j)}}\]
are constants and $r = 1$ or $r = 2$. 

For simplicity of notation, write
\[y =  \prod_{i = 1}^r\left( \frac{\varepsilon_i^{(k)}}{\varepsilon_i^{(j)}}\right)^{a_i}\prod_{i = 1}^{\nu} \left( \frac{\gamma_i^{(k)}}{\gamma_i^{(j)}}\right)^{n_i}, \quad 
x = \prod_{i = 1}^{r}\left( \frac{\varepsilon_i^{(i_0)}}{\varepsilon_i^{(j)}}\right)^{a_i} \prod_{i = 1}^{\nu} \left( \frac{\gamma_i^{(i_0)}}{\gamma_i^{(j)}}\right)^{n_i}\]
so that our equation is
\begin{equation} \label{eq:TrueSunit}
\delta_1y - 1 = \delta_2x.
\end{equation}
Let $z= \frac{1}{x} = \frac{\delta_2}{\lambda}$ and denote by $\Sigma$ the set of pairs $(x,y)$ satisfying \eqref{eq:TrueSunit}. That is, $\Sigma$ denotes the set of tuples $(n_1, \dots, n_{\nu}, a_1, \dots, a_r)$ giving $x,y$ which satisfy \eqref{eq:TrueSunit}.

Let $\mathbf{l},\mathbf{h}\in\mathbb{R}^{\nu + m}$ with $\mathbf{0}\leq \mathbf{l}\leq \mathbf{h}$. Then we define $\Sigma(\mathbf{l},\mathbf{h})$ as the set of all $(x,y)\in \Sigma$ such that $\left(h_v\left(\frac{\delta_2}{\lambda}\right)\right)\leq \mathbf{h}$ and such that $\left(h_v\left(\frac{\delta_2}{\lambda}\right)\right)\nleq \mathbf{l}$. 
\[\Sigma(\mathbf{l},\mathbf{h}) = \{(x,y) \in \Sigma \ | \ (h_v(z))\leq \mathbf{h} \text{ and } (h_v(z))\nleq \mathbf{l}\}.\]
Write $\Sigma(\mathbf{h})=\Sigma(\mathbf{l},\mathbf{h})$ if $\mathbf{l}=\mathbf{0}$. Further, for each place $w$, we denote by $\Sigma_w(\mathbf{l},\mathbf{h})$ the set of all $(x,y)\in\Sigma(\mathbf{h})$ such that $h_w(z)>l_w$. 

Recall that 
\[f(x,y) = x^3 + C_1 x^{2}y + C_2xy^2 + C_3y^3 = c p_1^{z_1} \cdots p_v^{z_v}.\]
and $\gcd(x,y)=1$ and $S = \{p_1, \dots, p_v\}$. Let 
\[N_S = \prod_{p\in S}p.\]
To measure an integer $b$ and the finite set $S$, we take
\begin{align*}
b_S	& = 1728 N_S^2 \prod_{p \notin S} p^{\min(2,\ord_p(b))}\\
	& = 1728 \prod_{p\in S}p^2 \prod_{\substack{p \notin S \\ p \mid b}} p^{\min(2,\ord_p(b))}.
\end{align*}
Recall further that the Weil height of an integer $n \in \mathbb{Z}\backslash {0}$ is given by
\[h(n) = \log|n|.\] 
Now, denote by $h(f-c)$ the maximum logarithmic Weil heights of the coefficients of the polynomial $f - c$,
\begin{align*}
h(f-c) & = \max(\log|C_1|, \log|C_2|, \log|C_3|, \log|c|),		
\end{align*}
where we recall that $C_i \in \mathbb{N}$.
Put $b = 432 \Delta c^2$ with $\Delta$ the discriminant of $F$. Now, let 
\[\Omega = 2b_S \log(b_S) + 172h(f-c).\]
By \edit{Rafael and Benjamin's paper}, 
\[\max(h(x),h(y))\leq \Omega. \]

We recall that  
\[\beta = X-Y\theta = \alpha \zeta \varepsilon_1^{a_1} \cdots \varepsilon_r^{a_r}\cdot \gamma_1^{n_1}\cdots \gamma_{\nu}^{n_{\nu}}\]
and we define
\[\Omega' = 2h(\alpha) + 4\Omega + 2h(\theta) + 2\log(2).\]

For $z \in K$, we recall
\[h(z)=\frac{1}{[K:\mathbb{Q}]}\sum_{w \in M_K} \log \max \left\{ \left\|z\right\|_{w}, 1\right\}\]
where $||z||_w$ are the usual norms and $M_K$ is a set of inequivalent absolute values on $K$. Now, 
\[(\alpha)\mathcal{O}_K = \mathfrak{p}_1^{A_1} \cdots \mathfrak{p}_n^{A_n} \quad \text{ and } \quad (\theta)\mathcal{O}_K = \mathfrak{p}_1^{B_1} \cdots \mathfrak{p}_m^{B_m}.\]
For $w = \mathfrak{p}$ a finite place, we have
\[ \log \max \{ \|z\|_{w}, 1\} = \max \left\{ \log\left(\frac{1}{N(\mathfrak{p}_i)^{\ord_{\mathfrak{p}_i}(\alpha)}} \right), 0\right\} = \max \left\{ \log\left(\frac{1}{p^{fA_i}} \right), 0\right\} = 0\]
and 
\[ \log \max \{ \|z\|_{w}, 1\} = \max \left\{ \log\left(\frac{1}{N(\mathfrak{p}_i)^{\ord_{\mathfrak{p}_i}(\theta)}} \right), 0\right\} = \max \left\{ \log\left(\frac{1}{p^{fB_i}} \right), 0\right\} = 0.\]
It follows that
\[h(\alpha)=\frac{1}{[K:\mathbb{Q}]}\sum_{w \in M_K} \log \max \left\{ \left\|\alpha\right\|_{w}, 1\right\}
	= \frac{1}{[K:\mathbb{Q}]}\sum_{\sigma:K \to \mathbb{C}} \log \max \left\{ |\sigma(\alpha)|, 1\right\}
\]
and 
\[h(\theta)=\frac{1}{[K:\mathbb{Q}]}\sum_{w \in M_K} \log \max \left\{ \left\|\theta\right\|_{w}, 1\right\}
	= \frac{1}{[K:\mathbb{Q}]}\sum_{\sigma:K \to \mathbb{C}} \log \max \left\{ |\sigma(\theta)|, 1\right\}.
\]
Now, 
\begin{align*}
\Omega'	
	& = 2h(\alpha) + 4\Omega + 2h(\theta) + 2\log(2)\\
	& = \frac{2}{[K:\mathbb{Q}]}\sum_{\sigma:K \to \mathbb{C}} \log \max \left\{ |\sigma(\alpha)|, 1\right\} + 4\Omega + \frac{2}{[K:\mathbb{Q}]}\sum_{\sigma:K \to \mathbb{C}} \log \max \left\{ |\sigma(\theta)|, 1\right\} + 2\log(2)
\end{align*}

\begin{lemma}
Let ${\mathbf{m} = (n_1, \dots, n_{\nu}, a_1, \dots, a_r) \in \mathbb{R}^{r + \nu}}$ be any solution of \eqref{Eq:main3}. If $\mathbf{h} \in\mathbb{R}^{\nu + r}$ with $\mathbf{h} = (\Omega')$, then $\mathbf{m}\in \Sigma(h)$, where
\[\Sigma(h) = \Sigma(0,h)= \{(x,y) \in \Sigma \ | \ (h_v(z))\leq h \text{ and }  (h_v(z))\nleq 0 \},\]
\end{lemma}
That is, all solutions $(x,y) \in \Sigma$ satisfy $\mathbf{m}\in \Sigma(h)$ if $\mathbf{h} = (\Omega')$.

\begin{proof}
Let $(x,y) \in \Sigma$. Then $(x,y)$ satisfy $\mu_0y-\lambda_0x=1$. We must show that the resulting value of $z:= \frac{1}{x} = \frac{\delta_2}{\lambda}$ arising from this choice of $x,y$ satisfies
\[0 < \left(h_v\left(\frac{\delta_2}{\lambda}\right)\right)\leq h.\]
Now, via Rafael and Benjamin, for a solution $X,Y$ of $f(X,Y) = c p_1^{z_1}\cdots p_v^{z_v}$, we have
\[\max(h(X),h(Y)) \leq \Omega.\]
We use the following height properties
\begin{enumerate}
\item For a non-zero rational number $a/b$ where $\gcd(a,b) = 1$,
\[h(a/b) = \max \{\log{|a|}, \log{|b|}\}\]
\item For $\alpha \in \overline{\mathbb{Q}}$, $n \in \mathbb{N}$, we have
\[h(n \alpha) = n h(\alpha).\]
\item For $\alpha, \beta \in \overline{\mathbb{Q}}$, we have
\[h(\alpha + \beta) \leq h(\alpha) + h(\beta) + \log{2}.\]
\item For $\alpha, \beta \in \overline{\mathbb{Q}}$, we have
\[h(\alpha\beta) \leq h(\alpha) + h(\beta).\]
\item For $\alpha \in \overline{\mathbb{Q}}$, we have
\[h(1/\alpha) = h(\alpha).\]
\end{enumerate}

Now, applying these properties to $\beta = X-\theta Y$, we obtain
\begin{align*}
h(\beta)	& = h(X-\theta Y)\\
		& \leq h(X) + h(-\theta Y) + \log{2}\\
		& \leq h(X) + h(-\theta) + h(Y) + \log{2}\\
		& = h(X) + h(\theta) + h(Y) + \log{2}\\
		& \leq 2\Omega + h(\theta) + \log{2}.
\end{align*}
Now, $h(\beta) = h(\beta^{(i)})$, hence
\[h(\beta^{(i)}) \leq 2\Omega + h(\theta) + \log{2}.\]
Further, we have
\begin{align*}
\delta_2x	& =  \delta_2 \prod_{i = 1}^{r}\left( \frac{\varepsilon_i^{(i_0)}}{\varepsilon_i^{(j)}}\right)^{a_i} \prod_{i = 1}^{\nu} \left( \frac{\gamma_i^{(i_0)}}{\gamma_i^{(j)}}\right)^{n_i} \\
		& =  \frac{\theta^{(j)} - \theta^{(k)}}{\theta^{(k)} - \theta^{(i_0)}}\cdot \frac{\alpha^{(i_0)}\zeta^{(i_0)}}{\alpha^{(j)}\zeta^{(j)}} \prod_{i = 1}^{r}\left( \frac{\varepsilon_i^{(i_0)}}{\varepsilon_i^{(j)}}\right)^{a_i} \prod_{i = 1}^{\nu} \left( \frac{\gamma_i^{(i_0)}}{\gamma_i^{(j)}}\right)^{n_i} \\
		& = \frac{\theta^{(j)} - \theta^{(k)}}{\theta^{(k)} - \theta^{(i_0)}}\cdot \frac{\beta^{(i_0)}}{\beta^{(j)}}.
\end{align*}
This means that 
\begin{align*}
x	& = \frac{\theta^{(j)} - \theta^{(k)}}{\theta^{(k)} - \theta^{(i_0)}}\cdot \frac{\beta^{(i_0)}}{\beta^{(j)}}\cdot\frac{1}{\delta_2}\\
	& =  \frac{\theta^{(j)} - \theta^{(k)}}{\theta^{(k)} - \theta^{(i_0)}}\cdot \frac{\beta^{(i_0)}}{\beta^{(j)}}\cdot\frac{1}{\frac{\theta^{(j)} - \theta^{(k)}}{\theta^{(k)} - \theta^{(i_0)}}\cdot \frac{\alpha^{(i_0)}\zeta^{(i_0)}}{\alpha^{(j)}\zeta^{(j)}}}\\
	& = \frac{\theta^{(j)} - \theta^{(k)}}{\theta^{(k)} - \theta^{(i_0)}}\cdot \frac{\beta^{(i_0)}}{\beta^{(j)}}\cdot\frac{\theta^{(k)} - \theta^{(i_0)}}{\theta^{(j)} - \theta^{(k)}}\cdot \frac{\alpha^{(j)}\zeta^{(j)}}{\alpha^{(i_0)}\zeta^{(i_0)}}\\
	& =\frac{\beta^{(i_0)}}{\beta^{(j)}}\cdot \frac{\alpha^{(j)}\zeta^{(j)}}{\alpha^{(i_0)}\zeta^{(i_0)}}.
\end{align*}
Hence, 
\begin{align*}
h(x)	& = h\left( \frac{\beta^{(i_0)}}{\beta^{(j)}}\cdot \frac{\alpha^{(j)}\zeta^{(j)}}{\alpha^{(i_0)}\zeta^{(i_0)}}\right)\\
	& = h(\beta^{(i_0)}) + h\left( \frac{1}{\beta^{(j)}}\right) + h(\alpha^{(j)}) + h\left( \frac{1}{\alpha^{(i_0)}}\right)+ h(\zeta^{(j)}) + h\left( \frac{1}{\zeta^{(i_0)}}\right)\\
	& = 2h(\beta) + 2h(\alpha) + 2h(\zeta)\\
	& \leq 2(2\Omega + h(\theta) + \log{2}) +2h(\alpha) + 2h(\zeta)\\
	& = 4\Omega + 2h(\theta) + 2\log{2} + 2h(\alpha) + 2h(\zeta).
\end{align*}
Now, 
\begin{align*}
h(\zeta)	& =\frac{1}{[K:\mathbb{Q}]}\sum_{w \in M_K} \log \max \left\{ \left\|\zeta\right\|_{w}, 1\right\}\\
		& = \frac{1}{[K:\mathbb{Q}]}\sum_{\sigma:K \to \mathbb{C}} \log \max \left\{ |\sigma(\zeta)|, 1\right\}\\
		& = \frac{1}{[K:\mathbb{Q}]}\sum_{\sigma:K \to \mathbb{C}} \log \max \left\{ 1, 1\right\}\\
		& = 0.
\end{align*}
Therefore, 
\[h(x) \leq 4\Omega + 2h(\theta) + 2\log{2} + 2h(\alpha) = \Omega'\]
and hence
\[h(z) = h\left(\frac{\delta_2}{\lambda}\right) = h(1/x) = h(x) \leq \Omega'.\]
Together with $\displaystyle h_v\left(\frac{\delta_2}{\lambda}\right) \leq h\left(\frac{\delta_2}{\lambda}\right)$ implies
\[h_v\left(\frac{\delta_2}{\lambda}\right) \leq \Omega'\]
for each $v \in S^*.$
Similarly, by definition, we have $h_v\left(\frac{\delta_2}{\lambda}\right) \geq 0$. That is, $(x,y) \in \Sigma(h)$ as required. 
\end{proof}

\subsection{Coverings of $\Sigma$}

From the previous section, we see that all solutions $(x,y) \in \Sigma$ satisfy $\mathbf{m}\in \Sigma(h)$ if ${\mathbf{h} = (\Omega')}$.

Now, let $l,h\in\mathbb{R}^{\nu+r}$ with $0\leq l\leq h$. With the definitions of the previous section, we have that

\begin{lemma}\label{lem:covering}
It holds that $\Sigma(h)=\Sigma(l,h)\cup \Sigma(l)$ and $\Sigma(l,h)=\cup_{v\in S^*}\Sigma_v(l,h)$.
\end{lemma}
\begin{proof}
Recall that 
\[\Sigma(l,h) = \{(x,y) \in \Sigma \ | \ (h_v(z))\leq h \text{ and } (h_v(z))\nleq l\},\]
\[\Sigma(h) = \Sigma(0,h)= \{(x,y) \in \Sigma \ | \ (h_v(z))\leq h \text{ and }  (h_v(z))\nleq 0 \},\]
and for each $w\in S^*$
\[\Sigma_w(l,h) = \{(x,y) \in \Sigma \ | \ (h_v(z))\leq h \text{ and }  (h_v(z))\nleq 0 \text{ and } h_w(z)>l_w\}.\]

Suppose $(x,y) \in \Sigma(h)$. By definition this means, $(h_v(z))\leq h$ and that $h_v(z) > 0$ for at least one coordinate. Since $0 \leq l \leq h$, it follows that either $(h_v(z))\leq l$ or $(h_v(z))\nleq l$. That is, either all coordinates satisfy $h_v(z) \leq l_v$, or there is at least one coordinate for which $h_v(z) > l_v$, meaning that either $(x,y) \in \Sigma(l)$ or $(x,y) \in \Sigma(l,h)$. Hence $(x,y) \in \Sigma(l,h) \cup \Sigma(l)$. That is, $\Sigma(h) \subseteq \Sigma(l,h) \cup \Sigma(l)$.

Conversely, we suppose that $(x,y) \in \Sigma(l,h) \cup \Sigma(l)$. It follows that either $(h_v(z))\leq h \text{ and } (h_v(z))\nleq l$ or $(h_v(z))\leq l \text{ and } (h_v(z))\nleq 0$. In either case, this means that $(h_v(z)) \leq h$ and $(h_v(z)) \nleq 0$. Hence $(x,y) \in \Sigma(h)$. Thus $\Sigma(h) \supseteq \Sigma(l,h) \cup \Sigma(l)$. Together with the previous paragraph, this yields $\Sigma(h)=\Sigma(l,h)\cup \Sigma(l)$.

To prove the second point, let $(x,y) \in \Sigma(l,h)$. Then there exists $w\in S^*$ with $h_w(z)>l_w$ and thus $(x,y)$ lies in $\Sigma_w(l,h)$. Hence $\Sigma(l,h) \subseteq \cup_{v\in S^*}\Sigma_v(l,h)$. Lastly, since each set $\Sigma_v(l,h)$ is contained in $\Sigma(l,h)$ it follows that $\Sigma(l,h)=\cup_{v\in S^*}\Sigma_v(l,h)$.
\end{proof}

Suppose now we are given an initial bound $h_0$ with $\Sigma=\Sigma(h_0)$ and pairs $(l_n,h_n)\in \mathbb{R}^{\nu + r}\times \mathbb{R}^{\nu + r}$ with $0\leq l_n\leq h_n$ and $h_{n+1}=l_{n}$ for $n=0,\dotsc,N$. Then we can cover $\Sigma$: $$\Sigma=\Sigma(l_{N})\cup\bigl(\cup_{n=0}^{N}\cup_{v\in S^*}\Sigma_v(l_n,h_n)\bigl).$$
Indeed this follows directly by applying the above lemma $N$ times. In the subsequent sections, we shall show that one can efficiently enumerate each set $\Sigma_v(l_n,h_n)$ by finding all points in the intersection $\Gamma_v\cap \mathcal E_v$ of a lattice $\Gamma_v$ with an ellipsoid $\mathcal E_v$.





If $h_0=(b,\dotsc,b)$ for $b$ the initial height bound, then
Lemma~\ref{lem:covering} gives  $$\Sigma=\Sigma(h_0), \quad \Sigma(h)=\Sigma(l,h)\cup \Sigma(l) \quad \textnormal{and} \quad \Sigma(l,h)=\cup_{v\in S^*}\Sigma_v(l,h).$$
Thus, after choosing a good sequence of lower and upper bounds (i.e. $l,h\in R^{S^*}$ with $0\leq l\leq h$) covering the whole space $0\leq h_0$, we are reduced to compute $\Sigma_v(l,h)$.
%
%
%\subsubsection{Refined coverings}
%
%Next, for any nonempty subset $V\subseteq S^*$ we denote by $\Sigma_V(l,v)$ the set of all $(x,y)\in \Sigma(l,h)$ such that $h_v(z)>l_v$ for each $v\in V$ and such that $h_v(z)\leq l_v$ for each $v\in S^*\setminus V$.
%
%
%\begin{lemma}[Refined coverings]
%It holds that $\Sigma(l,h)
%=\cup_{V}\Sigma_V(l,h)$ with the union taken over all nonempty subsets $V\subseteq S^*$.
%\end{lemma}
%\begin{proof}
%To see that $\Sigma(l,h)\subseteq \cup_V\Sigma_V(l,h)$, we take $(x,y)$ in $\cup_V\Sigma_V(l,h)$ and we let $V$ be the set of $v\in S^*$ with $h_v(z)>l_v$. Then the set $V$ is nonempty since $(h_v(z))\nleq l$ and thus $(x,y)$ lies in $\Sigma_V(l,h)$. This proves that $\Sigma(l,h)= \cup_V\Sigma_V(l,h)$.
%\end{proof}
%Suppose again we are given an initial bound $h_0$ with $\Sigma=\Sigma(h_0)$ and pairs $(l_n,h_n)\in \RR^{S^*}\times \RR^{S^*}$ with $0\leq l_n\leq h_n$ and $h_{n+1}=l_{n}$ for $n=0,\dotsc,N$. Then we can cover $\Sigma$: $$\Sigma=\Sigma(l_{N})\cup\bigl(\cup_{n=0}^{N}\cup_{V\subseteq S^*}\Sigma_V(l_n,h_n)\bigl).$$
%Indeed this follows directly by applying the above lemma $N$ times. In the subsequent sections, we shall show that one can efficiently enumerate each set $\Sigma_V(l_n,h_n)$ by finding all points in the intersection $\Gamma_V\cap \mathcal E_V$ of a lattice $\Gamma_V=\cap_{v\in V}\Gamma_v$ with an ellipsoid $\mathcal E_V$.
%
%Note these refined coverings are not that efficient in practice. We should consider the refined coverings from the elllog sieve.
%

%---------------------------------------------------------------------------------------------------------------------------------------------%
\subsection{Controlling the exponents in terms of the Weil Height}

We now work with the norm $\|\cdot \|_\infty$. However, below we shall give much more precise estimates (which are essentially optimal) to make the volumes of the involved ellipsoids as small as possible. 

\subsubsection{Bounding $\{a_1, \dots, a_{r}\}$}

We next consider the quadratic form $q_f=A^TD^2A$ on $\mathbb{Z}^{\nu}$ and where $D^2$ is a $\nu \times \nu$ diagonal matrix with diagonal entries $\lfloor\frac{\log(p_i)^2}{\log(2)^2}\rfloor$ for $p_i \in S$. We note that $\lfloor(\log(2))^2\rfloor = 0$, so if the diagonal entries of $D$ were set to $\lfloor\log(p_i)^2$ and $2\in S$, our matrix $D$ would not be invertible. In this case, when generating the lattice and ellipsoid, this will yield a matrix which is not positive-definite, meaning that we will not be able to apply Fincke-Pohst. With this in mind, the quadratic form $q_f$ is positive definite since $A$ is invertible. 

\begin{lemma}
For any solution $(x,y, n_1, \dots, n_{\nu}, a_1, \dots, a_r)$ of \eqref{Eq:main3}, we have 
\[\frac{\log(2)^2}{[K:\mathbb{Q}]}q_f(\mathbf{n}) = \frac{\log(2)^2}{[K:\mathbb{Q}]}\sum_{l = 1}^{\nu} \left\lfloor\frac{\log(p_l)^2}{\log(2)^2}\right\rfloor|u_l -r_l|^2 < \left(h\left(\frac{\delta_2}{\lambda}\right)\right)^2.\] 
\end{lemma}

\begin{proof}
Recall that $\mathbf{n} = (n_1, \dots, n_{\nu})^{\text{T}}$ and 
\[A\mathbf{n} = \mathbf{u} - \mathbf{r}.\]
Assume first that $2 \notin S$. Now
\begin{align*}
q_f(\mathbf{n})	
	& = (A\mathbf{n})^{\text{T}}D^2A\mathbf{n}\\
	& = \mathbf{n}^{\text{T}}A^{\text{T}}D^2A\mathbf{n}\\
	& = (\mathbf{u} - \mathbf{r})^{\text{T}}D^2(\mathbf{u} - \mathbf{r})\\
	& = \begin{pmatrix} u_1 - r_1&  \dots & u_{\nu} - r_{\nu} \end{pmatrix} 
		\begin{pmatrix} \lfloor\frac{\log(p_1)^2}{\log(2)^2}\rfloor& 0 & \dots & 0\\ 0 &\lfloor\frac{\log(p_2)^2}{\log(2)^2}\rfloor & \dots & 0\\
		\vdots & \vdots & \ddots & \vdots\\ 0 & 0 & \dots & \lfloor\frac{\log(p_{\nu})^2}{\log(2)^2}\rfloor\\ 
		\end{pmatrix} 
		\begin{pmatrix} u_1 - r_1 \\  \vdots \\ u_{\nu} - r_{\nu} \end{pmatrix} \\
	& =\left\lfloor\frac{\log(p_1)^2}{\log(2)^2}\right\rfloor(u_1-r_1)^2 + \cdots + \left\lfloor\frac{\log(p_{\nu})^2}{\log(2)^2}\right\rfloor(u_{\nu}-r_{\nu})^2\\
	& = \sum_{l = 1}^{\nu}\left\lfloor\frac{\log(p_l)^2}{\log(2)^2}\right\rfloor|u_l-r_l|^2.
\end{align*}
Hence it follows that
\[ q_f(\mathbf{n}) = \sum_{l = 1}^{\nu}\left\lfloor\frac{\log(p_l)^2}{\log(2)^2}\right\rfloor|u_l-r_l|^2.\]
Now, recall that
\[h\left(\frac{\delta_2}{\lambda}\right) = \frac{1}{[K:\mathbb{Q}]}\sum_{l = 1}^{\nu} \log(p_l)|u_l - r_l| + \frac{1}{[L:\mathbb{Q}]}\sum_{w :L \to \mathbb{C}} \log \max \left\{ \left|w\left(\frac{\delta_2}{\lambda}\right)\right|, 1\right\}.\]
It follows that
\begin{align*}
\frac{\log(2)^2}{[K:\mathbb{Q}]}q_f(\mathbf{n}) 
	& = \frac{\log(2)^2}{[K:\mathbb{Q}]}\sum_{l = 1}^{\nu} \left\lfloor\frac{\log(p_l)^2}{\log(2)^2}\right\rfloor|u_l -r_l|^2 \\
	& \leq \frac{\log(2)^2}{[K:\mathbb{Q}]}\sum_{l = 1}^{\nu}\frac{\log(p_l)^2}{\log(2)^2}|u_l -r_l|^2 \\
	& = \frac{1}{[K:\mathbb{Q}]}\sum_{l = 1}^{\nu}\log(p_l)^2|u_l -r_l|^2 \\
%	& < \frac{1}{[L:\mathbb{Q}]}\sum_{w :L \to \mathbb{C}} \left(\log \max \left\{ \left|w\left(\frac{\delta_2}{\lambda}\right)\right|, 1\right\}\right)^2 + \frac{1}{[K:\mathbb{Q}]}\sum_{l = 1}^{\nu} \left(\log(p_l)|u_l - r_l|\right)^2\\
%	& < \left(h\left(\frac{\delta_2}{\lambda}\right)\right)^2 
\end{align*}
since all terms are positive. 

If $2 \in S$, we have
\begin{align*}
q_f(\mathbf{n})	
	& = (A\mathbf{n})^{\text{T}}D^2A\mathbf{n}\\
	& = \mathbf{n}^{\text{T}}A^{\text{T}}D^2A\mathbf{n}\\
	& = (\mathbf{u} - \mathbf{r})^{\text{T}}D^2(\mathbf{u} - \mathbf{r})\\
	& = \begin{pmatrix} u_1 - r_1&  \dots & u_{\nu} - r_{\nu} \end{pmatrix} 
		\begin{pmatrix} \lfloor\frac{\log(2)^2}{\log(2)^2}\rfloor& 0 & \dots & 0\\ 0 &\lfloor\frac{\log(p_2)^2}{\log(2)^2}\rfloor & \dots & 0\\
		\vdots & \vdots & \ddots & \vdots\\ 0 & 0 & \dots & \lfloor\frac{\log(p_{\nu})^2}{\log(2)^2}\rfloor\\ 
		\end{pmatrix} 
		\begin{pmatrix} u_1 - r_1 \\  \vdots \\ u_{\nu} - r_{\nu} \end{pmatrix} \\
	& = \begin{pmatrix} u_1 - r_1&  \dots & u_{\nu} - r_{\nu} \end{pmatrix} 
		\begin{pmatrix} 1 & 0 & \dots & 0\\ 0 &\lfloor\frac{\log(p_2)^2}{\log(2)^2}\rfloor & \dots & 0\\
		\vdots & \vdots & \ddots & \vdots\\ 0 & 0 & \dots & \lfloor\frac{\log(p_{\nu})^2}{\log(2)^2}\rfloor\\ 
		\end{pmatrix} 
		\begin{pmatrix} u_1 - r_1 \\  \vdots \\ u_{\nu} - r_{\nu} \end{pmatrix} \\
	& = (u_1-r_1)^2  + \left\lfloor\frac{\log(p_2)^2}{\log(2)^2}\right\rfloor(u_2-r_2)^2 + \cdots + \left\lfloor\frac{\log(p_{\nu})^2}{\log(2)^2}\right\rfloor(u_{\nu}-r_{\nu})^2\\
	& = |u_1 - r_1|^2 + \sum_{l = 2}^{\nu}\left\lfloor\frac{\log(p_l)^2}{\log(2)^2}\right\rfloor|u_l-r_l|^2.
\end{align*}
Hence it follows that
\[ q_f(\mathbf{n}) = |u_1 - r_1|^2 + \sum_{l = 2}^{\nu}\left\lfloor\frac{\log(p_l)^2}{\log(2)^2}\right\rfloor|u_l-r_l|^2.\]
Now, recall that
\[h\left(\frac{\delta_2}{\lambda}\right) = \frac{1}{[K:\mathbb{Q}]}\sum_{l = 1}^{\nu} \log(p_l)|u_l - r_l| + \frac{1}{[L:\mathbb{Q}]}\sum_{w :L \to \mathbb{C}} \log \max \left\{ \left|w\left(\frac{\delta_2}{\lambda}\right)\right|, 1\right\}.\]
It follows that
\begin{align*}
\frac{\log(2)^2}{[K:\mathbb{Q}]}q_f(\mathbf{n}) 
	& = \frac{\log(2)^2}{[K:\mathbb{Q}]}\left( |u_1 - r_1|^2 + \sum_{l = 2}^{\nu} \left\lfloor\frac{\log(p_l)^2}{\log(2)^2}\right\rfloor|u_l -r_l|^2\right) \\
	& \leq \frac{\log(2)^2}{[K:\mathbb{Q}]}\left( |u_1 - r_1|^2 + \sum_{l = 2}^{\nu} \frac{\log(p_l)^2}{\log(2)^2}|u_l -r_l|^2\right) \\
	& = \frac{1}{[K:\mathbb{Q}]}\left( \log(2)^2|u_1 - r_1|^2 + \sum_{l = 2}^{\nu} \log(p_l)^2|u_l -r_l|^2\right) \\
	& = \frac{1}{[K:\mathbb{Q}]}\sum_{l = 1}^{\nu}\log(p_l)^2|u_l -r_l|^2
\end{align*}
since all terms are positive. 
\end{proof}

Now, 
\[\frac{\log(2)^2}{[K:\mathbb{Q}]}q_f(\mathbf{n}) =  \frac{\log(2)^2}{[K:\mathbb{Q}]} \sum_{l = 1}^{\nu}\left\lfloor\frac{\log(p_l)^2}{\log(2)^2}\right\rfloor|u_l-r_l|^2.\]

Take $\mathbf{h}\in\mathbb{R}^{r+\nu}$ such that $\mathbf{h}\geq \mathbf{0}$. Let $\mathbf{m} = (n_1, \dots, n_{\nu}, a_1, \dots, a_r) \in \mathbb{R}^{r + \nu}$ be any solution of \eqref{Eq:main3}. Denote by $h_{v}\left(\frac{\delta_2}{\lambda}\right)$ the $v^{\text{th}}$ entry of the solution vector
\[\left(\log(p_1)|u_1 - r_1|, \dots, \log(p_{\nu})|u_{\nu} - r_{\nu}|, \log \max \left\{ \left|w_1\left(\frac{\delta_2}{\lambda}\right)\right|, 1\right\}, \dots, \log \max \left\{ \left|w_n\left(\frac{\delta_2}{\lambda}\right)\right|, 1\right\}  \right)\] 
and suppose $h_v(z)\leq h_v$ for all $v\in \{1, \dots, r+\nu\}$. Then we deduce
\[\log(2)^2q_f(\mathbf{n}) = \log(2)^2\sum_{k = 1}^{\nu}\left\lfloor\frac{\log(p_k)^2}{\log(2)^2}\right\rfloor|u_k-r_k|^2 \leq \sum_{k = 1}^{\nu} \log(p_k)^2|u_k -r_k|^2 \leq \sum_{k = 1}^{\nu} h_k^2.\]

Recall that for the degree $3$ Thue-Mahler equation, either $r = 1$ or $r = 2$. Choose a set $I$ of embeddings $L \rightarrow \mathbb{C}$ of cardinality $r$. For $r = 1$, this is simply
\[R = \begin{pmatrix}
	\log\left|\left(\frac{\varepsilon_1^{(j)}}{\varepsilon_1^{(i_0)}}\right)^{\iota_1}\right| \end{pmatrix}.\]
Clearly, as long as we choose $\iota_1$ such that $\log\left|\left(\frac{\varepsilon_1^{(j)}}{\varepsilon_1^{(i_0)}}\right)^{\iota_1}\right| \neq 0$, that is $\left|\left(\frac{\varepsilon_1^{(j)}}{\varepsilon_1^{(i_0)}}\right)^{\iota_1}\right| \neq 1$, this matrix is invertible, with inverse matrix
\[R^{-1} = \begin{pmatrix} \frac{1}{\log\left|\left(\frac{\varepsilon_1^{(j)}}{\varepsilon_1^{(i_0)}}\right)^{\iota_1}\right|}\end{pmatrix}.\]

When $r = 2$, we let $I$ be the set of embeddings $L \to \mathbb{C}$ of cardinality $2$ such that for any $\alpha \in K$, it holds that $I\alpha^{(i_0)} \cup I\alpha^{(j)} = \Gal(L/\mathbb{Q})\alpha$. Such a set $I$ exists. Then we consider the $2 \times 2$ matrix
\[R = \begin{pmatrix}
	\log\left|\left(\frac{\varepsilon_1^{(j)}}{\varepsilon_1^{(i_0)}}\right)^{\iota_1}\right| &
	\log\left|\left(\frac{\varepsilon_2^{(j)}}{\varepsilon_2^{(i_0)}}\right)^{\iota_1}\right|\\
	\log\left|\left(\frac{\varepsilon_1^{(j)}}{\varepsilon_1^{(i_0)}}\right)^{\iota_2}\right| &
	\log\left|\left(\frac{\varepsilon_2^{(j)}}{\varepsilon_2^{(i_0)}}\right)^{\iota_2}\right|\\
	\end{pmatrix}.\]
Here, we let $I$ be the set of embeddings $L \to \mathbb{C}$ of cardinality $2$ such that for any $\alpha \in K$, it holds that $I\alpha^{(i_0)} \cup I\alpha^{(j)} = \Gal(L/\mathbb{Q})\alpha$. Such a set $I$ exists. 
\begin{lemma}
When $r = 2$, the matrix $R$ has an inverse
\[R^{-1} = \begin{pmatrix}
	\overline{r}_{11} & \overline{r}_{12} \\
	\overline{r}_{21} & \overline{r}_{22}
\end{pmatrix}.\]
\end{lemma}
	
\begin{proof}
See Rafael's proof.
\end{proof}

\textbf{Bounding $\{a_1, \dots, a_r\}$ when $r = 1$.}\\

Suppose first that $r = 1$. Now, for any solution $(x,y, a_1, n_1, \dots, n_{\nu})$ of \eqref{Eq:main3}, set
\[\vec{\varepsilon} = \begin{pmatrix} a_1 \end{pmatrix}.\]
Now, 
\begin{align*}
R\vec{\varepsilon}
	& = \begin{pmatrix}
		\log\left|\left(\frac{\varepsilon_1^{(j)}}{\varepsilon_1^{(i_0)}}\right)^{\iota_1}\right| \end{pmatrix}
		\begin{pmatrix} a_1\end{pmatrix}\\
	& = \begin{pmatrix} 
		a_1\log\left|\left(\frac{\varepsilon_1^{(j)}}{\varepsilon_1^{(i_0)}}\right)^{\iota_1}\right| \end{pmatrix}\\
	& = \begin{pmatrix} 
		\log\left|\left(\frac{\varepsilon_1^{(j)}}{\varepsilon_1^{(i_0)}}\right)^{\iota_1}\right|^{a_1} 	 		\end{pmatrix}\\
	& = \begin{pmatrix} 
		\log\left|\left(\frac{\varepsilon_1^{(j)}}{\varepsilon_1^{(i_0)}}\right)^{\iota_1 \ a_1}\right| 		 \end{pmatrix}.
\end{align*}
Since $R$ is invertible, we find
\begin{align*}
\vec{\varepsilon} = \begin{pmatrix} a_1\end{pmatrix} 
& =  R^{-1} \begin{pmatrix} \log\left|\left(\frac{\varepsilon_1^{(j)}}{\varepsilon_1^{(i_0)}}\right)^{\iota_1 \ 		a_1}\right| \end{pmatrix}\\
& = \begin{pmatrix} \frac{1}{\log\left|\left(\frac{\varepsilon_1^{(j)}}{\varepsilon_1^{(i_0)}}\right)^{\iota_1}\right|}	\end{pmatrix}
	\begin{pmatrix} \log\left|\left(\frac{\varepsilon_1^{(j)}}{\varepsilon_1^{(i_0)}}\right)^{\iota_1 \ a_1}\right| 	\end{pmatrix}\\
& = \begin{pmatrix} \overline{r}_{11} \end{pmatrix}\begin{pmatrix} \log\left|\left(\frac{\varepsilon_1^{(j)}}		{\varepsilon_1^{(i_0)}}\right)^{\iota_1 \ a_1}\right| \end{pmatrix}\\
& = \begin{pmatrix} \overline{r}_{11}\log\left|\left(\frac{\varepsilon_1^{(j)}}{\varepsilon_1^{(i_0)}}\right)^{\iota_1 \ 	a_1}\right|\end{pmatrix}.
\end{align*}

It follows that 
\[a_1 =  \overline{r}_{11}\log\left|\left(\frac{\varepsilon_1^{(j)}}{\varepsilon_1^{(i_0)}}\right)^{\iota_1 \ 	a_1}\right|.\]

Now, to estimate $|a_1|$, we begin to estimate the sum on the right hand side. For this, we consider
\[\frac{\delta_2}{\lambda}= \left( \frac{\varepsilon_1^{(j)}}{\varepsilon_1^{(i_0)}}\right)^{a_1}\prod_{i = 1}^{\nu} \left( \frac{\gamma_i^{(j)}}{\gamma_i^{(i_0)}}\right)^{n_i}.\]
For any embedding $\iota: L \to \mathbb{C}$, we have 
\[\left(\frac{\delta_2}{\lambda}\right)^{\iota} \prod_{i = 1}^{\nu} \left( \frac{\gamma_i^{(i_0)}}{\gamma_i^{(j)}}\right)^{\iota \ n_i} =  \left( \frac{\varepsilon_1^{(j)}}{\varepsilon_1^{(i_0)}}\right)^{\iota \ a_1}.\] 

Taking absolute values, we obtain
\[\left|\left(\frac{\delta_2}{\lambda}\right)^{\iota} \prod_{i = 1}^{\nu} \left( \frac{\gamma_i^{(i_0)}}{\gamma_i^{(j)}}\right)^{\iota \ n_i}\right| = \left|\left( \frac{\varepsilon_1^{(j)}}{\varepsilon_1^{(i_0)}}\right)^{\iota \ a_1}\right|,\]
so that
\begin{align*}
\log\left|\left( \frac{\varepsilon_1^{(j)}}{\varepsilon_1^{(i_0)}}\right)^{\iota \ a_1}\right|
	& = \log\left|\left(\frac{\delta_2}{\lambda}\right)^{\iota} \prod_{i = 1}^{\nu} \left( \frac{\gamma_i^{(i_0)}}{\gamma_i^{(j)}}\right)^{\iota \ n_i}\right|\\
	& = \log\left|\left(\frac{\delta_2}{\lambda}\right)^{\iota}\right| + \log\left| \prod_{i = 1}^{\nu} \left( \frac{\gamma_i^{(i_0)}}{\gamma_i^{(j)}}\right)^{\iota \ n_i}\right|\\
	& = \log\left|\left(\frac{\delta_2}{\lambda}\right)^{\iota}\right| - \log\left| \prod_{i = 1}^{\nu} \left( \frac{\gamma_i^{(j)}}{\gamma_i^{(i_0)}}\right)^{\iota \ n_i}\right|. 
\end{align*}

Hence, 
\begin{align*}
a_1	& =  \overline{r}_{11}\log\left|\left(\frac{\varepsilon_1^{(j)}}{\varepsilon_1^{(i_0)}}\right)^{\iota_1 \ 	a_1}\right| \\
	& = \overline{r}_{11}\left( \log\left|\left(\frac{\delta_2}{\lambda}\right)^{\iota_1}\right| - \log\left| \prod_{i = 1}^{\nu} \left( \frac{\gamma_i^{(j)}}{\gamma_i^{(i_0)}}\right)^{\iota_1 \ n_i}\right|\right)\\
	& = \overline{r}_{11}\left(\log\left|\left(\frac{\delta_2}{\lambda}\right)^{\iota_1}\right| - 
	n_1\log\left| \left( \frac{\gamma_1^{(j)}}{\gamma_1^{(i_0)}}\right)^{\iota_1}\right| - \cdots -n_{\nu}\log \left|\left( \frac{\gamma_{\nu}^{(j)}}{\gamma_{\nu}^{(i_0)}}\right)^{\iota_1}\right|\right).
\end{align*}

Recall that 
\[A\mathbf{n} = \mathbf{u} - \mathbf{r}\]
so
\[\mathbf{n} = A^{-1}(\mathbf{u} - \mathbf{r}).\]
If 
\[A = \begin{pmatrix}
	a_{11} & a_{12} & \dots & a_{1\nu} \\ a_{21} & a_{22} & \dots & a_{2\nu}\\
	\vdots & \vdots & \ddots & \vdots\\ a_{\nu 1} & a_{\nu 2} & \dots & a_{\nu\nu}\\ 
	\end{pmatrix},\]
then 
\[A^{-1} = \begin{pmatrix}
	\overline{a}_{11} & \overline{a}_{12} & \dots & \overline{a}_{1\nu} \\ \overline{a}_{21} & \overline{a}_{22} 	& \dots & \overline{a}_{2\nu}\\
	\vdots & \vdots & \ddots & \vdots\\ \overline{a}_{\nu 1} & \overline{a}_{\nu 2} & \dots 
	& \overline{a}_{\nu\nu}\\ 
	\end{pmatrix},\]
and 
\begin{align*}
\begin{pmatrix} n_1 \\  \vdots \\ n_{\nu} \end{pmatrix}	
	& = \mathbf{n}  = A^{-1}(\mathbf{u} - \mathbf{r})\\
	& = \begin{pmatrix}
	\overline{a}_{11} & \overline{a}_{12} & \dots & \overline{a}_{1\nu} \\ \overline{a}_{21} & \overline{a}_{22} 	& \dots & \overline{a}_{2\nu}\\
	\vdots & \vdots & \ddots & \vdots\\ \overline{a}_{\nu 1} & \overline{a}_{\nu 2} & \dots 
	& \overline{a}_{\nu\nu}\\ 
	\end{pmatrix}
	\begin{pmatrix} u_1-r_1 \\  \vdots \\ u_{\nu}-r_{\nu} \end{pmatrix}\\
	& = \begin{pmatrix} \overline{a}_{11}(u_1-r_1) + \cdots + \overline{a}_{1\nu}(u_{\nu}-r_{\nu}) \\  \vdots \\ \overline{a}_{\nu 1}(u_1-r_1) + \cdots + \overline{a}_{\nu \nu}(u_{\nu}-r_{\nu}) \end{pmatrix}\\
	& = \begin{pmatrix} \sum_{k=1}^{\nu} \overline{a}_{1k}(u_k-r_k)\\  \vdots \\ \sum_{k=1}^{\nu} \overline{a}_{\nu k}(u_k-r_k) \end{pmatrix}\\
\end{align*}
Now, 
\begin{align*}
a_1	& = \overline{r}_{11}\left(\log\left|\left(\frac{\delta_2}{\lambda}\right)^{\iota_1}\right| - 
	n_1\log\left| \left( \frac{\gamma_1^{(j)}}{\gamma_1^{(i_0)}}\right)^{\iota_1}\right| - \cdots -n_{\nu}\log \left|\left( \frac{\gamma_{\nu}^{(j)}}{\gamma_{\nu}^{(i_0)}}\right)^{\iota_1}\right|\right)\\
	& = \overline{r}_{11}\left(\log\left|\left(\frac{\delta_2}{\lambda}\right)^{\iota_1}\right| - 
	 \left(\overline{a}_{11}(u_1-r_1) + \cdots + \overline{a}_{1\nu}(u_{\nu}-r_{\nu})\right)\log\left| \left( \frac{\gamma_1^{(j)}}{\gamma_1^{(i_0)}}\right)^{\iota_1}\right| - \cdots \right. \\
	 &  \quad \quad \left. \cdots - (\overline{a}_{\nu 1}(u_1-r_1) + \cdots + \overline{a}_{\nu \nu}(u_{\nu}-r_{\nu}))\log \left|\left( \frac{\gamma_{\nu}^{(j)}}{\gamma_{\nu}^{(i_0)}}\right)^{\iota_1}\right|\right)\\
	 & = \overline{r}_{11}\left[\log\left|\left(\frac{\delta_2}{\lambda}\right)^{\iota_1}\right| - (u_1-r_1)\left(\overline{a}_{11}\log\left| \left( \frac{\gamma_1^{(j)}}{\gamma_1^{(i_0)}}\right)^{\iota_1}\right| + \cdots + \overline{a}_{\nu 1}\log\left| \left( \frac{\gamma_{\nu}^{(j)}}{\gamma_{\nu}^{(i_0)}}\right)^{\iota_1}\right|\right) - \cdots\right.\\
	 & \quad \quad \cdots - \left.(u_{\nu}-r_{\nu})\left(\overline{a}_{1\nu}\log\left| \left( \frac{\gamma_1^{(j)}}{\gamma_1^{(i_0)}}\right)^{\iota_1}\right| + \cdots + \overline{a}_{\nu \nu}\log\left| \left( \frac{\gamma_{\nu}^{(j)}}{\gamma_{\nu}^{(i_0)}}\right)^{\iota_1}\right|\right)\right]\\
	 & = \overline{r}_{11}\left[\log\left|\left(\frac{\delta_2}{\lambda}\right)^{\iota_1}\right| - (u_1-r_1)\alpha_{\gamma 1} - \cdots - (u_{\nu} - r_{\nu})\alpha_{\gamma \nu}\right]\\
	 & = \overline{r}_{11}\left(\log\left|\left(\frac{\delta_2}{\lambda}\right)^{\iota_1}\right| - \sum_{k=1}^{\nu}(u_k-r_k)\alpha_{\gamma k}\right)
\end{align*}
where
\[\alpha_{\gamma k} = \overline{a}_{1k}\log\left| \left( \frac{\gamma_1^{(j)}}{\gamma_1^{(i_0)}}\right)^{\iota_1}\right| + \cdots + \overline{a}_{\nu k}\log\left| \left( \frac{\gamma_{\nu}^{(j)}}{\gamma_{\nu}^{(i_0)}}\right)^{\iota_1}\right|.\]
Taking absolute values, this yields
\begin{align*}
|a_1|	& = \left|\overline{r}_{11}\left(\log\left|\left(\frac{\delta_2}{\lambda}\right)^{\iota_1}\right| - \sum_{k=1}^{\nu}(u_k-r_k)\alpha_{\gamma k}\right)\right|\\
	& \leq |\overline{r}_{11}|\left|\log\left|\left(\frac{\delta_2}{\lambda}\right)^{\iota_1}\right|\right| + \left| - \sum_{k=1}^{\nu}(u_k-r_k)\alpha_{\gamma k}\overline{r}_{11}\right|\\
	& \leq |\overline{r}_{11}|\left|\log\left|\left(\frac{\delta_2}{\lambda}\right)^{\iota_1}\right|\right| + \sum_{k=1}^{\nu}|u_k-r_k||\alpha_{\gamma k}\overline{r}_{11}|. 
\end{align*}

Now, if $\log\left|\left(\frac{\delta_2}{\lambda}\right)^{\iota_1}\right| \geq 0$ we obtain
\begin{align*}
|a_1|	& \leq |\overline{r}_{11}|\left|\log\left|\left(\frac{\delta_2}{\lambda}\right)^{\iota_1}\right|\right| + \sum_{k=1}^{\nu}|u_k-r_k||\alpha_{\gamma k}\overline{r}_{11}|\\
		& \leq |\overline{r}_{11}|\log\left|\left(\frac{\delta_2}{\lambda}\right)^{\iota_1}\right| + \sum_{k=1}^{\nu}|u_k-r_k||\alpha_{\gamma k}\overline{r}_{11}|\\
		& \leq |\overline{r}_{11}|\log\max\left\{\left|\left(\frac{\delta_2}{\lambda}\right)^{\iota_1}\right|, 1 \right\} + \sum_{k=1}^{\nu}|u_k-r_k||\alpha_{\gamma k}\overline{r}_{11}|\\
		& \leq \sum_{\sigma: L \to \mathbb{C}}|\overline{r}_{11}|\log\max \left\{\left|\sigma \left(\frac{\delta_2}{\lambda}\right)\right|,1\right\} + \sum_{k=1}^{\nu}|\alpha_{\gamma k}\overline{r}_{11}||u_k-r_k|.
\end{align*}

Recall that $\frac{\delta_2}{\lambda}$ is a quotient of elements which are conjugate to one another. In other words, taking the norm on $L$ of $\frac{\delta_2}{\lambda}$, we obtain $N\left(\frac{\delta_2}{\lambda}\right) = 1.$ On the other hand, by definition, we have 
\[1 = N\left(\frac{\delta_2}{\lambda}\right) = \prod_{\sigma: L \to \mathbb{C}} \sigma \left(\frac{\delta_2}{\lambda}\right).\]
Taking absolute values and logarithms, 
\[0 = \sum_{\sigma: L \to \mathbb{C}} \log\left|\sigma \left(\frac{\delta_2}{\lambda}\right)\right|\]
so that
\[-\log\left|\left(\frac{\delta_2}{\lambda}\right)^{\iota_1}\right| = -\log\left|\iota_1 \left(\frac{\delta_2}{\lambda}\right)\right| = \sum_{\shortstack{\footnotesize $\sigma: L \to \mathbb{C} $\\ \footnotesize $ \sigma \neq \iota_1$}} \log\left|\sigma \left(\frac{\delta_2}{\lambda}\right)\right|.\]

Hence if $\log\left|\left(\frac{\delta_2}{\lambda}\right)^{\iota_1}\right| < 0$ we obtain
\begin{align*}
|a_1|	& \leq |\overline{r}_{11}|\left|\log\left|\left(\frac{\delta_2}{\lambda}\right)^{\iota_1}\right|\right| + \sum_{k=1}^{\nu}|u_k-r_k||\alpha_{\gamma k}\overline{r}_{11}|\\
		& \leq |\overline{r}_{11}|\left(-\log\left|\left(\frac{\delta_2}{\lambda}\right)^{\iota_1}\right|\right) + \sum_{k=1}^{\nu}|u_k-r_k||\alpha_{\gamma k}\overline{r}_{11}|\\
		& = |\overline{r}_{11}|\left(\sum_{\shortstack{\footnotesize $\sigma: L \to \mathbb{C} $\\ \footnotesize $ \sigma \neq \iota_1$}} \log\left|\sigma \left(\frac{\delta_2}{\lambda}\right)\right|\right) + \sum_{k=1}^{\nu}|u_k-r_k||\alpha_{\gamma k}\overline{r}_{11}|\\
		& \leq |\overline{r}_{11}|\sum_{\shortstack{\footnotesize $\sigma: L \to \mathbb{C} $\\ \footnotesize $ \sigma \neq \iota_1$}}\log\max\left\{\left|\sigma\left(\frac{\delta_2}{\lambda}\right)\right|, 1 \right\} + \sum_{k=1}^{\nu}|u_k-r_k||\alpha_{\gamma k}\overline{r}_{11}|\\
		& \leq \sum_{\sigma: L \to \mathbb{C}}|\overline{r}_{11}|\log\max \left\{\left|\sigma \left(\frac{\delta_2}{\lambda}\right)\right|,1\right\} + \sum_{k=1}^{\nu}|\alpha_{\gamma k}\overline{r}_{11}||u_k-r_k|.
\end{align*}

In both cases, it follows that
\[|a_1| \leq \sum_{\sigma: L \to \mathbb{C}}|\overline{r}_{11}|\log\max \left\{\left|\sigma \left(\frac{\delta_2}{\lambda}\right)\right|,1\right\} + \sum_{k=1}^{\nu}|\alpha_{\gamma k}\overline{r}_{11}||u_k-r_k|\]
for 
\[\alpha_{\gamma k} = \overline{a}_{1k}\log\left| \left( \frac{\gamma_1^{(j)}}{\gamma_1^{(i_0)}}\right)^{\iota_1}\right| + \cdots + \overline{a}_{\nu k}\log\left| \left( \frac{\gamma_{\nu}^{(j)}}{\gamma_{\nu}^{(i_0)}}\right)^{\iota_1}\right|\]
Recall that
\[h\left(\frac{\delta_2}{\lambda}\right) = \frac{1}{[L:\mathbb{Q}]}\sum_{w :L \to \mathbb{C}} \log \max \left\{ \left|w\left(\frac{\delta_2}{\lambda}\right)\right|, 1\right\} + \frac{1}{[K:\mathbb{Q}]}\sum_{k = 1}^{\nu} \log(p_k)|u_k - r_k|.\]
Hence
\begin{align*}
|a_1|	& \leq \sum_{\sigma: L \to \mathbb{C}}|\overline{r}_{11}|\log\max \left\{\left|\sigma \left(\frac{\delta_2}{\lambda}\right)\right|,1\right\} + \sum_{k=1}^{\nu}|\alpha_{\gamma k}\overline{r}_{11}||u_k-r_k|\\
	& = \frac{1}{[L:\mathbb{Q}]}\sum_{\sigma: L \to \mathbb{C}}|\overline{r}_{11}|[L:\mathbb{Q}]\log\max \left\{\left|\sigma \left(\frac{\delta_2}{\lambda}\right)\right|,1\right\} + \\
	& \quad \quad + \frac{1}{[K:\mathbb{Q}]}\sum_{k=1}^{\nu}|\alpha_{\gamma k}\overline{r}_{11}|\frac{[K:\mathbb{Q}]}{\log(p_k)}\log(p_k)|u_k-r_k|\\
	& = \frac{1}{[L:\mathbb{Q}]}\sum_{\sigma :L \to \mathbb{C}} w_{\varepsilon \sigma}\log \max \left\{ \left|\sigma\left(\frac{\delta_2}{\lambda}\right)\right|, 1\right\} + \frac{1}{[K:\mathbb{Q}]}\sum_{k= 1}^{\nu} w_{\gamma k}\log(p_k)|u_k - r_k|
\end{align*}
where
\[w_{\varepsilon \sigma} = |\overline{r}_{11}|[L:\mathbb{Q}] \quad \text{ and } \quad 
	w_{\gamma k} = |\alpha_{\gamma k}\overline{r}_{11}|\frac{[K:\mathbb{Q}]}{\log(p_l)}\]
and
\[\alpha_{\gamma k} = \overline{a}_{1k}\log\left| \left( \frac{\gamma_1^{(j)}}{\gamma_1^{(i_0)}}\right)^{\iota_1}\right| + \cdots + \overline{a}_{\nu k}\log\left| \left( \frac{\gamma_{\nu}^{(j)}}{\gamma_{\nu}^{(i_0)}}\right)^{\iota_1}\right|\]
for $k = 1, \dots, \nu$. That is, 
\begin{align*}
|a_1|	& \leq \frac{1}{[L:\mathbb{Q}]}\sum_{\sigma :L \to \mathbb{C}} w_{\varepsilon \sigma}\log \max \left\{ \left|\sigma\left(\frac{\delta_2}{\lambda}\right)\right|, 1\right\} + \frac{1}{[K:\mathbb{Q}]}\sum_{k = 1}^{\nu} w_{\gamma k}\log(p_k)|u_k - r_k|\\
	& \leq \max\{w_{\varepsilon \sigma}, w_{\gamma 1}, \dots, w_{\gamma \nu}\} h\left(\frac{\delta_2}{\lambda}\right)
\end{align*}

\textbf{Bounding $\{a_1, \dots, a_r\}$ when $r = 2$.}\\

Now, suppose $r = 2$. For any solution $(x,y, n_1, \dots, n_{\nu}, a_1, a_2)$ of \eqref{Eq:main3}, set
\[\vec{\varepsilon} = \begin{pmatrix} a_1 & a_2 \end{pmatrix}^{\text{T}}.\]
Now, 
\begin{align*}
R\vec{\varepsilon}
	& = \begin{pmatrix}
		\log\left|\left(\frac{\varepsilon_1^{(j)}}{\varepsilon_1^{(i_0)}}\right)^{\iota_1}\right| &
		\log\left|\left(\frac{\varepsilon_2^{(j)}}{\varepsilon_2^{(i_0)}}\right)^{\iota_1}\right|\\
		\log\left|\left(\frac{\varepsilon_1^{(j)}}{\varepsilon_1^{(i_0)}}\right)^{\iota_2}\right| &
		\log\left|\left(\frac{\varepsilon_2^{(j)}}{\varepsilon_2^{(i_0)}}\right)^{\iota_2}\right|\\
		\end{pmatrix}
		\begin{pmatrix} a_1 \\ a_2 \end{pmatrix}\\
	& = \begin{pmatrix} 
		a_1\log\left|\left(\frac{\varepsilon_1^{(j)}}{\varepsilon_1^{(i_0)}}\right)^{\iota_1}\right|  
		+ a_2\log\left|\left(\frac{\varepsilon_2^{(j)}}{\varepsilon_2^{(i_0)}}\right)^{\iota_1}\right|\\
		a_1\log\left|\left(\frac{\varepsilon_1^{(j)}}{\varepsilon_1^{(i_0)}}\right)^{\iota_2}\right|  
		+ a_2\log\left|\left(\frac{\varepsilon_2^{(j)}}{\varepsilon_2^{(i_0)}}\right)^{\iota_2}\right|
	 	 \end{pmatrix}\\
	& = \begin{pmatrix} 
		\log\left(\left|\left(\frac{\varepsilon_1^{(j)}}{\varepsilon_1^{(i_0)}}\right)^{\iota_1}\right|^{a_1} \cdot 
		 \left|\left(\frac{\varepsilon_2^{(j)}}{\varepsilon_2^{(i_0)}}\right)^{\iota_1}\right|^{a_2}\right)\\
		\log\left(\left|\left(\frac{\varepsilon_1^{(j)}}{\varepsilon_1^{(i_0)}}\right)^{\iota_2}\right|^{a_1} \cdot 
		 \left|\left(\frac{\varepsilon_2^{(j)}}{\varepsilon_2^{(i_0)}}\right)^{\iota_2}\right|^{a_2}\right)	 	 		\end{pmatrix}\\
	& = \begin{pmatrix} 
		\log\left|\left(\frac{\varepsilon_1^{(j)}}{\varepsilon_1^{(i_0)}}\right)^{\iota_1 \ a_1} \cdot 
		 \left(\frac{\varepsilon_2^{(j)}}{\varepsilon_2^{(i_0)}}\right)^{\iota_1 \ a_2}\right| \\ 
		\log\left|\left(\frac{\varepsilon_1^{(j)}}{\varepsilon_1^{(i_0)}}\right)^{\iota_2\ a_2} \cdot 
		 \left(\frac{\varepsilon_2^{(j)}}{\varepsilon_2^{(i_0)}}\right)^{\iota_2 \ a_2}\right|
		 \end{pmatrix}.
\end{align*}
Now, since $R$ is invertible with $R^{-1} = (\overline{r}_{nm})$, we find
\begin{align*}
\vec{\varepsilon} = \begin{pmatrix} a_1 \\ a_2 \end{pmatrix} 
	& =  R^{-1} \begin{pmatrix} 
		\log\left|\left(\frac{\varepsilon_1^{(j)}}{\varepsilon_1^{(i_0)}}\right)^{\iota_1 \ a_1} \cdot 
		 \left(\frac{\varepsilon_2^{(j)}}{\varepsilon_2^{(i_0)}}\right)^{\iota_1 \ a_2}\right| \\ 
		\log\left|\left(\frac{\varepsilon_1^{(j)}}{\varepsilon_1^{(i_0)}}\right)^{\iota_2\ a_2} \cdot 
		 \left(\frac{\varepsilon_2^{(j)}}{\varepsilon_2^{(i_0)}}\right)^{\iota_2 \ a_2}\right|
		 \end{pmatrix}\\
	& = \begin{pmatrix}
		\overline{r}_{11} & \overline{r}_{12} \\
		\overline{r}_{21} & \overline{r}_{22}
		\end{pmatrix}
		\begin{pmatrix} 
		\log\left|\left(\frac{\varepsilon_1^{(j)}}{\varepsilon_1^{(i_0)}}\right)^{\iota_1 \ a_1} \cdot 
		 \left(\frac{\varepsilon_2^{(j)}}{\varepsilon_2^{(i_0)}}\right)^{\iota_1 \ a_2}\right| \\ 
		\log\left|\left(\frac{\varepsilon_1^{(j)}}{\varepsilon_1^{(i_0)}}\right)^{\iota_2\ a_2} \cdot 
		 \left(\frac{\varepsilon_2^{(j)}}{\varepsilon_2^{(i_0)}}\right)^{\iota_2 \ a_2}\right|
		 \end{pmatrix}\\
	& = \begin{pmatrix} 
		\overline{r}_{11}\log\left|\left(\frac{\varepsilon_1^{(j)}}{\varepsilon_1^{(i_0)}}\right)^{\iota_1 \ a_1} 		\cdot \left(\frac{\varepsilon_2^{(j)}}{\varepsilon_2^{(i_0)}}\right)^{\iota_1 \ a_2}\right| + 
		\overline{r}_{12}\log\left|\left(\frac{\varepsilon_1^{(j)}}{\varepsilon_1^{(i_0)}}\right)^{\iota_2\ a_1}
		\cdot \left(\frac{\varepsilon_2^{(j)}}{\varepsilon_2^{(i_0)}}\right)^{\iota_2 \ a_2}\right| \\
		\overline{r}_{21}\log\left|\left(\frac{\varepsilon_1^{(j)}}{\varepsilon_1^{(i_0)}}\right)^{\iota_1 \ a_1} 		\cdot \left(\frac{\varepsilon_2^{(j)}}{\varepsilon_2^{(i_0)}}\right)^{\iota_1 \ a_2}\right| +
		\overline{r}_{22}\log\left|\left(\frac{\varepsilon_1^{(j)}}{\varepsilon_1^{(i_0)}}\right)^{\iota_2\ a_1}
		\cdot \left(\frac{\varepsilon_2^{(j)}}{\varepsilon_2^{(i_0)}}\right)^{\iota_2 \ a_2}\right| \\
		\end{pmatrix}.
\end{align*}
and so we have
\[a_l = \overline{r}_{l1}\log\left|\left(\frac{\varepsilon_1^{(j)}}{\varepsilon_1^{(i_0)}}\right)^{\iota_1 \ a_1} 		\cdot \left(\frac{\varepsilon_2^{(j)}}{\varepsilon_2^{(i_0)}}\right)^{\iota_1 \ a_2}\right| + 
	\overline{r}_{l2}\log\left|\left(\frac{\varepsilon_1^{(j)}}{\varepsilon_1^{(i_0)}}\right)^{\iota_2\ a_1}
	\cdot \left(\frac{\varepsilon_2^{(j)}}{\varepsilon_2^{(i_0)}}\right)^{\iota_2 \ a_2}\right|\]
for $l = 1,2$. 

Now, to estimate $|a_l|$, we begin to estimate the sum on the right hand side. For this, we consider
\[\frac{\delta_2}{\lambda}= \left( \frac{\varepsilon_1^{(j)}}{\varepsilon_1^{(i_0)}}\right)^{a_1}\left( \frac{\varepsilon_2^{(j)}}{\varepsilon_2^{(i_0)}}\right)^{a_2}\prod_{i = 1}^{\nu} \left( \frac{\gamma_i^{(j)}}{\gamma_i^{(i_0)}}\right)^{n_i}.\]
For any embedding $\iota: L \to \mathbb{C}$, we have 
\[\left(\frac{\delta_2}{\lambda}\right)^{\iota} \prod_{i = 1}^{\nu} \left( \frac{\gamma_i^{(i_0)}}{\gamma_i^{(j)}}\right)^{\iota \ n_i} =  \left( \frac{\varepsilon_1^{(j)}}{\varepsilon_1^{(i_0)}}\right)^{\iota \ a_1}\left( \frac{\varepsilon_2^{(j)}}{\varepsilon_2^{(i_0)}}\right)^{\iota \ a_2}.\] 

Taking absolute values, we obtain
\[\left|\left(\frac{\delta_2}{\lambda}\right)^{\iota} \prod_{i = 1}^{\nu} \left( \frac{\gamma_i^{(i_0)}}{\gamma_i^{(j)}}\right)^{\iota \ n_i}\right| = \left|\left( \frac{\varepsilon_1^{(j)}}{\varepsilon_1^{(i_0)}}\right)^{\iota \ a_1}\left( \frac{\varepsilon_2^{(j)}}{\varepsilon_2^{(i_0)}}\right)^{\iota \ a_2}\right|,\]
so that
\begin{align*}
\log\left|\left( \frac{\varepsilon_1^{(j)}}{\varepsilon_1^{(i_0)}}\right)^{\iota \ a_1}\left( \frac{\varepsilon_2^{(j)}}{\varepsilon_2^{(i_0)}}\right)^{\iota \ a_2}\right|
	& = \log\left|\left(\frac{\delta_2}{\lambda}\right)^{\iota} \prod_{i = 1}^{\nu} \left( \frac{\gamma_i^{(i_0)}}{\gamma_i^{(j)}}\right)^{\iota \ n_i}\right|\\
	& = \log\left|\left(\frac{\delta_2}{\lambda}\right)^{\iota}\right| + \log\left| \prod_{i = 1}^{\nu} \left( \frac{\gamma_i^{(i_0)}}{\gamma_i^{(j)}}\right)^{\iota \ n_i}\right|\\
	& = \log\left|\left(\frac{\delta_2}{\lambda}\right)^{\iota}\right| - \log\left| \prod_{i = 1}^{\nu} \left( \frac{\gamma_i^{(j)}}{\gamma_i^{(i_0)}}\right)^{\iota \ n_i}\right|. 
\end{align*}

Hence, for $l =1,2$,
\begin{align*}
a_l	& = \overline{r}_{l1}\log\left|\left(\frac{\varepsilon_1^{(j)}}{\varepsilon_1^{(i_0)}}\right)^{\iota_1 \ a_1} 		\cdot \left(\frac{\varepsilon_2^{(j)}}{\varepsilon_2^{(i_0)}}\right)^{\iota_1 \ a_2}\right| + 
	\overline{r}_{l2}\log\left|\left(\frac{\varepsilon_1^{(j)}}{\varepsilon_1^{(i_0)}}\right)^{\iota_2\ a_1}
	\cdot \left(\frac{\varepsilon_2^{(j)}}{\varepsilon_2^{(i_0)}}\right)^{\iota_2 \ a_2}\right|\\
	& = \overline{r}_{l1}\left( \log\left|\left(\frac{\delta_2}{\lambda}\right)^{\iota_1}\right| - \log\left| \prod_{i = 1}^{\nu} \left( \frac{\gamma_i^{(j)}}{\gamma_i^{(i_0)}}\right)^{\iota_1 \ n_i}\right|\right) + \\ 
	& \quad \quad + \overline{r}_{l2}\left( \log\left|\left(\frac{\delta_2}{\lambda}\right)^{\iota_2}\right| - \log\left| \prod_{i = 1}^{\nu} \left( \frac{\gamma_i^{(j)}}{\gamma_i^{(i_0)}}\right)^{\iota_2 \ n_i}\right|\right)\\
	& = \overline{r}_{l1}\left(\log\left|\left(\frac{\delta_2}{\lambda}\right)^{\iota_1}\right| - 
	n_1\log\left| \left( \frac{\gamma_1^{(j)}}{\gamma_1^{(i_0)}}\right)^{\iota_1}\right| - \cdots -n_{\nu}\log \left|\left( \frac{\gamma_{\nu}^{(j)}}{\gamma_{\nu}^{(i_0)}}\right)^{\iota_1}\right|\right) + \\
	& \quad \quad + \overline{r}_{l2}\left(\log\left|\left(\frac{\delta_2}{\lambda}\right)^{\iota_2}\right| - 
	n_1\log\left| \left( \frac{\gamma_1^{(j)}}{\gamma_1^{(i_0)}}\right)^{\iota_2}\right| - \cdots -n_{\nu}\log \left|\left( \frac{\gamma_{\nu}^{(j)}}{\gamma_{\nu}^{(i_0)}}\right)^{\iota_2}\right|\right).
\end{align*}

Now, for $l = 1,2$, 
\begin{align*}
a_l 	& = \overline{r}_{l1}\left(\log\left|\left(\frac{\delta_2}{\lambda}\right)^{\iota_1}\right| - 
	n_1\log\left| \left( \frac{\gamma_1^{(j)}}{\gamma_1^{(i_0)}}\right)^{\iota_1}\right| - \cdots -n_{\nu}\log \left|\left( \frac{\gamma_{\nu}^{(j)}}{\gamma_{\nu}^{(i_0)}}\right)^{\iota_1}\right|\right) + \\
	& \quad \quad + \overline{r}_{l2}\left(\log\left|\left(\frac{\delta_2}{\lambda}\right)^{\iota_2}\right| - 
	n_1\log\left| \left( \frac{\gamma_1^{(j)}}{\gamma_1^{(i_0)}}\right)^{\iota_2}\right| - \cdots -n_{\nu}\log \left|\left( \frac{\gamma_{\nu}^{(j)}}{\gamma_{\nu}^{(i_0)}}\right)^{\iota_2}\right|\right)\\
	& = \overline{r}_{l1}\log\left|\left(\frac{\delta_2}{\lambda}\right)^{\iota_1}\right| + \overline{r}_{l2}\log\left|\left(\frac{\delta_2}{\lambda}\right)^{\iota_2}\right| + \\
	& \quad \quad - n_1\left(\overline{r}_{l1} \log\left| \left( \frac{\gamma_1^{(j)}}{\gamma_1^{(i_0)}}\right)^{\iota_1}\right|+ \overline{r}_{l2}\log\left| \left( \frac{\gamma_1^{(j)}}{\gamma_1^{(i_0)}}\right)^{\iota_2}\right|\right) - \cdots \\
	& \quad \quad \cdots - n_{\nu} \left(\overline{r}_{l1}\log \left|\left( \frac{\gamma_{\nu}^{(j)}}{\gamma_{\nu}^{(i_0)}}\right)^{\iota_1}\right| + \overline{r}_{l2}\log \left|\left( \frac{\gamma_{\nu}^{(j)}}{\gamma_{\nu}^{(i_0)}}\right)^{\iota_2}\right| \right)  \\
	& = \overline{r}_{l1}\log\left|\left(\frac{\delta_2}{\lambda}\right)^{\iota_1}\right| + \overline{r}_{l2}\log\left|\left(\frac{\delta_2}{\lambda}\right)^{\iota_2}\right| - n_1\beta_{\gamma l 1} - \dots - n_{\nu}\beta_{\gamma l \nu},
\end{align*}
where
\[\beta_{\gamma l k} = \left(\overline{r}_{l1} \log\left| \left( \frac{\gamma_k^{(j)}}{\gamma_k^{(i_0)}}\right)^{\iota_1}\right|+ \overline{r}_{l2}\log\left| \left( \frac{\gamma_k^{(j)}}{\gamma_k^{(i_0)}}\right)^{\iota_2}\right|\right)\]
for $k = 1, \dots, \nu$. Recall that 
\[A\mathbf{n} = \mathbf{u} - \mathbf{r}\]
so
\[\mathbf{n} = A^{-1}(\mathbf{u} - \mathbf{r}).\]
If 
\[A = \begin{pmatrix}
	a_{11} & a_{12} & \dots & a_{1\nu} \\ a_{21} & a_{22} & \dots & a_{2\nu}\\
	\vdots & \vdots & \ddots & \vdots\\ a_{\nu 1} & a_{\nu 2} & \dots & a_{\nu\nu}\\ 
	\end{pmatrix},\]
then 
\[A^{-1} = \begin{pmatrix}
	\overline{a}_{11} & \overline{a}_{12} & \dots & \overline{a}_{1\nu} \\ \overline{a}_{21} & \overline{a}_{22} 	& \dots & \overline{a}_{2\nu}\\
	\vdots & \vdots & \ddots & \vdots\\ \overline{a}_{\nu 1} & \overline{a}_{\nu 2} & \dots 
	& \overline{a}_{\nu\nu}\\ 
	\end{pmatrix},\]
and 
\begin{align*}
\begin{pmatrix} n_1 \\  \vdots \\ n_{\nu} \end{pmatrix}	
	& = \mathbf{n}  = A^{-1}(\mathbf{u} - \mathbf{r})\\
	& = \begin{pmatrix}
	\overline{a}_{11} & \overline{a}_{12} & \dots & \overline{a}_{1\nu} \\ \overline{a}_{21} & \overline{a}_{22} 	& \dots & \overline{a}_{2\nu}\\
	\vdots & \vdots & \ddots & \vdots\\ \overline{a}_{\nu 1} & \overline{a}_{\nu 2} & \dots 
	& \overline{a}_{\nu\nu}\\ 
	\end{pmatrix}
	\begin{pmatrix} u_1-r_1 \\  \vdots \\ u_{\nu}-r_{\nu} \end{pmatrix}\\
	& = \begin{pmatrix} \overline{a}_{11}(u_1-r_1) + \cdots + \overline{a}_{1\nu}(u_{\nu}-r_{\nu}) \\  \vdots \\ \overline{a}_{\nu 1}(u_1-r_1) + \cdots + \overline{a}_{\nu \nu}(u_{\nu}-r_{\nu}) \end{pmatrix}\\
	& = \begin{pmatrix} \sum_{k=1}^{\nu} \overline{a}_{1k}(u_k-r_k)\\  \vdots \\ \sum_{k=1}^{\nu} \overline{a}_{\nu k}(u_k-r_k) \end{pmatrix}
\end{align*}

Now, 
\begin{align*}
a_l 	& = \overline{r}_{l1}\log\left|\left(\frac{\delta_2}{\lambda}\right)^{\iota_1}\right| + \overline{r}_{l2}\log\left|\left(\frac{\delta_2}{\lambda}\right)^{\iota_2}\right| - n_1\beta_{\gamma l 1} - \dots - n_{\nu}\beta_{\gamma l \nu}\\
	& = \overline{r}_{l1}\log\left|\left(\frac{\delta_2}{\lambda}\right)^{\iota_1}\right| + \overline{r}_{l2}\log\left|\left(\frac{\delta_2}{\lambda}\right)^{\iota_2}\right| - \left(\overline{a}_{11}(u_1-r_1) + \cdots + \overline{a}_{1\nu}(u_{\nu}-r_{\nu})\right)\beta_{\gamma l 1} - \cdots \\
	&  \quad \quad \cdots - (\overline{a}_{\nu 1}(u_1-r_1) + \cdots + \overline{a}_{\nu \nu}(u_{\nu}-r_{\nu}))\beta_{\gamma l \nu} \\
	& = \overline{r}_{l1}\log\left|\left(\frac{\delta_2}{\lambda}\right)^{\iota_1}\right| + \overline{r}_{l2}\log\left|\left(\frac{\delta_2}{\lambda}\right)^{\iota_2}\right| -(u_1-r_1)(\overline{a}_{11}\beta_{\gamma l 1} + \cdots + \overline{a}_{\nu 1}\beta_{\gamma l \nu}) - \cdots\\
	& \quad \quad \cdots - (u_{\nu}-r_{\nu})\left(\overline{a}_{1\nu}\beta_{\gamma l 1} + \cdots + \overline{a}_{\nu \nu}\beta_{\gamma l \nu}\right)\\
	& = \overline{r}_{l1}\log\left|\left(\frac{\delta_2}{\lambda}\right)^{\iota_1}\right| + \overline{r}_{l2}\log\left|\left(\frac{\delta_2}{\lambda}\right)^{\iota_2}\right| - (u_1-r_1)\alpha_{\gamma l 1} - \cdots - (u_{\nu} - r_{\nu})\alpha_{\gamma l \nu}\\
	& = \overline{r}_{l1}\log\left|\left(\frac{\delta_2}{\lambda}\right)^{\iota_1}\right| + \overline{r}_{l2}\log\left|\left(\frac{\delta_2}{\lambda}\right)^{\iota_2}\right| - \sum_{k=1}^{\nu}(u_k-r_k)\alpha_{\gamma l k}
\end{align*}
where
\[\alpha_{\gamma l k} = \overline{a}_{1k}\beta_{\gamma l 1} + \cdots + \overline{a}_{\nu k}\beta_{\gamma l \nu}\]
and
\[\beta_{\gamma l k} = \left(\overline{r}_{l1} \log\left| \left( \frac{\gamma_k^{(j)}}{\gamma_k^{(i_0)}}\right)^{\iota_1}\right|+ \overline{r}_{l2}\log\left| \left( \frac{\gamma_k^{(j)}}{\gamma_k^{(i_0)}}\right)^{\iota_2}\right|\right)\]
for $k = 1, \dots, \nu$.

Further, recall that $\frac{\delta_2}{\lambda}$ is a quotient of elements which are conjugate to one another. In other words, taking the norm on $L$ of $\frac{\delta_2}{\lambda}$, we obtain $N\left(\frac{\delta_2}{\lambda}\right) = 1.$ On the other hand, by definition, we have 
\[1 = N\left(\frac{\delta_2}{\lambda}\right) = \prod_{\sigma: L \to \mathbb{C}} \sigma \left(\frac{\delta_2}{\lambda}\right).\]
Taking absolute values and logarithms, 
\[0 = \sum_{\sigma: L \to \mathbb{C}} \log\left|\sigma \left(\frac{\delta_2}{\lambda}\right)\right|\]
so that
\[-\log\left|\left(\frac{\delta_2}{\lambda}\right)^{\iota}\right| = -\log\left|\iota \left(\frac{\delta_2}{\lambda}\right)\right| = \sum_{\shortstack{\footnotesize $\sigma: L \to \mathbb{C} $\\ \footnotesize $ \sigma \neq \iota$}} \log\left|\sigma \left(\frac{\delta_2}{\lambda}\right)\right|.\]

Taking absolute values, this yields
\begin{align*}
|a_l| 	& = \left|\overline{r}_{l1}\log\left|\left(\frac{\delta_2}{\lambda}\right)^{\iota_1}\right| + \overline{r}_{l2}\log\left|\left(\frac{\delta_2}{\lambda}\right)^{\iota_2}\right| - \sum_{k=1}^{\nu}(u_k-r_k)\alpha_{\gamma l k}\right|\\
	& \leq |\overline{r}_{l1}|\left|\log\left|\left(\frac{\delta_2}{\lambda}\right)^{\iota_1}\right|\right| + |\overline{r}_{l2}|\left|\log\left|\left(\frac{\delta_2}{\lambda}\right)^{\iota_2}\right|\right| + \left| - \sum_{k=1}^{\nu}(u_k-r_k)\alpha_{\gamma l k}\right|\\
	& \leq |\overline{r}_{l1}|\left|\log\left|\left(\frac{\delta_2}{\lambda}\right)^{\iota_1}\right|\right| + |\overline{r}_{l2}|\left|\log\left|\left(\frac{\delta_2}{\lambda}\right)^{\iota_2}\right|\right| + \sum_{k=1}^{\nu}|u_k-r_k||\alpha_{\gamma l k}|
\end{align*}
where
\[\alpha_{\gamma l k} = \overline{a}_{1k}\beta_{\gamma l 1} + \cdots + \overline{a}_{\nu k}\beta_{\gamma l \nu}\]
and
\[\beta_{\gamma l k} = \left(\overline{r}_{l1} \log\left| \left( \frac{\gamma_k^{(j)}}{\gamma_k^{(i_0)}}\right)^{\iota_1}\right|+ \overline{r}_{l2}\log\left| \left( \frac{\gamma_k^{(j)}}{\gamma_k^{(i_0)}}\right)^{\iota_2}\right|\right)\]
for $k = 1, \dots, \nu$.

Suppose $\log\left|\left(\frac{\delta_2}{\lambda}\right)^{\iota_1}\right| \geq 0$ and $\log\left|\left(\frac{\delta_2}{\lambda}\right)^{\iota_2}\right| \geq 0$. Then, we obtain
\begin{align*}
|a_l| 	& \leq |\overline{r}_{l1}|\left|\log\left|\left(\frac{\delta_2}{\lambda}\right)^{\iota_1}\right|\right| + |\overline{r}_{l2}|\left|\log\left|\left(\frac{\delta_2}{\lambda}\right)^{\iota_2}\right|\right| + \sum_{k=1}^{\nu}|u_k-r_k||\alpha_{\gamma l k}|\\
	& = |\overline{r}_{l1}|\log\left|\left(\frac{\delta_2}{\lambda}\right)^{\iota_1}\right| + |\overline{r}_{l2}|\log\left|\left(\frac{\delta_2}{\lambda}\right)^{\iota_2}\right| + \sum_{k=1}^{\nu}|u_k-r_k||\alpha_{\gamma l k}|\\
	& \leq \max\{|\overline{r}_{l1}|, |\overline{r}_{l2}|\}|\log\max\left\{\left|\left(\frac{\delta_2}{\lambda}\right)^{\iota_1}\right|,1\right\} + \\
	& \quad \quad +\max\{|\overline{r}_{l1}|, |\overline{r}_{l2}|\}\log\max\left\{\left|\left(\frac{\delta_2}{\lambda}\right)^{\iota_2}\right|,1\right\} + \sum_{k=1}^{\nu}|u_k-r_k||\alpha_{\gamma l k}|\\
	&  \leq \sum_{w: L \to \mathbb{C}}\max\{|\overline{r}_{l1}|, |\overline{r}_{l2}|\}|\log\max\left\{\left|w\left(\frac{\delta_2}{\lambda}\right)\right|,1\right\} + \sum_{k=1}^{\nu}|u_k-r_k||\alpha_{\gamma l k}|.
\end{align*}

Alternatively, suppose that both $\log\left|\left(\frac{\delta_2}{\lambda}\right)^{\iota_1}\right| < 0$ and $\log\left|\left(\frac{\delta_2}{\lambda}\right)^{\iota_2}\right| < 0$. Then
\begin{align*}
|a_l| 	& \leq |\overline{r}_{l1}|\left|\log\left|\left(\frac{\delta_2}{\lambda}\right)^{\iota_1}\right|\right| + |\overline{r}_{l2}|\left|\log\left|\left(\frac{\delta_2}{\lambda}\right)^{\iota_2}\right|\right| + \sum_{k=1}^{\nu}|u_k-r_k||\alpha_{\gamma l k}|\\
	& = |\overline{r}_{l1}|\left(-\log\left|\left(\frac{\delta_2}{\lambda}\right)^{\iota_1}\right|\right) + |\overline{r}_{l2}|\left(-\log\left|\left(\frac{\delta_2}{\lambda}\right)^{\iota_2}\right|\right) + \sum_{k=1}^{\nu}|u_k-r_k||\alpha_{\gamma l k}|\\
	& = |\overline{r}_{l1}|\sum_{\shortstack{\footnotesize $\sigma: L \to \mathbb{C} $\\ \footnotesize $ \sigma \neq \iota_1$}} \log\left|\sigma \left(\frac{\delta_2}{\lambda}\right)\right|+ |\overline{r}_{l2}|\left(-\log\left|\left(\frac{\delta_2}{\lambda}\right)^{\iota_2}\right|\right) + \sum_{k=1}^{\nu}|u_k-r_k||\alpha_{\gamma l k}|\\
	& \leq \max\{|\overline{r}_{l1}|, |\overline{r}_{l2}|\}\sum_{\shortstack{\footnotesize $\sigma: L \to \mathbb{C} $\\ \footnotesize $ \sigma \neq \iota_1$}} \log\left|\sigma \left(\frac{\delta_2}{\lambda}\right)\right| +\\
	& \quad \quad +\max\{|\overline{r}_{l1}|, |\overline{r}_{l2}|\}\left(-\log\left|\left(\frac{\delta_2}{\lambda}\right)^{\iota_2}\right|\right) + \sum_{k=1}^{\nu}|u_k-r_k||\alpha_{\gamma l k}|\\
	& \leq \max\{|\overline{r}_{l1}|, |\overline{r}_{l2}|\}\sum_{\shortstack{\footnotesize $w: L \to \mathbb{C} $\\ \footnotesize $ \sigma \neq \iota_1, \iota_2$}} \log\left|\sigma \left(\frac{\delta_2}{\lambda}\right)\right|  + \sum_{k=1}^{\nu}|u_k-r_k||\alpha_{\gamma l k}|\\
	&  \leq \sum_{w: L \to \mathbb{C}}\max\{|\overline{r}_{l1}|, |\overline{r}_{l2}|\}|\log\max\left\{\left|w\left(\frac{\delta_2}{\lambda}\right)\right|,1\right\} + \sum_{k=1}^{\nu}|u_k-r_k||\alpha_{\gamma l k}|.
\end{align*}

Lastly, if, without loss of generality, we have $\log\left|\left(\frac{\delta_2}{\lambda}\right)^{\iota_1}\right| < 0$ and $\log\left|\left(\frac{\delta_2}{\lambda}\right)^{\iota_2}\right| \geq 0$, then
\begin{align*}
|a_l| 	& \leq |\overline{r}_{l1}|\left|\log\left|\left(\frac{\delta_2}{\lambda}\right)^{\iota_1}\right|\right| + |\overline{r}_{l2}|\left|\log\left|\left(\frac{\delta_2}{\lambda}\right)^{\iota_2}\right|\right| + \sum_{k=1}^{\nu}|u_k-r_k||\alpha_{\gamma l k}|\\
	& = |\overline{r}_{l1}|\left(-\log\left|\left(\frac{\delta_2}{\lambda}\right)^{\iota_1}\right|\right) + |\overline{r}_{l2}|\log\left|\left(\frac{\delta_2}{\lambda}\right)^{\iota_2}\right| + \sum_{k=1}^{\nu}|u_k-r_k||\alpha_{\gamma l k}|\\
	& = |\overline{r}_{l1}|\sum_{\shortstack{\footnotesize $\sigma: L \to \mathbb{C} $\\ \footnotesize $ \sigma \neq \iota_1$}} \log\left|\sigma \left(\frac{\delta_2}{\lambda}\right)\right|+ |\overline{r}_{l2}|\log\left|\left(\frac{\delta_2}{\lambda}\right)^{\iota_2}\right| + \sum_{k=1}^{\nu}|u_k-r_k||\alpha_{\gamma l k}|\\
	& \leq \max\{|\overline{r}_{l1}|, |\overline{r}_{l2}|\}\sum_{w: L \to \mathbb{C}} \log \max \left\{\left|w\left(\frac{\delta_2}{\lambda}\right)\right|, 1\right\} +  |\overline{r}_{l2}|\log\left|\left(\frac{\delta_2}{\lambda}\right)^{\iota_2}\right| +\sum_{k=1}^{\nu}|u_k-r_k||\alpha_{\gamma l k}|\\
	& = \sum_{w: L \to \mathbb{C}}\max\{|\overline{r}_{l1}|, |\overline{r}_{l2}|\}|\log\max\left\{\left|w\left(\frac{\delta_2}{\lambda}\right)\right|,1\right\} + |\overline{r}_{l1}|\log\max\left\{\left|\iota_1\left(\frac{\delta_2}{\lambda}\right)\right|,1\right\}+ \\
	&\quad\quad + |\overline{r}_{l2}|\log\max\left\{\left|\iota_2\left(\frac{\delta_2}{\lambda}\right)\right|,1\right\}+ \sum_{k=1}^{\nu}|u_k-r_k||\alpha_{\gamma l k}|.
\end{align*}
In all cases, it follows that we have 
\begin{align*}
|a_l|	& \leq \sum_{w: L \to \mathbb{C}}\max\{|\overline{r}_{l1}|, |\overline{r}_{l2}|\}|\log\max\left\{\left|w\left(\frac{\delta_2}{\lambda}\right)\right|,1\right\} + |\overline{r}_{l1}|\log\max\left\{\left|\iota_1\left(\frac{\delta_2}{\lambda}\right)\right|,1\right\} + \\
	&\quad\quad + |\overline{r}_{l2}|\log\max\left\{\left|\iota_2\left(\frac{\delta_2}{\lambda}\right)\right|,1\right\}+ \sum_{k=1}^{\nu}|u_k-r_k||\alpha_{\gamma l k}|
\end{align*}
where
\[\alpha_{\gamma l k} = \overline{a}_{1k}\beta_{\gamma l 1} + \cdots + \overline{a}_{\nu k}\beta_{\gamma l \nu}\]
and
\[\beta_{\gamma l k} = \left(\overline{r}_{l1} \log\left| \left( \frac{\gamma_k^{(j)}}{\gamma_k^{(i_0)}}\right)^{\iota_1}\right|+ \overline{r}_{l2}\log\left| \left( \frac{\gamma_k^{(j)}}{\gamma_k^{(i_0)}}\right)^{\iota_2}\right|\right)\]
for $k = 1, \dots, \nu$.
That is, for $k = 1, \dots, \nu$, we have
\begin{align*}
\alpha_{\gamma l k}	
& = \overline{a}_{1k}\left(\overline{r}_{l1} \log\left| \left( \frac{\gamma_1^{(j)}}{\gamma_1^{(i_0)}}\right)^{\iota_1}\right|+ \overline{r}_{l2}\log\left| \left( \frac{\gamma_1^{(j)}}{\gamma_1^{(i_0)}}\right)^{\iota_2}\right|\right) + \cdots \\ 
	& \quad \quad  \cdots + \overline{a}_{\nu k}\left(\overline{r}_{l1} \log\left| \left( \frac{\gamma_{\nu}^{(j)}}{\gamma_{\nu}^{(i_0)}}\right)^{\iota_1}\right|+ \overline{r}_{l2}\log\left| \left( \frac{\gamma_{\nu}^{(j)}}{\gamma_{\nu}^{(i_0)}}\right)^{\iota_2}\right|\right) \\
& = \left(\overline{a}_{1k}\overline{r}_{l1} \log\left| \left( \frac{\gamma_1^{(j)}}{\gamma_1^{(i_0)}}\right)^{\iota_1}\right| + \cdots + \overline{a}_{\nu k}\overline{r}_{l1} \log\left| \left( \frac{\gamma_{\nu}^{(j)}}{\gamma_{\nu}^{(i_0)}}\right)^{\iota_1}\right| \right)+  \\ 
	& \quad \quad + \left(\overline{a}_{1k}\overline{r}_{l2} \log\left| \left( \frac{\gamma_1^{(j)}}{\gamma_1^{(i_0)}}\right)^{\iota_2}\right| + \cdots + \overline{a}_{\nu k}\overline{r}_{l2} \log\left| \left( \frac{\gamma_{\nu}^{(j)}}{\gamma_{\nu}^{(i_0)}}\right)^{\iota_2}\right| \right)\\
& = \overline{r}_{l1} \left(\overline{a}_{1k} \log\left| \left( \frac{\gamma_1^{(j)}}{\gamma_1^{(i_0)}}\right)^{\iota_1}\right| + \cdots + \overline{a}_{\nu k} \log\left| \left( \frac{\gamma_{\nu}^{(j)}}{\gamma_{\nu}^{(i_0)}}\right)^{\iota_1}\right| \right)+  \\ 
	& \quad \quad + \overline{r}_{l2}\left(\overline{a}_{1k} \log\left| \left( \frac{\gamma_1^{(j)}}{\gamma_1^{(i_0)}}\right)^{\iota_2}\right| + \cdots + \overline{a}_{\nu k}\log\left| \left( \frac{\gamma_{\nu}^{(j)}}{\gamma_{\nu}^{(i_0)}}\right)^{\iota_2}\right| \right).
\end{align*}

Recall that
\[h\left(\frac{\delta_2}{\lambda}\right) = \frac{1}{[L:\mathbb{Q}]}\sum_{w :L \to \mathbb{C}} \log \max \left\{ \left|w\left(\frac{\delta_2}{\lambda}\right)\right|, 1\right\} + \frac{1}{[K:\mathbb{Q}]}\sum_{l = 1}^{\nu} \log(p_l)|u_l - r_l|.\]
Hence for $l = 1,2$, 
\begin{align*}
|a_l|	& \leq \sum_{w: L \to \mathbb{C}}\max\{|\overline{r}_{l1}|, |\overline{r}_{l2}|\}|\log\max\left\{\left|w\left(\frac{\delta_2}{\lambda}\right)\right|,1\right\} + |\overline{r}_{l1}|\log\max\left\{\left|\iota_1\left(\frac{\delta_2}{\lambda}\right)\right|,1\right\} + \\
	&\quad\quad + |\overline{r}_{l2}|\log\max\left\{\left|\iota_2\left(\frac{\delta_2}{\lambda}\right)\right|,1\right\}+ \sum_{k=1}^{\nu}|u_k-r_k||\alpha_{\gamma l k}|\\
	& = \frac{1}{[L:\mathbb{Q}]}\sum_{w: L \to \mathbb{C}}\max\{|\overline{r}_{l1}|, |\overline{r}_{l2}|\}[L:\mathbb{Q}]\log\max \left\{\left|w \left(\frac{\delta_2}{\lambda}\right)\right|,1\right\} +\\
	& \quad \quad + \frac{[L:\mathbb{Q}]}{[L:\mathbb{Q}]}|\overline{r}_{l1}|\log\max\left\{\left|\iota_1\left(\frac{\delta_2}{\lambda}\right)\right|,1\right\} + \frac{[L:\mathbb{Q}]}{[L:\mathbb{Q}]}|\overline{r}_{l2}|\log\max\left\{\left|\iota_2\left(\frac{\delta_2}{\lambda}\right)\right|,1\right\} + \\
	& \quad \quad + \frac{1}{[K:\mathbb{Q}]}\sum_{k=1}^{\nu}\frac{[K:\mathbb{Q}]}{\log(p_k)}\log(p_k)|u_k-r_k||\alpha_{\gamma l k}|\\
	& = \frac{1}{[L:\mathbb{Q}]}\sum_{\sigma :L \to \mathbb{C}} w_{\varepsilon l \sigma}\log \max \left\{ \left|\sigma\left(\frac{\delta_2}{\lambda}\right)\right|, 1\right\} + \frac{1}{[K:\mathbb{Q}]}\sum_{k = 1}^{\nu} w_{\gamma l k}\log(p_k)|u_k - r_k|
\end{align*}
where
\[w_{\varepsilon l \sigma} = 
\begin{cases}
\max\{|\overline{r}_{l1}|, |\overline{r}_{l2}|\}[L:\mathbb{Q}] & \text{ for } \sigma \notin I\\
\left(\max\{|\overline{r}_{l1}|, |\overline{r}_{l2}|\} + |\overline{r}_{li}|\right)[L:\mathbb{Q}] & \text{ for } \sigma = \iota_i \in I\\
\end{cases}\]
and 
\[w_{\gamma l k} = |\alpha_{\gamma l k}|\frac{[K:\mathbb{Q}]}{\log(p_k)}\]
where
\begin{align*}
\alpha_{\gamma l k} 
	& = \overline{a}_{1k} \left(\overline{r}_{l1} \log\left| \left( \frac{\gamma_1^{(j)}}{\gamma_1^{(i_0)}}\right)^{\iota_1}\right|+ \overline{r}_{l2}\log\left| \left( \frac{\gamma_1^{(j)}}{\gamma_1^{(i_0)}}\right)^{\iota_2}\right|\right) + \cdots \\
	& \quad \quad \cdots + \overline{a}_{\nu k} \left(\overline{r}_{l1} \log\left| \left( \frac{\gamma_{\nu}^{(j)}}{\gamma_{\nu}^{(i_0)}}\right)^{\iota_1}\right|+ \overline{r}_{l2}\log\left| \left( \frac{\gamma_{\nu}^{(j)}}{\gamma_{\nu}^{(i_0)}}\right)^{\iota_2}\right|\right)
\end{align*}
for $k = 1, \dots, \nu$.

 That is, 
\begin{align*}
|a_l|	& \leq \frac{1}{[L:\mathbb{Q}]}\sum_{\sigma :L \to \mathbb{C}} w_{\varepsilon l \sigma}\log \max \left\{ \left|\sigma\left(\frac{\delta_2}{\lambda}\right)\right|, 1\right\} + \frac{1}{[K:\mathbb{Q}]}\sum_{k = 1}^{\nu} w_{\gamma l k}\log(p_k)|u_k - r_k|\\
	& \leq \max\{w_{\varepsilon l \sigma_1}, \dots, w_{\varepsilon \sigma_{?}}, w_{\gamma 1}, \dots, w_{\gamma \nu}\} h\left(\frac{\delta_2}{\lambda}\right).
\end{align*}

Together with the case $r = 1$, we have proven the following lemma
\begin{lemma}\label{lem:mepsbound}
For any solution $(x,y,a_1, \dots, a_r, n_1, \dots, n_{\nu})$ of \eqref{Eq:main3}, we have
\[|a_l| \leq \max\{w_{\varepsilon \sigma_1}, \dots, w_{\varepsilon \sigma_{?}}, w_{\gamma 1}, \dots, w_{\gamma \nu}\} h\left(\frac{\delta_2}{\lambda}\right)\]
where $l = 1, \dots, r$, where $r = 1, 2$. 
\end{lemma}

\begin{remark}\label{rem:i12}
In Lemma~\ref{lem:mepsbound},  one can take $w_{\epsilon 1}=[L:K]\|r_\epsilon\|_\infty$ for $v\in I$ if $|I|=1$ and the summand $\sum_{v:L\to \mathbb{C}}w_{\epsilon v}h_v(z)$ can be replaced by $3[L:K]\|r_\epsilon\|_\infty\max_{v:L\to\mathbb{C}}h_v(z)$ if $|I|=2$.
\end{remark}
\begin{proof}
In the case $|I|=1$, we  either have precisely one non-negative or precisely one negative. If precisely one non-negative then the claim follows, and if precisely one negative then we just get $\sum_{v\mid\infty}h_v(z)$ by above proof, which again proves the claim.

Consider now the case $|I|=2$. The claim follows if both are non-negative. If both are negative then we just apply once $N(z)=1$ (and product formula) and the claim follows again. Finally, if one is non-negative and one negative, then we compute that $$2\sum_{v\in I\setminus I^-}h_v(z)+\sum_{v\mid \infty,v\notin I}h_v(z)\leq 3\max_{v\mid \infty}h_v(z).$$ This follows since there are at most 3 positive ones in total and if there are indeed 3 positive ones, then the middle one cancels out (see proof of above lemma).
\end{proof}

\begin{question}
I should go through the above. But isn't it more accurate if we just compute the bound as is?
\end{question}

Now, 
\begin{equation}\label{def:bepsbound}
|a_l|^2 \leq \left( \frac{1}{[K:\mathbb{Q}]}\sum_{k = 1}^{\nu} w_{\gamma l k}\log(p_k)|u_k - r_k| + \frac{1}{[L:\mathbb{Q}]}\sum_{\sigma :L \to \mathbb{C}} w_{\varepsilon l \sigma}\log \max \left\{ \left|\sigma\left(\frac{\delta_2}{\lambda}\right)\right|, 1\right\} \right)^2.
\end{equation}

%In particular, if $\deg{g(t)}=3$ then 
%\[|a_l|^2 \leq
%\begin{cases}
%\left( \frac{2}{[L:\mathbb{Q}]}\max_{\sigma:L\to \mathbb{C}} w_{\varepsilon \sigma}\log \max \left\{ \left|\sigma\left(\frac{\delta_2}{\lambda}\right)\right|, 1\right\}  + \frac{1}{[K:\mathbb{Q}]}\sum_{k = 1}^{\nu} w_{\gamma k}\log(p_k)|u_k - r_k|\right)^2 & \text{ if } \sqrt{\Delta}\notin\mathbb{Q} \\
%\left( \frac{1}{[L:\mathbb{Q}]}\max_{\sigma:L\to \mathbb{C}} w_{\varepsilon \sigma}\log \max \left\{ \left|\sigma\left(\frac{\delta_2}{\lambda}\right)\right|, 1\right\}  + \frac{1}{[K:\mathbb{Q}]}\sum_{k = 1}^{\nu} w_{\gamma k}\log(p_k)|u_k - r_k|\right)^2 & \text{ if } \sqrt{\Delta}\in\mathbb{Q} \\
%\end{cases}\]
%NOTE: CAN'T CANCEL STUFF OUT IN THE AUTOMORPHISM BIT BECAUSE THE COEFFICIENTS WESIGMA ARE NOW NOT ALLOWING US TO CANCEL OUT THE LOGS
%CAN USE RAFAELS CANCELLATION INSTEAD, BUT THEN HAVE TO TAKE 3*MAX THING
%OR JUST DIRECTLY COMPUTE THIS FOR ALL AUTOS

Take $\mathbf{h}\in\mathbb{R}^{r+\nu}$ such that $\mathbf{h}\geq \mathbf{0}$. Let $\mathbf{m} = (n_1, \dots, n_{\nu}, a_1, \dots, a_r) \in \mathbb{R}^{r + \nu}$ be any solution of \eqref{Eq:main3}. Denote by $h_{v}\left(\frac{\delta_2}{\lambda}\right)$ the $v^{\text{th}}$ entry of the vector
\[\left(\log(p_1)|u_1 - r_1|, \dots, \log(p_{\nu})|u_{\nu} - r_{\nu}|, \log \max \left\{ \left|w_1\left(\frac{\delta_2}{\lambda}\right)\right|, 1\right\}, \dots, \log \max \left\{ \left|w_n\left(\frac{\delta_2}{\lambda}\right)\right|, 1\right\} \right)\] 
and suppose $h_v(z)\leq h_v$ for all $v\in \{1, \dots, r+\nu\}$. Then we deduce
\[|a_l|^2 \leq \left( \frac{1}{[K:\mathbb{Q}]}\sum_{k = 1}^{\nu} w_{\gamma l k}h_k + \frac{1}{[L:\mathbb{Q}]}\sum_{\sigma:L\to \mathbb{C}} w_{\varepsilon l \sigma}h_{\sigma}\right)^2 \]
%\left( \frac{1}{[L:\mathbb{Q}]}\max_{\sigma:L\to \mathbb{C}} w_{\varepsilon \sigma}h_{\sigma} + \frac{1}[K:
%\begin{cases}
%\left( \frac{2}{[L:\mathbb{Q}]}\max_{\sigma:L\to \mathbb{C}} w_{\varepsilon \sigma}h_{\sigma}  + \frac{1}{[K:\mathbb{Q}]}\sum_{k = 1}^{\nu} w_{\gamma k}h_k\right)^2 & \text{ if } \sqrt{\Delta}\notin\mathbb{Q} \\
%\left( \frac{1}{[L:\mathbb{Q}]}\max_{\sigma:L\to \mathbb{C}} w_{\varepsilon \sigma}h_{\sigma} + \frac{1}{[K:\mathbb{Q}]}\sum_{k = 1}^{\nu} w_{\gamma k}h_k\right)^2 & \text{ if } \sqrt{\Delta}\in\mathbb{Q} \\
%\end{cases}\right)\]

\subsection{Archimedean ellipsoid: real case, $r = 2$.}
We first consider the case when all roots of $f$ are real numbers. That is, there are $3$ real embeddings, hence $s=3, t = 0$ and therefore $r = s+t-1 = 2$. 

Let $\tau:L\to\mathbb{R} \subset \mathbb{C}$ be an embedding and let $l_\tau\geq c_\tau$ and $c>0$ be given real numbers for $c_\tau=\log^+(2|\tau(\delta_2)|)= \log \max\{2|\tau(\delta_2)|,1\}$. We define 
\[\alpha_0 = [c\log|\tau(\delta_1)|] \quad \text{ and } \quad \alpha_{\varepsilon 1} =  \left[c\log\left|\tau\left(\frac{\varepsilon_1^{(k)}}{\varepsilon_1^{(j)}}\right)\right|\right],\ \  \alpha_{\varepsilon 2} =  \left[c\log\left|\tau\left(\frac{\varepsilon_2^{(k)}}{\varepsilon_2^{(j)}}\right)\right|\right].\]
For $i = 1, \dots, \nu$, define
\[\alpha_{\gamma i} = \left[c\log\left|\tau\left(\frac{\gamma_i^{(k)}}{\gamma_i^{(j)}}\right)\right|\right].\]
Here, $[ \ \cdot\  ]$ denotes the nearest integer function. 
Recall that
\[h\left(\frac{\delta_2}{\lambda}\right) = \frac{1}{[K:\mathbb{Q}]}\sum_{l = 1}^{\nu} \log(p_l)|u_l - r_l| + \frac{1}{[L:\mathbb{Q}]}\sum_{w :L \to \mathbb{C}} \log \max \left\{ \left|w\left(\frac{\delta_2}{\lambda}\right)\right|, 1\right\}.\]
Let 
\[h_{\tau}\left(\frac{\delta_2}{\lambda}\right) =\log \max \left\{ \left|\tau\left(\frac{\delta_2}{\lambda}\right)\right|, 1\right\},\]
the $\tau^{\text{th}}$ entry in the second summand of $h\left(\frac{\delta_2}{\lambda}\right)$ and $k_{\tau} = \frac{3}{2}.$

\begin{lemma}\label{lem:archellest}
Suppose $(x,y, n_1, \dots, n_{\nu}, a_1, \dots, a_r)$ is a solution of \eqref{Eq:main3}. If ${h_{\tau} \left(\frac{\delta_2}{\lambda}\right) > c_{\tau}}$ and $\kappa_\tau=3/2$, then  
\begin{align*}
&\left|\alpha_0+\sum_{i = 1}^r a_i \alpha_{\varepsilon i} + \sum_{i = 1}^{\nu} n_i \alpha_{\gamma i}\right|\\
	& \leq \frac{1}{2}\left(\frac{1}{[K:\mathbb{Q}]}\sum_{l = 1}^{\nu}w_l \log(p_l)|u_l - r_l| + \frac{1}{[L:\mathbb{Q}]}\sum_{\sigma :L \to \mathbb{C}} w_{\sigma}\log \max \left\{ \left|\sigma\left(\frac{\delta_2}{\lambda}\right)\right|, 1\right\} \right) + \\
	& \quad \quad \quad + \left(\frac{1}{2} + c\kappa_{\tau}e^{-h_{\tau}\left(\frac{\delta_2}{\lambda}\right)}\right).
\end{align*} 
\end{lemma}

\begin{proof}
Let 
\begin{align*}
\alpha	
	& = \alpha_0+\sum_{i = 1}^r a_i \alpha_{\varepsilon i} + \sum_{i = 1}^{\nu} n_i \alpha_{\gamma i}\\
	& = [c\log|\tau(\delta_1)|] +\sum_{i = 1}^r a_i \left[c\log\left|\tau\left(\frac{\varepsilon_i^{(k)}}{\varepsilon_i^{(j)}}\right)\right|\right] + \sum_{i = 1}^{\nu} n_i \left[c\log\left|\tau\left(\frac{\gamma_i^{(k)}}{\gamma_i^{(j)}}\right)\right|\right]
\end{align*}
and
\[\Lambda_{\tau} = \log\left|\tau\left(\delta_1 \prod_{i = 1}^r\left( \frac{\varepsilon_i^{(k)}}{\varepsilon_i^{(j)}}\right)^{a_i}\prod_{i = 1}^{\nu} \left( \frac{\gamma_i^{(k)}}{\gamma_i^{(j)}}\right)^{n_i}\right)\right|= \log\left(\tau\left(\delta_1 \prod_{i = 1}^r\left( \frac{\varepsilon_i^{(k)}}{\varepsilon_i^{(j)}}\right)^{a_i}\prod_{i = 1}^{\nu} \left( \frac{\gamma_i^{(k)}}{\gamma_i^{(j)}}\right)^{n_i}\right)\right) \]
where the above equality follows from 
\[\tau\left(\delta_1 \prod_{i = 1}^r\left( \frac{\varepsilon_i^{(k)}}{\varepsilon_i^{(j)}}\right)^{a_i}\prod_{i = 1}^{\nu} \left( \frac{\gamma_i^{(k)}}{\gamma_i^{(j)}}\right)^{n_i}\right) > 0.\]
Indeed, by assumption, it holds that 
\[ h_{\tau}\left(\frac{\delta_2}{\lambda}\right) = \log \max \left\{ \left|\tau\left(\frac{\delta_2}{\lambda}\right)\right|, 1\right\} > c_\tau = \log \max\{2|\tau(\delta_2)|,1\}\]
Thus
\begin{align*}
\exp\left(h_{\tau}\left(\frac{\delta_2}{\lambda}\right)\right)	& > \exp(c_{\tau}) \\
\exp\left(\log \max \left\{ \left|\tau\left(\frac{\delta_2}{\lambda}\right)\right|, 1\right\}\right) & > \exp \left(\log \max\{2|\tau(\delta_2)|,1\}\right)\\
\max \left\{ \left|\tau\left(\frac{\delta_2}{\lambda}\right)\right|, 1\right\} & > \max\{2|\tau(\delta_2)|,1\}\\
\end{align*}
Now, we have
\[\max\{2|\tau(\delta_2)|,1\} < \max \left\{ \left|\tau\left(\frac{\delta_2}{\lambda}\right)\right|, 1\right\}.\]
If  
\[\max \left\{ \left|\tau\left(\frac{\delta_2}{\lambda}\right)\right|, 1\right\} = 1,\]
then
\[\max\{2|\tau(\delta_2)|,1\} < \max \left\{ \left|\tau\left(\frac{\delta_2}{\lambda}\right)\right|, 1\right\} = 1.\]
If $\max\{2|\tau(\delta_2)|,1\} = 1$, then
\[1 = \max\{2|\tau(\delta_2)|,1\} < \max \left\{ \left|\tau\left(\frac{\delta_2}{\lambda}\right)\right|, 1\right\} = 1,\]
which is impossible, so we must have that $2|\tau(\delta_2)| \geq 1$. In this case, 
\[1 \leq 2|\tau(\delta_2)| = \max\{2|\tau(\delta_2)|,1\} < \max \left\{ \left|\tau\left(\frac{\delta_2}{\lambda}\right)\right|, 1\right\} = 1,\]
which again is impossible. It follows that we must have
\[\max \left\{ \left|\tau\left(\frac{\delta_2}{\lambda}\right)\right|, 1\right\} = \left|\tau\left(\frac{\delta_2}{\lambda}\right)\right|.\]
In this case, 
\[\max\{2|\tau(\delta_2)|,1\} < \max \left\{ \left|\tau\left(\frac{\delta_2}{\lambda}\right)\right|, 1\right\} = \left|\tau\left(\frac{\delta_2}{\lambda}\right)\right|.\]
It follows that
\[2|\tau(\delta_2)| \leq \max\{2|\tau(\delta_2)|,1\} < \max \left\{ \left|\tau\left(\frac{\delta_2}{\lambda}\right)\right|, 1\right\} = \left|\tau\left(\frac{\delta_2}{\lambda}\right)\right|\]
and therefore
\[2|\tau(\delta_2)| < \left|\tau\left(\frac{\delta_2}{\lambda}\right)\right| = \frac{|\tau(\delta_2)|}{|\tau(\lambda)|} \implies |\tau(\lambda)| < \frac{1}{2}.\]
Now, since
\[\lambda = \delta_2 \prod_{i = 1}^{r}\left( \frac{\varepsilon_i^{(i_0)}}{\varepsilon_i^{(j)}}\right)^{a_i} \prod_{i = 1}^{\nu} \left( \frac{\gamma_i^{(i_0)}}{\gamma_i^{(j)}}\right)^{n_i} = \delta_1 \prod_{i = 1}^r\left( \frac{\varepsilon_i^{(k)}}{\varepsilon_i^{(j)}}\right)^{a_i}\prod_{i = 1}^{\nu} \left( \frac{\gamma_i^{(k)}}{\gamma_i^{(j)}}\right)^{n_i} - 1,\]
where
\[\mu =  \delta_1 \prod_{i = 1}^r\left( \frac{\varepsilon_i^{(k)}}{\varepsilon_i^{(j)}}\right)^{a_i}\prod_{i = 1}^{\nu} \left( \frac{\gamma_i^{(k)}}{\gamma_i^{(j)}}\right)^{n_i},\]
we have
\[\lambda = \mu - 1.\]
Applying $\tau$, this is
\[\tau(\lambda) = \tau(\mu) -1\]
and thus
\[|\tau(\lambda)| < \frac{1}{2} \implies \tau(\mu) = \tau(\lambda) + 1 > 0.\]
It follows that
\[\tau(\mu) = \tau\left(\delta_1 \prod_{i = 1}^r\left( \frac{\varepsilon_i^{(k)}}{\varepsilon_i^{(j)}}\right)^{a_i}\prod_{i = 1}^{\nu} \left( \frac{\gamma_i^{(k)}}{\gamma_i^{(j)}}\right)^{n_i}\right) > 0.\]
Now, 
\[\Lambda_{\tau} = \log\left|\tau\left(\delta_1 \prod_{i = 1}^r\left( \frac{\varepsilon_i^{(k)}}{\varepsilon_i^{(j)}}\right)^{a_i}\prod_{i = 1}^{\nu} \left( \frac{\gamma_i^{(k)}}{\gamma_i^{(j)}}\right)^{n_i}\right)\right|= \log\left(\tau\left(\delta_1 \prod_{i = 1}^r\left( \frac{\varepsilon_i^{(k)}}{\varepsilon_i^{(j)}}\right)^{a_i}\prod_{i = 1}^{\nu} \left( \frac{\gamma_i^{(k)}}{\gamma_i^{(j)}}\right)^{n_i}\right)\right) \]
and thus
\begin{align*}
\Lambda_{\tau}	
	& = \log\left(\tau\left(\delta_1 \prod_{i = 1}^r\left( \frac{\varepsilon_i^{(k)}}{\varepsilon_i^{(j)}}\right)^{a_i}\prod_{i = 1}^{\nu} \left( \frac{\gamma_i^{(k)}}{\gamma_i^{(j)}}\right)^{n_i}\right)\right)\\
	& = \log\left(\tau\left(\delta_1\right) \tau\left(\prod_{i = 1}^r\left( \frac{\varepsilon_i^{(k)}}{\varepsilon_i^{(j)}}\right)^{a_i}\right)\tau\left(\prod_{i = 1}^{\nu} \left( \frac{\gamma_i^{(k)}}{\gamma_i^{(j)}}\right)^{n_i}\right)\right)\\
	& = \log\left(\tau\left(\delta_1\right) \prod_{i = 1}^r\tau\left( \frac{\varepsilon_i^{(k)}}{\varepsilon_i^{(j)}}\right)^{a_i}\prod_{i = 1}^{\nu} \tau\left( \frac{\gamma_i^{(k)}}{\gamma_i^{(j)}}\right)^{n_i}\right)\\
	& = \log\left(\tau\left(\delta_1\right)\right) +\log\left(\prod_{i = 1}^r\tau\left( \frac{\varepsilon_i^{(k)}}{\varepsilon_i^{(j)}}\right)^{a_i} \right) + \log \left(\prod_{i = 1}^{\nu} \tau\left( \frac{\gamma_i^{(k)}}{\gamma_i^{(j)}}\right)^{n_i} \right)\\
	& = \log\left(\tau\left(\delta_1\right)\right) + \sum_{i=1}^r a_i\log\left(\tau\left( \frac{\varepsilon_i^{(k)}}{\varepsilon_i^{(j)}}\right) \right) + \sum_{i=1}^{\nu}n_i\log \left(\tau\left( \frac{\gamma_i^{(k)}}{\gamma_i^{(j)}}\right)\right)\\
\end{align*}
We have
\[|\alpha| = |\alpha + c\Lambda_{\tau} - c\Lambda_{\tau}|\] 
so that by the triangle inequality, 
\[|\alpha| \leq |\alpha - c\Lambda_{\tau}| + c|\Lambda_{\tau}|.\]
Now, 
\begin{align*}
|\alpha-c\Lambda_\tau|
	& = \left|[c\log(\tau(\delta_1))] +\sum_{i = 1}^r a_i \left[c\log\left(\tau\left(\frac{\varepsilon_i^{(k)}}{\varepsilon_i^{(j)}}\right)\right)\right] + \sum_{i = 1}^{\nu} n_i \left[c\log\left(\tau\left(\frac{\gamma_i^{(k)}}{\gamma_i^{(j)}}\right)\right)\right]\right. +\\
	& \quad \quad \left. - c \log\left(\tau\left(\delta_1 \prod_{i = 1}^r\left( \frac{\varepsilon_i^{(k)}}{\varepsilon_i^{(j)}}\right)^{a_i}\prod_{i = 1}^{\nu} \left( \frac{\gamma_i^{(k)}}{\gamma_i^{(j)}}\right)^{n_i}\right)\right)\right|\\
	& = \left|[c\log(\tau(\delta_1))] +\sum_{i = 1}^r a_i \left[c\log\left(\tau\left(\frac{\varepsilon_i^{(k)}}{\varepsilon_i^{(j)}}\right)\right)\right] + \sum_{i = 1}^{\nu} n_i \left[c\log\left(\tau\left(\frac{\gamma_i^{(k)}}{\gamma_i^{(j)}}\right)\right)\right]\right. +\\
	& \quad \quad \left. - c\log\left(\tau\left(\delta_1\right)\right) - \sum_{i=1}^r a_i c\log\left(\tau\left( \frac{\varepsilon_i^{(k)}}{\varepsilon_i^{(j)}}\right) \right) - \sum_{i=1}^{\nu} n_i c\log \left(\tau\left( \frac{\gamma_i^{(k)}}{\gamma_i^{(j)}}\right)\right)\right|\\
	& = \left| \left\{ [c\log(\tau(\delta_1))] - c\log\left(\tau\left(\delta_1\right)\right)\right\} + \sum_{i = 1}^r \left\{a_i \left[c\log\left(\tau\left(\frac{\varepsilon_i^{(k)}}{\varepsilon_i^{(j)}}\right)\right)\right] - a_i c\log\left(\tau\left( \frac{\varepsilon_i^{(k)}}{\varepsilon_i^{(j)}}\right) \right)\right\} \right.\\
	& \quad \quad \left. + \sum_{i = 1}^{\nu} \left\{n_i \left[c\log\left(\tau\left(\frac{\gamma_i^{(k)}}{\gamma_i^{(j)}}\right)\right)\right] - n_i c\log \left(\tau\left( \frac{\gamma_i^{(k)}}{\gamma_i^{(j)}}\right)\right)\right\}\right|\\
	& \leq \left| [c\log(\tau(\delta_1))] - c\log\left(\tau\left(\delta_1\right)\right)\right| + \left|\sum_{i = 1}^r \left\{a_i \left[c\log\left(\tau\left(\frac{\varepsilon_i^{(k)}}{\varepsilon_i^{(j)}}\right)\right)\right] - a_i c\log\left(\tau\left( \frac{\varepsilon_i^{(k)}}{\varepsilon_i^{(j)}}\right) \right)\right\} \right| \\
	& \quad \quad + \left| \sum_{i = 1}^{\nu} \left\{n_i \left[c\log\left(\tau\left(\frac{\gamma_i^{(k)}}{\gamma_i^{(j)}}\right)\right)\right] - n_i c\log \left(\tau\left( \frac{\gamma_i^{(k)}}{\gamma_i^{(j)}}\right)\right)\right\}\right|\\
	& \leq \left| [c\log(\tau(\delta_1))] - c\log\left(\tau\left(\delta_1\right)\right)\right| + \sum_{i = 1}^r |a_i|\left| \left[c\log\left(\tau\left(\frac{\varepsilon_i^{(k)}}{\varepsilon_i^{(j)}}\right)\right)\right] - c\log\left(\tau\left( \frac{\varepsilon_i^{(k)}}{\varepsilon_i^{(j)}}\right) \right)\right| \\
	& \quad \quad +  \sum_{i = 1}^{\nu} |n_i|\left| \left[c\log\left(\tau\left(\frac{\gamma_i^{(k)}}{\gamma_i^{(j)}}\right)\right)\right] - c\log \left(\tau\left( \frac{\gamma_i^{(k)}}{\gamma_i^{(j)}}\right)\right)\right|.
\end{align*}
Now, since $[ \ \cdot \ ]$ denotes the nearest integer function, it is clear that $|[ \ c \ ] - c| \leq 1/2$ for any integer $c$. Hence
\begin{align*}
|\alpha-c\Lambda_\tau|
	& \leq \left| [c\log(\tau(\delta_1))] - c\log\left(\tau\left(\delta_1\right)\right)\right| + \sum_{i = 1}^r |a_i|\left| \left[c\log\left(\tau\left(\frac{\varepsilon_i^{(k)}}{\varepsilon_i^{(j)}}\right)\right)\right] - c\log\left(\tau\left( \frac{\varepsilon_i^{(k)}}{\varepsilon_i^{(j)}}\right) \right)\right| \\
	& \quad \quad +  \sum_{i = 1}^{\nu} |n_i|\left| \left[c\log\left(\tau\left(\frac{\gamma_i^{(k)}}{\gamma_i^{(j)}}\right)\right)\right] - c\log \left(\tau\left( \frac{\gamma_i^{(k)}}{\gamma_i^{(j)}}\right)\right)\right|\\
	& \leq \frac{1}{2} + \frac{1}{2}\sum_{i = 1}^r |a_i| + \frac{1}{2}\sum_{i = 1}^{\nu} |n_i|\\
	& = \frac{1}{2} + \frac{1}{2}\sum_{i = 1}^r |a_i| + \frac{1}{2}\sum_{i = 1}^{\nu} \left|\sum_{k=1}^{\nu} \overline{a}_{ik}(u_k-r_k)\right|\\
	& = \frac{1}{2} + \frac{1}{2}\sum_{i = 1}^r |a_i| + \frac{1}{2}\sum_{i = 1}^{\nu} \left|\overline{a}_{i1}(u_1-r_1) + \cdots + \overline{a}_{i\nu}(u_{\nu}-r_{\nu})\right|\\
	& = \frac{1}{2} + \frac{1}{2}\sum_{i = 1}^r |a_i| + \frac{1}{2} \left|\left(\overline{a}_{11}(u_1-r_1) + \cdots + \overline{a}_{1\nu}(u_{\nu}-r_{\nu})\right) + \cdots \right.\\
	& \quad\quad \cdots + \left.\left(\overline{a}_{\nu 1}(u_1-r_1) + \cdots + \overline{a}_{\nu\nu}(u_{\nu}-r_{\nu})\right)\right|\\
	& = \frac{1}{2} + \frac{1}{2}\sum_{i = 1}^r |a_i| + \frac{1}{2} \left|(u_1-r_1)(\overline{a}_{11} + \cdots + \overline{a}_{\nu 1}) + \cdots + (u_{\nu} - r_{\nu})(\overline{a}_{1\nu} + \cdots + \overline{a}_{\nu\nu}) \right|\\
	& \leq \frac{1}{2}\left(1 + \sum_{i = 1}^r |a_i| + |u_1-r_1||\overline{a}_{11} + \cdots + \overline{a}_{\nu 1}| + \cdots + |u_{\nu} - r_{\nu}||\overline{a}_{1\nu} + \cdots + \overline{a}_{\nu\nu}|\right)\\
	& = \frac{1}{2}\left(1 + \sum_{i = 1}^r |a_i| + |u_1-r_1|\sum_{i=1}^{\nu}|\overline{a}_{i1}| + \cdots + |u_{\nu} - r_{\nu}| \sum_{i=1}^{\nu}|\overline{a}_{i\nu}|\right).
\end{align*}
Recall that when $r = 2$, we have, for $l = 1,2$
\[|a_l| \leq \frac{1}{[L:\mathbb{Q}]}\sum_{\sigma :L \to \mathbb{C}} w_{\varepsilon l \sigma}\log \max \left\{ \left|\sigma\left(\frac{\delta_2}{\lambda}\right)\right|, 1\right\} + \frac{1}{[K:\mathbb{Q}]}\sum_{k = 1}^{\nu} w_{\gamma l k}\log(p_k)|u_k - r_k|\]
where
\[w_{\varepsilon l \sigma} = 
\begin{cases}
\max\{|\overline{r}_{l1}|, |\overline{r}_{l2}|\}[L:\mathbb{Q}] & \text{ for } \sigma \notin I\\
\left(\max\{|\overline{r}_{l1}|, |\overline{r}_{l2}|\} + |\overline{r}_{li}|\right)[L:\mathbb{Q}] & \text{ for } \sigma = \iota_i \in I\\
\end{cases}\]
and 
\[w_{\gamma l k} = |\alpha_{\gamma l k}|\frac{[K:\mathbb{Q}]}{\log(p_k)}\]
where
\begin{align*}
\alpha_{\gamma l k} 
	& = \overline{a}_{1k} \left(\overline{r}_{l1} \log\left| \left( \frac{\gamma_1^{(j)}}{\gamma_1^{(i_0)}}\right)^{\iota_1}\right|+ \overline{r}_{l2}\log\left| \left( \frac{\gamma_1^{(j)}}{\gamma_1^{(i_0)}}\right)^{\iota_2}\right|\right) + \cdots \\
	& \quad \quad \cdots + \overline{a}_{\nu k} \left(\overline{r}_{l1} \log\left| \left( \frac{\gamma_{\nu}^{(j)}}{\gamma_{\nu}^{(i_0)}}\right)^{\iota_1}\right|+ \overline{r}_{l2}\log\left| \left( \frac{\gamma_{\nu}^{(j)}}{\gamma_{\nu}^{(i_0)}}\right)^{\iota_2}\right|\right)
\end{align*}
for $k = 1, \dots, \nu$.

Now, 
\begin{align*}
|\alpha-c\Lambda_\tau| 
	& \leq \frac{1}{2} + \frac{1}{2}\sum_{i = 1}^r |a_i| + \frac{1}{2}|u_1-r_1|\sum_{i=1}^{\nu}|\overline{a}_{i1}| + \cdots + \frac{1}{2}|u_{\nu} - r_{\nu}| \sum_{i=1}^{\nu}|\overline{a}_{i\nu}| \\
	& \leq \frac{1}{2} + \frac{1}{2}|u_1-r_1|\sum_{i=1}^{\nu}|\overline{a}_{i1}| + \cdots + \frac{1}{2}|u_{\nu} - r_{\nu}| \sum_{i=1}^{\nu}|\overline{a}_{i\nu}| + \\
	& \quad + \frac{1}{2}\left(\frac{1}{[L:\mathbb{Q}]}\sum_{\sigma :L \to \mathbb{C}} w_{\varepsilon_1 \sigma}\log \max \left\{ \left|\sigma\left(\frac{\delta_2}{\lambda}\right)\right|, 1\right\} + \frac{1}{[K:\mathbb{Q}]}\sum_{k = 1}^{\nu} w_{\gamma_1 k}\log(p_k)|u_k - r_k| \right) + \\
	&\quad + \frac{1}{2}\left(\frac{1}{[L:\mathbb{Q}]}\sum_{\sigma :L \to \mathbb{C}} w_{\varepsilon_2 \sigma}\log \max \left\{ \left|\sigma\left(\frac{\delta_2}{\lambda}\right)\right|, 1\right\} + \frac{1}{[K:\mathbb{Q}]}\sum_{k = 1}^{\nu} w_{\gamma_2 k}\log(p_k)|u_k - r_k| \right) \\
	& = \frac{1}{2} + \frac{1}{2}|u_1-r_1|\sum_{i=1}^{\nu}|\overline{a}_{i1}| + \cdots + \frac{1}{2}|u_{\nu} - r_{\nu}| \sum_{i=1}^{\nu}|\overline{a}_{i\nu}| + \\
	& \quad + \frac{1}{2}\left(\frac{1}{[L:\mathbb{Q}]}\sum_{\sigma :L \to \mathbb{C}} (w_{\varepsilon_1 \sigma} + w_{\varepsilon_2 \sigma})\log \max \left\{ \left|\sigma\left(\frac{\delta_2}{\lambda}\right)\right|, 1\right\}\right) + \\
	&\quad + \frac{1}{2}\left(\frac{1}{[K:\mathbb{Q}]}\sum_{k = 1}^{\nu} (w_{\gamma_1 k} + w_{\gamma_2 k})\log(p_k)|u_k - r_k| \right) \\
	& = \frac{1}{2} + \frac{1}{[L:\mathbb{Q}]}\sum_{\sigma :L \to \mathbb{C}} \frac{(w_{\varepsilon_1 \sigma} + w_{\varepsilon_2 \sigma})}{2}\log \max \left\{ \left|\sigma\left(\frac{\delta_2}{\lambda}\right)\right|, 1\right\} + \\
	&\quad + \frac{1}{[K:\mathbb{Q}]}\sum_{k = 1}^{\nu} \frac{(w_{\gamma_1 k} + w_{\gamma_2 k})}{2}\log(p_k)|u_k - r_k| \\
	& \quad + \frac{1}{2}|u_1-r_1|\sum_{i=1}^{\nu}|\overline{a}_{i1}| + \cdots + \frac{1}{2}|u_{\nu} - r_{\nu}| \sum_{i=1}^{\nu}|\overline{a}_{i\nu}|\\
	& = \frac{1}{2} + \frac{1}{[L:\mathbb{Q}]}\sum_{\sigma :L \to \mathbb{C}} \frac{(w_{\varepsilon_1 \sigma} + w_{\varepsilon_2 \sigma})}{2}\log \max \left\{ \left|\sigma\left(\frac{\delta_2}{\lambda}\right)\right|, 1\right\} + \\
	&\quad + \frac{1}{[K:\mathbb{Q}]}\left(\frac{(w_{\gamma_1 1} + w_{\gamma_2 1})}{2}\log(p_1)|u_1 - r_1|+ \cdots + \frac{(w_{\gamma_1 \nu} + w_{\gamma_2 \nu})}{2}\log(p_{\nu})|u_{\nu} - r_{\nu}|\right) \\
	& \quad + \frac{1}{2}|u_1-r_1|\sum_{i=1}^{\nu}|\overline{a}_{i1}| + \cdots + \frac{1}{2}|u_{\nu} - r_{\nu}| \sum_{i=1}^{\nu}|\overline{a}_{i\nu}|\\
	& = \frac{1}{2} + \frac{1}{[L:\mathbb{Q}]}\sum_{\sigma :L \to \mathbb{C}} \frac{(w_{\varepsilon_1 \sigma} + w_{\varepsilon_2 \sigma})}{2}\log \max \left\{ \left|\sigma\left(\frac{\delta_2}{\lambda}\right)\right|, 1\right\} + \\
	& \quad + |u_1 - r_1|\left( \frac{(w_{\gamma_1 1} + w_{\gamma_2 1})}{2[K:\mathbb{Q}]}\log(p_1) + \frac{1}{2}\sum_{i=1}^{\nu}|\overline{a}_{i1}|\right) + \cdots \\
	& \quad + |u_{\nu} - r_{\nu}|\left( \frac{(w_{\gamma_1 {\nu}} + w_{\gamma_2 {\nu}})}{2[K:\mathbb{Q}]}\log(p_{\nu}) + \frac{1}{2}\sum_{i=1}^{\nu}|\overline{a}_{i{\nu}}|\right). 
\end{align*}
Altogether, we have
\begin{align*}
|\alpha-c\Lambda_\tau| 
	& = \frac{1}{2} + \frac{1}{[L:\mathbb{Q}]}\sum_{\sigma :L \to \mathbb{C}} \frac{(w_{\varepsilon_1 \sigma} + w_{\varepsilon_2 \sigma})}{2}\log \max \left\{ \left|\sigma\left(\frac{\delta_2}{\lambda}\right)\right|, 1\right\} + \\
	& \quad + |u_1 - r_1|\left( \frac{(w_{\gamma_1 1} + w_{\gamma_2 1})}{2[K:\mathbb{Q}]}\log(p_1) + \frac{1}{2}\sum_{i=1}^{\nu}|\overline{a}_{i1}|\right) + \cdots \\
	& \quad + |u_{\nu} - r_{\nu}|\left( \frac{(w_{\gamma_1 {\nu}} + w_{\gamma_2 {\nu}})}{2[K:\mathbb{Q}]}\log(p_{\nu}) + \frac{1}{2}\sum_{i=1}^{\nu}|\overline{a}_{i{\nu}}|\right)\\
	& = \frac{1}{2} + \frac{1}{[L:\mathbb{Q}]}\sum_{\sigma :L \to \mathbb{C}} \frac{(w_{\varepsilon_1 \sigma} + w_{\varepsilon_2 \sigma})}{2}\log \max \left\{ \left|\sigma\left(\frac{\delta_2}{\lambda}\right)\right|, 1\right\} + \\
	& \quad + \sum_{k = 1}^{\nu} |u_k - r_k|\left( \frac{(w_{\gamma_1 k} + w_{\gamma_2 k})}{2[K:\mathbb{Q}]}\log(p_k) + \frac{1}{2}\sum_{i=1}^{\nu}|\overline{a}_{ik}|\right)\\
	& = \frac{1}{2} + \frac{1}{2[L:\mathbb{Q}]}\sum_{\sigma :L \to \mathbb{C}} (w_{\varepsilon_1 \sigma} + w_{\varepsilon_2 \sigma})\log \max \left\{ \left|\sigma\left(\frac{\delta_2}{\lambda}\right)\right|, 1\right\} + \\
	& \quad + \frac{1}{2[K:\mathbb{Q}]}\sum_{k = 1}^{\nu} \log(p_k)|u_k - r_k|\left( (w_{\gamma_1 k} + w_{\gamma_2 k}) + \frac{[K:\mathbb{Q}]}{\log(p_k)}\sum_{i=1}^{\nu}|\overline{a}_{ik}|\right).
\end{align*}
Now, let
\[w_{\sigma} = (w_{\varepsilon 1 \sigma} + w_{\varepsilon 2 \sigma}), \quad \quad w_k = (w_{\gamma 1 k} + w_{\gamma 2 k}) + \frac{[K:\mathbb{Q}]}{\log(p_k)}\sum_{i=1}^{\nu}|\overline{a}_{ik}|\]
for $\sigma: L \to \mathbb{C}$ and $k = 1, \dots, \nu$. That is, 
\begin{align*}
|\alpha-c\Lambda_\tau|
	& \leq \frac{1}{2} + \frac{1}{2[L:\mathbb{Q}]}\sum_{\sigma :L \to \mathbb{C}} w_{\sigma}\log \max \left\{ \left|\sigma\left(\frac{\delta_2}{\lambda}\right)\right|, 1\right\} + \frac{1}{2[K:\mathbb{Q}]}\sum_{l = 1}^{\nu}w_l \log(p_l)|u_l - r_l|\\
	& \leq \frac{1}{2} + \frac{1}{2}\left(\frac{1}{[L:\mathbb{Q}]}\sum_{\sigma :L \to \mathbb{C}} w_{\sigma}\log \max \left\{ \left|\sigma\left(\frac{\delta_2}{\lambda}\right)\right|, 1\right\} + \frac{1}{[K:\mathbb{Q}]}\sum_{l = 1}^{\nu}w_l \log(p_l)|u_l - r_l|\right).
\end{align*}

We compare this to $h\left(\frac{\delta_2}{\lambda}\right)$, which we recall is given by
\[h\left(\frac{\delta_2}{\lambda}\right) = \frac{1}{[L:\mathbb{Q}]}\sum_{\sigma :L \to \mathbb{C}} \log \max \left\{ \left|\sigma\left(\frac{\delta_2}{\lambda}\right)\right|, 1\right\} + \frac{1}{[K:\mathbb{Q}]}\sum_{l = 1}^{\nu} \log(p_l)|u_l - r_l|.\]

Now, we bound $c|\Lambda_{\tau}|$ to obtain a bound for $|\alpha|$. Via Rafael, we will see that this bound is
\[c|\Lambda_{\tau}| \leq c\kappa_{\tau}e^{-h_{\tau}\left(\frac{\delta_2}{\lambda}\right)}.\]
Since 
\[\tau(\lambda) = \tau\left(\delta_2 \prod_{i = 1}^{r}\left( \frac{\varepsilon_i^{(i_0)}}{\varepsilon_i^{(j)}}\right)^{a_i} \prod_{i = 1}^{\nu} \left( \frac{\gamma_i^{(i_0)}}{\gamma_i^{(j)}}\right)^{n_i} \right) = \tau(\mu) - 1 = e^{\Lambda_{\tau}} - 1,\]
the power series definition of exponential function gives
\[\tau(\lambda) = e^{\Lambda_{\tau}} - 1 = \sum_{n=0}^{\infty}\frac{\Lambda_{\tau}^n}{n!} - 1 = \sum_{n=1}^{\infty}\frac{\Lambda_{\tau}^n}{n!} = \Lambda_{\tau} + \sum_{n=2}^{\infty}\frac{\Lambda_{\tau}^{n}}{n!} = \Lambda_{\tau}\left( 1 + \sum_{n=2}^{\infty}\frac{\Lambda_{\tau}^{n-1}}{n!} \right).\]

If $\Lambda_\tau\geq 0$ then $1+\sum_{n\geq 2} (\Lambda_\tau)^{n-1}/n!>1$ which implies that 
\[|\Lambda_{\tau}| \leq |\Lambda_{\tau}|\left|1+\sum_{n\geq 2} \frac{(\Lambda_\tau)^{n-1}}{n!}\right| =|\tau(\lambda)| .\]

Suppose now that $\Lambda_\tau<0$. Our assumption 
\[ h_{\tau}\left(\frac{\delta_2}{\lambda}\right) = \log \max \left\{ \left|\tau\left(\frac{\delta_2}{\lambda}\right)\right|, 1\right\} > c_\tau = \log \max\{2|\tau(\delta_2)|,1\}\]
means that $|\tau(\lambda)| < 1/2$. 
That is, 
\[-\frac{1}{2} < \tau(\lambda) < \frac{1}{2} \implies \frac{1}{2} < \tau(\lambda) + 1 < \frac{3}{2} \implies \log\left(\frac{1}{2}\right) < \log(\tau(\lambda) + 1) < \log\left(\frac{3}{2}\right).\]
In particular, 
\[-\log\left(1/2\right) > -\log(\tau(\lambda) + 1).\]
Together with 
\[\tau(\lambda) + 1 = \tau(\mu) \implies \log(\tau(\lambda) + 1)= \log(\tau(\mu)) = \Lambda_{\tau} < 0,\]
this means that
\[|\Lambda_\tau|=-\log (\tau(\lambda)+1)\leq -\log(1/2)=\log 2.\]
Therefore
\begin{align*}
\left| \sum_{n\geq 2} \frac{(\Lambda_\tau)^{n-1}}{n!}\right| 
	& \leq \sum_{n\geq 2} \frac{|\Lambda_\tau|^{n-1}}{n!}\\	
	& =\sum_{n\geq 1} \frac{|\Lambda_\tau|^{n}}{(n+1)!}\\
	& = \frac{|\Lambda_\tau|}{1\cdot 2}+ \frac{|\Lambda_\tau|^{2}}{1\cdot 2\cdot 3}+\frac{|\Lambda_\tau|^{3}}{1\cdot 2\cdot 3\cdot 4} + \frac{|\Lambda_\tau|^{4}}{1\cdot 2\cdot 3\cdot 4\cdot 5} + \cdots\\
	& = \frac{1}{2} \left(\frac{|\Lambda_\tau|}{1}+ \frac{|\Lambda_\tau|^{2}}{1\cdot 3}+\frac{|\Lambda_\tau|^{3}}{1\cdot 3\cdot 4} + \frac{|\Lambda_\tau|^{4}}{1\cdot 3\cdot 4\cdot 5} + \cdots\right)\\
	& \leq \frac{1}{2} \left(\frac{|\Lambda_\tau|}{1}+ \frac{|\Lambda_\tau|^{2}}{1\cdot 2}+\frac{|\Lambda_\tau|^{3}}{1\cdot 2\cdot 3} + \frac{|\Lambda_\tau|^{4}}{1\cdot 2\cdot 3\cdot 4} + \cdots\right)\\
	&= \frac{1}{2} \left(\sum_{n \geq 1} \frac{|\Lambda_{\tau}|^n}{n!}\right)\\
	&= \frac{1}{2} \left(\sum_{n \geq 0} \frac{|\Lambda_{\tau}|^n}{n!} - 1\right)\\
	&= \frac{1}{2} (e^{|\Lambda_{\tau}|} - 1)\\
	& \leq \frac{1}{2} (e^{\log{2}} - 1) = \frac{1}{2}
\end{align*}
where the second inequality follows from the fact that $\frac{1}{1\cdot 3 \cdot 4 \cdots n} \leq \frac{1}{1\cdot 2\cdots (n-1)}$ since for $n \geq 3$,  
\[2 \leq n \implies 1\cdot 2 \cdot 3 \cdots (n-1) < 1\cdot 3 \cdot 4\cdots n.\]

More generally, applying the same idea as above for any even $N\geq 2$, we obtain 
\begin{align*}
\left| \sum_{n\geq 2} \frac{(\Lambda_\tau)^{n-1}}{n!}\right| 
	& =\left|\sum_{n\geq 1} \frac{\Lambda_\tau^{n}}{(n+1)!}\right|\\
	& = \left|\frac{\Lambda_\tau}{1\cdot 2}+ \frac{\Lambda_\tau^{2}}{1\cdot 2\cdot 3}+\frac{\Lambda_\tau^{3}}{1\cdot 2\cdot 3\cdot 4} + \frac{\Lambda_\tau^{4}}{1\cdot 2\cdot 3\cdot 4\cdot 5} + \cdots\right|\\
	& = \left|\sum_{n = 1}^N \frac{\Lambda_\tau^{n}}{(n+1)!} + \frac{\Lambda_\tau^{N+1}}{1\cdots (N+2)}+ \frac{\Lambda_\tau^{N+2}}{1\cdots (N+3)}+\frac{\Lambda_\tau^{N+3}}{1\cdots (N+4)} + \cdots\right|\\
	& = \left|\sum_{n = 1}^N \frac{\Lambda_\tau^{n}}{(n+1)!} + \frac{1}{N+2}\left(\frac{\Lambda_\tau^{N+1}}{1\cdots (N+1)}+ \frac{\Lambda_\tau^{N+2}}{1\cdots (N+1)\cdot(N+3)} + \cdots\right)\right|\\
	& \leq \left|\sum_{n = 1}^N \frac{\Lambda_\tau^{n}}{(n+1)!}\right| + \frac{1}{N+2}\left|\sum_{n = N+1}^{\infty} \frac{\Lambda_\tau^{n}}{n!}\right|\\
	& \leq \left|\sum_{n = 1}^N \frac{\Lambda_\tau^{n}}{(n+1)!}\right| + \frac{1}{N+2}\left(\sum_{n = N+1}^{\infty} \frac{|\Lambda_\tau|^{n}}{n!}\right)\\
	& = \left|\sum_{n = 1}^N \frac{\Lambda_\tau^{n}}{(n+1)!}\right| + \frac{1}{N+2}\left(\sum_{n =0}^{\infty} \frac{|\Lambda_\tau|^{n}}{n!} - \sum_{n =0}^{N} \frac{|\Lambda_\tau|^{n}}{n!}\right)\\
	& \leq \left|\sum_{n = 1}^N \frac{\Lambda_\tau^{n}}{(n+1)!}\right| + \frac{1}{N+2}\left(\sum_{n =0}^{\infty} \frac{|\Lambda_\tau|^{n}}{n!} \right)\\
	& = \left|\sum_{n = 1}^N \frac{\Lambda_\tau^{n}}{(n+1)!}\right| + \frac{1}{N+2}e^{|\Lambda_{\tau}|}\\
	& \leq \left|\sum_{n = 1}^N \frac{\Lambda_\tau^{n}}{(n+1)!}\right| + \frac{1}{N+2}e^{\log{2}}\\
	& = \left|\sum_{n = 1}^N \frac{\Lambda_\tau^{n}}{(n+1)!}\right| + \frac{2}{N+2}:= k_N.
\end{align*}

We now give an upper bound for $k_N$. Since $\Lambda_\tau<0$, we obtain
\begin{align*}
\sum_{n = 1}^N \frac{\Lambda_\tau^{n}}{(n+1)!}
	& = \frac{\Lambda_\tau}{2!}+ \frac{\Lambda_\tau^{2}}{3!}+\frac{\Lambda_\tau^{3}}{4!} + \frac{\Lambda_\tau^{4}}{5!} + \cdots\\
	& =  \frac{|\Lambda_\tau|^{2}}{3!} - \frac{-\Lambda_\tau}{2!}+\frac{|\Lambda_\tau|^{4}}{5!} -\frac{-\Lambda_\tau^{3}}{4!} + \cdots\\
	& =  \frac{|\Lambda_\tau|^{2}}{3!} - \frac{|\Lambda_\tau|}{2!}+\frac{|\Lambda_\tau|^{4}}{5!} -\frac{|\Lambda_\tau|^{3}}{4!} + \cdots\\
	& = \sum_{\substack{n = 2 \\ n \mid 2}}^N \left(\frac{|\Lambda_\tau|^{n}}{(n+1)!} - \frac{|\Lambda_\tau|^{n-1}}{n!}\right)\\
	& = \sum_{\substack{n = 2 \\ n \mid 2}}^N \frac{|\Lambda_\tau|^{n-1}}{n!}\left(\frac{|\Lambda_\tau|}{n+1} - 1\right)\\
	& =  \frac{|\Lambda_\tau|}{2}\left(\frac{|\Lambda_\tau|}{3} - 1\right)+ \sum_{\substack{n = 4 \\ n \mid 2}}^N \frac{|\Lambda_\tau|^{n-1}}{n!}\left(\frac{|\Lambda_\tau|}{n+1} - 1\right)\\
	& \geq  \frac{|\Lambda_\tau|}{2}\left(\frac{|\Lambda_\tau|}{4} - 1\right)+ \sum_{\substack{n = 4 \\ n \mid 2}}^N \frac{|\Lambda_\tau|^{n-1}}{n!}\left(\frac{|\Lambda_\tau|}{n+1} - 1\right)\\
\end{align*}
$$\sum_{n\geq 2} (\Lambda_\tau)^{n-1}/n!=\sum_{N\geq n\geq 1} (\Lambda_\tau)^{n}/(n+1)!=\sum_{N\geq n\geq 2, \, 2\mid n}\tfrac{|\Lambda_\tau|^n}{(n+1)!}-\tfrac{|\Lambda_\tau|^{n-1}}{n!}$$
$$=\sum_{N\geq n\geq 2, \, 2\mid n}\tfrac{|\Lambda_\tau|^{n-1}}{n!}(\tfrac{|\Lambda_\tau|}{n+1}-1)=\tfrac{|\Lambda_\tau|}{2}(\tfrac{|\Lambda_\tau|}{3}-1)+\sum_{N\geq n\geq 4, \, 2\mid n}\tfrac{|\Lambda_\tau|^{n-1}}{n!}(\tfrac{|\Lambda_\tau|}{n+1}-1)$$
$$\geq \tfrac{\log 2}{2}(\tfrac{\log 2}{4}-1)+\sum_{N\geq n\geq 4, \, 2\mid n}\tfrac{(\log 2)^{n-1}}{n!}(\tfrac{3/4(\log 2)}{n+1}-1):=-k_N.$$
The last inequality follows by distinguishing two cases whether  $|\Lambda_\tau|\leq 3/4\cdot \log 2$ or not; note that $ln(2)/2*(ln(2)/4-1)/(-ln (2)*3/8)\geq 1$.  Now, on using that $-k_N$ is negative, it follows that $|1+\sum_{n\geq 2} (\Lambda_\tau)^{n-1}/n!|\geq 1-|\sum_{n\geq 2} (\Lambda_\tau)^{n-1}/n!|\geq 1-k_N$ and thus 
$$|\Lambda_\tau|\leq \kappa_\tau|\tau(x)|, \quad \kappa_\tau=\tfrac{1}{1-k_N}|\tau(\lambda_0)|,  \quad c_\tau=\log^+(2|\lambda_0|).$$
%The constant $\kappa_\tau$ depends on $N$ which can be taken arbitrarily as long as $N\geq 2$ is even. Further, the value $k_N$ can be slightly improved when one finds the maximum of the functions $x^{n-1}(\tfrac{x}{n+1}-1)$ on the interval $[0,\log 2]$ for each even $n\geq 2$.\end{proof}

DETAILS HERE NEED TO BE CLARIFIED, SO FOR NOW, WE WILL JUST TAKE $k_N = 1/2$
That is, 
\[\left| \sum_{n\geq 2} \frac{(\Lambda_\tau)^{n-1}}{n!}\right| \leq \frac{1}{2}\]
hence... ACTUALLY I DON'T UNDERSTAND THIS PROOF AT ALL, SO WE WILL TAKE $k_N = 1/2$, giving us 
\[|\Lambda_\tau|\leq \kappa_\tau|\tau(x)|, \quad \kappa_\tau=\tfrac{1}{1-k_N}|\tau(\lambda_0)|,  \quad c_\tau=\log^+(2|\lambda_0|),\] 
hence
\[\kappa_\tau=\tfrac{1}{1-1/2}|\tau(\lambda_0)| = 2|\tau(\lambda_0)| \implies |\Lambda_\tau|\leq \kappa_\tau|\tau(x)| = 2|\tau(\lambda_0)||\tau(x)|.\]
Now, 
\begin{align*}
-h_{\tau}\left(\frac{\delta_2}{\lambda}\right) = -\log \max \left\{ \left|\tau\left(\frac{\delta_2}{\lambda}\right)\right|, 1\right\} \implies e^{-h_{\tau}\left(\frac{\delta_2}{\lambda}\right)} 
	&= e^{-\log \max \left\{ \left|\tau\left(\frac{\delta_2}{\lambda}\right)\right|, 1\right\}}\\
	&= e^{\log \left(\max \left\{ \left|\tau\left(\frac{\delta_2}{\lambda}\right)\right|, 1\right\}\right)^{-1}}\\
	& = \left(\max \left\{ \left|\tau\left(\frac{\delta_2}{\lambda}\right)\right|, 1\right\}\right)^{-1}\\
	& = \frac{1}{\max \left\{ \left|\tau\left(\frac{\delta_2}{\lambda}\right)\right|, 1\right\}}\\
	& = \max \left\{ \left|\tau\left(\frac{\lambda}{\delta_2}\right)\right|, 1\right\}\\
	& = \max \left\{ \left|\tau(x)\right|, 1\right\}.
\end{align*}
In addition, 
\[ \left|\tau\left(\frac{\delta_2}{\lambda}\right)\right| \leq \max \left\{ \left|\tau\left(\frac{\delta_2}{\lambda}\right)\right|, 1\right\} \implies \frac{1}{\max \left\{ \left|\tau\left(\frac{\delta_2}{\lambda}\right)\right|, 1\right\}} \leq \frac{1}{\left|\tau\left(\frac{\delta_2}{\lambda}\right)\right|} = {\left|\tau\left(\frac{\lambda}{\delta_2}\right)\right|}=| \tau(x)|.\]
Therefore, all together, we have
\[ |\Lambda_\tau|\leq \kappa_\tau|\tau(x)| \leq \kappa_{\tau}\max\left\{|\tau(x)|, 1\right\} = \kappa_{\tau}e^{-h_{\tau}\left(\frac{\delta_2}{\lambda}\right)} \leq \kappa_{\tau} e^{-l_{\tau}}\]
NO THIS STILL DOESN'T MAKE SENSE. LET'S JUST GO WITH RAFAELS BOUND with $k_{\tau} = 2|\tau(\delta_2)|$







Altogether, we now have
\begin{align*}
&\left|\alpha_0+\sum_{i = 1}^r a_i \alpha_{\varepsilon i} + \sum_{i = 1}^{\nu} n_i \alpha_{\gamma i}\right|\\
	& \leq \frac{1}{2}\left(\frac{1}{[L:\mathbb{Q}]}\sum_{\sigma :L \to \mathbb{C}} w_{\sigma}\log \max \left\{ \left|\sigma\left(\frac{\delta_2}{\lambda}\right)\right|, 1\right\} + \frac{1}{[K:\mathbb{Q}]}\sum_{l = 1}^{\nu}w_l \log(p_l)|u_l - r_l|\right) + \\
	& \quad \quad \quad + \left(\frac{1}{2} + c\kappa_{\tau}e^{-h_{\tau}\left(\frac{\delta_2}{\lambda}\right)}\right).
\end{align*} 

%
%If $v=p$ then Lemma~\ref{} gives optimal $c_v$ and $\kappa_v$. In the real case, if $v=\tau:L\to \mathbb{C}$ then we can take $\kappa_\tau$ as defined in \eqref{} and $c_v=\log 2$.
%
%Let now  $v=\tau$. Suppose first that we are in the real case. It holds that $\mu-1=\lambda$ and $\Lambda_\tau=\log \tau (\mu)$. Then, on using power series definition of exponential function, we obtain $$\Lambda_\tau(1+\sum_{n\geq 2} (\Lambda_\tau)^{n-1}/n!)=\Lambda_\tau+\sum_{n\geq 2} (\Lambda_\tau)^n/n!=e^{\Lambda_\tau}-1=\tau(\lambda).$$
%If $\Lambda_\tau\geq 0$ then $1+\sum_{n\geq 2} (\Lambda_\tau)^{n-1}/n!>1$ which implies that $|\Lambda_\tau|\leq |\tau(\lambda)|.$ Suppose now that $\Lambda_\tau<0$. Our assumption $h_v(z)\geq \log 2$ means that $|\tau(z)|\leq 1/2$ and thus $|\Lambda_\tau|=-\log (\tau(z)+1)\leq -\log(1/2)=\log 2.$ Therefore, the absolute value of $\sum_{n\geq 2} (\Lambda_\tau)^{n-1}/n!$ is at most $$\sum_{n\geq 2} |\Lambda_\tau|^{n-1}/n!=\sum_{n\geq 1} |\Lambda_\tau|^{n}/(n+1)!\leq \tfrac{1}{2}\sum_{n\geq 1} |\Lambda_\tau|^{n}/n!\leq
%\tfrac{1}{2}e^{\log 2}-1/2=1/2.$$
%More precisely, for any even $N\geq 2$, we obtain 
%$$|\sum_{n\geq 2} (\Lambda_\tau)^{n-1}/n!|=|\sum_{n\geq 1} (\Lambda_\tau)^{n}/(n+1)!|\leq |\sum_{N\geq n\geq 1} (\Lambda_\tau)^{n}/(n+1)!|+\tfrac{1}{N+2}|\sum_{n>N} (\Lambda_\tau)^{n}/n!|$$
%$$\leq |\sum_{N\geq n\geq 1} (\Lambda_\tau)^{n}/(n+1)!|+\tfrac{1}{N+2}e^{|\Lambda_\tau|}\leq |\sum_{N\geq n\geq 1} (\Lambda_\tau)^{n}/(n+1)!|+\tfrac{2}{N+2}:=k_N.$$
%We now give an upper bound for $k_N$. Since $\Lambda_\tau<0$, we obtain
%$$\sum_{n\geq 2} (\Lambda_\tau)^{n-1}/n!=\sum_{N\geq n\geq 1} (\Lambda_\tau)^{n}/(n+1)!=\sum_{N\geq n\geq 2, \, 2\mid n}\tfrac{|\Lambda_\tau|^n}{(n+1)!}-\tfrac{|\Lambda_\tau|^{n-1}}{n!}$$
%$$=\sum_{N\geq n\geq 2, \, 2\mid n}\tfrac{|\Lambda_\tau|^{n-1}}{n!}(\tfrac{|\Lambda_\tau|}{n+1}-1)=\tfrac{|\Lambda_\tau|}{2}(\tfrac{|\Lambda_\tau|}{3}-1)+\sum_{N\geq n\geq 4, \, 2\mid n}\tfrac{|\Lambda_\tau|^{n-1}}{n!}(\tfrac{|\Lambda_\tau|}{n+1}-1)$$
%$$\geq \tfrac{\log 2}{2}(\tfrac{\log 2}{4}-1)+\sum_{N\geq n\geq 4, \, 2\mid n}\tfrac{(\log 2)^{n-1}}{n!}(\tfrac{3/4(\log 2)}{n+1}-1):=-k_N.$$
%The last inequality follows by distinguishing two cases whether  $|\Lambda_\tau|\leq 3/4\cdot \log 2$ or not; note that $ln(2)/2*(ln(2)/4-1)/(-ln (2)*3/8)\geq 1$.  Now, on using that $-k_N$ is negative, it follows that $|1+\sum_{n\geq 2} (\Lambda_\tau)^{n-1}/n!|\geq 1-|\sum_{n\geq 2} (\Lambda_\tau)^{n-1}/n!|\geq 1-k_N$ and thus 
%$$|\Lambda_\tau|\leq \kappa_\tau|\tau(z)|, \quad \kappa_\tau=\tfrac{1}{1-k_N},  \quad c_\tau=\log 2.$$
%The constant $\kappa_\tau$ depends on $N$ which can be taken arbitrarily as long as $N\geq 2$ is even. Further, the value $k_N$ can be slightly improved when one finds the maximum of the functions $x^{n-1}(\tfrac{x}{n+1}-1)$ on the interval $[0,\log 2]$ for each even $n\geq 2$. 
\end{proof}

Recall that
\[h\left(\frac{\delta_2}{\lambda}\right) =  \frac{1}{[K:\mathbb{Q}]}\sum_{l = 1}^{\nu} \log(p_l)|u_l - r_l| + \frac{1}{[L:\mathbb{Q}]}\sum_{w :L \to \mathbb{C}} \log \max \left\{ \left|w\left(\frac{\delta_2}{\lambda}\right)\right|, 1\right\}.\]

Take $\mathbf{h}\in\mathbb{R}^{r+\nu}$ such that $\mathbf{h}\geq \mathbf{0}$. Let $\mathbf{m} = (n_1, \dots, n_{\nu}, a_1, \dots, a_r) \in \mathbb{R}^{r + \nu}$ be any solution of \eqref{Eq:main3} with 
\[h_{\tau}\left(\frac{\delta_2}{\lambda}\right) \geq l_{\tau}\]
where we denote by $h_{v}\left(\frac{\delta_2}{\lambda}\right)$ the $v^{\text{th}}$ entry of the vector
\[\left(\log (p_1)|u_1 - r_1|, \dots, \log(p_{\nu})|u_{\nu} - r_{\nu}|, \max \left\{ \left|\tau_1\left(\frac{\delta_2}{\lambda}\right)\right|, 1\right\}, \dots, \log \max \left\{ \left|\tau_n\left(\frac{\delta_2}{\lambda}\right)\right|, 1\right\}\right).\] 
Since 
\[h_{\tau}\left(\frac{\delta_2}{\lambda}\right) \geq l_{\tau} > c_{\tau},\]
the previous lemma holds. That is, 
\begin{align*}
&\left|\alpha_0+\sum_{i = 1}^r a_i \alpha_{\varepsilon i} + \sum_{i = 1}^{\nu} n_i \alpha_{\gamma i}\right|\\
	& \leq \frac{1}{2}\left(\frac{1}{[K:\mathbb{Q}]}\sum_{l = 1}^{\nu}w_l \log(p_l)|u_l - r_l| + \frac{1}{[L:\mathbb{Q}]}\sum_{\sigma :L \to \mathbb{C}} w_{\sigma}\log \max \left\{ \left|\sigma\left(\frac{\delta_2}{\lambda}\right)\right|, 1\right\} \right) + \\
	& \quad \quad \quad + \left(\frac{1}{2} + c\kappa_{\tau}e^{-h_{\tau}\left(\frac{\delta_2}{\lambda}\right)}\right).
\end{align*} 


Suppose $h_{v}\left(\frac{\delta_2}{\lambda}\right) \leq h_v$ for all $v\in \{1, \dots, r+\nu\}$. Then, since
\[l_{\tau} \leq h_{\tau}\left(\frac{\delta_2}{\lambda}\right) \leq h_{\tau} \implies -h_{\tau} \leq  -h_{\tau}\left(\frac{\delta_2}{\lambda}\right) \leq -l_{\tau} \implies e^{-h_{\tau}} \leq e^{-h_{\tau}\left(\frac{\delta_2}{\lambda}\right)} \leq e^{-l_{\tau}},\]
 we deduce
\begin{align*}
&\left|\alpha_0+\sum_{i = 1}^r a_i \alpha_{\varepsilon i} + \sum_{i = 1}^{\nu} n_i \alpha_{\gamma i}\right|\\
	& \leq \frac{1}{2}\left(\frac{1}{[K:\mathbb{Q}]}\sum_{l = 1}^{\nu}w_l \log(p_l)|u_l - r_l| + \frac{1}{[L:\mathbb{Q}]}\sum_{\sigma :L \to \mathbb{C}} w_{\sigma}\log \max \left\{ \left|\sigma\left(\frac{\delta_2}{\lambda}\right)\right|, 1\right\} \right) + \\
	& \quad \quad \quad + \left(\frac{1}{2} + c\kappa_{\tau}e^{-h_{\tau}\left(\frac{\delta_2}{\lambda}\right)}\right)\\
	& \leq \frac{1}{2}\left(\frac{1}{[K:\mathbb{Q}]}\sum_{l = 1}^{\nu}w_l h_l + \frac{1}{[L:\mathbb{Q}]}\sum_{\sigma :L \to \mathbb{C}} w_{\sigma}h_{\sigma} \right) + \left(\frac{1}{2} + c\kappa_{\tau}e^{-h_{\tau}\left(\frac{\delta_2}{\lambda}\right)}\right)\\
	& \leq \frac{1}{2}\left(\frac{1}{[K:\mathbb{Q}]}\sum_{l = 1}^{\nu}w_l h_l + \frac{1}{[L:\mathbb{Q}]}\sum_{\sigma :L \to \mathbb{C}} w_{\sigma}h_{\sigma} \right) + \frac{1}{2} + c\kappa_{\tau}e^{-l_{\tau}}
\end{align*} 
%
%\begin{align*}
%& \left|\alpha_0+\sum_{i = 1}^r a_i \alpha_{\varepsilon i} + \sum_{i = 1}^{\nu} n_i \alpha_{\gamma i}\right| ^2 \\ & \leq
%\begin{cases}
%\left( \frac{2}{[L:\mathbb{Q}]}\max_{\sigma:L\to \mathbb{C}} w_{\sigma}h_{\sigma}  + \frac{1}{[K:\mathbb{Q}]}\sum_{k = 1}^{\nu} w_{k}h_k + \frac{1}{2} + c\kappa_{\tau}e^{-l_{\tau}}\right)^2 & \text{ if } \sqrt{\Delta}\notin\mathbb{Q} \\
%\left( \frac{1}{[L:\mathbb{Q}]}\max_{\sigma:L\to \mathbb{C}} w_{\sigma}h_{\sigma}  + \frac{1}{[K:\mathbb{Q}]}\sum_{k = 1}^{\nu} w_{k}h_k + \frac{1}{2} + c\kappa_{\tau}e^{-l_{\tau}}\right)^2 & \text{ if } \sqrt{\Delta}\in\mathbb{Q}.\\
%\end{cases}
%\end{align*}

%We now fix $\epsilon^*\in\unit_\infty$ and we take $h\in\RR^{S^*}$ with $h\geq 0$.  Let $(x,y)\in\Sigma$ with $m\in\ZZ^\unit$ and $h_\tau(z)\geq l_\tau$, and suppose that $h_v(z)\leq h_v$ for all $v\in S^*$. Then, on combining Lemma~\ref{lem:archellest} with \eqref{eq:hvjbound}, we see that  $\left|\alpha_0+\sum_{u\in\unit}m_u\alpha_u\right|^2$ is at most
%\begin{equation}\label{def:bepsstarbound}
%b_{\epsilon^*}=\left(\sum_{v:L\to\CC}w_{v}h_v+\sum_{v\in T}\max\bigl(w_{v}h_v,w_{v^{(j)}}a_p\log N(v^{(j)})\bigl)+c\kappa_\tau e^{-l_{\tau}}\right)^2.
%\end{equation}
%By Remark~\ref{rem:i12}, we can replace $\sum_{v:L\to\CC}w_{v}h_v$ by $3[L:K]\|r_\epsilon\|_\infty\max_{v:L\to\CC}h_v$ if $|I|=2$.
%

We finally can define the ellipsoid. Let
\[b = \frac{1}{\log(2)^2}\sum_{k = 1}^{\nu} h_k^2\]
where
\[\log(2)^2q_f(\mathbf{n}) = \log(2)^2\sum_{k = 1}^{\nu}\left\lfloor\frac{\log(p_k)^2}{\log(2)^2}\right\rfloor|u_k-r_k|^2 \leq \sum_{k = 1}^{\nu} \log(p_k)^2|u_k -r_k|^2 \leq \sum_{k = 1}^{\nu} h_k^2.\]

%\[q_f(\mathbf{n}) = \sum_{l = 1}^{\nu} \lfloor\log(p_l)^2\rfloor|u_l -r_l|^2 \leq \sum_{l = 1}^{\nu} \log(p_l)^2|u_l -r_l|^2 \leq \sum_{l = 1}^{\nu} h_k^2:= b \]

%\[q_f(\mathbf{n}) = \frac{1}{[K:\mathbb{Q}]}\sum_{k = 1}^{\nu} \log(p_k)^2|u_k -r_k|^2 \leq \frac{1}{[K:\mathbb{Q}]}\sum_{k = 1}^{\nu} h_k^2 = b.\]
For each $\varepsilon_l$ in $\{\varepsilon_1, \dots, \varepsilon_r\}$ such that $\varepsilon_l \neq \varepsilon_l^*$, we define
%\[b_{\varepsilon_l} = 
%\begin{cases}
%\left( \frac{2}{[L:\mathbb{Q}]}\max_{\sigma:L\to \mathbb{C}} w_{\varepsilon \sigma}h_{\sigma}  + \frac{1}{[K:\mathbb{Q}]}\sum_{k = 1}^{\nu} w_{\gamma k}h_k\right)^2 & \text{ if } \sqrt{\Delta}\notin\mathbb{Q} \\
%\left( \frac{1}{[L:\mathbb{Q}]}\max_{\sigma:L\to \mathbb{C}} w_{\varepsilon \sigma}h_{\sigma} + \frac{1}{[K:\mathbb{Q}]}\sum_{k = 1}^{\nu} w_{\gamma k}h_k\right)^2 & \text{ if } \sqrt{\Delta}\in\mathbb{Q}, \\
%\end{cases}\]
%where
\[|a_l|^2 \leq \left( \frac{1}{[K:\mathbb{Q}]}\sum_{k = 1}^{\nu} w_{\gamma l k}h_k + \frac{1}{[L:\mathbb{Q}]}\sum_{\sigma:L\to \mathbb{C}} w_{\varepsilon l \sigma}h_{\sigma}\right)^2=:b_{\varepsilon_l} \]

%
%
%\[|a_l|^2 \leq b_{\varepsilon_l} =
%\begin{cases}
%\left( \frac{2}{[L:\mathbb{Q}]}\max_{\sigma:L\to \mathbb{C}} w_{\varepsilon \sigma}h_{\sigma}  + \frac{1}{[K:\mathbb{Q}]}\sum_{k = 1}^{\nu} w_{\gamma k}h_k\right)^2 & \text{ if } \sqrt{\Delta}\notin\mathbb{Q} \\
%\left( \frac{1}{[L:\mathbb{Q}]}\max_{\sigma:L\to \mathbb{C}} w_{\varepsilon \sigma}h_{\sigma} + \frac{1}{[K:\mathbb{Q}]}\sum_{k = 1}^{\nu} w_{\gamma k}h_k\right)^2 & \text{ if } \sqrt{\Delta}\in\mathbb{Q}. \\
%\end{cases}\]
Now, for $\varepsilon_l$ in $\{\varepsilon_1, \dots, \varepsilon_r\}$ such that $\varepsilon_l = \varepsilon_l^*$, we define
\begin{align*}
&\left|\alpha_0+\sum_{i = 1}^r a_i \alpha_{\varepsilon i} + \sum_{i = 1}^{\nu} n_i \alpha_{\gamma i}\right|^2\\
	& \leq \left(\frac{1}{2}\left(\frac{1}{[K:\mathbb{Q}]}\sum_{l = 1}^{\nu}w_l h_l + \frac{1}{[L:\mathbb{Q}]}\sum_{\sigma :L \to \mathbb{C}} w_{\sigma}h_{\sigma}\right) + \frac{1}{2} + c\kappa_{\tau}e^{-l_{\tau}}\right)^2=:b_{\varepsilon_l}. 
\end{align*} 

%
%\[b_{\varepsilon_l} = 
%\begin{cases}
%\left( \frac{2}{[L:\mathbb{Q}]}\max_{\sigma:L\to \mathbb{C}} w_{\sigma}h_{\sigma}  + \frac{1}{[K:\mathbb{Q}]}\sum_{k = 1}^{\nu} w_{k}h_k + \frac{1}{2} + c\kappa_{\tau}e^{-l_{\tau}}\right)^2 & \text{ if } \sqrt{\Delta}\notin\mathbb{Q} \\
%\left( \frac{1}{[L:\mathbb{Q}]}\max_{\sigma:L\to \mathbb{C}} w_{\sigma}h_{\sigma}  + \frac{1}{[K:\mathbb{Q}]}\sum_{k = 1}^{\nu} w_{k}h_k + \frac{1}{2} + c\kappa_{\tau}e^{-l_{\tau}}\right)^2 & \text{ if } \sqrt{\Delta}\in\mathbb{Q}\\
%\end{cases}\]
%where
%\begin{align*}
%\left|\alpha_0+\sum_{i = 1}^r a_i \alpha_{\varepsilon i} + \right.& \left.\sum_{i = 1}^{\nu} n_i \alpha_{\gamma i}\right| ^2 
%	\leq b_{\varepsilon_l} \\
%\\ & =
%\begin{cases}
%\left( \frac{2}{[L:\mathbb{Q}]}\max_{\sigma:L\to \mathbb{C}} w_{\sigma}h_{\sigma}  + \frac{1}{[K:\mathbb{Q}]}\sum_{k = 1}^{\nu} w_{k}h_k + \frac{1}{2} + c\kappa_{\tau}e^{-l_{\tau}}\right)^2 & \text{ if } \sqrt{\Delta}\notin\mathbb{Q} \\
%\left( \frac{1}{[L:\mathbb{Q}]}\max_{\sigma:L\to \mathbb{C}} w_{\sigma}h_{\sigma}  + \frac{1}{[K:\mathbb{Q}]}\sum_{k = 1}^{\nu} w_{k}h_k + \frac{1}{2} + c\kappa_{\tau}e^{-l_{\tau}}\right)^2 & \text{ if } \sqrt{\Delta}\in\mathbb{Q}.\\
%\end{cases}
%\end{align*}

Let
\[\mathbf{x} = (x_1, \dots, x_{\nu}, x_{\varepsilon_1}, \dots, x_{\varepsilon_{r}}) \in \mathbb{R}^{\nu + r}.\]
Then we define the ellipsoid $\mathcal{E_\tau}\subseteq \mathbb{R}^{r+\nu}$ by
\begin{align*}\label{def:ellreal}
& \mathcal{E_\tau}=\{q_\tau(\mathbf{x})\leq (1 + r)(bb_{\varepsilon_1}\cdots b_{\varepsilon_r}); \ \mathbf{x}\in\mathbb{R}^{r+\nu}\}, \quad \text{ where }\\
&\quad \quad \quad q_{\tau}(\mathbf{x})= (b_{\varepsilon_1}\cdots b_{\varepsilon_r})\left( q_f(x_1, \dots, x_{\nu}) + \sum_{i = 1}^r\frac{b}{b_{\varepsilon_i}}x_{\varepsilon_i}^2\right)\\
&\quad \quad \quad q_{\tau}(\mathbf{x})=\left((b_{\varepsilon_1}\cdots b_{\varepsilon_r})\cdot q_f(x_1, \dots, x_{\nu}) + (b_{\varepsilon_1}\cdots b_{\varepsilon_r})\sum_{i = 1}^r\frac{b}{b_{\varepsilon_i}}x_{\varepsilon_i}^2\right)\\
&\quad \quad \quad q_{\tau}(\mathbf{x})=\left((b_{\varepsilon_1}\cdots b_{\varepsilon_r})\cdot q_f(x_1, \dots, x_{\nu}) + \sum_{i = 1}^rb(b_{\varepsilon_1}\cdots b_{\varepsilon_{i-1}}b_{\varepsilon_{i+1}}b_{\varepsilon_r})x_{\varepsilon_i}^2\right)
\end{align*}
where
\[q_f(\mathbf{y}) = (A\mathbf{y})^{\text{T}}D^2A\mathbf{y}.\]

Now, if 
\[\mathbf{x} = (x_1, \dots, x_{\nu}, x_{\varepsilon_1}, \dots, x_{\varepsilon_{r}}) \in \mathbb{R}^{\nu + r}\]
is a solution, it follows that
\begin{align*}
q_{\tau}(\mathbf{x})
	& =\left((b_{\varepsilon_1}\cdots b_{\varepsilon_r})\cdot q_f(x_1, \dots, x_{\nu}) + \sum_{i = 1}^rb(b_{\varepsilon_1}\cdots b_{\varepsilon_{i-1}}b_{\varepsilon_{i+1}}b_{\varepsilon_r})x_{\varepsilon_i}^2\right)\\
	& \leq (b_{\varepsilon_1}\cdots b_{\varepsilon_r})\cdot b + \sum_{i = 1}^rb(b_{\varepsilon_1}\cdots b_{\varepsilon_{i-1}}b_{\varepsilon_{i+1}}b_{\varepsilon_r})b_{\varepsilon_i}\\
	& = (1+r)(bb_{\varepsilon_1}\cdots b_{\varepsilon_r}).
\end{align*}

%---------------------------------------------------------------------------------------------------------------------------------------------%
\subsection{Archimedean sieve: Real case, $r = 2$}

Suppose that all roots of $f$ are real so that $r = 2$. Let $\tau:L\to\mathbb{C}$ be an embedding. We take $l,h \in \mathbb{R}^{m+\nu}$ with $0\leq l\leq h$ and $l_\tau\geq \log 2$. Then we consider the 
translated lattice $\Gamma_{\tau}\subset \mathbb{Z}^{r + \nu}$ defined by 
\[\Gamma_{\tau} =\Phi_{\tau}(\mathbb{Z}^{r+\nu})+w\]
where $w=(0,\dotsc,0,\alpha_0)^{\text{T}}$ for $c$ a constant of the size $e^{l_\tau}$ and where $\Phi_\tau$ is a linear transformation which is the identity on $\mathbb{Z}^{u+r - 1}$ and which sends 
\[(0, \dots, 0, 1) \mapsto (\alpha_{\gamma 1}, \dots, \alpha_{\gamma {\nu}}, \alpha_{\varepsilon 1}, \dots, \alpha_{\varepsilon {r}}).\]
That is, 
\[\left( \left[c\log\left(\tau\left(\frac{\gamma_1^{(k)}}{\gamma_1^{(j)}}\right)\right)\right], \dots, \left[c\log\left(\tau\left(\frac{\gamma_{\nu}^{(k)}}{\gamma_{\nu}^{(j)}}\right)\right)\right], \left[c\log\left(\tau\left(\frac{\varepsilon_1^{(k)}}{\varepsilon_1^{(j)}}\right)\right)\right], \dots, \left[c\log\left(\tau\left(\frac{\varepsilon_r^{(k)}}{\varepsilon_r^{(j)}}\right)\right)\right] \right).\]
The matrix associated to this lattice is therefore
\[\Gamma_{\tau} = \begin{pmatrix}
	1 & 0 & \dots &  \dots & 0 & 0\\ 
	0 & 1	& \dots & \dots & 0 & 0\\
	\vdots & \vdots & \ddots & \dots & \vdots & \vdots \\ 
	0 & 0 & \dots &  \dots & 1 & 0\\ 
	\alpha_{\gamma 1} & \dots &\alpha_{\gamma {\nu}} & \alpha_{\varepsilon 1} & \dots & \alpha_{\varepsilon {r}}
\end{pmatrix}.\]

%for some fixed $\epsilon^*\in\unit_\infty$ and whose row indexed by $\epsilon^*$ is given by $(\alpha_u)\in\ZZ^\unit$. 
Let $\mathcal E_{\tau}=\mathcal E_{\tau}(h,l_{\tau})$ be the ellipsoid constructed in \eqref{def:ellreal}. Let ${\mathbf{m} = (n_1, \dots, n_{\nu}, a_1, \dots, a_r) \in \mathbb{R}^{r + \nu}}$ be any solution of \eqref{Eq:main3}. We say that $\mathbf{m}$ is determined by some $\mathbf{y} \in \Gamma_{\tau}$ if 
\[\mathbf{y} = (y_1, \dots, y_{r+ \nu}) = \left(n_1, \dots, n_{\nu}, a_1, \dots, a_{r-1}, \alpha_0+\sum_{i = 1}^r a_i \alpha_{\varepsilon i} + \sum_{i = 1}^{\nu} n_i \alpha_{\gamma i}\right)\]
where the missing element $a_{i}$ corresponds to $\varepsilon^*$.

\begin{lemma}\label{lem:archsieve}
Let ${\mathbf{m} = (n_1, \dots, n_{\nu}, a_1, \dots, a_r) \in \mathbb{R}^{r + \nu}}$ be any solution of \eqref{Eq:main3} which lies in $\Sigma_\tau(l,h)$. Then $\mathbf{m}$ is determined by some $\gamma\in \Gamma_\tau\cap\mathcal E_\tau.$
\end{lemma}

%\begin{proof}
%We define $\gamma\in \ZZ^\unit$ with $\gamma_u=m_u$ for each $u\in\unit\setminus \epsilon^*$ and $\gamma_{\epsilon^*}=\alpha_0+\sum_{u\in\unit} m_u \alpha_u$. Then $\gamma\in \Gamma_\tau$ and \eqref{def:bepsstarbound} implies that $\gamma_{\epsilon^*}^2\leq b_{\epsilon^*}$. Further \eqref{def:bbound} gives that $q_f(\gamma_\delta)\leq b$, and  \eqref{def:bepsbound} provides that $\gamma_\epsilon^2\leq b_\epsilon$ for each $\epsilon\in\unit_\infty$ with $\epsilon\neq \epsilon^*$. It follows that $$q_\tau(\gamma)\leq \frac{1}{|\unit|}(|\unit_S|b+\sum_{\epsilon\in\unit_\infty}\frac{b}{b_\epsilon}b_\epsilon)\leq b.$$
%This proves that $\gamma\in\mathcal E_\tau$ and hence the statement follows.
%\end{proof}
%
Suppose that $\gamma\in \Gamma_\tau\cap \mathcal E_\tau$. Let $M=M_\tau$ be the matrix defining the ellipsoid 
\[\mathcal E_\tau: z^tM^tMz\leq (1 + r)(bb_{\varepsilon_1}\cdots b_{\varepsilon_r}),\]
that is,  
\begin{align*}
M &=\sqrt{b_{\varepsilon_1}\cdots b_{\varepsilon_r}}\begin{pmatrix}
	DA & 0 & \dots & 0 & 0\\
	0 & \sqrt{\frac{b}{b_{\varepsilon 1}}} & \dots & 0 & 0\\
	0 & 0  & \sqrt{\frac{b}{b_{\varepsilon 2}}} & \dots & 0\\
	\vdots & \vdots &0 &  \ddots & \vdots\\ 
	0 & 0 & \dots & \dots & \sqrt{\frac{b}{b_{\varepsilon^*}}} \\
	\end{pmatrix}.
\end{align*}	
Note that we never need to compute $M$, but rather $M^TM$ so that we do not need to worry about precision. In this case, 
\begin{align*}
M^TM &= b_{\varepsilon_1}\cdots b_{\varepsilon_r}\begin{pmatrix}
	A^TD^2A & 0 & \dots & 0 & 0\\
	0 & \frac{b}{b_{\varepsilon 1}} & \dots & 0 & 0\\
	0 & 0  & \frac{b}{b_{\varepsilon 2}} & \dots & 0\\
	\vdots & \vdots &0 &  \ddots & \vdots\\ 
	0 & 0 & \dots & \dots & \frac{b}{b_{\varepsilon^*}} \\
	\end{pmatrix}\\
	& = \begin{pmatrix}
	(b_{\varepsilon_1}\cdots b_{\varepsilon_r})A^TD^2A & 0 & \dots & 0 & 0\\
	0 & b b_{\varepsilon_2}\cdots b_{\varepsilon_r} & \dots & 0 & 0\\
	0 & 0  & bb_{\varepsilon_1}b_{\varepsilon_3}\cdots b_{\varepsilon_r} & \dots & 0\\
	\vdots & \vdots &0 &  \ddots & \vdots\\ 
	0 & 0 & \dots & \dots & bb_{\varepsilon_1}\cdots b_{\varepsilon_{r-1}} \\
	\end{pmatrix}.
\end{align*}	
	
Since $\gamma\in \Gamma_\tau\cap \mathcal E_\tau$, there exists $x\in \mathbb{R}^{r + \nu}$ such that $\gamma=\Gamma_\tau x+w$ and ${\gamma^tM^tM\gamma\leq (1 + r)(bb_{\varepsilon_1}\cdots b_{\varepsilon_r})}$. We thus have
\[(\Gamma_\tau x+w)^tM^tM(\Gamma_\tau x+w) \leq (1 + r)(bb_{\varepsilon_1}\cdots b_{\varepsilon_r}).\]
As $\Gamma_{\tau}$ is clearly invertible, with matrix inverse
\[\Gamma_{\tau}^{-1} = \begin{pmatrix}
	1 & 0 & \dots &  \dots & 0 & 0\\ 
	0 & 1	& \dots & \dots & 0 & 0\\
	\vdots & \vdots & \ddots & \dots & \vdots & \vdots \\ 
	0 & 0 & \dots &  \dots & 1 & 0\\ 
	-\frac{\alpha_{\gamma 1}}{\alpha_{\varepsilon {r}}} & \dots &-\frac{\alpha_{\gamma {\nu}}}{\alpha_{\varepsilon {r}}} & -\frac{\alpha_{\varepsilon 1}}{\alpha_{\varepsilon {r}}} & \dots & \frac{1}{\alpha_{\varepsilon {r}}}
\end{pmatrix},\]
we can find a vector $c$ such that $\Gamma_{\tau}c = -w$. Indeed, this vector is $c = \Gamma_{\tau}^{-1}(-w)$, where
\[c = \Gamma_{\tau}^{-1}w = \begin{pmatrix}
	1 & 0 & \dots &  \dots & 0 & 0\\ 
	0 & 1	& \dots & \dots & 0 & 0\\
	\vdots & \vdots & \ddots & \dots & \vdots & \vdots \\ 
	0 & 0 & \dots &  \dots & 1 & 0\\ 
	-\frac{\alpha_{\gamma 1}}{\alpha_{\varepsilon {r}}} & \dots &-\frac{\alpha_{\gamma {\nu}}}{\alpha_{\varepsilon {r}}} & -\frac{\alpha_{\varepsilon 1}}{\alpha_{\varepsilon r}} & \dots & \frac{1}{\alpha_{\varepsilon {r}}}
\end{pmatrix}
	\begin{pmatrix}
	0 \\ 0 \\ \vdots \\ 0 \\ -\alpha_0
	\end{pmatrix}
	= \begin{pmatrix}
	0 \\ 0 \\ \vdots \\ 0 \\ -\frac{\alpha_0}{\alpha_{\varepsilon r}}
\end{pmatrix}.\]
Now, 
\begin{align*}
(1 + r)(bb_{\varepsilon_1}\cdots b_{\varepsilon_r})
	& \geq (\Gamma_\tau x+w)^tM^tM(\Gamma_\tau x+w) \\
	& =  (\Gamma_\tau x- (-w))^tM^tM(\Gamma_\tau x-(-w)) \\
	& = (\Gamma_\tau x-\Gamma_{\tau}c)^tM^tM(\Gamma_\tau x-\Gamma_{\tau}c)\\
	& = (\Gamma_\tau (x-c))^tM^tM(\Gamma_\tau (x-c))\\
	& = (x-c)^t(M\Gamma_{\tau})^tM\Gamma_\tau(x-c)\\
	& = (x-c)^tB^tB(x-c)
\end{align*}
where $B = M\Gamma_\tau$. That is, we are left to solve
\[(x-c)B^tB(x-c) \leq (1 + r)(bb_{\varepsilon_1}\cdots b_{\varepsilon_r}).\]
%Here we observe that the vector $c$ is likely not an integral vector; that is, $-\frac{\alpha_0}{\alpha_{\varepsilon r}} \notin \mathbb{Z}$. To amend this, we modify the bound as follows
%\begin{align*}
%\alpha_{\varepsilon r}^2(x-c)^tB^tB(x-c) & \leq \alpha_{\varepsilon r}^2(1 + r)(bb_{\varepsilon_1}\cdots b_{\varepsilon_r})\\
%(\alpha_{\varepsilon r}x-\alpha_{\varepsilon r}c)^tB^tB(\alpha_{\varepsilon r}x-\alpha_{\varepsilon r}c) & \leq \alpha_{\varepsilon r}^2(1 + r)(bb_{\varepsilon_1}\cdots b_{\varepsilon_r})\\
%(z-d)^tB^tB(z-d) & \leq \alpha_{\varepsilon r}^2(1 + r)(bb_{\varepsilon_1}\cdots b_{\varepsilon_r})
%\end{align*}
%where $d = (0 \dots 0 -\alpha_0)$ and $z = \alpha_{\varepsilon r}x$. 
%Is this necessary? Our thing should work over the rationals regardless


%---------------------------------------------------------------------------------------------------------------------------------------------%
%---------------------------------------------------------------------------------------------------------------------------------------------%

%
%
%
%
%
%
% Since
%\[M\gamma=M(\Gamma_\tau x+w) = M\Gamma_{\tau} x + Mw.\]
%Here, it is clear that $M$ is invertible, with inverse
%
%
%
%, it is also invertible. Hence we can find a vector $c$ such that $Bc = v$. 
%
%\begin{align*}
%B = &M\Gamma_{\tau} =\sqrt{b_{\varepsilon_1}\cdots b_{\varepsilon_r}}\begin{pmatrix}
%	DA & 0 & \dots & 0 & 0\\
%	0 & \sqrt{\frac{b}{b_{\varepsilon 1}}} & \dots & 0 & 0\\
%	0 & 0  & \sqrt{\frac{b}{b_{\varepsilon 2}}} & \dots & 0\\
%	\vdots & \vdots &0 &  \ddots & \vdots\\ 
%	0 & 0 & \dots & \dots & \sqrt{\frac{b}{b_{\varepsilon^*}}} \\
%	\end{pmatrix}
%	\begin{pmatrix}
%	1 & 0 & \dots &  \dots & \dots & 0 & 0\\ 
%	0 & 1	& \dots & \dots & \dots & 0 & 0\\
%	\vdots & \vdots & \ddots & \dots & \dots & \vdots & \vdots \\ 
%	0 & 0 & \dots &  \dots & \dots & 1 & 0\\ 
%	\alpha_0 & \alpha_{\gamma 1} & \dots &\alpha_{\gamma {\nu}} & \alpha_{\varepsilon 1} & \dots & \alpha_{\varepsilon^*}
%	\end{pmatrix}\\
%& = \sqrt{b_{\varepsilon_1}\cdots b_{\varepsilon_r}}\begin{pmatrix}
%	DA & \dots & \dots & \dots& 0 & 0\\
%	 0  & \sqrt{\frac{b}{b_{\varepsilon 1}}}& \dots & \dots & \vdots & \vdots \\
%	\vdots & \vdots &\vdots & \ddots & \vdots & \vdots\\ 
%	0 & 0  & \dots & \dots & \sqrt{\frac{b}{b_{\varepsilon r-1}}} &0\\
%	\alpha_0 \sqrt{\frac{b}{b_{\varepsilon^*}}}  & \dots & \dots & \alpha_{\varepsilon 1}\sqrt{\frac{b}{b_{\varepsilon^*}}} & \dots & \alpha_{\varepsilon^*}\sqrt{\frac{b}{b_{\varepsilon^*}}}
%	\end{pmatrix}\\
%	& =\begin{pmatrix}
%	 \sqrt{b_{\varepsilon_1}\cdots b_{\varepsilon_r}}DA & \dots & \dots & \dots& 0 & 0\\
%	 0  &  \sqrt{bb_{\varepsilon_2}\cdots b_{\varepsilon_r}}& \dots & \dots & \vdots & \vdots \\
%	\vdots & \vdots &\vdots & \ddots & \vdots & \vdots\\ 
%	0 & 0  & \dots & \dots &  \sqrt{bb_{\varepsilon_1}\cdots b_{\varepsilon_{r-2}}b_{\varepsilon_{r}}} &0\\
%	\alpha_0 \sqrt{bb_{\varepsilon_1}\cdots b_{\varepsilon_{r-1}}}  & \dots & \dots & \alpha_{\varepsilon 1}\sqrt{bb_{\varepsilon_1}\cdots b_{\varepsilon_{r-1}}} & \dots & \alpha_{\varepsilon^*}\sqrt{bb_{\varepsilon_1}\cdots b_{\varepsilon_{r-1}}}
%	\end{pmatrix}.
%\end{align*}
%
%It follows that
%\begin{align*}
%(1 + r)(bb_{\varepsilon_1}\cdots b_{\varepsilon_r})	
%	& \geq \gamma^tM^tM\gamma \\
%	& = (M\gamma)^tM\gamma \\
%	& = (Bx + v)^t(Bx + v).
%%	& = ((Bx)^t+v^t)(Bx+v)\\
%%	& = (x^tB^t + v^t)(Bx+v)\\
%%	& = x^tB^tBx + (Bx)^tv+v^tBx+v^tv.
%\end{align*}
%Now, since $B$ is positive-definite, it is also invertible. Hence we can find a vector $c$ such that $Bc = v$. 
%
%Now, 
%\[(Bx)^tv = (M\gamma - v)^tv = (M\gamma)^tv -v^tv\]
%and
%\[v^tBx = v^t(M\gamma - v) = v^t(M\gamma) -v^tv\]
%and thus
%\begin{align*}
%(1 + r)(bb_{\varepsilon_1}\cdots b_{\varepsilon_r})
%	& \geq x^tB^tBx + (Bx)^tv+v^tBx+v^tv\\
%	& = x^tB^tBx + (M\gamma)^tv -v^tv + v^t(M\gamma) -v^tv +v^tv\\
%	& = x^tB^tBx + (M\gamma)^tv + v^t(M\gamma) -v^tv \\
%	& = x^tB^tBx + 2(v^t(M\gamma)) -v^tv.
%\end{align*}
%Next, we observe that
%\begin{align*}
%|v^t(M\gamma)|
%	& = |(Mw)^t(M\gamma)|\\
%	& = |wM^TM\gamma|\\
%	& = |\alpha_0 bb_{\varepsilon_1}\cdots b_{\varepsilon_{r-1}} \gamma_n|\\
%	& \leq |\alpha_0| bb_{\varepsilon_1}\cdots b_{\varepsilon_{r-1}} \sqrt{b_{\varepsilon}^*},
%\end{align*}
%%
%%
%%\[|v^t(M\gamma)| \leq \frac{b}{b_{\epsilon^*}^{1/2}}|\alpha_0|\]
%and thus
%\[x^tB^tBx\leq (1 + r)(bb_{\varepsilon_1}\cdots b_{\varepsilon_r})+v^tv - 2v^t(M\gamma)\]
%so that 
%\begin{align*}
%|x^tB^tBx| 
%	& \leq |(1 + r)(bb_{\varepsilon_1}\cdots b_{\varepsilon_r})+v^tv - 2v^t(M\gamma)|\\
% 	& \leq |(1 + r)(bb_{\varepsilon_1}\cdots b_{\varepsilon_r})+ v^tv| + 2|v^t(M\gamma)|\\
%	& \leq \left|(1 + r)(bb_{\varepsilon_1}\cdots b_{\varepsilon_r})+ v^tv\right| + 2 |\alpha_0| bb_{\varepsilon_1}\cdots b_{\varepsilon_{r-1}} \sqrt{b_{\varepsilon}^*}.
%\end{align*}
%
%Now, following the steps of the Fincke-Pohst algorithm, we first generate the matrix $R$ via Cholesky decomposition applied to $B^{\text{T}}B$. In particular, this means that we don't need to know $B$ explicitly, but rather $B^TB$, where
%\[B^TB = (M\Gamma_{\tau})^T(M\Gamma_{\tau}) = \Gamma_{\tau}^T(M^TM)\Gamma_{\tau}\]
%where both $M^TM$ and $\Gamma_{\tau}$ are integral matrices. 
%
%
%\subsection{Non-Archimedean sieve}
%%Let $v\in T$. To simplify our exposition, we slightly abuse notation and we write $v=\ord_p:\bar{\QQ}_p\to\QQ$. We take $l,h\in\RR^{S^*}$ with $0\leq l\leq h$ and $l_v/\log p\geq \max\bigl(\tfrac{1}{p-1},v(\mu_0)\bigl)-v(\lambda_0)$, and then we consider the 
%%translated lattice $\Gamma_v\subset \ZZ^\unit$ defined below.  We say that $(x,y)\in\Sigma$ with $m\in\ZZ^{\unit}$ is determined by some $\gamma\in \Gamma_v$ if the entries of $\gamma$ are a (fixed) permutation of the entries of $m$. Let $\mathcal E_v\subseteq \RR^\unit$ be the ellipsoid constructed in \eqref{def:ellnonarch}.
%%\begin{lemma}\label{lem:nonarchsieve}
%%Any $(x,y)\in\Sigma_v(l,h)$ is determined by some $\gamma\in \Gamma_v\cap\mathcal E_v.$
%%\end{lemma}
%%
%%In the remaining of this section we prove this lemma.
%
%---------------------------------------------------------------------------------------------------------------------------------------------%
%---------------------------------------------------------------------------------------------------------------------------------------------%

\subsection{Archimedean Real Case Summary}
If $(n_1, \dots, n_{\nu}, a_1, \dots, a_r) \in \mathbb{R}^{r+\nu}$ is a solution which lies in $\Sigma_{\tau}(l,h)$, then, by definition, it corresponds to a solution $(x,y)$ satisfying
\[\Sigma_{\tau}(l,h) = \{(x,y) \in \Sigma \ | \ (h_v(z))\leq h \text{ and }  (h_v(z))\nleq 0 \text{ and } h_{\tau}(z)>l_{\tau}\}.\]
Here, $l_{\tau}$ is defined as some constant such that 
\[l_{\tau} > c_{\tau}.\]
By the computations above (see page 99), it follows that
\[\left|\alpha_0+\sum_{i = 1}^r a_i \alpha_{\varepsilon i} + \sum_{i = 1}^{\nu} n_i \alpha_{\gamma i}\right|
 \leq \frac{1}{2}\left(\frac{1}{[K:\mathbb{Q}]}\sum_{l = 1}^{\nu}w_l h_l + \frac{1}{[L:\mathbb{Q}]}\sum_{\sigma :L \to \mathbb{C}} w_{\sigma}h_{\sigma} \right) + \frac{1}{2} + c\kappa_{\tau}e^{-l_{\tau}}.\]

Now, consider the vector 
\[\gamma = (n_1, \dots, n_{\nu}, a_1, \dots, a_{r-1}, \alpha_0+\sum_{i = 1}^r a_i \alpha_{\varepsilon i} + \sum_{i = 1}^{\nu} n_i \alpha_{\gamma i})\]
and the lattice defined by $\Gamma_{\tau}x + w$
\[\begin{pmatrix}
	1 & 0 & \dots &  \dots & 0 & 0\\ 
	0 & 1	& \dots & \dots & 0 & 0\\
	\vdots & \vdots & \ddots & \dots & \vdots & \vdots \\ 
	0 & 0 & \dots &  \dots & 1 & 0\\ 
	\alpha_{\gamma 1} & \dots &\alpha_{\gamma {\nu}} & \alpha_{\varepsilon 1} & \dots & \alpha_{\varepsilon {r}}
\end{pmatrix}
\begin{pmatrix}
	x_1 \\ x_2 \\ \vdots \\ x_{\nu+r - 1} \\ x_{\nu+r}\\ 
\end{pmatrix}+
\begin{pmatrix}
	0 \\ 0 \\ \vdots \\ 0 \\ \alpha_0\\ 
\end{pmatrix}\]
for some vector $(x_1, \dots, x_{\nu+r}) \in \mathbb{Z}^{\nu+r}$.
If  $(x_1, \dots, x_{\nu+r}) = (n_1, \dots, n_{\nu}, a_1, \dots, a_{r})$, then a quick computation shows that 
\[\Gamma_{\tau}x + w = 
\begin{pmatrix}
	x_1 \\ x_2 \\ \vdots \\ x_{\nu+r - 1} \\ \alpha_0+\sum_{i = 1}^r x_i \alpha_{\varepsilon i} + \sum_{i = 1}^{\nu} x_i \alpha_{\gamma i}\\ 
\end{pmatrix} = 
\begin{pmatrix}
	n_1 \\ n_2 \\ \vdots \\ a_{r - 1} \\ \alpha_0+\sum_{i = 1}^r a_i \alpha_{\varepsilon i} + \sum_{i = 1}^{\nu} n_i \alpha_{\gamma i}\\ 
\end{pmatrix} = \gamma^T.\]
Hence, $\gamma$ is in the lattice $\Gamma$.

Now, consider the ellipsoid $\mathcal{E}_\tau$. We claim that $\gamma \in \mathcal{E}_{\tau}$. Indeed, this means that 
\[\gamma^tM^tM\gamma  \leq (1 + r)(bb_{\varepsilon_1}\cdots b_{\varepsilon_r}).\]
In particular
\begin{align*}
& \gamma^TM^TM\gamma \\ 
	& = \begin{pmatrix}
	n_1 \\ n_2 \\ \vdots \\ a_{r - 1} \\ \alpha_0+\sum_{i = 1}^r a_i \alpha_{\varepsilon i} + \sum_{i = 1}^{\nu} n_i 	\alpha_{\gamma i}\\ 
	\end{pmatrix}
	\begin{pmatrix}
	(b_{\varepsilon_1}\cdots b_{\varepsilon_r})A^TD^2A & 0 & \dots & 0\\
	0 & b b_{\varepsilon_2}\cdots b_{\varepsilon_r} & \dots & 0 \\
	\vdots & \vdots  &  \ddots & \vdots\\ 
	0 & 0 & \dots & bb_{\varepsilon_1}\cdots b_{\varepsilon_{r-1}} \\
	\end{pmatrix}
	\gamma\\
	& = \begin{pmatrix}
		\begin{pmatrix} 
		n_1 \\ n_2 \\ \vdots \\ n_{\nu}
		\end{pmatrix}(b_{\varepsilon_1}\cdots b_{\varepsilon_r})A^TD^2A\\
		a_1 b b_{\varepsilon_2}\cdots b_{\varepsilon_r} \\ 
		\vdots \\
		a_{r - 1} b b_{\varepsilon_1}\cdots b_{\varepsilon_{r-2}}b_{\varepsilon_r} \\ 
		\left(\alpha_0+\sum_{i = 1}^r a_i \alpha_{\varepsilon i} + \sum_{i = 1}^{\nu} n_i \alpha_{\gamma i}\right)b b_{\varepsilon_1}\cdots b_{\varepsilon_{r-1}}\\ 
	\end{pmatrix}
	\begin{pmatrix}
	 n_1 & \dots & a_{r-1} & \alpha_0+\sum_{i = 1}^r a_i \alpha_{\varepsilon i} + \sum_{i = 1}^{\nu} n_i \alpha_{\gamma i}
	 \end{pmatrix}\\
	 & = \begin{pmatrix} 
		n_1 \\ n_2 \\ \vdots \\ n_{\nu}
		\end{pmatrix}(b_{\varepsilon_1}\cdots b_{\varepsilon_r})A^TD^2A
		\begin{pmatrix}
	 n_1 & \dots & n_{\nu} \end{pmatrix} + a_1^2 b b_{\varepsilon_2}\cdots b_{\varepsilon_r} + \cdots + a_{r - 1}^2 b b_{\varepsilon_1}\cdots b_{\varepsilon_{r-2}}b_{\varepsilon_r} + \\
	 & \quad \quad \quad + \left(\alpha_0+\sum_{i = 1}^r a_i \alpha_{\varepsilon i} + \sum_{i = 1}^{\nu} n_i \alpha_{\gamma i}\right)^2b b_{\varepsilon_1}\cdots b_{\varepsilon_{r-1}}.
\end{align*}	
Now, by definition, we have 
\begin{align*}
\gamma^TM^TM\gamma 
	& = \begin{pmatrix} 
		n_1 \\ n_2 \\ \vdots \\ n_{\nu}
		\end{pmatrix}(b_{\varepsilon_1}\cdots b_{\varepsilon_r})A^TD^2A
		\begin{pmatrix}
	 n_1 & \dots & n_{\nu} \end{pmatrix} + a_1^2 b b_{\varepsilon_2}\cdots b_{\varepsilon_r} + \cdots + a_{r - 1}^2 b b_{\varepsilon_1}\cdots b_{\varepsilon_{r-2}}b_{\varepsilon_r} + \\
	 & \quad \quad \quad + \left(\alpha_0+\sum_{i = 1}^r a_i \alpha_{\varepsilon i} + \sum_{i = 1}^{\nu} n_i \alpha_{\gamma i}\right)^2b b_{\varepsilon_1}\cdots b_{\varepsilon_{r-1}}\\
& \leq (b_{\varepsilon_1}\cdots b_{\varepsilon_r}) q_f(n_1 \dots n_{\nu}) +					 			b_{\varepsilon_1} b b_{\varepsilon_2}\cdots b_{\varepsilon_r} + \cdots + b_{\varepsilon_{r - 1}} b 		b_{\varepsilon_1}\cdots b_{\varepsilon_{r-2}}b_{\varepsilon_r} + b_{\varepsilon_r}b 					b_{\varepsilon_1}\cdots b_{\varepsilon_{r-2}}b_{\varepsilon_{r-1}}\\
& \leq (b_{\varepsilon_1}\cdots b_{\varepsilon_r})b + r(bb_{\varepsilon_1}\cdots b_{\varepsilon_r})\\
& = (1+r)(bb_{\varepsilon_1}\cdots b_{\varepsilon_r}).
\end{align*}
Thus $\gamma \in \mathcal{E}_{\tau}$. Thus, if we assume that our solution lies in $\Sigma_{\tau}(l,h)$, it follows that $\gamma \in \Gamma \cap \mathcal{E}_{\tau}$. Now, by Rafael, 
\[\Sigma=\Sigma\left(h_{0}\right), \quad \Sigma(h)=\Sigma(l, h) \cup \Sigma(l) \quad \text { and } \quad \Sigma(l, h)=\cup_{v \in S^{*}} \Sigma_{v}(l, h).\]
Thus, we collect all solutions from $\Sigma_{\tau}(l,h)$ and continue to generate all solutions at the other places. 

MAYBE COULD USE MORE DETAIL HERE
%---------------------------------------------------------------------------------------------------------------------------------------------%
%---------------------------------------------------------------------------------------------------------------------------------------------%

\section{Non-Archimedean Case}

\subsection{Non-Archimedean sieve}

Note that in this section we might use $v$ and $l$ interchangeably. Eventually this will be fixed to be consistent...

Let $v \in \{1, \dots, \nu\}$. We take vectors $l,h \in \mathbb{R}^{\nu+r}$ with $0 \leq l \leq h$ and 
\[\frac{l_v}{\log(p)} \geq \max\left( \frac{1}{p-1}, \ord_{p_v}(\delta_1)\right) - \ord_{p_v}(\delta_2)\]
and then consider the translated lattice $\Gamma_v \subseteq \mathbb{Z}^{\nu + r}$ defined below. We say that $(x,y) \in \Sigma$ with ${\mathbf{m} = (n_1, \dots, n_{\nu}, a_1, \dots, a_r) \in \mathbb{R}^{r + \nu}}$ is determined by some $\gamma \in \Gamma_v$ if the entries of $\gamma$ are a (fixed) permutation of the entries of $\mathbf{m}$. Let $\mathcal{E}_v$ be the ellipsoid constructed in \eqref{def:ellp}. 

\begin{lemma}
And $(x,y) \in \Sigma_v(l,h)$ is determined by some $\gamma \in \Gamma_v \cap \mathcal{E}_v$. 
\end{lemma}

In the remainder of this section, we prove this lemma. 

\subsubsection{Computing $u_l - r_l = \sum_{i = 1}^{\nu} n_ia_{li}$}

Recall that $z \in \mathbb{C}_p$ having $\ord_p(z) = 0$ is called a $p$-adic unit. 

Let $l \in \{1, \dots, \nu \}$ and consider the prime $p = p_l$. For every $i \in \{1, \dots, r\}$, part (ii) of the Corollary of Lemma 2 of Tzanakis-de Weger tells us that $\frac{\varepsilon_1^{(i_0)}}{\varepsilon_1^{(j)}}$ and $\frac{\varepsilon_i^{(k)}}{\varepsilon_i^{(j)}}$ (for $i = 1, \dots, r$) are $p_l$-adic units. 

From now on we make the following choice for the index $i_0$. Let $g_l(t)$ be the irreducible factor of $g(t)$ in $\mathbb{Q}_{p_l}[t]$ corresponding to the prime ideal $\mathfrak{p}_l$. Since $\mathfrak{p}_l$ has ramification index and residue degree equal to $1$, $\deg(g_l[t]) = 1$. We choose $i_0 \in \{1,2,3\}$ so that $\theta^{(i_0)}$ is the root of $g_l(t)$. The indices of $j,k$ are fixed, but arbitrary. 

\begin{lemma} \label{Lem:units} \
\begin{enumerate}
\item[(i)] Let $i \in \{1, \dots, \nu\}$. Then $\frac{\gamma_i^{(k)}}{\gamma_i^{(j)}}$ are $p_l$-adic units. 
\item[(ii)] Let $i \in \{1, \dots, \nu\}$. Then $\ord_{p_l}\left(\frac{\gamma_i^{(i_0)}}{\gamma_i^{(j)}}\right) = a_{li}$, where $\mathbf{a_i} = (a_{1i}, \dots, a_{\nu i})$. 
\end{enumerate}
\end{lemma}

\begin{proof}
Consider the factorization of $g(t)$ in $\mathbb{Q}_{p_l}[t]: g(t) = g_1(t) \cdots g_m(t)$. Note $\theta^{(j)}$ is a root of some $g_h(t) \neq g_l(t)$. Let $\mathfrak{p}_h$ be the corresponding prime ideal above $p_l$ and $e_h$ be its ramification index. Then $\mathfrak{p} \neq \mathfrak{p}_l$ and since 
\[(\gamma_i)\mathcal{O}_K = \mathfrak{p}_1^{a_{1i}} \cdots \mathfrak{p}_{\nu}^{a_{\nu i}},\]
we have 
\[\ord_{p_l}(\gamma_i^{(j)}) = \frac{1}{e_h}\ord_{\mathfrak{p}_h}(\gamma_i) = 0.\]
An analogous argument gives $\ord_{p_l}(\gamma_i^{(k)}) = 0$. On the other hand, 
\[\ord_{p_l}(\gamma_i^{(i_0)}) = \frac{1}{e_l}\ord_{\mathfrak{p}_l}(\gamma_i) = \ord_{\mathfrak{p}_l}(\mathfrak{p}_1^{a_{1i}} \cdots \mathfrak{p}_{\nu}^{a_{\nu i}}) = a_{li}.\]

\end{proof}

We consider the form
\[\Lambda_{p_l} = \log_{p_l}(\delta_1) + \sum_{i=1}^r a_i\log_{p_l}\left( \frac{\varepsilon_i^{(k)}}{\varepsilon_i^{(j)}}\right) + \sum_{i=1}^{\nu} n_i \log_{p_l} \left( \frac{\gamma_i^{(k)}}{\gamma_i^{(j)}}\right).\]
To simplify our exposition, we introduce the following notation. 
\[b_1 = 1, \quad b_{1+i} = n_i \ \text{ for } i \in \{1, \dots, \nu\},\]
and
\[ b_{1 + \nu+i} = a_i \ \text{ for } i \in \{1, \dots, r\}.\]
Put
\[\alpha_1 = \log_{p_l} \delta_1, \quad \alpha_{1+i} = \log_{p_l}\left( \frac{\gamma_i^{(k)}}{\gamma_i^{(l)}}\right)  \ \text{ for } i \in \{1, \dots, \nu\},\]
and
\[\alpha_{1+ \nu+i} = \log_{p_l}\left( \frac{\varepsilon_i^{(k)}}{\varepsilon_i^{(l)}}\right)
\ \text{ for } i \in \{1, \dots, r\}.\]
Now, 
\[\Lambda_{p_l} = \log_{p_l}(\delta_1) + \sum_{i=1}^r a_i\log_{p_l}\left( \frac{\varepsilon_i^{(k)}}{\varepsilon_i^{(j)}}\right) + \sum_{i=1}^{\nu} n_i \log_{p_l} \left( \frac{\gamma_i^{(k)}}{\gamma_i^{(j)}}\right) = \sum_{i = 1}^{1 + \nu + r} b_i\alpha_i.\]


The next lemma deals with a special case in which the $n_l$ can be computed directly. 

\begin{lemma}
Let $l \in \{1, \dots, \nu \}$. If $\ord_{p_l}(\delta_1) \neq 0$, then 
\[ \sum_{i = 1}^{\nu} n_ia_{li} = \min\{\ord_{p_l}(\delta_1), 0\} - \ord_{p_l}(\delta_2).\]
\end{lemma}

%From here it follows that
%\[z_l = u_l + t_l = \sum_{j = 1}^{\nu}n_ja_{lj} + r_l + t_l.\]

\begin{proof}
Apply the Corollary of Lemma $2$ of Tzanakis-de Weger and Lemma~\ref{Lem:units} to both expressions of $\lambda$ in \eqref{Eq:Sunit}. On the one hand, we obtain $\ord_{p_l}(\lambda) = \min\{\ord_{p_l}(\delta_1), 0\}$, and on the other hand, we obtain 
\[\begin{split}
\ord_{p_l}(\lambda)
& = \ord_{p_l}(\delta_2) + \sum_{i = 1}^{\nu} \ord_{p_l}\left( \frac{\gamma_i^{(i_0)}}{\gamma_i^{(j)}}\right)^{n_i}\\
& = \ord_{p_l}(\delta_2) + \sum_{i = 1}^{\nu} n_ia_{li}.
\end{split}\]
\end{proof}

That is, 
\[\sum_{i = 1}^{\nu} n_ia_{li} = \begin{cases}
- \ord_{p_l}(\delta_2) & \text{ if } \ord_{p_l}(\delta_1) > 0 \\
 \ord_{p_l}(\delta_1)- \ord_{p_l}(\delta_2) = \ord_{p_l}(\delta_1/\delta_2) & \text{ if } \ord_{p_l}(\delta_1) < 0 \\
\end{cases}\]


From here, we will need to assume that $\ord_{p_l}(\delta_1) = 0$. We first recall the following result of the $p_l$-adic logarithm:

\begin{lemma}\label{Lem:padic}
Let $z_1, \dots, z_m \in \overline{\mathbb{Q}}_p$ be $p$-adic units and let $b_1, \dots, b_m \in \mathbb{Z}$. If
\[\ord_p(z_1^{b_1}\cdots z_m^{b_m} - 1) > \frac{1}{p-1}\]
then
\[\ord_p(b_1\log_p z_1 + \cdots + b_m \log_p z_m) = \ord_p(z_1^{b_1}\cdots z_m^{b_m} - 1) \]
\end{lemma}

NOT CLEAR TO ME IF THE BELOW IS AN EFFICIENT COMPUTATION TO MAKE, OR THAT IT HAPPENS OFTEN ENOUGH TO TEST.

For $l \in \{1, \dots \nu\}$, we identify conditions in which $n_l$ can be bounded by a small explicit constant.

Let $L$ be a finite extension of $\mathbb{Q}_{p_l}$ containing $\delta_1$, $\frac{\gamma_i^{(k)}}{\gamma_i^{(l)}}$ (for $i = 1, \dots, \nu$), and $ \frac{\varepsilon_i^{(k)}}{\varepsilon_i^{(l)}}$ (for $i = 1, \dots, r$). Since finite $p$-adic fields are complete, $\alpha_i \in L$ for $i = 1, \dots, 1+ \nu +r$ as well. Choose $\phi \in \overline{\mathbb{Q}_{p_l}}$ such that $L = \mathbb{Q}_{p_l}(\phi)$ and $\ord_{p_l}(\phi) > 0 $. Let $G(t)$ be the minimal polynomial of $\phi$ over $\mathbb{Q}_{p_l}$ and let $S$ be its degree. For $i = 1, \dots, 1 + \nu + r$ write
\[\alpha_i = \sum_{h = 1}^S \alpha_{ih}\phi^{h - 1}, \quad \alpha_{ih} \in \mathbb{Q}_{p_l}.\]
Then
\begin{equation} \label{Eq:lambdalh}
\Lambda_l = \sum_{h = 1}^S \Lambda_{lh}\phi^{h-1},
\end{equation}
with
\[\Lambda_{lh} = \sum_{i = 1}^{1 + \nu + r } b_i \alpha_{ih}\]
for $h = 1, \dots, S$. 

\begin{lemma}\label{Lem:discG}
For every $h \in \{1, \dots, S\}$, we have
\[\ord_{p_l}(\Lambda_{lh}) > \ord_{p_l}(\Lambda_l) - \frac{1}{2}\ord_{p_l}(\text{Disc}(G(t))).\]
\end{lemma}

\begin{proof}
Taking the images of \eqref{Eq:lambdalh} under conjugation $\phi \mapsto \phi^{(h)}$ ($h = 1, \dots, S$) gives
\[\begin{bmatrix}
\Lambda_l^{(1)} \\
\vdots \\
\Lambda_l^{(S)}	\\
\end{bmatrix}
=
\begin{bmatrix}
1 		& \phi^{(1)} 	& \cdots 	& \phi^{(1)S-1}\\
\vdots 	& \vdots 		& 		& \vdots \\
1 		& \phi^{(S)} 	& \cdots  	& \phi^{(S)S-1}\\
\end{bmatrix}
\begin{bmatrix}
\Lambda_{l1}\\
\vdots \\
\Lambda_{lS}\\
\end{bmatrix}\]
The $s \times s$ matrix $(\phi^{(h)i-1})$ above is invertible, with inverse
\[\frac{1}{\displaystyle \prod_{1\leq j<k\leq S} (\phi^{(k)} - \phi^{(j)})}
\begin{bmatrix}
\gamma_{11} 	& \cdots 	& \gamma_{1S}\\
\vdots 		& 		& \vdots\\
\gamma_{s1} 	& \cdots 	& \gamma_{SS}\\
\end{bmatrix},\]
where $\gamma_{jk}$ is a polynomial in the entries of $(\phi^{(h)i-1})$ having integer coefficients. Since $\ord_{p_l}(\phi) > 0$ and since $\ord_{p_l}(\phi^{(h)}) = \ord_{p_l}(\phi)$ for all $h = 1, \dots, S$, it follows that $\ord_{p_l}(\gamma_{jk}) > 0 $ for every $\gamma_{jk}$. Therefore, since 
\[\Lambda_{lh} = \frac{1}{\displaystyle \prod_{1\leq j<k\leq S}(\phi^{(k)} - \phi^{(j)})}\sum_{i = 1}^S \gamma_{hi}\Lambda_l^{(i)},\]
we have 
\[\begin{split}
\ord_{p_l}(\Lambda_{lh}) 
	& = \min_{1 \leq i \leq S} \left\{\ord_{p_l}(\gamma_{hi}) + \ord_{p_l}(\Lambda_l^{(i)})\right\} -\frac{1}{2}\ord_{p_l}(\text{Disc}(G(t)))\\
	& \geq \min_{1 \leq i \leq S} \ord_{p_l}(\Lambda_l^{(i)}) +  \min_{1 \leq i \leq S} \ord_{p_l}(\gamma_{hi}) - \frac{1}{2}\ord_{p_l}(\text{Disc}(G(t)))\\
	& = \ord_{p_l}\Lambda_l + \min_{1 \leq i \leq S} \ord_{p_l}(\gamma_{hi}) - \frac{1}{2}\ord_{p_l}(\text{Disc}(G(t)))
\end{split}\]
for every $h \in \{1, \dots, S\}$. 
%\min_{1 \leq i \leq s} \left\{\ord_{p_l}(\gamma_{hi}) + \ord_{p_l}(\Lambda_l^{(i)}) -\frac{1}{2}\ord_{p_l}(\text{Disc}(G(t)))\right\}\]
\end{proof}

\textbf{Remark.} The above can made more precise using the $\gamma_{jk}$, possibly giving us a tighter subsequent bound on the $n_l$. 

\begin{lemma} \label{Lem:Lambda}
If $\ord_{p_l}(\delta_1) = 0$ and 
\[\sum_{i = 1}^{\nu} n_{i}a_{li} > \frac{1}{p_l-1} - \ord_{p_l}(\delta_2),\]
then
\[\ord_{p_l}(\Lambda_l) = \sum_{i = 1}^{\nu} n_{i}a_{li} + \ord_{p_l}(\delta_2).\]
\end{lemma}

\begin{proof}
Immediate from Lemma~\ref{Lem:padic}.
\end{proof}
Said another way, this means
\[ \sum_{i = 1}^{\nu} n_{i}a_{li} = \ord_{p_l}(\Lambda_l) - \ord_{p_l}(\delta_2) =  \ord_{p_l}(\Lambda_l/\delta_2).\]

\begin{lemma} \label{Lem:specialcase} \
Suppose $\ord_{p_l}(\delta_1) = 0$. 
\begin{enumerate}
\item[(i)] If $\ord_{p_l}(\alpha_1) < \displaystyle \min_{2 \leq i \leq 1 + \nu + r} \ord_{p_l}(\alpha_i)$, then
\[\sum_{i = 1}^{\nu} n_i a_{li} \leq \max \left\{ \bigg\lfloor{\frac{1}{p-1} - \ord_{p_l}(\delta_2)}\bigg\rfloor,  \bigg \lceil\displaystyle \min_{2 \leq i \leq 1 + \nu + r} \ord_{p_l}(\alpha_{i}) - \ord_{p_l}(\delta_2) \bigg \rceil - 1 \right\}\]

\item[(ii)] For all $h \in \{1, \dots, S\}$, if $\ord_{p_l}(\alpha_{1h}) < \displaystyle \min_{2 \leq i \leq 1+ \nu + r} \ord_{p_l}(\alpha_{ih})$, then
\[\sum_{i = 1}^{\nu} n_i a_{li} \leq \max \left\{ \bigg\lfloor{\frac{1}{p-1} - \ord_{p_l}(\delta_2)}\bigg\rfloor, \bigg \lceil \displaystyle \min_{2 \leq i \leq 1+\nu+r} \ord_{p_l}(\alpha_{ih})- \ord_{p_l}(\delta_2) + \nu_l \bigg \rceil - 1\right\},\]
where 
\[\nu_l = \frac{1}{2}\ord_{p_l}(\text{Disc}(G(t)))\]
\end{enumerate}
\end{lemma}

AGAIN, IT'S NOT CLEAR HOW EFFICIENT THIS COMPUTATION IS... WE SHALL SEE. PART (1) IS ACTUALLY NEEDED IN THE REST OF THE COMPUTATIONS, BUT PART (2) MIGHT BE THE INEFFICIENT PART OF THIS. 

\begin{proof} \
\begin{enumerate}
\item[(i)] We prove the contrapositive. Suppose
\[\sum_{i = 1}^{\nu} n_i a_{li} > \frac{1}{p-1} - \ord_{p_l}(\delta_2), \]
and
\[\sum_{i = 1}^{\nu} n_i a_{li}  \geq \displaystyle \min_{2 \leq i \leq 1 +\nu + r} \ord_{p_l}(\alpha_{i}) - \ord_{p_l}(\delta_2).\]
Observe that
\[\begin{split}
\ord_{p_l}(\alpha_{1}) 	
	& = \ord_{p_l}\left( \Lambda_{l} - \sum_{i = 2}^{1+\nu +r}b_i\alpha_{i}\right) \\
	& \geq \min\left\{ \ord_{p_l}(\Lambda_{l}), \min_{2 \leq i \leq 1+\nu+r} \ord_{p_l}(b_i\alpha_{i})\right\}.
\end{split}\]
Therefore, it suffices to show that 
\[\ord_{p_l}(\Lambda_{l}) \geq \min_{2 \leq i \leq 1 + \nu + r} \ord_{p_l}(b_i\alpha_{i}).\]
By Lemma~\ref{Lem:padic}, the first inequality implies $\ord_{p_l}(\Lambda_{l}) = \displaystyle \sum_{i = 1}^{\nu} n_ia_{li} + \ord_{p_l}(\delta_2)$, from which the result follows. 

\item[(ii)] We prove the contrapositive. Let $h \in \{1, \dots, S\}$ and suppose
\[\sum_{i = 1}^{\nu} n_i a_{li} > \frac{1}{p-1} - \ord_{p_l}(\delta_2), \]
and
\[\sum_{i = 1}^{\nu} n_i a_{li}  \geq \nu_l + \displaystyle \min_{2 \leq i \leq 1+\nu+r} \ord_{p_l}(\alpha_{ih}) - \ord_{p_l}(\delta_2).\]
Observe that 
\[\begin{split}
\ord_{p_l}(\alpha_{1h}) 	
	& = \ord_{p_l}\left( \Lambda_{lh} - \sum_{i = 2}^{1+\nu+r}b_i\alpha_{ih}\right) \\
	& \geq \min\left\{ \ord_{p_l}(\Lambda_{lh}), \min_{2 \leq i \leq 1+\nu+r} \ord_{p_l}(b_i\alpha_{ih})\right\}
\end{split}\]
Therefore, it suffices to show that 
\[\ord_{p_l}(\Lambda_{lh}) \geq \min_{2 \leq i \leq 1+\nu+r} \ord_{p_l}(b_i\alpha_{ih}).\]
By Lemma~\ref{Lem:padic}, the first inequality implies $\ord_{p_l}(\Lambda_{l}) = \displaystyle \sum_{i = 1}^{\nu}n_ia_{li} + \ord_{p_l}(\delta_2)$. Combining this with Lemma~\ref{Lem:discG} yields
\[\ord_{p_l}(\Lambda_{lh}) \geq \displaystyle \sum_{i = 1}^{\nu} n_ia_{li} + \ord_{p_l}(\delta_2) - \nu_l.\]
The results now follow from our second assumption. 
\end{enumerate}
\end{proof}


We now set some notation and give some preliminaries for the $p_l$-adic reduction procedures. Consider a fixed index $l \in \{1, \dots, v\}$. Following Lemma \ref{Lem:specialcase}, we have
\[\ord_{p_l}(\alpha_1) \geq \min_{2\leq i\leq 1+ \nu+ r} \ord_{p_l}(\alpha_i) \quad \text{ and } \quad \ord_{p_l}(\alpha_{1h}) \geq \min_{2\leq i\leq 1+ \nu+ r}(\alpha_{ih}) \quad h = (1, \dots, s).\]
and 
\[\ord_{p_l}(\delta_1) = 0.\]

Let $I$ be the set of all indices $i' \in \{2, \dots, 1+ \nu + r\}$ for which
\[\ord_{p_l}(\alpha_{i'}) = \min_{2\leq i\leq 1+ \nu+ r} \ord_{p_l}(\alpha_i).\]
We will identify two cases, the \textit{special case} and the \textit{general case}. The special case occurs when there is some index $i' \in I$ such that $\alpha_i/\alpha_{i'} \in \mathbb{Q}_{p_l}$ for $i = 1, \dots, 1+ \nu+ r$. The general case is when there is no such index. 

We now assume that our Thue-Mahler equation has degree $3$ to assure that our linear form in $p$-adic logs has coefficients in $\mathbb{Q}_p$. DETAILS NEEDED HERE [p51 of HAMBROOK]. This means that we are indeed always in the Special Case of TdW/Hambrook. 

Thus, let $\hat{i}$ be an arbitrary index in $I$ for which $\alpha_i/\alpha_{\hat{i}} \in \mathbb{Q}_{p_l}$ for every $i = 1, \dots, 1+ \nu+ r$. We further define
\[\beta_i = - \frac{\alpha_i}{\alpha_{\hat{i}}} \quad i = 1, \dots, 1+ \nu+ r,\]
and 
\[\Lambda'_l = \frac{1}{\alpha_{\hat{i}}}\Lambda_l = \sum_{i = 1}^{1+ \nu+ r} b_i(-\beta_i).\]
Now, we have $\beta_i \in \mathbb{Z}_{p_l}$ for $i = 1, \dots, 1+ \nu+ r$. 

\begin{lemma} \label{Lem:19.1}
Suppose $\ord_{p_l}(\delta_1) = 0$ and 
\[\sum_{i = 1}^v n_{i}a_{li} > \frac{1}{p_l-1} - \ord_{p_l}(\delta_2).\]
Then
\[\ord_{p_l}(\Lambda_l') = \sum_{i = 1}^v n_{i}a_{li} + \ord_{p_l}(\delta_2) - \ord_{p_l}(\alpha_{\hat{i}}).\]
\end{lemma}

\begin{proof}
Immediate from Lemma \ref{Lem:discG} and Lemma \ref{Lem:Lambda}. 
\end{proof}

We now describe the $p_l$-adic reduction procedure. Recall that $l_v$ is a constant such that
\[\frac{l_v}{\log(p)} \geq \max\left( \frac{1}{p-1}, \ord_{p_v}(\delta_1)\right) - \ord_{p_v}(\delta_2).\]
Now, let $l_v'$ (denoted $\mu$ in BeGhKr and $m$ in TdW) be the largest element of $\mathbb{Z}_{\geq 0}$ at most
\[l_v' \leq \frac{l_v}{\log(p)} - \ord_{p_l}(\alpha_{\hat{i}}) + \ord_{p_l}(\delta_2).\]
We will use the notation $l_v'$ and $\mu$ interchangeably. Eventually we should use consistent notation here, but we will just use $\mu$ for now in place of $l_v'$. 

For each $x \in \mathbb{Z}_{p_l}$, let $x^{\{\mu\}}$ denote the unique rational integer in $[0,p_l^{\mu} - 1]$ such that $\ord_{p_l}(x - x^{\mu}) \geq \mu$ (ie. $x \equiv x^{\{\mu\}} \pmod{p_l^{\mu}}$). That is, 
\[x \equiv x^{\{\mu\}} \pmod{p_l^{\mu}} \implies x - x^{\{\mu\}} =\alpha p_l^{\mu}\]
for some $\alpha \in \mathbb{Z}$. Hence $x \equiv x^{\{\mu\}} \pmod{p_l^{j}}$ for $j =1, \dots, \mu$. In other words, we must have
\[x = a_0+ a_1 p + \cdots + a_n p^n + \cdots \quad \text{ and } \quad x^{\{\mu\}} = a_0+ a_1 p + \cdots + a_{\mu - 1}p^{\mu - 1}.\]
Then 
\[x - x^{\{\mu\}} = a_{\mu}p^{\mu} + \cdots + a_n p^n + \cdots \implies x - x^{\{\mu\}} \equiv 0 \pmod{p^{\mu}}\]
so that the highest power dividing $x - x^{\{\mu\}}$ is at least $\mu$. Recall, the order is the first non-zero term appearing in the series expansion of $x - x^{\{\mu\}}$, and thus $a_{\mu}$ may or may not be the first non-zero term, hence the order is at least $\mu$, though can be larger.

Let $\Gamma_{\mu}$ be the $(\nu+r)$-dimensional translated lattice $A_{\mu}x + w$, where $A_{\mu}$ is the diagonal matrix having $\hat{i}^{\text{th}}$ row 
\[\left(\beta_2^{\{\mu\}}, \cdots, \beta_{\hat{i} - 1}^{\{\mu\}}, p_l^{\mu}, \beta_{\hat{i} + 1}^{\{\mu\}}, \cdots, \beta_{1+ \nu+ r}^{\{\mu\}}\right) \in \mathbb{Z}^{\nu+r}.\]
Here, $p_l^{\mu}$ is the $(\hat{i},\hat{i})$ entry of $A_{\mu}$. That is, 
\[A_{\mu} = 
\begin{pmatrix}
1	& 		&		&		&		&		&	\\
	& \ddots	& 		&		& 0		& 		&	\\
	&		& 1		&		&		&		&	\\
	\beta_2^{\{\mu\}}& \cdots & \beta_{\hat{i} - 1}^{\{\mu\}} & p_l^{\mu} & \beta_{\hat{i} + 1}^{\{\mu\}}& \cdots &\beta_{1+ \nu+ r}^{\{\mu\}}\\
	& 		& 		& 		& 1		&		&	\\	
	& 0		& 		& 		&		& \ddots	&	\\	
	& 		& 		& 		&		& 		& 1	\\	
\end{pmatrix}.\]
Additionally, $w$ is the vector whose only non-zero entry is the $\hat{i}^{\text{th}}$ element, $ \beta_1^{\{\mu\}}$,
\[w = (0, \dots 0, \beta_1^{\{\mu\}},0, \dots, 0)^T \in \mathbb{Z}^{\nu + r}.\]

Of course, we must compute the $\beta_i$ to $p_l$-adic precision at least $\mu$ in order to avoid errors here. 
Let $\gamma = (n_1, \dots, n_{\nu}, a_1, \dots, a_r) \in \mathbb{R}^{\nu + r}$ be a solution to our $S$-unit equation. 

\begin{lemma}
Suppose $\ord_{p_l}(\delta_1) = 0$ and 
\[\sum_{i = 1}^v n_{i}a_{li} > \frac{1}{p_l-1} - \ord_{p_l}(\delta_2).\]
Then the following equivalence holds: 
\begin{align*}
\sum_{i = 1}^v n_{i}a_{li}  \geq \mu - \ord_{p_l}(\delta_2) + \ord_{p_l}(\alpha_{\hat{i}}) 
	& \quad \text{ if and only if } \quad \ord_{p_l}(\Lambda_l') \geq \mu \\
	& \quad \text{ if and only if } \quad \gamma \in\Gamma_v.
\end{align*}
\end{lemma} 

\begin{remark}
Note that the conditions $\ord_{p_l}(\delta_1) = 0$ and 
\[\sum_{i = 1}^v n_{i}a_{li} > \frac{1}{p_l-1} - \ord_{p_l}(\delta_2)\]
are equivalent to 
\[\sum_{i = 1}^v n_{i}a_{li} > \max\left\{\frac{1}{p_l-1}, \ord_{p_l}(\delta_1)\right\} - \ord_{p_l}(\delta_2).\]
\end{remark}

\begin{proof}
By Lemma \ref{Lem:19.1}, the assumption means that 
\[\ord_{p_l}(\Lambda_l') = \sum_{i = 1}^v n_{i}a_{li} + \ord_{p_l}(\delta_2) - \ord_{p_l}(\alpha_{\hat{i}}).\]

Now, suppose 
\[\sum_{i = 1}^v n_{i}a_{li}  \geq \mu - \ord_{p_l}(\delta_2) + \ord_{p_l}(\alpha_{\hat{i}}).\]
We thus have
\begin{align*}
\ord_{p_l}(\Lambda_l')	
	& = \sum_{i = 1}^v n_{i}a_{li} + \ord_{p_l}(\delta_2) - \ord_{p_l}(\alpha_{\hat{i}})\\
	& \geq  \mu - \ord_{p_l}(\delta_2) + \ord_{p_l}(\alpha_{\hat{i}}) + \ord_{p_l}(\delta_2) - \ord_{p_l}(\alpha_{\hat{i}})\\	
	& = \mu.
\end{align*}
Conversely, suppose $\ord_{p_l}(\Lambda_l') \geq \mu$. Then
\[\mu \leq \ord_{p_l}(\Lambda_l') = \sum_{i = 1}^v n_{i}a_{li} + \ord_{p_l}(\delta_2) - \ord_{p_l}(\alpha_{\hat{i}}).\]
That is, 
\[\sum_{i = 1}^v n_{i}a_{li} \geq \mu - \ord_{p_l}(\delta_2) + \ord_{p_l}(\alpha_{\hat{i}}).\]
Hence, it follows that $\displaystyle \sum_{i = 1}^v n_{i}a_{li} \geq \mu - \ord_{p_l}(\delta_2) + \ord_{p_l}(\alpha_{\hat{i}})$ if and only if $\ord_{p_l}(\Lambda_l') \geq \mu$.

Now, suppose $\gamma = (n_1, \dots, n_{\nu}, a_1, \dots, a_r) \in \mathbb{R}^{\nu + r}$ is a solution to our $S$-unit equation. Suppose further that $\displaystyle \sum_{i = 1}^v n_{i}a_{li} \geq \mu - \ord_{p_l}(\delta_2) + \ord_{p_l}(\alpha_{\hat{i}})$ so that $\ord_{p_l}(\Lambda_l') \geq \mu$. Let
\[\lambda = \frac{1}{p^{\mu}}\sum_{i = 1}^{\nu + r + 1}b_i(-\beta_i^{\{\mu\}})\]
and consider the $(\nu + r)$-dimensional vector
\[x= (n_1, \dots, n_{\hat{i} -1}, \lambda, n_{\hat{i} +1}, \dots, n_{\nu}, a_1, \dots, a_r)^T.\]
We claim $x \in \mathbb{Z}^{\nu + r}$. That is, $\lambda \in \mathbb{Z}$, meaning that $\sum_{i = 1}^{\nu + r + 1}b_i(-\beta_i^{\{\mu\}})$ is divisible by $p^{\mu}$, or equivalently, 
\[\ord_p\left(\sum_{i = 1}^{\nu + r + 1}b_i(-\beta_i^{\{\mu\}})\right) \geq \mu.\]
Indeed, since 
\[\ord_{p_{l}}\left(\beta_{i}^{\{\mu\}}-\beta_{i}\right) \geq \mu \quad \text { for } i=1, \dots, 1+\nu+r,\]
by definition, it follows that $\beta_{i}^{\{\mu\}}$ and $\beta_{i}$ share the first $\mu - 1$ terms and thus $\ord_p(\beta_i) = \ord_p(\beta_i^{\{\mu\}})$.
Now, to compute this order, we only need to concern ourselves with the first non-zero term in the series expansion of $\sum_{i = 1}^{\nu + r + 1}b_i(-\beta_i^{\{\mu\}})$. Since $\beta_{i}^{\{\mu\}}$ and $\beta_{i}$ share the first $\mu - 1$ terms, it follows that showing
\[\ord_p\left(\sum_{i = 1}^{\nu + r + 1}b_i(-\beta_i^{\{\mu\}})\right) \geq \mu\]
is equivalent to showing that 
\[\ord_p\left(\sum_{i = 1}^{\nu + r + 1}b_i(-\beta_i)\right) \geq \mu \implies \ord_{p_l}(\Lambda_l') \geq \mu.\]
This latter inequality is true by assumption. Thus $\lambda \in \mathbb{Z}$. 

Then, computing $A_{\mu}x + w$ yields
\begin{align*}
A_{\mu}x + w & = 
\begin{pmatrix}
1	& 		&		&		&		&		&	\\
	& \ddots	& 		&		& 0		& 		&	\\
	&		& 1		&		&		&		&	\\
	\beta_2^{\{\mu\}}& \cdots & \beta_{\hat{i} - 1}^{\{\mu\}} & p_l^{\mu} & \beta_{\hat{i} + 1}^{\{\mu\}}& \cdots &\beta_{1+ \nu+ r}^{\{\mu\}}\\
	& 		& 		& 		& 1		&		&	\\	
	& 0		& 		& 		&		& \ddots	&	\\	
	& 		& 		& 		&		& 		& 1	\\	
\end{pmatrix}
\begin{pmatrix}
n_1 \\ \vdots \\ n_{\hat{i} -1} \\ \lambda \\ n_{\hat{i} +1} \\ \vdots \\ n_{\nu} \\ a_1 \\ \vdots \\ a_r \end{pmatrix}
+ \begin{pmatrix}
0 \\ \vdots \\ 0 \\ \beta_1^{\{\mu\}} \\ 0\\ \vdots \\ \vdots \\ \vdots \\ 0
\end{pmatrix}\\
&= \begin{pmatrix}
1	& 		&		&		&		&		&	\\
	& \ddots	& 		&		& 0		& 		&	\\
	&		& 1		&		&		&		&	\\
	\beta_2^{\{\mu\}}& \cdots & \beta_{\hat{i} - 1}^{\{\mu\}} & p_l^{\mu} & \beta_{\hat{i} + 1}^{\{\mu\}}& \cdots &\beta_{1+ \nu+ r}^{\{\mu\}}\\
	& 		& 		& 		& 1		&		&	\\	
	& 0		& 		& 		&		& \ddots	&	\\	
	& 		& 		& 		&		& 		& 1	\\	
\end{pmatrix}\begin{pmatrix}
b_2 \\ \vdots \\ b_{\hat{i} -1} \\ \lambda \\ b_{\hat{i} +1} \\ \vdots \\ b_{\nu + r + 1} 
\end{pmatrix}
+ \begin{pmatrix}
0 \\ \vdots \\ 0 \\ \beta_1^{\{\mu\}} \\ 0 \\ \vdots \\ 0 
\end{pmatrix}\\
& = \begin{pmatrix}
b_2 \\ \vdots \\ b_{\hat{i} -1} \\ 
 b_2\beta_2^{\{\mu\}} + \cdots + b_{\hat{i} - 1}\beta_{\hat{i} - 1}^{\{\mu\}} + \lambda p_l^{\mu} + b_{\hat{i} + 1}\beta_{\hat{i} + 1}^{\{\mu\}} + \cdots + b_{\nu+r+ 1}\beta_{1+\nu+r}^{\{\mu\}} + \beta_1^{\{\mu\}} \\
b_{\hat{i} +1} \\ \vdots \\ b_{\nu+r+1} \\
\end{pmatrix}\\
\end{align*}
Now, 
\[ \lambda p_l^{\mu} = p^{\mu}\frac{1}{p^{\mu}}\sum_{i = 1}^{\nu + r + 1}b_i(-\beta_i^{\{\mu\}}) = \sum_{i = 1}^{\nu + r + 1}b_i(-\beta_i^{\{\mu\}}),\]
hence
\begin{align*}
& b_2\beta_2^{\{\mu\}} + \cdots + b_{\hat{i} - 1}\beta_{\hat{i} - 1}^{\{\mu\}} + b_{\hat{i} + 1}\beta_{\hat{i} + 1}^{\{\mu\}} + \cdots + b_{\nu+r+ 1}\beta_{1+\nu+r}^{\{\mu\}} + \lambda p_l^{\mu} + \beta_1^{\{\mu\}}\\
& = b_1 \beta_1^{\{\mu\}} + b_2\beta_2^{\{\mu\}} + \cdots + b_{\hat{i} - 1}\beta_{\hat{i} - 1}^{\{\mu\}} + b_{\hat{i} + 1}\beta_{\hat{i} + 1}^{\{\mu\}} + \cdots + b_{\nu+r+ 1}\beta_{1+\nu+r}^{\{\mu\}} + \sum_{i = 1}^{\nu + r + 1}b_i(-\beta_i^{\{\mu\}}) \\
& = b_{\hat{i}}(-\beta_{\hat{i}}^{\{\mu\}})\\
& = b_{\hat{i}}
\end{align*}
where the last equality follows from the fact that 
\[-\beta_i = \frac{\alpha_{\hat{i}}}{\alpha_{\hat{i}}} =1.\]
Thus, 
\[A_{\mu}x + w = \begin{pmatrix}
b_2 \\ \vdots \\ b_{\hat{i} -1} \\ b_{\hat{i}} \\ b_{\hat{i} +1} \\ \vdots \\ b_{\nu+r+1}
\end{pmatrix} = 
\begin{pmatrix}
n_1 \\ \vdots \\ n_{\nu} \\ a_1 \\ \vdots \\ a_r \end{pmatrix} = \gamma.\]
Thus, it follows that $\gamma \in \Gamma_v$. 
CONVERSELY STILL NEED TO SHOW THE CONVERSE, THAT IS 
\[m'\in\Gamma_v \quad \text{ implies } \quad \sum_{i = 1}^v n_{i}a_{li}  \geq \mu - \ord_{p_l}(\delta_2) + \ord_{p_l}(\alpha_{\hat{i}}).\]


%Our assumption gives $n_p-a_p\geq \max(0,\ord_p(\mu_0))-\ord_p(\lambda_0)$ and then Lemma~\ref{lem:padiccomp} implies that $\ord_p(\mu_0)=0$. Therefore, on using again our assumption which assures that $n_p-a_p> \tfrac{1}{p-1}-\ord_p(\lambda_0)$, we see that an application of Lemma~\ref{lem:padiccomp}  gives  $$(n_p-a_p)+\ord_p(\lambda_0)=\ord_p(\Lambda_p)=\ord_p(\Lambda'_p)+\ord_p(\xi_p).$$
%Thus $n_p-a_p\geq l_v'-\ord_p(\xi_p/\lambda_0)$ if and only if $\ord_p(\Lambda'_p)\geq l_v'$. Further, the TdW arguments show that $\ord_p(\Lambda'_p)\geq l_v'$ if and only if $m'\in\Gamma_v$. On combining we deduce the lemma.
\end{proof}

\begin{remark}
In the case $n=3$, the construction of $\Lambda'_p$ is simpler since there are only two cases to consider (either $g_p$ splits completely over $\mathbb{Q}_p$, or it has square factor).
\end{remark}


%
%
%
%
%
%
%
%
%
%
%
%Put
%\[Q = \sum_{i = 2}^{v+2} W_i^2 B_i^2.\]
%
%\begin{lem} \label{lem:LLL}
%If $\ell(\Gamma_{\mu},\mathbf{y}) > Q^{1/2}$ then
%\[\sum_{i = 1}^v n_{i}a_{li} \leq \max\left\{ \frac{1}{p_l-1} - \ord_{p_l}(\delta_2), \mu - d_l - 1,0\right\}\]
%\end{lem}
%
%\begin{proof}
%We prove the contrapositive. Assume 
%\[\sum_{i = 1}^v n_{i}a_{li} > \frac{1}{p_l-1} - \ord_{p_l}(\delta_2), \quad \sum_{i = 1}^v n_{i}a_{li} > \mu - d_l 
%\quad \text{ and } \quad \sum_{i = 1}^v n_{i}a_{li} > 0.\]
%Consider the vector
%\[\mathbf{x} = A_{\mu}
%\begin{pmatrix}
%b_2\\
%\vdots\\
%b_{\hat{i}-1}\\
%b_{\hat{i}+1}\\
%\vdots\\
%b_{v+2}\\
%\lambda
%\end{pmatrix}
%= 
%\begin{pmatrix}
%W_2b_2\\
%\vdots\\
%W_{\hat{i}-1}b_{\hat{i}-1}\\
%W_{\hat{i}+1}b_{\hat{i}+1}\\
%\vdots\\
%W_{v+2}b_{v+2}\\
%-W_{\hat{i}}b_{\hat{i}}
%\end{pmatrix}
%+ \mathbf{y}.\]
%By Lemma~\ref{Lem:19.1},  
%\[\ord_{p_l}\left( \sum_{i=1}^{v+2}b_i(-\beta_i)\right) = \ord_{p_l}(\Lambda_l') \geq\sum_{i = 1}^v n_{i}a_{li} + d_l \geq \mu.\]
%Since $\ord_{p_l}(\beta_i^{\{\mu\}} - \beta_i) \geq \mu$ for $i = 1, \dots, v+2$, it follows that
%\[\ord_{p_l}\left( \sum_{i=1}^{v+2}b_i(-\beta_i^{\{\mu\}})\right) \geq \mu,\]
%so that $\lambda \in \mathbb{Z}$. Hence $\mathbf{x} \in \Gamma_{\mu}$. Now $\sum_{i = 1}^v n_{i}a_{li} > 0$ so that there exists some $i$ such that $n_ia_{li} \neq 0$, and in particular, $b_{1+i} = n_i \neq 0$. Thus we cannot have $\mathbf{x} = \mathbf{y}$. Therefore, 
%\[\ell(\Gamma_{\mu}, \mathbf{y})^2 \leq |\mathbf{x} - \mathbf{y}|^2 = \sum_{i = 2}^{v+2}W_i^2 b_i^2
%\leq  \sum_{i = 2}^{v+2}W_i^2 |b_i|^2 \leq  \sum_{i = 2}^{v+2}W_i^2 B_i^2 = Q.\]
%\end{proof}

%The reduction procedure works as follows. Taking $A_{\mu}$ as input, we first compute an LLL-reduced basis for $\Gamma_{\mu}$. Then, we find a lower bound for $\ell(\Gamma_{\mu}, \mathbf{y})$. If the lower bound is not greater than $Q^{1/2}$ so that Lemma \ref{lem:LLL} does not give a new upper bound, we increase $\mu$ and try the procedure again. If we find that several increases of $\mu$ have failed to yield a new upper bound $N_l$ and that the value of $\mu$ has become significantly larger than it was initially, we move onto the next $l \in \{1, \dots, v\}$.
%
%If the lower bound is greater than $Q^{1/2}$, Lemma \ref{lem:LLL} gives a new upper bound $N_l$ for $\sum_{i = 1}^v n_{i}a_{li}$ and hence for $m$
%\[m = \frac{\sum_{j = 1}^{v}n_ja_{lj} + r_l + t_l}{\alpha_l} < \frac{N_l+ r_l + t_l}{\alpha_l} = M.\]
%If $M < 3000$, we exit the algorithm and enter the brute force search. Otherwise, we update the bounds $N_1, \dots, N_{l-1}, N_{l+1}, \dots, N_v$ via
%\[\sum_{j=1}^v n_ja_{ij} = m\alpha_i - r_i - t_i \leq M\alpha_i - r_i - t_i = N_i.\]
%Then using 
%\[|n_l| \leq \max_{1 \leq i \leq v}|n_i| \leq ||A^{-1}||_{\infty}\max_{1 \leq i\leq v}\sum_{j = 1}^v n_j a_{ij}
%\leq ||A^{-1}||_{\infty} \max_{1 \leq i\leq v}(N_i) = B_{l+1}.\]
%we update the $B_i$ and repeat the above procedure until $M < 3000$ or until no further improvement can be made on the $B_i$, in which case we move onto the next $l \in \{1, \dots, v\}$.

We define 
\[c_p=\log p\left(\max\left(\tfrac{1}{p-1},\ord_{p_l}(\delta_1)\right)-\ord_{p_l}(\delta_2)\right).\]

\begin{corollary}
Assume that $h_{p_l}(z)>\max(0,c_p)$. Then the following equivalence holds: 
\[h_{p_l}(z)\geq \log{p_l}\left(\mu - \ord_{p_l}(\delta_2) + \ord_{p_l}(\alpha_{\hat{i}})\right) \quad \text{ if and only if } \quad \gamma \in\Gamma_v.\]
\end{corollary}
\begin{proof}
Recall from Proposition~\ref{prop:heightdecomp} that 
\[h_{p_l}(z) = 
\begin{cases}
\log(p_l)|u_l - r_l| \\
0
\end{cases}.\]
Since $h_{p_l}(z) > 0$, it follows that $h_{p_l}(z) = \log(p_l)|u_l - r_l|$. Hence the assumption becomes
\begin{align*}
&\log(p_l)|u_l - r_l| = h_{p_l}(z) > \log p\left(\max\left(\tfrac{1}{p-1},\ord_{p_l}(\delta_1)\right)-\ord_{p_l}(\delta_2)\right)\\
&|u_l - r_l| = h_{p_l}(z) > \left(\max\left(\tfrac{1}{p-1},\ord_{p_l}(\delta_1)\right)-\ord_{p_l}(\delta_2)\right) \\
& \sum_{j = 1}^{\nu}n_ja_{lj} > \left(\max\left(\tfrac{1}{p-1},\ord_{p_l}(\delta_1)\right)-\ord_{p_l}(\delta_2)\right)
\end{align*}
with conclusion
\begin{align*}
h_{p_l}(z)\geq \log{p_l}\left(\mu - \ord_{p_l}(\delta_2) + \ord_{p_l}(\alpha_{\hat{i}})\right) 
	& \quad \text{ if and only if } \quad \gamma \in\Gamma_v\\
\log(p_l)|u_l - r_l| \geq \log{p_l}\left(\mu - \ord_{p_l}(\delta_2) + \ord_{p_l}(\alpha_{\hat{i}})\right) 
	&\quad \text{ if and only if } \quad \gamma \in\Gamma_v\\
|u_l - r_l| \geq \left(\mu - \ord_{p_l}(\delta_2) + \ord_{p_l}(\alpha_{\hat{i}})\right) 
	& \quad \text{ if and only if } \quad \gamma \in\Gamma_v\\
\sum_{j = 1}^{\nu}n_ja_{lj} \geq \left(\mu - \ord_{p_l}(\delta_2) + \ord_{p_l}(\alpha_{\hat{i}})\right) 
	&\quad \text{ if and only if } \quad \gamma \in\Gamma_v,
\end{align*}
which is the previous lemma. 

%This follows from the above lemma, since Proposition~\ref{prop:heightdecomp} together with $h_p(z)>0$ implies that $h_p(z)=\log p(n_p-a_p)$.
\end{proof}

Recall that we wish to prove the following lemma:
\begin{lemma}\label{lem:nonarchsieve}
Any $(x,y) \in \Sigma_v(l,h)$ is determined by some $\gamma \in \Gamma_v \cap \mathcal{E}_v$. 
\end{lemma}

\begin{proof}[Proof of Lemma~\ref{lem:nonarchsieve}]
If $(n_1, \dots, n_{\nu}, a_1, \dots, a_r) \in \mathbb{R}^{r+\nu}$ is a solution which lies in $\Sigma_v(l,h)$, then, by definition, it corresponds to a solution $(x,y)$ satisfying
\[\Sigma_v(l,h) = \{(x,y) \in \Sigma \ | \ (h_w(z))\leq h \text{ and }  (h_w(z))\nleq 0 \text{ and } h_v(z)>l_v\}.\]
Hence $h_v(z)>l_v$, where $l_v$ is a constant such that
\[\frac{l_v}{\log(p)} \geq \max\left( \frac{1}{p-1}, \ord_{p_v}(\delta_1)\right) - \ord_{p_v}(\delta_2).\]
That is, 
\[h_v(z) > l_v \geq \log(p)\left(\max\left( \frac{1}{p-1}, \ord_{p_v}(\delta_1)\right) - \ord_{p_v}(\delta_2)\right) = c_p.\]
Now, recall that $l \geq 0$ so that $l_v \geq 0$. It thus follows that 
\[h_v(z) > l_v \geq
\begin{cases}
0\\
c_p
\end{cases}
\implies h_v(z) >\max(0,c_p).\]
In other words, the condition of the previous corollary is satisfied. 

Now, recall that $l_v'$ (sometimes denoted $\mu$) is the largest element of $\mathbb{Z}_{\geq 0}$ at most
\[l_v' \leq \frac{l_v}{\log(p)} - \ord_{p_l}(\alpha_{\hat{i}}) + \ord_{p_l}(\delta_2).\]
That is
\[\frac{l_v}{\log(p)} \geq l_v' + \ord_{p_l}(\alpha_{\hat{i}}) - \ord_{p_l}(\delta_2)\]
so that
\[h_v(z) > l_v \geq \log(p)\left(\l_v' + \ord_{p_l}(\alpha_{\hat{i}}) - \ord_{p_l}(\delta_2)\right).\]
Now, by the previous corollary, we must have $\gamma \in \Gamma_v$. This shows that $(x,y)$ is determined by $\gamma=m'\in\Gamma_v$, which proves Lemma~\ref{lem:nonarchsieve}.
%
%If $h_v(z)>l_v$ then $h_v(z)>\max(0,c_p)$ since $l\geq 0$ and $l_v\geq c_p$ by assumption. Further, it holds that $l_v/\log p\geq l_v'-v(\xi_p/\lambda_0)$ by the definition of $l_v'$. Hence $h_v(z)\geq \log p(l_v'-v(\xi_p/\lambda_0))$ and then the above corollary implies that $m'\in\Gamma_v$. This shows that $(x,y)$ is determined by $\gamma=m'\in\Gamma_v$, which proves Lemma~\ref{lem:nonarchsieve}.
\end{proof}


\subsection{Non-Archimedean ellipsoid.} 
Recall that
\[h\left(\frac{\delta_2}{\lambda}\right) =  \frac{1}{[K:\mathbb{Q}]}\sum_{l = 1}^{\nu} \log(p_l)|u_l - r_l| + \frac{1}{[L:\mathbb{Q}]}\sum_{w :L \to \mathbb{C}} \log \max \left\{ \left|w\left(\frac{\delta_2}{\lambda}\right)\right|, 1\right\}.\]

We now restrict our attention to those $p \in \{p_1, \dots, p_{\nu}\}$ and study the $p$-adic valuations of the numbers appearing in \eqref{Eq:Sunit}. Let $l \in \{1, \dots, \nu\}$, corresponding to $p_l \in \{p_1, \dots, p_{\nu}\}$. Take $\mathbf{h}\in\mathbb{R}^{r+\nu}$ such that $\mathbf{h}\geq \mathbf{0}$. Let
\[b = \frac{1}{\log(2)^2}\sum_{k = 1}^{\nu} h_k^2\]
where
\[\log(2)^2q_f(\mathbf{n}) = \log(2)^2\sum_{k = 1}^{\nu}\left\lfloor\frac{\log(p_k)^2}{\log(2)^2}\right\rfloor|u_k-r_k|^2 \leq \sum_{k = 1}^{\nu} \log(p_k)^2|u_k -r_k|^2 \leq \sum_{k = 1}^{\nu} h_k^2.\]

%\[q_f(\mathbf{n}) = \sum_{l = 1}^{\nu} \lfloor\log(p_l)^2\rfloor|u_l -r_l|^2 \leq \sum_{l = 1}^{\nu} \log(p_l)^2|u_l -r_l|^2 \leq \sum_{l = 1}^{\nu} h_k^2:= b \]

%\[q_f(\mathbf{n}) = \frac{1}{[K:\mathbb{Q}]}\sum_{k = 1}^{\nu} \log(p_k)^2|u_k -r_k|^2 \leq \frac{1}{[K:\mathbb{Q}]}\sum_{k = 1}^{\nu} h_k^2 = b.\]
For each $\varepsilon_l$ in $\{\varepsilon_1, \dots, \varepsilon_r\}$, we define
%\[b_{\varepsilon_l} = 
%\begin{cases}
%\left( \frac{2}{[L:\mathbb{Q}]}\max_{\sigma:L\to \mathbb{C}} w_{\varepsilon \sigma}h_{\sigma}  + \frac{1}{[K:\mathbb{Q}]}\sum_{k = 1}^{\nu} w_{\gamma k}h_k\right)^2 & \text{ if } \sqrt{\Delta}\notin\mathbb{Q} \\
%\left( \frac{1}{[L:\mathbb{Q}]}\max_{\sigma:L\to \mathbb{C}} w_{\varepsilon \sigma}h_{\sigma} + \frac{1}{[K:\mathbb{Q}]}\sum_{k = 1}^{\nu} w_{\gamma k}h_k\right)^2 & \text{ if } \sqrt{\Delta}\in\mathbb{Q}, \\
%\end{cases}\]
%where
\[|a_l|^2 \leq \left( \frac{1}{[K:\mathbb{Q}]}\sum_{k = 1}^{\nu} w_{\gamma l k}h_k + \frac{1}{[L:\mathbb{Q}]}\sum_{\sigma:L\to \mathbb{C}} w_{\varepsilon l \sigma}h_{\sigma}\right)^2=:b_{\varepsilon_l}.\]
Note that here, we do not distinguish any $\varepsilon_l^*$. 

We define the ellipsoid $\mathcal{E}_l \subseteq \mathbb{R}^{\nu + r}$ by 
\begin{align}\label{def:ellp}
& \mathcal{E}_l=\{q_l(\mathbf{x})\leq (1 + r)(bb_{\varepsilon_1}\cdots b_{\varepsilon_r}); \ \mathbf{x}\in\mathbb{R}^{r+\nu}\}, \quad \text{ where }\\
&\quad \quad \quad q_l(\mathbf{x})= (b_{\varepsilon_1}\cdots b_{\varepsilon_r})\left( q_f(x_1, \dots, x_{\nu}) + \sum_{i = 1}^r\frac{b}{b_{\varepsilon_i}}x_{\varepsilon_i}^2\right)\\
&\quad \quad \quad q_l(\mathbf{x})=\left((b_{\varepsilon_1}\cdots b_{\varepsilon_r})\cdot q_f(x_1, \dots, x_{\nu}) + (b_{\varepsilon_1}\cdots b_{\varepsilon_r})\sum_{i = 1}^r\frac{b}{b_{\varepsilon_i}}x_{\varepsilon_i}^2\right)\\
&\quad \quad \quad q_l(\mathbf{x})=\left((b_{\varepsilon_1}\cdots b_{\varepsilon_r})\cdot q_f(x_1, \dots, x_{\nu}) + \sum_{i = 1}^rb(b_{\varepsilon_1}\cdots b_{\varepsilon_{i-1}}b_{\varepsilon_{i+1}}b_{\varepsilon_r})x_{\varepsilon_i}^2\right)
\end{align}
where
\[q_f(\mathbf{y}) = (A\mathbf{y})^{\text{T}}D^2A\mathbf{y}.\]

To generate the matrix for this ellipsoid, recall that $I$ is the set of all indices $i' \in \{2, \dots, 1+ \nu + r\}$ for which
\[\ord_{p_l}(\alpha_{i'}) = \min_{2\leq i\leq 1+ \nu+ r} \ord_{p_l}(\alpha_i).\]
We note that we are always in the so-called \textit{special case}, where there is some index $i' \in I$ such that $\alpha_i/\alpha_{i'} \in \mathbb{Q}_{p_l}$ for $i = 1, \dots, 1+ \nu+ r$. 

Now we state several relatively-easy-to-check conditions that each imply that we are always in the special case for degree $3$ Thue-Mahler equations. Moreover, each condition implies that we have $\frac{\alpha_{i_1}}{\alpha_{i_2}} \in \mathbb{Q}_p$ for every $i_1, i_2 \in\{1, \dots, 1+\nu+r\}$.

\begin{enumerate}[(a)]
\item $\alpha_1, \dots, \alpha_{1+\nu+r} \in \mathbb{Q}_p$
\item $g(t)$ has three or more linear factors in $\mathbb{Q}_p[t]$ and $\theta^{(i_0)}, \theta^{(j)}, \theta^{(k)}$ are roots of such polynomials. 
\item $g(t)$ has an irreducible factor in $\mathbb{Q}_p[t]$ of degree two, and $\theta^{(j)}, \theta^{(k)}$ are roots of this
\item $g(t)$ has a non-linear irreducible factor in $\mathbb{Q}_p[t]$ that splits completely in the extension of $\mathbb{Q}_p$ that it generates and $\theta^{(j)}, \theta^{(k)}$ are roots of this factor
\end{enumerate}

\begin{proof}
It is obvious that (a) implies $\alpha_{i_1}/\alpha_{i_2} \in \mathbb{Q}_p$ for every $i_1, i_2 \in\{1, \dots, 1+\nu+r\}$. If (b) holds, then $\delta_1, \gamma_{i}^{(k)} / \gamma_{i}^{(j)} (i=1, \ldots, \nu), \varepsilon_{i}^{(k)} / \varepsilon_{i}^{(j)} (i=1, \ldots, r)$ all belong to $\mathbb{Q}_p$, which, since $\mathbb{Q}_p$ is complete, implies (a). Now, (c) implies (d). We claim that (d) implies $\alpha_{i_1}/\alpha_{i_2} \in \mathbb{Q}_p$ for every $i_1, i_2 \in\{1, \dots, 1+\nu+r\}$. To see this, assume (d), let $L$ be the extension of $\mathbb{Q}_p$ generated by the factor of $g(t)$ in question, and consider any $\alpha, \beta \in L$. The automorphisms on $L$ that maps $\theta^{(j)}$ to $\theta^{(k)}$ multiplies the logarithms $\log _{p_{l}}\left(\alpha^{(k)} / \alpha^{(j)}\right)$ and $\log _{p_{l}}\left(\beta^{(k)} / \beta^{(j)}\right)$ by $-1$ and hence fixes the quotient
\begin{equation}\label{logs}
\frac{\log _{p_{l}}\left(\alpha^{(k)} / \alpha^{(j)}\right)}{\log _{p_{l}}\left(\beta^{(k)} / \beta^{(j)}\right)}.
\end{equation}
Therefore, since $L$ is Galois, this quotient belongs to $\mathbb{Q}_p$. Since $\alpha_{i_1}/\alpha_{i_2}$ is of the form \eqref{logs} for every $i_1, i_2 \in\{1, \dots, 1+\nu+r\}$, the claim is proved. 
\end{proof}

Now, recall that if our Thue-Mahler is only of degree $3$, it follows that $g(t)$ can only split in 3 ways in $\mathbb{Q}_p$. 
\begin{enumerate}[(a)]
\item $g(t) = g_1(t)$, where $\deg(g_1(t)) = 3$
\item $g(t) = g_1(t)g_2(t)$ where $\deg(g_1(t)) = 1$ and $\deg(g_2(t)) = 2$ (without loss of generality) 
\item $g(t) = g_1(t)g_2(t)g_3(t)$ where $\deg(g_i(t)) = 3$ for $i = 1, 2, 3$. 
\end{enumerate}
In the event that that $g(t)$ is irreducible (a), the corresponding prime ideal $\mathfrak{p}$ in $K$ has $ef = 3$, and is therefore bounded. That is, it does not appear in the set of unbounded ideals $\{\mathfrak{p}_1, \dots, \mathfrak{p}_{\nu}\}$, and we can ignore this case. The other 2 cases appear in the list above, therefore guaranteeing that $\alpha_{i_1}/\alpha_{i_2} \in \mathbb{Q}_p$ for every $i_1, i_2 \in\{1, \dots, 1+\nu+r\}$. 


% and take $h\in\mathbb{R}^{}$ with $h\geq 0$. Let $b=b(h)$ be the real number defined in \eqref{def:bbound} and for any $\epsilon\in\unit_\infty$ we define $b_\epsilon=b_\epsilon(h)$ as in \eqref{def:bepsbound}. Then we define the ellipsoid $\mathcal E_v\subseteq \RR^{\unit}$ by
%\begin{equation}\label{def:ellnonarch}
%\mathcal E_v=\{q_v(x)\leq b; \ x\in\RR^\unit\}, \quad q_v(x)=\frac{1}{|\unit|}\left(|\unit_S|q_f(x_\delta)+\sum_{\epsilon\in\unit_\infty}\frac{b}{b_\epsilon}x_{\epsilon}^2\right)
%\end{equation}
%where $x=(x_\delta)\oplus(x_\epsilon)$ with $(x_\delta)\in\RR^{\unit_S}$ and $(x_\epsilon)\in \RR^{\unit_\infty}$.
%
%

Finally, suppose that $\gamma\in \Gamma_v\cap \mathcal E_v$. Let $M=M_v$ be the matrix defining the ellipsoid 
\[\mathcal E_\tau: z^tM^tMz\leq (1 + r)(bb_{\varepsilon_1}\cdots b_{\varepsilon_r}),\]
that is,  
\begin{align*}
M &=\sqrt{b_{\varepsilon_1}\cdots b_{\varepsilon_r}}\begin{pmatrix}
	DA & 0 & \dots & 0 & 0\\
	0 & \sqrt{\frac{b}{b_{\varepsilon 1}}} & \dots & 0 & 0\\
	0 & 0  & \sqrt{\frac{b}{b_{\varepsilon 2}}} & \dots & 0\\
	\vdots & \vdots &0 &  \ddots & \vdots\\ 
	0 & 0 & \dots & \dots & \sqrt{\frac{b}{b_{\varepsilon}}} \\
	\end{pmatrix}.
\end{align*}	
Note that we never need to compute $M$, but rather $M^TM$ so that we do not need to worry about precision. In this case, 
\begin{align*}
M^TM &= b_{\varepsilon_1}\cdots b_{\varepsilon_r}\begin{pmatrix}
	A^TD^2A & 0 & \dots & 0 & 0\\
	0 & \frac{b}{b_{\varepsilon 1}} & \dots & 0 & 0\\
	0 & 0  & \frac{b}{b_{\varepsilon 2}} & \dots & 0\\
	\vdots & \vdots &0 &  \ddots & \vdots\\ 
	0 & 0 & \dots & \dots & \frac{b}{b_{\varepsilon}} \\
	\end{pmatrix}\\
	& = \begin{pmatrix}
	(b_{\varepsilon_1}\cdots b_{\varepsilon_r})A^TD^2A & 0 & \dots & 0 & 0\\
	0 & b b_{\varepsilon_2}\cdots b_{\varepsilon_r} & \dots & 0 & 0\\
	0 & 0  & bb_{\varepsilon_1}b_{\varepsilon_3}\cdots b_{\varepsilon_r} & \dots & 0\\
	\vdots & \vdots &0 &  \ddots & \vdots\\ 
	0 & 0 & \dots & \dots & bb_{\varepsilon_1}\cdots b_{\varepsilon_{r-1}} \\
	\end{pmatrix}.
\end{align*}	
	
Recall that 
\[A_{\mu}= \begin{pmatrix}
1	& 		&		&		&		&		&	\\
	& \ddots	& 		&		& 0		& 		&	\\
	&		& 1		&		&		&		&	\\
	\beta_2^{\{\mu\}}& \cdots & \beta_{\hat{i} - 1}^{\{\mu\}} & p_l^{\mu} & \beta_{\hat{i} + 1}^{\{\mu\}}& \cdots &\beta_{1+ \nu+ r}^{\{\mu\}}\\
	& 		& 		& 		& 1		&		&	\\	
	& 0		& 		& 		&		& \ddots	&	\\	
	& 		& 		& 		&		& 		& 1	\\	
\end{pmatrix}\]
$A_{\mu}x + w$ define the lattice $\Gamma_v$ where $\gamma\in \Gamma_v\cap \mathcal E_v$. In particular, since $\gamma\in \Gamma_v\cap \mathcal E_v$, there exists $x\in \mathbb{R}^{r + \nu}$ such that $\gamma=\Gamma_v x+w$ and ${\gamma^tM^tM\gamma\leq (1 + r)(bb_{\varepsilon_1}\cdots b_{\varepsilon_r})}$. We thus have
\[(\Gamma_v x+w)^tM^tM(\Gamma_\tau x+w) \leq (1 + r)(bb_{\varepsilon_1}\cdots b_{\varepsilon_r}).\]
As $A_{\tau}$ is clearly invertible, with matrix inverse
\[A_{\tau}^{-1} = \begin{pmatrix}
1	& 		&		&		&		&		&	\\
	& \ddots	& 		&		& 0		& 		&	\\
	&		& 1		&		&		&		&	\\
	-\frac{\beta_2^{\{\mu\}}}{p_l^{\mu}} & \cdots & -\frac{\beta_{\hat{i} - 1}^{\{\mu\}}}{p_l^{\mu}} & \frac{1}{p_l^{\mu}} & -\frac{\beta_{\hat{i} + 1}^{\{\mu\}}}{p_l^{\mu}}& \cdots &-\frac{\beta_{1+ \nu+ r}^{\{\mu\}}}{p_l^{\mu}}\\
	& 		& 		& 		& 1		&		&	\\	
	& 0		& 		& 		&		& \ddots	&	\\	
	& 		& 		& 		&		& 		& 1	\\
	\end{pmatrix},\]
we can find a vector $c$ such that $A_{\tau}c = -w$. Indeed, this vector is $c = A_{\tau}^{-1}(-w)$, where
\[c = A_{\tau}^{-1}w =  \begin{pmatrix}
1	& 		&		&		&		&		&	\\
	& \ddots	& 		&		& 0		& 		&	\\
	&		& 1		&		&		&		&	\\
	-\frac{\beta_2^{\{\mu\}}}{p_l^{\mu}} & \cdots & -\frac{\beta_{\hat{i} - 1}^{\{\mu\}}}{p_l^{\mu}} & \frac{1}{p_l^{\mu}} & -\frac{\beta_{\hat{i} + 1}^{\{\mu\}}}{p_l^{\mu}}& \cdots &-\frac{\beta_{1+ \nu+ r}^{\{\mu\}}}{p_l^{\mu}}\\
	& 		& 		& 		& 1		&		&	\\	
	& 0		& 		& 		&		& \ddots	&	\\	
	& 		& 		& 		&		& 		& 1	\\
	\end{pmatrix}\begin{pmatrix}
	0 \\ \vdots \\ 0 \\ -\beta_1^{\{\mu\}} \\ 0 \\ \vdots \\ 0
	\end{pmatrix}
	= \begin{pmatrix}
	0 \\ \vdots \\ 0 \\-\frac{\beta_1^{\{\mu\}}}{p^{\{\mu\}}} \\ 0 \\ \vdots \\ 0
\end{pmatrix}.\]
Now, 
\begin{align*}
(1 + r)(bb_{\varepsilon_1}\cdots b_{\varepsilon_r})
	& \geq (\Gamma_\tau x+w)^tM^tM(\Gamma_\tau x+w) \\
	& =  (\Gamma_\tau x- (-w))^tM^tM(\Gamma_\tau x-(-w)) \\
	& = (\Gamma_\tau x-\Gamma_{\tau}c)^tM^tM(\Gamma_\tau x-\Gamma_{\tau}c)\\
	& = (\Gamma_\tau (x-c))^tM^tM(\Gamma_\tau (x-c))\\
	& = (x-c)^t(M\Gamma_{\tau})^tM\Gamma_\tau(x-c)\\
	& = (x-c)^tB^tB(x-c)
\end{align*}
where $B = M\Gamma_\tau$. That is, we are left to solve
\[(x-c)B^tB(x-c) \leq (1 + r)(bb_{\varepsilon_1}\cdots b_{\varepsilon_r}).\]




Recall that $\gamma \in \Gamma_v$ means
\[A_{\mu}x + w = \begin{pmatrix}
b_2 \\ \vdots \\ b_{\hat{i} -1} \\ b_{\hat{i} +1} \\ \vdots \\ b_{\nu+r+1} \\ b_{\hat{i}}
\end{pmatrix} = \gamma.\]


%---------------------------------------------------------------------------------------------------------------------------------------------%

\endinput

Any text after an \endinput is ignored.
You could put scraps here or things in progress.


%    3. Notes
%    4. Footnotes

%    5. Bibliography
\begin{singlespace}
\raggedright
\bibliographystyle{abbrvnat}
\bibliography{biblio}
\end{singlespace}

\appendix
%    6. Appendices (including copies of all required UBC Research
%       Ethics Board's Certificates of Approval)
%\include{reb-coa}	% pdfpages is useful here
\chapter{Supporting Materials}

This would be any supporting material not central to the dissertation.
For example:
\begin{itemize}
\item additional details of methodology and/or data;
\item diagrams of specialized equipment developed.;
\item copies of questionnaires and survey instruments.
\end{itemize}


\backmatter
%    7. Index
% See the makeindex package: the following page provides a quick overview
% <http://www.image.ufl.edu/help/latex/latex_indexes.shtml>


\end{document}

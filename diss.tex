%--------------------------------------------------------------------------------------------------------------------------------------------%
% Template for a UBC-compliant dissertation
% At the minimum, you will need to change the information found
% after the "Document meta-data"
%
%!TEX TS-program = pdflatex
%!TEX encoding = UTF-8 Unicode

%% The ubcdiss class provides several options:
%%   gpscopy (aka fogscopy)
%%       set parameters to exactly how GPS specifies
%%         * single-sided
%%         * page-numbering starts from title page
%%         * the lists of figures and tables have each entry prefixed
%%           with 'Figure' or 'Table'
%%       This can be tested by `\ifgpscopy ... \else ... \fi'
%%   10pt, 11pt, 12pt
%%       set default font size
%%   oneside, twoside
%%       whether to format for single-sided or double-sided printing
%%   balanced
%%       when double-sided, ensure page content is centred
%%       rather than slightly offset (the default)
%%   singlespacing, onehalfspacing, doublespacing
%%       set default inter-line text spacing; the ubcdiss class
%%       provides \textspacing to revert to this configured spacing
%%   draft
%%       disable more intensive processing, such as including
%%       graphics, etc.
%%

% For submission to GPS
\documentclass[gpscopy,onehalfspacing,11pt]{ubcdiss}

% For your own copies (looks nicer)
% \documentclass[balanced,twoside,11pt]{ubcdiss}

%--------------------------------------------------------------------------------------------------------------------------------------------%%--------------------------------------------------------------------------------------------------------------------------------------------%
%%
%% FONTS:
%% 
%% The defaults below configures Times Roman for the serif font,
%% Helvetica for the sans serif font, and Courier for the
%% typewriter-style font.  Configuring fonts can be time
%% consuming; we recommend skipping to END FONTS!
%% 
%% If you're feeling brave, have lots of time, and wish to use one
%% your platform's native fonts, see the commented out bits below for
%% XeTeX/XeLaTeX.  This is not for the faint at heart. 
%% (And shouldn't you be writing? :-)
%%

%% NFSS font specification (New Font Selection Scheme)
\usepackage{times,courier}
\usepackage[scaled=.92]{helvet}

%% Math or theory people may want to include the handy AMS macros
%\usepackage{amssymb}
%\usepackage{amsmath}
%\usepackage{amsfonts}

%% The pifont package provides access to the elements in the dingbat font.   
%% Use \ding{##} for a particular dingbat (see p7 of psnfss2e.pdf)
%%   Useful:
%%     51,52 different forms of a checkmark
%%     54,55,56 different forms of a cross (saltyre)
%%     172-181 are 1-10 in open circle (serif)
%%     182-191 are 1-10 black circle (serif)
%%     192-201 are 1-10 in open circle (sans serif)
%%     202-211 are 1-10 in black circle (sans serif)
%% \begin{dinglist}{##}\item... or dingautolist (which auto-increments)
%% to create a bullet list with the provided character.
\usepackage{pifont}

%--------------------------------------------------------------------------------------------------------------------------------------------%
%% Configure fonts for XeTeX / XeLaTeX using the fontspec package.
%% Be sure to check out the fontspec documentation.
%\usepackage{fontspec,xltxtra,xunicode}	% required
%\defaultfontfeatures{Mapping=tex-text}	% recommended
%% Minion Pro and Myriad Pro are shipped with some versions of
%% Adobe Reader.  Adobe representatives have commented that these
%% fonts can be used outside of Adobe Reader.
%\setromanfont[Numbers=OldStyle]{Minion Pro}
%\setsansfont[Numbers=OldStyle,Scale=MatchLowercase]{Myriad Pro}
%\setmonofont[Scale=MatchLowercase]{Andale Mono}

%% Other alternatives:
%\setromanfont[Mapping=tex-text]{Adobe Caslon}
%\setsansfont[Scale=MatchLowercase]{Gill Sans}
%\setsansfont[Scale=MatchLowercase,Mapping=tex-text]{Futura}
%\setmonofont[Scale=MatchLowercase]{Andale Mono}
%\newfontfamily{\SYM}[Scale=0.9]{Zapf Dingbats}
%% END FONTS
%--------------------------------------------------------------------------------------------------------------------------------------------%%--------------------------------------------------------------------------------------------------------------------------------------------%



%--------------------------------------------------------------------------------------------------------------------------------------------%%--------------------------------------------------------------------------------------------------------------------------------------------%
%%
%% Recommended packages
%%
\usepackage{checkend}	% better error messages on left-open environments
\usepackage{graphicx}	% for incorporating external images

%% booktabs: provides some special commands for typesetting tables as used
%% in excellent journals.  Ignore the examples in the Lamport book!
\usepackage{booktabs}

%% listings: useful support for including source code listings, with
%% optional special keyword formatting.  The \lstset{} causes
%% the text to be typeset in a smaller sans serif font, with
%% proportional spacing.
\usepackage{listings}
\lstset{basicstyle=\sffamily\scriptsize,showstringspaces=false,fontadjust}

%% The acronym package provides support for defining acronyms, providing
%% their expansion when first used, and building glossaries.  See the
%% example in glossary.tex and the example usage throughout the example
%% document.
%% NOTE: to use \MakeTextLowercase in the \acsfont command below,
%%   we *must* use the `nohyperlinks' option -- it causes errors with
%%   hyperref otherwise.  See Section 5.2 in the ``LaTeX 2e for Class
%%   and Package Writers Guide'' (clsguide.pdf) for details.
\usepackage[printonlyused,nohyperlinks]{acronym}
%% The ubcdiss.cls loads the `textcase' package which provides commands
%% for upper-casing and lower-casing text.  The following causes
%% the acronym package to typeset acronyms in small-caps
%% as recommended by Bringhurst.
\renewcommand{\acsfont}[1]{{\scshape \MakeTextLowercase{#1}}}

%% color: add support for expressing colour models.  Grey can be used
%% to great effect to emphasize other parts of a graphic or text.
%% For an excellent set of examples, see Tufte's "Visual Display of
%% Quantitative Information" or "Envisioning Information".
\usepackage{color}
\definecolor{greytext}{gray}{0.5}

%% comment: provides a new {comment} environment: all text inside the
%% environment is ignored.
%%   \begin{comment} ignored text ... \end{comment}
\usepackage{comment}

%% The natbib package provides more sophisticated citing commands
%% such as \citeauthor{} to provide the author names of a work,
%% \citet{} to produce an author-and-reference citation,
%% \citep{} to produce a parenthetical citation.
%% We use \citeeg{} to provide examples
\usepackage[numbers,sort&compress]{natbib}
\newcommand{\citeeg}[1]{\citep[e.g.,][]{#1}}

%% The titlesec package provides commands to vary how chapter and
%% section titles are typeset.  The following uses more compact
%% spacings above and below the title.  The titleformat that follow
%% ensure chapter/section titles are set in singlespace.
\usepackage[compact]{titlesec}
\titleformat*{\section}{\singlespacing\raggedright\bfseries\Large}
\titleformat*{\subsection}{\singlespacing\raggedright\bfseries\large}
\titleformat*{\subsubsection}{\singlespacing\raggedright\bfseries}
\titleformat*{\paragraph}{\singlespacing\raggedright\itshape}

%% The caption package provides support for varying how table and
%% figure captions are typeset.
\usepackage[format=hang,indention=-1cm,labelfont={bf},margin=1em]{caption}

%% url: for typesetting URLs and smart(er) hyphenation.
%% \url{http://...} 
\usepackage{url}
\urlstyle{sf}	% typeset urls in sans-serif


%--------------------------------------------------------------------------------------------------------------------------------------------%%--------------------------------------------------------------------------------------------------------------------------------------------%
%%
%% Possibly useful packages: you may need to explicitly install
%% these from CTAN if they aren't part of your distribution;
%% teTeX seems to ship with a smaller base than MikTeX and MacTeX.
%%
%\usepackage{pdfpages}	% insert pages from other PDF files
%\usepackage{longtable}	% provide tables spanning multiple pages
%\usepackage{chngpage}	% support changing the page widths on demand
%\usepackage{tabularx}	% an enhanced tabular environment

%% enumitem: support pausing and resuming enumerate environments.
%\usepackage{enumitem}

%% rotating: provides two environments, sidewaystable and sidewaysfigure,
%% for typesetting tables and figures in landscape mode.  
%\usepackage{rotating}

%% subfig: provides for including subfigures within a figure,
%% and includes being able to separately reference the subfigures.
%\usepackage{subfig}

%% ragged2e: provides several new new commands \Centering, \RaggedLeft,
%% \RaggedRight and \justifying and new environments Center, FlushLeft,
%% FlushRight and justify, which set ragged text and are easily
%% configurable to allow hyphenation.
%\usepackage{ragged2e}

%% The ulem package provides a \sout{} for striking out text and
%% \xout for crossing out text.  The normalem and normalbf are
%% necessary as the package messes with the emphasis and bold fonts
%% otherwise.
%\usepackage[normalem,normalbf]{ulem}    % for \sout

%--------------------------------------------------------------------------------------------------------------------------------------------%
%% HYPERREF:
%% The hyperref package provides for embedding hyperlinks into your
%% document.  By default the table of contents, references, citations,
%% and footnotes are hyperlinked.
%%
%% Hyperref provides a very handy command for doing cross-references:
%% \autoref{}.  This is similar to \ref{} and \pageref{} except that
%% it automagically puts in the *type* of reference.  For example,
%% referencing a figure's label will put the text `Figure 3.4'.
%% And the text will be hyperlinked to the appropriate place in the
%% document.
%%
%% Generally hyperref should appear after most other packages

%% The following puts hyperlinks in very faint grey boxes.
%% The `pagebackref' causes the references in the bibliography to have
%% back-references to the citing page; `backref' puts the citing section
%% number.  See further below for other examples of using hyperref.
%% 2009/12/09: now use `linktocpage' (Jacek Kisynski): GPS now prefers
%%   that the ToC, LoF, LoT place the hyperlink on the page number,
%%   rather than the entry text.
\usepackage[bookmarks,bookmarksnumbered,%
    allbordercolors={0.8 0.8 0.8},%
    pagebackref,linktocpage%
    ]{hyperref}
%% The following change how the the back-references text is typeset in a
%% bibliography when `backref' or `pagebackref' are used
%%
%% Change \nocitations if you'd like some text shown where there
%% are no citations found (e.g., pulled in with \nocite{xxx})
\newcommand{\nocitations}{\relax}
%%\newcommand{\nocitations}{No citations}
%%
%\renewcommand*{\backref}[1]{}% necessary for backref < 1.33
\renewcommand*{\backrefsep}{,~}%
\renewcommand*{\backreftwosep}{,~}% ', and~'
\renewcommand*{\backreflastsep}{,~}% ' and~'
\renewcommand*{\backrefalt}[4]{%
\textcolor{greytext}{\ifcase #1%
\nocitations%
\or
\(\rightarrow\) page #2%
\else
\(\rightarrow\) pages #2%
\fi}}


%% The following uses most defaults, which causes hyperlinks to be
%% surrounded by colourful boxes; the colours are only visible in
%% PDFs and don't show up when printed:
%\usepackage[bookmarks,bookmarksnumbered]{hyperref}

%% The following disables the colourful boxes around hyperlinks.
%\usepackage[bookmarks,bookmarksnumbered,pdfborder={0 0 0}]{hyperref}

%% The following disables all hyperlinking, but still enabled use of
%% \autoref{}
%\usepackage[draft]{hyperref}

%% The following commands causes chapter and section references to
%% uppercase the part name.
\renewcommand{\chapterautorefname}{Chapter}
\renewcommand{\sectionautorefname}{Section}
\renewcommand{\subsectionautorefname}{Section}
\renewcommand{\subsubsectionautorefname}{Section}

%% If you have long page numbers (e.g., roman numbers in the 
%% preliminary pages for page 28 = xxviii), you might need to
%% uncomment the following and tweak the \@pnumwidth length
%% (default: 1.55em).  See the tocloft documentation at
%% http://www.ctan.org/tex-archive/macros/latex/contrib/tocloft/
% \makeatletter
% \renewcommand{\@pnumwidth}{3em}
% \makeatother

%--------------------------------------------------------------------------------------------------------------------------------------------%%--------------------------------------------------------------------------------------------------------------------------------------------%
%%
%% Some special settings that controls how text is typeset
%%
% \raggedbottom		% pages don't have to line up nicely on the last line
% \sloppy		% be a bit more relaxed in inter-word spacing
% \clubpenalty=10000	% try harder to avoid orphans
% \widowpenalty=10000	% try harder to avoid widows
% \tolerance=1000

%% And include some of our own useful macros
% This file provides examples of some useful macros for typesetting
% dissertations.  None of the macros defined here are necessary beyond
% for the template documentation, so feel free to change, remove, and add
% your own definitions.

% We recommend that you define macros to separate the semantics
% of the things you write from how they are presented.  For example,
% you'll see definitions below for a macro \file{}: by using
% \file{} consistently in the text, we can change how filenames
% are typeset simply by changing the definition of \file{} in
% this file.

%% The following is a directive for TeXShop to indicate the main file
%%!TEX root = diss.tex

% relevant packages
\usepackage[parfill]{parskip} 	% Activate to begin paragraphs with an empty line rather than an indent
\usepackage{graphicx}
\usepackage{amssymb}
\usepackage{amsmath}
\usepackage{mathrsfs }
\usepackage{amsthm}
\usepackage{epstopdf}
\usepackage{enumerate}
\usepackage{tikz}
\usetikzlibrary{matrix}
\usepackage{listings}
\usepackage{color}
\usepackage[all]{xy}
\usepackage[english]{babel}
\usepackage{setspace}

% relevant environments
\newtheorem{theorem}{Theorem}[section]
\newtheorem{conjecture}[theorem]{Conjecture}
\newtheorem{corollary}[theorem]{Corollary}
\newtheorem{lemma}[theorem]{Lemma}
\newtheorem{properties}[theorem]{Properties}
\newtheorem{proposition}[theorem]{Proposition}
\newtheorem{problem}[theorem]{Problem}
\newtheorem{question}[theorem]{Question}

\theoremstyle{definition}
\newtheorem{Algorithm}[theorem]{Algorithm}
\newtheorem{definition}[theorem]{Definition}
\newtheorem{example}[theorem]{Example}
\newtheorem{remark}[theorem]{Remark}

% math operators
\DeclareMathOperator{\ord}{ord}
\DeclareMathOperator{\sgn}{sgn}
\DeclareMathOperator{\Cl}{Cl}
\DeclareMathOperator{\Gal}{Gal}
\newcommand{\eps}{\varepsilon}

\newcommand{\NA}{\textsc{n/a}}	% for "not applicable"
\newcommand{\eg}{e.g.,\ }	% proper form of examples (\eg a, b, c)
\newcommand{\ie}{i.e.,\ }	% proper form for that is (\ie a, b, c)
\newcommand{\etal}{\emph{et al}}

% macros for typesetting terms.
\newcommand{\file}[1]{\texttt{#1}}
\newcommand{\class}[1]{\texttt{#1}}
\newcommand{\latexpackage}[1]{\href{http://www.ctan.org/macros/latex/contrib/#1}{\texttt{#1}}}
\newcommand{\latexmiscpackage}[1]{\href{http://www.ctan.org/macros/latex/contrib/misc/#1.sty}{\texttt{#1}}}
\newcommand{\env}[1]{\texttt{#1}}
\newcommand{\BibTeX}{Bib\TeX}

% draft edit commands
\usepackage{soul}
\newcommand{\aaron}[1]{\textcolor{purple}{\footnotesize #1}}
\newcommand{\strike}[1]{\textcolor{red}{\st{#1}}}
% The following definition will also output a warning to the console
\newcommand{\edit}[1]{\typeout{**EDIT** #1}{\textcolor{blue}{#1}}}


% Define a command \doi{} to typeset a digital object identifier (DOI).
% Note: if the following definition raise an error, then you likely
% have an ancient version of url.sty.  Either find a more recent version
% (3.1 or later work fine) and simply copy it into this directory,  or
% comment out the following two lines and uncomment the third.
\DeclareUrlCommand\DOI{}
\newcommand{\doi}[1]{\href{http://dx.doi.org/#1}{\DOI{doi:#1}}}
%\newcommand{\doi}[1]{\href{http://dx.doi.org/#1}{doi:#1}}

% Useful macro to reference an online document with a hyperlink
% as well with the URL explicitly listed in a footnote
% #1: the URL
% #2: the anchoring text
\newcommand{\webref}[2]{\href{#1}{#2}\footnote{\url{#1}}}

% END


%--------------------------------------------------------------------------------------------------------------------------------------------%%--------------------------------------------------------------------------------------------------------------------------------------------%
%%
%% Document meta-data: be sure to also change the \hypersetup information
%%

\title{Computing elliptic curves over $\mathbb{Q}$ via Thue-Mahler equations and related problems}
%\subtitle{If you want a subtitle}

\author{Adela Gherga}
\previousdegree{B.Sc. Mathematics, McMaster University, 2006}
\previousdegree{M.Sc. Mathematics, McMaster University, 2010}

% What is this dissertation for?
\degreetitle{Doctor of Philosophy}

\institution{The University of British Columbia}
\campus{Vancouver}

\faculty{The Faculty of Graduate and Postdoctoral Studies}
\department{Mathematics}
\submissionmonth{July}
\submissionyear{2019}

% details of your examining committee
%\examiningcommittee{John Smith, Materials Engineering}{Supervisor}
%\examiningcommittee{Mary Maker, Materials Engineering}%
%    {Supervisory Committee Member}
%\examiningcommittee{Nebulous Name, Department}{Supervisory Committee Member}
%\examiningcommittee{Magnus Monolith, Other Department}{Additional Examiner}
%
%% details of your supervisory committee
%\supervisorycommittee{Ira Crater, Materials Engineering}%
%    {Supervisory Committee Member}
%\supervisorycommittee{Adeline Long, \textsc{CEO} of Aerial Machine
%    Transportation, Inc.}{Supervisory Committee Member}

%% hyperref package provides support for embedding meta-data in .PDF
%% files
\hypersetup{
  pdftitle={Change this title!  (DRAFT: \today)},
  pdfauthor={Adela Gherga},
  pdfkeywords={Your keywords here}
}

%--------------------------------------------------------------------------------------------------------------------------------------------%%--------------------------------------------------------------------------------------------------------------------------------------------%
%% 
%% The document content
%%

%% LaTeX's \includeonly commands causes any uses of \include{} to only
%% include files that are in the list.  This is helpful to produce
%% subsets of your thesis (e.g., for committee members who want to see
%% the dissertation chapter by chapter).  It also saves time by 
%% avoiding reprocessing the entire file.
%\includeonly{intro,conclusions}
%\includeonly{discussion}

\begin{document}

%--------------------------------------------------------------------------------------------------------------------------------------------%
%% From Thesis Components: Tradtional Thesis
%% <http://www.grad.ubc.ca/current-students/dissertation-thesis-preparation/order-components>

% Preliminary Pages (numbered in lower case Roman numerals)
%    1. Title page (mandatory)
\maketitle

%    2. Committee page (mandatory): lists supervisory committee and,
%    if applicable, the examining committee
%\makecommitteepage

%    3. Abstract (mandatory - maximum 350 words)
%% The following is a directive for TeXShop to indicate the main file
%%!TEX root = diss.tex

\chapter{Abstract}

We present a practical and efficient algorithm for solving an arbitrary Thue-Mahler equation. This algorithm uses explicit height bounds with refined sieves, combining Diophantine approximation techniques of Tzanakis-de Weger with new geometric ideas. We begin by using methods of algebraic number theory to reduce the problem of solving the Thue-Mahler equation to the problem of solving a finite collection of related Diophantine equations. In the first part of this thesis, we establish the key results which allow us to drastically reduce the number of such Diophantine equations and subsequently reduce the running time. 

In the second part of this thesis, we show that if $n \geq 3$ is a fixed integer, then there exists an effectively computable constant $c (n)$ such that if $x, y$ and $m$ are integers satisfying
$$
\frac{x^m-1}{x-1} = \frac{y^n-1}{y-1}, \; \; y>x>1, \; m > n,
$$
with $\gcd(m-1,n-1)>1$,
then $\max \{ x, y, m \} < c (n)$. In case $n \in \{ 3, 4, 5 \}$, we solve the equation completely, subject to this non-coprimality condition.  In case $n=5$, our resulting computations require a variety of innovations for solving Ramanujan-Nagell equations of the shape $f(x)=y^n$, where $f(x)$ is a given polynomial with integer coefficients (and degree at least two), and $y$ is a fixed integer. In particular, we modify our Thue-Mahler algorithm and specialize our refinements to the case of Ramanujan-Nagell equations, enabling us to fully resolve the $n = 5$ case. 

In the third part, we discuss an algorithm for finding all elliptic curves over $\mathbb{Q}$ with a given conductor. Though based on 
classical ideas derived from reducing the problem to one of solving associated Thue-Mahler equations,  our approach, in many cases at least, appears to be reasonably efficient  computationally. We provide 
details of the output derived from running the algorithm, concentrating on the cases of conductor $p$ or $p^2$, for $p$ prime, with comparisons to existing 
data. 

Finally, we specialize the Thue-Mahler algorithm to degree $3$, applying an analogue of Matshke-von K\:anel's elliptic logarithm sieve to construct a global sieve, leading to reduced search spaces. The algorithm is implemented in the Magma computer algebra system, and is part of an on-going collaborative project. 



%--------------------------------------------------------------------------------------------------------------------------------------------%

\endinput

This document provides brief instructions for using the \class{ubcdiss}
class to write a \acs{UBC}-conformant dissertation in \LaTeX.  This
document is itself written using the \class{ubcdiss} class and is
intended to serve as an example of writing a dissertation in \LaTeX.
This document has embedded \acp{URL} and is intended to be viewed
using a computer-based \ac{PDF} reader.

Note: Abstracts should generally try to avoid using acronyms.

Note: at \ac{UBC}, both the \ac{GPS} Ph.D. defence programme and the
Library's online submission system restricts abstracts to 350
words.

% Consider placing version information if you circulate multiple drafts
%\vfill
%\begin{center}
%\begin{sf}
%\fbox{Revision: \today}
%\end{sf}
%\end{center}

Any text after an \endinput is ignored.
You could put scraps here or things in progress.
\cleardoublepage

%    4. Lay Summary (Effective May 2017, mandatory - maximum 150 words)
%% The following is a directive for TeXShop to indicate the main file
%%!TEX root = diss.tex

%% https://www.grad.ubc.ca/current-students/dissertation-thesis-preparation/preliminary-pages
%% 
%% LAY SUMMARY Effective May 2017, all theses and dissertations must
%% include a lay summary.  The lay or public summary explains the key
%% goals and contributions of the research/scholarly work in terms that
%% can be understood by the general public. It must not exceed 150
%% words in length.

\chapter{Lay Summary}

Consider any collection of prime numbers $\{p_1, \dots, p_v\}$ and any collection of integers $c, c_0, \dots, c_n$. Our main result involves the \textit{Thue-Mahler} equation
\[F(x,y) = c_0x^n + c_1x^{n-1}y + \cdots + c_{n-1}xy^{n-1} + c_ny^n = cp_1^{z_1}\cdots p_v^{z_v},\]
where the values $x, y$, and $z_1, \dots, z_v$ are unknown. In particular, for any such equation, we know that there are only finitely many values of $x,y$, and $z_1, \dots, z_n$ which satisfy it. In our work, we develop an algorithm to find all of these solutions for any given collection of primes and coefficients $c_i$. The solutions to these Thue-Mahler have many important mathematical applications, and we modify and refine our algorithm for use in those applications.


\endinput

The lay or public summary explains the key goals and contributions of
the research\slash{}scholarly work in terms that can be understood by the
general public. It must not exceed 150 words in length.

\cleardoublepage

%    5. Preface
%% The following is a directive for TeXShop to indicate the main file
%%!TEX root = diss.tex

\chapter{Preface}

The work presented in \autoref{ch:Goormaghtigh} is joint work with Dr. M. Bennett and Dr. D. Kreso and has been submitted for publication \cite{BeGhKr}. I was responsible for modifying and specializing the Thue-Mahler algorithm to resolve the remaining cases, $0 \leq x \leq 720$, for $n = 5$. I implemented the resulting algorithm in Magma and performed the tests on each remaining case, as well as wrote \autoref{TM}. The remainder of the work submitted for publication was originally drafted by M. Bennett and D. Kreso. 

\autoref{ch:EllipticCurves} is work completed in collaboration with Dr. M. Bennett and Dr. A. Rechnitzer. A version of this chapter has been published and appears in M. A. Bennett, A. Gherga and A. Rechnitzer, \emph{Computing elliptic curves over $\mathbb{Q}$}, Math. Comp. 88 (2019), no. 317, 1341-1390. In this work, I modified and implemented all of the code needed to resolve the reducible and irreducible forms. Furthermore, I was responsible for running this code to generate all of the solutions and resulting elliptic curves to the forms in the section ``Examples''. I drafted the majority of this section, while the remainder of the paper was originally drafted by M. Bennett and A. Rechnitzer.

\autoref{ch:AlgorithmsForTM} and \autoref{ch:EfficientTMSolver} is part of an ongoing collaborative project, currently in preparation \cite{GhKaMaSi} with Dr. B. Matshke, Dr. R. von K\"anel, and Dr. S. Siksek. The ideas presented in \autoref{sec:PIRL} and \autoref{subsec:FactorizationTMwithoutOK} are attributed to S. Siksek. The work in \autoref{ch:EfficientTMSolver} is joint work with R. von K\"anel, to whom the new ideas are attributed. Here, I helped to develop the theory and details behind these ideas, as well as implemented and tested the algorithm presented in both chapters.



\cleardoublepage

%    6. Table of contents (mandatory - list all items in the preliminary pages
%    starting with the abstract, followed by chapter headings and
%    subheadings, bibliographies and appendices)
\tableofcontents
\cleardoublepage	% required by tocloft package

%    7. List of tables (mandatory if thesis has tables)
\listoftables
\cleardoublepage	% required by tocloft package

%    8. List of figures (mandatory if thesis has figures)
\listoffigures
\cleardoublepage	% required by tocloft package

%    9. List of illustrations (mandatory if thesis has illustrations)
%   10. Lists of symbols, abbreviations or other (optional)

%   11. Glossary (optional)
%% The following is a directive for TeXShop to indicate the main file
%%!TEX root = diss.tex

\chapter{Glossary}

This glossary uses the handy \latexpackage{acroynym} package to automatically
maintain the glossary.  It uses the package's \texttt{printonlyused}
option to include only those acronyms explicitly referenced in the
\LaTeX\ source.

% use \acrodef to define an acronym, but no listing
\acrodef{UI}{user interface}
\acrodef{UBC}{University of British Columbia}

% The acronym environment will typeset only those acronyms that were
% *actually used* in the course of the document
\begin{acronym}[ANOVA]
\acro{ANOVA}[ANOVA]{Analysis of Variance\acroextra{, a set of
  statistical techniques to identify sources of variability between groups}}
\acro{API}{application programming interface}
\acro{CTAN}{\acroextra{The }Common \TeX\ Archive Network}
\acro{DOI}{Document Object Identifier\acroextra{ (see
    \url{http://doi.org})}}
\acro{GPS}[GPS]{Graduate and Postdoctoral Studies}
\acro{PDF}{Portable Document Format}
\acro{RCS}[RCS]{Revision control system\acroextra{, a software
    tool for tracking changes to a set of files}}
\acro{TLX}[TLX]{Task Load Index\acroextra{, an instrument for gauging
  the subjective mental workload experienced by a human in performing
  a task}}
\acro{UML}{Unified Modelling Language\acroextra{, a visual language
    for modelling the structure of software artefacts}}
\acro{URL}{Unique Resource Locator\acroextra{, used to describe a
    means for obtaining some resource on the world wide web}}
\acro{W3C}[W3C]{\acroextra{the }World Wide Web Consortium\acroextra{,
    the standards body for web technologies}}
\acro{XML}{Extensible Markup Language}
\end{acronym}

% You can also use \newacro{}{} to only define acronyms
% but without explictly creating a glossary
% 
% \newacro{ANOVA}[ANOVA]{Analysis of Variance\acroextra{, a set of
%   statistical techniques to identify sources of variability between groups.}}
% \newacro{API}[API]{application programming interface}
% \newacro{GOMS}[GOMS]{Goals, Operators, Methods, and Selection\acroextra{,
%   a framework for usability analysis.}}
% \newacro{TLX}[TLX]{Task Load Index\acroextra{, an instrument for gauging
%   the subjective mental workload experienced by a human in performing
%   a task.}}
% \newacro{UI}[UI]{user interface}
% \newacro{UML}[UML]{Unified Modelling Language}
% \newacro{W3C}[W3C]{World Wide Web Consortium}
% \newacro{XML}[XML]{Extensible Markup Language}
	% always input, since other macros may rely on it

\textspacing		% begin one-half or double spacing

%   12. Acknowledgements (optional)
%% The following is a directive for TeXShop to indicate the main file
%%!TEX root = diss.tex

\chapter{Acknowledgments}

I am indebted to Dr. Michael A. Bennett for suggesting to me the line of research on which this thesis is based and for numerous comments and suggestions that helped me to improve this thesis. 

This research was funded in part by a National Sciences and Engineering Research Council Postgraduate Scholarship.

\endinput

Thank those people who helped you. 

Don't forget your parents or loved ones.

You may wish to acknowledge your funding sources.


%   13. Dedication (optional)

% Body of Thesis (not all sections may apply)
\mainmatter

\acresetall	% reset all acronyms used so far

%    1. Introduction
%% The following is a directive for TeXShop to indicate the main file
%%!TEX root = diss.tex

\chapter{Introduction}
\label{ch:Introduction}

\aaron{
i mean, the beginning is the part you're not comfortable writing, right? the longer it went on, the better it flowed. at that point you're quoting and weaving results you know well, referencing the little mental web you have woven. it seems cohesive, but also i don't understand it. 
the beginning bit seems thrown together � like Mike told you to include bits about DEs and so you begrudgingly injected something ?? 
like the very very beginning bit
anyway, i'll e-mail you back the tex file and the pdf. 
i know you're not asking for this advice but it's coming from ozgur and yaniv and they are very smart and i trust them lots:}

\aaron{
\begin{enumerate}
\item be very careful about whether you're using colloquial language, and how it might be interpreted. e.g. be careful not to insult people's work, and try to not to flip flop on how hand wavey you are being. I think I have a couple of notes in the file pertaining to each of these points
\item when citing work, either use the author names every time or don't. don't mix and match unless appropriate. why would you deny some the respect of appearing in your work, but not others? 
\item if you're going to write notes to yourself in your thesis/papers, you must have a way of ensuring that you'll see them later before you send it off. caps lock is not sufficient and yaniv and ozgur can provide examples if you need. I included a little \edit{} command for you so that you can just Cmd+F (or C-s ?? ) for all appearances of \edit in the .tex file if you use it. Has the added advantage of making PDF text blue so that everyone reading too knows that it doesn't belong. 
\end{enumerate}}

\aaron{
also my disclaimer for edits: 
1. for some reason my brain is tired today;
2. I don't know the culture of your field nor some of the very elementary things you're presenting
3. Because of 1, I tried to communicate what I wanted to say using the best language I could, but may not have always succeeded at clarity/intent/approachability ?? 
So basically, remember that it's possible that my edits deserve to be treated with a grain of salt. ??}




\aaron{This start feels outside the realm of where you're going --- it seems at once abrupt and off-topic. It would be nice to have an introductory sentence or two to get the reader on track before discussing the ``required background'' material.}
A Diophantine equation is a polynomial equation in several variables defined over the integers. The term \textit{Diophantine} refers to the Greek mathematician Diophantus of Alexandria, who studied such equations in the 3rd century A.D.  \aaron{why the history lesson? maybe you could use this as one way of motivating/introducing DEs: ``look at these things. look how long they've been studied. here's why, and here are the ways people study them. \ldots or something\ldots}

\aaron{remove separate paragraph if it's the same thought --- $f$ is a DE right? If so, then these next lines are providing additional information to what was given above, not starting a new thread.}
Let $f(x_1, \dots, x_n)$ be a polynomial with integer coefficients. We wish to study the set of solutions $(x_1, \dots, x_n) \in \mathbb{Z}^n$ to the equation
\begin{equation}\label{Introduction:Diophantine}
f(x_1, \dots, x_n) = 0.
\end{equation}
There are several different approaches for doing so, arising from three basic
problems concerning Diophantine equations. The first such problem is to
determine whether \strike{or not} \eqref{Introduction:Diophantine} has any
solutions \strike{at all} \aaron{(too colloquial imo)}. Indeed, one of the most
famous theorems in mathematics, Fermat's Last Theorem, proven by Wiles in 1995,
states that for $f(x,y,z) = x^n + y^n - z^n$, where $n \geq 3$, there are no
solutions in the positive integers $x,y,z$ \aaron{(there are so many commas in this sentence. You can remove at least 2 of 7 by splicing and/or rearranging)}. Qualitative questions of this type
are often studied using algebraic methods.

Suppose now that \eqref{Introduction:Diophantine} is solvable, that is, has at least one solution. The second basic problem is to determine whether the number of solutions is finite or infinite. For example, consider the \textit{Thue equation}, 
\begin{equation}\label{Introduction:Thue}
f(x,y) = a,
\end{equation}
where $f(x,y)$ is an integral binary form of degree $n \geq 3$ \aaron{(feels
  like you really jump into the language here. you spelled out what a DE was,
  but now assume the reader knows the definition of an integral binary
  form. Personally, I knew the former but the latter reads like domain-specific
  jargon to me)} and $a$ is a fixed nonzero rational integer. In 1909, Thue
[\textcolor{red}{REF}] proved that this equation has only finitely many
solutions. This result followed from a sharpening of Liouville's inequality, an
observation that algebraic numbers do not admit very strong approximation by
rational numbers. That is, if $\alpha$ is a real algebraic number of degree
$n \geq 2$ and $p,q$ are integers, Liouville's ([\textcolor{red}{REF}])
observation states that
\begin{equation}\label{Introduction:Liouville}
\left|\alpha - \frac{p}{q}\right| > \frac{c_1}{q^{n}},
\end{equation}
where $c_1 >0 $ is a value depending explicitly on $\alpha$. The finitude of the number of solutions to \eqref{Introduction:Thue} follows directly from a sharpening of \eqref{Introduction:Liouville} of the type
\begin{equation}\label{Introduction:Sharpening}
\left|\alpha - \frac{p}{q}\right| > \frac{\lambda(q)}{q^{n}}, \quad \lambda(q) \to \infty.
\end{equation}
\aaron{what is the limit $\lambda \to \infty$ with respect to?}  Indeed, if
$\alpha$ is a real root of $f(x,1)$ and $\alpha^{(i)}$, $i = 1, \dots, n$ are
its conjugates, it follows from \eqref{Introduction:Thue} that
\[\prod_{i=1}^n\left|\alpha^{(i)}-\frac{x}{y}\right| = \frac{a}{|a_0||y|^n}\]
where $a_0$ is the leading coefficient of the polynomial $f(x,1)$. If the Thue equation has integer solutions with arbitrarily large $|y|$, the product $\prod_{i=1}^n|\alpha^{(i)}-x/y|$ must take arbitrarily small values for solutions $x,y$ of \eqref{Introduction:Thue}. As all the $\alpha^{(i)}$ are different, $x/y$ must be correspondingly close to one of the real numbers $\alpha^{(i)}$, say $\alpha$. Thus we obtain
\[\left|\alpha - \frac{x}{y}\right| < \frac{c_2}{|y|^n}\]
where $c_2$ depends only on $a_0$, $n$, and the conjugates $\alpha^{(i)}$. Comparison of this inequality with 
\eqref{Introduction:Sharpening} shows that $|y|$ cannot be arbitrarily large, and so the number of solutions of the Thue equation is finite. Using this argument, an explicit bound can be constructed on the solutions of \eqref{Introduction:Thue} provided that an effective \aaron{(descriptive? explicit? tight? tractable?)} inequality \eqref{Introduction:Sharpening} is known. The sharpening of the Liouville inequality however, especially in effective form, proved to be very difficult. \aaron{REF? also ``very difficult'' seems a subjective qualification; is that okay for your audience?}

In [\textcolor{red}{REF:THUE}], Thue published a proof that 
\[\left|\alpha - \frac{p}{q}\right| < \frac{1}{q^{\frac{n}{2} + 1 + \varepsilon}}\]
has only finitely many solutions in integers $p,q > 0$ for all algebraic
numbers $\alpha$ of degree $n \geq 3$ and any $\varepsilon > 0$. In essence, he
obtained the inequality \eqref{Introduction:Sharpening} with
$\lambda(q) = c_3q^{\frac{1}{2}n - 1 - \varepsilon}$ \aaron{this function does
  not match the one appearing above in displaymath. is that supposed to be the
  case? might have something to do with the $<$ not matching the $>$ in
  \eqref{Introduction:Sharpening}? it is not clear to me, but hopefully it will
  be to typical reader}, where $c_3 > 0$ depends on $\alpha$ and $\varepsilon$,
thereby confirming that all Thue equations have only finitely many
solutions. Unfortunately, Thue's arguments do not allow one to find the
explicit dependence of $c_3$ on $\alpha$ and $\varepsilon$, and so the bound
for the number of solutions of the Thue equation cannot be given in explicit
form either. That is, Thue's proof is ineffective, meaning that it
  provides no means to \strike{actually} find the solutions to \eqref{Introduction:Thue}. \aaron{I feel like I would dance more carefully around calling someone's proof ineffective.}

Nonetheless, the investigation of Thue's equation and its generalizations was central to the development of the theory of Diophantine equations in the early 20th century when it was discovered that many Diophantine equations in two unknowns could be reduced to it. In particular, the thorough development and enrichment of Thue's method led Siegel to his theorem on the finitude of the number of integral points on an algebraic curve of genus greater than zero \aaron{[REF?]}. However, as Siegel's result relies on Thue's rational approximation to algebraic numbers, it too is ineffective \aaron{in the above sense}. 

Shortly following Thue's result, Goormaghtigh conjectured that the only non-trivial integer solutions of the exponential Diophantine equation
\begin{equation} \label{Introduction:Goormaghtigh}
\frac{x^m-1}{x-1} = \frac{y^n-1}{y-1}
\end{equation}
satisfying $x > y > 1$ and $n,m > 2$ are
\[31 = \frac{2^{5}-1}{2-1} = \frac{5^{3}-1}{5-1} \quad \text{ and } \quad 8191 = \frac{2^{13}-1}{2-1}=\frac{90^{3}-1}{90-1}.\]
These correspond to the known solutions $(x,y,m,n)=(2,5,5,3)$ and $(2,90,13,3)$ to what is nowadays termed {\it Goormaghtigh's equation}. The Diophantine equation \eqref{Introduction:Goormaghtigh} asks for integers having all digits equal to one with respect to two distinct bases, yet whether it has finitely many solutions is still unknown. By fixing the exponents $m$ and $n$ however, Davenport, Lewis, and Schinzel ([REF]) were able to prove that \eqref{Introduction:Goormaghtigh} has only finitely many solutions. Unfortunately, this result rests on Siegel's aforementioned finiteness theorem, and is therefore ineffective. 

In 1933, Mahler [\textcolor{red}{REF}] published a paper on the investigation of the Diophantine equation
\[f(x,y) = p_1^{z_1}\cdots p_v^{z_v}, \quad (x,y) = 1,\] in which
$S= \{p_1, \dots, p_v\}$ denote\aaron{s} a fixed set of prime numbers,
$x,y,z_i \geq 0$, $i = 1, \dots, v$ are unknown integers, and $f(x,y)$ is an
integral irreducible binary form of degree $n \geq 3$. Generalizing the
classical result of Thue, Mahler proved that this equation has only finitely
many solutions. Unfortunately, like Thue, Mahler's argument is also
ineffective \aaron{each time I read this, I believe more strongly that a different word should be used to describe their work. ineffective seems like an attack, and a broad stroke that misses the precise critique you're looking to discuss}.

This leads us to the third basic problem regarding Diophantine equations and the main focus of this thesis: given a solvable Diophantine equation, determine all of its solutions. Until long after Thue's work, no method was known for the construction of bounds for the number of solutions of a Thue equation in terms of the parameters of the equation. Only in 1968 was such a method introduced by Baker [REF], based on his theory of bounds for linear forms in the logarithms of algebraic numbers. Generalizing Baker's ground-breaking result to the $p$-adic case, Sprind\u zuk and Vinogradov [CITE] and Coates [CITE] proved that the solutions of any \textit{Thue-Mahler equation},
\begin{equation}\label{Introduction:ThueMahler}
f(x,y) = ap_1^{z_1}\cdots p_v^{z_v}, \quad (x,y) = 1,
\end{equation}
where $a$ is a fixed integer, could, at least in principal, be effectively determined.  The first practical method for solving the general Thue-Mahler equation \eqref{Introduction:ThueMahler} over $\mathbb{Z}$ is attributed to Tzanakis and de Weger [CITE], whose ideas were inspired in part by the method of Agrawal, Coates, Hunt, and van der Poorten [CITE] in their work to solve the specific Thue-Mahler equation
\[x^3 - x^2y + xy^2 + y^3 = \pm 11^{z_1}.\]
Using optimized bounds arising from the theory of linear forms in logarithms, a refined, automated version of this explicit method has since been implemented by Hambrook as a MAGMA package \aaron{[REF?]}. 

As for Goormaghtigh's equation, when $m$ and $n$ are fixed and 
\begin{equation}\label{Introduction:GoormaghtighCondition}
\gcd(m-1,n-1) > 1,
\end{equation}
Davenport, Lewis, and Schinzel ([REF]) were able to replace Siegel's result by an effective argument due to Runge. This result was improved by Nesterenko and Shorey ([REF]) and Bugeaud and Shorey ([REF]) using Baker's theory of linear forms in logarithms. In either case, in order to deduce effectively computable bounds \aaron{(I like this use of effectively)} upon the polynomial variables $x$ and $y$, one must impose the constraints upon $m$ and $n$ that either $m=n+1$, or that the assumption \eqref{Introduction:GoormaghtighCondition} holds. In the extensive literature on this problem, there are a number of striking results that go well beyond what we have mentioned here. By way of example, work of Balasubramanian and Shorey ([REF]) shows that equation \eqref{Introduction:Goormaghtigh} has at most finitely many solutions if we fix only the set of prime divisors of $x$ and $y$, while Bugeaud and Shorey ([REF]) prove an analogous finiteness result, under the additional assumption of \eqref{Introduction:GoormaghtighCondition}, provided the quotient $(m-1)/(n-1)$ is bounded above. Additional results on special cases of equation \eqref{Introduction:Goormaghtigh} are available in, for example, \cite{HeTo}, \cite{Le1}, \cite{Le2} and \cite{Le3}.  An excellent overview of results on this problem  can be found in the survey of Shorey \cite{ShoSur}.

%---------------------------------------------------------------------------------------------------------------------------------------------%

\section{Statement of the results}

The novel contributions of this thesis concern the development and implementation of efficient algorithms to determine all solutions of certain Goormaghtigh equations and Thue-Mahler equations. In particular, we follow [REF: BeGhKr] to prove that, in fact, under assumption \eqref{Introduction:GoormaghtighCondition}, equation \eqref{Introduction:Goormaghtigh} has at most finitely many solutions which may be found effectively, even if we fix only a single exponent. \\

\begin{theorem}[BeGhKr]\label{IntroductionTheorem:Goormaghtigh1}
If there is a solution in integers $x,y, n$ and $m$ to equation \eqref{Introduction:Goormaghtigh}, satisfying \eqref{Introduction:GoormaghtighCondition}, then
\begin{equation} \label{IntroductionTheorem:Goormaghtigh1Eq}
x <  (3d)^{4n/d} \leq  36^n.
\end{equation}
In particular, if $n$ is fixed, there is an effectively computable constant $c=c(n)$ such that
$\max \{ x, y, m \} < c$.
\end{theorem}
We note that the latter conclusion here follows immediately from \eqref{IntroductionTheorem:Goormaghtigh1Eq}, in conjunction with, for example, work of Baker ([REF]). The constants present in our upper bound \eqref{IntroductionTheorem:Goormaghtigh1Eq} may be sharpened somewhat at the cost of increasing the complexity of our argument. By refining our approach, in conjunction with some new results from computational Diophantine approximation, we are able to achieve the complete solution of equation \eqref{Introduction:Goormaghtigh}, subject to condition \eqref{Introduction:GoormaghtighCondition}, for small fixed values of $n$. \\

\begin{theorem}[BeGhKr] \label{IntroductionTheorem:Goormaghtigh2Eq}
If there is a solution in integers $x,y$ and $m$ to equation \eqref{Introduction:Goormaghtigh}, with $n \in \{ 3, 4, 5 \}$ and satisfying \eqref{Introduction:GoormaghtighCondition}, then
\[(x,y,m,n) = (2,5,5,3)  \; \mbox{ and } \; (2,90,13,3).\]
\end{theorem}

In the case $n = 5$ of Theorem \eqref{IntroductionTheorem:Goormaghtigh2Eq} ``off-the-shelf'' techniques for finding integral points on models of elliptic curves or for solving {\it Ramanujan-Nagell} equations of the shape $F(x)=z^n$ (where $F$ is a polynomial and $z$ a fixed integer) do not apparently permit the full resolution of this problem in a reasonable amount of time. Instead, we sharpen the existing techniques of [TdW] and [Hambrook] for solving Thue-Mahler equations and specialize them to this problem. 

A direct consequence and primary motivation for developing an efficient Thue-Mahler algorithm is the computation of elliptic curves over $\mathbb{Q}$. Let $S$ be a finite set of rational primes. In $1963$, Shafarevich [CITE] proved that there are at most finitely many $\mathbb{Q}$-isomorphism classes of elliptic curves defined over $\mathbb{Q}$ having good reduction outside $S$. The first effective proof of this statement was provided by Coates [CITE] in 1970 for the case $K = \mathbb{Q}$ and $S = \{2,3\}$ using bounds for linear forms in $p$-adic and complex logarithms. Early attempts to make these results explicit for fixed sets of small primes overlap with the arguments of [COATES], in that they reduce the problem to that of solving a number of degree $3$ Thue-Mahler equations of the form
\[F(x,y) = au,\]
where $u$ is an integer whose prime factors all lie in $S$. 

In the $1950$'s and $1960$'s, Taniyama and Weil asked whether all elliptic curves over $\mathbb{Q}$ of a given conductor $N$ are related to modular functions. While this conjecture is now known as the Modularity Theorem, until its proof in $2001$ \cite{Breuil}, attempts to verify it sparked a large effort to tabulate all elliptic curves over $\mathbb{Q}$ of given conductor $N$. In 1966, Ogg (\cite{Ogg1}, \cite{Ogg2}) determined all elliptic curves defined over $\mathbb{Q}$ with conductor of the form $2^a$. Coghlan, in his dissertation \cite{Coghlan}, studied the curves of conductor $2^a3^b$ independently of Ogg, while Setzer \cite{Setzer} computed all $\mathbb{Q}$-isomorphism classes of elliptic curves of conductor $p$ for certain small primes $p$. Each of these examples corresponds, via the [BR] approach, to cases with reducible forms. The first analysis on irreducible forms in \eqref{Eq:tmEquation} was carried out by Agrawal, Coates, Hunt and van der Poorten \cite{Agrawal}, who determined all elliptic curves of conductor $11$ defined over $\mathbb{Q}$ to verify the (then) conjecture of Taniyama-Weil.

There are very few, if any, subsequent attempts in the literature to find elliptic curves of given conductor via Thue-Mahler equations. Instead, many of the approaches involve a completely different method to the problem, using modular forms. This method relies upon the Modularity Theorem of Breuil, Conrad, Diamond and Taylor \cite{Breuil}, which was still a conjecture (under various guises) when these ideas were first implemented. Much of the success of this approach can be attributed to Cremona (see e.g. \cite{Cremona1}, \cite{Cremona2}) and his collaborators, who have devoted decades of work to it. In fact, using this method, all elliptic curves over $\mathbb{Q}$ of conductor $N$ have been determined for values of $N$ as follows
\begin{itemize} \itemsep0em
\item Antwerp IV ($1972$): $N \leq 200$
\item Tingley ($1975$): $N \leq 320$
\item Cremona ($1988$): $N \leq 600$
\item Cremona ($1990$): $N \leq 1000$
\item Cremona ($1997$): $N \leq 5077$
\item Cremona ($2001$): $N \leq 10000$
\item Cremona ($2005$): $N \leq 130000$
\item Cremona ($2014$): $N \leq 350000$
\item Cremona ($2015$): $N \leq 364000$
\item Cremona ($2016$): $N \leq 390000$.
\end{itemize}

In this thesis, we follow [BeGhRe] wherein we return to techniques based upon solving Thue-Mahler equations, using a number of results from classical invariant theory. In particular, we illustrate the connection between elliptic curves over $\mathbb{Q}$ and cubic forms and subsequently describe an effective algorithm for determining all elliptic curves over $\mathbb{Q}$ having good reduction outside $S$. This result can be summarized as follows. If we wish to find an elliptic curves $E$ of conductor $N = p_1^{a_1}\cdots p_v^{a_v}$ for some $a_i \in \mathbb{N}$, by Theorem 1 of [BeGhRe], there exists an integral binary cubic form $F$ of discriminant $N_0 \mid 12 N$ and relatively prime integers $u,v$ satisfying
\[F(u,v) = w_0u^3 + w_1u^2v + w_2uv^2 + w_3v^3 = 2^{\alpha_1}3^{\beta_1}\prod_{p|N_0}p^{\kappa_p}\]
for some $\alpha_1, \beta_1, \kappa_p$. Then $E$ is isomorphic over $\mathbb{Q}$ to the elliptic curve $E_{\mathcal{D}}$, where $E_{\mathcal{D}}$ is determined by the form $F$ and $(u,v)$. It is worth noting that Theorem 1 of [BeGhRe] very explicitly describes how to generate $E_{\mathcal{D}}$; once a solution $(u,v)$ to the Thue-Mahler equation $F$ is known, a quick computation of the Hessian and Jacobian discriminant of $F$ evaluated at $(u,v)$ yields the coefficients of $E_{\mathcal{D}}$. Using this theorem, all $E/\mathbb{Q}$ of conductor $N$ may be computed by generating all of the relevant binary cubic forms, solving the corresponding Thue-Mahler equations, and outputting the elliptic curves that arise. The first and last steps of this process are straightforward. Indeed, Bennett and Rechnitzer describe an efficient algorithm for carrying out the first step \aaron{REF}. In fact, they having carried out a one-time computation of all irreducible forms that can arise in Theorem 1 of absolute discriminant bounded by $10^{10}$. The bulk of the work is therefore concentrated in step 2, solving a large number of degree 3 Thue-Mahler equations. 

Unfortunately, despite many refinements, [Hambrook's] MAGMA implementation of a Thue-Mahler solver encounters a multitude of bottlenecks which often yield unavoidable timing and memory problems, even when parallelization is considered. As our aim is to use the results of [BeGhRe] to generate all elliptic curves over $\mathbb{Q}$ of conductor $N < 10^6$, in its current state, the Hambrook algorithm is inefficient for this task, and in many cases, simply unusable due to its memory requirements. The main novel contribution of this thesis is therefore the efficient resolution of an arbitrary degree $3$ Thue-Mahler equation and the implementation of this algorithm as a MAGMA package. This work is based on ideas of Matshke, von Kanel [CITE], and Siksek and is summarized in the following steps.



% OLD INTRO

%A Thue-Mahler equation is a Diophantine equation of the form
%\begin{equation} \label{Eq:tmEquation}
%F(x,y) = ap_1^{z_1}\cdots p_v^{z_v},
%\end{equation}
%where
%\[F(x,y) = f_0x^n + f_1x^{n-1}y + \cdots + f_{n-1}xy^{n-1} + f_ny^n\]
%is an irreducible binary form of degree at least $3$, $a$ is a nonzero integer, and $p_1, \dots, p_v$ are rational primes. The number of solutions in relatively prime integers $x$ and $y$ is known to be finite via the work of Mahler [CITE]. Though this proof is ineffective, it generalizes a classical result of Thue [CITE], who had proved an analogous statement for equations of the form $F(x,y) = a$. In the mid-1960's, Baker [CITE] proved his ground-breaking results on effective lower bounds for linear forms in logarithms of algebraic numbers. 
%
%Generalizing Baker's result to the $p$-adic case, Sprind\u zuk and Vinogradov [CITE] and Coates [CITE] proved that the solutions of any Thue-Mahler equation could, at least in principal, be effectively determined. The first practical method for solving the general Thue-Mahler equation over $\mathbb{Z}$ is attributed to Tzanakis and de Weger [CITE], whose ideas were inspired in part by the method of Agrawal, Coates, Hunt, and van der Poorten [CITE] in their work to solve the specific Thue-Mahler equation
%\[x^3 - x^2y + xy^2 + y^3 = \pm 11^{z_1}.\]

%In 2011, using optimized bounds arising from the theory of linear forms in logarithms, Hambrook [CITE] implemented a refined, automated version of the explicit method of [TdW] as a MAGMA package. Similar to [TdW], this algorithm begins by reducing the problem to one of solving a collection of finitely many $S$-unit equations in a certain algebraic number field $K$. Of course, by an $S$-unit, we mean an integer whose prime factors all lie in $S$. For each such equation, a very large upper bound on the solutions is generated using the theory of linear forms in logarithms. This bound is then reduced via Diophantine approximation techniques. Finally, the algorithm searches below this reduced bound using a combination of clever sieves and brute force. 
%
%Unfortunately, despite Hambrook's refinements, this algorithm encounters a multitude of bottlenecks which often yield unavoidable timing and memory problems, even when parallelization is considered. As we will outline shortly, one of our primary aims in this thesis is to solve a very large number of Thue-Mahler equations. In its current state, the Hambrook algorithm is inefficient for this task, and in many cases, simply unusable due to its memory requirements. The main novel contribution of this thesis is therefore the efficient resolution of an arbitrary Thue-Mahler equation and the implementation of this algorithm as a MAGMA package. This work is based on ideas of Matshke, von Kanel [CITE], and Siksek and is summarized in the following steps.

\textbf{Step 1.}
Following [TdW] and [Hambrook], we reduce the problem of solving the given Thue-Mahler equation to the problem of solving a collection of finitely many $S$-unit equations in a certain algebraic number field $K$. These are equations of the form
\begin{equation}\label{Introduction:SUnit}
\mu_0 y - \lambda_0 x = 1
\end{equation}
for some $\mu_0, \lambda_0 \in K$ and unknowns $x,y$. The collection of forms is such that if we know the solutions of each equation in the collection, then we can easily derive all of the solutions of the Thue-Mahler equation. This reduction is performed in two steps. First, \eqref{Introduction:ThueMahler} is reduced to a finite number of ideal equations over $K$. Here, we employ new results by Siksek [Cite?] to significantly reduce the number of ideal equations to consider. Next, we reduce each ideal equation to a number of certain $S$-unit equations \eqref{Introduction:SUnit} via a finite number of principalization tests. The method of [TdW] reduces \eqref{Introduction:ThueMahler} to $(m/2) h^v$ $S$-unit equations, where $m$ is the number of roots of unity of $K$, $h$ is the class number, and $v$ is the number of rational primes $p_1, \dots, p_v$. The method of Siksek that we employ gives only $m/2$ $S$-unit equations. The principle computational work here consists of computing an integral basis, a system of fundamental units, and a splitting field of $K$, as well as computing the class group of $K$ and the factorizations of the primes $p_1,\dots,p_v$ into prime ideals in the ring of integers of $K$. 

The remaining steps are performed for each of the $S$-unit equations in our collection. 

\textbf{Step 2.}
In place of the logarithmic sieves used in [TdW] to derive a large upper bound, we work with the global logarithmic Weil height
\[h: \mathbb{G}_m(\overline{\mathbb{Q}}) \to \mathbb{R}_{\geq 0}.\]
For a given \eqref{Introduction:SUnit}, we show that the height $h(1/x)$ admits a decomposition into local heights at each place of $K$ appearing in the $S$-unit equation. Using [CITE : Matshke, von Kanel], we generate a very large upper bound on the height $h(1/x)$, and subsequently, on the local heights. This step is a straightforward computation, whereas the analogous step in Hambrook and TdW is a complex and lengthy derivation which involves factoring rational primes into prime ideals in a splitting field of $K$ and computing heights of certain elements of the splitting field. 

\textbf{Step 3.}
For each place of $K$ appearing in \eqref{Introduction:SUnit}, we drastically reduce the upper bounds derived in Step 2 by using computational Diophantine approximation techniques applied to the intersection of a certain ellipsoid and translated lattice. This technique involves using the Finke-Pohst algorithm to enumerate all short vectors in the intersection. Here, working with the Weil height $h(1/x)$ has the advantage that it leads to ellipsoids whose volumes are smaller than the ellipsoids implicitly used in [TdW] by a factor of $\sim r^{r/2}$ for $r$ the number of places of $K$ appearing in our $S$-unit equation. In this way, we reduce the number of short vectors appearing from the Fincke-Pohst algorithm, and consequently reduce our running time and memory requirements. 

\textbf{Step 4.}
Samir's sieve - this may not be done in time as we only just received Samir's writeup and explanation as pertaining to Thue-Mahler equations.

\textbf{Step 5.}
Finally, we use a sieving procedure to find all the solutions of the Diophantine equation that live in the box defined by the bounds derived in the previous steps. To carry out this step, we run through all the possible solutions in the box and �sieve� out the vast majority of non-solutions. This is done via certain low-cost congruence tests. The candidate solutions passing this test are then verified directly against \eqref{Introduction:SUnit}. Though we expect the bounds defining the box to be small, there can still be a very large number of possible solutions to check, especially if the number of rational primes involved in the Thue-Mahler equation is large. The computations performed on each individual candidate solution are relatively simple, but the sheer number of candidates often makes this step the computational bottleneck of the entire algorithm. 

\textbf{Step 6.}
Having performed Steps 2-5 for each $S$-unit equation in our collection, we now have all the solutions of each such equation, and we use this knowledge to determine all the solutions of the Thue-Mahler equation. 

The reader will notice several parallels between this refined algorithm and the aforementioned Goormaghtigh equation solver in the case $n=5$. In particular, both algorithms share the same setup and refinements of the [TdW] and [Hambrook] solver. For \eqref{Introduction:Goormaghtigh}, however, we are left to solve
\[f(y) = x^m,\]
a Thue-Mahler-like equation of degree $4$ in explicit values of $x$ and unknown integers $y$ and $m$. In this case, we are permitted simplifications which allow us to omit the Fincke-Pohst algorithm and final congruence sieves. Instead, for each $x$, we rely on only a few iterations of the LLL algorithm to reduce our initial bound on the exponents before entering a naive search to complete our computation. Of course, this algorithm can be refined further for efficiency, however, in the context of [BeGhKr], such improvements are not needed. 

The outline of this thesis is as follows. ADD


%--------------------------------------------------------------------------------------------------------------------------------------------%
%--------------------------------------------------------------------------------------------------------------------------------------------%

\endinput

Any text after an \endinput is ignored.
You could put scraps here or things in progress.



This document provides a quick set of instructions for using the
\class{ubcdiss} class to write a dissertation in \LaTeX. 
Unfortunately this document cannot provide an introduction to using
\LaTeX.  The classic reference for learning \LaTeX\ is
\citeauthor{lamport-1994-ladps}'s
book~\cite{lamport-1994-ladps}.  There are also many freely-available
tutorials online;
\webref{http://www.andy-roberts.net/misc/latex/}{Andy Roberts' online
    \LaTeX\ tutorials}
seems to be excellent.
The source code for this docment, however, is intended to serve as
an example for creating a \LaTeX\ version of your dissertation.

We start by discussing organizational issues, such as splitting
your dissertation into multiple files, in
\autoref{sec:SuggestedThesisOrganization}.
We then cover the ease of managing cross-references in \LaTeX\ in
\autoref{sec:CrossReferences}.
We cover managing and using bibliographies with \BibTeX\ in
\autoref{sec:BibTeX}. 
We briefly describe typesetting attractive tables in
\autoref{sec:TypesettingTables}.
We briefly describe including external figures in
\autoref{sec:Graphics}, and using special characters and symbols
in \autoref{sec:SpecialSymbols}.
As it is often useful to track different versions of your dissertation,
we discuss revision control further in
\autoref{sec:DissertationRevisionControl}. 
We conclude with pointers to additional sources of information in
\autoref{sec:Conclusions}.

%--------------------------------------------------------------------------------------------------------------------------------------------%
%--------------------------------------------------------------------------------------------------------------------------------------------%
\section{Suggested Thesis Organization}
\label{sec:SuggestedThesisOrganization}

The \acs{UBC} \acf{GPS} specifies a particular arrangement of the
components forming a thesis.\footnote{See
    \url{http://www.grad.ubc.ca/current-students/dissertation-thesis-preparation/order-components}}
This template reflects that arrangement.

In terms of writing your thesis, the recommended best practice for
organizing large documents in \LaTeX\ is to place each chapter in
a separate file.  These chapters are then included from the main
file through the use of \verb+\include{file}+.  A thesis might
be described as six files such as \file{intro.tex},
\file{relwork.tex}, \file{model.tex}, \file{eval.tex},
\file{discuss.tex}, and \file{concl.tex}.

We also encourage you to use macros for separating how something
will be typeset (\eg bold, or italics) from the meaning of that
something. 
For example, if you look at \file{intro.tex}, you will see repeated
uses of a macro \verb+\file{}+ to indicate file names.
The \verb+\file{}+ macro is defined in the file \file{macros.tex}.
The consistent use of \verb+\file{}+ throughout the text not only
indicates that the argument to the macro represents a file (providing
meaning or semantics), but also allows easily changing how
file names are typeset simply by changing the definition of the
\verb+\file{}+ macro.
\file{macros.tex} contains other useful macros for properly typesetting
things like the proper uses of the latinate \emph{exempli grati\={a}}
and \emph{id est} (\ie \verb+\eg+ and \verb+\ie+), 
web references with a footnoted \acs{URL} (\verb+\webref{url}{text}+),
as well as definitions specific to this documentation
(\verb+\latexpackage{}+).

%--------------------------------------------------------------------------------------------------------------------------------------------%
\section{Making Cross-References}
\label{sec:CrossReferences}

\LaTeX\ make managing cross-references easy, and the \latexpackage{hyperref}
package's\ \verb+\autoref{}+ command\footnote{%
    The \latexpackage{hyperref} package is included by default in this
    template.}
makes it easier still. 

A thing to be cross-referenced, such as a section, figure, or equation,
is \emph{labelled} using a unique, user-provided identifier, defined
using the \verb+\label{}+ command.  
The thing is referenced elsewhere using the \verb+\autoref{}+ command.
For example, this section was defined using:
\begin{lstlisting}
    \section{Making Cross-References}
    \label{sec:CrossReferences}
\end{lstlisting}
References to this section are made as follows:
\begin{lstlisting}
    We then cover the ease of managing cross-references in \LaTeX\
    in \autoref{sec:CrossReferences}.
\end{lstlisting}
\verb+\autoref{}+ takes care of determining the \emph{type} of the 
thing being referenced, so the example above is rendered as
\begin{quote}
    We then cover the ease of managing cross-references in \LaTeX\
    in \autoref{sec:CrossReferences}.
\end{quote}

The label is any simple sequence of characters, numbers, digits,
and some punctuation marks such as ``:'' and ``--''; there should
be no spaces.  Try to use a consistent key format: this simplifies
remembering how to make references.  This document uses a prefix
to indicate the type of the thing being referenced, such as \texttt{sec}
for sections, \texttt{fig} for figures, \texttt{tbl} for tables,
and \texttt{eqn} for equations.

For details on defining the text used to describe the type
of \emph{thing}, search \file{diss.tex} and the \latexpackage{hyperref}
documentation for \texttt{autorefname}.


%--------------------------------------------------------------------------------------------------------------------------------------------%
\section{Managing Bibliographies with \BibTeX}
\label{sec:BibTeX}

One of the primary benefits of using \LaTeX\ is its companion program,
\BibTeX, for managing bibliographies and citations.  Managing
bibliographies has three parts: (i) describing references,
(ii)~citing references, and (iii)~formatting cited references.

\subsection{Describing References}

\BibTeX\ defines a standard format for recording details about a
reference.  These references are recorded in a file with a
\file{.bib} extension.  \BibTeX\ supports a broad range of
references, such as books, articles, items in a conference proceedings,
chapters, technical reports, manuals, dissertations, and unpublished
manuscripts. 
A reference may include attributes such as the authors,
the title, the page numbers, the \ac{DOI}, or a \ac{URL}.  A reference
can also be augmented with personal attributes, such as a rating,
notes, or keywords.

Each reference must be described by a unique \emph{key}.\footnote{%
    Note that the citation keys are different from the reference
    identifiers as described in \autoref{sec:CrossReferences}.}
A key is a simple sequence of characters, numbers, digits, and some
punctuation marks such as ``:'' and ``--''; there should be no spaces. 
A consistent key format simiplifies remembering how to make references. 
For example:
\begin{quote}
   \fbox{\emph{last-name}}\texttt{-}\fbox{\emph{year}}\texttt{-}\fbox{\emph{contracted-title}}
\end{quote}
where \emph{last-name} represents the last name for the first author,
and \emph{contracted-title} is some meaningful contraction of the
title.  Then \citeauthor{kiczales-1997-aop}'s seminal article on
aspect-oriented programming~\cite{kiczales-1997-aop} (published in
\citeyear{kiczales-1997-aop}) might be given the key
\texttt{kiczales-1997-aop}.

An example of a \BibTeX\ \file{.bib} file is included as
\file{biblio.bib}.  A description of the format a \file{.bib}
file is beyond the scope of this document.  We instead encourage
you to use one of the several reference managers that support the
\BibTeX\ format such as
\webref{http://jabref.sourceforge.net}{JabRef} (multiple platforms) or
\webref{http://bibdesk.sourceforge.net}{BibDesk} (MacOS\,X only). 
These front ends are similar to reference manages such as
EndNote or RefWorks.


\subsection{Citing References}

Having described some references, we then need to cite them.  We
do this using a form of the \verb+\cite+ command.  For example:
\begin{lstlisting}
    \citet{kiczales-1997-aop} present examples of crosscutting 
    from programs written in several languages.
\end{lstlisting}
When processed, the \verb+\citet+ will cause the paper's authors
and a standardized reference to the paper to be inserted in the
document, and will also include a formatted citation for the paper
in the bibliography.  For example:
\begin{quote}
    \citet{kiczales-1997-aop} present examples of crosscutting 
    from programs written in several languages.
\end{quote}
There are several forms of the \verb+\cite+ command (provided
by the \latexpackage{natbib} package), as demonstrated in
\autoref{tbl:natbib:cite}.
Note that the form of the citation (numeric or author-year) depends
on the bibliography style (described in the next section).
The \verb+\citet+ variant is used when the author names form
an object in the sentence, whereas the \verb+\citep+ variant
is used for parenthetic references, more like an end-note.
Use \verb+\nocite+ to include a citation in the bibliography
but without an actual reference.
\nocite{rowling-1997-hpps}
\begin{table}
\caption{Available \texttt{cite} variants; the exact citation style
    depends on whether the bibliography style is numeric or author-year.}
\label{tbl:natbib:cite}
\centering
\begin{tabular}{lp{3.25in}}\toprule
Variant & Result \\
\midrule
% We cheat here to simulate the cite/citep/citet for APA-like styles
\verb+\cite+ & Parenthetical citation (\eg ``\cite{kiczales-1997-aop}''
    or ``(\citeauthor{kiczales-1997-aop} \citeyear{kiczales-1997-aop})'') \\
\verb+\citet+ & Textual citation: includes author (\eg
    ``\citet{kiczales-1997-aop}'' or
    or ``\citeauthor{kiczales-1997-aop} (\citeyear{kiczales-1997-aop})'') \\
\verb+\citet*+ & Textual citation with unabbreviated author list \\
\verb+\citealt+ & Like \verb+\citet+ but without parentheses \\
\verb+\citep+ & Parenthetical citation (\eg ``\cite{kiczales-1997-aop}''
    or ``(\citeauthor{kiczales-1997-aop} \citeyear{kiczales-1997-aop})'') \\
\verb+\citep*+ & Parenthetical citation with unabbreviated author list \\
\verb+\citealp+ & Like \verb+\citep+ but without parentheses \\
\verb+\citeauthor+ & Author only (\eg ``\citeauthor{kiczales-1997-aop}'') \\
\verb+\citeauthor*+ & Unabbreviated authors list 
    (\eg ``\citeauthor*{kiczales-1997-aop}'') \\
\verb+\citeyear+ & Year of citation (\eg ``\citeyear{kiczales-1997-aop}'') \\
\bottomrule
\end{tabular}
\end{table}

\subsection{Formatting Cited References}

\BibTeX\ separates the citing of a reference from how the cited
reference is formatted for a bibliography, specified with the
\verb+\bibliographystyle+ command. 
There are many varieties, such as \texttt{plainnat}, \texttt{abbrvnat},
\texttt{unsrtnat}, and \texttt{vancouver}.
This document was formatted with \texttt{abbrvnat}.
Look through your \TeX\ distribution for \file{.bst} files. 
Note that use of some \file{.bst} files do not emit all the information
necessary to properly use \verb+\citet{}+, \verb+\citep{}+,
\verb+\citeyear{}+, and \verb+\citeauthor{}+.

There are also packages available to place citations on a per-chapter
basis (\latexpackage{bibunits}), as footnotes (\latexpackage{footbib}),
and inline (\latexpackage{bibentry}).
Those who wish to exert maximum control over their bibliography
style should see the amazing \latexpackage{custom-bib} package.

%--------------------------------------------------------------------------------------------------------------------------------------------%
\section{Typesetting Tables}
\label{sec:TypesettingTables}

\citet{lamport-1994-ladps} made one grievous mistake
in \LaTeX: his suggested manner for typesetting tables produces
typographic abominations.  These suggestions have unfortunately
been replicated in most \LaTeX\ tutorials.  These
abominations are easily avoided simply by ignoring his examples
illustrating the use of horizontal and vertical rules (specifically
the use of \verb+\hline+ and \verb+|+) and using the
\latexpackage{booktabs} package instead.

The \latexpackage{booktabs} package helps produce tables in the form
used by most professionally-edited journals through the use of
three new types of dividing lines, or \emph{rules}.
% There are times that you don't want to use \autoref{}
Tables~\ref{tbl:natbib:cite} and~\ref{tbl:LaTeX:Symbols} are two
examples of tables typeset with the \latexpackage{booktabs} package.
The \latexpackage{booktabs} package provides three new commands
for producing rules:
\verb+\toprule+ for the rule to appear at the top of the table,
\verb+\midrule+ for the middle rule following the table header,
and \verb+\bottomrule+ for the bottom-most at the end of the table.
These rules differ by their weight (thickness) and the spacing before
and after.
A table is typeset in the following manner:
\begin{lstlisting}
    \begin{table}
    \caption{The caption for the table}
    \label{tbl:label}
    \centering
    \begin{tabular}{cc}
    \toprule
    Header & Elements \\
    \midrule
    Row 1 & Row 1 \\
    Row 2 & Row 2 \\
    % ... and on and on ...
    Row N & Row N \\
    \bottomrule
    \end{tabular}
    \end{table}
\end{lstlisting}
See the \latexpackage{booktabs} documentation for advice in dealing with
special cases, such as subheading rules, introducing extra space
for divisions, and interior rules.

%--------------------------------------------------------------------------------------------------------------------------------------------%
\section{Figures, Graphics, and Special Characters}
\label{sec:Graphics}

Most \LaTeX\ beginners find figures to be one of the more challenging
topics.  In \LaTeX, a figure is a \emph{floating element}, to be
placed where it best fits.
The user is not expected to concern him/herself with the placement
of the figure.  The figure should instead be labelled, and where
the figure is used, the text should use \verb+\autoref+ to reference
the figure's label.
\autoref{fig:latex-affirmation} is an example of a figure.
\begin{figure}
    \centering
    % For the sake of this example, we'll just use text
    %\includegraphics[width=3in]{file}
    \Huge{\textsf{\LaTeX\ Rocks!}}
    \caption{Proof of \LaTeX's amazing abilities}
    \label{fig:latex-affirmation}   % label should change
\end{figure}
A figure is generally included as follows:
\begin{lstlisting}
    \begin{figure}
    \centering
    \includegraphics[width=3in]{file}
    \caption{A useful caption}
    \label{fig:fig-label}   % label should change
    \end{figure}
\end{lstlisting}
There are three items of note:
\begin{enumerate}
\item External files are included using the \verb+\includegraphics+
    command.  This command is defined by the \latexpackage{graphicx} package
    and can often natively import graphics from a variety of formats.
    The set of formats supported depends on your \TeX\ command processor.
    Both \texttt{pdflatex} and \texttt{xelatex}, for example, can
    import \textsc{gif}, \textsc{jpg}, and \textsc{pdf}.  The plain
    version of \texttt{latex} only supports \textsc{eps} files.

\item The \verb+\caption+ provides a caption to the figure. 
    This caption is normally listed in the List of Figures; you
    can provide an alternative caption for the LoF by providing
    an optional argument to the \verb+\caption+ like so:
    \begin{lstlisting}
    \caption[nice shortened caption for LoF]{%
	longer detailed caption used for the figure}
    \end{lstlisting}
    \ac{GPS} generally prefers shortened single-line captions
    in the LoF: multiple-line captions are a bit unwieldy.

\item The \verb+\label+ command provides for associating a unique, user-defined,
    and descriptive identifier to the figure.  The figure can be
    can be referenced elsewhere in the text with this identifier
    as described in \autoref{sec:CrossReferences}.
\end{enumerate}
See Keith Reckdahl’s excellent guide for more details,
\webref{http://www.ctan.org/tex-archive/info/epslatex.pdf}{\emph{Using
imported graphics in LaTeX2e}}.

\section{Special Characters and Symbols}
\label{sec:SpecialSymbols}

\LaTeX\ appropriates many common symbols for its own purposes,
with some used for commands (\ie \verb+\{}&%+) and
mathematics (\ie \verb+$^_+), and others are automagically transformed
into typographically-preferred forms (\ie \verb+-`'+) or to
completely different forms (\ie \verb+<>+).
\autoref{tbl:LaTeX:Symbols} presents a list of common symbols and
their corresponding \LaTeX\ commands.  A much more comprehensive list 
of symbols and accented characters is available at:
\url{http://www.ctan.org/tex-archive/info/symbols/comprehensive/}
\begin{table}
\caption{Useful \LaTeX\ symbols}\label{tbl:LaTeX:Symbols}
\centering\begin{tabular}{ccp{0.5cm}cc}\toprule
\LaTeX & Result && \LaTeX & Result \\
\midrule
    \verb+\texttrademark+ & \texttrademark && \verb+\&+ & \& \\
    \verb+\textcopyright+ & \textcopyright && \verb+\{ \}+ & \{ \} \\
    \verb+\textregistered+ & \textregistered && \verb+\%+ & \% \\
    \verb+\textsection+ & \textsection && \verb+\verb!~!+ & \verb!~! \\
    \verb+\textdagger+ & \textdagger && \verb+\$+ & \$ \\
    \verb+\textdaggerdbl+ & \textdaggerdbl && \verb+\^{}+ & \^{} \\
    \verb+\textless+ & \textless && \verb+\_+ & \_ \\
    \verb+\textgreater+ & \textgreater && \\
\bottomrule
\end{tabular}
\end{table}

%--------------------------------------------------------------------------------------------------------------------------------------------%\section{Changing Page Widths and Heights}

The \class{ubcdiss} class is based on the standard \LaTeX\ \class{book}
class~\cite{lamport-1994-ladps} that selects a line-width to carry
approximately 66~characters per line.  This character density is
claimed to have a pleasing appearance and also supports more rapid
reading~\cite{bringhurst-2002-teots}.  I would recommend that you
not change the line-widths!

\subsection{The \texttt{geometry} Package}

Some students are unfortunately saddled with misguided supervisors
or committee members whom believe that documents should have the
narrowest margins possible.  The \latexpackage{geometry} package is
helpful in such cases.  Using this package is as simple as:
\begin{lstlisting}
    \usepackage[margin=1.25in,top=1.25in,bottom=1.25in]{geometry}
\end{lstlisting}
You should check the package's documentation for more complex uses.

\subsection{Changing Page Layout Values By Hand}

There are some miserable students with requirements for page layouts
that vary throughout the document.  Unfortunately the
\latexpackage{geometry} can only be specified once, in the document's
preamble.  Such miserable students must set \LaTeX's layout parameters
by hand:
\begin{lstlisting}
    \setlength{\topmargin}{-.75in}
    \setlength{\headsep}{0.25in}
    \setlength{\headheight}{15pt}
    \setlength{\textheight}{9in}
    \setlength{\footskip}{0.25in}
    \setlength{\footheight}{15pt}

    % The *sidemargin values are relative to 1in; so the following
    % results in a 0.75 inch margin
    \setlength{\oddsidemargin}{-0.25in}
    \setlength{\evensidemargin}{-0.25in}
    \setlength{\textwidth}{7in}       % 1.1in margins (8.5-2*0.75)
\end{lstlisting}
These settings necessarily require assuming a particular page height
and width; in the above, the setting for \verb+\textwidth+ assumes
a \textsc{US} Letter with an 8.5'' width.
The \latexpackage{geometry} package simply uses the page height and
other specified values to derive the other layout values.
The
\href{http://tug.ctan.org/tex-archive/macros/latex/required/tools/layout.pdf}{\texttt{layout}}
package provides a
handy \verb+\layout+ command to show the current page layout
parameters. 


\subsection{Making Temporary Changes to Page Layout}

There are occasions where it becomes necessary to make temporary
changes to the page width, such as to accomodate a larger formula. 
The \latexmiscpackage{chngpage} package provides an \env{adjustwidth}
environment that does just this.  For example:
\begin{lstlisting}
    % Expand left and right margins by 0.75in
    \begin{adjustwidth}{-0.75in}{-0.75in}
    % Must adjust the perceived column width for LaTeX to get with it.
    \addtolength{\columnwidth}{1.5in}
    \[ an extra long math formula \]
    \end{adjustwidth}
\end{lstlisting}


%--------------------------------------------------------------------------------------------------------------------------------------------%
\section{Keeping Track of Versions with Revision Control}
\label{sec:DissertationRevisionControl}

Software engineers have used \acf{RCS} to track changes to their
software systems for decades.  These systems record the changes to
the source code along with context as to why the change was required.
These systems also support examining and reverting to particular
revisions from their system's past.

An \ac{RCS} can be used to keep track of changes to things other
than source code, such as your dissertation.  For example, it can
be useful to know exactly which revision of your dissertation was
sent to a particular committee member.  Or to recover an accidentally
deleted file, or a badly modified image.  With a revision control
system, you can tag or annotate the revision of your dissertation
that was sent to your committee, or when you incorporated changes
from your supervisor.

Unfortunately current revision control packages are not yet targetted
to non-developers.  But the Subversion project's
\webref{http://tortoisesvn.net/docs/release/TortoiseSVN_en/}{TortoiseSVN}
has greatly simplified using the Subversion revision control system
for Windows users.  You should consult your local geek.

A simpler alternative strategy is to create a GoogleMail account
and periodically mail yourself zipped copies of your dissertation.

%--------------------------------------------------------------------------------------------------------------------------------------------%
\section{Recommended Packages}

The real strength to \LaTeX\ is found in the myriad of free add-on
packages available for handling special formatting requirements.
In this section we list some helpful packages.

\subsection{Typesetting}

\begin{description}
\item[\latexpackage{enumitem}:]
    Supports pausing and resuming enumerate environments.

\item[\latexpackage{ulem}:]
    Provides two new commands for striking out and crossing out text
    (\verb+\sout{text}+ and \verb+\xout{text}+ respectively)
    The package should likely
    be used as follows:
    \begin{verbatim}
    \usepackage[normalem,normalbf]{ulem}
    \end{verbatim}
    to prevent the package from redefining the emphasis and bold fonts.

\item[\latexpackage{chngpage}:]
    Support changing the page widths on demand.

\item[\latexpackage{mhchem}:] 
    Support for typesetting chemical formulae and reaction equations.

\end{description}

Although not a package, the
\webref{http://www.ctan.org/tex-archive/support/latexdiff/}{\texttt{latexdiff}}
command is very useful for creating changebar'd versions of your
dissertation.


\subsection{Figures, Tables, and Document Extracts}

\begin{description}
\item[\latexpackage{pdfpages}:]
    Insert pages from other PDF files.  Allows referencing the extracted
    pages in the list of figures, adding labels to reference the page
    from elsewhere, and add borders to the pages.

\item[\latexpackage{subfig}:]
    Provides for including subfigures within a figure, and includes
    being able to separately reference the subfigures.  This is a
    replacement for the older \texttt{subfigure} environment.

\item[\latexpackage{rotating}:]
    Provides two environments, sidewaystable and sidewaysfigure,
    for typesetting tables and figures in landscape mode.  

\item[\latexpackage{longtable}:]
    Support for long tables that span multiple pages.

\item[\latexpackage{tabularx}:]
    Provides an enhanced tabular environment with auto-sizing columns.

\item[\latexpackage{ragged2e}:]
    Provides several new commands for setting ragged text (\eg forms
    of centered or flushed text) that can be used in tabular
    environments and that support hyphenation.

\end{description}


\subsection{Bibliography Related Packages}

\begin{description}
\item[\latexpackage{bibunits}:]
    Support having per-chapter bibliographies.

\item[\latexpackage{footbib}:]
    Cause cited works to be rendered using footnotes.

\item[\latexpackage{bibentry}:] 
    Support placing the details of a cited work in-line.

\item[\latexpackage{custom-bib}:]
    Generate a custom style for your bibliography.

\end{description}


%--------------------------------------------------------------------------------------------------------------------------------------------%
\section{Moving On}
\label{sec:Conclusions}

At this point, you should be ready to go.  Other handy web resources:
\begin{itemize}
\item \webref{http://www.ctan.org}{\ac{CTAN}} is \emph{the} comprehensive
    archive site for all things related to \TeX\ and \LaTeX. 
    Should you have some particular requirement, somebody else is
    almost certainly to have had the same requirement before you,
    and the solution will be found on \ac{CTAN}.  The links to
    various packages in this document are all to \ac{CTAN}.

\item An online
    \webref{http://www.ctan.org/get/info/latex2e-help-texinfo/latex2e.html}{%
	reference to \LaTeX\ commands} provides a handy quick-reference
    to the standard \LaTeX\ commands.

\item The list of 
    \webref{http://www.tex.ac.uk/cgi-bin/texfaq2html?label=interruptlist}{%
	Frequently Asked Questions about \TeX\ and \LaTeX}
    can save you a huge amount of time in finding solutions to
    common problems.

\item The \webref{http://www.tug.org/tetex/tetex-texmfdist/doc/}{te\TeX\
    documentation guide} features a very handy list of the most useful
    packages for \LaTeX\ as found in \ac{CTAN}.

\item The
\webref{http://www.ctan.org/tex-archive/macros/latex/required/graphics/grfguide.pdf}{\texttt{color}}
    package, part of the graphics bundle, provides handy commands
    for changing text and background colours.  Simply changing
    text to various levels of grey can have a very 
    \textcolor{greytext}{dramatic effect}.


\item If you're really keen, you might want to join the
    \webref{http://www.tug.org}{\TeX\ Users Group}.

\end{itemize}




%    2. Main body
% Generally recommended to put each chapter into a separate file
%\include{relatedwork}
%\include{model}
%\include{impl}
%\include{discussion}
%\include{conclusions}

%% The following is a directive for TeXShop to indicate the main file
%%!TEX root = diss.tex

\chapter{Preliminaries}
\label{ch:Preliminaries}

%--------------------------------------------------------------------------------------------------------------------------------------------%
%--------------------------------------------------------------------------------------------------------------------------------------------%

\section{Algebraic number theory} 
\label{sec:AlgebraicNumberTheory}

In this section we recall some basic results from algebraic number theory that we use throughout the remaining chapters. We refer to \cite{Mar} and \cite{Neuk} for full details. 

Let $K$ be a finite algebraic extension of $\mathbb{Q}$ of degree $n = [K:\mathbb{Q}]$. There are $n$ embeddings $\sigma: K \to \mathbb{C}$. These embeddings can be described by writing $K = \mathbb{Q}(\theta)$ for some $\theta \in \mathbb{C}$ and observing that $\theta$ can be sent to any one of its conjugates. 
Let $s$ denote the number of real embeddings of $K$ and let $t$ denote the number of conjugate pairs of complex embeddings of $K$, where $n = s + 2t$. By Dirichlet's Unit Theorem, the group of units of $K$ is the direct product of a finite cyclic group consisting of the roots of unity in $K$ and a free abelian group of rank $r = s + t -1$. Equivalently, there exists a system of $r$ independent units $\eps_1, \dots, \eps_r$ such that the group of units of $K$ is given by 
\[\left\{\zeta \cdot \eps_1^{a_1}\cdots \eps_r^{a_r} \ : \ \zeta \text{ a root of unity}, a_i \in \mathbb{Z} \text{ for } i = 1, \dots, r\right\}.\]
Any set of independent units that generate the torsion-free part of the unit group is called a system of \textit{fundamental units}. 

An element $\alpha \in K$ is called an \textit{algebraic integer} if its minimal polynomial over $\mathbb{Z}$ is monic. The set of algebraic integers in $K$ forms a ring, denoted $\mathcal{O}_K$. We refer to this ring as the \textit{ring of integers} or \textit{number ring} corresponding to the number field $K$. For any $\alpha \in K$, we define the \textit{norm} of $\alpha$ as 
\[N_{K/\mathbb{Q}}(\alpha) = \prod_{\sigma:K \to \mathbb{C}} \sigma(\alpha)\]
where the product is taken over all embeddings $\sigma$ of $K$. For algebraic integers, $N_{K/\mathbb{Q}}(\alpha) \in \mathbb{Z}$. The units are precisely the elements of norm $\pm 1$. Two elements $\alpha, \beta$ of $K$ are called \textit{associates} if there exists a unit $\eps$ such that $\alpha = \eps \beta$. Let $(\alpha)\mathcal{O}_K$ denote the ideal generated by $\alpha$. Associated elements generate the same ideal, and distinct generators of an ideal are associated. There exist only finitely many non-associated algebraic integers in $K$ with given norm. 

Any element of the ring of integers can be written as a product of \textit{irreducible} elements. These are non-zero non-unit elements of $\mathcal{O}_K$ which have no integral divisors but their own associates. Unfortunately, number rings are not alway unique factorization domains: this decomposition into irreducible elements may not be unique. However, every number ring is a Dedekind domain. This means that every ideal can be decomposed into a product of prime ideals and this decomposition is unique. A \textit{principal} ideal is an ideal generated by a single element $\alpha$. Two fractional ideals are called equivalent if their quotient is principal. It is well known that there are only finitely many equivalence classes of fractional ideals and the set of all such classes forms a finite abelian group called the \textit{ideal class group}, $\Cl(K)$. The number of ideal classes, $\#\Cl(K)$, is called the \textit{class number} of $\mathcal{O}_K$ and is denoted by $h_K$. For an ideal $\mathfrak{a}$, it is always true that $\mathfrak{a}^{h_K}$ is principal. The norm of the (integral) ideal $\mathfrak{a}$ is defined by $N_{K/\mathbb{Q}}(\mathfrak{a}) = \#\left(\mathcal{O}_K/\mathfrak{a}\right)$. If $\mathfrak{a} = (\alpha) \mathcal{O}_K$ is a principal ideal, then $N_{K/\mathbb{Q}}(\mathfrak{a}) = \left|N_{K/\mathbb{Q}}(\alpha)\right|$. 

Let $L$ be a finite field extension of $K$ with ring of integers $\mathcal{O}_L$. Every prime ideal $\mathfrak{P}$ of $\mathcal{O}_L$ \textit{lies over} a unique prime ideal $\mathfrak{p}$ in $\mathcal{O}_K$. That is, $\mathfrak{P}$ divides $\mathfrak{p}$. The \textit{ramification index} $e({\mathfrak{P}}|\mathfrak{p})$ is the largest power to which $\mathfrak{P}$ divides $\mathfrak{p}$. The field $\mathcal{O}_L/\mathfrak{P}$ is an extension of finite degree $f(\mathfrak{P}|\mathfrak{p})$ over $\mathcal{O}_K/\mathfrak{p}$. We call $f(\mathfrak{P}|\mathfrak{p})$ the \textit{inertial degree} of $\mathfrak{P}$ over $\mathfrak{p}$. For $\mathfrak{p}$ lying over the rational prime $p$, this is the integer such that 
\[N_{K/\mathbb{Q}}(\mathfrak{p}) = p^{f(\mathfrak{p}|p)}.\]
The ramification index and inertial degree are multiplicative in a tower of fields. In particular, if $\mathfrak{P}$ lies over $\mathfrak{p}$ which lies over the rational prime $p$, then
\[e({\mathfrak{P}}|p) = e({\mathfrak{P}}|\mathfrak{p})e({\mathfrak{p}}|p) \quad \text{ and } \quad f({\mathfrak{P}}|p) = f({\mathfrak{P}}|\mathfrak{p})f({\mathfrak{p}}|p).\]
Let $\mathfrak{P}_1, \dots, \mathfrak{P}_m$ be the primes of $\mathcal{O}_L$ lying over a prime ideal $\mathfrak{p}$ of $\mathcal{O}_K$. Denote by $e({\mathfrak{P}}_1|\mathfrak{p}),\dots, e({\mathfrak{P}}_m|\mathfrak{p})$ and $f({\mathfrak{P}}_1|\mathfrak{p}), \dots, f({\mathfrak{P}}_m|\mathfrak{p})$ the corresponding ramification indices and inertial degrees. Then
\[\sum_{i=1}^m e({\mathfrak{P}}_i|\mathfrak{p})f({\mathfrak{P}}_i|\mathfrak{p}) = [L:K].\]

If $L$ is normal over $K$ and $\mathfrak{P}_i$ and $\mathfrak{P}_j$ are two prime ideals lying over $\mathfrak{p}$, then $e({\mathfrak{P}}_i|\mathfrak{p}) = e({\mathfrak{P}}_j|\mathfrak{p})$ and $f({\mathfrak{P}}_i|\mathfrak{p}) = f({\mathfrak{P}}_j|\mathfrak{p})$. In this case, $\mathfrak{p}$ factors as
\[\mathfrak{p}\mathcal{O}_L = \left( \mathfrak{P}_1 \cdots \mathfrak{P}_m\right)^e\]
in $\mathcal{O}_L$, where the $\mathfrak{P}_i$ are distinct prime ideals all having the same ramification degree $e$ and inertial degree $f$ over $\mathfrak{p}$. It follows that 
\[mef = [L:K].\]

%--------------------------------------------------------------------------------------------------------------------------------------------%
%--------------------------------------------------------------------------------------------------------------------------------------------%

\section{$p$-adic valuations}
\label{sec:pAdicValuations}

In this section we give a concise exposition of $p$-adic valuations. As references for this material we give \cite{BS} (especially Theorem 3 in Chapter 4, Section 2), \cite{Ca} (especially Lemma 2.1 in Chapter 9), \cite{Has2} (especially Chapter 18), \cite{Ko} (especially Chapter 3, Section 2), and \cite{Nark} (especially Theorem 6.1).

We denote the algebraic closure of $\mathbb{Q}_p$ by $\overline{\mathbb{Q}}_p$. The completion of $\overline{\mathbb{Q}}_p$ with respect to the absolute value of $\overline{\mathbb{Q}}_p$ is denoted by $\mathbb{C}_p$.

Let $K$ be an arbitrary number field. A homomorphism $v: K^* \to \mathbb{R}_{\geq 0}$ of the multiplicative group of $K$ into the group of positive real numbers is called a \textit{valuation} if it satisfies the condition
\[v(x+y) \leq v(x) + v(y).\]
This definition may be extended to all of $K$ by setting $v(0) = 0$. If
\[v(x+y) \leq \max(v(x),v(y))\]
holds for all $x,y \in K$, then $v$ is called a \textit{non-Archimedean valuation}. All remaining valuations on $K$ are called \textit{Archimedean}. 

Every valuation $v$ induces on $K$ the structure of a metric topological space which may or may not be complete. We say that two valuations are \textit{equivalent} if they define the same topology and we call an equivalence class of absolute values a \textit{place} of $K$. It is an elementary result of topology that every metric space may be embedded in a complete metric space, and this can be done in an essentially unique way. For the field $K$, the resulting complete metric space may be given a field structure. Equivalently, there exists a field $L$ with a valuation $w$ such that $L$ is complete in the topology induced by $w$. The field $K$ is contained in $L$ and the valuations $v$ and $w$ coincide in $K$. Moreover, the completion $L$ of $K$ is unique up to topological isomorphism.

For any non-zero prime ideal $\mathfrak{p}$ of $\mathcal{O}_K$, let $\ord_{\mathfrak{p}}(\mathfrak{a})$ denote the exact power to which $\mathfrak{p}$ divides the ideal $\mathfrak{a}$. For fractional ideals $\mathfrak{a}$ this number may be negative. For $\alpha \in K$, we write $\ord_{\mathfrak{p}}(\alpha)$ for $\ord_{\mathfrak{p}}\left((\alpha)\mathcal{O}_K\right)$. Every prime ideal defines a discrete non-Archimedean valuation on $K$ via
\[v(x):= \left(\frac{1}{N_{K/\mathbb{Q}}(\mathfrak{p})}\right)^{\ord_{\mathfrak{p}}(x)}.\]
Furthermore, every embedding of $K$ into the complex field defines an Archimedean valuation. Conversely, every discrete valuation on $K$ arises in this way by a prime ideal of $\mathcal{O}_K$, while every Archimedean valuation of $K$ is equivalent to $|\sigma(x)|$, where $\sigma$ is an embedding of $K$ into $\mathbb{C}$. Valuations defined by different prime ideals are non-equivalent, and two valuations defined by different embeddings of $K$ into $\mathbb{C}$ are equivalent if and only if those embeddings are complex conjugated. The topology induced in $K$ by a prime ideal $\mathfrak{p}$ of $\mathcal{O}_K$ is called the \textit{$\mathfrak{p}$-adic topology}. The completion of $K$ under this valuation is denoted by $K_{\mathfrak{p}}$ or $K_v$ and called the \textit{$\mathfrak{p}$-adic field}. Let $V$ be the set of all valuations of an algebraic number field $K$. Then for every non-zero element $\alpha \in K$ we have 
\[\prod_{v \in V} v(\alpha) = 1.\]

In the ring of integers of $\mathbb{Q}$, the prime ideals are generated by the rational primes $p$, and the resulting topology in the field $\mathbb{Q}$ is called the \textit{$p$-adic topology}. The completion of $\mathbb{Q}$ under this valuation is denoted by $\mathbb{Q}_p$. If $v(x)$ is a non-trivial valuation of $\mathbb{Q}$, then either $v(x)$ is equivalent to the ordinary absolute value $|x|$, or it is equivalent to one of the $p$-adic valuations induced by rational primes. Analogous to $\ord_{\mathfrak{p}}$, for any prime $p$ we define the $p$-adic order of $x \in \mathbb{Q}$ as the largest exponent of $p$ dividing $x$. Then, the $p$-adic valuation $v$ is defined as
\[v(x) = p^{-\ord_p(x)}.\]
If $K_{\mathfrak{p}}$ is a $\mathfrak{p}$-adic field, it is necessarily a finite extension of a certain $\mathbb{Q}_p$. 

Consider now $K/\mathbb{Q}$ where $n = [K:\mathbb{Q}]$ and let $g(t)$ denote the minimal polynomial of $K$ over $\mathbb{Q}$. Suppose $p$ is a rational prime and let $g(t) = g_1(t) \cdots g_m(t)$ be the decomposition of $g(t)$ into irreducible polynomials $g_i(t) \in \mathbb{Q}_p[t]$ of degree $n_i = \deg g_i(t)$. The prime ideals in $K$ dividing $p$ are in one-to-one correspondence with $g_1(t), \dots, g_m(t)$. More precisely, we have in $K$ the following decomposition of $(p)\mathcal{O}_K$
\[(p)\mathcal{O}_K = \mathfrak{p}_1^{e(\mathfrak{p}_1|p)} \cdots \mathfrak{p}_m^{e(\mathfrak{p}_m|p)},\]
with $\mathfrak{p}_1, \dots, \mathfrak{p}_m$ distinct prime ideals and ramification indices $e(\mathfrak{p}_1 | p), \dots, e(\mathfrak{p}_m | p) \in \mathbb{N}$. For $i = 1, \dots, m$ the inertial degree of $\mathfrak{p}_i$ is denoted by $f(\mathfrak{p}_i|p)$. Then $n_i = e(\mathfrak{p}_i | p)f(\mathfrak{p}_i | p)$ and $K_{\mathfrak{p}_i} \simeq \mathbb{Q}_p(\theta_i)$, where $g(\theta_i) = 0$. 

By $\overline{\mathbb{Q}_p}$ we denote the algebraic closure of $\mathbb{Q}_p$. There are $n$ embeddings of $K$ into $\overline{\mathbb{Q}_p}$, and each one fixes $\mathbb{Q}$ and maps $\theta$ to a root of $g$ in $\overline{\mathbb{Q}_p}$. Let $\theta_i^{(1)}, \dots, \theta_i^{(n_i)}$ denote the roots of $g_i(t)$ in $\overline{\mathbb{Q}_p}$. For $i = 1, \dots, m$ and $j = 1, \dots, n_i$, let $\sigma_{ij}$ be the embedding of $K$ into $\mathbb{Q}_p(\theta_i^{(j)})$ defined by $\theta \mapsto \theta_i^{(j)}$. The $m$ classes of conjugate embeddings are $\{\sigma_{i1}, \dots, \sigma_{in_i}\}$ for $i = 1, \dots, m$. Note that $\sigma_{ij}$ coincides with the embedding $K \hookrightarrow K_{\mathfrak{p}_i} \simeq \mathbb{Q}_p(\theta_i) \simeq \mathbb{Q}_p(\theta_i^{(j)})$. 

For any finite extension $L$ of $\mathbb{Q}_p$, the $p$-adic valuation $v$ of $\mathbb{Q}_p$ extends uniquely to $L$ as 
\[v(x) = |N_{L/\mathbb{Q}_p}(x)|^{1/[L:\mathbb{Q}_p]}.\]
Here, we define the $p$-adic order of $x \in L$ by
\[\ord_p(x) = \frac{1}{[L:\mathbb{Q}_p]}\ord_p(N_{L/\mathbb{Q}_p}(x)).\]
This definition is independent of the field $L$ containing $x$. So, since each element of $\overline{\mathbb{Q}_p}$ is by definition contained in some finite extension of $\mathbb{Q}_p$, this definition can be used to define the $p$-adic valuation $v$ of any $x \in \overline{\mathbb{Q}_p}$. Every finite extension of $\mathbb{Q}_p$ is complete with respect to $v$, but $\overline{\mathbb{Q}_p}$ is not. The completion of $\overline{\mathbb{Q}_p}$ with respect to $v$ is denoted by $\mathbb{C}_p$. 

The $m$ extensions of the $p$-adic valuation on $\mathbb{Q}$ to $K$ are just multiples of the $\mathfrak{p}_i$-adic valuation on $K$:
\[\ord_p(x) = \frac{1}{e_i}\ord_{\mathfrak{p}_i}(x) \quad \text{ for } i = 1, \dots, m.\]
We also view these extensions as arising from various embeddings of $K$ into $\overline{\mathbb{Q}_p}$. Indeed, the extension to $\mathbb{Q}_p(\theta_i^{(j)})$ of the $p$-adic valuation on $\mathbb{Q}_p$ induces a $p$-adic valuation on $K$ via the embedding $\sigma_{ij}$ as 
\[v(x) = |N_{K_{\mathfrak{p}_i}/\mathbb{Q}_p}(\sigma_{ij}(x))|^{1/n_i}.\]
Here, as before, $n_i = \deg g_i(t) = [K_{\mathfrak{p}_i} : \mathbb{Q}_p]$. Furthermore, 
\[\ord_p(x) = \ord_p(\sigma_{ij}(x)),\]
and we have
\[\ord_p(\sigma_{ij}(x)) =  \frac{1}{e_i}\ord_{\mathfrak{p}_i}(x) \quad \text{ for } i = 1, \dots, m,\ j = 1, \dots, n_i.\]

Of course, in the special case $x \in \mathbb{Q}_p$, we can write
\[x = \sum_{i=k}^{\infty} u_ip^i\]
where $k = \ord_p(x)$ and the $p$-adic digits $u_i$ are in $\{0, \dots, p-1\}$ with $u_k \neq 0$. If $\ord_p(x) \geq 0$ then $x$ is called a $p$-adic \textit{integer}. The set of $p$-adic integers is denoted $\mathbb{Z}_p$. A $p$-adic \textit{unit} is an $x \in \mathbb{Q}_p$ with $\ord_p(x) = 0$. For any $p$-adic integer $\alpha$ and $\mu \in \mathbb{N}_0$ there exists a unique rational integer $x^{(\mu)} = \sum_{i=0}^{\mu-1}u_ip^i$ such that 
\[\ord_p(x-x^{(\mu)}) \geq \mu, \quad \text{ and } \quad 0 \leq x^{(\mu)} \leq p^{\mu} - 1.\]
For $\ord_p(x) \geq k$ we also write $x \equiv 0 \mod{p^k}$.

%--------------------------------------------------------------------------------------------------------------------------------------------%
%--------------------------------------------------------------------------------------------------------------------------------------------%

\section{$p$-adic logarithms}
\label{sec:pAdicLogarithms}

We have seen how to extend $p$-adic valuations to algebraic extensions of $\mathbb{Q}$. For any $z \in \mathbb{C}_p$ with $\ord_p(z-1) > 0$, we can also define the $p$-adic logarithm of $z$ by
\[\log_p(z) = -\sum_{i=1}^{\infty} \frac{(1-z)^i}{i}.\]
By the $n^{\text{th}}$ term test, this series converges precisely in the region where ${\ord_p(z-1) > 0}$. Three important properties of the $p$-adic logarithm are
\begin{enumerate}
\item $\log_p(xy) = \log_p(x) + \log_p(y)$ whenever $\ord_p(x-1) > 0$ and $\ord_p(y-1) > 0$.
\item $\log_p(z^k) = k \log(p)$ whenever $\ord_p(z-1) > 0$ and $k \in \mathbb{Z}$. 
\item $\ord_p(\log_p(z)) = \ord_p(z-1)$ whenever $\ord_p(z-1) > 1/(p-1)$.
\end{enumerate}
Proofs of the first and last property can be found in \cite{Has2} (pp. 264-265). The second property follows from the first.

We will use the following lemma to extend the definition of the $p$-adic logarithm to all $p$-adic units in $\overline{\mathbb{Q}_p}$. 
\begin{lemma} \label{lem: pAdicLogarithms}
Let $z$ be a $p$-adic unit belonging to a finite extensions $L$ of $\mathbb{Q}_p$. Let $e$ and $f$ be the ramification index and inertial degree of $L$. 
\begin{enumerate}[(a)]
\item There is a positive integer $r$ such that $\ord_p(z^r-1) >0$.
\item If $r$ is the smallest positive integer having $\ord_p(z^r-1) >0$, then $r$ divides $p^f-1$, and an integer $q$ satisfies $\ord_p(z^q-1) >0$ if and only if it is a multiple of $r$.
\item If $r$ is a nonzero integer with $\ord_p(z^r-1) >0$, and if $k$ is an integer with $p^k(p-1) > e$, then
\[\ord_p(z^{rp^k}-1) >\frac{1}{p-1}.\]
\end{enumerate}
\end{lemma}

For $z$ a $p$-adic unit in $\overline{\mathbb{Q}_p}$ we define
\[\log_p{z} = \frac{1}{q}\log_p{z^q},\]
where $q$ is an arbitrary non-zero integer such that $\ord_p(z^q-1) >0$. To see that this definition is independent of $q$, let $r$ be the smallest positive integer with $\ord_p(z^r-1) >0$, note that $q/r$ is an integer, and use the second property of $p$-adic logarithms above to write
\[\frac{1}{q}\log_p{z^q} = \frac{1}{r(q/r)}\log_p{z^{r(q/r)}} = \frac{1}{r}\log_p{z^r}.\]
Choosing $q$ such that $\ord_p(z^q-1) > 1/(p-1)$ helps to speed up and control the convergence of the series defining $\log_p$ (cf. \cite{Sm} (pp. 28-30) and \cite{Coh2} (pp. 263-265)).

It is straightforward to see that Properties 1 and 2 above extend to the case where $x,y,z$ are $p$-adic units. Combining this with Property 3, we obtain
\begin{lemma}\label{lem:pAdicLogarithms2}
Let $z_1, \dots, z_m \in \overline{\mathbb{Q}_p}$ be $p$-adic units and let $b_1, \dots, b_m \in \mathbb{Z}$. If 
\[\ord_p(z_1^{b_1}\cdots z_m^{b_m} - 1) > \frac{1}{p-1}\]
then 
\[\ord_p(b_1\log_p{z_1} + \cdots + b_m \log_p{z_m}) = \ord_p(z_1^{b_1}\cdots z_m^{b_m} - 1).\]
\end{lemma}

%--------------------------------------------------------------------------------------------------------------------------------------------%
%--------------------------------------------------------------------------------------------------------------------------------------------%

\section{The Weil height}
\label{sec:WeilHeight}

Let $K$ be a number field and at each place $v$ of $K$, let $K_v$ denote the completion of $K$ at $v$. Then
\[\sum_{v|p} [K_v:\mathbb{Q}_v] = [K:\mathbb{Q}]\]
for all places $p$ of $\mathbb{Q}$. We will use two absolute values $| \cdot |_v$ and $\| \cdot \|_v$ on $K$ which we now define. If $v|\infty$, then $\| \cdot \|_v$ restricted to $\mathbb{Q}$ is the usual Archimedean absolute value; if $v|p$ for a rational prime $p$, then $\| \cdot \|_v$ restricted to $\mathbb{Q}$ is the usual $p$-adic valuation. We then set
\[ | \cdot |_v = \| \cdot \|_v^{[K_v:\mathbb{Q}_v]/[K:\mathbb{Q}]}.\]
Let $x \in K$ and let $\log^+(\cdot)$ denote the real-valued function $\max\{\log(\cdot),0\}$ on $\mathbb{R}_{\geq 0}$. We define the \textit{logarithmic Weil height} $h(x)$ by 
\[h(x) = \frac{1}{[K:\mathbb{Q}]}\sum_v \log^+|x|_v,\]
where the sum is take over all places $v$ of $K$. If $x$ is an algebraic unit, then $|x|_v = 1$ for all non-Archimedean places $v$, and therefore $h(x)$ can be taken over the Archimedean places only. 
In particular, if $x \in \mathbb{Q}$, then with $x = p/q$ for $p,q \in \mathbb{Z}$ with $\gcd(p,q) = 1$, we have $h(x) = \log\max\{|p|,|q|\}$, and if $x \in \mathbb{Z}$ then $h(x) = \log|x|$. 

%reference: BoGu

%--------------------------------------------------------------------------------------------------------------------------------------------%
%--------------------------------------------------------------------------------------------------------------------------------------------%

\section{Elliptic curves}
\label{sec:EllipticCurves}

Let $K$ be a field of characteristic $\text{char}(K) \neq 2,3$. An \textit{elliptic curve} $E$ over $K$ is a nonsingular curve of the form 
\begin{equation} \label{eq:EllipticCurve}
E: y^2 + a_1xy + a_3y = x^3 + a_2x^2 + a_4x + a_6
\end{equation}
with $a_i \in K$ having a specified base point, $\mathcal{O}\in E$. An equation of the form (\ref{eq:EllipticCurve}) is called a \textit{Weierstrass equation}. This equation is unique up to a coordinate transformation of the form
\[x = u^2x' + r, \quad\quad y = u^3y' + su^2x' + t, \]
with $r,s,t,u \in K, u\neq 0$. 
Applying several linear changes of variables and writing 
\[b_2 = a_1^2 + 4a_2, \quad b_4 = a_1a_3 + 2a_4, \quad b_6 = a_3^2 + 4a_6,\]
\[b_8 = a_1^2a_6 + 4a_2a_6 - a_1a_3a_4 + a_2a_3^2 - a_4^2,\]
\[ c_4 = b_2^2 - 24b_4, \quad \text{ and } \quad c_6 = -b_2^3 + 36b_2b_4 + 9b_2b_4b_6,\]
$E$ can be written as
\[E: y^2 = x^3 - 27c_4x - 54c_6.\]
Associated to this curve are the quantities 
\[\Delta = -b_2^2b_8 - 8b_4^3 - 27b_6^2 + 9b_2b_4b_6 \quad \text{ and } \quad j = c_4^3/\Delta,\]
where $\Delta$ is called the \textit{discriminant} of the Weierstrass equation and the quantity $j$ is called the \textit{j-invariant} of the elliptic curve. The condition of being nonsingular is equivalent to $\Delta$ being non-zero. Two elliptic curves are isomorphic over $\bar{K}$, the algebraic closure of $K$, if and only if they both have the same $j$-invariant.

When $K = \mathbb{Q}$, the Weierstrass model (\ref{eq:EllipticCurve}) can be chosen so that $\Delta$ has minimal $p$-adic order for each rational prime $p$ and $a_i \in \mathbb{Z}$. Suppose (\ref{eq:EllipticCurve}) is such a global minimal model for an elliptic curve $E$ over $\mathbb{Q}$. Reducing the coefficients modulo a rational prime $p$ yields a (possibly singular) curve over $\mathbb{F}_p$
\begin{equation}
\tilde{E}: y^2 + \tilde{a_1}xy + \tilde{a_3}y = x^3 + \tilde{a_2}x^2 + \tilde{a_4}x + \tilde{a_6},
\end{equation}
where $\tilde{a_i} \in \mathbb{F}_p$. This ``reduced" curve $\tilde{E}/\mathbb{F}_p$ is called the \textit{reduction of $E$ modulo} $p$. It is nonsingular provided that $\Delta \not \equiv 0 \mod{p}$, in which case it is an elliptic curve defined over $\mathbb{F}_p$. The curve $E$ is said to have \textit{good reduction} modulo $p$ if $\tilde{E}/\mathbb{F}_p$ is nonsingular, otherwise, we say $E$ has \textit{bad reduction} modulo $p$. 

The reduction type of $E$ at a rational prime $p$ is measured by the \textit{conductor}, 
\[N = \prod_{p}p^{f_p}\]
where the product runs over all primes $p$ and $f_p = 0$ for all but finitely many primes. In particular, $f_p \neq 0$ if $p$ does not divide $\Delta$. Equivalently, $E$ has bad reduction at $p$ if and only if $p \mid N$. Suppose $E$ has bad reduction at $p$ so that $f_p \neq 0$. The reduction type of $E$ at $p$ is said to be \textit{multiplicative} ($E$ has a node over $\mathbb{F}_p$) or \textit{additive} ($E$ has a cusp over $\mathbb{R}_p$) depending on whether $f_p = 1$ or $f_p \geq 2$, respectively. The $f_p$, hence the conductor, are invariant under isogeny. 

%--------------------------------------------------------------------------------------------------------------------------------------------%
%--------------------------------------------------------------------------------------------------------------------------------------------%

\section{Cubic forms}
\label{sec:CubicForms}

Let $a,b,c$ and $d$ be integers and consider the binary cubic form
\[F(x,y) = ax^3 + bx^2y + cxy^2 + dy^3.\]
Two such forms $F_1$ and $F_2$ are called \textit{equivalent} if they are equivalent under the $GL_{2}(\mathbb{Z})$-action. That is, if there exist integers $a_1, a_2, a_3$, and $a_4$ such that 
\[F_1(a_1x + a_2y, a_3x + a_4y) = F_2(x,y)\]
for all $x,y$, where $a_1a_4 - a_2a_3 = \pm 1$. In this case, we write $F_1 \sim F_2$. The \textit{discriminant} $D_F$ of such a form is given by 
\[D_F = -27a^2d^2 + b^2c^2 + 18abcd - 4ac^3 - 4b^3d = a^4 \prod_{i < j} (\alpha_i - \alpha_j)^2,\]
where $\alpha_1, \alpha_2$ and $\alpha_3$ are the roots of the polynomial $F(x,1)$. We observe that if $F_1 \sim F_2$, then $D_{F_1} = D_{F_2}$. 

Associated to $F$ is the Hessian $H_F(x,y)$, given by
\begin{align*}
H_F(x,y) & = -\frac{1}{4}\left( \frac{\partial^2F}{\partial x^2} \frac{\partial^2F}{\partial y^2} - \left(\frac{\partial^2F}{\partial x \partial y}\right)^2\right)\\
& = (b^2 - 3ac)x^2 + (bc - 9ad)xy + (c^2 - 3bd)y^2,
\end{align*}
and the Jacobian determinant of $F$ and $H_F$, a cubic form $G_F(x,y)$ defined by
\begin{align*}
G_F(x,y) &= \frac{\partial F}{\partial x} \frac{\partial H_F}{\partial y} - \frac{\partial F}{\partial y} \frac{\partial H_F}{\partial x} \\
& =  (-27a^2d + 9abc -2b^3)x^3 + (-3b^2c - 27abd + 18ac^2)x^2y +  \\
& \quad \quad + (3bc^2 - 18b^2d + 27acd)xy^2 + (-9bcd + 2c^3 + 27ad^2)y^3.
\end{align*}

%--------------------------------------------------------------------------------------------------------------------------------------------%
%--------------------------------------------------------------------------------------------------------------------------------------------%

\section{Lattices}
\label{sec:Lattices}

An $n$-dimensional lattice is a discrete subgroup of $\mathbb{R}^n$ of the form
\[\Gamma = \left\{ \sum_{i=1}^n x_i \mathbf{b}_i \ : \ x_i \in \mathbb{Z} \right\},\]
where $\mathbf{b}_1, \dots, \mathbf{b_n}$ are vectors forming a basis for $\mathbb{R}^n$. We say that the vectors $\mathbf{b}_1, \dots, \mathbf{b_n}$ form a \textit{basis} for $\Gamma$, or that they generate $\Gamma$. Let $B$ denote the matrix whose columns are the vectors $\mathbf{b}_1, \dots, \mathbf{b_n}$. Any lattice element $\mathbf{v}$ may be expressed as $\mathbf{v} = B\mathbf{x}$ for some $\mathbf{x} \in \mathbb{Z}^n$. We call $\mathbf{v}$ the \textit{embedded vector} and $\mathbf{x}$ the \textit{coordinate vector}.

A \textit{bilinear form} on a lattice $\Gamma$ is a function $\Phi: \Gamma \times \Gamma \to \mathbb{Z}$ satisfying
\begin{enumerate}
\item $\Phi(\mathbf{u}, \mathbf{v}+\mathbf{w}) = \Phi(\mathbf{u},\mathbf{v}) + \Phi(\mathbf{u},\mathbf{w})$
\item $\Phi(\mathbf{u}+\mathbf{v}, \mathbf{w}) = \Phi(\mathbf{u},\mathbf{w}) + \Phi(\mathbf{v},\mathbf{w})$
\item $\Phi(a\mathbf{u}, \mathbf{w}) = a\Phi(\mathbf{u},\mathbf{w})$
\item $\Phi(\mathbf{u}, a\mathbf{w}) = a\Phi(\mathbf{u},\mathbf{w})$
\end{enumerate}
for all $\mathbf{u}, \mathbf{v}$, and $\mathbf{w}$ in $\Gamma$ and any $a \in \mathbb{Z}$. 

Given a basis, we can define a specific bilinear form on our lattice $\Gamma$ as part of its structure. This form describes a kind of distance between elements $\mathbf{u}$ and $\mathbf{v}$ and we say the lattice is \textit{defined} by $\Phi$. Associated to this bilinear form is a quadratic form $Q: \Gamma \to \mathbb{Z}$ defined by $Q(\mathbf{v}) = \Phi(\mathbf{v}, \mathbf{v})$. A lattice is called \textit{positive definite} if its quadratic form is positive definite. 

The bilinear forms (and their associated quadratic forms) that we will be using come from the usual inner product on vectors in $\mathbb{R}^n$. This is simply the dot product $\Phi(\mathbf{u},\mathbf{v}) = \mathbf{u} \cdot \mathbf{v}$ for embedded vectors, $\mathbf{u},\mathbf{v}$. For the coordinate vectors $\mathbf{x},\mathbf{y}$ associated to these vectors, this translates to multiplication with the basis matrix. Precisely, if $\mathbf{u} = B\mathbf{x}$ and $\mathbf{v} = B\mathbf{y}$, we have $\Phi(\mathbf{u},\mathbf{v}) = \mathbf{x}^TB^TB\mathbf{y}$. 

If $\mathbf{v} = B\mathbf{x}$, the \textit{norm} of the vector $\mathbf{v} \in \Gamma$ is defined to be the inner product $\Phi(\mathbf{v},\mathbf{v})$. In terms of the corresponding coordinate vector $\mathbf{x}$, this is
\[\mathbf{v}^T\mathbf{v} = \mathbf{x}^TB^TB\mathbf{x}.\]
Equivalently, we write $\mathbf{x}^TA\mathbf{x}$ where $A = B^TB$ is the Gram matrix of $\Gamma$ with basis $B$ and bilinear form $\Phi$. The entries of the matrix $A$ are $a_{ij} = \Phi(\mathbf{b}_i,\mathbf{b}_j)$.

Two basis matrices $B_1$ and $B_2$ define the same lattice $\Gamma$ if and only if there is a unimodular matrix $U$ such that $B_1U = B_2$. The bilinear form on $\Gamma$ can be written with respect to either embedded or coordinate vectors. Using another basis to express the lattice elements is possible, and sometimes preferable. However, the Gram matrix is specific to the bilinear form on the lattice and should not change when operating on embedded vectors. If it is operating on coordinate vectors, the change of basis must be accounted for. 

%--------------------------------------------------------------------------------------------------------------------------------------------%

\endinput

Any text after an \endinput is ignored.
You could put scraps here or things in progress.
%% The following is a directive for TeXShop to indicate the main file
%%!TEX root = diss.tex

\chapter{Algorithms for Thue-Mahler Equations}
\label{ch:AlgorithmsForTM}

In this chapter, we give some of primary algorithms needed to solve an arbitrary Thue-Mahler equation. The methods presented here follow somewhat \cite{Ham} and \cite{TW3}, with new results and modifications from \cite{GhKaMaSi}. 

%--------------------------------------------------------------------------------------------------------------------------------------------%
%--------------------------------------------------------------------------------------------------------------------------------------------%

\section{First steps}
\label{sec:FirstSteps}

Fix a nonzero integer $c$ and let $S=\{p_1,\dotsc,p_v\}$ be a set of rational primes. Let
\[F(X,Y) = c_0 X^n + c_1 X^{n-1}Y + \cdots + c_{n-1}XY^{n-1} + c_nY^n\]
be an irreducible binary form over $\mathbb{Z}$ of degree $n \geq 3$. We want to solve the Thue--Mahler equation
\begin{equation} \label{eq:ThueMahler}
F(X,Y) = c p_1^{Z_1}\cdots p_v^{Z_v}
\end{equation}
for unknowns $X,Y, Z_1, \dots, Z_v$ with $\gcd(X,Y) = 1$ and $Z_i \geq 0$ for $i = 1,\dots, v$. To do so, we first reduce \eqref{eq:ThueMahler} to the special case where $c_0 = 1$ and $\gcd(c,p_i) = 1$ for $i = 1, \dots, v$, loosely following \cite{Ham}. 

As $F$ is irreducible by assumption, at least one of the coefficients $c_0$ and $c_n$ is nonzero. Hence, we may transform the given Thue--Mahler equation to one with $c_0 \neq 0$ by interchanging $X$ and $Y$ and by renaming the coefficients $c_i$ appropriately. In particular, solving \eqref{eq:ThueMahler} is equivalent to solving 
\[ c_0' \overline{X}^n + c_1' \overline{X}^{n-1}\overline{Y} + \cdots + c_{n-1}'\overline{X}\overline{Y}^{n-1} + c_n'\overline{Y}^n = c p_1^{Z_1}\cdots p_v^{Z_v},\]
where $c_i' = c_{n-1}$ for $i = 0, \dots, n$, $\overline{X} = Y$, and $\overline{Y} = X$. 

Denote by $\mathcal{D}$ the set of all positive rational integers $m$ dividing $c_0$ such that ${\ord_p(m)\leq \ord_p(c)}$ for each rational prime $p\notin S$. Equivalently, $\mathcal{D}$ is precisely the set of all possible integers $d$ such that $d = \gcd(c_0,Y)$. To see this, let $q_1, \dots, q_{w}$ denote the distinct prime divisors of $a$ not contained in $S$. Then 
\[c = \prod_{i=1}^w q_i^{b_i}\cdot \prod_{i=1}^v p_i^{\ord_{p_i}(c)}\]
for some integers $b_i >0$. If $(X,Y,Z_1, \dots, Z_v)$ is a solution of the Thue-Mahler equation in question, it follows that
\[F(X,Y) = cp_1^{Z_1}\dots p_v^{Z_v} =  \prod_{i=1}^w q_i^{b_i}\cdot \prod_{i=1}^v p_i^{\ord_{p_i}(c) + Z_i}.\]
Suppose $\gcd(c_0,Y) = d$. Since $d$ divides $F(X,Y)$, it necessarily divides 
\[{\prod_{i=1}^w q_i^{b_i}\cdot \prod_{i=1}^v p_i^{\ord_{p_i}(c) + Z_i}}.\] 
In particular, 
\[d = \prod_{i=1}^w q_i^{s_i}\cdot \prod_{i=1}^v p_i^{t_i}\]
for some non-negative integers $s_1, \dots, s_w, t_1, \dots, t_v$ such that 
\[s_i \leq \min\{\ord_{q_i}(c), \ord_{q_i}(c_0)\} \quad \text{ and } \quad 
	t_i \leq \min\{\ord_{p_i}(c) + Z_i, \ord_{p_i}(c_0)\}.\] 
From here, it is easy to see that ${\ord_p(d)\leq \ord_p(c)}$ for each rational prime $p\notin S$ so that $d \in \mathcal{D}$. 

Conversely, suppose $d \in \mathcal{D}$ so that $\ord_{p}(d) \leq \ord_{p}(c)$ for all $p \notin S$. That is, the right-hand side of 
\[\ord_{p}(d) \leq \ord_{p}(c) = 
\ord_p\left(\prod_{i=1}^w q_i^{b_i}\cdot \prod_{i=1}^v p_i^{\ord_{p_i}(c)}\right)\]
is non-trivial only at the primes $\{q_1, \dots, q_w\}$. In particular, 
\[d = \prod_{i=1}^w q_i^{s_i}\cdot \prod_{i=1}^v p_i^{t_i}\]
for non-negative integers $s_1, \dots, s_w, t_1, \dots, t_v$ such that 
\[s_i \leq \min\{\ord_{q_i}(c), \ord_{q_i}(c_0)\} \quad \text{ and } \quad 
	t_i \leq \ord_{p_i}(c_0).\] 
It follows that $d = \gcd(c_0,Y)$ for some solution $(X,Y,Z_1, \dots, Z_v)$ of equation~\eqref{eq:ThueMahler}. 

For any $d\in \mathcal{D}$, we define the rational numbers 
\[u_d = c_0^{n-1}/d^n \quad \textnormal{and}\quad c_d = \sgn(u_dc)\prod_{p\notin S} p^{\ord_p(u_dc)}.\]
On using that $d\in \mathcal{D}$, we see that the rational number $c_d$ is in fact an integer coprime to $S$. 

Suppose $(X,Y,Z_1, \dots, Z_v)$ is a solution of \eqref{eq:ThueMahler} with ${\gcd(X,Y) = 1}$ and $d = \gcd(c_0,Y)$. Define the homogeneous polynomial $f(x,y) \in \mathbb{Z}[x,y]$ of degree $n$ by
\[f(x,y) = x^n + C_1 x^{n-1}y + \dots + C_{n-1}xy^{n-1} + C_ny^n,\]
where
\[x=\tfrac{c_0X}{d},\quad y=\tfrac{Y}{d} \quad \text{ and } \quad C_i = c_ic_0^{i-1} \quad \text{ for } i = 1, \dots, n.\]
Since $\gcd(X,Y) = 1$, the numbers $x$ and $y$ are also coprime integers by definition of $d$. We observe that 
\[f(x,y) = u_dF(X,Y) = u_dc \prod_{i = 1}^v p_i^{Z_i} = c_d\prod_{p \in S}p^{Z_i + \ord_p(u_dc)}.\]
Setting $z_i = Z_i + \ord_p(u_dc)$ for all $i \in \{1, \dots, v\}$, we obtain
\begin{equation} \label{eq:ThueMahler2}
f(x,y) = x^n + C_1 x^{n-1}y + \dots + C_{n-1}xy^{n-1} + C_ny^n = c_d p_1^{z_1}\cdots p_v^{z_v}, 
\end{equation}
where $\gcd(x,y) = 1$ and $\gcd(c_d,p_i) = 1$ for all $i = 1, \dots, v$. 

Since there are only finitely many choices for $d = \gcd(c_0, Y)$, there are only finitely many choices for $\{c_d,u_d,d\}$. Then, solving \eqref{eq:ThueMahler} is equivalent to solving the finitely many Thue-Mahler equations \eqref{eq:ThueMahler2} for each choice of $\{c_d,u_d,d\}$.  For each such choice, the solution $\{x,y,z_1, \dots, z_v\}$ is related to $\{X,Y, Z_1, \dots, Z_v\}$ via
\[X = \frac{dx}{c_0},\quad Y=dy \quad \text{ and } \quad Z_i = z_i - \ord_p(u_dc).\]

Lastly, we observe that the polynomial $f(x,y)$ of \eqref{eq:ThueMahler2} remains the same for any choice of $\{c_d,u_d,d\}$. Thus, to solve the family of equations \eqref{eq:ThueMahler2}, we need only to enumerate over every possible $c_d$. Now, if $\mathcal{C}$ denotes the set of all $\{c_d,u_d,d\}$ and $d_1, d_2 \in \mathcal{D}$, we may have $\{c_{d_1},u_{d_1}, d_1\}, \{c_{d_2},u_{d_2}, d_2\} \in \mathcal{C}$ where $c_{d_1} = c_{d_2}$. That is, $d_1, d_2$ may yield the same value of $c_d$, reiterating that we need only solve \eqref{eq:ThueMahler2} for each distinct $c_d$. 

%--------------------------------------------------------------------------------------------------------------------------------------------%
%--------------------------------------------------------------------------------------------------------------------------------------------%

\section{The relevant algebraic number field}
\label{sec:RelevantAlgNumField}

For the remainder of this chapter, we consider the Thue-Mahler equation
\begin{equation} \label{eq:ThueMahler3}
f(x,y) = x^n + C_1 x^{n-1}y + \dots + C_{n-1}xy^{n-1} + C_ny^n = c p_1^{z_1} \cdots p_v^{z_v}
\end{equation}
where $\gcd(x,y) = 1$ and $\gcd(c,p_i) = 1$ for $i = 1, \dots, p_v$.

Following \cite{TW3}, put
\[g(t) = f(t,1) = t^n + C_1 t^{n-1} + \dots + C_{n-1}t + C_n\]
and note that $g(t)$ is irreducible in $\mathbb{Z}[t]$. Let $K = \mathbb{Q}(\theta)$ with $g(\theta) = 0$. Now \eqref{eq:ThueMahler3} is equivalent to the norm equation
\begin{equation} \label{eq:normTM}
N_{K/\mathbb{Q}}(x-y\theta) = cp_1^{z_1}\dots p_v^{z_v}.
\end{equation}

Let $p_i$ be any rational prime and let 
\[(p_i)\mathcal{O}_K = \prod_{j = 1}^{m_i} \mathfrak{p}_{ij}^{e(\mathfrak{p}_{ij}|p_i)}\]
be the factorization of $p_i$ into prime ideals in the ring of integers $\mathcal{O}_K$ of $K$. Let $f(\mathfrak{p}_{ij}|p_i)$ be the inertial degree of $\mathfrak{p}_{ij}$ over $p_i$. Since $N(\mathfrak{p}_{ij}) = p_i^{f_{ij}}$, \eqref{eq:normTM} leads to finitely many ideal equations of the form
\begin{equation} \label{eq:idealTM}
(x-y\theta)\mathcal{O}_K = \mathfrak{a} \prod_{j = 1}^{m_1} \mathfrak{p}_{1j}^{z_{1j}} \cdots \prod_{j = 1}^{m_v} \mathfrak{p}_{vj}^{z_{vj}}
\end{equation}
where $\mathfrak{a}$ is an ideal of norm $|c|$ and the $z_{ij}$ are unknown integers related to $z_i$ by 
\[\sum_{j = 1}^{m_i} f(\mathfrak{p}_{ij}|p_i)z_{ij} = z_i\]
for $i \in \{1, \dots, v\}$.

Our first task is to cut down the number of variables appearing in \eqref{eq:idealTM}. We will do this by showing that only a few prime ideals can divide $(x-y\theta)\mathcal{O}_K$ to a large power. 

%--------------------------------------------------------------------------------------------------------------------------------------------%
%--------------------------------------------------------------------------------------------------------------------------------------------%

\section{The prime ideal removing lemma}
\label{sec:PIRL}

In this section, we establish some key results that will allow us to cut down the number of prime ideals that can appear to a large power in the factorization of $(x-y\theta)\mathcal{O}_K$. It is of particular importance to note that we do not appeal to the Prime Ideal Removing Lemma of Tzanakis and de Weger (\cite{TW3}) here and instead apply the following results of \cite{GhKaMaSi}. 

Let $p \in \{p_1, \dots, p_v\}$. We will produce the following two finite lists $L_p$ and $M_p$. The list $L_p$ will
consist of certain ideals $\mathfrak{b}$ of $\mathcal{O}_K$ supported at the prime ideals above $p$. The list $M_p$ will consist of certain pairs $(\mathfrak{b},\mathfrak{p})$ where $\mathfrak{b}$ is supported at the prime ideals above $p$ and $\mathfrak{p}$ is a prime ideal lying over $p$ satisfying $e(\mathfrak{p}|p)=f(\mathfrak{p}|p)=1$. These lists will satisfy the following property: if $(x,y,z_1,\dots,z_v)$ is a solution to the Thue-Mahler equation \eqref{eq:ThueMahler3} then
\begin{enumerate}[(i)]
\item either there is some $\mathfrak{b} \in L_p$
such that
\begin{equation} \label{eq:Lp}
\mathfrak{b} \mid (x-y\theta )\mathcal{O}_K, \qquad \text{$(x-y\theta)\mathcal{O}_K/\mathfrak{b}$ is coprime to $(p)\mathcal{O}_K$};
\end{equation}
\item or there is a pair $(\mathfrak{b},\mathfrak{p}) \in M_p$ and a non-negative integer $v_p$ such that
\begin{equation} \label{eq:Mp}
(\mathfrak{b} \mathfrak{p}^{v_p}) \mid (x-y\theta)\mathcal{O}_K, \qquad \text{$(x-y\theta)\mathcal{O}_K/(\mathfrak{b} \mathfrak{p}^{v_p})$ is coprime to $(p)\mathcal{O}_K$}.
\end{equation}
\end{enumerate}

To generate the lists $M_p$, $L_p$ we consider two affine patches, $p \nmid y$ and $p \mid y$. We begin with the following lemmata.

\begin{lemma} \label{lem:AffinePatch1}
Let $(x,y,z_1, \dots, z_v)$ be a solution of \eqref{eq:ThueMahler3} with $p \nmid y$, let $t$ be a positive integer, and suppose $x/y \equiv u \pmod{p^t}$, where ${u \in \{0,1,2,\dotsc,p^{t}-1\}}$. If $\mathfrak{q}$ is a prime ideal of $\mathcal{O}_K$ lying over $p$, then
\[\ord_{\mathfrak{q}}(x-y\theta)\ge \min\{\ord_{\mathfrak{q}}(u-\theta), t \cdot e(\mathfrak{q}|p)\}.\]
Moreover, if $\ord_{\mathfrak{q}}(u-\theta) < t \cdot e(\mathfrak{q}|p)$, then
\[\ord_\mathfrak{q}(x-y\theta) = \ord_{\mathfrak{q}}(u-\theta).\]
\end{lemma}

\begin{lemma} \label{lem:AffinePatch2}
Let $(x,y,z_1, \dots, z_v)$ be a solution of \eqref{eq:ThueMahler3} with $p \mid y$ (and thus $p \nmid x$), let $t$ be a positive integer, and suppose $y/x \equiv u \pmod{p^t}$, where $u \in \{0,1,2,\dotsc,p^{t}-1\}$. If $\mathfrak{q}$ is a prime ideal of $\mathcal{O}_K$ lying over $p$, then
\[\ord_{\mathfrak{q}}(x-y\theta)\ge \min\{\ord_{\mathfrak{q}}(1-\theta u), t \cdot e(\mathfrak{q}|p)\}.\]
Moreover, if $\ord_{\mathfrak{q}}(1-\theta u) < t \cdot e(\mathfrak{q}|p)$, then
\[\ord_\mathfrak{q}(x-y\theta) = \ord_{\mathfrak{q}}(1 - \theta u).\]
\end{lemma}

\begin{proof}[Proof of Lemmas~\ref{lem:AffinePatch1} and \ref{lem:AffinePatch2}]
Suppose $p \nmid y$. Thus $\ord_{\mathfrak{q}}(y) = 0$ and hence 
\[\ord_{\mathfrak{q}}(x-y\theta) = \ord_{\mathfrak{q}}(x/y - \theta).\]
Since $x/y-\theta = u - \theta + x/y - u,$ we have
\[\begin{array}{ll}
\ord_\mathfrak{q}(x/y-\theta)	& = \ord_{\mathfrak{q}}(u - \theta + x/y - u) \\
						& \geq \min\{\ord_{\mathfrak{q}}(u - \theta), \ord_{\mathfrak{q}}(x/y - u)\}. 
\end{array}\]
By assumption, 
\[\ord_{\mathfrak{q}}(x/y-u) \geq \ord_{\mathfrak{q}}(p^t) = t \cdot e(\mathfrak{q}|p),\]
 %Thus $\ord_\fq(x-\theta)=\ord_\fq(u-\theta)$,
completing the proof of Lemma~\ref{lem:AffinePatch1}. The proof of Lemma~\ref{lem:AffinePatch2} is similar. 
\end{proof}

The following algorithm computes the lists $L_p$ and $M_p$ that come from the first patch $p \nmid y$. We denote these respectively by $\mathcal{L}_p$ and $\mathcal{M}_p$. 

\begin{algorithm} \label{alg:AffinePatch1}
To compute
$\mathcal{L}_p$ and $\mathcal{M}_p$:

\begin{enumerate}[Step (1)]
\item Let 
\[\mathcal{L}_p \leftarrow \emptyset, \qquad \mathcal{M}_p \leftarrow \emptyset,\]
\[ t \leftarrow 1, \quad \mathcal{U} \leftarrow \{w : w \in \{0,1,\dots,p-1\} \}.\]
\item Let
\[\mathcal{U}^\prime \leftarrow \emptyset.\]
Loop through the elements $u \in \mathcal{U}$. Let 
\[\mathcal{P}_u= \{\mathfrak{q} \text{ lying above } p \ : \ \ord_{\mathfrak{q}}(u-\theta) \geq t \cdot e(\mathfrak{q}|p)\}\]
and
\[ \mathfrak{b}_u 	= \prod_{\mathfrak{q} \mid p} \mathfrak{q}^{\min\{\ord_\mathfrak{q}(u-\theta), t \cdot e(\mathfrak{q}|p)\}} 
				= (u-\theta) \mathcal{O}_K+p^t \mathcal{O}_K.\]
\begin{enumerate}[(i)]
\item If $\mathcal{P}_u = \emptyset$ then
\[\mathcal{L}_p \leftarrow \mathcal{L}_p \cup \{\mathfrak{b}_u\}.\]
\item Else if $\mathcal{P}_u = \{\mathfrak{p}\}$ with $e(\mathfrak{p}|p)=f(\mathfrak{p}|p)=1$ and there is at least one $\mathbb{Z}_p$-root $\alpha$ of $g(t)$ satisfying $\alpha \equiv u \pmod{p^t}$, then
\[\mathcal{M}_p \leftarrow \mathcal{M}_p \cup \{ (\mathfrak{b}_u,\mathfrak{p})\}.\]
\item Else 
\[\mathcal{U}^\prime \leftarrow \mathcal{U} \cup \{ u+p^{t}w : w \in \{0,\dots,p-1\} \}.\]
\end{enumerate}

\item If $\mathcal{U}^\prime \ne \emptyset$ then let
\[t \leftarrow t+1, \qquad \mathcal{U} \leftarrow \mathcal{U}^{\prime},\]
and return to Step (2). Else output $\mathcal{L}_p$, $\mathcal{M}_p$.
\end{enumerate}
\end{algorithm}

\begin{lemma}
Algorithm~\ref{alg:AffinePatch1} terminates.
\end{lemma}

\begin{proof}
Suppose otherwise. Write $t_0=1$ and $t_i=t_0+i$ for $i=1,2,3,\dots$. Then there is an infinite sequence of congruence classes $u_i \mod{p^{t_i}}$ such that ${u_{i+1} \equiv u_i \mod{p^{t_i}}}$, and such that the $u_i$ fail the hypotheses of both (i) and (ii). This means that $\mathcal{P}_{u_i}$ is non-empty for every $i \in \mathbb{N}_{>0}$. By the pigeon-hole principle, some prime ideal $\mathfrak{p}$ of $\mathcal{O}_K$ appears in infinitely many of the $\mathcal{P}_{u_i}$. Thus ${\ord_{\mathfrak{p}}(u_i-\theta) \ge t_i\cdot e(\mathfrak{p}|p)}$ infinitely often. However, the sequence $\{u_i\}_{i=1}^{\infty}$ converges to some $\alpha \in \mathbb{Z}_p$ so that $\alpha=\theta$ in $K_\mathfrak{p}$. This forces $e(\mathfrak{p}|p)=f(\mathfrak{p}|p)=1$ and $\alpha$ to be a $\mathbb{Z}_p$-root of $g(t)$. In this case, $\mathfrak{p}$ corresponds to the factor $(t-\alpha)$ in the $p$-adic factorisation of $g(t)$. There can be at most one such $\mathfrak{p}$, forcing $\mathcal{P}_{u_i}=\{\mathfrak{p}\}$ for all $i$. In particular, the hypothesis of (ii) are satisfied and we reach a contradiction.
\end{proof}

\begin{lemma}\label{lem:AffinePatch1Check}
Let $p \in \{p_1, \dots, p_v\}$ and let $\mathcal{L}_p$, $\mathcal{M}_p$ be as given by Algorithm~\ref{alg:AffinePatch1}. Let $(x,y,z_1,\dots, z_v)$ be a solution to \eqref{eq:ThueMahler3}. Then
\begin{itemize}
\item either there is some $\mathfrak{b} \in \mathcal{L}_p$ such that \eqref{eq:Lp} is satisfied; 
\item or there is some $(\mathfrak{b},\mathfrak{p}) \in \mathcal{M}_p$ with $e(\mathfrak{p}|p)=f(\mathfrak{p}|p)=1$ and integer $v_p \geq 0$ such that \eqref{eq:Mp} is satisfied.
\end{itemize}
\end{lemma}

\begin{proof}
Let 
\[t_0 = 1 \quad \text{ and } \quad \mathcal{U}_0=\{w \; :\;  w \in \{0,1,\dots,p-1\}\}\]
be the initial values for $t$ and $\mathcal{U}$ in the algorithm. Then $x/y \equiv u_0 \pmod{p^{t_0}}$ for some $u_0 \in \mathcal{U}_0$. Write $\mathcal{U}_i$ for the value of $\mathcal{U}$ after $i$ iterations of the algorithm  and let $t_i=t_0+i$. As the algorithm terminates, $\mathcal{U}_i = \emptyset$ for some sufficiently large $i$. Hence there is some $i$ such that $x/y \equiv u_i \mod{p^{t_i}}$ where $u_i \in \mathcal{U}_i$, but there is no element in $\mathcal{U}_{i+1}$ congruent to $x/y$ modulo $p^{t_{i+1}}$. In other words, $u_i$ must satisfy the hypotheses of either step (i) or (ii) of algorithm~\ref{alg:AffinePatch1}. Write $u=u_i$ and $t=t_i$ for $x/y \equiv u \mod{p^t}$ and consider the ideal $\mathfrak{b}_u$ generated in this step. By Lemma~\ref{lem:AffinePatch1}, $\mathfrak{b}_u$ divides $(x-y\theta) \mathcal{O}_K$. Furthermore, by definition of $\mathcal{P}_u$, if $\mathfrak{q}$ is a prime ideal of $\mathcal{O}_K$ not contained in $\mathcal{P}_u$, then $(x-y\theta)\mathcal{O}_K/\mathfrak{b}_u$ is not divisible by $\mathfrak{q}$. 

Suppose first that the hypothesis of (i) is satisfied: $\mathcal{P}_u = \emptyset$. The algorithm adds $\mathfrak{b}_u$ to the set $\mathcal{L}_p$, with the above remarks ensuring that \eqref{eq:Lp} is satisfied.

Suppose next that the hypothesis of (ii) is satisfied: $\mathcal{P}_u=\{\mathfrak{p}\}$ where ${e(\mathfrak{p}|p)=f(\mathfrak{p}|p)=1}$ and there is a unique $\mathbb{Z}_p$ root $\alpha$ of $g(t)$ such that $\alpha \equiv u \mod{p^t}$. The algorithm adds $(\mathfrak{b}_u,\mathfrak{p})$ to the set $\mathcal{M}_p$. By the above, $(x-y\theta)\mathcal{O}_K/\mathfrak{b}_u$ is an integral ideal, not divisible by any prime ideal $\mathfrak{q} \neq \mathfrak{p}$ lying over $p$. Thus there is some positive integer $v_p \geq 0$ such that \eqref{eq:Mp} is satisfied, concluding the proof. 
\end{proof}

Having computed the lists arising from the affine patch $p \nmid y$, we initialize $L_p$ and $M_p$ as $\mathcal{L}_p$ and $\mathcal{M}_p$, respectively, and append to these lists the elements from the second patch, $p \mid y$, using the following algorithm.  

\begin{algorithm}\label{alg:AffinePatch2}
To compute $L_p$ and $M_p$.

\begin{enumerate}[Step (1)]
\item Let 
\[ L_p \leftarrow \mathcal{L}_p, \qquad M_p \leftarrow \mathcal{M}_p,\]
where $\mathcal{L}_p$, $\mathcal{M}_p$ are computed by Algorithm~\ref{alg:AffinePatch1}.
\item Let
\[ t \leftarrow 2, \qquad \mathcal{U} \leftarrow \{pw \; : \; w \in \{0,1,\dots,p-1\} \}.\]
\item Let
\[ \mathcal{U}^{\prime} \leftarrow \emptyset.\]
Loop through the elements $u \in \mathcal{U}$. Let 
\[\mathcal{P}_u=\{\mathfrak{q} \text{ lying above } p \ : \ \ord_{\mathfrak{q}}(1-u\theta ) \ge t \cdot e(\mathfrak{q}|p)\},\]
and
\[ \mathfrak{b}_u=\prod_{\mathfrak{q} \mid p} \mathfrak{q}^{\min\{\ord_{\mathfrak{q}}(1-u\theta ), t \cdot e(\mathfrak{q}|p)\}} =(1-u\theta) \mathcal{O}_K+p^t \mathcal{O}_K.\]

\begin{enumerate}[(i)]
\item If $\mathcal{P}_u=\emptyset$  
%$\ord_p(\Norm(u-\theta)) \ge (n-1) c_0$ 
then
\[L_p \leftarrow L_p \cup \{\mathfrak{b}_u\}.\]
%\item[(ii)] Else if $\cP_u=\{\fp\}$ with
%$e(\fp/p)=f(\fp/p)=1$, 
%and
%$\ord_p(\Norm(u-\theta)) \ge (n-1) c_0$,
%and there is at least on $\Z_p$-root $\alpha$ of 
%$f$ satisfying $\alpha \equiv u \pmod{p^t}$,
%then
%\[
%\cM_p \leftarrow \cM_p \cup \{ (\fb_u,\fp)\}.
%\]
\item Else 
\[\mathcal{U}^\prime \leftarrow \mathcal{U}^\prime \cup \{ u+p^{t}w : w \in \{0,\dotsc,p-1\} \}.\]
\end{enumerate}

\item If $\mathcal{U}^\prime \ne \emptyset$ then let
\[t \leftarrow t+1, \qquad \mathcal{U} \leftarrow \mathcal{U}^\prime,\]
and return to Step (3). Else output $L_p$, $M_p$.
\end{enumerate}
%\noindent \textbf{Output:} $\cL_p$, $\cM_p$.
\end{algorithm}

\begin{lemma} 
Algorithm~\ref{alg:AffinePatch2} terminates. 
\end{lemma}

\begin{proof}
Suppose that the algorithm does not terminate. Let $t_0=2$ and $t_i=t_0+i$ for $i \in \mathbb{N}$. Then there is an infinite sequence of congruence classes $\{u_i\}_{i = 0}^{\infty}$ and corresponding sets $\{\mathcal{P}_{u_i}\}_{i=0}^{\infty}$ such that $u_{i+1} \equiv u_i \mod{t_i}$ and $\mathcal{P}_{u_i} \ne \emptyset$ for all $i$. Moreover, $p \mid u_0$. Let $\alpha$ be the limit of $\{u_i\}_{i=0}^{\infty}$ in $\mathbb{Z}_p$. By the pigeon-hole principle, there is some ideal $\mathfrak{q}$ in $\mathcal{O}_K$ above $p$ which appears in infinitely many sets $\mathcal{P}_{u_i}$. It follows that $\ord_{\mathfrak{q}}(1 -u_i \theta) \ge t_i \cdot e(\mathfrak{q}|p)$ and thus $1-\alpha \theta=0$ in $K_{\mathfrak{q}}$. But as $p \mid u_0$, we have $\ord_p(\alpha) \ge 1$, and so $\ord_{\mathfrak{q}}(\theta)<0$. This contradicts the fact that $\theta$ is an algebraic integer. Therefore the algorithm must terminate.
\end{proof}

\begin{lemma}\label{lem:AffinePatch2Check}
Let $p \in \{p_1, \dots, p_v\}$ and let $L_p$, $M_p$ be as given by Algorithm~\ref{alg:AffinePatch2}. Let $(x,y,z_1,\dots, z_v)$ be a solution to \eqref{eq:ThueMahler3}. Then
\begin{itemize}
\item either there is some $\mathfrak{b} \in L_p$ such that \eqref{eq:Lp} is satisfied; 
\item or there is some $(\mathfrak{b},\mathfrak{p}) \in M_p$ with $e(\mathfrak{p}|p)=f(\mathfrak{p}|p)=1$ and integer $v_p \geq 0$ such that \eqref{eq:Mp} is satisfied.
\end{itemize}
\end{lemma}

\begin{proof}
Let $(x,y,z_1,\dots, z_v)$ be a solution to \eqref{eq:ThueMahler3}. In view of Lemma~\ref{lem:AffinePatch1Check} we may suppose $p \mid y$. Then $\ord_{\mathfrak{q}}(x) = 0$ and $\ord_{\mathfrak{q}}(x-y\theta)=\ord_{\mathfrak{q}}(1 - (y/x) \theta)$ for any prime ideal $\mathfrak{q}$ lying over $p$. The remainder of the proof is analogous to the proof of Lemma~\ref{lem:AffinePatch1Check}.
\end{proof}

%--------------------------------------------------------------------------------------------------------------------------------------------%

\subsection{Computational remarks and refinements}
\label{subsec:PIRLRemarks}

In implementing Algorithms~\ref{alg:AffinePatch1} and \ref{alg:AffinePatch2}, we reduce the number of prime ideals appearing in the factorization of $(x-y\theta)\mathcal{O}_K$ to a large power. The Prime Ideal Removing Lemma, as originally stated in Tzanakis - de Weger outlines a similar process by comparing the valuations of $(x-y\theta)\mathcal{O}_K$ at two prime ideals $\mathfrak{p}_1$ and $\mathfrak{p}_2$ above $p$. Of course if $\mathfrak{p}_1 \mid (x-y\theta)\mathcal{O}_K$, we restrict the possible values for $x$ and $y$ modulo $p$. However any choice of $x$ and $y$ modulo $p$ affects the valuations of $(x-y\theta)\mathcal{O}_K$ at all prime ideals above $p$. In the present refinement outlined by Lemma~\ref{lem:AffinePatch1} and Lemma~\ref{lem:AffinePatch2}, we instead study the valuations of $(x-y\theta)\mathcal{O}_K$ at all prime ideals above $p$ simultaneously. This presents us with considerably less ideal equations of the form \ref{eq:idealTM} to resolve. 

Moreover, this variant of the Prime Ideal Removing Lemma permits the following additional refinements:
\begin{itemize}
\item Let $\mathfrak{b} \in L_p$. If there exists a pair $(\mathfrak{b}^\prime,\mathfrak{p}) \in M_p$ with $\mathfrak{b}^\prime \mid \mathfrak{b}$ and $\mathfrak{b}/\mathfrak{b}^\prime=\mathfrak{p}^w$
for some $w \ge 0$, then we may delete $\mathfrak{b}$ from $L_p$. In doing so, the conclusion to Lemma~\ref{lem:AffinePatch2Check} continues to hold.
\item Suppose $(\mathfrak{b},\mathfrak{p})$, $(\mathfrak{b}^\prime,\mathfrak{p}) \in M_p$ with $\mathfrak{b}^{\prime} \mid \mathfrak{b}$, and $\mathfrak{b}/\mathfrak{b}^{\prime}=\mathfrak{p}^w$ for some ${w \geq 0}$. Then, we may delete $(\mathfrak{b},\mathfrak{p})$ from $M_p$ without affecting the conclusion to Lemma~\ref{lem:AffinePatch2Check}. 
\end{itemize}

%--------------------------------------------------------------------------------------------------------------------------------------------%
%--------------------------------------------------------------------------------------------------------------------------------------------%

\section{Factorization of the Thue-Mahler equation}
\label{sec:FactorizationTM}

After applying Algorithm~\ref{alg:AffinePatch1} and Algorithm~\ref{alg:AffinePatch2}, we are reduced to solving finitely many ideal equations of the form
\begin{equation}\label{eq:TMfactored}
(x-y\theta)\mathcal{O}_K=\mathfrak{a} \mathfrak{p}_1^{u_1}\cdots \mathfrak{p}_{\nu}^{u_{\nu}}
\end{equation}
in integer variables $x,y,u_1, \dots, u_{\nu}$ with $u_i \geq 0$ for $i = 1, \dots, \nu$, where ${0 \leq \nu \leq v}$. Here
\begin{itemize}
\item for $i \in \{1, \dots, \nu\}$, $\mathfrak{p}_i$ is a prime ideal of $\mathcal{O}_K$ arising from Algorithm~\ref{alg:AffinePatch1} and Algorithm~\ref{alg:AffinePatch2} applied to $p \in \{p_1, \dots, p_v\}$, such that $(\mathfrak{b}, \mathfrak{p}_i) \in M_p$ for some ideal $\mathfrak{b}$;
\item for $i \in \{\nu+1, \dots, v\}$, the corresponding rational prime $p_i \in S$ yields $M_{p_i} = \emptyset$, in which case we set $u_i = 0$;
\item $\mathfrak{a}$ is an ideal of $\mathcal{O}_K$ of norm $|c|\cdot p_1^{t_1} \cdots p_v^{t_v}$ such that
$u_i + t_i =  z_i$. 
\end{itemize}

For each choice of $\mathfrak{a}$ and prime ideals $\mathfrak{p}_1, \dots, \mathfrak{p}_{\nu}$, we reduce equation~\eqref{eq:TMfactored} to a number of so-called ``$S$-unit equations''. We present two different algorithms for doing so and outline the advantages and disadvantages of each. In practicality, we do not know a priori which of these two options is more efficient. Instead, we implement and use both algorithms simultaneously and selecting the most computationally efficient option as it appear. 

%--------------------------------------------------------------------------------------------------------------------------------------------%

\subsection{Avoiding the class group $\Cl(K)$}
\label{subsec:FactorizationTMwithoutOK}

For $i = 1, \dots, {\nu}$ let $h_i$ be the smallest positive integer for which $\mathfrak{p}_i^{h_i}$ is principal and let 
$r_i$ be a positive integer satisfying $0 \leq r_i < h_i$. Let
\[\mathbf{a}_i = (a_{1i}, \dots, a_{{\nu}i}).\]
where $a_{ii} = h_i$ and $a_{ji} = 0$ for $j \neq i$. We let $A$ be the matrix with columns $\mathbf{a}_1, \dots, \mathbf{a}_{\nu}$. Hence $A$ is a $\nu \times \nu$ diagonal matrix over $\mathbb{Z}$ with diagonal entries $h_i$. Now, if \eqref{eq:TMfactored} has a solution $\mathbf{u} = (u_1, \dots, u_{\nu})$, it necessarily must be of the form $\mathbf{u} = A\mathbf{n} + \mathbf{r}$, where $\mathbf{n} = (n_1, \dots, n_{\nu})$ and $\mathbf{r} = (r_1, \dots, r_{\nu})$. The vector $\mathbf{n}$ is comprised of integers $n_i$ which we solve for. The vector $\mathbf{r}$ is comprised of the values $r_i$ satisfying $0 \leq r_i < h_i$ for $i = 1, \dots, \nu$. 

Using the above notation, we let
\[\mathfrak{c}_i = \tilde{\mathfrak{p}}^{\mathbf{a}_i}=\mathfrak{p}_1^{a_{1i}}\cdot \mathfrak{p}_2^{a_{2i}} \cdots \mathfrak{p}_{\nu}^{a_{{\nu}i}} = \mathfrak{p}_i^{h_i} \]
for all $i \in \{1, \dots, {\nu}\}$.

Thus, we can write \eqref{eq:TMfactored} as
\[ (x-y\theta) \mathcal{O}_K = \mathfrak{a} \tilde{\mathfrak{p}}^{\mathbf{u}}  = (\mathfrak{a} \cdot \tilde{\mathfrak{p}}^\mathbf{r}) \cdot \mathfrak{c}_1^{n_1}\cdots \mathfrak{c}_{\nu}^{n_{\nu}}.\]

By definition of $h_i$, each $i \in \{1, \dots, {\nu}\}$ yields an element $\gamma_i \in K^*$ such that 
\[\mathfrak{c}_i = (\gamma_i) \mathcal{O}_K.\]
Furthermore, if $\mathbf{u}$ is a solution of \eqref{eq:TMfactored} with corresponding vectors $\mathbf{n}, \mathbf{r}$, there exists some $\alpha \in K^*$ such that 
\[\mathfrak{a} \cdot \tilde{\mathfrak{p}}^\mathbf{r}= (\alpha)\mathcal{O}_K.\]

%--------------------------------------------------------------------------------------------------------------------------------------------%

\subsection{Using the class group $\Cl(K)$}
\label{subsec:FactorizationTMwithOK}

Let $\mathbf{u}=(u_1,\dots, u_{\nu})$ be a solution of \eqref{eq:TMfactored} and consider the map
\[\phi : \mathbb{Z}^{\nu} \rightarrow \text{Cl}(K), \qquad (x_1,\dots ,x_{\nu}) \mapsto [\mathfrak{p}_1]^{x_1}\cdots [\mathfrak{p}_{\nu}]^{x_{\nu}},\]
where $[ \mathfrak{q} ]$ denotes the equivalence class of the fractional ideal $\mathfrak{q}$. 
Since the product of $\mathfrak{a}$ and $\mathfrak{p}_1^{u_1}\cdots \mathfrak{p}_{\nu}^{u_{\nu}}$ defines a principal ideal, the map $\phi$ implies
\[\phi(\mathbf{u})=[\mathfrak{a}]^{-1}.\]
In particular, if $[\mathfrak{a}]^{-1}$ does not belong to the image of $\phi$ then \eqref{eq:TMfactored} has no solutions. We therefore suppose that $[\mathfrak{a}]^{-1}$ belongs to the image. Let $\mathbf{r}=(r_1,\dotsc,r_{\nu})$ denote a preimage of $[\mathfrak{a}]^{-1}$ and observe that $\mathbf{u} - \mathbf{r}$ belongs to the kernel of $\phi$. The kernel is a subgroup of $\mathbb{Z}^v$ of rank $\nu$. Let $\mathbf{a}_1,\dots,\mathbf{a}_{\nu}$ be a basis for the kernel, where
\[\mathbf{a}_i = (a_{1i}, \dots, a_{\nu i}) \quad \text{ for } i = 1, \dots, \nu.\]
Let
\[\mathbf{u}-\mathbf{r}=n_1 \mathbf{a}_1+\cdots + n_{\nu} \mathbf{a}_{\nu}\]
for some integers $n_i \in \mathbb{Z}$ and let $A$ denote the $\nu \times \nu$ matrix over $\mathbb{Z}$ with columns $\mathbf{a}_1,\dots,\mathbf{a}_{\nu}$. It follows that $\mathbf{u}= A\mathbf{n}+\mathbf{r}$ where $\mathbf{n} = (n_1,\dots,n_{\nu})$.

For $\mathbf{a}_i=(a_{1i},\dotsc,a_{\nu i}) \in \mathbb{Z}^{\nu}$, we adopt the notation 
\[\tilde{\mathfrak{p}}^\mathbf{a} :=\mathfrak{p}_1^{a_{1i}}\cdot \mathfrak{p}_2^{a_{2i}} \cdots \mathfrak{p}_{\nu}^{a_{\nu i}}.\]
Let
\[\mathfrak{c}_1= \tilde{\mathfrak{p}}^{\mathbf{a}_1},\dotsc,\mathfrak{c}_{\nu}= \tilde{\mathfrak{p}}^{\mathbf{a}_{\nu}}.\]
Thus, we can rewrite \eqref{eq:TMfactored} as
\[(x-y\theta) \mathcal{O}_K = \mathfrak{a} \tilde{\mathfrak{p}}^{\mathbf{u}} = (\mathfrak{a} \cdot \tilde{\mathfrak{p}}^\mathbf{r}) \cdot \mathfrak{c}_1^{n_1}\cdots \mathfrak{c}_{\nu}^{n_{\nu}}.\]

Consider the ideal equivalence class of $(\mathfrak{a} \cdot \tilde{\mathfrak{p}}^\mathbf{r})$ in $\Cl(K)$ and note that
\[[\mathfrak{a} \cdot \tilde{\mathfrak{p}}^\mathbf{r}] 
	= [\mathfrak{a}] \cdot [\mathfrak{p}_1]^{r_1}\cdots [\mathfrak{p}_{\nu}]^{r_{\nu}} 
	= [\mathfrak{a}]\cdot \phi(r_1,\dotsc,r_{\nu})=[1]\]
as $\phi(r_1,\dotsc,r_{\nu})=[\mathfrak{a}]^{-1}$ by construction. This means 
\[\mathfrak{a} \cdot \tilde{\mathfrak{p}}^\mathbf{r}= (\alpha) \mathcal{O}_K\]
for some $\alpha \in K^*$. Furthermore, 
\[[\mathfrak{c}_i] = [\tilde{\mathfrak{p}}^{\mathbf{a}_i}] = \phi(\mathbf{a}_i) = [1] \quad \text{ for } i = 1, \dots, \nu,\]
as the $\mathbf{a}_i$ are a basis for the kernel of $\phi$. For all $i \in \{1, \dots, {\nu}\}$, we therefore have
\[\mathfrak{c}_i= (\gamma_i) \mathcal{O}_K\]
for some $\gamma_i \in K^*$.

%--------------------------------------------------------------------------------------------------------------------------------------------%

\subsection{The $S$-unit equation}
\label{subsec:SUnitEquation}

\autoref{subsec:FactorizationTMwithoutOK} and  \autoref{subsec:FactorizationTMwithOK} outline two different algorithms to reduce the ideal equation~\eqref{eq:TMfactored} to a number of certain ``$S$-unit equations'', which we define shortly. Regardless of which method we use, under both algorithms outlined above, equation~\eqref{eq:TMfactored} becomes
\begin{equation} \label{eq:TMprincipal}
(x-y\theta) \mathcal{O}_K= (\alpha \cdot \gamma_1^{n_1} \cdots \gamma_{\nu}^{n_{\nu}}) \mathcal{O}_K
\end{equation}
for some vector $\mathbf{n} = (n_1, \dots, n_{\nu}) \in \mathbb{Z}^{\nu}$. The ideal generated by $\alpha$ in $K$ has norm 
\[|c|\cdot p_1^{t_1 + r_1} \cdots p_{\nu}^{t_{\nu} + r_{\nu}}p_{\nu +1}^{t_{\nu +1}} \cdots p_v^{t_v}\]
and the $n_i$ are related to the $z_i$ via
\[z_i = u_i + t_i = \sum_{j = 1}^{\nu}n_ja_{ij} + r_i + t_i \quad \text{ for } i =1, \dots, v.\]
where $u_i = r_i = 0$ for all $i \in \{\nu + 1, \dots, v\}$. 

Fix a complete set of fundamental units $\{\eps_1, \dots, \eps_r\}$ of $\mathcal{O}_K$. Here $r = s + t -1$, where $s$ denotes the number of real embeddings of $K$ into $\mathbb{C}$ and $t$ denotes the number of complex conjugate pairs of non-real embeddings of $K$ into $\mathbb{C}$. Then, under either method, equation~\eqref{eq:TMfactored} reduces to a finite number of equations in $K$ of the form
\begin{equation} \label{eq:TMinK}
x-y\theta = \alpha \zeta \varepsilon_1^{a_1} \cdots \varepsilon_r^{a_r}\gamma_1^{n_1}\cdots \gamma_{\nu}^{n_{\nu}}
\end{equation}
with unknowns $a_i \in \mathbb{Z}$, $n_i \in \mathbb{Z}$, and $\zeta$ in the set $T$ of roots of unity in $\mathcal{O}_K$. Since $T$ is finite, we treat $\zeta$ as another parameter. 

Let $p \in \{p_1, \dots, p_v, \infty\}$. Recall that $g(t)$ is an irreducible polynomial in $\mathbb{Z}[t]$ arising from \eqref{eq:ThueMahler3} such that
\[g(t) = f(t,1) = t^n + C_1 t^{n-1} + \dots + C_{n-1}t + C_n.\]
Denote the roots of $g(t)$ in $\overline{\mathbb{Q}_p}$ (where $\overline{\mathbb{Q}_{\infty}} = \overline{\mathbb{R}} = \mathbb{C}$) by $\theta^{(1)}, \dots, \theta^{(n)}$. Let $i_0, j, k \in \{1,\dots, n\}$ be distinct indices and consider the three embeddings of $K$ into $\overline{\mathbb{Q}_p}$ defined by $\theta \mapsto \theta^{(i_0)}, \theta^{(j)}, \theta^{(k)}$. We use $z^{(i)}$ to denote the image of $z$ under the embedding $\theta \mapsto \theta^{(i)}$. From the Siegel identity
\[(\theta^{(i_0)} - \theta^{(j)})(x-y\theta^{(k)}) + (\theta^{(j)} - \theta^{(k)})(x-y\theta^{(i_0)}) + (\theta^{(k)} - \theta^{(i_0)})(x-y\theta^{(j)}) = 0,\]
applying the embeddings to $\beta = x-y\theta$ yields the so-called ``$S$-unit equation''
\begin{equation} \label{eq:Sunit}
\delta_1 \prod_{i = 1}^r\left( \frac{\varepsilon_i^{(k)}}{\varepsilon_i^{(j)}}\right)^{a_i}\prod_{i = 1}^{\nu} \left( \frac{\gamma_i^{(k)}}{\gamma_i^{(j)}}\right)^{n_i} - 1 = \delta_2 \prod_{i = 1}^{r}\left( \frac{\varepsilon_i^{(i_0)}}{\varepsilon_i^{(j)}}\right)^{a_i} \prod_{i = 1}^{\nu} \left( \frac{\gamma_i^{(i_0)}}{\gamma_i^{(j)}}\right)^{n_i},
\end{equation}
where
\[\delta_1 = \frac{\theta^{(i_0)} - \theta^{(j)}}{\theta^{(i_0)} - \theta^{(k)}}\cdot\frac{\alpha^{(k)}\zeta^{(k)}}{\alpha^{(j)}\zeta^{(j)}}, \quad \delta_2 = \frac{\theta^{(j)} - \theta^{(k)}}{\theta^{(k)} - \theta^{(i_0)}}\cdot \frac{\alpha^{(i_0)}\zeta^{(i_0)}}{\alpha^{(j)}\zeta^{(j)}}\]
are constants. 

To summarize, our original problem of solving \eqref{eq:ThueMahler3} for $(x,y,z_1,\dots, z_v)$ has been reduced to solving finitely many equations of the form \eqref{eq:Sunit} for the variables $(x,y, n_1, \dots, n_{\nu},a_1,\dots,a_r)$.

%--------------------------------------------------------------------------------------------------------------------------------------------%

\subsection{Computational remarks and comparisons}
\label{subsec:FactorizationRemarks}

In \autoref{subsec:FactorizationTMwithoutOK}, we follow closely the method of \cite{TW3} to reduce the ideal equation~\eqref{eq:TMfactored} to the $S$-unit equation~\eqref{eq:Sunit}. To implement this reduction, we begin by computing all $h_i$ for which $\mathfrak{p}_i^{h_i}$ is principal for $i = 1, \dots, \nu$. In doing so, we generate all possible values for $r_i$, the non-negative integer satisfying $0 \leq r_i < h_i$. We then generate every possible vector $\mathbf{r} = (r_1, \dots, r_{\nu})$ and test the corresponding ideal product $\mathfrak{a} \cdot \tilde{\mathfrak{p}}^{\mathbf{r}}$ for principality. Those vectors which pass this test yield an $S$-unit equation~\eqref{eq:Sunit}. In the worst case scenario, this method reduces to $h_K^{\nu}$ such equations, where $h_K$ is the class number of $K$. Moreover, this process needs to be applied to every ideal equation~\eqref{eq:TMfactored}, yielding what may be a very large number of principalization tests and subsequent large number of $S$-unit equations to solve. 

In contrast, the method in \autoref{subsec:FactorizationTMwithOK} reduces \eqref{eq:TMfactored} to only $\#T/2$ $S$-unit equations to solve, where $T$ is the set of roots of unity in $K$. In particular, the sum total of $S$-unit equations does not drastically increase. If $[\mathbf{a}]^{-1}$ is not in the image of $\phi$, we reach a contradiction. If $[\mathbf{a}]^{-1}$ is in the image of $\phi$ then we obtain only $\#T/2$ corresponding equations \eqref{eq:Sunit}. In particular, the number of principalization tests in this method is limited by the number of ideal equations~\eqref{eq:TMfactored}, where each such equation yields only $(1+\nu)$ tests. 

However, when generating the vectors $\mathbf{r} = (r_1, \dots, r_{\nu})$ using the class group, we observe that some of the integers $r_i$ may be negative, so we do not expect $\alpha$ to be an algebraic integer in general. This can be problematic later in the algorithm when we compute the embedding of $K$ into our $p$-adic fields. In those instances, the precision on our $p$-adic fields may not be high enough, and as a result, some non-zero elements of $K$ may be erroneously mapped to $0$. To avoid this, we force the $r_i$ to be positive by adding sufficiently many multiples of the class number. 

In most cases, the method described in \autoref{subsec:FactorizationTMwithOK} is far more efficient than that of \autoref{subsec:FactorizationTMwithoutOK}. However, computing the class group may be a very costly computation. Indeed, for some Thue-Mahler equations, this may be the bottle-neck of the algorithm. In this case, it may happen that computing the class group will take longer than directly checking each potential $S$-unit equation arising from the alternative method. Unfortunately, we cannot know a piori how long computing $\Cl(K)$ will take in so much that we cannot know a priori how long solving all $S$-unit equations from the other algorithm will take. In practicality, generating the class group in Magma is a process which cannot be terminated without exiting the program. For this reason, we cannot simply apply a timeout in Magma if computing $\Cl(K)$ is exceeding what we deem a reasonable amount of time. Adding to this, Magma does not support parallelization, so we cannot implement both algorithms simultaneously. Our compromise to solve a single Thue-Mahler equation is to run two separate instances of Magma in parallel, each generating the $S$-unit equations using the two aforementioned algorithms. When one of these instances finishes, the other is forced to terminate. In this way, though far from ideal, we are able to select the most computationally efficient option. 

%--------------------------------------------------------------------------------------------------------------------------------------------%
%--------------------------------------------------------------------------------------------------------------------------------------------%

\section{A small upper bound for $u_l$ in a special case}
\label{sec:SmallBoundForSpecialCase}

We now restrict our attention to those $p \in \{p_1, \dots, p_{\nu}\}$ and study the $p$-adic valuations of the numbers appearing in \eqref{eq:Sunit}. In particular, for $l \in \{1, \dots, \nu\}$, we identify conditions in which $\sum_{j = 1}^{\nu} n_ja_{lj}$ can be bounded by a small explicit constant, where $a_{lj}$ is the $(l,j)^{\text{th}}$ entry of the matrix $A$ derived in either \autoref{subsec:FactorizationTMwithoutOK} or \autoref{subsec:FactorizationTMwithOK}. Recall that $u_l + r_l = \sum_{j = 1}^{\nu} n_ja_{lj}$, where $r_l$ is known, so that a bound on $\sum_{j = 1}^{\nu} n_ja_{lj}$ yields a bound on the exponent $u_l$ in \eqref{eq:TMfactored}.

Fix a rational prime $p_l \in \{p_1, \dots, p_{\nu}\}$ and recall that $z \in \mathbb{C}_{p_l}$ having $\ord_{p_l}(z) = 0$ is called a $p_l$-adic unit. Part (i) of the Corollary of Lemma 7.2 of \cite{TW3} tells us that $\frac{\eps_1^{(i_0)}}{\varepsilon_1^{(j)}}, \dots, \frac{\eps_r^{(i_0)}}{\varepsilon_r^{(j)}}$ and $\frac{\varepsilon_1^{(k)}}{\varepsilon_1^{(j)}}, \dots, \frac{\varepsilon_r^{(k)}}{\varepsilon_r^{(j)}}$ are $p_l$-adic units. 

Let $g_l(t)$ be the irreducible factor of $g(t)$ in $\mathbb{Q}_{p_l}[t]$ corresponding to the prime ideal $\mathfrak{p}_l$. Since $\mathfrak{p}_l$ has ramification index and residue degree equal to $1$, $\deg(g_l(t)) = 1$. We now choose $i_0 \in \{1, \dots, 4\}$ so that $\theta^{(i_0)}$ is the root of $g_l(t)$. We fix this choice of index $i_0$ for the remainder of this chapter. The indices of $j,k$ are fixed, but arbitrary. 

\begin{lemma} \label{lem:SunitUnits} \
\begin{enumerate}
\item[(i)] Let $i \in \{1, \dots, \nu\}$. Then $\frac{\gamma_i^{(k)}}{\gamma_i^{(j)}}$ are $p_l$-adic units. 
\item[(ii)] Let $i \in \{1, \dots, \nu\}$. Then $\ord_{p_l}\left(\frac{\gamma_i^{(i_0)}}{\gamma_i^{(j)}}\right) = a_{li}$, where $\mathbf{a_i} = (a_{1i}, \dots, a_{vi})$ is the $i^{\text{th}}$ column of the matrix $A$ of either \autoref{subsec:FactorizationTMwithoutOK} or \autoref{subsec:FactorizationTMwithOK}. 
\end{enumerate}
\end{lemma}

\begin{proof}
Consider the factorization $g(t) = g_1(t) \cdots g_m(t)$ of $g(t)$ in $\mathbb{Q}_{p_l}[t]$. Note that $\theta^{(j)}$ is a root of some $g_h(t) \neq g_l(t)$. Let $\mathfrak{p}_h$ be the corresponding prime ideal above $p_l$ and $e(\mathfrak{p}_h|p_l)$ be its ramification index. Then $\mathfrak{p} \neq \mathfrak{p}_l$ and since 
\[(\gamma_i)\mathcal{O}_K = \mathfrak{p}_1^{a_{1i}} \cdots \mathfrak{p}_v^{a_{vi}},\]
we have 
\[\ord_{p_l}(\gamma_i^{(j)}) = \frac{1}{e(\mathfrak{p}_h|p_l)}\ord_{\mathfrak{p}_h}(\gamma_i) = 0.\]
An analogous argument gives $\ord_{p_l}(\gamma_i^{(k)}) = 0$. On the other hand, 
\[\ord_{p_l}(\gamma_i^{(i_0)}) = \frac{1}{e(\mathfrak{p}_l|p_l)}\ord_{\mathfrak{p}_l}(\gamma_i) = \ord_{\mathfrak{p}_l}(\mathfrak{p}_1^{a_{1i}} \cdots \mathfrak{p}_v^{a_{vi}}) = a_{li}.\]
\end{proof}

The next lemma deals with a special case in which the sum $\sum_{j = 1}^{\nu} n_ja_{lj}$ can be computed directly. This lemma is analogous to Lemma 7.3 of \cite{TW3}.

Recall the constants
\[\delta_1 = \frac{\theta^{(i_0)} - \theta^{(j)}}{\theta^{(i_0)} - \theta^{(k)}}\cdot\frac{\alpha^{(k)}\zeta^{(k)}}{\alpha^{(j)}\zeta^{(j)}}, \quad \delta_2 = \frac{\theta^{(j)} - \theta^{(k)}}{\theta^{(k)} - \theta^{(i_0)}}\cdot \frac{\alpha^{(i_0)}\zeta^{(i_0)}}{\alpha^{(j)}\zeta^{(j)}}\]
of \eqref{eq:Sunit}.
\begin{lemma}\label{lem:Delta1}
Let $l \in \{1, \dots, v\}$. If $\ord_{p_l}(\delta_1) \neq 0$, then 
\[ \sum_{i = 1}^{\nu} n_ia_{li} = \min\{\ord_{p_l}(\delta_1), 0\} - \ord_{p_l}(\delta_2).\]
\end{lemma}

\begin{proof}
Apply the Corollary of Lemma $7.2$ of \cite{TW3} and Lemma~\ref{lem:SunitUnits} to both expressions of $\lambda$ in \eqref{eq:Sunit}. On the one hand, we obtain that $\ord_{p_l}(\lambda) = \min\{\ord_{p_l}(\delta_1), 0\}$, and on the other hand, 
\begin{align*}
\ord_{p_l}(\lambda)
& = \ord_{p_l}(\delta_2) + \sum_{i = 1}^{\nu} \ord_{p_l}\left( \frac{\gamma_i^{(i_0)}}{\gamma_i^{(j)}}\right)^{n_i}\\
& = \ord_{p_l}(\delta_2) + \sum_{i = 1}^{\nu} n_ia_{li}.
\end{align*}
\end{proof}

For the remainder of this section, we assume $\ord_{p_l}(\delta_1) = 0$. Here, it is convenient to use the notation
\[b_1 = 1, \quad b_{1+i} = n_i \ \text{ for } i \in \{1, \dots, \nu\},\] 
and
\[ b_{1+{\nu}+i} = a_i \ \text{ for } i  \in \{1, \dots, r\}.\]
Put
\[\alpha_1 = \log_{p_l} \delta_1, \quad \alpha_{1+i} = \log_{p_l}\left( \frac{\gamma_i^{(k)}}{\gamma_i^{(j)}}\right)  \ \text{ for } i \in \{1, \dots, \nu\},\]
and
\[\alpha_{1+\nu+i} = \log_{p_l}\left( \frac{\varepsilon_i^{(k)}}{\varepsilon_i^{(j)}}\right) \ \text{ for } i  \in \{1, \dots, r\}.\]
Define
\[\Lambda_l = \sum_{i = 1}^{1+\nu+r} b_i\alpha_i.\]

Let $L$ be a finite extension of $\mathbb{Q}_{p_l}$ containing $\delta_1$, $\frac{\gamma_1^{(k)}}{\gamma_1^{(j)}}, \dots, \frac{\gamma_{\nu}^{(k)}}{\gamma_{\nu}^{(j)}}$, and $\frac{\varepsilon_1^{(k)}}{\varepsilon_1^{(j)}}, \dots, \frac{\varepsilon_r^{(k)}}{\varepsilon_r^{(j)}}$. Since finite $p_l$-adic fields are complete, $\alpha_i \in L$ for $i = 1, \dots, 1+\nu+r$ as well. Choose $\phi \in \overline{\mathbb{Q}_{p_l}}$ such that $L = \mathbb{Q}_{p_l}(\phi)$ and $\ord_{p_l}(\phi) > 0 $. Let $G(t)$ be the minimal polynomial of $\phi$ over $\mathbb{Q}_{p_l}$ and let $s$ be its degree. For $i = 1, \dots, 1+\nu+r$ write
\[\alpha_i = \sum_{h = 1}^s \alpha_{ih}\phi^{h - 1}, \quad \alpha_{ih} \in \mathbb{Q}_{p_l}.\]
Then
\begin{equation} \label{eq:LambdaL}
\Lambda_l = \sum_{h = 1}^s \Lambda_{lh}\phi^{h-1},
\end{equation}
with
\[\Lambda_{lh} = \sum_{i = 1}^{1+\nu+r} b_i \alpha_{ih}\]
for $h = 1, \dots, s$. 

\begin{lemma}\label{lem:DiscG}
For every $h \in \{1, \dots, s\}$, we have
\[\ord_{p_l}(\Lambda_{lh}) > \ord_{p_l}(\Lambda_l) - \frac{1}{2}\ord_{p_l}(\text{Disc}(G(t))).\]
\end{lemma}

\begin{proof}
Taking the images of \eqref{eq:LambdaL} under conjugation $\phi \mapsto \phi^{(h)}$ ($h = 1, \dots, s$) yields
\[\begin{bmatrix}
\Lambda_l^{(1)} \\
\vdots \\
\Lambda_l^{(s)}	\\
\end{bmatrix}
=
\begin{bmatrix}
1 		& \phi^{(1)} 	& \cdots 	& \phi^{(1)s-1}\\
\vdots 	& \vdots 		& 		& \vdots \\
1 		& \phi^{(s)} 	& \cdots  	& \phi^{(s)s-1}\\
\end{bmatrix}
\begin{bmatrix}
\Lambda_{l1}\\
\vdots \\
\Lambda_{ls}\\
\end{bmatrix}\]
The $s \times s$ matrix $(\phi^{(h)i-1})$ above is invertible, with inverse
\[\frac{1}{\displaystyle \prod_{1\leq j<k\leq s} (\phi^{(k)} - \phi^{(j)})}
\begin{bmatrix}
\gamma_{11} 	& \cdots 	& \gamma_{1s}\\
\vdots 		& 		& \vdots\\
\gamma_{s1} 	& \cdots 	& \gamma_{ss}\\
\end{bmatrix},\]
where $\gamma_{jk}$ is an integral polynomial in the entries of $(\phi^{(h)i-1})$. Since $\ord_{p_l}(\phi) > 0$ and $\ord_{p_l}(\phi^{(h)}) = \ord_{p_l}(\phi)$ for all $h = 1, \dots, s$, it follows that $\ord_{p_l}(\gamma_{jk}) > 0 $ for every $\gamma_{jk}$. Therefore, as 
\[\Lambda_{lh} = \frac{1}{\displaystyle \prod_{1\leq j<k\leq s}(\phi^{(k)} - \phi^{(j)})}\sum_{i = 1}^s \gamma_{hi}\Lambda_l^{(i)},\]
we have 
\begin{align*}
\ord_{p_l}(\Lambda_{lh}) 
	& = \min_{1 \leq i \leq s} \left\{\ord_{p_l}(\gamma_{hi}) + \ord_{p_l}(\Lambda_l^{(i)})\right\} -\frac{1}{2}\ord_{p_l}(\text{Disc}(G(t)))\\
	& \geq \min_{1 \leq i \leq s} \ord_{p_l}(\Lambda_l^{(i)}) +  \min_{1 \leq i \leq s} \ord_{p_l}(\gamma_{hi}) - \frac{1}{2}\ord_{p_l}(\text{Disc}(G(t)))\\
	& = \ord_{p_l}\Lambda_l + \min_{1 \leq i \leq s} \ord_{p_l}(\gamma_{hi}) - \frac{1}{2}\ord_{p_l}(\text{Disc}(G(t)))
\end{align*}
for every $h \in \{1, \dots, s\}$. 
%\min_{1 \leq i \leq s} \left\{\ord_{p_l}(\gamma_{hi}) + \ord_{p_l}(\Lambda_l^{(i)}) -\frac{1}{2}\ord_{p_l}(\text{Disc}(G(t)))\right\}\]
\end{proof}

\begin{lemma} \label{lem:Lambda}
If 
\[\sum_{i = 1}^{\nu} n_{i}a_{li} > \frac{1}{p_l-1} - \ord_{p_l}(\delta_2),\]
then
\[\ord_{p_l}(\Lambda_l) = \sum_{i = 1}^{\nu} n_{i}a_{li} + \ord_{p_l}(\delta_2).\]
\end{lemma}

\begin{proof}
Immediate from Lemma~\ref{lem:pAdicLogarithms2}.
\end{proof}

\begin{lemma} \label{lem:specialcase} \
\begin{enumerate}[(i)]
\item If $\ord_{p_l}(\alpha_1) < \displaystyle \min_{2 \leq i \leq 1+\nu+r} \ord_{p_l}(\alpha_i)$, then
\[\sum_{i = 1}^{\nu} n_i a_{li} \leq \max \left\{ \bigg\lfloor{\frac{1}{p_l-1} - \ord_{p_l}(\delta_2)}\bigg\rfloor,  \bigg \lceil\displaystyle \min_{2 \leq i \leq 1+\nu+r} \ord_{p_l}(\alpha_{i}) - \ord_{p_l}(\delta_2) \bigg \rceil - 1 \right\}\]
\item For all $h \in \{1, \dots, s\}$, if $\ord_{p_l}(\alpha_{1h}) < \displaystyle \min_{2 \leq i \leq 1+\nu+r} \ord_{p_l}(\alpha_{ih})$, then
\[\sum_{i = 1}^{\nu} n_i a_{li} \leq \max \left\{ \bigg\lfloor{\frac{1}{p_l-1} - \ord_{p_l}(\delta_2)}\bigg\rfloor, \bigg \lceil \displaystyle \min_{2 \leq i \leq 1+\nu+r} \ord_{p_l}(\alpha_{ih})- \ord_{p_l}(\delta_2) + w_l \bigg \rceil - 1\right\},\]
where 
\[w_l = \frac{1}{2}\ord_{p_l}(\text{Disc}(G(t))).\]
\end{enumerate}
\end{lemma}

\begin{proof} \
\begin{enumerate}[(i)]
\item We prove the contrapositive. Suppose
\[\sum_{i = 1}^{\nu} n_i a_{li} > \frac{1}{p_l-1} - \ord_{p_l}(\delta_2), \]
and
\[\sum_{i = 1}^{\nu} n_i a_{li}  \geq \displaystyle \min_{2 \leq i \leq 1+\nu+r} \ord_{p_l}(\alpha_{i}) - \ord_{p_l}(\delta_2).\]
Observe that
\begin{align*}
\ord_{p_l}(\alpha_{1}) 	
	& = \ord_{p_l}\left( \Lambda_{l} - \sum_{i = 2}^{1+\nu+r}b_i\alpha_{i}\right) \\
	& \geq \min\left\{ \ord_{p_l}(\Lambda_{l}), \min_{2 \leq i \leq 1+\nu+r} \ord_{p_l}(b_i\alpha_{i})\right\}.
\end{align*}
Therefore, it suffices to show that 
\[\ord_{p_l}(\Lambda_{l}) \geq \min_{2 \leq i \leq 1+\nu+r} \ord_{p_l}(b_i\alpha_{i}).\]
By Lemma~\ref{lem:pAdicLogarithms2}, the first inequality implies ${\ord_{p_l}(\Lambda_{l}) = \displaystyle \sum_{i = 1}^{\nu} n_ia_{li} + \ord_{p_l}(\delta_2)}$, from which the result follows. 

\item Similar to the proof of (i).
%\item[(ii)] We prove the contrapositive. Let $h \in \{1, \dots, s\}$ and suppose
%\[\sum_{i = 1}^v n_i a_{li} > \frac{1}{p-1} - \ord_{p_l}(\delta_2), \]
%and
%\[\sum_{i = 1}^v n_i a_{li}  \geq \nu_l + \displaystyle \min_{2 \leq i \leq v+2} \ord_{p_l}(\alpha_{ih}) - \ord_{p_l}(\delta_2).\]
%Observe that 
%\[\begin{split}
%\ord_{p_l}(\alpha_{1h}) 	
%	& = \ord_{p_l}\left( \Lambda_{lh} - \sum_{i = 2}^{v+2}b_i\alpha_{ih}\right) \\
%	& \geq \min\left\{ \ord_{p_l}(\Lambda_{lh}), \min_{2 \leq i \leq v+2} \ord_{p_l}(b_i\alpha_{ih})\right\}
%\end{split}\]
%Therefore, it suffices to show that 
%\[\ord_{p_l}(\Lambda_{lh}) \geq \min_{2 \leq i \leq v+2} \ord_{p_l}(b_i\alpha_{ih}).\]
%By Lemma~\ref{Lem:padic}, the first inequality implies $\ord_{p_l}(\Lambda_{l}) = \displaystyle \sum_{i = 1}^v n_ia_{li} + \ord_{p_l}(\delta_2)$. Combining this with Lemma~\ref{Lem:discG} yields
%\[\ord_{p_l}(\Lambda_{lh}) \geq \displaystyle \sum_{i = 1}^v n_ia_{li} + \ord_{p_l}(\delta_2) - \nu_l.\]
%The results now follow from our second assumption. 
\end{enumerate}
\end{proof}

%--------------------------------------------------------------------------------------------------------------------------------------------%
%--------------------------------------------------------------------------------------------------------------------------------------------%

\section{Lattice-Based Reduction}
\label{sec:LatticeReduction}

At this point in solving the Thue-Mahler equation, we proceed to solve each $S$-unit equation~\eqref{eq:Sunit} for the exponents $(n_1, \dots, n_{\nu}, a_1, \dots, a_r)$. To do so, we generate a very large upper bound on the exponents and reduce this bound via Diophantine approximation computations. The specific details of this process are described in \autoref{ch:EfficientTMSolver} and \autoref{ch:Goormaghtigh}. In general, from each $S$-unit equation, we generate several linear forms in logarithms to which we associate an integral lattice $\Gamma$. It will be important in this reduction process to enumerate all short vectors in these lattices. In this section, we describe two algorithms used in the short vector enumeration process. 

%--------------------------------------------------------------------------------------------------------------------------------------------%

\subsection{The $L^3$-lattice basis reduction algorithm}
\label{subsec:LLL}

Let $\Gamma$ be an $n$-dimensional lattice with basis vectors $\mathbf{b}_1, \dots, \mathbf{b}_n$ equipped with a bilinear form $\Phi: \Gamma \times \Gamma \to \mathbb{Z}$. Recall that $\Phi$ defines a norm on $\Gamma$ via the usual inner product on $\mathbb{R}^n$. For $i = 1, \dots, n$, define the vectors $\mathbf{b}_i^*$ inductively by
\[\mathbf{b}_i^* = \mathbf{b}_i - \sum_{j=1}^{i-1}\mu_{ij}\mathbf{b}_j^*, \quad \mu_{ij} = \frac{\Phi(\mathbf{b}_i,\mathbf{b}_j^*)}{\Phi(\mathbf{b}_j^*,\mathbf{b}_j)},\]
where $\mu_{ij} \in \mathbb{R}$ for $1\leq j < i \leq n$. This is the usual Gram-Schmidt process. The basis $\mathbf{b}_1,\dots, \mathbf{b}_n$ is called \textit{LLL-reduced} if
\[|\mu_{ij}| \leq \frac{1}{2} \quad \text{ for } 1\leq j < i \leq n, \]
\[\frac{3}{4}|\mathbf{b}_{i-1}^*|^2 \leq |\mathbf{b}_i^* + \mu_{ii-1}\mathbf{b}_{i-1}^*|^2 \quad \text{ for } 1 <i \leq n,\]
where $| \cdot |$ is the usual Euclidean norm in $\mathbb{R}^n$, 
\[|\mathbf{v}| = \Phi(\mathbf{v},\mathbf{v}) = \mathbf{v}^{T}\mathbf{v}.\]

These properties imply that an LLL-reduced basis is approximately orthogonal, and that, generically, its constituent vectors are roughly of the same length. Every $n$-dimensional lattice has an LLL-reduced basis and such a basis can be computed very quickly using the so-called LLL algorithm (\cite{LLL}). This algorithm takes as input an arbitrary basis for a lattice and outputs an LLL-reduced basis. The algorithm is typically modified to additionally output a unimodular matrix $U$ such that $A = BU$, where $B$ is the matrix whose column-vectors are the input basis and $A$ is the matrix whose column-vectors are the LLL-reduced output basis. Several versions of this algorithm are implemented in Magma, including de Weger's exact integer version. (\cite{Weg0}).

We remark that a lattice may have more than one reduced basis, and that the ordering of the basis vectors is not arbitrary. The properties of reduced bases that are of most interest to us are the following. Let $\mathbf{v}$ a vector in $\mathbb{R}^n$ and denote by $l(\Gamma,\mathbf{v})$ the distance from $\mathbf{v}$ to the nearest point in the lattice $\Gamma$, viz.
\[l(\Gamma,\mathbf{v}) = \min_{\mathbf{u} \in \Gamma \backslash\{\mathbf{v}\}} |\mathbf{u} - \mathbf{v}|.\]
From an LLL-reduced basis for $\Gamma$, we can compute lower bounds for $l(\Gamma,\mathbf{v})$, according to the following results. 

\begin{lemma} \label{lem:LLL}
Let $\Gamma$ be a lattice with LLL-reduced basis $\mathbf{c}_1, \dots, \mathbf{c}_n$ and let $\mathbf{v}$ be a vector in $\mathbb{R}^n$. 
\begin{enumerate}[(a)]
\item If $\mathbf{v} = \mathbf{0}$, then $l(\Gamma,\mathbf{v}) \geq 2^{-(n-1)/2}|\mathbf{c}_1|$.
\item Assume $\mathbf{v} = s_1\mathbf{c}_1 + \cdots + s_n \mathbf{c}_n$, where $s_1, \dots, s_n \in \mathbb{R}$ with not all $s_i \in \mathbb{Z}$. Put 
\[J = \{j \in \{1, \dots, n\} \ : \ s_j \notin \mathbb{Z} \}.\]
For $j \in J$, set 
\[\delta(j) = 
\begin{cases}
\max_{i > j} \|s_i \| |\mathbf{c}_i| 	& \text{ if } j < n\\
0 							& \text{ if } j = n,
\end{cases}\]
where $\| \cdot \|$ denotes the distance to the nearest integer. We have
\[l(\Gamma,\mathbf{v}) \geq \max_{j \in J}\left(2^{-(n-1)/2}\| s_j\| |\mathbf{c}_1| - (n-j)\delta(j)\right).\]
\end{enumerate}
\end{lemma}
Lemma~\ref{lem:LLL} (a) is Proposition 1.11 in \cite{LLL}; proofs can be found in \cite{LLL}, \cite{Weg0} (Section 3.4), or \cite{Sm} (Section V.3). Lemma~\ref{lem:LLL} (b) is a combination of Lemmas 3.5 and 3.6 in \cite{Weg0}. Note that the assumption in Lemma~\ref{lem:LLL} (b) is equivalent to ${\mathbf{v} \notin \Gamma}$. 

We see that the vector $\mathbf{c}_1$ in a reduced basis is, in a very precise sense, not too far from being the shortest non-zero vector of $\Gamma$. As has already been mentioned, what makes this result so valuable is that there is a very simple and efficient algorithm to find a reduced basis in a lattice, namely the LLL algorithm.

%--------------------------------------------------------------------------------------------------------------------------------------------%

\subsection{The Fincke-Pohst algorithm}
\label{subsec:FinckePohst}

Sometimes it is not sufficient to have a lower bound for $l(\Gamma,\mathbf{v})$ only. It may be useful to know exactly all vectors $\mathbf{u} \in \Gamma$ such that $|\mathbf{u}|  = \Phi(\mathbf{u}, \mathbf{u}) \leq C$ for a given constant $C$. This can be done efficiently using an algorithm of Fincke-Pohst (cf. \cite{FP}, \cite{Coh1}). A version of this algorithm with some improvements due to Stehl\'e is implemented in Magma. As input this algorithm takes a matrix $B$, whose columns span the lattice $\Gamma$, and a constant $C > 0$. The output is a list of all lattice points $\mathbf{u} \in \Gamma$ with $|\mathbf{u}| \leq C$, apart from $\mathbf{u} = \mathbf{0}$. In this section, we outline the main steps in this algorithm. 

We begin by letting $B$ denote the basis matrix associated to the lattice $\Gamma$, with corresponding bilinear form $\Phi$. We call a vector $\mathbf{u} \in \Gamma$ \textit{small} if its norm $\Phi(\mathbf{u}, \mathbf{u})$ is less than a constant $C$. As an element of the lattice, $\mathbf{u} = B\mathbf{x}$ for some coordinate vector $\mathbf{x} \in \mathbb{Z}^n$. Let $Q$ be the quadratic form associated to $\Phi$ and let $A=B^TB$. Now finding the short vectors $\mathbf{u} \in \Gamma$ is equivalent to solving 
\begin{equation} \label{eq:ShortVector}
Q(\mathbf{x}) = \mathbf{x}^TA\mathbf{x} \leq C.
\end{equation}

Let $\mathbf{x} = (x_1, \dots, x_n)$. To solve this inequality, we first rearrange the terms of the quadratic form via quadratic completion. Here we assume that $\Gamma$ is positive definite so that every nonzero element of the lattice has a positive norm. With this, we find the Cholesky decomposition $A = R^TR$, where $R$ is an upper triangular matrix, and express $Q$ as
\[ Q(\mathbf{x}) = \sum_{i=1}^n q_{ii}\left( x_i + \sum_{j=i+1}^n q_{ij}x_j\right)^2.\]
The coefficients $q_{ij}$ are defined from $R$ and stored in a matrix $\tilde{Q}$ for convenience. In particular, 
\begin{equation} \label{eq:CholeskyCoeffs}
q_{ij} =
\begin{cases}
\frac{r_{ij}}{r_{ii}} & \text{ if } i < j\\
r_{ii}^2 & \text{ if } i = j.
\end{cases}
\end{equation}
Since $R$ is upper triangular, the matrix $\tilde{Q}$ is as well. This yields the following reformulation of \eqref{eq:ShortVector}
\[ \sum_{i=1}^n q_{ii}\left( x_i + \sum_{j=i+1}^n q_{ij}x_j\right)^2 \leq C.\]
From here we observe that the individual term $q_{nn}x_n^2$ must also be less than $C$. Specifically, 
\[x_n^2 \leq \frac{C}{q_{nn}}\]
so that $x_n$ is bounded above by $\sqrt{C/q_{nn}}$ and below by $-\sqrt{C/q_{nn}}$. This illustrates the first step in establishing bounds on a specific entry $x_i$. Adding more terms from the outer sum to this sequence, a pattern emerges. Let
\[U_k = \sum_{j = k+1}^n q_{kj}x_j,\]
where $U_n = 0$, and rewrite $Q(\mathbf{x})$ as 
\[Q(\mathbf{x}) = \sum_{i=1}^n q_{ii}\left( x_i + \sum_{j=i+1}^n q_{ij}x_j\right)^2 = \sum_{i=1}^n q_{ii}\left( x_i + U_i\right)^2.\]
In general, 
\[q_{kk}(x_k + U_k)^2 \leq C - \sum_{i = k+1}^n q_{ii}(x_i + U_i)^2.\]
Let $T_k$ denote the bound on the right-hand side, 
\[T_k = C - \sum_{i = k+1}^n q_{ii}(x_i + U_i)^2.\]
We set $T_n = C$ and find each subsequent $T_k$ by subtracting the next term from the outer summand,
\[T_k = T_{k+1} - q_{k+1,k+1}(x_{k+1} + U_{k+1})^2.\]
This yields the upper bound
\[q_{kk}(x_k + U_k)^2 \leq T_k\]
so that $x_k$ is bounded above by $\sqrt{T_k/q_{kk}} - U_k$ and below by ${-\sqrt{T_k/q_{kk}} - U_k}$. In this way, we iteratively enumerate all vectors $\mathbf{x}$ satisfying $Q(\mathbf{x}) \leq C$, beginning with the entry $x_n$ of $\mathbf{x}$ and working down towards $x_1$.  

%--------------------------------------------------------------------------------------------------------------------------------------------%

\subsection{Computational remarks and translated lattices}
\label{subsec:FinckePohstRemarks}

Recall that the Cholesky decomposition of $A = B^TB$ yields the upper triangular matrix $R$ where $A = R^TR$. It is noted in the \cite{FP} that if we label the columns of $R$ by $\mathbf{r}_i$ and the rows of $R^{-1}$ by $\mathbf{r}'_i$, then 
\[x_k^2 = \left( \mathbf{r}'^{\ T}_k \cdot \sum_{i=1}^n x_i \mathbf{r}_i \right)^2 \leq \mathbf{r}'^{\ T}_k \mathbf{r}_k (\mathbf{x}^TR^TR\mathbf{x}) \leq | \mathbf{r}'_k |^2C.\]
To reduce the search space, it is thus beneficial to reduce the rows of $R^{-1}$. Furthermore, rearranging the columns of $R$ so that the shortest column vector is first helps reduce the total running time of the Fincke-Pohst algorithm. In particular, doing so leads to progressively smaller intervals in which $x_k$ may exist. 

We express this reduction with a unimodular matrix $V^{-1}$ so that $R_1^{-1} = V^{-1}R^{-1}$. Applying an appropriate permutation matrix $P$, we then reorder the columns of $R_1$. Since $R_1 = RV$, this yields $R_2 = (RV)P$. Finally, we compute the solutions $\mathbf{y}$ to $\mathbf{y}^TR_2^TR_2\mathbf{y}\leq C$ and recover the short vectors $\mathbf{x}$ satisfying the original inequality \eqref{eq:ShortVector} via $\mathbf{x} = VP\mathbf{y}$. 

As before, let $\Gamma$ be an $n$-dimensional lattice with basis matrix $B$, quadratic form $\Phi$, and associated bilinear form $Q$. In \autoref{subsec:FinckePohst}, it is noted that an implementation of the Fincke-Pohst algorithm is available in Magma. Unfortunately, this implementation does not support \textit{translated} lattices, a variant of the Fincke-Pohst algorithm which we will need in \autoref{ch:EfficientTMSolver}. By a translated lattice, we mean the discrete subgroup of $\mathbb{R}^n$ of the form
\[\Gamma + \mathbf{w} = \left\{ \sum_{i=1}^n x_i \mathbf{b}_i + \mathbf{w}\ : \ x_i \in \mathbb{Z} \right\},\]
where $\mathbf{b}_1, \dots, \mathbf{b}_n$ form the columns of $B$ and $\mathbf{w} \in \mathbb{R}^n$. In the remainder of this section, we describe how to modify the Fincke-Pohst algorithm and its refinements to support translated lattices. 

Analogous to the non-translated case, any embedded vector $\mathbf{u}$ of $\Gamma + \mathbf{w}$ may be expressed as $\mathbf{u} = B\mathbf{x} + \mathbf{w}$ for a corresponding coordinate vector $\mathbf{x}$. In this case, we call the vector $\mathbf{u} \in \Gamma + \mathbf{w}$ \textit{small} if 
\begin{equation} \label{eq:TransShortVector}
(\mathbf{x}-\mathbf{c})^TB^TB(\mathbf{x}-\mathbf{c}) \leq C
\end{equation}
for some $C \geq 0$, where $\mathbf{c} = -\mathbf{w}$. 

As in the usual short vectors process, we begin by applying Cholesky decomposition to the positive definite matrix $A=B^TB$ to obtain an upper triangular matrix $R$ satisfying $A = R^TR$. We then generate the matrices $R_1, R_2, V,$ and $P$ described earlier in this section. This allows us to write $A = U^TGU$ for a unimodular matrix $U$ and Gram matrix $G$ given by
\[U = P^{-1}V^{-1} \quad \text{ and } \quad G = R_2^TR_2.\]
Thus the inequality~\eqref{eq:TransShortVector} becomes
\begin{equation} \label{eq:TransShortVector2}
(\mathbf{y}-\mathbf{d})^TG(\mathbf{y}-\mathbf{d}) \leq C
\end{equation}
where
\[\mathbf{y} = U\mathbf{x} \quad \text{ and } \quad \mathbf{d} = U\mathbf{c}.\]
To enumerate the vectors $\mathbf{y}$ which satisfy this inequality, we consider the bilinear form $Q$ associated to the lattice $\Gamma$. We express this form as
\[ Q(\mathbf{y}-\mathbf{d}) = \sum_{i=1}^n q_{ii}\left( y_i - d_i + \sum_{j=i+1}^n q_{ij}(y_j - d_j)\right)^2.\]
As in the usual Fincke-Pohst algorithm, the coefficients $q_{ij}$ are defined from the matrix $R$ via equation~\eqref{eq:CholeskyCoeffs}. Let
\[U_k = -d_k + \sum_{j = k+1}^n q_{kj}(y_j - d_j),\]
where $U_n = -d_n$, and rewrite $Q(\mathbf{y}-\mathbf{d})$ as
\[ Q(\mathbf{y}-\mathbf{d}) = \sum_{i=1}^n q_{ii}\left( y_i - d_i + \sum_{j=i+1}^n q_{ij}(y_j - d_j)\right)^2 = \sum_{i=1}^n q_{ii}\left( y_i + U_i\right)^2.\]
From here, we proceed as in the usual Fincke-Pohst algorithm described in \autoref{subsec:FinckePohst}. Once we compute all vectors $\mathbf{y}$ which satisfy \eqref{eq:TransShortVector2}, we recover $\mathbf{x}$ using $\mathbf{x} = U^{-1}\mathbf{y}$. 

As a final remark about Fincke-Pohst for translated lattices, it is worth noting that one could use the variant implemented in Magma simply by increasing the dimension of the lattice $\Gamma$ and appropriately redefining the basis vectors $\mathbf{b}_i$. This is highly ill-advised as it increases the search space and subsequent running time of the algorithm.  

Generally speaking, the use of Fincke-Pohst in our applications poses one of the main bottlenecks in solving Thue-Mahler and Thue-Mahler-like equations. Specifically, this algorithm often yields upwards of hundreds of millions of short vectors, each one needing to be stored and, in our case, appropriately manipulated. This creates both timing and memory problems, often leading to gigabytes of data usage. Deleting these vectors does not release the memory and, as with the class group function, Magma's built-in Fincke-Pohst process cannot be terminated without exiting the program. The primary advantage of implementing and using our own version of Fincke-Pohst, as described in this section, is therefore the ability to add a fail-stop should the number of vectors found become too large. 

%--------------------------------------------------------------------------------------------------------------------------------------------%
%--------------------------------------------------------------------------------------------------------------------------------------------%

\endinput

Any text after an \endinput is ignored.
You could put scraps here or things in progress.






%    3. Notes
%    4. Footnotes

%    5. Bibliography
\begin{singlespace}
\raggedright
\bibliographystyle{abbrvnat}
\bibliography{biblio}
\end{singlespace}

\appendix
%    6. Appendices (including copies of all required UBC Research
%       Ethics Board's Certificates of Approval)
%\include{reb-coa}	% pdfpages is useful here
\chapter{Supporting Materials}

This would be any supporting material not central to the dissertation.
For example:
\begin{itemize}
\item additional details of methodology and/or data;
\item diagrams of specialized equipment developed.;
\item copies of questionnaires and survey instruments.
\end{itemize}


\backmatter
%    7. Index
% See the makeindex package: the following page provides a quick overview
% <http://www.image.ufl.edu/help/latex/latex_indexes.shtml>


\end{document}
